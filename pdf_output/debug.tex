% Options for packages loaded elsewhere
\PassOptionsToPackage{unicode}{hyperref}
\PassOptionsToPackage{hyphens}{url}
%
\documentclass[
  11pt,
  a4paper,
]{article}
\usepackage{amsmath,amssymb}
\usepackage{setspace}
\usepackage{iftex}
\ifPDFTeX
  \usepackage[T1]{fontenc}
  \usepackage[utf8]{inputenc}
  \usepackage{textcomp} % provide euro and other symbols
\else % if luatex or xetex
  \usepackage{unicode-math} % this also loads fontspec
  \defaultfontfeatures{Scale=MatchLowercase}
  \defaultfontfeatures[\rmfamily]{Ligatures=TeX,Scale=1}
\fi
\usepackage{lmodern}
\ifPDFTeX\else
  % xetex/luatex font selection
  \setmainfont[]{DejaVu Serif}
  \setsansfont[]{DejaVu Sans}
  \setmonofont[]{DejaVu Sans Mono}
\fi
% Use upquote if available, for straight quotes in verbatim environments
\IfFileExists{upquote.sty}{\usepackage{upquote}}{}
\IfFileExists{microtype.sty}{% use microtype if available
  \usepackage[]{microtype}
  \UseMicrotypeSet[protrusion]{basicmath} % disable protrusion for tt fonts
}{}
\makeatletter
\@ifundefined{KOMAClassName}{% if non-KOMA class
  \IfFileExists{parskip.sty}{%
    \usepackage{parskip}
  }{% else
    \setlength{\parindent}{0pt}
    \setlength{\parskip}{6pt plus 2pt minus 1pt}}
}{% if KOMA class
  \KOMAoptions{parskip=half}}
\makeatother
\usepackage{xcolor}
\usepackage[margin=2cm]{geometry}
\usepackage{longtable,booktabs,array}
\usepackage{calc} % for calculating minipage widths
% Correct order of tables after \paragraph or \subparagraph
\usepackage{etoolbox}
\makeatletter
\patchcmd\longtable{\par}{\if@noskipsec\mbox{}\fi\par}{}{}
\makeatother
% Allow footnotes in longtable head/foot
\IfFileExists{footnotehyper.sty}{\usepackage{footnotehyper}}{\usepackage{footnote}}
\makesavenoteenv{longtable}
\setlength{\emergencystretch}{3em} % prevent overfull lines
\providecommand{\tightlist}{%
  \setlength{\itemsep}{0pt}\setlength{\parskip}{0pt}}
\setcounter{secnumdepth}{5}
\ifLuaTeX
\usepackage[bidi=basic]{babel}
\else
\usepackage[bidi=default]{babel}
\fi
\babelprovide[main,import]{russian}
\ifPDFTeX
\else
\babelfont{rm}[]{DejaVu Serif}
\fi
% get rid of language-specific shorthands (see #6817):
\let\LanguageShortHands\languageshorthands
\def\languageshorthands#1{}
\ifLuaTeX
  \usepackage{selnolig}  % disable illegal ligatures
\fi
\IfFileExists{bookmark.sty}{\usepackage{bookmark}}{\usepackage{hyperref}}
\IfFileExists{xurl.sty}{\usepackage{xurl}}{} % add URL line breaks if available
\urlstyle{same}
\hypersetup{
  pdflang={ru},
  hidelinks,
  pdfcreator={LaTeX via pandoc}}

\author{}
\date{}

\begin{document}

{
\setcounter{tocdepth}{4}
\tableofcontents
}
\setstretch{1.15}
\hypertarget{ux431ux43bux43eux43a-1.-ux444ux438ux437ux438ux43aux430-ux430ux442ux43cux43eux441ux444ux435ux440ux44b}{%
\section{Блок 1. «Физика
атмосферы»}\label{ux431ux43bux43eux43a-1.-ux444ux438ux437ux438ux43aux430-ux430ux442ux43cux43eux441ux444ux435ux440ux44b}}

\hypertarget{ux442ux435ux43cux430-ux441ux442ux440ux43eux435ux43dux438ux435-ux441ux43eux441ux442ux430ux432-ux441ux432ux43eux439ux441ux442ux432ux430-ux430ux442ux43cux43eux441ux444ux435ux440ux44b}{%
\subsection{1.1. Тема «Строение, состав, свойства
атмосферы»}\label{ux442ux435ux43cux430-ux441ux442ux440ux43eux435ux43dux438ux435-ux441ux43eux441ux442ux430ux432-ux441ux432ux43eux439ux441ux442ux432ux430-ux430ux442ux43cux43eux441ux444ux435ux440ux44b}}

\hypertarget{ux43fux440ux435ux434ux43cux435ux442-ux438-ux43cux435ux442ux43eux434-ux43cux435ux442ux435ux43eux440ux43eux43bux43eux433ux438ux438-ux435ux435-ux43cux435ux441ux442ux43e-ux441ux440ux435ux434ux438-ux434ux440ux443ux433ux438ux445-ux43dux430ux443ux43a-ux438-ux441ux432ux44fux437ux44c-ux441-ux43dux438ux43cux438}{%
\subsubsection{\texorpdfstring{\textbf{Предмет и метод метеорологии, ее
место среди других наук и связь с
ними}}{Предмет и метод метеорологии, ее место среди других наук и связь с ними}}\label{ux43fux440ux435ux434ux43cux435ux442-ux438-ux43cux435ux442ux43eux434-ux43cux435ux442ux435ux43eux440ux43eux43bux43eux433ux438ux438-ux435ux435-ux43cux435ux441ux442ux43e-ux441ux440ux435ux434ux438-ux434ux440ux443ux433ux438ux445-ux43dux430ux443ux43a-ux438-ux441ux432ux44fux437ux44c-ux441-ux43dux438ux43cux438}}

\textbf{Метеорология} --- это наука, изучающая атмосферу с
использованием фундаментальных законов физики (в частности,
гидродинамики, термодинамики и теории излучения) и химии. Основной
задачей является не только описание, но и прогнозирование состояния
атмосферы на различных пространственно-временных масштабах. Это
достигается путем решения системы уравнений, описывающих поведение
атмосферы как сплошной среды.

\textbf{Иерархия атмосферных движений:} Процессы в атмосфере
классифицируются по масштабам, и для каждого масштаба применяются свои
допущения и упрощения фундаментальных уравнений. Анализ масштабов (scale
analysis) является ключевым методом в динамической метеорологии,
позволяющим отбросить второстепенные члены в уравнениях и выделить
главные физические балансы.

\begin{itemize}
\tightlist
\item
  \textbf{Микромасштаб (\textless{} 2 км):} Турбулентность, отдельные
  конвективные ячейки. Здесь доминируют силы плавучести и вязкости. Сила
  Кориолиса пренебрежимо мала.
\item
  \textbf{Мезомасштаб (2--2000 км):} Грозовые системы, бризы,
  горно-долинные циркуляции. На этих масштабах становится важным учёт
  силы Кориолиса, но гидростатическое приближение (равновесие между
  силой тяжести и вертикальным градиентом давления) всё ещё может быть
  неточным в областях с сильными вертикальными ускорениями (например, в
  грозовых облаках).
\item
  \textbf{Синоптический (крупный) масштаб (\textgreater{} 2000 км):}
  Циклоны и антициклоны. Движения на этом масштабе хорошо описываются
  \textbf{гидростатическим} и \textbf{геострофическим} приближениями.
  Анализ этих систем является ядром классической синоптической
  метеорологии.
\item
  \textbf{Планетарный масштаб:} Длинные волны Россби, общая циркуляция
  атмосферы. Эти движения определяют фоновые условия для синоптических
  процессов.
\end{itemize}

\textbf{Методы исследования:} Современная метеорология опирается на три
столпа: \textbf{теорию}, \textbf{наблюдения} и \textbf{численное
моделирование}.

\begin{enumerate}
\def\labelenumi{\arabic{enumi}.}
\tightlist
\item
  \textbf{Теория (Динамическая метеорология):} Фундаментом служат
  \textbf{уравнения гидротермодинамики} --- система, описывающая
  эволюцию атмосферных полей.
\item
  \textbf{Наблюдения (Синоптическая метеорология):} Классический
  синоптический метод --- это совместный анализ карт погоды на разных
  высотах для построения трёхмерной картины атмосферных процессов.
\item
  \textbf{Численное моделирование (NWP):} Уравнения гидротермодинамики
  решаются численно. Важнейшей теоретической проблемой является
  \textbf{предсказуемость}, ограниченная хаотической природой атмосферы.
\end{enumerate}

\hypertarget{ux43eux441ux43dux43eux432ux43dux44bux435-ux43cux435ux442ux435ux43eux440ux43eux43bux43eux433ux438ux447ux435ux441ux43aux438ux435-ux432ux435ux43bux438ux447ux438ux43dux44b-ux438-ux430ux442ux43cux43eux441ux444ux435ux440ux43dux44bux435-ux44fux432ux43bux435ux43dux438ux44f}{%
\subsubsection{\texorpdfstring{\textbf{Основные метеорологические
величины и атмосферные
явления}}{Основные метеорологические величины и атмосферные явления}}\label{ux43eux441ux43dux43eux432ux43dux44bux435-ux43cux435ux442ux435ux43eux440ux43eux43bux43eux433ux438ux447ux435ux441ux43aux438ux435-ux432ux435ux43bux438ux447ux438ux43dux44b-ux438-ux430ux442ux43cux43eux441ux444ux435ux440ux43dux44bux435-ux44fux432ux43bux435ux43dux438ux44f}}

\begin{itemize}
\tightlist
\item
  \textbf{Давление (\(p\)):} В синоптическом масштабе давление
  практически всегда находится в \textbf{гидростатическом равновесии} с
  силой тяжести (\(dp = -g\rho dz\)). Горизонтальные градиенты давления
  создают \textbf{силу барического градиента}, которая является основной
  движущей силой горизонтальных движений.
\item
  \textbf{Температура (\(T\)):} Мера внутренней энергии воздуха.
  Горизонтальные градиенты температуры являются причиной
  \textbf{бароклинности} атмосферы и приводят к возникновению
  \textbf{термического ветра}.
\item
  \textbf{Потенциальная температура (\(\theta\)):} Температура, которую
  принял бы объём воздуха при его адиабатическом приведении к
  стандартному давлению (1000 гПа). \(\theta = T(p_0/p)^{R/c_p}\).
  Является \textbf{консервативной величиной} при адиабатических
  процессах.
\item
  \textbf{Ветер (\(\vec{V}\)):} В свободной атмосфере синоптического
  масштаба ветер близок к \textbf{геострофическому балансу} ---
  равновесию между силой барического градиента и силой Кориолиса.
\item
  \textbf{Влажность:} \textbf{Удельная влажность (\(q\))} и
  \textbf{отношение смеси (\(r\))} являются консервативными величинами
  при адиабатических процессах без фазовых переходов.
\item
  \textbf{Атмосферные явления:} Наблюдаемые процессы, такие как грозы,
  туманы, метели, которые являются результатом сложных взаимодействий
  динамических и термодинамических полей.
\end{itemize}

\hypertarget{ux441ux43eux441ux442ux430ux432-ux430ux442ux43cux43eux441ux444ux435ux440ux43dux43eux433ux43e-ux432ux43eux437ux434ux443ux445ux430-ux43fux43eux441ux442ux43eux44fux43dux43dux44bux435-ux438-ux43fux435ux440ux435ux43cux435ux43dux43dux44bux435-ux441ux43eux441ux442ux430ux432ux43dux44bux435-ux447ux430ux441ux442ux438}{%
\subsubsection{\texorpdfstring{\textbf{Состав атмосферного воздуха:
постоянные и переменные составные
части}}{Состав атмосферного воздуха: постоянные и переменные составные части}}\label{ux441ux43eux441ux442ux430ux432-ux430ux442ux43cux43eux441ux444ux435ux440ux43dux43eux433ux43e-ux432ux43eux437ux434ux443ux445ux430-ux43fux43eux441ux442ux43eux44fux43dux43dux44bux435-ux438-ux43fux435ux440ux435ux43cux435ux43dux43dux44bux435-ux441ux43eux441ux442ux430ux432ux43dux44bux435-ux447ux430ux441ux442ux438}}

\begin{itemize}
\tightlist
\item
  \textbf{Постоянные газы:} Определяют основные термодинамические
  свойства воздуха. К ним относятся \textbf{Азот (\(N_2\))}
  (\textasciitilde78\%) и \textbf{Кислород (\(O_2\))}
  (\textasciitilde21\%), а также \textbf{Аргон (Ar)}
  (\textasciitilde0.9\%).
\item
  \textbf{Переменные газы и аэрозоли:}

  \begin{itemize}
  \tightlist
  \item
    \textbf{Водяной пар (\(H_2O\)):} Его фазовые переходы сопровождаются
    выделением/поглощением \textbf{скрытого тепла}, которое служит
    основным источником энергии для многих мощных атмосферных явлений.
  \item
    \textbf{Озон (\(O_3\)):} Поглощение озоном УФ-радиации является
    причиной существования \textbf{стратосферной температурной
    инверсии}.
  \item
    \textbf{Аэрозоли:} Служат \textbf{ядрами конденсации (CCN)} и
    \textbf{ядрами кристаллизации (IN)}, без которых образование облаков
    в реальной атмосфере было бы затруднено.
  \end{itemize}
\end{itemize}

\hypertarget{ux430ux43dux442ux440ux43eux43fux43eux433ux435ux43dux43dux43eux435-ux432ux43eux437ux434ux435ux439ux441ux442ux432ux438ux435-ux43dux430-ux430ux442ux43cux43eux441ux444ux435ux440ux443}{%
\subsubsection{\texorpdfstring{\textbf{Антропогенное воздействие на
атмосферу}}{Антропогенное воздействие на атмосферу}}\label{ux430ux43dux442ux440ux43eux43fux43eux433ux435ux43dux43dux43eux435-ux432ux43eux437ux434ux435ux439ux441ux442ux432ux438ux435-ux43dux430-ux430ux442ux43cux43eux441ux444ux435ux440ux443}}

Хозяйственная деятельность человека достигла масштаба, влияющего на
глобальный климат и состав атмосферы.

\begin{itemize}
\tightlist
\item
  \textbf{Рост концентрации парниковых газов:} Сжигание ископаемого
  топлива привело к систематическому увеличению концентрации \(CO_2\) в
  атмосфере. Растут также концентрации метана (\(CH_4\)) и закиси азота
  (\(N_2O\)). Это усиливает парниковый эффект и является основной
  причиной современного глобального потепления.
\item
  \textbf{Разрушение озонового слоя:} Выбросы хлорфторуглеводородов
  (фреонов) привели к истощению стратоферного озона, особенно в полярных
  широтах («озоновые дыры»).
\item
  \textbf{Загрязнение воздуха:} Выбросы оксидов серы и азота приводят к
  образованию \textbf{кислотных дождей}. Промышленные и транспортные
  выбросы в городах создают \textbf{смог}.
\end{itemize}

\hypertarget{ux438ux437ux43cux435ux43dux435ux43dux438ux435-ux441ux43eux441ux442ux430ux432ux430-ux432ux43eux437ux434ux443ux445ux430-ux441-ux432ux44bux441ux43eux442ux43eux439}{%
\subsubsection{\texorpdfstring{\textbf{Изменение состава воздуха с
высотой}}{Изменение состава воздуха с высотой}}\label{ux438ux437ux43cux435ux43dux435ux43dux438ux435-ux441ux43eux441ux442ux430ux432ux430-ux432ux43eux437ux434ux443ux445ux430-ux441-ux432ux44bux441ux43eux442ux43eux439}}

\begin{itemize}
\tightlist
\item
  \textbf{Гомосфера:} Простирается от земной поверхности до высоты
  примерно 100 км. В этом слое процентное соотношение основных газов
  (азота, кислорода, аргона) практически постоянно благодаря
  интенсивному турбулентному перемешиванию.
\item
  \textbf{Гетеросфера:} Расположена выше 100 км. Здесь доминирует
  процесс гравитационного разделения (диффузии), при котором лёгкие газы
  (гелий, водород) концентрируются в верхних слоях, а тяжёлые --- в
  нижних. Также происходит диссоциация молекул на атомы под действием
  УФ-излучения.
\end{itemize}

\hypertarget{ux432ux435ux440ux442ux438ux43aux430ux43bux44cux43dux43eux435-ux441ux442ux440ux43eux435ux43dux438ux435-ux430ux442ux43cux43eux441ux444ux435ux440ux44b}{%
\subsubsection{\texorpdfstring{\textbf{Вертикальное строение
атмосферы}}{Вертикальное строение атмосферы}}\label{ux432ux435ux440ux442ux438ux43aux430ux43bux44cux43dux43eux435-ux441ux442ux440ux43eux435ux43dux438ux435-ux430ux442ux43cux43eux441ux444ux435ux440ux44b}}

По характеру вертикального распределения температуры атмосфера делится
на несколько слоёв (сфер), разделённых переходными зонами (паузами).

\begin{itemize}
\tightlist
\item
  \textbf{Тропосфера:}

  \begin{itemize}
  \tightlist
  \item
    \textbf{Высота:} От поверхности до 8--10 км в полярных широтах,
    10--12 км в умеренных и 16--18 км в тропиках.
  \item
    \textbf{Температура:} Убывает с высотой от среднего значения +15°C у
    поверхности до --50\ldots--60°C в умеренных широтах и до --80°C в
    тропиках на верхней границе. Средний вертикальный градиент
    температуры (\(\gamma\)) составляет \textbf{0,65°C/100 м}.
  \item
    \textbf{Давление:} Уменьшается экспоненциально с высотой, падая от
    \textasciitilde1013 гПа у поверхности до 200--250 гПа на верхней
    границе (тропопаузе).
  \item
    \textbf{Физические процессы:} Содержит \textasciitilde80\% массы
    атмосферы и почти весь водяной пар. Это область интенсивной
    конвекции, турбулентности, облако- и осадкообразования. Здесь
    развиваются все основные погодные системы.
  \end{itemize}
\item
  \textbf{Стратосфера:}

  \begin{itemize}
  \tightlist
  \item
    \textbf{Высота:} От тропопаузы до \textasciitilde50 км
    (стратопауза).
  \item
    \textbf{Температура:} В среднем \textbf{растёт с высотой} от
    --50\ldots--60°C до значений около 0°C в стратопаузе. Этот рост
    обусловлен поглощением УФ-радиации \textbf{озоновым слоем}.
  \item
    \textbf{Давление:} Падает от \textasciitilde200 гПа до
    \textasciitilde1 гПа на верхней границе.
  \item
    \textbf{Физические процессы:} Высокая статическая устойчивость из-за
    температурной инверсии, что подавляет вертикальные движения. Здесь
    распространяются и разрушаются планетарные волны.
  \end{itemize}
\item
  \textbf{Мезосфера:}

  \begin{itemize}
  \tightlist
  \item
    \textbf{Высота:} От \textasciitilde50 км до \textasciitilde85 км
    (мезопауза).
  \item
    \textbf{Температура:} Вновь \textbf{убывает с высотой}, достигая в
    мезопаузе абсолютного минимума для атмосферы (в среднем --90°C,
    иногда до --130°C). Охлаждение происходит за счёт радиационного
    излучения молекул \(CO_2\).
  \item
    \textbf{Давление:} Падает от \textasciitilde1 гПа до сотых долей
    гПа.
  \item
    \textbf{Физические процессы:} Здесь сгорает большинство метеоров,
    иногда наблюдаются серебристые облака.
  \end{itemize}
\item
  \textbf{Термосфера:}

  \begin{itemize}
  \tightlist
  \item
    \textbf{Высота:} От \textasciitilde85 км до \textasciitilde600 км.
  \item
    \textbf{Температура:} \textbf{Резко возрастает с высотой} до
    1000--1500 К и выше из-за поглощения жёсткой УФ и рентгеновской
    радиации Солнца.
  \item
    \textbf{Давление:} Чрезвычайно низкое, атмосфера крайне разрежена.
  \item
    \textbf{Физические процессы:} Здесь происходят полярные сияния.
  \end{itemize}
\item
  \textbf{Экзосфера:}

  \begin{itemize}
  \tightlist
  \item
    \textbf{Высота:} Выше \textasciitilde600 км. Внешняя часть
    атмосферы, где происходит ускользание лёгких газов (водорода, гелия)
    в мировое пространство.
  \end{itemize}
\end{itemize}

\hypertarget{ux43eux437ux43eux43dux43eux441ux444ux435ux440ux430-ux438-ux438ux43eux43dux43eux441ux444ux435ux440ux430}{%
\subsubsection{\texorpdfstring{\textbf{Озоносфера и
Ионосфера}}{Озоносфера и Ионосфера}}\label{ux43eux437ux43eux43dux43eux441ux444ux435ux440ux430-ux438-ux438ux43eux43dux43eux441ux444ux435ux440ux430}}

Эти слои выделяются не по температурному признаку, а по
физико-химическим свойствам и накладываются на температурную структуру.

\begin{itemize}
\tightlist
\item
  \textbf{Озоносфера:} Слой с максимальной концентрацией озона
  (\(O_3\)), расположенный в основном в стратосфере на высотах 20--30
  км. Озон поглощает биологически опасную УФ-радиацию Солнца, защищая
  жизнь на Земле.
\item
  \textbf{Ионосфера:} Ионизированная часть верхней атмосферы (от
  \textasciitilde70 км и выше), охватывающая часть мезосферы и всю
  термосферу. Содержит значительное количество свободных электронов и
  ионов, что делает её электропроводной и способной отражать радиоволны.
\end{itemize}

\hypertarget{ux43fux43eux43dux44fux442ux438ux435-ux43fux43eux433ux440ux430ux43dux438ux447ux43dux43eux433ux43e-ux438-ux43fux440ux438ux437ux435ux43cux43dux43eux433ux43e-ux441ux43bux43eux44f-ux430ux442ux43cux43eux441ux444ux435ux440ux44b}{%
\subsubsection{\texorpdfstring{\textbf{Понятие пограничного и приземного
слоя
атмосферы}}{Понятие пограничного и приземного слоя атмосферы}}\label{ux43fux43eux43dux44fux442ux438ux435-ux43fux43eux433ux440ux430ux43dux438ux447ux43dux43eux433ux43e-ux438-ux43fux440ux438ux437ux435ux43cux43dux43eux433ux43e-ux441ux43bux43eux44f-ux430ux442ux43cux43eux441ux444ux435ux440ux44b}}

\begin{itemize}
\tightlist
\item
  \textbf{Планетарный пограничный слой (ППС) или слой трения:} Нижний
  слой тропосферы толщиной 1--1.5 км, в котором на движение воздуха
  существенное влияние оказывает трение о земную поверхность и
  термическое воздействие. Здесь наблюдается суточный ход метеовеличин и
  формируется \textbf{спираль Экмана}.
\item
  \textbf{Приземный слой:} Самая нижняя часть ППС (50--100 м), где
  турбулентные потоки тепла, влаги и количества движения можно считать
  практически постоянными с высотой.
\end{itemize}

\hypertarget{ux43fux43eux43dux44fux442ux438ux435-ux43e-ux432ux43eux437ux434ux443ux448ux43dux44bux445-ux43cux430ux441ux441ux430ux445-ux438-ux444ux440ux43eux43dux442ux430ux445}{%
\subsubsection{\texorpdfstring{\textbf{Понятие о воздушных массах и
фронтах}}{Понятие о воздушных массах и фронтах}}\label{ux43fux43eux43dux44fux442ux438ux435-ux43e-ux432ux43eux437ux434ux443ux448ux43dux44bux445-ux43cux430ux441ux441ux430ux445-ux438-ux444ux440ux43eux43dux442ux430ux445}}

\begin{itemize}
\tightlist
\item
  \textbf{Воздушная масса:} Обширная область тропосферы с относительно
  однородными свойствами.
\item
  \textbf{Атмосферный фронт:} Узкая переходная зона между двумя
  различными воздушными массами. Классическая \textbf{формула Маргулеса}
  описывает наклон фронтальной поверхности как функцию от разности
  температур и скоростей ветра по обе стороны от фронта.
\end{itemize}

\hypertarget{ux443ux440ux430ux432ux43dux435ux43dux438ux435-ux441ux43eux441ux442ux43eux44fux43dux438ux44f-ux441ux443ux445ux43eux433ux43e-ux438-ux432ux43bux430ux436ux43dux43eux433ux43e-ux432ux43eux437ux434ux443ux445ux430.-ux432ux438ux440ux442ux443ux430ux43bux44cux43dux430ux44f-ux442ux435ux43cux43fux435ux440ux430ux442ux443ux440ux430}{%
\subsubsection{\texorpdfstring{\textbf{Уравнение состояния сухого и
влажного воздуха. Виртуальная
температура}}{Уравнение состояния сухого и влажного воздуха. Виртуальная температура}}\label{ux443ux440ux430ux432ux43dux435ux43dux438ux435-ux441ux43eux441ux442ux43eux44fux43dux438ux44f-ux441ux443ux445ux43eux433ux43e-ux438-ux432ux43bux430ux436ux43dux43eux433ux43e-ux432ux43eux437ux434ux443ux445ux430.-ux432ux438ux440ux442ux443ux430ux43bux44cux43dux430ux44f-ux442ux435ux43cux43fux435ux440ux430ux442ux443ux440ux430}}

Соотношение между давлением (p), плотностью (\(\rho\)) и температурой
(T) для сухого воздуха описывается \textbf{уравнением состояния}:
\(p = \rho R_d T\). Для учёта влияния влажности, которая уменьшает
плотность воздуха, вводится понятие \textbf{виртуальной температуры
(\(T_v\))}: \(T_v \approx T(1 + 0.61q)\). Использование \(T_v\)
позволяет применять то же уравнение состояния для влажного воздуха, что
критически важно для корректного расчёта силы плавучести и статической
устойчивости.

\hypertarget{ux445ux430ux440ux430ux43aux442ux435ux440ux438ux441ux442ux438ux43aux438-ux432ux43bux430ux436ux43dux43eux433ux43e-ux432ux43eux437ux434ux443ux445ux430-ux438-ux441ux432ux44fux437ux44c-ux43cux435ux436ux434ux443-ux43dux438ux43cux438}{%
\subsubsection{\texorpdfstring{\textbf{Характеристики влажного воздуха и
связь между
ними}}{Характеристики влажного воздуха и связь между ними}}\label{ux445ux430ux440ux430ux43aux442ux435ux440ux438ux441ux442ux438ux43aux438-ux432ux43bux430ux436ux43dux43eux433ux43e-ux432ux43eux437ux434ux443ux445ux430-ux438-ux441ux432ux44fux437ux44c-ux43cux435ux436ux434ux443-ux43dux438ux43cux438}}

\begin{itemize}
\tightlist
\item
  \textbf{Давление насыщенного пара (E):} Максимально возможное
  парциальное давление водяного пара при данной температуре. Его
  зависимость от температуры описывается \textbf{уравнением
  Клаузиуса-Клайперона}, которое объясняет экспоненциальный рост \(E\) с
  температурой.
\item
  \textbf{Консервативные переменные:} Для анализа термодинамических
  процессов используются консервативные величины. Для ненасыщенного
  воздуха это \textbf{потенциальная температура (\(\theta\))} и
  \textbf{отношение смеси (\(r\))}. Для насыщенных процессов, когда
  происходит конденсация, используется
  \textbf{эквивалентно-потенциальная температура (\(\theta_e\))},
  которая сохраняется даже при выделении скрытого тепла.
\item
  \textbf{Термодинамические диаграммы:} Для анализа процессов в
  атмосфере (например, для оценки устойчивости и потенциала развития
  конвекции) используются специальные диаграммы (аэрологические
  диаграммы), на которых нанесены линии, соответствующие различным
  термодинамическим процессам (изобары, изотермы, сухие и влажные
  адиабаты, линии постоянного отношения смеси).
\end{itemize}

ewpage

\hypertarget{ux43fux440ux435ux434ux43cux435ux442-ux438-ux43cux435ux442ux43eux434-ux43cux435ux442ux435ux43eux440ux43eux43bux43eux433ux438ux438-ux435ux435-ux43cux435ux441ux442ux43e-ux441ux440ux435ux434ux438-ux434ux440ux443ux433ux438ux445-ux43dux430ux443ux43a-ux438-ux441ux432ux44fux437ux44c-ux441-ux43dux438ux43cux438-1}{%
\subsubsection{Предмет и метод метеорологии, ее место среди других наук
и связь с
ними}\label{ux43fux440ux435ux434ux43cux435ux442-ux438-ux43cux435ux442ux43eux434-ux43cux435ux442ux435ux43eux440ux43eux43bux43eux433ux438ux438-ux435ux435-ux43cux435ux441ux442ux43e-ux441ux440ux435ux434ux438-ux434ux440ux443ux433ux438ux445-ux43dux430ux443ux43a-ux438-ux441ux432ux44fux437ux44c-ux441-ux43dux438ux43cux438-1}}

Как мы с вами знаем, метеорология --- это весьма обширная и многогранная
наука, центральной задачей которой является изучение воздушной оболочки
Земли --- атмосферы, ее строения, физических свойств и протекающих в ней
процессов. Это, безусловно, ключевая геофизическая дисциплина, поскольку
она исследует физические законы, применимые к нашей планете.

Рассмотрим более детально ее предмет, основные методы исследования и
место в системе наук.

\hypertarget{ux43fux440ux435ux434ux43cux435ux442-ux43cux435ux442ux435ux43eux440ux43eux43bux43eux433ux438ux438}{%
\paragraph{Предмет
метеорологии}\label{ux43fux440ux435ux434ux43cux435ux442-ux43cux435ux442ux435ux43eux440ux43eux43bux43eux433ux438ux438}}

Предмет метеорологии охватывает колоссальный спектр атмосферных явлений
и процессов, от микромасштабных до планетарных. В общем смысле,
метеорология изучает:

\begin{itemize}
\tightlist
\item
  \textbf{Строение атмосферы}: ее вертикальное распределение на основные
  слои, такие как тропосфера, стратосфера, мезосфера, термосфера и
  экзосфера, а также характер изменения температуры в этих слоях. Важное
  внимание уделяется гомосфере и гетеросфере, различающимся по газовому
  составу.
\item
  \textbf{Свойства атмосферы}: физические характеристики воздуха, такие
  как атмосферное давление, температура, влажность воздуха, скорость и
  направление ветра, а также плотность. Изучаются их пространственное
  распределение и временные изменения, которые определяют погодные
  условия.
\item
  \textbf{Протекающие в ней процессы}:

  \begin{itemize}
  \tightlist
  \item
    \textbf{Атмосферные движения}: От микромасштабных турбулентных
    вихрей (которые, кстати, вызывают флуктуации метеорологических
    величин) и местных ветров (бризы, фены, бора, смерчи, шквалы) до
    синоптических вихрей (циклоны, антициклоны) и общей циркуляции
    атмосферы (ОЦА), включая муссоны и пассаты. Динамическая
    метеорология конкретно фокусируется на изучении атмосферных движений
    и связанных с ними преобразований энергии.
  \item
    \textbf{Тепло- и влагообмен}: Процессы теплообмена,
    теплопроводности, радиационного баланса (включая эффективное
    излучение и поглощенную солнечную радиацию), а также фазовых
    переходов воды (испарение, конденсация, сублимация) и их влияние на
    погоду и климат.
  \item
    \textbf{Формирование облаков и осадков}: Изучение условий
    образования и развития облаков различных форм (кучевые, слоистые,
    перисто-кучевые и др.), их водности, а также выпадения осадков.
    Особое внимание уделяется влиянию вертикальных движений и
    бароклинности среды.
  \item
    \textbf{Атмосферные фронты и воздушные массы}: Их образование,
    перемещение, эволюция и взаимодействие, определяющие погодные
    условия на значительных территориях. Изучаются различные типы
    воздушных масс (теплые, холодные, нейтральные; устойчивые,
    неустойчивые) и фронтов (теплые, холодные, окклюзии), а также
    процессы фронтогенеза и фронтолиза.
  \item
    \textbf{Особые явления погоды}: Туманы (радиационные, адвективные,
    испарения, смешения, городские, морозные), метели, пыльные бури,
    грозы, шквалы, обледенение (воздушных судов и кораблей).
  \item
    \textbf{Антропогенное воздействие}: Загрязнение атмосферы, изменение
    климата и влияние городской среды на метеорологический режим. Это
    включает изучение парниковых газов, прямых выбросов тепла и их
    последствий.
  \end{itemize}
\end{itemize}

\hypertarget{ux43cux435ux442ux43eux434ux44b-ux43cux435ux442ux435ux43eux440ux43eux43bux43eux433ux438ux438}{%
\paragraph{Методы
метеорологии}\label{ux43cux435ux442ux43eux434ux44b-ux43cux435ux442ux435ux43eux440ux43eux43bux43eux433ux438ux438}}

Метеорология использует комплекс методов для достижения своих целей,
которые можно разделить на несколько ключевых направлений:

\begin{enumerate}
\def\labelenumi{\arabic{enumi}.}
\tightlist
\item
  \textbf{Эмпирический (наблюдательный) метод}:

  \begin{itemize}
  \tightlist
  \item
    Основой метеорологии и климатологии являются данные регулярных,
    синхронных метеорологических наблюдений за состоянием атмосферы,
    формирующие глобальную сеть.
  \item
    Используются \textbf{прямые (контактные) измерения}, осуществляемые
    наземными метеорологическими станциями (стандартными стационарными,
    переносными автоматическими, приборными комплексами высотных
    сооружений), корабельными комплексами и самолетным зондированием
    (попутные наблюдения, летающие лаборатории). Наземные станции,
    работающие по единому регламенту ВМО, считаются наиболее надежным и
    точным источником информации, служа базой для обработки всех
    остальных данных. Важно отметить, что систематические
    инструментальные наблюдения в России начались в 1725 году.
  \item
    Применяются \textbf{дистанционные (косвенные) измерения}, получаемые
    с метеорологических спутников Земли (полярно-орбитальных и
    геостационарных), метеорологических радиолокаторов (для регистрации
    осадков и облачности) и лидаров (для вертикальных профилей
    прозрачности). Изобретение радиозонда Молчановым в 1930 году, а
    затем запуск ИСЗ с метеорологической аппаратурой с 1958 года,
    произвели настоящую революцию в службе погоды, предоставляя
    возможность изучения вертикального строения атмосферы и огромный
    объем данных.
  \item
    Теоретические выводы проверяются путем сопоставления с фактическими
    данными наблюдений, что делает метеорологическую практику как
    источником данных, так и критерием правильности теории.
  \end{itemize}
\item
  \textbf{Теоретический (аналитический) метод}:

  \begin{itemize}
  \tightlist
  \item
    Включает в себя преобразование и решение общих уравнений
    гидротермодинамики, адаптированных к физическим условиям атмосферы.
    Эти уравнения являются математическим выражением фундаментальных
    законов физики: сохранения импульса (второй закон Ньютона),
    сохранения энергии, и сохранения массы.
  \item
    Из-за исключительной сложности и турбулентного характера атмосферных
    процессов (с резкими беспорядочными колебаниями скорости, давления,
    плотности, называемыми пульсациями или флуктуациями), приходится
    использовать осредненные уравнения. Однако это приводит к появлению
    новых неизвестных величин, требующих введения коэффициентов
    турбулентности (горизонтальной и вертикальной) для замыкания системы
    уравнений.
  \item
    Упрощение уравнений динамики достигается путем оценки влияния
    различных факторов и сил, учета порядка метеорологических величин
    или применения теории подобия.
  \end{itemize}
\item
  \textbf{Численный (гидродинамический) метод}:

  \begin{itemize}
  \tightlist
  \item
    Решение большинства вопросов и задач динамической метеорологии
    связано с определением пространственного распределения
    метеорологических величин в данный момент времени или с прогнозом
    погоды.
  \item
    Поскольку точные аналитические решения нелинейных дифференциальных
    уравнений в частных производных, описывающих нестационарные
    атмосферные процессы, найти невозможно, они решаются численными
    методами на высокопроизводительных компьютерах.
  \item
    Этапы работы вычислительных центров включают сбор и первичную
    обработку поступающих сводок, численный анализ метеорологических
    полей (интерполяция значений на регулярную сетку с применением таких
    методов, как полиномиальная интерполяция, последовательные
    уточнения, оптимальная интерполяция), расчет прогнозов по различным
    прогностическим схемам и выдачу результатов в удобном для
    потребителя виде.
  \item
    Важным аспектом является объективный анализ и усвоение несинхронной
    информации, например, со спутников, что требует четырехмерного
    анализа.
  \end{itemize}
\item
  \textbf{Синоптический метод}:

  \begin{itemize}
  \tightlist
  \item
    Оформился как наука во второй половине XIX века.
  \item
    Это метод анализа и прогноза атмосферных процессов и погодных
    условий на больших территориях с использованием синоптических карт и
    вспомогательных средств (аэрологических диаграмм, вертикальных
    разрезов). Его появление связано с составлением Брандесом первых
    синоптических карт для Европы в 1820-х годах и открытием закона
    Бейс-Бало.
  \item
    Основной прием синоптического анализа заключается в одновременном
    обозрении метеорологических элементов на значительных пространствах.
  \item
    Термин ``синоптический'' (от греч. ΣΥΝΟΠΤΙΚΟΣ) означает ``способный
    всё обозреть''.
  \item
    Синоптический метод включает прогноз перемещения и эволюции
    воздушных масс, атмосферных фронтов, циклонических и
    антициклонических образований. Этот прогноз является первым
    подготовительным этапом прогноза погоды и его основой. Он особенно
    эффективен для комплексных прогнозов, поскольку предсказанное
    синоптическое положение служит общей базой для всех индивидуальных
    прогнозов.
  \end{itemize}
\item
  \textbf{Статистический метод}:

  \begin{itemize}
  \tightlist
  \item
    Используется для характеристики турбулентного движения атмосферы
    посредством средних значений метеорологических величин, поскольку их
    мгновенные значения рассматриваются как случайные, и пользоваться
    ими практически невозможно.
  \item
    Применяется для разработки вероятностных моделей, устанавливающих
    связь между прошлым состоянием атмосферы и вероятностью будущего
    события. Это включает анализ предикторов (прогностических признаков)
    и предиктантов (прогнозируемых характеристик).
  \item
    Основан на статистической обработке архивных метеорологических
    данных. При разработке статистических методов прогноза необходимо
    учитывать такие факторы, как статистическая стационарность процесса
    и ограниченность архивного материала.
  \end{itemize}
\end{enumerate}

\hypertarget{ux43cux435ux441ux442ux43e-ux43cux435ux442ux435ux43eux440ux43eux43bux43eux433ux438ux438-ux441ux440ux435ux434ux438-ux434ux440ux443ux433ux438ux445-ux43dux430ux443ux43a-ux438-ux435ux435-ux441ux432ux44fux437ux44c-ux441-ux43dux438ux43cux438}{%
\paragraph{Место метеорологии среди других наук и ее связь с
ними}\label{ux43cux435ux441ux442ux43e-ux43cux435ux442ux435ux43eux440ux43eux43bux43eux433ux438ux438-ux441ux440ux435ux434ux438-ux434ux440ux443ux433ux438ux445-ux43dux430ux443ux43a-ux438-ux435ux435-ux441ux432ux44fux437ux44c-ux441-ux43dux438ux43cux438}}

Метеорология не является изолированной наукой; она глубоко интегрирована
в систему естественных наук и имеет многочисленные связи с ними.

\begin{enumerate}
\def\labelenumi{\arabic{enumi}.}
\tightlist
\item
  \textbf{Геофизические науки}: Метеорология является частью геофизики,
  изучая физические процессы в одной из земных оболочек.
\item
  \textbf{Физика}: Является фундаментальной основой. Динамическая
  метеорология прямо опирается на общие законы физики, такие как
  гидромеханика и термодинамика. Исходные уравнения динамической
  метеорологии выражают фундаментальные физические законы: сохранения
  импульса (второй закон Ньютона), сохранения энергии, и сохранения
  массы. Принципы оптических и акустических явлений в атмосфере также
  являются частью метеорологии, выделяясь в отдельные дисциплины ---
  атмосферная оптика и атмосферная акустика.
\item
  \textbf{Математика}: Необходима для количественного анализа
  атмосферных процессов. Используются дифференциальное и интегральное
  исчисления (градиент, дивергенция, вихрь, лапласиан, якобиан), решение
  систем нелинейных дифференциальных уравнений в частных производных.
  История науки показывает, что математика была необходима для развития
  естествознания, как подчеркивалось еще в Академии Платона, а Ньютон
  строил свою ``натуральную философию'' на математических и
  экспериментальных началах.
\item
  \textbf{География и науки о Земле}: Климатология, как большой раздел
  метеорологии, рассматривается как практически самостоятельная
  дисциплина и является географической наукой, так как климат --- это
  одна из физико-географических характеристик местности. Метеорология
  изучает пространственное распределение метеорологических величин и их
  изменение, что тесно связано с географией.
\item
  \textbf{Климатология}: Изучает климат как совокупность атмосферных
  условий, свойственных месту на земном шаре. Тесно связана с
  динамической и синоптической метеорологией, аэрологией, и
  экспериментальной метеорологией.
\item
  \textbf{Океанология и гидрология}: Атмосфера неразрывно связана с
  Мировым океаном. Изучается взаимодействие атмосферы и океана
  (например, влияние температуры поверхности океана на циклогенез,
  Эль-Ниньо). Гидрология, в свою очередь, занимается водными ресурсами и
  водным балансом Земли, что является частью общего влагооборота в
  природе.
\item
  \textbf{Аэрология}: Учение о методах исследования свободной атмосферы,
  как правило, выше 1 км.
\item
  \textbf{Прикладные науки}: Метеорология имеет множество прикладных
  направлений, которые обеспечивают специфические потребности различных
  отраслей деятельности человека. К ним относятся авиационная,
  сельскохозяйственная, медицинская, космическая (спутниковая) и морская
  метеорология. Информация о погоде и климате критически важна для
  планирования и обслуживания работы авиации, ракетно-космических
  систем, различных видов транспорта, а также для строительства,
  энергетики, здравоохранения и сельскохозяйственного производства.
\item
  \textbf{Информационные технологии и инженерия}: Развитие метеорологии
  тесно связано с развитием приборостроения (например, первые термометры
  Галилея, барометры Торричелли, дождемеры Кастелли и Гука, флюгеры
  Вильда, анемометры Робинсона, радиозонды Молчанова). Современные
  методы прогноза погоды требуют использования высокопроизводительных
  компьютеров и сложных алгоритмов обработки данных. Служба погоды --
  это высокотехнологичная отрасль, занимающаяся непрерывным сбором,
  обработкой и использованием данных о состоянии атмосферы.
  Автоматизированная обработка метеорологической информации является
  неотъемлемой частью современного прогностического процесса.
\end{enumerate}

Таким образом, метеорология является фундаментальной наукой, чьи
теоретические основы уходят корнями в физику и математику, а
практические приложения охватывают широкий круг междисциплинарных задач,
связанных с функционированием всей геосферы и повседневной деятельностью
человечества. Ее развитие требует постоянного совершенствования методов
наблюдений, численного моделирования и статистического анализа, что в
свою очередь стимулирует прогресс в смежных областях науки и техники.

ewpage

\textbf{Основные метеорологические величины}

Атмосферные движения, процессы тепло- и влагообмена, а также связанные с
ними изменения погоды, определяются пространственным распределением так
называемых метеорологических величин. Пространство, каждая точка
которого характеризуется определенным значением метеорологической
величины, называется полем этой величины. Поля, как и сами величины,
делятся на скалярные и векторные. К скалярным полям относятся поля
температуры, давления, влажности воздуха, а к векторным -- поля ветра,
силы тяжести, силы Кориолиса и других векторных величин.

\begin{enumerate}
\def\labelenumi{\arabic{enumi}.}
\item
  \textbf{Атмосферное давление (P)}: Это одна из ключевых скалярных
  метеорологических величин, представляющая собой силу, с которой
  атмосфера действует на единицу площади. Измеряется оно в гектопаскалях
  (гПа). Давление в атмосфере быстро убывает с высотой, и его
  горизонтальное распределение формирует барическое поле, где изолинии
  равного давления (изобары) показывают области пониженного (циклоны, Н)
  и повышенного (антициклоны, В) давления. Важно отметить, что изменение
  давления с высотой также зависит от температуры воздушного столба, что
  приводит к горизонтальной неоднородности поля давления.
\item
  \textbf{Температура воздуха (T)}: Эта скалярная величина, измеряемая в
  градусах Цельсия (°С) или в абсолютных Кельвинах (T = t + 273.2 K),
  характеризует физическое состояние атмосферного воздуха. Температура
  воздуха непрерывно меняется в пространстве и во времени, имея широкий
  диапазон значений на Земле. Вертикальное распределение температуры
  позволяет выделить основные слои атмосферы: тропосферу, стратосферу,
  мезосферу, термосферу и экзосферу, разделенные переходными слоями,
  такими как тропопауза и мезопауза. В тропосфере температура, как
  правило, убывает с высотой, а в стратосфере -- возрастает. Суточные и
  годовые колебания температуры также являются важной характеристикой. В
  метеорологии широко используются понятия потенциальной температуры
  (связанной с адиабатическими процессами) и виртуальной температуры,
  которая учитывает влияние влажности воздуха на его плотность.
\item
  \textbf{Влажность воздуха (q, e, f)}: Характеристики влажности
  описывают содержание водяного пара в воздухе. К ним относятся:

  \begin{itemize}
  \tightlist
  \item
    \textbf{Парциальное давление водяного пара (е)}: Давление,
    создаваемое молекулами воды в воздухе, поступающими в него при
    испарении.
  \item
    \textbf{Давление насыщенного водяного пара (E)}: Максимальное
    парциальное давление водяного пара, возможное при данной
    температуре.
  \item
    \textbf{Абсолютная влажность (ρп)}: Масса водяного пара,
    содержащаяся в единице объема воздуха (т.е. плотность водяного
    пара).
  \item
    \textbf{Относительная влажность (f)}: Отношение фактического
    парциального давления водяного пара к давлению насыщенного водяного
    пара при данной температуре.
  \item
    \textbf{Точка росы (Td)}: Температура, при которой водяной пар,
    содержащийся в воздухе, достигает состояния насыщения при данном
    давлении.
  \item
    \textbf{Дефицит насыщения}: Разность между насыщающей и фактической
    упругостью водяного пара.
  \item
    \textbf{Дефицит точки росы}: Разность между температурой воздуха и
    точкой росы.
  \item
    Измерение влажности обычно производится психрометрическим методом,
    основанным на разности показаний сухого и смоченного термометров.
  \end{itemize}
\item
  \textbf{Ветер (V)}: Это горизонтальная составляющая движения воздуха
  относительно земной поверхности. Ветер характеризуется направлением (в
  угловых градусах) и скоростью (в м/с). Поле ветра тесно связано с
  полем давления, что проявляется в соответствии барических систем
  определенным системам синоптического масштаба. В свободной атмосфере
  действительный ветер очень близок к геострофическому, который является
  математической абстракцией, характеризующейся балансом барического
  градиента и силы Кориолиса и направленной параллельно изобарам. В
  метеорологии также изучаются дифференциальные характеристики поля
  ветра, такие как дивергенция скорости ветра (скалярная величина,
  характеризующая сходимость или расходимость потоков) и вихрь скорости
  (векторная величина, описывающая вращение воздушной частицы).
  Атмосферные движения, как правило, носят турбулентный характер, что
  проявляется в резких беспорядочных колебаниях скорости, давления и
  плотности, называемых пульсациями.
\item
  \textbf{Облачность (N)}: Это совокупность облаков в атмосфере,
  характеризующаяся количеством (в баллах) и формой. Облака играют
  важнейшую роль в формировании погоды и климата, влияя на радиационный
  баланс. Существует международная классификация, которая делит облака
  на 10 основных родов, а также множество видов и разновидностей в
  зависимости от их формы, внутренней структуры и высоты расположения
  (например, кучевые, слоистые, перистые, кучево-дождевые). Образование
  облаков тесно связано с вертикальными движениями воздуха и
  конденсацией водяного пара.
\item
  \textbf{Атмосферные осадки}: Любая форма воды, выпадающая из облаков
  или осаждающаяся на земной поверхности в результате конденсации или
  сублимации водяного пара. Осадки классифицируются по типу (ливневые,
  обложные, морось), фазовому состоянию (дождь, снег, крупа, град,
  изморозь) и интенсивности. Выпадение осадков обусловлено процессами
  слияния (коагуляции) облачных частиц, что требует достаточной
  вертикальной протяженности облака и интенсивных вертикальных движений.
\item
  \textbf{Дальность видимости (MDV)}: Характеризует прозрачность воздуха
  и определяется максимальным расстоянием, с которого видны определенные
  объекты. Ухудшение видимости часто связано с такими явлениями, как
  туманы, дымки, а также наличие взвешенных примесей в воздухе (пыль,
  сажа).
\end{enumerate}

\textbf{Атмосферные явления}

Помимо перечисленных метеорологических величин, существуют
многочисленные атмосферные явления, которые формируют общую картину
погоды:

\begin{enumerate}
\def\labelenumi{\arabic{enumi}.}
\item
  \textbf{Туманы и дымки}: Это помутнения воздуха в приземном слое.
  \textbf{Туман} определяется горизонтальной видимостью менее 1 км, а
  \textbf{дымка} -- видимостью от 1 до 10 км. Туманы классифицируются по
  механизму образования:

  \begin{itemize}
  \tightlist
  \item
    \textbf{Туманы охлаждения}: Радиационные (охлаждение поверхности
    ночью), адвективные (охлаждение теплого воздуха при движении над
    холодной поверхностью), адвективно-радиационные, орографические
    (адиабатическое охлаждение при подъеме воздуха по склонам гор).
  \item
    \textbf{Туманы испарения}: Например, парения водоемов (когда
    холодный воздух движется над теплой водой).
  \item
    \textbf{Туманы смешения}: Образуются при смешении воздушных масс с
    разными температурами и влажностью.
  \item
    \textbf{Городские туманы}: Имеют особенности, связанные с
    загрязнением воздуха и дополнительными источниками пара в городской
    среде.
  \end{itemize}
\item
  \textbf{Грозы}: Комплекс явлений, связанных с мощными кучево-дождевыми
  облаками (Cb), включающий электрические разряды (молнии), гром,
  ливневые осадки, а иногда и град. Грозы часто связаны с прохождением
  атмосферных фронтов, особенно холодных.
\item
  \textbf{Метели}: Перенос снега ветром, сопровождающийся ухудшением
  видимости. Различают общую метель (со снегопадом), низовую метель (без
  снегопада) и поземок (перенос снега непосредственно над поверхностью).
  Возникновение метелей зависит от силы ветра, температуры воздуха (ниже
  0°С) и наличия снежного покрова.
\item
  \textbf{Шквалы}: Внезапные и резкие усиления ветра, часто
  сопровождающие грозы. Скорость ветра при шквалах может достигать 30
  м/с и более.
\item
  \textbf{Пыльные (песчаные) бури}: Перенос сильным ветром большого
  количества пыли или песка, что приводит к значительному ухудшению
  видимости. Это явление чаще наблюдается в теплый период года в
  засушливых регионах.
\item
  \textbf{Гололед, изморозь, гололедица}: Явления, связанные с
  отложением льда на поверхностях. \textbf{Гололед} -- отложение
  плотного льда на предметах, \textbf{изморозь} -- рыхлые ледяные
  отложения, \textbf{гололедица} -- образование ледяной корки на
  дорогах. Эти явления могут приводить к обледенению самолетов и судов.
\end{enumerate}

\textbf{Синоптические объекты и циркуляционные системы}

Синоптическая метеорология изучает физические процессы в атмосфере,
определяющие погоду на значительных территориях. Для этого используются
синоптические карты и другие средства анализа. Ключевыми объектами
синоптического анализа являются:

\begin{enumerate}
\def\labelenumi{\arabic{enumi}.}
\item
  \textbf{Воздушные массы}: Обширные объемы воздуха в тропосфере,
  относительно однородные по физическим свойствам. Классифицируются по
  термодинамическим признакам (теплые, холодные, устойчивые,
  неустойчивые) и географическим очагам формирования (арктический,
  полярный/умеренных широт, тропический, экваториальный; морские или
  континентальные). Воздушные массы в процессе перемещения
  трансформируются, изменяя свои свойства.
\item
  \textbf{Атмосферные фронты}: Узкие переходные зоны, разделяющие
  воздушные массы с различными свойствами. Фронты могут быть основными
  (тропосферными) или вторичными (приземными), а также простыми (теплые,
  холодные, стационарные) или сложными (фронты окклюзии). Процессы
  образования (фронтогенез) и размывания (фронтолиз) фронтов тесно
  связаны с горизонтальными градиентами температуры и давления. В зонах
  фронтов наиболее ярко проявляется бароклинность атмосферы --
  пересечение изобарических и изотермических поверхностей.
\item
  \textbf{Синоптические вихри (Циклоны и Антициклоны)}:

  \begin{itemize}
  \tightlist
  \item
    \textbf{Циклоны}: Области пониженного атмосферного давления с
    характерной циркуляцией воздуха (против часовой стрелки в Северном
    полушарии). Циклоны умеренных широт часто возникают на атмосферных
    фронтах (фронтальные циклоны). \textbf{Тропические циклоны} (также
    известные как тайфуны или ураганы) -- это особо интенсивные вихри,
    формирующиеся в тропиках, с экстремальными скоростями ветра и
    ливнями, способные приносить огромные разрушения.
  \item
    \textbf{Антициклоны}: Области повышенного атмосферного давления с
    циркуляцией воздуха по часовой стрелке в Северном полушарии. Они
    могут быть промежуточными, заключительными или термическими.
  \end{itemize}
\item
  \textbf{Местные ветры}: Циркуляции воздуха, ограниченные определенными
  географическими районами и обусловленные локальными особенностями
  подстилающей поверхности или рельефа. Примеры включают \textbf{бризы}
  (ветер, дующий между водоемом и сушей, изменяющий направление днем и
  ночью), \textbf{горно-долинные ветры} (аналогичные бризам, но в горной
  местности), \textbf{фёны} (теплые сухие ветры, спускающиеся с гор) и
  \textbf{бору} (холодный порывистый ветер, стекающий с гор).
\item
  \textbf{Струйные течения}: Высотные узкие зоны сильных ветров в
  тропосфере или стратосфере, играющие важную роль в общей циркуляции
  атмосферы и влияющие на погоду нижележащих слоев.
\item
  \textbf{Общая циркуляция атмосферы (ОЦА)}: Это система
  крупномасштабных воздушных течений над земным шаром. Ее существование
  обусловлено неравномерным распределением солнечной радиации (разность
  температур между полярными и экваториальными областями) и вращением
  Земли. ОЦА включает в себя такие элементы, как пассаты, муссоны, а
  также планетарные фронтальные зоны и центры действия атмосферы
  (например, Азорский и Тихоокеанский максимумы, Исландский и Алеутский
  минимумы, Сибирский и Канадский антициклоны). ОЦА также проявляется в
  явлениях, таких как Эль-Ниньо, связанных с крупномасштабным
  взаимодействием океана и атмосферы.
\end{enumerate}

\textbf{Методы измерения и анализа}

В основе метеорологии лежат систематические наблюдения за состоянием
атмосферы, которые формируют глобальную сеть. Для измерений используются
как \textbf{прямые (контактные)} методы (наземные метеорологические
станции, корабли погоды, самолетное зондирование), так и
\textbf{дистанционные (косвенные)} методы (метеорологические спутники,
радиолокаторы, содары, лидары). Полученные данные затем анализируются и
используются для составления синоптических карт, аэрологических диаграмм
и вертикальных разрезов.

Современный прогноз погоды основывается на \textbf{численных методах},
которые решают сложные уравнения гидротермодинамики, а также на
\textbf{физико-статистических} подходах. Все это позволяет не только
диагностировать текущее состояние атмосферы, но и прогнозировать ее
изменения, несмотря на исключительную сложность атмосферных процессов.

ewpage

\textbf{1. Общее определение и роль атмосферы}

Атмосфера -- это газовая оболочка Земли, которая находится в непрерывном
движении. Она активно участвует в суточном и годовом вращении нашей
планеты, и именно в ней зарождаются атмосферные возмущения самых
различных масштабов. Атмосфера также постоянно обменивается с
подстилающей поверхностью, а также с биосферой -- растительным и
животным миром. В целом, атмосферные движения, процессы тепло- и
влагообмена в совокупности представляют собой основные факторы,
определяющие погоду и климат. Метеорология как наука изучает атмосферу,
ее строение, свойства и протекающие в ней процессы.

\textbf{2. Состав атмосферного воздуха: постоянные, переменные
компоненты и аэрозоли}

Химический состав атмосферного воздуха представляет собой сложную
механическую смесь, которую можно условно разделить на три основные
категории компонентов:

\begin{itemize}
\tightlist
\item
  Постоянные газы.
\item
  Переменные компоненты.
\item
  Атмосферные аэрозоли.
\end{itemize}

\textbf{2.1. Постоянные газы} К постоянным газам относятся те, объемные
концентрации которых стабильны в нижних слоях атмосферы, примерно до
высоты 100 км. Сухой чистый воздух, состоящий из этих газов, в
большинстве метеорологических расчетов можно рассматривать как единый
идеальный газ, поскольку температура и давление атмосферных газов на
Земле достаточно далеки от их тройной точки.

Основные постоянные компоненты:

\begin{itemize}
\tightlist
\item
  \textbf{Азот (N2):} Составляет 78.084 \% от всей массы сухого чистого
  воздуха или 78\% по объему.
\item
  \textbf{Кислород (O2):} Составляет 20.946 \% от массы сухого воздуха
  или 20.95\% (округленно 21\%) по объему.
\item
  \textbf{Аргон (Ar):} Содержится в количестве 0.934 \% по массе, или
  0.9\% по объему.
\item
  \textbf{Диоксид углерода (CO2):} Составляет 0.033 \% по массе, или
  0.036\% по объему.
\end{itemize}

Кроме этих четырех газов, к постоянным компонентам, составляющим около
1\% объема, относятся неон (Ne), гелий (He), метан (CH4), криптон (Kr),
водород (H2), ксенон.

\textbf{2.2. Переменные компоненты} Количество этих компонентов в
атмосфере значительно варьируется.

\begin{itemize}
\tightlist
\item
  \textbf{Водяной пар (H2O):} Это наиболее изменчивый и один из
  важнейших компонентов атмосферы. Он образуется за счет испарения
  жидкой воды, снега или льда. Парциальное давление водяного пара
  (обозначаемое как \(e\)) пропорционально количеству молекул воды и
  является ключевой характеристикой влажности воздуха. Если воздух
  содержит достаточно водяного пара для образования устойчивой жидкой
  фазы в виде капель, такой пар называется насыщенным, а его парциальное
  давление --- давлением насыщения (\(E_a\) или \(E_И\)), которое
  зависит только от температуры. Водяной пар является важнейшим
  парниковым газом, и его содержание регулируется процессами
  влагооборота и общей циркуляцией атмосферы. Влажность воздуха также
  характеризуется абсолютной влажностью (плотность водяного пара),
  относительной влажностью, удельной влажностью, отношением смеси и
  точкой росы.
\item
  \textbf{Озон (O3):} Слой с повышенной концентрацией озона (озоносфера)
  расположен на высоте 10-50 км. Озон поглощает ультрафиолетовое
  излучение Солнца, препятствуя его излишнему поступлению на земную
  поверхность и благоприятствуя существованию жизни.
\item
  Другие переменные газы могут включать аммиак, перекись водорода, йод,
  радон.
\end{itemize}

\textbf{2.3. Аэрозоли} Это взвешенные в воздухе твердые частицы или
мельчайшие капельки. К ним относятся дым, сажа, пепел, морская соль,
пыльца, споры, а также микроорганизмы. Размеры аэрозолей весьма малы,
варьируясь от 0.001 до 5 мкм. Они влияют на рассеяние солнечной радиации
и могут служить ядрами конденсации для образования облаков.

\textbf{3. Эволюция состава атмосферы}

Первичная атмосфера Земли, по гипотезе Л. Пастера, не содержала
свободного кислорода. Предполагается, что она состояла из газов
вулканического происхождения, таких как водяной пар, углекислый газ,
азот и сероводород. Кислород начал накапливаться в атмосфере в
результате жизнедеятельности первых одноклеточных автотрофных
организмов, осуществлявших фотосинтез. Постепенно, по мере охлаждения
Земли, водяной пар конденсировался, что привело к формированию океанов,
морей и озер.

\textbf{4. Физические характеристики и объемно-массовые свойства
атмосферы}

Атмосферный воздух удерживается у поверхности Земли благодаря силе
тяжести. Молекулы воздуха на любом уровне подвергаются давлению со
стороны вышележащих слоев, что определяет форму атмосферы, повторяющую
форму Земли, и создает атмосферное давление.

\begin{itemize}
\tightlist
\item
  На уровне моря плотность воздуха составляет приблизительно 1.3 кг/м³,
  а давление может быть 101004 Н/м². С увеличением высоты плотность и
  давление воздуха быстро убывают.
\item
  Около половины всей массы атмосферы сосредоточено в нижнем
  5-километровом слое, 9/10 --- в нижних 20 км, а 99.5\% от всей массы
  --- в нижних 80 км. Несмотря на это, атмосфера не имеет резкой верхней
  границы и простирается на тысячи километров.
\item
  Для влажного воздуха используется уравнение состояния, аналогичное
  уравнению идеального газа, но с введением \textbf{виртуальной
  температуры} (\(T_v\)). Эта температура определяется соотношением
  \(T_v = T(1 + 0.608q)\) или \(T_v = T(1 + 0.61q)\), где \(T\) ---
  реальная температура, а \(q\) --- удельная влажность. Виртуальная
  температура всегда выше реальной температуры воздуха, что указывает на
  то, что плотность влажного воздуха всегда меньше плотности сухого при
  прочих равных условиях.
\end{itemize}

\textbf{5. Взаимодействие атмосферы с излучением}

Составные части атмосферы играют ключевую роль во взаимодействии с
солнечной и земной радиацией. Атмосферный воздух слабо поглощает
лучистую энергию. Однако различные компоненты атмосферы, в частности
водяной пар, углекислый газ и озон, являются основными поглотителями
длинноволновой (земной) радиации.

\begin{itemize}
\tightlist
\item
  Это поглощение приводит к так называемому \textbf{парниковому
  эффекту}, который уменьшает потери тепла земной поверхностью и
  способствует повышению температуры воздуха у поверхности.
\item
  Помимо поглощения, компоненты атмосферы вызывают рассеяние и рефракцию
  света, что приводит к наблюдаемым оптическим явлениям, например,
  голубому цвету неба или миражам. Облачность также значительно влияет
  на радиационный баланс, уменьшая приток солнечной радиации к земной
  поверхности и ее эффективное излучение.
\end{itemize}

Таким образом, состав атмосферы, включая как постоянные, так и
переменные компоненты, а также аэрозоли, определяет ее физические
свойства, динамику и энергетический баланс, что в конечном итоге
формирует погоду и климат.

ewpage

\hypertarget{ux430ux43dux442ux440ux43eux43fux43eux433ux435ux43dux43dux43eux435-ux432ux43eux437ux434ux435ux439ux441ux442ux432ux438ux435-ux43dux430-ux430ux442ux43cux43eux441ux444ux435ux440ux443-1}{%
\subsubsection{Антропогенное воздействие на
атмосферу}\label{ux430ux43dux442ux440ux43eux43fux43eux433ux435ux43dux43dux43eux435-ux432ux43eux437ux434ux435ux439ux441ux442ux432ux438ux435-ux43dux430-ux430ux442ux43cux43eux441ux444ux435ux440ux443-1}}

Как Вы хорошо знаете, человеческая деятельность оказывает существенное
влияние на атмосферу, приводя к ее загрязнению. Эти воздействия не
ограничиваются лишь прямыми выбросами, но и изменяют физические свойства
подстилающей поверхности, что в совокупности приводит к комплексным
изменениям метеорологического режима.

Основные аспекты антропогенного воздействия включают:

\begin{enumerate}
\def\labelenumi{\arabic{enumi}.}
\tightlist
\item
  \textbf{Изменение метеорологических величин и явлений:}

  \begin{itemize}
  \tightlist
  \item
    \textbf{Оптически активные примеси:} Антропогенные газообразные,
    жидкие и твердые примеси, особенно в больших городах (с населением
    свыше 500 тыс. человек), являются оптически активными. Они
    интенсивно поглощают радиацию, преимущественно в инфракрасном
    диапазоне (2-120 мкм). Это приводит к обратному влиянию на состояние
    атмосферы. В результате в крупных городах, по сравнению с окружающей
    сельской местностью, изменяются практически все метеорологические
    величины и явления: температура, влажность воздуха, потоки солнечной
    и земной радиации, скорость ветра, а также вероятность образования и
    интенсивность туманов, дымок, облаков и осадков.
  \item
    \textbf{Мезомасштабные процессы:} Особенности метеорологического
    режима городов, отличающие их от невозмущенной сельской местности,
    обусловлены процессами мезомасштаба, охватывающими территории от
    нескольких до 100-200 км.
  \item
    \textbf{Городской ``остров тепла'':} Исследования показывают, что
    город перегрет по сравнению с окрестностями, причем этот эффект
    заметнее ночью, чем днем. Это указывает на то, что прямые выбросы
    тепла от промышленности и транспорта, которые в основном происходят
    днем, не являются определяющим фактором формирования ``острова
    тепла''. Ключевую роль играет изменение радиационного режима под
    влиянием атмосферных примесей. Дополнительное количество водяного
    пара, образующееся при сжигании различных видов топлива и
    выбрасываемое в атмосферу, значительно влияет на радиационный, а
    следовательно, и на термический режим города.
  \item
    \textbf{Водяной пар антропогенного происхождения:} Выбросы водяного
    пара антропогенного происхождения увеличивают его содержание в
    воздухе. Зимой, когда испарение с поверхности города мало отличается
    от окрестностей, именно антропогенный фактор обусловливает большее
    содержание водяного пара в городе по сравнению с окрестностями (Δe
    \textgreater{} 0). Аналогично, в ночное время суток летом, при
    инверсиях температуры и ослабленном турбулентном обмене,
    антропогенный вклад также преобладает.
  \item
    \textbf{Влияние на видимость и облачность:} Загрязнение воздуха
    примесями (включая водяной пар) существенно изменяет условия
    образования туманов и дымок. В то время как количество туманов в
    городе может уменьшаться из-за изменения температурно-влажностного
    режима, загрязнение способствует увеличению числа дымок.
    Конденсационные следы от самолетов, автомобилей и промышленных
    предприятий могут разрастаться, приводя к образованию плотных
    облаков верхнего яруса, а также туманов и низкой облачности в
    промышленных центрах.
  \end{itemize}
\item
  \textbf{Изменение состава воздуха и глобальные последствия:}

  \begin{itemize}
  \tightlist
  \item
    \textbf{Смог, кислотные дожди, озоновый слой:} Рост загрязнения
    приводит к необратимым изменениям физических свойств подстилающей
    поверхности и опасным изменениям состава воздуха. Это вызывает
    повышение повторяемости опасных явлений, таких как смог и кислотные
    дожди, а также увеличение мутности тропосферы и стратосферы (что
    приводит к повышению альбедо) и уменьшение толщины озонового слоя.
  \item
    \textbf{Парниковый эффект и климатические изменения:} Изменения
    состава атмосферы оказывают наиболее сильное влияние на климат через
    парниковый эффект.

    \begin{itemize}
    \tightlist
    \item
      \textbf{Водяной пар} является важнейшим парниковым газом, его
      содержание регулируется влагооборотом и общей циркуляцией
      атмосферы.
    \item
      \textbf{Диоксид углерода (CO2) и метан (CH4)} также являются
      важными парниковыми газами. Их концентрация сильно зависит от
      деятельности человека. Например, за последние 150 лет концентрация
      CO2 в атмосфере возросла примерно на 30\% за счет увеличения
      объемов сжигаемого топлива.
    \item
      \textbf{Последствия роста парниковых газов:} Увеличение
      концентрации парниковых газов приводит к росту температуры воздуха
      у поверхности Земли и одновременному падению температуры в верхней
      тропосфере. Это, в свою очередь, может усиливать неустойчивость
      атмосферной циркуляции и увеличивать частоту опасных явлений.
    \item
      \textbf{Метан:} В последние годы, в связи с ростом техногенных
      выбросов, прогнозируется увеличение роли метана в поглощении
      земной радиации, хотя пока его роль еще очень мала. При этом любой
      газ, поглощающий радиацию в ``окне прозрачности'' атмосферы
      (8,5-12 мкм), может вызвать серьезные изменения энергетического
      баланса планеты при росте его количества.
    \end{itemize}
  \end{itemize}
\end{enumerate}

Таким образом, антропогенное воздействие ведет не только к локальным
изменениям погодных условий, но и к глобальным климатическим сдвигам,
затрагивающим температурный режим, водный баланс и циркуляцию атмосферы.

\hypertarget{ux438ux437ux43cux435ux43dux435ux43dux438ux435-ux441ux43eux441ux442ux430ux432ux430-ux432ux43eux437ux434ux443ux445ux430-ux441-ux432ux44bux441ux43eux442ux43eux439-1}{%
\subsubsection{Изменение состава воздуха с
высотой}\label{ux438ux437ux43cux435ux43dux435ux43dux438ux435-ux441ux43eux441ux442ux430ux432ux430-ux432ux43eux437ux434ux443ux445ux430-ux441-ux432ux44bux441ux43eux442ux43eux439-1}}

Теперь перейдем к вертикальной структуре атмосферы и изменению ее
состава.

Атмосферный воздух, как Вы знаете, удерживается у поверхности Земли
силой тяжести. Молекулы воздуха на любом уровне находятся под давлением
вышележащих слоев, что и формирует атмосферное давление, убывающее с
высотой. Воздух является сжимаемой средой, и его плотность также быстро
убывает с высотой. При этом около половины всей массы атмосферы
сосредоточено в нижних 5 км, 9/10 --- в нижних 20 км, а 99,5\% --- в
нижних 80 км {[}предыдущий ответ, основывается на общем знании
метеорологии{]}.

С точки зрения газового состава, атмосферу принято делить на два
основных слоя:

\begin{enumerate}
\def\labelenumi{\arabic{enumi}.}
\tightlist
\item
  \textbf{Гомосфера (Homosphere):}

  \begin{itemize}
  \tightlist
  \item
    Это нижний слой атмосферы, простирающийся приблизительно до высоты
    90-100 км.
  \item
    В гомосфере состав сухого воздуха остается практически постоянным,
    за исключением локальных изменений, связанных с содержанием
    углекислого газа, озона и водяного пара. Это объясняется непрерывным
    турбулентным перемешиванием, которое препятствует гравитационному
    разделению газов по молекулярной массе.
  \item
    \textbf{Постоянные компоненты сухого воздуха} в гомосфере имеют
    следующую объемную концентрацию:

    \begin{itemize}
    \tightlist
    \item
      Азот (N2) -- около 78,084\% (или 78\%)
    \item
      Кислород (O2) -- около 20,946\% (или 21\%)
    \item
      Аргон (Ar) -- около 0,934\% (или 0,9\%)
    \item
      Углекислый газ (CO2) -- около 0,033\% (или 0,036\%)
    \item
      Другие постоянные газы, составляющие около 1\% объема: неон (Ne),
      гелий (He), метан (CH4), криптон (Kr), водород (H2), ксенон.
    \end{itemize}
  \item
    \textbf{Переменные компоненты}:

    \begin{itemize}
    \tightlist
    \item
      \textbf{Водяной пар (H2O):} Это наиболее изменчивый компонент
      атмосферы. Его содержание резко убывает с высотой, причем удельная
      влажность падает быстрее всего по сравнению с давлением,
      температурой и плотностью. Практически весь водяной пар атмосферы
      (более 4/5 всей массы воздуха) сосредоточен в тропосфере (нижние
      10-20 км). Водяной пар является основным источником осадков и
      одним из главных поглотителей земной (инфракрасной) радиации,
      играя ключевую роль в парниковом эффекте.
    \item
      \textbf{Озон (O3):} Слой с повышенной концентрацией озона ---
      озоносфера --- расположен на высотах от 10 до 50 км. Озон
      эффективно поглощает ультрафиолетовое излучение Солнца,
      предотвращая его избыточное поступление на земную поверхность, что
      критически важно для существования жизни. Он также поглощает
      инфракрасную радиацию.
    \end{itemize}
  \end{itemize}
\item
  \textbf{Гетеросфера (Heterosphere):}

  \begin{itemize}
  \tightlist
  \item
    Расположена выше 90-100 км.
  \item
    В этом слое состав воздуха значительно меняется с высотой. Это
    происходит из-за того, что на этих высотах турбулентное
    перемешивание ослабевает, и начинается гравитационное разделение
    газов по их молекулярной массе, причем более легкие газы (водород,
    гелий) преобладают на больших высотах.
  \item
    \textbf{Другие слои по физико-химическим процессам:}

    \begin{itemize}
    \tightlist
    \item
      \textbf{Нейтросфера:} От поверхности до 70-80 км, где преобладают
      незаряженные частицы.
    \item
      \textbf{Ионосфера:} Выше 70-80 км и до около 400 км,
      характеризуется высокой концентрацией положительных молекулярных и
      атомных ионов и свободных электронов, что обеспечивает возможность
      радиосвязи.
    \item
      \textbf{Хемосфера:} От стратосферы до нижней части термосферы,
      область, где происходят фотохимические реакции с участием
      кислорода, озона, азота, гидроксила и натрия.
    \end{itemize}
  \end{itemize}
\end{enumerate}

Следует отметить, что газы в атмосфере поглощают радиацию селективно, то
есть каждый газ поглощает свойственные ему длины волн. Основные
атмосферные газы (азот и кислород) поглощают инфракрасную радиацию
Солнца в верхних слоях атмосферы.

Таким образом, вертикальное распределение как постоянных, так и
переменных компонентов атмосферы, а также изменение их плотности и
давления, определяют уникальные физические свойства и процессы в каждом
слое, что, безусловно, имеет первостепенное значение для понимания
динамики атмосферы.

ewpage

\hypertarget{ux442ux440ux43eux43fux43eux441ux444ux435ux440ux430}{%
\subsubsection{Тропосфера}\label{ux442ux440ux43eux43fux43eux441ux444ux435ux440ux430}}

Тропосфера является нижней и основной частью атмосферы, простирающейся
от земной поверхности до высоты 9-17 км (по другим данным, 10-20 км).
Это самый сложный слой, наиболее подверженный воздействию подстилающей
поверхности.

\begin{itemize}
\tightlist
\item
  \textbf{Температурный режим:} Температура в тропосфере, как правило,
  убывает с высотой примерно на 0,65 °C на каждые 100 м. В среднем,
  температура в тропосфере также уменьшается от тропиков к полюсам,
  причем летом это уменьшение менее выражено, чем зимой.
\item
  \textbf{Основные характеристики:}

  \begin{itemize}
  \tightlist
  \item
    В тропосфере сосредоточено более 4/5 всей массы атмосферного воздуха
    и практически весь водяной пар.
  \item
    Именно здесь формируются все основные виды облаков и выпадают
    атмосферные осадки.
  \item
    Здесь зарождаются сложные атмосферные процессы, определяющие погоду.
  \item
    Нижний слой тропосферы, на высоте 500-1500 м, называют пограничным
    слоем атмосферы или слоем трения, а приземный подслой толщиной 30-50
    м выделяется в его пределах.
  \item
    Крупномасштабный западный перенос воздуха, охватывающий всю
    тропосферу, является глобальным вихрем.
  \item
    В целом, в крупномасштабных процессах горизонтальная составляющая
    скорости в тропосфере на два и более порядка превышает вертикальную.
  \end{itemize}
\end{itemize}

\hypertarget{ux442ux440ux43eux43fux43eux43fux430ux443ux437ux430}{%
\subsubsection{Тропопауза}\label{ux442ux440ux43eux43fux43eux43fux430ux443ux437ux430}}

Между тропосферой и стратосферой находится переходный слой ---
тропопауза, открытая в 1899 году. Ее высота колеблется, и в полярной
атмосфере она значительно ниже и теплее, чем в тропической. На
формирование тропопаузы существенное влияние оказывают турбулентное
перемешивание и вертикальные движения, а колебания ее высоты связаны с
горизонтальными перемещениями теплых и холодных воздушных масс.

\hypertarget{ux441ux442ux440ux430ux442ux43eux441ux444ux435ux440ux430}{%
\subsubsection{Стратосфера}\label{ux441ux442ux440ux430ux442ux43eux441ux444ux435ux440ux430}}

Стратосфера располагается непосредственно над тропосферой, простираясь
до высот 45-55 км (по другим данным, от 8-16 км до 45-50 км).

\begin{itemize}
\tightlist
\item
  \textbf{Температурный режим:} В нижнем слое стратосферы температура
  практически постоянна и составляет от -45 до -75 °C в зависимости от
  широты и времени года. В верхнем слое температура начинает расти с
  высотой, достигая значений от -20 до +20 °C. Над полюсами летом
  температура в стратосфере выше, чем в тропиках.
\item
  \textbf{Основные характеристики:}

  \begin{itemize}
  \tightlist
  \item
    В пределах стратосферы (10-50 км) находится озоносфера, слой с
    повышенной концентрацией озона (O3). Озон поглощает ультрафиолетовое
    излучение, предотвращая его излишнее поступление на земную
    поверхность и создавая благоприятные условия для жизни.
  \item
    Стратосфера является частью хемосферы, где происходят важные
    фотохимические реакции с участием кислорода, озона, азота и других
    газов.
  \item
    В нижней стратосфере наблюдаются высотные фронтальные зоны, которые
    могут быть настолько узкими, что их называют стратосферными
    фронтами. Важно отметить, что направление горизонтальных градиентов
    температуры в стратосфере часто (особенно летом) противоположно
    направлению в тропосфере, что может приводить к существованию
    стратосферного холодного фронта над тропосферным теплым фронтом.
  \item
    Формирование стратосферы происходит в основном под влиянием
    радиационных процессов.
  \item
    Давление воздуха у верхней границы стратосферы (50-55 км) составляет
    около 0,1 мбар.
  \end{itemize}
\end{itemize}

\hypertarget{ux441ux442ux440ux430ux442ux43eux43fux430ux443ux437ux430}{%
\subsubsection{Стратопауза}\label{ux441ux442ux440ux430ux442ux43eux43fux430ux443ux437ux430}}

Переходный слой между стратосферой и мезосферой, расположенный на
высотах около 50-55 км. Температура на этом уровне приближается к
0\ldots-10 °C.

\hypertarget{ux43cux435ux437ux43eux441ux444ux435ux440ux430}{%
\subsubsection{Мезосфера}\label{ux43cux435ux437ux43eux441ux444ux435ux440ux430}}

Мезосфера простирается над стратосферой до высоты 80-85 км (или 50-80
км).

\begin{itemize}
\tightlist
\item
  \textbf{Температурный режим:} Температура в мезосфере снова понижается
  с высотой, достигая значений от 0 °C до -90 °C.
\item
  \textbf{Основные характеристики:}

  \begin{itemize}
  \tightlist
  \item
    Мезопауза, верхняя граница мезосферы, является областью с наиболее
    низкими температурами в атмосфере, достигающими -75\ldots-85 °C.
    Интересно, что температура на уровне мезопаузы зимой выше, чем
    летом.
  \item
    Мезосфера также является частью хемосферы.
  \item
    Она входит в состав гомосферы (до 90-100 км), где состав воздуха
    остается относительно постоянным с высотой благодаря непрерывному
    перемешиванию.
  \item
    Кроме того, она относится к нейтросфере (до 70-80 км), где
    незаряженные частицы значительно преобладают над заряженными.
  \end{itemize}
\end{itemize}

\hypertarget{ux43cux435ux437ux43eux43fux430ux443ux437ux430}{%
\subsubsection{Мезопауза}\label{ux43cux435ux437ux43eux43fux430ux443ux437ux430}}

Переходный слой, разделяющий мезосферу и термосферу.

\hypertarget{ux442ux435ux440ux43cux43eux441ux444ux435ux440ux430}{%
\subsubsection{Термосфера}\label{ux442ux435ux440ux43cux43eux441ux444ux435ux440ux430}}

Термосфера простирается от мезопаузы до высот 600-1000 км (по другим
данным, 80-500 км).

\begin{itemize}
\tightlist
\item
  \textbf{Температурный режим:} В термосфере наблюдается быстрый рост
  температуры с высотой, достигая 800-1200 °C на высотах 200-300 км,
  после чего она остается примерно постоянной.
\item
  \textbf{Основные характеристики:}

  \begin{itemize}
  \tightlist
  \item
    В пределах термосферы находится ионосфера (выше 70-80 км, до
    \textasciitilde400 км), характеризующаяся высокой концентрацией
    положительных молекулярных и атомных ионов, а также свободных
    электронов. Благодаря этой электрической природе становятся
    возможными многие виды радиосвязи.
  \item
    Термосфера входит в состав гетеросферы, где состав воздуха
    значительно меняется с высотой из-за расслоения газов по их
    относительной молекулярной массе.
  \end{itemize}
\end{itemize}

\hypertarget{ux44dux43aux437ux43eux441ux444ux435ux440ux430}{%
\subsubsection{Экзосфера}\label{ux44dux43aux437ux43eux441ux444ux435ux440ux430}}

Экзосфера является самым внешним слоем атмосферы. Источники указывают,
что условная верхняя граница атмосферы находится на высоте около 2000
км, подразумевая, что экзосфера простирается от верхней части термосферы
и далее в космическое пространство.

\begin{itemize}
\tightlist
\item
  \textbf{Основные характеристики:}

  \begin{itemize}
  \tightlist
  \item
    Как и термосфера, экзосфера относится к гетеросфере.
  \item
    Физические и химические процессы в экзосфере значительно отличаются
    от тех, что происходят в нижних слоях, и их взаимосвязь пока изучена
    недостаточно. Иногда упоминается и магнитосфера как область
    атмосферы, где проявляется действие магнитного поля Земли.
  \end{itemize}
\end{itemize}

Таким образом, атмосфера представляет собой динамичную и многослойную
систему, каждый элемент которой играет свою роль в формировании погоды и
климата нашей планеты.

ewpage

По мере развития метеорологии и, в частности, аэрологии, было
установлено, что атмосфера имеет слоистое строение, несмотря на то, что
до начала XX века она считалась однородной. Вертикальное деление
атмосферы на различные слои производится на основе нескольких критериев,
включая физико-химические процессы и газовый состав.

\textbf{1. Гомосфера и Гетеросфера (по газовому составу)}

При рассмотрении атмосферы с точки зрения ее газового состава, мы
традиционно выделяем два макрослоя:

\begin{itemize}
\tightlist
\item
  \textbf{Гомосфера:} Это нижний слой атмосферы, простирающийся
  приблизительно до высот 90-100 км. Основной характеристикой гомосферы
  является относительная стабильность газового состава воздуха по
  высоте. Термин ``гомосфера'' происходит от греческого ``homos'', что
  означает ``то же самое'', ``равный'' или ``однородный''. В этом слое
  постоянные газы, такие как азот (N2) и кислород (O2), сохраняют свои
  объемные концентрации, поскольку турбулентное перемешивание (в том
  числе, вызванное крупномасштабными атмосферными движениями)
  обеспечивает относительно равномерное распределение компонентов. Тем
  не менее, стоит отметить, что даже в гомосфере наблюдаются изменения в
  содержании переменных компонентов, таких как углекислый газ, озон и
  водяной пар.
\item
  \textbf{Гетеросфера:} Располагается выше гомосферы, то есть от 90-100
  км и далее. В гетеросфере газовый состав воздуха значительно
  изменяется с высотой. Это связано с тем, что на этих высотах
  гравитационная сепарация газов по их относительной молекулярной массе
  начинает преобладать над турбулентным перемешиванием, что приводит к
  расслоению атмосферы на более легкие и тяжелые газы. Термин
  ``гетеросфера'' происходит от греческого ``heteros'', означающего
  ``другой'' или ``различный''.
\end{itemize}

\textbf{2. Озоносфера (по физико-химическим процессам)}

Озоносфера -- это отдельный, крайне важный слой атмосферы, выделяемый по
его физико-химическим свойствам.

\begin{itemize}
\tightlist
\item
  \textbf{Расположение и Характеристики:} Озоносфера находится на
  высотах от 10 до 50 км. Ее ключевая особенность -- повышенная
  концентрация озона (O3). Эта область также является частью хемосферы,
  где активно протекают фотохимические реакции с участием кислорода,
  озона, азота, гидроксила и натрия.
\item
  \textbf{Экологическая роль:} Озон в этом слое играет критическую роль
  в поглощении ультрафиолетового (УФ) излучения Солнца. Это поглощение
  предотвращает чрезмерное поступление вредного УФ-излучения на земную
  поверхность, создавая таким образом энергетический баланс,
  благоприятный для существования земных форм жизни. Изучение озонового
  слоя Земли является одним из направлений научных исследований
  Росгидромета. Отмечу, что ЦАО (Центральная аэрологическая
  обсерватория) является международным центром по оперативному анализу
  состояния озонового слоя.
\end{itemize}

\textbf{3. Ионосфера (по физико-химическим процессам)}

Ионосфера -- это еще один ключевой слой, характеризующийся
специфическими физико-химическими процессами, а именно ионизацией.

\begin{itemize}
\tightlist
\item
  \textbf{Расположение и Характеристики:} Ионосфера располагается выше
  нейтросферы (которая достигает 70-80 км), начинаясь примерно на высоте
  70-80 км и простираясь до высот около 400 км. В этом слое наблюдается
  высокая концентрация положительных молекулярных и атомных ионов, а
  также свободных электронов.
\item
  \textbf{Значение:} Благодаря своей электрической природе, ионосфера
  играет фундаментальную роль в обеспечении многих видов радиосвязи на
  Земле, поскольку она способна отражать радиоволны. Аэрономия -- это
  наука, изучающая высокие слои атмосферы, к которым относится
  ионосфера.
\end{itemize}

Таким образом, разделение атмосферы на гомо- и гетеросферу, а также
выделение озоносферы и ионосферы, позволяет нам лучше систематизировать
и понимать вертикальную структуру нашей газовой оболочки, ее химический
состав, энергетические процессы и их влияние на жизнь и технологии на
Земле.

ewpage

\hypertarget{ux43fux43eux433ux440ux430ux43dux438ux447ux43dux44bux439-ux441ux43bux43eux439-ux430ux442ux43cux43eux441ux444ux435ux440ux44b-ux43fux43bux430ux43dux435ux442ux430ux440ux43dux44bux439-ux43fux43eux433ux440ux430ux43dux438ux447ux43dux44bux439-ux441ux43bux43eux439}{%
\subsubsection{Пограничный слой атмосферы (Планетарный пограничный
слой)}\label{ux43fux43eux433ux440ux430ux43dux438ux447ux43dux44bux439-ux441ux43bux43eux439-ux430ux442ux43cux43eux441ux444ux435ux440ux44b-ux43fux43bux430ux43dux435ux442ux430ux440ux43dux44bux439-ux43fux43eux433ux440ux430ux43dux438ux447ux43dux44bux439-ux441ux43bux43eux439}}

\textbf{Определение и характеристики:} Пограничный слой атмосферы (ПСА),
также известный как планетарный пограничный слой, представляет собой
нижний слой тропосферы, простирающийся от поверхности Земли до высоты,
где влияние турбулентного трения о подстилающую поверхность становится
пренебрежимо малым. В этом слое силы турбулентного трения и отклоняющая
сила вращения Земли (сила Кориолиса) имеют сопоставимый порядок.

\textbf{Толщина:} Типичная толщина пограничного слоя варьируется в
пределах от 1000 до 1500 метров. В некоторых случаях источники указывают
диапазон от 500 до 1500 метров.

\textbf{Доминирующие процессы и силы:}

\begin{enumerate}
\def\labelenumi{\arabic{enumi}.}
\tightlist
\item
  \textbf{Трение:} Действие трения о земную поверхность является
  определяющим фактором в ПСА, вызывая постоянное ослабление скорости
  воздушных течений и изменение их направления. Вследствие трения
  направление ветра в приземном слое отклоняется от касательной к
  изобаре в сторону низкого давления на 15-40°, а его скорость
  уменьшается по сравнению с геострофическим ветром (например, до 0.7 Vg
  над морем и 0.5 Vg над сушей в умеренных широтах).
\item
  \textbf{Турбулентный обмен:} ПСА характеризуется интенсивным
  турбулентным обменом количества движения, тепла и влаги. Пульсации
  скорости, давления и плотности воздуха являются характерной чертой
  турбулентного движения в этом слое. Именно турбулентность обеспечивает
  распространение суточных колебаний температуры до высоты около 1.5 км,
  что является одним из ключевых проявлений взаимодействия атмосферы и
  подстилающей поверхности. Инверсии, как предельно устойчивые слои,
  сильно подавляют турбулентность, препятствуя проникновению тепла,
  влаги и примесей.
\item
  \textbf{Сила Кориолиса:} Хотя турбулентное трение доминирует у
  поверхности, на более высоких уровнях ПСА сила Кориолиса становится
  соизмеримой с силой трения.
\item
  \textbf{Вертикальные движения:} Отклонение ветра в сторону низкого
  давления из-за трения приводит к конвергенции (сходимости) воздуха у
  поверхности в циклонах и ложбинах, вызывая восходящие движения и
  образование облаков. Напротив, в антициклонах и гребнях наблюдается
  дивергенция (расходимость), что приводит к нисходящим движениям,
  способствующим развитию инверсий и накоплению загрязняющих веществ у
  поверхности.
\item
  \textbf{Тепло- и влагообмен:} ПСА играет ключевую роль в процессах
  тепло- и влагообмена между атмосферой и подстилающей поверхностью, что
  в совокупности с атмосферными движениями определяет погоду и климат.
\end{enumerate}

\textbf{Слой Экмана:} Верхняя (большая) часть пограничного слоя иногда
называется слоем Экмана. В этом слое характеристики турбулентного обмена
изучены значительно слабее, чем в приземном, но здесь уже отчетливо
проявляется влияние отклоняющей силы вращения Земли на изменение модуля
и направления скорости ветра с высотой.

\textbf{Взаимодействие с фронтами:} Фронтальные поверхности в
пограничном слое могут быть сильно размыты из-за интенсивного
турбулентного перемешивания, особенно при сильных ветрах. Ширина
переходной зоны между воздушными массами во фронтах в ПСА обычно не
превышает 100 км, тогда как выше она достигает 200-300 км.

\hypertarget{ux43fux440ux438ux437ux435ux43cux43dux44bux439-ux441ux43bux43eux439-ux430ux442ux43cux43eux441ux444ux435ux440ux44b-ux43fux440ux438ux437ux435ux43cux43dux44bux439-ux43fux43eux434ux441ux43bux43eux439}{%
\subsubsection{Приземный слой атмосферы (Приземный
подслой)}\label{ux43fux440ux438ux437ux435ux43cux43dux44bux439-ux441ux43bux43eux439-ux430ux442ux43cux43eux441ux444ux435ux440ux44b-ux43fux440ux438ux437ux435ux43cux43dux44bux439-ux43fux43eux434ux441ux43bux43eux439}}

\textbf{Определение и характеристики:} Приземный слой атмосферы
представляет собой самый нижний подслой пограничного слоя,
непосредственно прилегающий к земной поверхности.

\textbf{Толщина:} Его толщина значительно меньше, чем у всего
пограничного слоя, и составляет всего несколько десятков метров.
Конкретные значения варьируются: от 20-30 м при сильно устойчивой
(инверсионной) термической стратификации до 200-300 м при неустойчивой
стратификации.

\textbf{Доминирующие процессы и силы:}

\begin{enumerate}
\def\labelenumi{\arabic{enumi}.}
\tightlist
\item
  \textbf{Турбулентная вязкость:} В приземном слое отклоняющая сила
  вращения Земли крайне мала по сравнению с силой турбулентной вязкости.
  Это делает турбулентное трение абсолютным доминантом в динамике
  приземного слоя.
\item
  \textbf{Коэффициент турбулентного обмена:} Особенностью приземного
  слоя является то, что коэффициент турбулентного обмена (или
  турбулентной температуропроводности) уменьшается по мере приближения к
  земной поверхности.
\item
  \textbf{Термическая стратификация:} В приземном слое наблюдаются
  резкие изменения термического градиента. Например, в ясный летний день
  термический градиент может быть больше 0.2 °С/м, а в ночных приземных
  инверсиях --- меньше -0.1 °С/м. В целом, в ясный солнечный день
  стратификация пограничного слоя (включая приземный) является
  неустойчивой или безразличной, а в ясную ночь --- устойчивой.
\item
  \textbf{Прямое взаимодействие с поверхностью:} Это слой, где
  происходит наиболее непосредственное и интенсивное взаимодействие
  атмосферы с подстилающей поверхностью, влияющее на режимы температуры
  и влажности.
\end{enumerate}

\textbf{Значение для наблюдений:} Метеорологические измерения, особенно
скорости и направления ветра, часто проводятся на стандартной высоте
10-12 метров, что соответствует приземному слою.

\textbf{Резюмируя,} пограничный слой атмосферы --- это область, где
превалирует турбулентность, и заметно влияние трения и силы Кориолиса,
определяющие большинство погодных явлений. Приземный слой является его
нижним подслоем, где влияние трения о поверхность абсолютно доминирует
над силой Кориолиса, а вертикальные градиенты метеорологических величин,
особенно температуры, достигают максимальных значений. Эти слои
находятся в постоянном динамическом и энергетическом взаимодействии с
подстилающей поверхностью, что делает их критически важными для
метеорологического анализа и прогнозирования.

ewpage

\hypertarget{ux432ux43eux437ux434ux443ux448ux43dux44bux435-ux43cux430ux441ux441ux44b-ux43eux441ux43dux43eux432ux44b-ux438-ux43aux43bux430ux441ux441ux438ux444ux438ux43aux430ux446ux438ux44f}{%
\subsubsection{1. Воздушные Массы: Основы и
Классификация}\label{ux432ux43eux437ux434ux443ux448ux43dux44bux435-ux43cux430ux441ux441ux44b-ux43eux441ux43dux43eux432ux44b-ux438-ux43aux43bux430ux441ux441ux438ux444ux438ux43aux430ux446ux438ux44f}}

\textbf{Определение и формирование:} Воздушная масса -- это обширная
область тропосферы, в пределах которой воздух обладает сравнительно
однородными свойствами по температуре, влажности и запыленности. Эти
однородные свойства формируются в так называемых ``очагах
формирования'', то есть в районах с однородной подстилающей поверхностью
(например, обширные участки суши, океана, снежного или ледяного
покрова), где воздушные потоки пребывают достаточно долго, часто в
условиях антициклонических областей. В процессе длительного нахождения
над такой поверхностью, воздух постепенно приобретает её термические и
влажностные характеристики, достигая ``температуры равновесия'',
типичной для данного района и сезона. После этого сформировавшаяся
воздушная масса может сохранять эти свойства в течение длительного
времени при перемещении по Земле.

Горизонтальные размеры воздушных масс измеряются тысячами километров, в
то время как вертикальные -- несколькими километрами. Они могут
простираться от поверхности земли до тропопаузы или наслаиваться друг на
друга, причем, как правило, более теплая воздушная масса располагается
над более холодной.

\textbf{Классификация воздушных масс:} Воздушные массы классифицируются
по нескольким критериям:

\textbf{а) По географическому происхождению (географическая
классификация):} Эта классификация, предложенная ещё в 1920-х годах
Бержероном, основывается на положении основных термических поясов
земного шара и до сих пор сохраняет свою значимость. Выделяют четыре
основных типа:

\begin{itemize}
\tightlist
\item
  \textbf{Арктические (АВ) / Антарктические (ААВ):} Формируются над
  снежным и ледяным покровом Арктики (или Антарктики) или над
  незамерзшими арктическими морями. Эти массы наиболее прозрачны,
  видимость в них может превышать 300 км. В кАВ над островом Диксон
  средний вертикальный градиент температуры в слое 200-1000 м составляет
  около 0.13°C/100 м. Зимой кАВ обычно очень холодный и сухой. При
  вторжении на материки он может приобретать значительную
  неустойчивость.
\item
  \textbf{Умеренных широт (УВ) / Полярные (ПВ):} Формируются в умеренных
  широтах. Континентальный умеренный воздух (кУВ) формируется в
  центральных и восточных районах материков, зимой в зоне 30-50° с.ш.,
  летом -- 50-70° с.ш.. Его вертикальный градиент температуры обычно
  0.2-0.5°C/100 м, часто наблюдаются приземные и внутримассовые
  инверсии. Морской умеренный воздух (мУВ) вторгается на материки
  преимущественно в тылу циклонов за холодными фронтами.
\item
  \textbf{Тропические (ТВ):} Формируются в тропических широтах.
  Континентальный тропический воздух (кТВ) летом может формироваться над
  материками в зоне 15-50° с.ш. при длительной малооблачной погоде и
  слабых ветрах. Зимой кТВ формируется над северной частью Африканского
  континента. В кТВ вертикальный градиент температуры обычно
  0.7-0.8°C/100 м. Морской тропический воздух (мТВ) формируется над
  океанами в тропических широтах.
\item
  \textbf{Экваториальные (ЭВ):} Формируются вблизи экватора.
\end{itemize}

Для каждого из этих типов (кроме экваториального) дополнительно
учитывается их происхождение: \textbf{морское (m)} или
\textbf{континентальное (c)}. Например, мУВ, кАВ. Иногда выделяют также
муссонный воздух (МВ), который может иметь различные свойства в
зависимости от района и сезона, например, над Индией летом муссонный
воздух очень теплый и влажный, сопровождается ливнями и грозами, а зимой
-- сухой.

\textbf{б) По термодинамическим свойствам:} Эта классификация основана
на соотношении температуры воздушной массы с температурой подстилающей
поверхности и её вертикальной устойчивости.

\begin{itemize}
\tightlist
\item
  \textbf{Теплая воздушная масса:} Температура которой в данном районе
  выше температуры равновесия, и она постепенно охлаждается. Теплые
  массы, как правило, являются \textbf{устойчивыми}. В них вертикальный
  градиент температуры γ меньше влажноадиабатического γ\_ва,
  конвективные движения не развиваются, и кучевые облака не образуются.
  Типичная погода: значительная и сплошная слоистая или слоисто-кучевая
  облачность, иногда слабые осадки, адвективные туманы. Исключение:
  теплая \textbf{неустойчивая} воздушная масса может наблюдаться над
  материками летом или над океанами зимой, особенно при перемещении
  относительно теплого воздуха на более теплую водную поверхность. Для
  нее характерны кучевая и кучево-дождевая облачность, ливневые осадки,
  грозы.
\item
  \textbf{Холодная воздушная масса:} Температура которой ниже
  температуры равновесия, и она постепенно прогревается. Холодные массы,
  как правило, являются \textbf{неустойчивыми}. В них γ \textgreater{}
  γ\_ва, что способствует развитию конвективных движений и образованию
  кучевых облаков. Типичная погода: кучевая и кучево-дождевая
  облачность, ливневые осадки, грозы. Исключение: холодная
  \textbf{устойчивая} воздушная масса наблюдается над льдами Арктики и
  Антарктики, а также в центральных частях антициклонов, где преобладает
  безоблачная морозная погода, иногда с радиационными туманами.
\item
  \textbf{Нейтральная (местная) воздушная масса:} Сохраняет свои
  основные свойства без существенных изменений день за днем. Она может
  быть как устойчивой, так и неустойчивой, в зависимости от начальных
  свойств и трансформации.
\end{itemize}

\textbf{Трансформация воздушных масс:} Перемещаясь из очага своего
формирования в другие районы, воздушные массы непрерывно изменяют свои
свойства под влиянием взаимодействия с подстилающей поверхностью и
изменившихся условий радиационного баланса. Этот процесс, называемый
трансформацией, продолжается до достижения новой температуры равновесия.
Воздушные массы формируются непрерывно в любом географическом районе при
любых условиях циркуляции, и одновременно может существовать несколько
воздушных масс с различными свойствами, при этом одни находятся в
равновесии, другие претерпевают трансформацию.

Например, холодный неустойчивый воздух (мАВ или мУВ) по мере продвижения
вглубь материка превращается в устойчивый. Формирование антициклона
способствует быстрому повышению устойчивости холодного воздуха из-за
нисходящих движений. Если воздушная масса теплее подстилающей
поверхности, её вертикальный градиент температуры в приземном слое
быстро уменьшается, может стать отрицательным (инверсия температуры), и
такая теплая воздушная масса, как правило, становится устойчивой.

\hypertarget{ux430ux442ux43cux43eux441ux444ux435ux440ux43dux44bux435-ux444ux440ux43eux43dux442ux44b-ux441ux442ux440ux443ux43aux442ux443ux440ux430-ux438-ux434ux438ux43dux430ux43cux438ux43aux430}{%
\subsubsection{2. Атмосферные Фронты: Структура и
Динамика}\label{ux430ux442ux43cux43eux441ux444ux435ux440ux43dux44bux435-ux444ux440ux43eux43dux442ux44b-ux441ux442ux440ux443ux43aux442ux443ux440ux430-ux438-ux434ux438ux43dux430ux43cux438ux43aux430}}

\textbf{Определение и характеристики:} Атмосферный фронт -- это
наклонная переходная поверхность (или зона) раздела между двумя
различными по своим свойствам воздушными массами. На картах погоды он
часто выражается четкой линией раздела. Переходная зона между массами
может быть достаточно широкой (200-500 км), но если горизонтальные
градиенты температуры в ней велики, она называется фронтальной зоной. В
вертикальной плоскости этой зоне соответствует наклонный переходный
слой, называемый фронтальным. Длина фронтальной зоны измеряется тысячами
километров. Ширина зоны фронта в пограничном слое не превышает 100 км, а
выше этого слоя достигает 200-300 км.

\textbf{Классификация атмосферных фронтов:} Фронты классифицируются по
нескольким признакам:

\textbf{а) По горизонтальной и вертикальной протяженности и
циркуляционной значимости:}

\begin{itemize}
\tightlist
\item
  \textbf{Основные (тропосферные, высокие):} На них развиваются
  внетропические циклоны и их семейства. Могут прослеживаться на картах
  погоды в течение нескольких дней.
\item
  \textbf{Вторичные (приземные, низкие):} Менее протяженные и менее
  устойчивые, чем основные.
\item
  \textbf{Верхние:} Хорошо выраженная высотная фронтальная зона (ВФЗ),
  не связанная с приземным фронтом.
\end{itemize}

\textbf{б) По направлению перемещения и термическим изменениям:}

\begin{itemize}
\tightlist
\item
  \textbf{Теплый фронт:} Перемещается в сторону относительно холодной
  воздушной массы. За ним перемещается теплая масса, а холодная
  отступает. Теплые фронты преимущественно являются анафронтами, то есть
  восходящие движения в них преобладают, теплая воздушная масса натекает
  на клин холодного воздуха.

  \begin{itemize}
  \tightlist
  \item
    \textbf{Облачность и осадки:} Перед приземной линией теплого фронта,
    вдоль клина холодной воздушной массы, располагается обширная система
    облаков Ci-Cs и As-Ns, под которыми могут быть разорванные St fr..
    Часто выпадают обложные осадки (снег зимой, дождь летом), иногда
    метели, гололед. В начальной стадии циклогенеза или вблизи центра
    циклона, облачность теплого фронта может быть более мощной и
    занимать широкую зону.
  \end{itemize}
\item
  \textbf{Холодный фронт:} Перемещается в сторону относительно теплой
  воздушной массы. За ним перемещается холодная масса, а теплая
  предфронтальная масса отступает. Холодные фронты преимущественно
  являются катафронтами, то есть теплая воздушная масса совершает
  нисходящее движение вдоль клина холодной массы.

  \begin{itemize}
  \tightlist
  \item
    \textbf{Различают два типа:}

    \begin{itemize}
    \tightlist
    \item
      \textbf{Холодные фронты 1-го рода:} Медленно перемещающиеся,
      облачная система и осадки расположены в основном за линией фронта,
      напоминает зеркальное отражение теплого фронта.
    \item
      \textbf{Холодные фронты 2-го рода:} Быстро движущиеся, облачность
      и осадки (мощные Cb) наблюдаются в виде узкого вала
      непосредственно перед линией фронта. Часто сопровождаются шквалами
      и грозами. Ширина зоны осадков 50-100 км.
    \end{itemize}
  \item
    \textbf{Облачность и осадки:} Основная форма облаков -- мощные Cb,
    которые могут давать Ci, Cc, Ac, Sc при растекании, а под ними St
    fr. или Cu fr. плохой погоды в зоне ливневых осадков. Иногда система
    облаков холодного фронта сильно расслоена.
  \end{itemize}
\item
  \textbf{Малоподвижные (стационарные) фронты:} Участки основного
  фронта, не претерпевающие существенного перемещения, когда нормальные
  к линии фронта составляющие ветра отсутствуют или направлены
  противоположно.
\item
  \textbf{Фронты окклюзии (сомкнутые фронты):} Образуются в результате
  слияния холодного и теплого фронтов, что начинается вблизи центра
  циклона. Наиболее теплая воздушная масса вытесняется вверх и на
  переднюю периферию циклона. Характерны для поздней стадии развития
  циклона. В зависимости от соотношения температур воздуха по обе
  стороны фронта окклюзии, различают теплые и холодные фронты окклюзии,
  а также нейтральные.

  \begin{itemize}
  \tightlist
  \item
    \textbf{Облачность:} Соединяют в себе черты теплого и холодного
    фронтов, но часто выражены менее резко. Облачность в окклюдированном
    циклоне располагается в виде спирали, закрученной приблизительно к
    центру циклона.
  \end{itemize}
\end{itemize}

\textbf{Динамика фронтов:}

\begin{itemize}
\tightlist
\item
  \textbf{Наклон фронтальной поверхности:} Угол наклона фронтальной
  поверхности к горизонту (α) определяется уравновешивающим действием
  силы Кориолиса, которое препятствует растеканию холодного воздуха по
  горизонтали. В свободной атмосфере выше пограничного слоя ветер вдоль
  стационарного фронта направлен параллельно его линии. Угол наклона
  теплого фронта (анафронта) составляет примерно 0.01-0.03, а холодного
  (катафронта) --- 0.001-0.002.
\item
  \textbf{Фронтогенез и фронтолиз:} Фронты существуют ограниченный
  период времени, подвергаясь процессам образования (фронтогенез) и
  размывания (фронтолиз). Фронтогенезу благоприятствуют условия, при
  которых усиливается горизонтальный градиент температуры и происходит
  сходимость ветров, например, адвекция холода или тепла. Фронтолиз
  происходит при обратных условиях -- ослабление градиентов,
  расходимость ветров, трансформация воздушных масс.
\item
  \textbf{Роль бароклинности:} Атмосферные фронты являются бароклинными
  системами, где изобарические и изотермические поверхности
  пересекаются. Бароклинность (горизонтальные контрасты температуры и
  влажности, или геострофическая адвекция виртуальной температуры)
  играет определяющую роль в возникновении и эволюции вихрей (циклонов и
  антициклонов) и фронтов. Например, адвекция холодного воздуха
  способствует циклогенезу, а адвекция теплого -- антициклогенезу.
  Вертикальные движения воздуха в циклоне обусловлены бароклинностью и
  турбулентным обменом.
\item
  \textbf{Влияние орографии:} Горы могут оказывать существенное влияние
  на фронты, вызывая их деформацию, усиление или ослабление. Существует
  ``топографический фронтогенез'' или ``береговой эффект'', при котором
  сходимость ветра вдоль побережья (из-за различий в турбулентном трении
  над сушей и морем) способствует усилению ветра и образованию местных
  фронтов, особенно перед холодными фронтами, подходящими к берегу.
\end{itemize}

\textbf{Связь с циклонами и антициклонами:} Перемещения воздушных масс и
фронтов тесно связаны с характером и эволюцией барического поля, то есть
с циклонической деятельностью. Циклоны умеренных широт (фронтальные)
обычно возникают на атмосферных фронтах, особенно на медленных холодных
и стационарных фронтах, реже у точки окклюзии, и очень редко на теплых
фронтах. Антициклоны образуются в холодной воздушной массе недалеко от
линии фронта. Возникновение волны на фронте является первым признаком
начавшегося циклогенеза. Погода в различных секторах циклона
определяется взаимодействием воздушных масс, вовлеченных в циркуляцию, и
связанных с ними фронтов.

Таким образом, воздушные массы и фронты -- это не просто теоретические
конструкты, а динамически взаимодействующие системы, определяющие всю
сложность и многообразие погодных явлений. Их изучение и прогнозирование
требуют глубокого понимания гидротермодинамических процессов в
атмосфере.

ewpage

\hypertarget{ux443ux440ux430ux432ux43dux435ux43dux438ux435-ux441ux43eux441ux442ux43eux44fux43dux438ux44f-ux441ux443ux445ux43eux433ux43e-ux438-ux432ux43bux430ux436ux43dux43eux433ux43e-ux432ux43eux437ux434ux443ux445ux430}{%
\subsubsection{Уравнение состояния сухого и влажного
воздуха}\label{ux443ux440ux430ux432ux43dux435ux43dux438ux435-ux441ux43eux441ux442ux43eux44fux43dux438ux44f-ux441ux443ux445ux43eux433ux43e-ux438-ux432ux43bux430ux436ux43dux43eux433ux43e-ux432ux43eux437ux434ux443ux445ux430}}

Атмосферный воздух рассматривается в метеорологии как сжимаемая среда, и
его термодинамическое состояние описывается основными параметрами:
плотностью (\(\rho\)), давлением (\(P\)) и абсолютной температурой
(\(T\)). Абсолютная температура (\(T\)) связана с температурой по шкале
Цельсия (\(t\)) соотношением \(T = t + 273.15\) K.

\textbf{1. Уравнение состояния сухого воздуха:} Сухой воздух, по сути,
представляет собой смесь идеальных газов с постоянными объемными
концентрациями основных компонентов до высоты примерно 100 км, таких как
азот (78\%), кислород (21\%), аргон (0.9\%), углекислый газ (0.036\%) и
другие. Для сухого воздуха справедливо уравнение состояния идеального
газа, известное как уравнение Клапейрона-Менделеева, которое в
метеорологии обычно записывается в удельной форме: \(P = \rho R T\) где
\(R\) -- удельная газовая постоянная сухого воздуха. Для сухого воздуха
ее значение составляет приблизительно \(R = 287.1\) Дж/(кг·К).

\textbf{2. Уравнение состояния влажного воздуха:} В атмосфере всегда
присутствует то или иное количество водяного пара, который находится в
смеси с сухим воздухом. Водяной пар является единственным газом в
атмосфере, который может переходить в жидкую и твердую фазы при
наблюдаемых на Земле температурах. Согласно закону Дальтона, общее
давление смеси газов равно сумме парциальных давлений каждого газа, а
плотность смеси равна сумме плотностей составляющих газов. Применительно
к влажному воздуху, состоящему из сухого воздуха и водяного пара, мы
имеем: \(P = P_c + e\) (где \(P_c\) -- парциальное давление сухого
воздуха, \(e\) -- парциальное давление водяного пара)
\(\rho = \rho_c + \rho_n\) (где \(\rho_c\) -- плотность сухого воздуха,
\(\rho_n\) -- плотность водяного пара) Уравнения состояния для сухого
воздуха и водяного пара по отдельности можно записать как:
\(P_c = \rho_c R T\) \(e = \rho_n R_n T\) (где \(R_n\) -- удельная
газовая постоянная водяного пара, \(R_n = 461.5\) Дж/(кг·К))

Сложив эти два уравнения, получаем уравнение состояния влажного воздуха:
\(P = (\rho_c + \rho_n)T R\) Однако, для практических расчетов,
поскольку удельная газовая постоянная влажного воздуха варьируется в
зависимости от содержания водяного пара, целесообразно преобразовать
уравнение так, чтобы в нем использовалась только удельная газовая
постоянная сухого воздуха. Отношение плотности водяного пара к плотности
влажного воздуха называется удельной влажностью (\(q\)):
\(q = \rho_n / \rho\). С учетом этого, уравнение состояния влажного
воздуха принимает вид: \(P = \rho R T (1 + 0.608 q)\) (в некоторых
источниках коэффициент может быть округлен до 0.61)

\hypertarget{ux432ux438ux440ux442ux443ux430ux43bux44cux43dux430ux44f-ux442ux435ux43cux43fux435ux440ux430ux442ux443ux440ux430}{%
\subsubsection{Виртуальная
температура}\label{ux432ux438ux440ux442ux443ux430ux43bux44cux43dux430ux44f-ux442ux435ux43cux43fux435ux440ux430ux442ux443ux440ux430}}

Для удобства использования газовой постоянной сухого воздуха (\(R\)) для
влажного воздуха вводится понятие \textbf{виртуальной температуры}
(\(T_v\)). Виртуальная температура определяется выражением:
\(T_v = T(1 + 0.608 q)\) Если заменить действительную температуру
(\(T\)) на виртуальную (\(T_v\)), то уравнение состояния влажного
воздуха принимает вид, аналогичный уравнению состояния сухого воздуха:
\(P = \rho R T_v\)

\textbf{Физический смысл виртуальной температуры:} Виртуальная
температура -- это такая температура, которую должен был бы иметь сухой
воздух при данном давлении, чтобы его плотность была такой же, как у
рассматриваемого влажного воздуха. Поскольку водяной пар имеет меньшую
молекулярную массу, чем средняя молекулярная масса сухого воздуха
(отношение газовых постоянных \(R_n/R \approx 1.608\)), влажный воздух
при той же температуре и давлении легче сухого воздуха. Следовательно,
для влажного воздуха виртуальная температура всегда выше действительной
температуры (\(T_v > T\)). Это напрямую следует из определения: если
\(q > 0\), то \(1 + 0.608q > 1\), и \(T_v\) будет больше \(T\).

Использование виртуальной температуры упрощает динамические и
термодинамические расчеты, так как позволяет использовать единую газовую
постоянную \(R\) для всех атмосферных процессов, независимо от
влажности. Например, она важна при расчетах барометрического
нивелирования, поскольку скорость убывания давления с высотой зависит от
средней температуры слоя воздуха, и в более теплом столбе (или с большей
виртуальной температурой) давление падает медленнее. Атмосфера всегда и
везде бароклинна, и уравнение состояния, включающее виртуальную
температуру, явно это демонстрирует: плотность зависит от температуры
сильнее, чем от давления.

\hypertarget{ux445ux430ux440ux430ux43aux442ux435ux440ux438ux441ux442ux438ux43aux438-ux432ux43bux430ux436ux43dux43eux433ux43e-ux432ux43eux437ux434ux443ux445ux430-ux438-ux441ux432ux44fux437ux44c-ux43cux435ux436ux434ux443-ux43dux438ux43cux438-1}{%
\subsubsection{Характеристики влажного воздуха и связь между
ними}\label{ux445ux430ux440ux430ux43aux442ux435ux440ux438ux441ux442ux438ux43aux438-ux432ux43bux430ux436ux43dux43eux433ux43e-ux432ux43eux437ux434ux443ux445ux430-ux438-ux441ux432ux44fux437ux44c-ux43cux435ux436ux434ux443-ux43dux438ux43cux438-1}}

Погода, включая такие элементы как температура и влажность воздуха,
определяется пространственным распределением метеорологических величин.
Характеристики влажности количественно определяют содержание водяного
пара в воздухе. К основным из них относятся:

\begin{enumerate}
\def\labelenumi{\arabic{enumi}.}
\item
  \textbf{Парциальное давление водяного пара (\(e\))} (часто называется
  «упругостью водяного пара»): Давление, создаваемое молекулами водяного
  пара в воздухе. Является важнейшей характеристикой влажности,
  измеряется в гектопаскалях (гПа), миллибарах (мбар) или миллиметрах
  ртутного столба (мм рт. ст.). Чем больше разность между давлением
  насыщенного водяного пара и парциальным давлением, тем суше воздух.
  Парциальное давление и абсолютная влажность меняются при нагревании
  или охлаждении воздуха, хоть и не очень сильно.
\item
  \textbf{Давление насыщенного водяного пара (\(E\) или \(E_a\))} (также
  «упругость насыщения» или максимальное давление): Максимально
  возможное парциальное давление водяного пара при данной температуре
  воздуха. Оно зависит только от температуры: чем выше температура, тем
  больше водяного пара может содержаться в воздухе до наступления
  насыщения. Приближение фактического давления водяного пара к давлению
  насыщения указывает на близость воздуха к состоянию насыщения.
\item
  \textbf{Абсолютная влажность (\(\rho_n\) или \(a\))}: Масса водяного
  пара в граммах, содержащаяся в 1 м\(^3\) воздуха (г/м\(^3\)). Это
  просто плотность водяного пара.
\item
  \textbf{Массовая доля водяного пара (\(q\))}: Безразмерная величина,
  представляющая отношение массы водяного пара к массе влажного воздуха
  в том же объеме. Выражается в промилле (‰) или г\(_{вп}\)/кг\(_{вв}\).
  Она очень близка к отношению смеси. До начала конденсации, удельная
  влажность и отношение смеси остаются практически постоянными.
\item
  \textbf{Отношение смеси (\(r\))}: Отношение массы водяного пара к
  массе сухого воздуха в том же объеме. Также безразмерная величина,
  выражаемая в г\(_{вп}\)/кг\(_{св}\).
\item
  \textbf{Относительная влажность (\(f\))}: Отношение фактического
  парциального давления водяного пара в атмосфере (\(e\)) к давлению
  насыщенного водяного пара (\(E\)) при той же температуре, выражается в
  процентах (\%). Относительная влажность показывает, насколько воздух
  близок к состоянию насыщения. Она сильнее зависит от изменений
  температуры, чем от изменений влажности воздуха. Когда воздух насыщен,
  относительная влажность равна 100\%.
\item
  \textbf{Температура точки росы (\(T_d\) или \(\tau\))}: Температура,
  до которой необходимо охладить воздух при постоянном давлении, чтобы
  содержащийся в нем водяной пар достиг состояния насыщения и началась
  конденсация. Выражается в градусах Цельсия (°C). На уровне конденсации
  температура поднимающегося воздуха становится равной температуре точки
  росы.
\item
  \textbf{Дефицит точки росы}: Разность между температурой воздуха и
  температурой точки росы, выражается в градусах Цельсия (°C). Чем
  больше эта разность, тем суше воздух.
\item
  \textbf{Дефицит насыщения}: Разность между насыщающей и фактической
  упругостью водяного пара при данных температуре и давлении, выражается
  в гектопаскалях (гПа).
\end{enumerate}

\textbf{Связь между характеристиками:}

\begin{itemize}
\tightlist
\item
  Перечисленные характеристики влажности тесно взаимосвязаны и
  используются для описания влажностного режима атмосферы. Например,
  относительная влажность напрямую связывает парциальное давление
  водяного пара с давлением насыщения, которое, в свою очередь, зависит
  от температуры.
\item
  Точка росы является критической температурой для начала конденсации.
  При охлаждении воздуха до точки росы, даже без изменения общего
  количества водяного пара, достигается насыщение.
\item
  Массовая доля водяного пара и отношение смеси являются более
  «консервативными» характеристиками влажности, так как их можно
  изменить только путем добавления или удаления водяного пара из частицы
  воздуха (до начала конденсации). Это делает их полезными для сравнения
  влажности воздушных масс, находящихся в разных условиях. Парциальное
  давление и абсолютная влажность, хотя и зависят от влажности, также
  изменяются при нагревании или охлаждении воздуха.
\item
  Испарение и конденсация -- это процессы, связанные с фазовыми
  переходами воды в атмосфере, которые напрямую зависят от влажностных
  характеристик и температуры.
\item
  Влажность воздуха оказывает существенное влияние на радиационный режим
  атмосферы (водяной пар является основным поглотителем земной радиации
  в инфракрасном диапазоне), на образование облаков и осадков, а также
  на вертикальную устойчивость атмосферы (через влажно-адиабатический
  градиент температуры, который меньше сухоадиабатического).
\end{itemize}

\textbf{Измерение влажности:} На практике влажность воздуха измеряется
психрометрическим или гигрометрическим методами. Психрометрический метод
основан на использовании сухого и смоченного термометров и
психрометрических таблиц для определения различных характеристик
влажности.

Эти характеристики и их взаимосвязи критически важны для понимания и
прогнозирования погодных явлений, особенно тех, что связаны с фазовыми
переходами воды в атмосфере.

ewpage

\hypertarget{ux431ux43bux43eux43a-1.-ux444ux438ux437ux438ux43aux430-ux430ux442ux43cux43eux441ux444ux435ux440ux44b-1}{%
\section{Блок 1. «Физика
атмосферы»}\label{ux431ux43bux43eux43a-1.-ux444ux438ux437ux438ux43aux430-ux430ux442ux43cux43eux441ux444ux435ux440ux44b-1}}

\hypertarget{ux442ux435ux43cux430-ux441ux442ux430ux442ux438ux43aux430-ux438-ux442ux435ux440ux43cux43eux434ux438ux43dux430ux43cux438ux43aux430-ux430ux442ux43cux43eux441ux444ux435ux440ux44b}{%
\subsection{1.2. Тема «Статика и термодинамика
атмосферы»}\label{ux442ux435ux43cux430-ux441ux442ux430ux442ux438ux43aux430-ux438-ux442ux435ux440ux43cux43eux434ux438ux43dux430ux43cux438ux43aux430-ux430ux442ux43cux43eux441ux444ux435ux440ux44b}}

\hypertarget{ux441ux438ux43bux44b-ux434ux435ux439ux441ux442ux432ux443ux44eux449ux438ux435-ux432-ux430ux442ux43cux43eux441ux444ux435ux440ux435-ux432-ux441ux43eux441ux442ux43eux44fux43dux438ux438-ux440ux430ux432ux43dux43eux432ux435ux441ux438ux44f.-ux443ux440ux430ux432ux43dux435ux43dux438ux435-ux441ux442ux430ux442ux438ux43aux438-ux438-ux435ux433ux43e-ux441ux43bux435ux434ux441ux442ux432ux438ux435}{%
\subsubsection{\texorpdfstring{\textbf{Силы, действующие в атмосфере в
состоянии равновесия. Уравнение статики и его
следствие}}{Силы, действующие в атмосфере в состоянии равновесия. Уравнение статики и его следствие}}\label{ux441ux438ux43bux44b-ux434ux435ux439ux441ux442ux432ux443ux44eux449ux438ux435-ux432-ux430ux442ux43cux43eux441ux444ux435ux440ux435-ux432-ux441ux43eux441ux442ux43eux44fux43dux438ux438-ux440ux430ux432ux43dux43eux432ux435ux441ux438ux44f.-ux443ux440ux430ux432ux43dux435ux43dux438ux435-ux441ux442ux430ux442ux438ux43aux438-ux438-ux435ux433ux43e-ux441ux43bux435ux434ux441ux442ux432ux438ux435}}

В состоянии покоя или при крупномасштабных движениях, где вертикальные
ускорения пренебрежимо малы, атмосфера находится в
\textbf{гидростатическом равновесии}. Это равновесие определяется
балансом двух основных сил, действующих на элементарный объём воздуха по
вертикали:

\begin{enumerate}
\def\labelenumi{\arabic{enumi}.}
\tightlist
\item
  \textbf{Сила тяжести:} Направлена вниз и равна \(g \rho dV\), где
  \(g\) --- ускорение свободного падения, \(\rho\) --- плотность
  воздуха, а \(dV\) --- объём.
\item
  \textbf{Сила вертикального барического градиента:} Возникает из-за
  разности давлений на верхней и нижней гранях объёма. Она направлена
  вверх, в сторону убывания давления, и равна
  \(- \frac{\partial p}{\partial z} dV\).
\end{enumerate}

Условие равновесия этих сил приводит к \textbf{основному уравнению
статики}: \[\frac{\partial p}{\partial z} = -g\rho\] Или, для единицы
массы: \[\frac{1}{\rho} \frac{\partial p}{\partial z} = -g\]
\textbf{Следствие:} Это уравнение показывает, что давление в любой точке
атмосферы равно весу вышележащего столба воздуха единичного сечения.
Проинтегрировав уравнение от высоты \(z\) до верха атмосферы
(\(z_{top}\)), где давление стремится к нулю, получаем:
\[p(z) = \int_{z}^{z_{top}} g\rho(z') dz'\] Это приближение является
одним из самых точных в метеорологии для синоптических и планетарных
масштабов и позволяет использовать давление в качестве вертикальной
координаты.

\hypertarget{ux43fux43eux43dux44fux442ux438ux435-ux43bux43eux43aux430ux43bux44cux43dux43eux439-ux438-ux43fux43eux43bux43dux43eux439-ux43fux440ux43eux438ux437ux432ux43eux434ux43dux43eux439-ux43cux435ux442ux435ux43eux440ux43eux43bux43eux433ux438ux447ux435ux441ux43aux438ux445-ux432ux435ux43bux438ux447ux438ux43d}{%
\subsubsection{\texorpdfstring{\textbf{Понятие локальной и полной
производной метеорологических
величин}}{Понятие локальной и полной производной метеорологических величин}}\label{ux43fux43eux43dux44fux442ux438ux435-ux43bux43eux43aux430ux43bux44cux43dux43eux439-ux438-ux43fux43eux43bux43dux43eux439-ux43fux440ux43eux438ux437ux432ux43eux434ux43dux43eux439-ux43cux435ux442ux435ux43eux440ux43eux43bux43eux433ux438ux447ux435ux441ux43aux438ux445-ux432ux435ux43bux438ux447ux438ux43d}}

При анализе атмосферных полей используются два подхода:

\begin{itemize}
\tightlist
\item
  \textbf{Эйлеров подход:} Наблюдение за изменением метеорологической
  величины \(F\) в фиксированной точке пространства. Скорость этого
  изменения описывается \textbf{локальной (частной) производной по
  времени}, \(\frac{\partial F}{\partial t}\).
\item
  \textbf{Лагранжев подход:} Наблюдение за изменением величины \(F\) при
  следовании за движущейся воздушной частицей. Скорость этого изменения
  описывается \textbf{полной (субстанциональной или индивидуальной)
  производной по времени}, \(\frac{dF}{dt}\).
\end{itemize}

Связь между ними устанавливается через \textbf{оператор полной
производной}:
\[\frac{dF}{dt} = \frac{\partial F}{\partial t} + u\frac{\partial F}{\partial x} + v\frac{\partial F}{\partial y} + w\frac{\partial F}{\partial z} = \frac{\partial F}{\partial t} + \vec{V} \cdot \nabla F\]
\textbf{Физический смысл:} Полное изменение свойства воздушной частицы
(\(\frac{dF}{dt}\)) складывается из локального изменения в точке
(\(\frac{\partial F}{\partial t}\)) и \textbf{адвективного изменения}
(\(\vec{V} \cdot \nabla F\)), которое происходит из-за того, что частица
перемещается в область с другим значением величины \(F\).

\hypertarget{ux43fux43eux43dux44fux442ux438ux435-ux433ux440ux430ux434ux438ux435ux43dux442ux430-ux43cux435ux442ux435ux43eux440ux43eux43bux43eux433ux438ux447ux435ux441ux43aux43eux439-ux432ux435ux43bux438ux447ux438ux43dux44b.-ux431ux430ux440ux438ux447ux435ux441ux43aux438ux439-ux433ux440ux430ux434ux438ux435ux43dux442-ux438-ux431ux430ux440ux438ux447ux435ux441ux43aux430ux44f-ux441ux442ux443ux43fux435ux43dux44c}{%
\subsubsection{\texorpdfstring{\textbf{Понятие градиента
метеорологической величины. Барический градиент и барическая
ступень}}{Понятие градиента метеорологической величины. Барический градиент и барическая ступень}}\label{ux43fux43eux43dux44fux442ux438ux435-ux433ux440ux430ux434ux438ux435ux43dux442ux430-ux43cux435ux442ux435ux43eux440ux43eux43bux43eux433ux438ux447ux435ux441ux43aux43eux439-ux432ux435ux43bux438ux447ux438ux43dux44b.-ux431ux430ux440ux438ux447ux435ux441ux43aux438ux439-ux433ux440ux430ux434ux438ux435ux43dux442-ux438-ux431ux430ux440ux438ux447ux435ux441ux43aux430ux44f-ux441ux442ux443ux43fux435ux43dux44c}}

\textbf{Градиент} скалярной метеорологической величины \(F\) (например,
давления или температуры) --- это вектор (\(\nabla F\)), направленный в
сторону её наибыстрейшего возрастания. Его компоненты --- это частные
производные по осям координат:
\[\nabla F = \frac{\partial F}{\partial x}\vec{i} + \frac{\partial F}{\partial y}\vec{j} + \frac{\partial F}{\partial z}\vec{k}\]
\textbf{Барический градиент (\(\nabla p\))} имеет особое значение. Его
горизонтальная составляющая (\(\nabla_h p\)) направлена от высокого
давления к низкому (с учётом знака ``минус'' в уравнении движения) и
определяет \textbf{силу барического градиента}, являющуюся основной
движущей силой горизонтальных ветров. Вертикальная составляющая
(\(\frac{\partial p}{\partial z}\vec{k}\)) практически полностью
уравновешивается силой тяжести.

\textbf{Барическая ступень} --- это величина, обратная модулю
вертикального барического градиента. Она показывает, на какую высоту
нужно подняться или опуститься, чтобы давление изменилось на 1 гПа. Из
уравнения статики следует, что барическая ступень прямо пропорциональна
температуре и обратно пропорциональна давлению. У поверхности Земли она
составляет примерно 8 м/гПа.

\hypertarget{ux431ux430ux440ux43eux43cux435ux442ux440ux438ux447ux435ux441ux43aux438ux435-ux444ux43eux440ux43cux443ux43bux44b-ux434ux43bux44f-ux440ux430ux437ux43bux438ux447ux43dux44bux445-ux43cux43eux434ux435ux43bux435ux439-ux430ux442ux43cux43eux441ux444ux435ux440ux44b}{%
\subsubsection{\texorpdfstring{\textbf{Барометрические формулы для
различных моделей
атмосферы}}{Барометрические формулы для различных моделей атмосферы}}\label{ux431ux430ux440ux43eux43cux435ux442ux440ux438ux447ux435ux441ux43aux438ux435-ux444ux43eux440ux43cux443ux43bux44b-ux434ux43bux44f-ux440ux430ux437ux43bux438ux447ux43dux44bux445-ux43cux43eux434ux435ux43bux435ux439-ux430ux442ux43cux43eux441ux444ux435ux440ux44b}}

Интегрирование уравнения статики \(dp = -g\rho dz\) с использованием
уравнения состояния \(p = \rho R_d T\) приводит к
\textbf{гипсометрическому уравнению}:
\[z_2 - z_1 = \frac{R_d}{g} \int_{p_2}^{p_1} T_v \frac{dp}{p}\] Для его
решения необходимо знать зависимость \(T_v(p)\). В зависимости от
принятой модели атмосферы получают разные барометрические формулы:

\begin{itemize}
\tightlist
\item
  \textbf{Однородная атмосфера (\(\rho = const\)):} Это простейшая,
  гипотетическая модель, в которой плотность воздуха не меняется с
  высотой. Интегрирование уравнения статики дает линейную зависимость:
  \(p_2 - p_1 = -g\rho(z_2 - z_1)\). Модель физически нереалистична, так
  как воздух сжимаем, но она полезна для введения понятия \textbf{высоты
  однородной атмосферы} \(H = p_0 / (g\rho_0) \approx 8\) км. Эта
  величина представляет собой высоту, которую имела бы атмосфера, если
  бы вся её масса имела постоянную плотность, равную плотности у
  поверхности. \(H\) используется как характерный вертикальный масштаб
  для многих атмосферных процессов.
\item
  \textbf{Изотермическая атмосфера (\(T = const\)):} В этой модели
  предполагается, что температура во всем слое постоянна. Это
  приближение может быть оправдано для некоторых слоев стратосферы.
  Интегрирование гипсометрического уравнения дает: \[
    p_2 = p_1 \exp\left[-\frac{g(z_2 - z_1)}{R_d T}\right]
    \] Эта формула показывает, что давление убывает с высотой
  \textbf{экспоненциально}. Величина \(H_{iso} = R_d T / g\) в этой
  модели также называется шкалой высот и показывает высоту, на которой
  давление уменьшается в \(e\) (≈2.718) раз.
\item
  \textbf{Политропная атмосфера (\(T(z) = T_0 - \gamma z\), где
  \(\gamma = const\)):} Эта модель является более реалистичной, так как
  предполагает линейное изменение температуры с высотой, что хорошо
  описывает условия в тропосфере. Интегрирование приводит к степенной
  зависимости: \[
    p_2 = p_1 \left(\frac{T_2}{T_1}\right)^{\frac{g}{R_d\gamma}} = p_1 \left(1 - \frac{\gamma(z_2-z_1)}{T_1}\right)^{\frac{g}{R_d\gamma}}
    \] Эта формула лежит в основе построения \textbf{Стандартной
  атмосферы}, которая состоит из нескольких слоёв, в каждом из которых
  градиент температуры \(\gamma\) постоянен.
\item
  \textbf{Реальная атмосфера:} В реальной атмосфере вертикальный профиль
  температуры сложен и не описывается простой функцией. Поэтому для
  практических расчётов интеграл в гипсометрическом уравнении
  вычисляется с использованием \textbf{средней виртуальной температуры
  слоя} \(\bar{T_v}\). Формула приобретает вид: \[
    z_2 - z_1 = \frac{R_d \bar{T_v}}{g} \ln\left(\frac{p_1}{p_2}\right)
    \] Это основная формула, используемая на практике. Величина
  \(z_2 - z_1\) называется \textbf{геопотенциальной толщиной} (или
  относительным геопотенциалом) слоя между изобарическими поверхностями
  \(p_1\) и \(p_2\). Формула показывает, что толщина слоя прямо
  пропорциональна его средней температуре: тёплые слои ``раздуты''
  (имеют большую толщину), а холодные --- ``сжаты''. Это является
  основой для построения карт барической топографии и анализа
  термического поля в свободной атмосфере. Практическое использование
  включает \textbf{приведение давления к уровню моря} (расчёт
  гипотетического давления на уровне моря по данным станции,
  расположенной на высоте) и \textbf{барометрическую альтиметрию}
  (определение высоты по измерению давления).
\end{itemize}

\hypertarget{ux438ux437ux43cux435ux43dux435ux43dux438ux435-ux43fux43bux43eux442ux43dux43eux441ux442ux438-ux432ux43eux437ux434ux443ux445ux430-ux441-ux432ux44bux441ux43eux442ux43eux439.-ux441ux442ux430ux43dux434ux430ux440ux442ux43dux430ux44f-ux430ux442ux43cux43eux441ux444ux435ux440ux430}{%
\subsubsection{\texorpdfstring{\textbf{Изменение плотности воздуха с
высотой. Стандартная
атмосфера}}{Изменение плотности воздуха с высотой. Стандартная атмосфера}}\label{ux438ux437ux43cux435ux43dux435ux43dux438ux435-ux43fux43bux43eux442ux43dux43eux441ux442ux438-ux432ux43eux437ux434ux443ux445ux430-ux441-ux432ux44bux441ux43eux442ux43eux439.-ux441ux442ux430ux43dux434ux430ux440ux442ux43dux430ux44f-ux430ux442ux43cux43eux441ux444ux435ux440ux430}}

Плотность воздуха, как и давление, убывает с высотой почти
экспоненциально. Из уравнения состояния \(\rho = p/(R_d T)\). Так как
давление убывает быстрее, чем растёт температура (в стратосфере) или
падает медленнее (в тропосфере), плотность всегда убывает с высотой.
Примерно 90\% массы атмосферы сосредоточено в нижних 16 км.

\textbf{Стандартная атмосфера (СА)} --- это идеализированная,
усреднённая модель вертикального распределения температуры, давления и
плотности. Она используется в качестве единого эталона для калибровки
приборов (например, высотомеров), проектирования летательных аппаратов и
проведения теоретических расчётов. В модели СА (ГОСТ 4401-81, ICAO
Standard Atmosphere) задан профиль температуры, а давление и плотность
рассчитываются по уравнению статики и состояния.

\hypertarget{ux43fux435ux440ux432ux43eux435-ux43dux430ux447ux430ux43bux43e-ux442ux435ux440ux43cux43eux434ux438ux43dux430ux43cux438ux43aux438-ux43fux440ux438ux43cux435ux43dux438ux442ux435ux43bux44cux43dux43e-ux43a-ux430ux442ux43cux43eux441ux444ux435ux440ux435.-ux430ux434ux438ux430ux431ux430ux442ux438ux447ux435ux441ux43aux438ux435-ux43fux440ux43eux446ux435ux441ux441ux44b}{%
\subsubsection{\texorpdfstring{\textbf{Первое начало термодинамики
применительно к атмосфере. Адиабатические
процессы}}{Первое начало термодинамики применительно к атмосфере. Адиабатические процессы}}\label{ux43fux435ux440ux432ux43eux435-ux43dux430ux447ux430ux43bux43e-ux442ux435ux440ux43cux43eux434ux438ux43dux430ux43cux438ux43aux438-ux43fux440ux438ux43cux435ux43dux438ux442ux435ux43bux44cux43dux43e-ux43a-ux430ux442ux43cux43eux441ux444ux435ux440ux435.-ux430ux434ux438ux430ux431ux430ux442ux438ux447ux435ux441ux43aux438ux435-ux43fux440ux43eux446ux435ux441ux441ux44b}}

Первое начало термодинамики (закон сохранения энергии) для единицы массы
воздуха записывается в виде: \[dQ = c_v dT + p d\alpha\] где \(dQ\) ---
приток тепла, \(c_v\) --- удельная теплоёмкость при постоянном объёме,
\(T\) --- температура, \(p\) --- давление, \(\alpha = 1/\rho\) ---
удельный объём. Более удобная для метеорологии форма, использующая
давление: \[dQ = c_p dT - \alpha dp\] где \(c_p\) --- удельная
теплоёмкость при постоянном давлении.

\textbf{Адиабатические процессы} --- это процессы, протекающие без
теплообмена с окружающей средой (\(dQ = 0\)). В атмосфере многие быстрые
процессы (например, вертикальные движения воздуха) можно считать
адиабатическими.

\hypertarget{ux441ux443ux445ux43eux434ux438ux430ux431ux430ux442ux438ux447ux435ux441ux43aux438ux439-ux433ux440ux430ux434ux438ux435ux43dux442.-ux43fux43eux442ux435ux43dux446ux438ux430ux43bux44cux43dux430ux44f-ux442ux435ux43cux43fux435ux440ux430ux442ux443ux440ux430-ux438-ux435ux435-ux441ux432ux43eux439ux441ux442ux432ux430}{%
\subsubsection{\texorpdfstring{\textbf{Суходиабатический градиент.
Потенциальная температура и ее
свойства}}{Суходиабатический градиент. Потенциальная температура и ее свойства}}\label{ux441ux443ux445ux43eux434ux438ux430ux431ux430ux442ux438ux447ux435ux441ux43aux438ux439-ux433ux440ux430ux434ux438ux435ux43dux442.-ux43fux43eux442ux435ux43dux446ux438ux430ux43bux44cux43dux430ux44f-ux442ux435ux43cux43fux435ux440ux430ux442ux443ux440ux430-ux438-ux435ux435-ux441ux432ux43eux439ux441ux442ux432ux430}}

Для сухого адиабатического процесса (\(dQ=0\)) из первого начала
термодинамики следует: \(c_p dT = \alpha dp = \frac{1}{\rho}dp\).
Подставляя сюда уравнение статики \(dp = -g\rho dz\), получаем:
\[\frac{dT}{dz} = -\frac{g}{c_p} \equiv -\gamma_d\]
\textbf{Суходиабатический градиент (\(\gamma_d\))} --- это скорость
падения температуры в поднимающейся частице сухого воздуха. Его значение
постоянно и составляет примерно \textbf{9.8 °C/км}.

\textbf{Потенциальная температура (\(\theta\))} --- это температура,
которую приобрела бы частица воздуха, если бы её адиабатически
переместили на стандартный уровень давления \(p_0\) (обычно 1000 гПа).
\[\theta = T \left(\frac{p_0}{p}\right)^{R_d/c_p}\] \textbf{Свойства:}
\(\theta\) является \textbf{консервативной величиной} при сухих
адиабатических процессах. Это делает её идеальным трассером для
отслеживания движения воздушных масс. Вертикальный градиент \(\theta\)
служит основным критерием статической устойчивости.

\hypertarget{ux43fux435ux440ux432ux43eux435-ux43dux430ux447ux430ux43bux43e-ux442ux435ux440ux43cux43eux434ux438ux43dux430ux43cux438ux43aux438-ux43fux440ux438-ux432ux43bux430ux436ux43dux43eux434ux438ux430ux431ux430ux442ux438ux447ux435ux441ux43aux43eux43c-ux43fux440ux43eux446ux435ux441ux441ux435.-ux432ux43bux430ux436ux43dux43eux434ux438ux430ux431ux430ux442ux438ux447ux435ux441ux43aux438ux439-ux433ux440ux430ux434ux438ux435ux43dux442}{%
\subsubsection{\texorpdfstring{\textbf{Первое начало термодинамики при
влажнодиабатическом процессе. Влажнодиабатический
градиент}}{Первое начало термодинамики при влажнодиабатическом процессе. Влажнодиабатический градиент}}\label{ux43fux435ux440ux432ux43eux435-ux43dux430ux447ux430ux43bux43e-ux442ux435ux440ux43cux43eux434ux438ux43dux430ux43cux438ux43aux438-ux43fux440ux438-ux432ux43bux430ux436ux43dux43eux434ux438ux430ux431ux430ux442ux438ux447ux435ux441ux43aux43eux43c-ux43fux440ux43eux446ux435ux441ux441ux435.-ux432ux43bux430ux436ux43dux43eux434ux438ux430ux431ux430ux442ux438ux447ux435ux441ux43aux438ux439-ux433ux440ux430ux434ux438ux435ux43dux442}}

Если поднимающийся воздух достигает насыщения, начинается конденсация.
Выделение \textbf{скрытого тепла парообразования (\(L_v\))} является
неадиабатическим источником тепла: \(dQ = -L_v dq_s\), где \(q_s\) ---
удельная влажность насыщения. Первое начало термодинамики принимает вид:
\[c_p dT - \alpha dp = -L_v dq_s\] Этот процесс называется
\textbf{влажноадиабатическим (или насыщенно-адиабатическим)}.
Выделяющееся тепло компенсирует часть адиабатического охлаждения,
поэтому температура в поднимающейся насыщенной частице падает медленнее.
Скорость падения температуры называется \textbf{влажноадиабатическим
градиентом (\(\gamma_s\))}. В отличие от \(\gamma_d\), \(\gamma_s\)
\textbf{не является постоянной величиной}. Она сильно зависит от
температуры и давления:

\begin{itemize}
\tightlist
\item
  При высоких температурах (в тёплом, влажном воздухе) \(\gamma_s\)
  значительно меньше \(\gamma_d\) (может быть до 4 °C/км), так как при
  конденсации выделяется много скрытого тепла.
\item
  При низких температурах (в холодном, сухом воздухе) \(\gamma_s\)
  приближается к \(\gamma_d\), так как содержание влаги и,
  соответственно, выделение тепла невелики.
\end{itemize}

\hypertarget{ux43fux441ux435ux432ux434ux43eux434ux438ux430ux431ux430ux442ux438ux447ux435ux441ux43aux438ux435-ux43fux440ux43eux446ux435ux441ux441ux44b.-ux44dux43aux432ux438ux432ux430ux43bux435ux43dux442ux43dux43e-ux43fux43eux442ux435ux43dux446ux438ux430ux43bux44cux43dux430ux44f-ux438-ux43fux441ux435ux432ux434ux43eux43fux43eux442ux435ux43dux446ux438ux430ux43bux44cux43dux430ux44f-ux442ux435ux43cux43fux435ux440ux430ux442ux443ux440ux430}{%
\subsubsection{\texorpdfstring{\textbf{Псевдодиабатические процессы.
Эквивалентно-потенциальная и псевдопотенциальная
температура}}{Псевдодиабатические процессы. Эквивалентно-потенциальная и псевдопотенциальная температура}}\label{ux43fux441ux435ux432ux434ux43eux434ux438ux430ux431ux430ux442ux438ux447ux435ux441ux43aux438ux435-ux43fux440ux43eux446ux435ux441ux441ux44b.-ux44dux43aux432ux438ux432ux430ux43bux435ux43dux442ux43dux43e-ux43fux43eux442ux435ux43dux446ux438ux430ux43bux44cux43dux430ux44f-ux438-ux43fux441ux435ux432ux434ux43eux43fux43eux442ux435ux43dux446ux438ux430ux43bux44cux43dux430ux44f-ux442ux435ux43cux43fux435ux440ux430ux442ux443ux440ux430}}

\textbf{Псевдоадиабатический процесс} --- это упрощённая модель
влажноадиабатического процесса, в которой предполагается, что все
сконденсировавшиеся капли воды или кристаллы льда немедленно выпадают из
воздушной частицы. Это делает процесс необратимым.

\textbf{Эквивалентно-потенциальная температура (\(\theta_e\))} --- это
температура, которую примет частица, если всю содержащуюся в ней влагу
сконденсировать (выделившееся тепло нагреет частицу), а затем
адиабатически привести её к уровню 1000 гПа. \textbf{Свойства:}
\(\theta_e\) является \textbf{консервативной величиной} как при сухих,
так и при влажных адиабатических процессах. Это делает её универсальным
трассером и важнейшим параметром для анализа \textbf{конвективной
(потенциальной) неустойчивости}.

\hypertarget{ux43fux43eux43dux44fux442ux438ux435-ux43e-ux43dux435ux430ux434ux438ux430ux431ux430ux442ux438ux447ux435ux441ux43aux438ux445-ux43fux440ux43eux446ux435ux441ux441ux430ux445}{%
\subsubsection{\texorpdfstring{\textbf{Понятие о неадиабатических
процессах}}{Понятие о неадиабатических процессах}}\label{ux43fux43eux43dux44fux442ux438ux435-ux43e-ux43dux435ux430ux434ux438ux430ux431ux430ux442ux438ux447ux435ux441ux43aux438ux445-ux43fux440ux43eux446ux435ux441ux441ux430ux445}}

Это процессы, при которых происходит теплообмен воздушной частицы с
окружающей средой (\(dQ \neq 0\)). Они изменяют потенциальную
температуру частицы. Основные неадиабатические процессы:

\begin{enumerate}
\def\labelenumi{\arabic{enumi}.}
\tightlist
\item
  \textbf{Радиационный теплообмен:} Нагрев за счёт поглощения солнечной
  радиации и охлаждение за счёт излучения длинноволновой радиации.
\item
  \textbf{Выделение/поглощение скрытого тепла:} При конденсации,
  испарении, замерзании и таянии.
\item
  \textbf{Турбулентный теплообмен:} Перенос тепла за счёт турбулентного
  перемешивания с окружающей средой, особенно в пограничном слое.
\end{enumerate}

\hypertarget{ux438ux437ux43cux435ux43dux435ux43dux438ux435-ux43fux430ux440ux430ux43cux435ux442ux440ux43eux432-ux432ux43eux437ux434ux443ux448ux43dux43eux439-ux447ux430ux441ux442ux438ux446ux44b-ux43fux440ux438-ux435ux435-ux432ux435ux440ux442ux438ux43aux430ux43bux44cux43dux44bux445-ux43fux435ux440ux435ux43cux435ux449ux435ux43dux438ux44fux445.-ux43aux440ux438ux432ux430ux44f-ux441ux43eux441ux442ux43eux44fux43dux438ux44f}{%
\subsubsection{\texorpdfstring{\textbf{Изменение параметров воздушной
частицы при ее вертикальных перемещениях. Кривая
состояния}}{Изменение параметров воздушной частицы при ее вертикальных перемещениях. Кривая состояния}}\label{ux438ux437ux43cux435ux43dux435ux43dux438ux435-ux43fux430ux440ux430ux43cux435ux442ux440ux43eux432-ux432ux43eux437ux434ux443ux448ux43dux43eux439-ux447ux430ux441ux442ux438ux446ux44b-ux43fux440ux438-ux435ux435-ux432ux435ux440ux442ux438ux43aux430ux43bux44cux43dux44bux445-ux43fux435ux440ux435ux43cux435ux449ux435ux43dux438ux44fux445.-ux43aux440ux438ux432ux430ux44f-ux441ux43eux441ux442ux43eux44fux43dux438ux44f}}

\textbf{Кривая состояния} --- это линия на аэрологической диаграмме,
показывающая изменение температуры воздушной частицы при её
адиабатическом подъёме от поверхности. Она строится следующим образом:

\begin{enumerate}
\def\labelenumi{\arabic{enumi}.}
\tightlist
\item
  От начальной точки (температура и точка росы у поверхности) частица
  поднимается вдоль \textbf{сухой адиабаты} (линии постоянной
  \(\theta\)).
\item
  На \textbf{уровне конденсации} (LCL), где температура частицы
  становится равной её точке росы, начинается конденсация.
\item
  Выше LCL частица поднимается вдоль \textbf{влажной адиабаты} (линии
  постоянной \(\theta_e\)).
\end{enumerate}

\hypertarget{ux443ux440ux43eux432ux435ux43dux44c-ux43aux43eux43dux434ux435ux43dux441ux430ux446ux438ux438.-ux443ux440ux43eux432ux435ux43dux44c-ux43aux43eux43dux432ux435ux43aux446ux438ux438.-ux44dux43dux435ux440ux433ux438ux44f-ux43dux435ux443ux441ux442ux43eux439ux447ux438ux432ux43eux441ux442ux438}{%
\subsubsection{\texorpdfstring{\textbf{Уровень конденсации. Уровень
конвекции. Энергия
неустойчивости}}{Уровень конденсации. Уровень конвекции. Энергия неустойчивости}}\label{ux443ux440ux43eux432ux435ux43dux44c-ux43aux43eux43dux434ux435ux43dux441ux430ux446ux438ux438.-ux443ux440ux43eux432ux435ux43dux44c-ux43aux43eux43dux432ux435ux43aux446ux438ux438.-ux44dux43dux435ux440ux433ux438ux44f-ux43dux435ux443ux441ux442ux43eux439ux447ux438ux432ux43eux441ux442ux438}}

\begin{itemize}
\tightlist
\item
  \textbf{Уровень поднятой конденсации (LCL):} Высота, на которой
  температура поднимаемой от поверхности частицы становится равной её
  точке росы. Это основание конвективных облаков.
\item
  \textbf{Уровень свободной конвекции (LFC):} Высота, на которой
  температура поднимаемой частицы (по кривой состояния) становится выше
  температуры окружающей среды. Выше этого уровня частица получает
  положительную плавучесть и ускоряется вверх без внешнего воздействия.
\item
  \textbf{Уровень равновесия (EL):} Высота, на которой температура
  частицы снова сравнивается с температурой окружения. Это примерная
  высота вершины конвективного облака.
\item
  \textbf{Энергия неустойчивости (CAPE - Convective Available Potential
  Energy):} Интеграл положительной плавучести (разности температур
  частицы и среды) от LFC до EL. CAPE пропорциональна площади на
  аэрологической диаграмме между кривой состояния и кривой
  стратификации. Является мерой потенциальной интенсивности конвекции и
  гроз.
\end{itemize}

\hypertarget{ux430ux44dux440ux43eux43bux43eux433ux438ux447ux435ux441ux43aux430ux44f-ux434ux438ux430ux433ux440ux430ux43cux43cux430-ux43fux440ux438ux43dux446ux438ux43fux44b-ux43fux43eux441ux442ux440ux43eux435ux43dux438ux44f-ux438-ux438ux441ux43fux43eux43bux44cux437ux43eux432ux430ux43dux438ux44f}{%
\subsubsection{\texorpdfstring{\textbf{Аэрологическая диаграмма:
принципы построения и
использования}}{Аэрологическая диаграмма: принципы построения и использования}}\label{ux430ux44dux440ux43eux43bux43eux433ux438ux447ux435ux441ux43aux430ux44f-ux434ux438ux430ux433ux440ux430ux43cux43cux430-ux43fux440ux438ux43dux446ux438ux43fux44b-ux43fux43eux441ux442ux440ux43eux435ux43dux438ux44f-ux438-ux438ux441ux43fux43eux43bux44cux437ux43eux432ux430ux43dux438ux44f}}

\textbf{Аэрологическая (термодинамическая) диаграмма} --- это
графический инструмент для анализа вертикальной структуры атмосферы и
оценки её устойчивости. Наиболее распространена диаграмма \textbf{Skew-T
log-P}.

\begin{itemize}
\tightlist
\item
  \textbf{Оси:} Вертикальная ось --- давление (в логарифмическом
  масштабе), горизонтальная --- температура. Изотермы наклонены под
  углом 45° (Skew-T).
\item
  \textbf{Линии:} На диаграмму нанесены семейства линий: изобары
  (горизонтальные), изотермы (наклонные), сухие адиабаты (слегка
  изогнутые), влажные адиабаты (сильно изогнутые) и линии постоянного
  отношения смеси.
\item
  \textbf{Использование:} На диаграмму наносят данные радиозонда
  (температуру и точку росы) --- \textbf{кривую стратификации}. Затем,
  поднимая частицу от поверхности, строят \textbf{кривую состояния}.
  Сравнивая эти две кривые, определяют LCL, LFC, EL, рассчитывают CAPE и
  другие индексы неустойчивости, что является основой для прогноза
  конвективных явлений.
\end{itemize}

\hypertarget{ux441ux442ux440ux430ux442ux438ux444ux438ux43aux430ux446ux438ux44f-ux430ux442ux43cux43eux441ux444ux435ux440ux44b.-ux43aux440ux438ux442ux435ux440ux438ux438-ux43eux446ux435ux43dux43aux438-ux432ux435ux440ux442ux438ux43aux430ux43bux44cux43dux43eux439-ux442ux435ux440ux43cux438ux447ux435ux441ux43aux43eux439-ux443ux441ux442ux43eux439ux447ux438ux432ux43eux441ux442ux438-ux43cux435ux442ux43eux434-ux447ux430ux441ux442ux438ux446ux44b}{%
\subsubsection{\texorpdfstring{\textbf{Стратификация атмосферы. Критерии
оценки вертикальной термической устойчивости (метод
частицы)}}{Стратификация атмосферы. Критерии оценки вертикальной термической устойчивости (метод частицы)}}\label{ux441ux442ux440ux430ux442ux438ux444ux438ux43aux430ux446ux438ux44f-ux430ux442ux43cux43eux441ux444ux435ux440ux44b.-ux43aux440ux438ux442ux435ux440ux438ux438-ux43eux446ux435ux43dux43aux438-ux432ux435ux440ux442ux438ux43aux430ux43bux44cux43dux43eux439-ux442ux435ux440ux43cux438ux447ux435ux441ux43aux43eux439-ux443ux441ux442ux43eux439ux447ux438ux432ux43eux441ux442ux438-ux43cux435ux442ux43eux434-ux447ux430ux441ux442ux438ux446ux44b}}

\textbf{Стратификация} --- это вертикальное распределение температуры в
атмосфере, описываемое \textbf{кривой стратификации}.
\textbf{Устойчивость} оценивается \textbf{методом частицы}: мысленно
смещают воздушную частицу по вертикали и сравнивают её температуру с
температурой окружающей среды.

\begin{itemize}
\tightlist
\item
  \textbf{Абсолютно устойчивая стратификация:} Вертикальный градиент
  температуры среды (\(\gamma\)) меньше влажноадиабатического
  (\(\gamma < \gamma_s\)). Смещённая частица всегда холоднее (плотнее)
  окружения и стремится вернуться обратно. Подавляет вертикальные
  движения.
\item
  \textbf{Абсолютно неустойчивая стратификация:} Градиент среды больше
  сухоадиабатического (\(\gamma > \gamma_d\)). Смещённая частица всегда
  теплее (легче) окружения и продолжает ускоряться. Приводит к развитию
  бурной конвекции.
\item
  \textbf{Условно-неустойчивая стратификация:} Градиент среды находится
  между влажно- и сухоадиабатическим (\(\gamma_s < \gamma < \gamma_d\)).
  Атмосфера устойчива для сухих (ненасыщенных) частиц, но неустойчива
  для насыщенных. Это наиболее распространённое состояние атмосферы, при
  котором возможны мощные грозы, если существует механизм, способный
  поднять воздух до уровня свободной конвекции.
\end{itemize}

ewpage

\hypertarget{ux441ux438ux43bux44b-ux434ux435ux439ux441ux442ux432ux443ux44eux449ux438ux435-ux432-ux430ux442ux43cux43eux441ux444ux435ux440ux435}{%
\subsubsection{Силы, действующие в
атмосфере}\label{ux441ux438ux43bux44b-ux434ux435ux439ux441ux442ux432ux443ux44eux449ux438ux435-ux432-ux430ux442ux43cux43eux441ux444ux435ux440ux435}}

При изучении атмосферных движений, динамическая метеорология
рассматривает атмосферу как сплошную среду, и применяет к ней общие
законы гидродинамики и термодинамики. Все силы, действующие в атмосфере,
подразделяются на две основные категории: массовые и поверхностные. Эти
силы в совокупности с тепло- и влагообменом являются основными
факторами, определяющими погоду и климат.

\hypertarget{ux43cux430ux441ux441ux43eux432ux44bux435-ux441ux438ux43bux44b}{%
\paragraph{Массовые
силы}\label{ux43cux430ux441ux441ux43eux432ux44bux435-ux441ux438ux43bux44b}}

Массовые силы --- это силы, приложенные ко всем точкам сплошной среды,
независимо от того, находятся ли эти точки на поверхности,
ограничивающей среду, или внутри нее. В атмосфере к массовым силам
относятся сила тяжести и отклоняющая сила вращения Земли (сила
Кориолиса).

\begin{itemize}
\tightlist
\item
  \textbf{Сила тяжести:} Эта сила притяжения Земли действует на каждую
  частицу воздуха. В первом приближении ускорение свободного падения,
  обусловленное силой тяжести, можно считать постоянным, хотя оно
  незначительно изменяется в зависимости от широты и высоты. В
  сферической системе координат сила тяжести в любой точке пространства
  направлена по радиус-вектору к центру Земли.
\item
  \textbf{Сила Кориолиса:} Это инерционная сила, возникающая вследствие
  переносного вращательного движения Земли вокруг своей оси и
  одновременного движения частиц воздуха относительно земной
  поверхности. Она действует на движущиеся относительно Земли частицы
  воздуха, то есть на воздушные течения. Ускорение Кориолиса,
  испытываемое движущейся частицей воздуха, определяется двумя
  факторами: разностью абсолютных величин и разностью направлений
  линейной скорости вращательного движения различных точек земной
  поверхности, над которыми проходит данная частица. Величина силы
  Кориолиса зависит от угловой скорости вращения Земли и относительной
  скорости движения частицы воздуха. Для единицы массы воздуха сила
  Кориолиса выражается как удвоенное векторное произведение вектора
  угловой скорости вращения Земли на вектор скорости относительного
  движения (скорости ветра). На экваторе влияние вращения Земли
  приближается к нулю, тогда как в высоких и умеренных широтах оно
  наиболее существенно.
\end{itemize}

\hypertarget{ux43fux43eux432ux435ux440ux445ux43dux43eux441ux442ux43dux44bux435-ux441ux438ux43bux44b}{%
\paragraph{Поверхностные
силы}\label{ux43fux43eux432ux435ux440ux445ux43dux43eux441ux442ux43dux44bux435-ux441ux438ux43bux44b}}

В отличие от массовых сил, поверхностные силы действуют в одной и той же
точке на различные площадки в различных направлениях, характеризуя
силовое взаимодействие между различными частицами воздуха. К
поверхностным силам относятся:

\begin{itemize}
\tightlist
\item
  \textbf{Сила барического градиента:} Это сила, возникающая из-за
  неравномерного распределения давления в атмосфере. Давление в
  атмосфере не только отражает вес вышележащего столба воздуха, но и
  действует во все стороны одинаково, как показал Паскаль. Разность
  давления приводит воздух в движение. Сила барического градиента
  направлена от области высокого давления к области низкого давления и
  является одной из ключевых сил, определяющих движение воздуха.
\item
  \textbf{Сила трения (вязкости):} В атмосфере действуют также силы
  трения, обусловленные вязкостью воздуха. В метеорологии касательные
  напряжения поверхностных сил, связанные с деформацией частицы воздуха,
  могут быть выражены через кинематический коэффициент вязкости воздуха
  и плотность воздуха. Влиянием сил молекулярной вязкости воздуха на
  атмосферные движения обычно можно пренебречь, поскольку числа
  Рейнольдса для атмосферных движений очень велики. Однако в пограничном
  слое атмосферы, ограниченном высотой 1.0--1.5 км, трение о земную
  поверхность существенно уменьшает скорость воздушных течений и
  изменяет их направление. Вязкость воздуха также играет роль в
  турбулентном обмене, который характеризуется беспорядочным движением
  множества отдельных частиц и вихрей, приводящих к перемешиванию и
  турбулентной вязкости.
\end{itemize}

\hypertarget{ux443ux441ux43bux43eux432ux438ux435-ux440ux430ux432ux43dux43eux432ux435ux441ux438ux44f-ux432-ux430ux442ux43cux43eux441ux444ux435ux440ux435-ux441ux442ux430ux442ux438ux43aux430-ux430ux442ux43cux43eux441ux444ux435ux440ux44b}{%
\subsubsection{Условие равновесия в атмосфере (Статика
атмосферы)}\label{ux443ux441ux43bux43eux432ux438ux435-ux440ux430ux432ux43dux43eux432ux435ux441ux438ux44f-ux432-ux430ux442ux43cux43eux441ux444ux435ux440ux435-ux441ux442ux430ux442ux438ux43aux430-ux430ux442ux43cux43eux441ux444ux435ux440ux44b}}

В условиях равновесия, или стационарного движения, силы, действующие на
частицу воздуха, уравновешивают друг друга, и ускорение обращается в
нуль, что означает, что скорость не меняется, и тело либо покоится, либо
движется равномерно.

Наиболее важным примером равновесия в атмосфере является
\textbf{гидростатическое равновесие}, которое описывает баланс сил в
вертикальном направлении. В этом случае сила тяжести, направленная вниз,
уравновешивается вертикальной составляющей силы барического градиента,
направленной вверх. Уравнение статики позволяет оценить потенциальную
энергию воздуха в атмосферном столбе и является основным для вычисления
давления в зависимости от высоты. Прогностическая модель атмосферы,
использующая уравнение статики, получается в так называемом
квазистатическом приближении, которое предполагает, что вертикальные
ускорения пренебрежимо малы.

Важно отметить, что равновесие действующих сил в атмосфере является
неустойчивым. Атмосферные движения преимущественно нестационарны, так
как постоянно возникающие ускорения изменяют характер воздушных течений.
Неравномерное нагревание различных частей земного шара и связанные с
этим барические градиенты являются первопричиной атмосферных движений.

ewpage

Коллега, давайте рассмотрим уравнение статики и его следствия, поскольку
это фундаментальный аспект динамической метеорологии, который мы
постоянно используем в нашей работе.

\hypertarget{ux443ux440ux430ux432ux43dux435ux43dux438ux435-ux441ux442ux430ux442ux438ux43aux438-ux441ux443ux442ux44c-ux438-ux444ux43eux440ux43cux443ux43bux438ux440ux43eux432ux43aux430}{%
\subsubsection{Уравнение Статики: Суть и
Формулировка}\label{ux443ux440ux430ux432ux43dux435ux43dux438ux435-ux441ux442ux430ux442ux438ux43aux438-ux441ux443ux442ux44c-ux438-ux444ux43eux440ux43cux443ux43bux438ux440ux43eux432ux43aux430}}

Уравнение статики описывает условие вертикального равновесия атмосферы,
то есть состояние, когда силы, действующие на объём воздуха в
вертикальном направлении, сбалансированы. В идеализированной неподвижной
атмосфере это равновесие обеспечивается балансом двух основных сил:

\begin{enumerate}
\def\labelenumi{\arabic{enumi}.}
\tightlist
\item
  \textbf{Сила тяжести}: Действует вниз на каждый объём воздуха,
  пропорциональна его массе. Для элементарного объёма воздуха плотностью
  \$\rho \$ и высотой \$\Delta z \$ с площадью основания 1 м², сила
  тяжести равна \$ g \rho \Delta z \$, где \$ g \$ --- ускорение
  свободного падения.
\item
  \textbf{Сила градиента давления}: Действует вверх и является
  результатом разности сил давления на нижнюю и верхнюю границы
  рассматриваемого объёма воздуха. Если давление на нижней границе \$ P
  \$, а на верхней \$ P + \Delta P \$, то сила градиента давления равна
  \$ -(P + \Delta P - P) = -\Delta P \$.
\end{enumerate}

Таким образом, условие равновесия выражается равенством этих двух сил:
\$ g \rho \Delta z = -\Delta P \$.

Переходя к дифференциальной форме, получаем классическое уравнение
статики: \$\frac{dP}{dz} = -g \rho \$.

Здесь \$ dP \$ --- изменение давления с высотой \$ dz \$, а знак минус
указывает на то, что давление уменьшается с увеличением высоты.

\hypertarget{ux43aux432ux430ux437ux438ux441ux442ux430ux442ux438ux447ux435ux441ux43aux43eux435-ux43fux440ux438ux431ux43bux438ux436ux435ux43dux438ux435}{%
\subsubsection{Квазистатическое
Приближение}\label{ux43aux432ux430ux437ux438ux441ux442ux430ux442ux438ux447ux435ux441ux43aux43eux435-ux43fux440ux438ux431ux43bux438ux436ux435ux43dux438ux435}}

Хотя уравнение статики строго справедливо для неподвижной атмосферы, оно
с высокой степенью точности выполняется и в движущейся атмосфере,
особенно для крупномасштабных процессов. Это обусловлено тем, что в
крупномасштабных движениях нарушение статичности наблюдается лишь в
отдельных, быстро приспосабливающихся случаях. В связи с этим в
динамической метеорологии принято говорить о \textbf{квазистатичности}
атмосферных процессов, а не о строгой статичности.

Это приближение играет ключевую роль в упрощении общих уравнений
гидродинамики применительно к атмосфере, особенно вертикальной
составляющей уравнения количества движения. В изобарической системе
координат, например, третье уравнение движения сводится к уравнению
статики.

\hypertarget{ux441ux43bux435ux434ux441ux442ux432ux438ux44f-ux443ux440ux430ux432ux43dux435ux43dux438ux44f-ux441ux442ux430ux442ux438ux43aux438}{%
\subsubsection{Следствия Уравнения
Статики}\label{ux441ux43bux435ux434ux441ux442ux432ux438ux44f-ux443ux440ux430ux432ux43dux435ux43dux438ux44f-ux441ux442ux430ux442ux438ux43aux438}}

Уравнение статики имеет ряд важных следствий и применений, которые лежат
в основе многих метеорологических расчетов и концепций:

\begin{enumerate}
\def\labelenumi{\arabic{enumi}.}
\item
  \textbf{Барометрическая Формула}: Основное назначение уравнения
  статики --- вычисление давления в зависимости от высоты. Для этого
  плотность \$\rho \$ исключается из уравнения статики с помощью
  уравнения состояния идеального газа. Поскольку атмосферные газы можно
  считать идеальными (температура и давление далеки от тройной точки),
  уравнение состояния имеет вид \$ P = \rho R T \$, где \$ R \$ ---
  удельная газовая постоянная воздуха, а \$ T \$ --- абсолютная
  температура. Подставляя \$\rho = P/(RT) \$ в уравнение статики,
  получаем: \$\frac{dP}{dz} = -g \frac{P}{RT} \$ или \$\frac{dP}{P} =
  -\frac{g}{RT} dz \$. Интегрируя это выражение, мы получаем различные
  формы барометрической формулы, связывающие давление с высотой и
  температурой, при различных допущениях об изменении температуры и
  плотности с высотой.
\item
  \textbf{Зависимость Давления от Температуры и Высоты}: Из
  барометрической формулы следует, что в тёплом воздухе давление падает
  с высотой медленнее, чем в холодном. Это означает, что на определённой
  высоте давление в столбе тёплого воздуха будет выше, чем в столбе
  холодного воздуха, если у поверхности они имели одинаковое давление.
  Это является фундаментальным принципом, объясняющим, например,
  формирование высотных барических систем. Однако, как отмечают
  источники, в реальной атмосфере это может быть перекрыто более мощными
  динамическими (например, бароклинными) факторами.
\item
  \textbf{Основа для Численного Прогноза Погоды}: Уравнение статики
  является неотъемлемой частью полной системы уравнений
  гидротермодинамики, используемой для численного прогноза погоды. Оно
  входит в систему уравнений движения атмосферы, позволяя упростить
  вертикальный баланс сил. Хотя оно упрощает систему, это приводит к
  тому, что в изобарической системе координат ``теряются'' бароклинные
  члены, что влияет на описание вихревых движений в вертикальных
  плоскостях. Тем не менее, квазистатическое приближение позволяет
  эффективно моделировать крупномасштабные процессы.
\item
  \textbf{Связь с Вертикальными Движениями}: Уравнение статики описывает
  равновесное состояние окружающей атмосферы. При вертикальных
  перемещениях частицы воздуха, изолированной от окружающей среды
  (например, при адиабатических процессах), её ускорение будет зависеть
  от силы тяжести и барического градиента внутри самой частицы, а также
  от свойств окружающей атмосферы, описываемых уравнением статики.
\end{enumerate}

Таким образом, уравнение статики -- это краеугольный камень динамической
метеорологии, позволяющий нам количественно анализировать вертикальное
распределение давления и температуры, и формирующий основу для сложных
прогностических моделей, несмотря на свою идеализированную природу.

ewpage

Коллега, давайте углубимся в важные понятия локальной и полной
производной метеорологических величин. Это краеугольный камень
динамической метеорологии, позволяющий нам описывать изменения
атмосферных параметров как в фиксированной точке, так и при движении
воздушной частицы.

\hypertarget{ux43fux43eux43dux44fux442ux438ux435-ux43bux43eux43aux430ux43bux44cux43dux43eux439-ux438-ux43fux43eux43bux43dux43eux439-ux43fux440ux43eux438ux437ux432ux43eux434ux43dux43eux439-ux43cux435ux442ux435ux43eux440ux43eux43bux43eux433ux438ux447ux435ux441ux43aux438ux445-ux432ux435ux43bux438ux447ux438ux43d-1}{%
\subsubsection{Понятие локальной и полной производной метеорологических
величин}\label{ux43fux43eux43dux44fux442ux438ux435-ux43bux43eux43aux430ux43bux44cux43dux43eux439-ux438-ux43fux43eux43bux43dux43eux439-ux43fux440ux43eux438ux437ux432ux43eux434ux43dux43eux439-ux43cux435ux442ux435ux43eux440ux43eux43bux43eux433ux438ux447ux435ux441ux43aux438ux445-ux432ux435ux43bux438ux447ux438ux43d-1}}

Изменения любой метеорологической величины, обозначаемой как \(\phi\), с
течением времени можно рассматривать с двух принципиально разных точек
зрения. Это различие крайне важно для формулирования и решения уравнений
гидротермодинамики, которые лежат в основе теоретической метеорологии.

\hypertarget{ux43bux43eux43aux430ux43bux44cux43dux430ux44f-ux43cux435ux441ux442ux43dux430ux44f-ux43fux440ux43eux438ux437ux432ux43eux434ux43dux430ux44f}{%
\paragraph{Локальная (местная)
производная}\label{ux43bux43eux43aux430ux43bux44cux43dux430ux44f-ux43cux435ux441ux442ux43dux430ux44f-ux43fux440ux43eux438ux437ux432ux43eux434ux43dux430ux44f}}

\textbf{Локальное (или местное) изменение} величины \(\phi\) --- это
изменение ее значения в неподвижной точке пространства относительно
выбранной системы отсчета, в то время как через эту точку проходят
различные частицы воздуха. Если метеорологическая величина \(\phi\)
является функцией времени (\(t\)) и пространственных координат
(\(x, y, z\)), то есть \(\phi = \phi(x, y, z, t)\), то локальное
изменение во времени определяется частной производной по времени,
обозначаемой \(\frac{\partial \phi}{\partial t}\).

Например, когда мы измеряем температуру воздуха на метеорологической
станции, мы получаем локальное изменение температуры. Мы видим, как
температура изменяется в данной точке, не отслеживая при этом конкретную
воздушную частицу. Практическое значение локального изменения в
метеорологии очень велико, поскольку большинство наших наблюдений, таких
как данные со станций, носят именно локальный характер.

\hypertarget{ux43fux43eux43bux43dux430ux44f-ux438ux43dux434ux438ux432ux438ux434ux443ux430ux43bux44cux43dux430ux44f-ux43fux440ux43eux438ux437ux432ux43eux434ux43dux430ux44f}{%
\paragraph{Полная (индивидуальная)
производная}\label{ux43fux43eux43bux43dux430ux44f-ux438ux43dux434ux438ux432ux438ux434ux443ux430ux43bux44cux43dux430ux44f-ux43fux440ux43eux438ux437ux432ux43eux434ux43dux430ux44f}}

\textbf{Индивидуальное изменение} величины \(\phi\) --- это изменение ее
значения, которое происходит в одной и той же, движущейся воздушной
частице. Поскольку координаты этой частицы (\(x, y, z\)) изменяются со
временем, значение величины \(\phi\) для движущейся частицы является
сложной функцией времени. Следовательно, индивидуальное изменение во
времени величины \(\phi\) определяется как \textbf{полная производная по
времени} от \(\phi\). Поэтому полная производная по времени также
называется \textbf{индивидуальной производной}.

Математически, если \(\phi(x, y, z, t)\) --- сложная функция, где
\(x, y, z\) сами являются функциями времени \(t\) (т.е., \(x(t)\),
\(y(t)\), \(z(t)\)), то полная производная по времени выражается как:

\$\frac{d\phi}{dt} = \frac{\partial \phi}{\partial t} +
\frac{\partial \phi}{\partial x} \frac{dx}{dt} +
\frac{\partial \phi}{\partial y} \frac{dy}{dt} +
\frac{\partial \phi}{\partial z} \frac{dz}{dt} \$

Здесь \(\frac{dx}{dt}\), \(\frac{dy}{dt}\), \(\frac{dz}{dt}\) --- это
составляющие скорости движения воздушной частицы в направлениях
соответствующих осей: \(u = \frac{dx}{dt}\), \(v = \frac{dy}{dt}\),
\(w = \frac{dz}{dt}\).

Подставляя эти обозначения скоростей, мы получаем окончательное
выражение для индивидуального изменения метеорологической величины
\(\phi\):

\$\frac{d\phi}{dt} = \frac{\partial \phi}{\partial t} + u
\frac{\partial \phi}{\partial x} + v \frac{\partial \phi}{\partial y} +
w \frac{\partial \phi}{\partial z} \$

Это уравнение является фундаментальным в динамической метеорологии.

\hypertarget{ux444ux438ux437ux438ux447ux435ux441ux43aux438ux439-ux441ux43cux44bux441ux43b-ux447ux43bux435ux43dux43eux432-ux443ux440ux430ux432ux43dux435ux43dux438ux44f}{%
\paragraph{Физический смысл членов
уравнения}\label{ux444ux438ux437ux438ux447ux435ux441ux43aux438ux439-ux441ux43cux44bux441ux43b-ux447ux43bux435ux43dux43eux432-ux443ux440ux430ux432ux43dux435ux43dux438ux44f}}

Каждый член в правой части уравнения полной производной имеет свой
физический смысл:

\begin{itemize}
\item
  \textbf{\(\frac{\partial \phi}{\partial t}\) (локальное изменение):}
  Как уже говорилось, это изменение величины \(\phi\) в фиксированной
  точке пространства. Оно показывает, как изменяется значение \(\phi\)
  во времени, не учитывая движение воздуха через эту точку.
\item
  \textbf{\(u \frac{\partial \phi}{\partial x} + v \frac{\partial \phi}{\partial y} + w \frac{\partial \phi}{\partial z}\)
  (адвективное и конвективное изменение):} Этот набор членов описывает
  изменение величины \(\phi\) в движущейся частице воздуха, вызванное ее
  перемещением из одной точки пространства в другую, где значения
  \(\phi\) отличаются. Этот блок называют \textbf{адвективным членом}
  или \textbf{скоростным переносом}.

  \begin{itemize}
  \tightlist
  \item
    Члены \(u \frac{\partial \phi}{\partial x}\) и
    \(v \frac{\partial \phi}{\partial y}\) представляют собой
    \textbf{горизонтальную адвекцию} (перенос) величины \(\phi\),
    вызванную горизонтальным движением воздуха (\(u\) и \(v\) -
    горизонтальные составляющие скорости).
  \item
    Член \(w \frac{\partial \phi}{\partial z}\) представляет собой
    \textbf{вертикальную адвекцию} (или конвекцию) величины \(\phi\),
    связанную с вертикальными движениями воздуха (\(w\) - вертикальная
    составляющая скорости). В метеорологии, локальное изменение величины
    в данной точке, обусловленное перемещением частиц воздуха,
    подразделяется на адвективное (горизонтальный перенос) и
    конвективное (вертикальные движения).
  \end{itemize}
\end{itemize}

Таким образом, полная производная (\(d\phi/dt\)) описывает полное
изменение метеорологической величины для конкретной воздушной частицы,
которое включает в себя как изменение в фиксированной точке, так и
изменение, вызванное движением этой частицы через неоднородное поле
данной величины. Понимание этой взаимосвязи является ключевым для
решения задач прогноза погоды и анализа атмосферных процессов.

ewpage

Коллега, рад, что мы углубляемся в следующую часть. Понимание понятия
градиента метеорологической величины, особенно барического градиента и
барической ступени, является краеугольным камнем в динамической
метеорологии и играет ключевую роль в анализе и прогнозе атмосферных
процессов.

\hypertarget{ux43fux43eux43dux44fux442ux438ux435-ux43cux435ux442ux435ux43eux440ux43eux43bux43eux433ux438ux447ux435ux441ux43aux43eux433ux43e-ux433ux440ux430ux434ux438ux435ux43dux442ux430}{%
\subsubsection{Понятие Метеорологического
Градиента}\label{ux43fux43eux43dux44fux442ux438ux435-ux43cux435ux442ux435ux43eux440ux43eux43bux43eux433ux438ux447ux435ux441ux43aux43eux433ux43e-ux433ux440ux430ux434ux438ux435ux43dux442ux430}}

В динамической метеорологии мы часто оперируем скалярными полями, такими
как поля температуры, давления и влажности воздуха. Для количественного
анализа изменений этих величин в пространстве вводится понятие
градиента. Градиент скалярной величины \(\phi\) -- это вектор, который
указывает направление наибольшего роста этой величины, а его модуль
равен производной по этому направлению.

Математически градиент \(\phi\) в декартовой системе координат с
единичными векторами \(\vec{i}, \vec{j}, \vec{k}\) вдоль осей
\(x, y, z\) соответственно, может быть представлен как векторная сумма
его составляющих:

\(\vec{\nabla}\phi = \frac{\partial\phi}{\partial x}\vec{i} + \frac{\partial\phi}{\partial y}\vec{j} + \frac{\partial\phi}{\partial z}\vec{k}\)
(1.2.4)

Абсолютная величина градиента, по сути, представляет собой длину
диагонали прямоугольного параллелепипеда, ребра которого равны проекциям
градиента на координатные оси:

\(|\vec{\nabla}\phi| = \sqrt{\left(\frac{\partial\phi}{\partial x}\right)^2 + \left(\frac{\partial\phi}{\partial y}\right)^2 + \left(\frac{\partial\phi}{\partial z}\right)^2}\)
(1.2.5)

Для краткости и удобства часто используется символический вектор
``набла'' (\(\nabla\)), или дифференциальный оператор Гамильтона. В
метеорологической практике, особенно при работе с дискретно заданными
полями (например, на картах погоды или в узлах расчетной сетки),
производные для вычисления градиента определяются методами численного
дифференцирования, используя конечные разности, например, метод
центральных разностей.

\hypertarget{ux431ux430ux440ux438ux447ux435ux441ux43aux438ux439-ux433ux440ux430ux434ux438ux435ux43dux442}{%
\subsubsection{Барический
Градиент}\label{ux431ux430ux440ux438ux447ux435ux441ux43aux438ux439-ux433ux440ux430ux434ux438ux435ux43dux442}}

Переходя к конкретному случаю, \textbf{барический градиент} (или
градиент давления) является ключевой векторной величиной,
характеризующей степень неоднородности поля давления в атмосфере. Его
направление определяется от высокого давления к низкому, то есть
противоположно вектору \(\nabla p\) в математическом смысле, где
\(\nabla p\) обычно указывает на наибольший рост давления. Модуль
горизонтального градиента давления оценивается путем деления разности
давлений между двумя точками на прямой, перпендикулярной к изобаре, на
расстояние между этими точками.

Иными словами, чем гуще нанесены изобары на карте (линии одинакового
давления, проводимые обычно через 5 гПа), тем больше значение
барического градиента. В циклонической области барический градиент
направлен внутрь, к центру низкого давления, а в антициклонической ---
наружу, от центра высокого давления.

Важно отметить, что сам по себе градиент давления является силой,
действующей на единицу площади. В метеорологии часто оперируют силой
градиента давления (G), отнесенной к единичной массе, которая выражается
формулой:

\(\vec{G} = -\frac{1}{\rho}\nabla p\)

где \(\rho\) --- плотность воздуха. Это фундаментальная сила, приводящая
воздух в движение, и именно она ответственна за возникновение ветра при
наличии неравномерного распределения давления.

Изменение барического градиента с высотой также имеет большое значение.
Если температура воздушного столба однородна по горизонтали, то с ростом
высоты барический градиент ослабевает. Однако, при наличии термической
неоднородности (различий в температуре по горизонтали), роль этой
неоднородности в определении значения барического градиента возрастает с
увеличением высоты. Это объясняет, почему сильные барические градиенты,
а следовательно, и сильные ветры, могут наблюдаться на больших высотах,
даже если у поверхности земли они слабее.

\hypertarget{ux431ux430ux440ux438ux447ux435ux441ux43aux430ux44f-ux441ux442ux443ux43fux435ux43dux44c}{%
\subsubsection{Барическая
Ступень}\label{ux431ux430ux440ux438ux447ux435ux441ux43aux430ux44f-ux441ux442ux443ux43fux435ux43dux44c}}

Понятие \textbf{барической ступени} тесно связано с вертикальным
распределением давления и температуры в атмосфере. Барическая ступень
характеризует вертикальное расстояние между двумя соседними
изобарическими поверхностями. Это расстояние не является постоянным и
сильно зависит от температуры воздушной массы.

Согласно уравнению статики, которое связывает давление с высотой и
температурой, при одной и той же разности давлений (например, между
двумя изобарическими поверхностями) высота столба воздуха между ними
будет больше в более теплой воздушной массе, чем в более холодной.
Следовательно, в теплой воздушной массе барическая ступень (т.е.
расстояние между изобарическими поверхностями по вертикали) будет
больше, чем в холодной. Это обусловлено тем, что в теплом воздухе
давление с высотой падает медленнее.

Этот принцип имеет критическое значение для понимания
\textbf{бароклинности атмосферы}. В бароклинной атмосфере изобарические
(постоянного давления) и изотермические (постоянной температуры)
поверхности пересекаются, образуя так называемые термодинамические
соленоиды. Чем больше таких соленоидов (то есть чем больше
горизонтальные градиенты температуры и давления), тем выше
бароклинность. В зонах атмосферных фронтов, где горизонтальные градиенты
температуры и давления наиболее велики, наблюдается наибольшее сгущение
термодинамических соленоидов, что способствует усилению циркуляции и
увеличению скоростей ветра.

\hypertarget{ux432ux43bux438ux44fux43dux438ux435-ux43dux430-ux430ux442ux43cux43eux441ux444ux435ux440ux43dux44bux435-ux43fux440ux43eux446ux435ux441ux441ux44b}{%
\subsubsection{Влияние на Атмосферные
Процессы}\label{ux432ux43bux438ux44fux43dux438ux435-ux43dux430-ux430ux442ux43cux43eux441ux444ux435ux440ux43dux44bux435-ux43fux440ux43eux446ux435ux441ux441ux44b}}

Барический градиент является основной движущей силой атмосферных
движений. В свободной атмосфере, где влияние силы трения относительно
мало, он в значительной степени уравновешивается силой Кориолиса,
приводя к формированию \textbf{геострофического ветра}, который дует
вдоль изобар, а не поперек них. Понимание этого баланса сил позволяет
метеорологам по полю изобар на картах давления оценить направление и
скорость ветра даже в тех районах, где прямые измерения отсутствуют.

Неравномерное распределение давления, вызванное процессами теплообмена,
является первопричиной движения воздуха. Горизонтальная бароклинность,
обусловленная различиями в температуре и плотности, играет важную роль в
зарождении и эволюции синоптических вихрей, таких как циклоны и
антициклоны. Например, адвекция холода (перенос более холодной воздушной
массы) или тепла способствует зарождению или усилению циклонических или
антициклонических вихрей соответственно.

Эти фундаментальные концепции --- градиент метеорологической величины,
барический градиент и барическая ступень --- лежат в основе всех
прогностических моделей, будь то численные гидродинамические модели или
традиционный синоптический анализ. Их количественная оценка необходима
для понимания текущего состояния атмосферы (диагноз) и предсказания ее
будущих изменений (прогноз).

ewpage

Коллега, продолжим наше обсуждение фундаментальных аспектов динамической
метеорологии. Сегодня мы рассмотрим барометрические формулы, которые
являются прямым следствием уравнения статики и имеют колоссальное
практическое значение в нашей работе.

\hypertarget{ux431ux430ux440ux43eux43cux435ux442ux440ux438ux447ux435ux441ux43aux438ux435-ux444ux43eux440ux43cux443ux43bux44b-ux442ux435ux43eux440ux435ux442ux438ux447ux435ux441ux43aux438ux435-ux43eux441ux43dux43eux432ux44b-ux438-ux43fux440ux430ux43aux442ux438ux447ux435ux441ux43aux43eux435-ux43fux440ux438ux43cux435ux43dux435ux43dux438ux435}{%
\subsubsection{Барометрические Формулы: Теоретические Основы и
Практическое
Применение}\label{ux431ux430ux440ux43eux43cux435ux442ux440ux438ux447ux435ux441ux43aux438ux435-ux444ux43eux440ux43cux443ux43bux44b-ux442ux435ux43eux440ux435ux442ux438ux447ux435ux441ux43aux438ux435-ux43eux441ux43dux43eux432ux44b-ux438-ux43fux440ux430ux43aux442ux438ux447ux435ux441ux43aux43eux435-ux43fux440ux438ux43cux435ux43dux435ux43dux438ux435}}

Барометрические формулы представляют собой интегралы основного уравнения
статики, полученные при различных допущениях относительно изменения
плотности (\(\rho\)) и температуры (\(T\)) воздуха с высотой. Они
позволяют нам количественно описывать распределение давления в атмосфере
по вертикали.

\hypertarget{ux443ux440ux430ux432ux43dux435ux43dux438ux435-ux441ux442ux430ux442ux438ux43aux438-ux43aux430ux43a-ux444ux443ux43dux434ux430ux43cux435ux43dux442}{%
\paragraph{Уравнение Статики как
Фундамент}\label{ux443ux440ux430ux432ux43dux435ux43dux438ux435-ux441ux442ux430ux442ux438ux43aux438-ux43aux430ux43a-ux444ux443ux43dux434ux430ux43cux435ux43dux442}}

Напомним, что основным уравнением статики, описывающим вертикальное
равновесие атмосферы, является: \$\frac{dP}{dz} = -g \rho \$. В
динамической метеорологии, особенно для крупномасштабных процессов, мы
используем его в квазистатическом приближении, поскольку даже в
движущейся атмосфере оно сохраняет высокую степень точности.

Для практического применения этой формулы необходимо исключить плотность
(\(\rho\)) с помощью уравнения состояния атмосферного воздуха. Уравнение
состояния для сухого воздуха имеет вид \(P = \rho RT\), где \(R\) ---
удельная газовая постоянная сухого воздуха, а \(T\) --- абсолютная
температура. Для влажного воздуха используется концепция виртуальной
температуры \(T_v = T(1 + 0.61q)\), где \(q\) --- массовая доля водяного
пара. Виртуальную температуру нельзя измерить напрямую, её можно только
вычислить. Таким образом, плотность можно выразить как
\(\rho = \frac{P}{RT_v}\). Подставляя это выражение для плотности в
уравнение статики, мы получаем: \$\frac{dP}{P} = -\frac{g}{RT_v} dz \$.
Интегрирование этого уравнения позволяет получить различные
барометрические формулы в зависимости от того, какие предположения
делаются относительно изменения температуры \(T_v\) с высотой \(z\).

\hypertarget{ux43cux43eux434ux435ux43bux44c-ux43eux434ux43dux43eux440ux43eux434ux43dux43eux439-ux433ux43eux43cux43eux433ux435ux43dux43dux43eux439-ux430ux442ux43cux43eux441ux444ux435ux440ux44b}{%
\paragraph{1. Модель Однородной (Гомогенной)
Атмосферы}\label{ux43cux43eux434ux435ux43bux44c-ux43eux434ux43dux43eux440ux43eux434ux43dux43eux439-ux433ux43eux43cux43eux433ux435ux43dux43dux43eux439-ux430ux442ux43cux43eux441ux444ux435ux440ux44b}}

Эта модель является наиболее простой и идеализированной. В гомогенной
атмосфере предполагается, что плотность воздуха \(\rho\) постоянна по
всей высоте. При этом температура также должна быть постоянной или
изменяться таким образом, чтобы плотность оставалась неизменной. Если
\(\rho = \text{const}\), то интегрирование уравнения статики
\(dP = -g \rho dz\) от уровня \(z_1\) (с давлением \(P_1\)) до уровня
\(z_2\) (с давлением \(P_2\)) даёт: \$ P\_2 - P\_1 = -g \rho (z\_2 -
z\_1) \$ Отсюда: \$ P\_2 = P\_1 - g \rho (z\_2 - z\_1)
\(. Эта формула показывает, что давление линейно уменьшается с высотой. Концепция **высоты однородной атмосферы** (\)H\_0
= \frac{P}{\rho g}\$) напрямую связана с этой моделью. Она представляет
собой высоту столба воздуха с постоянной плотностью, соответствующей
плотности на данном уровне, который создавал бы наблюдаемое давление.
Для земной поверхности это значение составляет примерно 8000 м. Несмотря
на свою упрощённость, модель однородной атмосферы даёт базовое
представление об убывании давления с высотой и служит масштабом высоты
атмосферы.

\hypertarget{ux43cux43eux434ux435ux43bux44c-ux438ux437ux43eux442ux435ux440ux43cux438ux447ux435ux441ux43aux43eux439-ux430ux442ux43cux43eux441ux444ux435ux440ux44b}{%
\paragraph{2. Модель Изотермической
Атмосферы}\label{ux43cux43eux434ux435ux43bux44c-ux438ux437ux43eux442ux435ux440ux43cux438ux447ux435ux441ux43aux43eux439-ux430ux442ux43cux43eux441ux444ux435ux440ux44b}}

В изотермической атмосфере предполагается, что абсолютная температура
\(T\) (или виртуальная температура \(T_v\)) остаётся постоянной с
высотой (\(T_v = \text{const}\)). Это значительно упрощает
интегрирование, поскольку \(R, g, T_v\) становятся константами.
Интегрируя \$\frac{dP}{P} = -\frac{g}{RT_v} dz \$ от \(P_1\) на \(z_1\)
до \(P_2\) на \(z_2\): \$\int\emph{\{P\_1\}\^{}\{P\_2\} \frac{dP}{P} =
-\frac{g}{RT_v} \int}\{z\_1\}\^{}\{z\_2\} dz \$
\$\ln\left(\frac{P_2}{P_1}\right) = -\frac{g}{RT_v}(z\_2 - z\_1) \$ Или
в экспоненциальной форме: \$ P\_2 = P\_1
\exp\left[-\frac{g}{RT_v}(z_2 - z_1)\right] \(. Эта формула показывает, что давление в изотермической атмосфере убывает с высотой по экспоненциальному закону. Она часто используется для расчетов в слоях, где температура меняется незначительно, или при расчёте средней температуры слоя (\)T\_m\$).

\hypertarget{ux43cux43eux434ux435ux43bux44c-ux43fux43eux43bux438ux442ux440ux43eux43fux43dux43eux439-ux430ux442ux43cux43eux441ux444ux435ux440ux44b}{%
\paragraph{3. Модель Политропной
Атмосферы}\label{ux43cux43eux434ux435ux43bux44c-ux43fux43eux43bux438ux442ux440ux43eux43fux43dux43eux439-ux430ux442ux43cux43eux441ux444ux435ux440ux44b}}

Политропический процесс характеризуется постоянным значением
теплоемкости \(c_\pi\). В атмосфере это часто означает, что температура
изменяется с высотой линейно, т.е. \(T(z) = T_0 - \gamma z\), где
\(\gamma\) --- постоянный вертикальный градиент температуры. Для
политропной атмосферы, используя уравнение состояния и линейное
изменение температуры, можно вывести более сложную барометрическую
формулу. Если \(T = T_0 - \gamma z\), то: \$\frac{dP}{P} =
-\frac{g}{R(T_0 - \gamma z)} dz \$ Интегрирование этого выражения дает:
\$\ln\left(\frac{P_2}{P_1}\right) = \frac{g}{R\gamma}
\ln\left(\frac{T_0 - \gamma z_2}{T_0 - \gamma z_1}\right) \$ \$ P\_2 =
P\_1 \left(\frac{T_2}{T_1}\right)\^{}\{-\frac{g}{R\gamma}\} = P\_1
\left(\frac{T_0 - \gamma z_2}{T_0 - \gamma z_1}\right)\^{}\{-\frac{g}{R\gamma}\}
\$. Эта формула более реалистична, чем изотермическая, поскольку
учитывает типичное уменьшение температуры с высотой в тропосфере. Случаи
изотермии и сухоадиабатического процесса являются частными случаями
политропного процесса.

\hypertarget{ux43cux43eux434ux435ux43bux44c-ux440ux435ux430ux43bux44cux43dux43eux439-ux430ux442ux43cux43eux441ux444ux435ux440ux44b}{%
\paragraph{4. Модель Реальной
Атмосферы}\label{ux43cux43eux434ux435ux43bux44c-ux440ux435ux430ux43bux44cux43dux43eux439-ux430ux442ux43cux43eux441ux444ux435ux440ux44b}}

В реальной атмосфере температура и плотность не являются постоянными и
меняются сложным образом в зависимости от высоты, географического
положения и времени. Атмосферный воздух также содержит переменное
количество водяного пара, что влияет на его плотность и требует
использования виртуальной температуры (\(T_v\)) в уравнении состояния.
Более того, атмосфера является бароклинной, то есть изопикнические,
изобарические и изотермические поверхности пересекаются, что делает её
сложной для аналитического описания.

Для работы с реальной атмосферой барометрические формулы адаптируются
следующим образом:

\begin{itemize}
\tightlist
\item
  \textbf{Использование средней барометрической температуры}: На
  практике для слоя между двумя уровнями \(z_1\) и \(z_2\) вводится
  средняя барометрическая температура слоя \(T_m\). Она может быть
  средней арифметической температурой слоя или более сложной средней
  величиной. В этом случае изотермическая формула преобразуется в: \$
  P\_2 = P\_1 \exp\left[-\frac{g}{RT_m}(z_2 - z_1)\right] \$.
\item
  \textbf{Виртуальная температура}: В \(T_m\) обязательно должна
  учитываться влажность воздуха через виртуальную температуру \(T_v\).
\item
  \textbf{Послойные расчеты}: Атмосфера делится на условные слои, в
  каждом из которых можно применять одну из упрощенных моделей
  (например, политропную или изотермическую), или для каждого слоя
  определяется своя средняя температура. Это позволяет получить более
  точные результаты.
\end{itemize}

\hypertarget{ux43fux440ux430ux43aux442ux438ux447ux435ux441ux43aux43eux435-ux438ux441ux43fux43eux43bux44cux437ux43eux432ux430ux43dux438ux435-ux431ux430ux440ux43eux43cux435ux442ux440ux438ux447ux435ux441ux43aux438ux445-ux444ux43eux440ux43cux443ux43b}{%
\subsubsection{Практическое Использование Барометрических
Формул}\label{ux43fux440ux430ux43aux442ux438ux447ux435ux441ux43aux43eux435-ux438ux441ux43fux43eux43bux44cux437ux43eux432ux430ux43dux438ux435-ux431ux430ux440ux43eux43cux435ux442ux440ux438ux447ux435ux441ux43aux438ux445-ux444ux43eux440ux43cux443ux43b}}

Барометрические формулы имеют фундаментальное значение для решения
множества задач в метеорологии и смежных областях:

\begin{enumerate}
\def\labelenumi{\arabic{enumi}.}
\tightlist
\item
  \textbf{Расчет Высот Изобарических Поверхностей (Альтиметрия)}:

  \begin{itemize}
  \tightlist
  \item
    Основное применение --- это определение геопотенциальных высот
    (\(H\)) стандартных изобарических поверхностей (например, 850, 700,
    500 мбар) относительно уровня моря. Эти высоты наносятся на карты
    абсолютной топографии (АТ-карты), которые являются ключевым
    инструментом в прогностической работе.
  \item
    Формула позволяет вычислить высоту одной изобарической поверхности
    над другой. Зная давление на уровне станции и используя среднюю
    температуру слоя, можно вычислить высоту изобарической поверхности
    над уровнем моря.
  \item
    В авиации барометрические формулы используются для определения
    высоты полёта, что критически важно для безопасности и навигации.
  \end{itemize}
\item
  \textbf{Анализ Бароклинности Атмосферы}:

  \begin{itemize}
  \tightlist
  \item
    Атмосфера всегда бароклинна, что проявляется в пересечении изобар и
    изотерм. Барометрические формулы, особенно их дифференциальная
    форма, позволяют анализировать изменение барического градиента с
    высотой.
  \item
    Наличие термической неоднородности (разницы температур в соседних
    столбах воздуха при равном давлении) приводит к тому, что сила
    барического градиента с высотой заметно зависит от этой
    неоднородности. В верхних слоях атмосферы она направлена
    перпендикулярно изотермам от более теплого воздуха к более
    холодному. Это лежит в основе концепции термического ветра.
  \item
    Бароклинность играет важную роль в формировании синоптических вихрей
    (циклонов и антициклонов) и атмосферных фронтов.
  \end{itemize}
\item
  \textbf{Прогноз Погоды}:

  \begin{itemize}
  \tightlist
  \item
    \textbf{Прогноз синоптического положения}: Прогноз перемещения и
    эволюции воздушных масс, атмосферных фронтов, циклонических и
    антициклонических образований является первым подготовительным
    этапом прогноза погоды.
  \item
    \textbf{Вертикальные движения и облакообразование}: Барометрические
    формулы косвенно используются при расчете вертикальных скоростей,
    которые критически важны для прогноза облачности, осадков, туманов,
    гроз и других явлений. Например, при адиабатическом подъёме воздуха
    происходит его охлаждение, что может привести к конденсации водяного
    пара и образованию облаков.
  \item
    \textbf{Температурный режим}: Формулы также используются для
    прогноза температуры воздуха на различных высотах и стратификации
    атмосферы.
  \item
    \textbf{Анализ устойчивости атмосферы}: Сопоставление вертикального
    градиента температуры с адиабатическими градиентами
    (сухоадиабатическим \(\gamma_a\) и влажноадиабатическим
    \(\gamma_{ва}\)) позволяет определить устойчивость атмосферы и
    предсказать развитие конвекции. Это имеет прямое отношение к
    образованию кучевых облаков и ливневых осадков.
  \item
    \textbf{Энергия неустойчивости}: Энергия неустойчивости, которая
    является показателем потенциала развития конвективных процессов,
    может быть графически определена с помощью аэрологических диаграмм,
    сопоставляя кривые стратификации и состояния.
  \end{itemize}
\item
  \textbf{Авиационная Метеорология}:

  \begin{itemize}
  \tightlist
  \item
    Прогнозы обледенения и турбулентности требуют знания вертикального
    распределения температуры, влажности и ветра, которые выводятся из
    данных, обработанных с использованием барометрических формул.
  \item
    Вся работа по метеорологическому обеспечению гражданской авиации,
    включая определение погодного минимума аэродрома, самолёта, экипажа,
    напрямую зависит от точности определения высоты и вертикальных
    параметров.
  \end{itemize}
\item
  \textbf{Общая Циркуляция Атмосферы}:

  \begin{itemize}
  \tightlist
  \item
    Барометрические формулы и связанные с ними расчеты позволяют
    описывать и изучать крупномасштабные воздушные течения, формирующие
    общую циркуляцию атмосферы.
  \end{itemize}
\end{enumerate}

Таким образом, барометрические формулы, хотя и основаны на
идеализированных моделях, являются незаменимым инструментом в арсенале
метеоролога, обеспечивая как теоретическое понимание атмосферных
процессов, так и их практическое прогнозирование.

ewpage

Коллега, крайне важно для понимания атмосферных процессов детально
рассмотреть, как изменяется плотность воздуха с высотой, и что
представляет собой концепция стандартной атмосферы, которая служит
своего рода референсной точкой для наших расчетов и моделей.

\hypertarget{ux438ux437ux43cux435ux43dux435ux43dux438ux435-ux43fux43bux43eux442ux43dux43eux441ux442ux438-ux432ux43eux437ux434ux443ux445ux430-ux441-ux432ux44bux441ux43eux442ux43eux439}{%
\subsubsection{Изменение плотности воздуха с
высотой}\label{ux438ux437ux43cux435ux43dux435ux43dux438ux435-ux43fux43bux43eux442ux43dux43eux441ux442ux438-ux432ux43eux437ux434ux443ux445ux430-ux441-ux432ux44bux441ux43eux442ux43eux439}}

Как известно, атмосфера Земли представляет собой газовую оболочку,
находящуюся в постоянном движении. Воздух -- это сжимаемая среда, и его
плотность (ρ) является одной из ключевых метеорологических величин,
характеризующих состояние атмосферы.

\textbf{Основные закономерности изменения плотности с высотой:}

\begin{itemize}
\tightlist
\item
  \textbf{Быстрое убывание:} Плотность воздуха, подобно атмосферному
  давлению, быстро уменьшается с высотой. Это обусловлено тем, что
  воздух является сжимаемой средой: нижние слои атмосферы несут на себе
  вес вышележащих слоев, что приводит к их сжатию и, соответственно,
  большей плотности.
\item
  \textbf{Распределение массы:} Большая часть массы атмосферы
  сосредоточена в нижних слоях. Так, не менее половины всей массы
  атмосферы располагается на высотах до 6 км. Более точные данные
  указывают, что около 50\% массы атмосферы сосредоточено в слое
  толщиной 5 км, 90\% --- в слое 16 км, и 99\% --- в слое 32 км. Даже на
  уровне 50-55 км давление воздуха составляет менее 0,0001 от
  приземного, что наглядно демонстрирует быстрое падение плотности.
\item
  \textbf{Зависимость от других параметров:} Плотность воздуха не
  является постоянной величиной и тесно связана с давлением (P) и
  температурой (T) воздуха через уравнение состояния идеального газа:
  \(P = \rho RT\). Здесь \(R\) --- универсальная газовая постоянная,
  или, как мы чаще используем, удельная газовая постоянная для сухого
  воздуха (287 Дж/(кг·К)). Для влажного воздуха используется концепция
  виртуальной температуры (\(T_v\)), что позволяет сохранить удельную
  газовую постоянную сухого воздуха в уравнении состояния.
\item
  \textbf{Закон сохранения массы:} Изменение плотности воздуха в
  движущейся среде тесно связано с распределением скорости движения в
  пространстве и выражается через скалярную величину, называемую
  дивергенцией скорости. Это связь устанавливается уравнением
  неразрывности, которое является выражением закона сохранения массы. В
  общем виде уравнение неразрывности для сжимаемого воздуха имеет вид:
  \(\frac{\partial \rho}{\partial t} + \vec{\nabla} \cdot (\rho \vec{V}) = 0\),
  где \(\vec{V}\) --- вектор скорости воздуха. В случае несжимаемой
  атмосферы, когда плотность постоянна, это уравнение упрощается до
  \(\vec{\nabla} \cdot \vec{V} = 0\), то есть дивергенция скорости равна
  нулю.
\item
  \textbf{Масштабы изменений:} Важно отметить, что масштабы изменения
  плотности в вертикальном направлении значительно превышают изменения в
  горизонтальном.
\end{itemize}

\hypertarget{ux441ux442ux430ux43dux434ux430ux440ux442ux43dux430ux44f-ux430ux442ux43cux43eux441ux444ux435ux440ux430}{%
\subsubsection{Стандартная
атмосфера}\label{ux441ux442ux430ux43dux434ux430ux440ux442ux43dux430ux44f-ux430ux442ux43cux43eux441ux444ux435ux440ux430}}

Поскольку метеорологические величины, такие как давление, температура и
плотность, непрерывно изменяются в пространстве и во времени, для
удобства расчетов, анализа и сравнения данных была введена концепция
\textbf{стандартной атмосферы}. Это идеализированная модель,
представляющая собой вертикальное распределение атмосферных параметров
(температуры, давления, плотности) в среднем, усредненном состоянии.
Хотя источники явно не используют термин ``стандартная атмосфера'' в
контексте ее определения как единой концепции, они предоставляют все
необходимые параметры для ее понимания через описание вертикальной
структуры и свойств атмосферы.

\textbf{Ключевые характеристики стандартной атмосферы:}

\begin{enumerate}
\def\labelenumi{\arabic{enumi}.}
\tightlist
\item
  \textbf{Вертикальное деление на слои:}

  \begin{itemize}
  \tightlist
  \item
    \textbf{Тропосфера:} Нижний слой, простирающийся от земной
    поверхности до высоты 9-17 км (в зависимости от широты и сезона). В
    этом слое сосредоточено более 4/5 всей массы атмосферного воздуха и
    практически весь водяной пар. Температура в тропосфере в среднем
    падает с высотой примерно на 0.65°C на каждые 100 метров, или
    6.5°C/км. Именно здесь формируются основные погодные явления.
  \item
    \textbf{Стратосфера:} Располагается над тропосферой, до высот 50-55
    км. В нижнем слое стратосферы температура практически постоянна (от
    -45°C до -75°C), а в верхнем слое она начинает расти (до
    -20°C\ldots20°C). Стратосфера отделена от тропосферы тропопаузой.
  \item
    \textbf{Мезосфера:} Находится над стратосферой, на высотах от 50 до
    80-85 км. В этом слое температура вновь понижается с высотой,
    достигая значений до -90°C. Мезосфера отделена от стратосферы
    стратопаузой.
  \item
    \textbf{Термосфера:} Самый верхний слой, простирающийся до 600-1000
    км, характеризующийся быстрым повышением температуры с высотой.
  \item
    \textbf{Экзосфера:} Выше термосферы, внешний слой атмосферы, где
    частицы могут покидать гравитационное поле Земли.
  \end{itemize}
\item
  \textbf{Распределение давления с высотой:}

  \begin{itemize}
  \tightlist
  \item
    Давление в атмосфере экспоненциально уменьшается с высотой. Это
    описывается \textbf{уравнением статики атмосферы}, которое выражает
    равновесие сил тяжести и барического градиента в вертикальном
    направлении: \(dP = -g\rho dz\). Здесь \(dP\) --- изменение
    давления, \(g\) --- ускорение свободного падения, \(\rho\) ---
    плотность воздуха, \(dz\) --- изменение высоты.
  \item
    Сочетание уравнения статики с уравнением состояния газа позволяет
    получить \textbf{барометрическую формулу}, которая связывает
    давление с высотой и температурой.
  \item
    Примеры типичных значений давления на различных высотах
    (приближенно): 0 км -- 1013 гПа, 5 км -- 500 гПа, 10 км -- 250 гПа,
    15 км -- 120 гПа, 20 км -- 50 гПа.
  \item
    Для решения инженерных и научных задач иногда используется концепция
    \textbf{высоты однородной атмосферы (H)}, которая представляет собой
    высоту гипотетического слоя воздуха постоянной плотности,
    оказывающего то же давление, что и вся атмосфера. Ее значение
    составляет примерно 8000 м.
  \end{itemize}
\end{enumerate}

Концепция стандартной атмосферы и детальное понимание изменения
плотности воздуха с высотой являются фундаментальными для решения задач
динамической метеорологии, моделирования атмосферных процессов и,
конечно же, для прогнозирования погоды.

ewpage

\hypertarget{ux43fux435ux440ux432ux43eux435-ux43dux430ux447ux430ux43bux43e-ux442ux435ux440ux43cux43eux434ux438ux43dux430ux43cux438ux43aux438-ux43fux440ux438ux43cux435ux43dux438ux442ux435ux43bux44cux43dux43e-ux43a-ux430ux442ux43cux43eux441ux444ux435ux440ux435}{%
\subsection{Первое Начало Термодинамики Применительно к
Атмосфере}\label{ux43fux435ux440ux432ux43eux435-ux43dux430ux447ux430ux43bux43e-ux442ux435ux440ux43cux43eux434ux438ux43dux430ux43cux438ux43aux438-ux43fux440ux438ux43cux435ux43dux438ux442ux435ux43bux44cux43dux43e-ux43a-ux430ux442ux43cux43eux441ux444ux435ux440ux435}}

Динамическая метеорология, как вы знаете, изучает атмосферные процессы,
опираясь на фундаментальные законы физики, включая гидромеханику и
термодинамику. Основным методом исследования здесь является
преобразование и решение общих уравнений гидротермодинамики,
адаптированных к специфическим физическим условиям атмосферы.

Первое начало термодинамики представляет собой экспериментально
установленный исходный принцип, выражающий общий закон сохранения
энергии в применении к термодинамическим процессам. Суть этого закона
заключается в том, что энергия в природе не исчезает и не появляется из
ниоткуда, а лишь трансформируется из одной формы в другую.

Для единицы массы воздуха, занимающей удельный объем \(v\) и имеющей
температуру \(T\), первое начало термодинамики может быть выражено через
изменение внутренней энергии \(dU\), приток тепла \(dQ\) и работу
\(dW\), совершаемую этой массой:

\(dU = dQ - dW\)

Внутренняя энергия единицы массы сухого воздуха (\(dU\)) связана с
изменением абсолютной температуры (\(dT\)) и удельной теплоемкостью при
постоянном объеме (\(c_v\)) следующим образом: \(dU = c_v dT\). Для
сухого воздуха \(c_v\) в большинстве расчетов принимается постоянной и
равной примерно 718 Дж/(кг·К).

Работа расширения или сжатия частицы воздуха (\(dW\)) связана с
изменением объема (\(dV\)) и давлением (\(P\)) соотношением:
\(dW = Pdv\). Важно отметить, что работа, выполняемая частицей,
считается отрицательной, а работа, выполняемая средой над частицей, --
положительной.

Подставляя эти выражения в основную формулу, получаем:
\(dQ = c_v dT + Pdv\).

Это уравнение может быть преобразовано с использованием уравнения
состояния идеального газа (\(Pv = RT\)), где \(R\) --- газовая
постоянная для сухого воздуха (приблизительно 287 Дж/(кг·К)), \(P\) ---
давление, \(v\) --- удельный объем, \(T\) --- абсолютная температура.

Также в метеорологии широко используется форма первого начала
термодинамики, выраженная через удельную теплоемкость при постоянном
давлении (\(c_p\)). Для сухого воздуха \(c_p \approx 1005\) Дж/(кг·К).
Между \(c_p\), \(c_v\) и \(R\) существует соотношение: \(c_p = c_v + R\)
или \(R = c_p - c_v\).

Часто используемая в метеорологии форма первого начала термодинамики,
полученная после подстановки \(Pv=RT\) и дифференцирования в выражение
(3.2.2) (аналогично \(dQ = c_v dT + Pdv\)): \(dQ = c_p dT - vdP\).

Таким образом, уравнения (3.2.1), (3.2.2), (3.2.4) и (3.2.5) из
источника являются математическими выражениями первого начала
термодинамики. Они позволяют количественно анализировать атмосферные
процессы, такие как процессы перехода тепловой энергии в механическую и
обратно.

\hypertarget{ux430ux434ux438ux430ux431ux430ux442ux438ux447ux435ux441ux43aux438ux435-ux43fux440ux43eux446ux435ux441ux441ux44b}{%
\subsection{Адиабатические
Процессы}\label{ux430ux434ux438ux430ux431ux430ux442ux438ux447ux435ux441ux43aux438ux435-ux43fux440ux43eux446ux435ux441ux441ux44b}}

Адиабатическим называется термодинамический процесс, при котором
отсутствует теплообмен между рассматриваемой системой (например,
частицей воздуха) и окружающей средой. Это означает, что \(dQ = 0\).

В атмосфере вертикальные перемещения воздуха часто можно в первом
приближении считать адиабатическими. Это обусловлено тем, что при таких
перемещениях, особенно при значительных перепадах давления, работа
расширения или сжатия воздушной массы значительно превосходит приток
тепла извне.

Для адиабатических процессов первое начало термодинамики принимает вид:
\(c_p dT - vdP = 0\).

Это уравнение совместно с уравнением состояния газа и уравнением статики
атмосферы позволяет описывать изменения состояния воздуха при
вертикальных движениях. В частных случаях адиабатические процессы
являются частными случаями политропических процессов (которые протекают
при постоянном значении теплоемкости \(\pi c\)). Адиабатический процесс
соответствует показателю политропы \(n = k = c_p / c_v\).

\hypertarget{ux441ux443ux445ux43eux430ux434ux438ux430ux431ux430ux442ux438ux447ux435ux441ux43aux438ux439-ux433ux440ux430ux434ux438ux435ux43dux442}{%
\subsubsection{Сухоадиабатический
Градиент}\label{ux441ux443ux445ux43eux430ux434ux438ux430ux431ux430ux442ux438ux447ux435ux441ux43aux438ux439-ux433ux440ux430ux434ux438ux435ux43dux442}}

Сухоадиабатическим вертикальным градиентом температуры (\(\gamma_a\))
называется величина, на которую изменяется температура единицы массы
сухого воздуха при ее адиабатическом перемещении на единицу высоты.

Для определения \(\gamma_a\) используется первое начало термодинамики
для адиабатических процессов в сухом воздухе (\(dQ=0\)) и уравнение
статики атмосферы, которое в квазистатическом приближении выражается как
\(dP = -\rho g dz\).

Из \(c_p dT - vdP = 0\), подставляя \(v=1/\rho\) и \(dP = -\rho g dz\),
получаем: \(c_p dT - (1/\rho)(-\rho g dz) = 0\) \(c_p dT + g dz = 0\)
\(dT/dz = -g/c_p\).

Таким образом, сухоадиабатический вертикальный градиент температуры:
\(\gamma_a = -dT/dz = g/c_p\).

При подстановке численных значений (\(g \approx 9.81\) м/с² и
\(c_p \approx 1005\) Дж/(кг·К)):
\(\gamma_a \approx 9.81 / 1005 \approx 0.00976\) К/м, или примерно
\(0.98\) °С на 100 м, или \(9.8\) °С на 1 км. В метеорологии часто
используется округленное значение \(1\) °С на 100 м, или \(10\) °С на 1
км.

Сухоадиабатический градиент является ключевым показателем для оценки
устойчивости атмосферы.

\hypertarget{ux43fux43eux442ux435ux43dux446ux438ux430ux43bux44cux43dux430ux44f-ux442ux435ux43cux43fux435ux440ux430ux442ux443ux440ux430-ux438-ux435ux435-ux441ux432ux43eux439ux441ux442ux432ux430}{%
\subsection{Потенциальная Температура и ее
Свойства}\label{ux43fux43eux442ux435ux43dux446ux438ux430ux43bux44cux43dux430ux44f-ux442ux435ux43cux43fux435ux440ux430ux442ux443ux440ux430-ux438-ux435ux435-ux441ux432ux43eux439ux441ux442ux432ux430}}

Потенциальная температура (\(\theta\)) --- это температура, которую
приобрела бы единица массы сухого воздуха, если бы она была
адиабатически приведена к стандартному давлению, обычно 1000 гПа (или
1000 мбар).

Выражение для потенциальной температуры (часто называемое формулой
Пуассона) выводится из уравнения адиабатических процессов:
\(\theta = T \left( \frac{1000}{P} \right)^{R/c_p}\). Где \(T\) --
текущая абсолютная температура, \(P\) -- текущее давление в гПа, \(R\)
-- газовая постоянная для сухого воздуха, \(c_p\) -- удельная
теплоемкость при постоянном давлении. Величина \(R/c_p\) приблизительно
равна 0.286.

\hypertarget{ux432ux430ux436ux43dux435ux439ux448ux438ux435-ux441ux432ux43eux439ux441ux442ux432ux430-ux43fux43eux442ux435ux43dux446ux438ux430ux43bux44cux43dux43eux439-ux442ux435ux43cux43fux435ux440ux430ux442ux443ux440ux44b}{%
\subsubsection{Важнейшие Свойства Потенциальной
Температуры}\label{ux432ux430ux436ux43dux435ux439ux448ux438ux435-ux441ux432ux43eux439ux441ux442ux432ux430-ux43fux43eux442ux435ux43dux446ux438ux430ux43bux44cux43dux43eux439-ux442ux435ux43cux43fux435ux440ux430ux442ux443ux440ux44b}}

\begin{enumerate}
\def\labelenumi{\arabic{enumi}.}
\item
  \textbf{Сохранение при сухоадиабатических процессах
  (Консервативность):} Главнейшее свойство потенциальной температуры
  заключается в том, что она не изменяется при сухоадиабатических
  процессах. Это легко доказать, прологарифмировав и продифференцировав
  выражение для \(\theta\):
  \(d(\ln \theta) = d(\ln T) - (R/c_p) d(\ln P)\)
  \(d\theta/\theta = dT/T - (R/c_p) dP/P\) Умножив на \(c_p\):
  \(c_p d\theta/\theta = c_p dT/T - R dP/P\) Используя уравнение
  состояния \(P = \rho RT\) или \(v = RT/P\), и соотношение
  \(R = c_p - c_v\), а также первое начало термодинамики в форме
  \(dQ = c_p dT - vdP\), можно показать, что:
  \(c_p d\theta/\theta = dQ/T\). Таким образом, если процесс
  адиабатический (\(dQ = 0\)), то потенциальная температура остается
  постоянной: \(\theta = \text{const}\). Это делает потенциальную
  температуру исключительно важной ``консервативной характеристикой''
  воздушной массы при ее сухих адиабатических перемещениях, позволяющей
  отслеживать воздушные частицы.
\item
  \textbf{Показатель Устойчивости Атмосферы:} Потенциальная температура
  является прямым индикатором статической устойчивости атмосферы к
  вертикальным смещениям сухого воздуха. Соотношение между изменением
  потенциальной температуры с высотой и сухоадиабатическим градиентом
  выражается так:
  \(\frac{\partial\theta}{\partial z} = \frac{\theta}{T} (\gamma_a - \gamma)\).
  Где \(\gamma = -dT/dz\) -- фактический вертикальный градиент
  температуры окружающей атмосферы.

  \begin{itemize}
  \tightlist
  \item
    \textbf{Устойчивая стратификация:} Если
    \(\partial\theta/\partial z > 0\) (потенциальная температура
    возрастает с высотой), то \(\gamma_a - \gamma > 0\), то есть
    \(\gamma < \gamma_a\). Это означает, что воздух устойчив к
    вертикальным смещениям. Поднимающаяся частица будет охлаждаться
    быстрее окружающей среды и опускаться обратно.
  \item
    \textbf{Неустойчивая стратификация:} Если
    \(\partial\theta/\partial z < 0\) (потенциальная температура убывает
    с высотой), то \(\gamma_a - \gamma < 0\), то есть
    \(\gamma > \gamma_a\). Воздух неустойчив, и поднятая частица будет
    продолжать подниматься, а опущенная -- опускаться.
  \item
    \textbf{Безразличная (нейтральная) стратификация:} Если
    \(\partial\theta/\partial z = 0\) (потенциальная температура
    постоянна с высотой), то \(\gamma_a - \gamma = 0\), то есть
    \(\gamma = \gamma_a\). Воздух находится в безразличном равновесии.
  \end{itemize}
\item
  \textbf{Использование в Аэрологических Диаграммах:} На аэрологических
  диаграммах (например, эмаграммах), изолинии потенциальной температуры
  называются сухими адиабатами и имеют постоянное значение. Это
  значительно упрощает анализ термодинамического состояния атмосферы и
  прогнозирование вертикальных движений.
\end{enumerate}

ewpage

\hypertarget{ux43fux435ux440ux432ux43eux435-ux43dux430ux447ux430ux43bux43e-ux442ux435ux440ux43cux43eux434ux438ux43dux430ux43cux438ux43aux438-ux43fux440ux438-ux432ux43bux430ux436ux43dux43e-ux430ux434ux438ux430ux431ux430ux442ux438ux447ux435ux441ux43aux43eux43c-ux43fux440ux43eux446ux435ux441ux441ux435}{%
\subsubsection{Первое начало термодинамики при влажно-адиабатическом
процессе}\label{ux43fux435ux440ux432ux43eux435-ux43dux430ux447ux430ux43bux43e-ux442ux435ux440ux43cux43eux434ux438ux43dux430ux43cux438ux43aux438-ux43fux440ux438-ux432ux43bux430ux436ux43dux43e-ux430ux434ux438ux430ux431ux430ux442ux438ux447ux435ux441ux43aux43eux43c-ux43fux440ux43eux446ux435ux441ux441ux435}}

Первое начало термодинамики представляет собой фундаментальный закон
сохранения энергии, применимый к термодинамическим процессам. Для
единицы массы воздуха, рассматриваемой как идеальный газ, это начало
может быть выражено в виде: \(dQ = c_v dT + Pdv\), где \(dQ\) -- приток
тепла к единице массы воздуха, \(c_v\) -- удельная теплоемкость при
постоянном объеме, \(dT\) -- приращение температуры, \(P\) -- давление,
и \(dv\) -- приращение удельного объема.

Это уравнение также может быть преобразовано к виду, который чаще
используется в метеорологии, путем замены \(Pdv\) с помощью уравнения
состояния и соотношения Майера (\(c_p = c_v + R\)), что приводит к:
\(dQ = c_p dT - \frac{RT}{P} dP\), где \(c_p\) -- удельная теплоемкость
при постоянном давлении, а \(R\) -- удельная газовая постоянная воздуха.

Влажно-адиабатический процесс, или псевдоадиабатический процесс,
является частным случаем, когда отсутствуют теплообмен с окружающей
средой, но при этом происходит фазовый переход водяного пара
(конденсация или испарение) внутри воздушной массы. Тепло, выделяющееся
при конденсации (или поглощаемое при испарении), воздействует на
температуру внутри воздушной частицы, изменяя ее внутреннюю энергию.
Таким образом, несмотря на отсутствие внешнего притока тепла,
температура в такой частице изменяется не только за счет работы
расширения или сжатия.

Учет этого внутреннего источника/стока тепла от фазовых переходов
водяного пара приводит к следующей форме уравнения первого начала
термодинамики для влажно-адиабатического процесса:
\(c_p dT - \frac{RT}{P} dP + Ldq = 0\). Здесь \(L\) -- удельная теплота
фазового перехода (например, конденсации), а \(dq\) -- изменение
массовой доли водяного пара вследствие этого перехода. Термин \(Ldq\)
представляет собой внутренний ``приток'' тепла, обусловленный
конденсацией (при \(dq < 0\) и выделении тепла) или ``отток'' тепла,
обусловленный испарением (при \(dq > 0\) и поглощении тепла).

\hypertarget{ux432ux43bux430ux436ux43dux43e-ux430ux434ux438ux430ux431ux430ux442ux438ux447ux435ux441ux43aux438ux439-ux433ux440ux430ux434ux438ux435ux43dux442-ux442ux435ux43cux43fux435ux440ux430ux442ux443ux440ux44b}{%
\subsubsection{Влажно-адиабатический градиент
температуры}\label{ux432ux43bux430ux436ux43dux43e-ux430ux434ux438ux430ux431ux430ux442ux438ux447ux435ux441ux43aux438ux439-ux433ux440ux430ux434ux438ux435ux43dux442-ux442ux435ux43cux43fux435ux440ux430ux442ux443ux440ux44b}}

Влажно-адиабатический градиент температуры (\(\gamma_{ва}\)) -- это
величина, на которую изменяется температура насыщенной водяным паром
порции воздуха при ее адиабатическом вертикальном перемещении на единицу
расстояния. В отличие от сухоадиабатического градиента
(\(\gamma_a \approx 9.74 \, \text{°С/100 м}\)), влажно-адиабатический
градиент всегда меньше, поскольку выделение скрытой теплоты конденсации
при подъеме (или поглощение при опускании) уменьшает скорость изменения
температуры.

Формула для влажно-адиабатического градиента температуры может быть
представлена в виде:
\(\gamma_{ва} = \gamma_a \frac{1 + \frac{L q_w}{c_p T}}{1 + \frac{L^2 q_w}{c_p R T^2}}\),
где \(\gamma_a\) -- сухоадиабатический градиент, \(L\) -- удельная
теплота конденсации, \(q_w\) -- массовая доля насыщенного водяного пара
(насыщающая удельная влажность), \(c_p\) -- удельная теплоемкость сухого
воздуха при постоянном давлении, \(R\) -- удельная газовая постоянная
сухого воздуха, и \(T\) -- абсолютная температура.

\textbf{Зависимость от температуры и давления:}

\begin{enumerate}
\def\labelenumi{\arabic{enumi}.}
\tightlist
\item
  \textbf{Зависимость от температуры:} Влажно-адиабатический градиент
  (\(\gamma_{ва}\)) сильно зависит от температуры. Насыщающая упругость
  водяного пара (\(E_a\)) и, следовательно, насыщающая удельная
  влажность (\(q_w\)) экспоненциально возрастают с температурой. При
  более высоких температурах в воздухе может содержаться значительно
  больше водяного пара. Это означает, что при адиабатическом подъеме и
  охлаждении насыщенной частицы воздуха, на каждый градус понижения
  температуры будет конденсироваться больше водяного пара, выделяя,
  соответственно, больше скрытой теплоты. Эта выделяемая теплота
  компенсирует адиабатическое охлаждение, приводя к более медленному
  падению температуры с высотой.

  \begin{itemize}
  \tightlist
  \item
    Таким образом, при высоких температурах \(\gamma_{ва}\) мал
    (например, летом при \(T = 10 \, \text{°С}\) на уровне 800 гПа,
    \(\gamma_{ва}\) может составлять около \(4.9 \, \text{°С/км}\)).
  \item
    По мере понижения температуры, количество водяного пара, которое
    может конденсироваться, уменьшается, и влияние выделяемой скрытой
    теплоты становится менее значительным. В результате \(\gamma_{ва}\)
    возрастает и приближается к сухоадиабатическому градиенту
    \(\gamma_a\). Например, зимой при \(T = -20 \, \text{°С}\) на уровне
    800 гПа, \(\gamma_{ва}\) может достигать \(8.3 \, \text{°С/км}\). В
    условиях очень низких температур, когда содержание водяного пара
    становится ничтожным, влажно-адиабатический процесс практически
    неотличим от сухоадиабатического.
  \end{itemize}
\item
  \textbf{Зависимость от давления:} Зависимость \(\gamma_{ва}\) от
  давления проявляется через насыщающую удельную влажность \(q_w\),
  которая определяется как \(q_w = 0.622 \frac{E_a}{P}\). Хотя \(E_a\)
  зависит только от температуры, \(P\) (давление) напрямую входит в
  формулу \(q_w\).

  \begin{itemize}
  \tightlist
  \item
    При более низком давлении (например, на больших высотах), при той же
    температуре, значение \(q_w\) будет выше. Это, в свою очередь, через
    формулу для \(\gamma_{ва}\) (содержащую \(q_w\) в числителе и
    знаменателе) влияет на его величину. Однако доминирующей
    зависимостью является зависимость от температуры, поскольку
    насыщающая упругость водяного пара (и, как следствие, \(q_w\))
    меняется с температурой экспоненциально, в то время как зависимость
    от давления носит обратный характер.
  \end{itemize}
\end{enumerate}

В целом, понимание влажно-адиабатических процессов и градиента является
ключевым для прогнозирования образования и развития облаков и осадков,
поскольку именно эти процессы определяют термодинамическое состояние
насыщенного воздуха в атмосфере.

ewpage

Коллега, рад помочь с подготовкой к экзаменам. Вот подробные заметки по
запрошенным темам.

\hypertarget{ux43fux441ux435ux432ux434ux43eux434ux438ux430ux431ux430ux442ux438ux447ux435ux441ux43aux438ux435-ux43fux440ux43eux446ux435ux441ux441ux44b}{%
\section{Псевдодиабатические
Процессы}\label{ux43fux441ux435ux432ux434ux43eux434ux438ux430ux431ux430ux442ux438ux447ux435ux441ux43aux438ux435-ux43fux440ux43eux446ux435ux441ux441ux44b}}

При изучении атмосферных процессов, особенно связанных с вертикальными
перемещениями воздуха, мы часто сталкиваемся с адиабатическими
приближениями. В первом приближении, когда работа расширения или сжатия
значительно превосходит приток тепла извне, изменение термодинамического
состояния движущегося воздуха можно считать \textbf{сухоадиабатическим},
то есть предполагается, что воздух не получает и не отдает тепло в
процессе движения. Однако в реальной атмосфере, особенно при конденсации
водяного пара, этот ``идеальный'' адиабатический процесс усложняется.

\textbf{Псевдодиабатический процесс} -- это адиабатический процесс,
применимый к \textbf{насыщенному влажному воздуху}, при котором
предполагается, что вся сконденсировавшаяся вода (в виде капель или
ледяных кристаллов) \textbf{немедленно удаляется} из рассматриваемой
частицы воздуха. Иными словами, хотя скрытая теплота фазовых переходов
(конденсации или сублимации) выделяется в частицу и влияет на её
температуру, сама конденсированная влага не остается в частице, не
испаряется обратно и не вносит вклад в её теплоёмкость.

Для вертикально движущейся порции влажного воздуха уравнение первого
начала термодинамики принимает вид, учитывающий теплоту фазовых
переходов:

\(c_p dT - vdP = -Ldq\)

где \(L\) --- скрытая теплота конденсации (или испарения), а \(dq\) ---
изменение удельной влажности (или количества сконденсировавшейся влаги).
При подъёме насыщенной влажной частицы воздуха, выделение скрытой
теплоты конденсации замедляет понижение её температуры, обусловленное
адиабатическим расширением. Это приводит к тому, что скорость падения
температуры с высотой в насыщенном воздухе (влажно-адиабатический
градиент) меньше, чем в сухом воздухе (сухоадиабатический градиент).

Псевдоадиабатические процессы являются ключевыми для понимания
образования мощных кучево-дождевых облаков, где происходит интенсивное
выделение скрытой теплоты. Численные модели прогноза обложных осадков
часто строятся на предположении псевдоадиабатичности.

\hypertarget{ux44dux43aux432ux438ux432ux430ux43bux435ux43dux442ux43dux43e-ux43fux43eux442ux435ux43dux446ux438ux430ux43bux44cux43dux430ux44f-ux438-ux43fux441ux435ux432ux434ux43eux43fux43eux442ux435ux43dux446ux438ux430ux43bux44cux43dux430ux44f-ux442ux435ux43cux43fux435ux440ux430ux442ux443ux440ux430-ux438ux445-ux441ux432ux43eux439ux441ux442ux432ux430}{%
\section{Эквивалентно-Потенциальная и Псевдопотенциальная Температура,
их
Свойства}\label{ux44dux43aux432ux438ux432ux430ux43bux435ux43dux442ux43dux43e-ux43fux43eux442ux435ux43dux446ux438ux430ux43bux44cux43dux430ux44f-ux438-ux43fux441ux435ux432ux434ux43eux43fux43eux442ux435ux43dux446ux438ux430ux43bux44cux43dux430ux44f-ux442ux435ux43cux43fux435ux440ux430ux442ux443ux440ux430-ux438ux445-ux441ux432ux43eux439ux441ux442ux432ux430}}

В термодинамике атмосферы для характеристики состояния воздушных масс,
особенно с учётом влажности и фазовых переходов, используются такие
важные консервативные характеристики, как потенциальная,
эквивалентно-потенциальная и псевдопотенциальная температуры.

\textbf{Потенциальная температура (\(\theta\))}: Мы уже обсуждали, что
потенциальная температура не изменяется при сухоадиабатических
процессах. Это означает, что если частица сухого воздуха перемещается по
вертикали без теплообмена, её потенциальная температура остаётся
постоянной.

\textbf{Псевдопотенциальная температура (\(\theta_p\))}: Это величина,
которая сохраняет постоянное значение в движущейся воздушной массе при
\textbf{псевдоадиабатическом процессе}. Это означает, что если
насыщенная влажная частица воздуха поднимается, конденсирующаяся влага
немедленно удаляется, а скрытая теплота конденсации полностью передается
частице, то её псевдопотенциальная температура остаётся неизменной.

\textbf{Эквивалентно-потенциальная температура (\(\theta_e\))}: Этот
термин часто используется как взаимозаменяемый с псевдопотенциальной
температурой. Эквивалентно-потенциальная температура определяется как
температура, которую приобрела бы воздушная частица, если бы она:

\begin{enumerate}
\def\labelenumi{\arabic{enumi}.}
\tightlist
\item
  Была адиабатически поднята до уровня конденсации.
\item
  Затем поднята влажно-адиабатически (псевдоадиабатически) до такого
  уровня, где весь водяной пар сконденсировался и удалился.
\item
  Затем сухоадиабатически опущена до стандартного уровня давления
  (обычно 1000 гПа).
\end{enumerate}

\textbf{Основные свойства эквивалентно-потенциальной/псевдопотенциальной
температуры:}

\begin{itemize}
\tightlist
\item
  \textbf{Консервативность}: Являются наиболее консервативными
  характеристиками воздушной массы \textbf{до начала конденсации
  водяного пара}. При наступлении конденсации и протекании
  псевдоадиабатического процесса, псевдопотенциальная температура
  становится инвариантом. Это делает их чрезвычайно полезными для
  идентификации и прослеживания воздушных масс, поскольку их значения
  мало меняются в суточном ходе или при вертикальных перемещениях
  воздушных частиц, если нет значительных неадиабатических притоков
  тепла (кроме теплоты фазовых переходов).
\item
  \textbf{Учёт скрытой теплоты}: В отличие от потенциальной температуры,
  которая учитывает только внутреннюю и потенциальную энергию сухого
  воздуха, эквивалентно-потенциальная температура включает в себя и
  скрытую теплоту, содержащуюся в водяном паре. Чем больше водяного
  пара, тем больше выделяется скрытой теплоты при его конденсации, и тем
  выше будет эквивалентно-потенциальная температура.
\item
  \textbf{Диагностический инструмент}: Их значения позволяют оценить
  стабильность атмосферы. По их вертикальному распределению можно судить
  о бароклинности среды и условиях для конвекции.
\end{itemize}

\hypertarget{ux43fux43eux43dux44fux442ux438ux435-ux43e-ux43dux435ux430ux434ux438ux430ux431ux430ux442ux438ux447ux435ux441ux43aux438ux445-ux43fux440ux43eux446ux435ux441ux441ux430ux445-1}{%
\section{Понятие о Неадиабатических
Процессах}\label{ux43fux43eux43dux44fux442ux438ux435-ux43e-ux43dux435ux430ux434ux438ux430ux431ux430ux442ux438ux447ux435ux441ux43aux438ux445-ux43fux440ux43eux446ux435ux441ux441ux430ux445-1}}

Как мы уже выяснили, \textbf{адиабатический процесс} -- это изменение
состояния термодинамической системы (частицы воздуха), происходящее без
теплообмена с окружающей средой, то есть \(dQ = 0\). Изменение
температуры в этом случае обусловлено исключительно работой сжатия или
расширения частицы. Однако в реальной атмосфере, кроме вертикальных
перемещений, постоянно происходят процессы, связанные с притоком или
оттоком тепла. Эти процессы называются \textbf{неадиабатическими}.

Неадиабатические процессы играют фундаментальную роль в формировании
погоды и климата, поскольку они являются источниками или стоками энергии
в атмосферной системе. Включение неадиабатических факторов значительно
усложняет уравнения гидротермодинамики.

Основные источники неадиабатического притока/оттока тепла в атмосфере
включают:

\begin{enumerate}
\def\labelenumi{\arabic{enumi}.}
\item
  \textbf{Радиационные процессы}: Это приток или отток тепла за счет
  электромагнитного излучения.

  \begin{itemize}
  \tightlist
  \item
    \textbf{Поглощение солнечной радиации}: Воздух слабо поглощает
    коротковолновую солнечную радиацию, но некоторые газы (водяной пар,
    углекислый газ, озон) и аэрозоли поглощают её частично, превращая в
    тепловую энергию. Земная поверхность, нагреваясь Солнцем, также
    излучает в инфракрасном диапазоне.
  \item
    \textbf{Излучение Земли и атмосферы}: Земля и сама атмосфера
    излучают длинноволновую (инфракрасную) радиацию. Атмосферные газы,
    такие как водяной пар и углекислый газ, сильно поглощают и излучают
    в этом диапазоне, создавая так называемый ``парниковый эффект''.
    Атмосфера не находится в состоянии радиационного равновесия, и эти
    потоки энергии приводят к её нагреву или охлаждению.
  \end{itemize}
\item
  \textbf{Приток/отток тепла при фазовых переходах воды}: Эти процессы
  особенно существенны в облаках и туманах.

  \begin{itemize}
  \tightlist
  \item
    \textbf{Конденсация и сублимация}: При конденсации водяного пара
    (образование капель) или сублимации (образование кристаллов льда)
    выделяется скрытая теплота, которая нагревает окружающий воздух. Эта
    теплота является мощным источником энергии для развития конвективных
    систем, таких как кучево-дождевые облака и тропические циклоны.
  \item
    \textbf{Испарение и таяние}: При испарении (переход воды в пар) или
    таянии (переход льда в воду) скрытая теплота поглощается из
    окружающей среды, что приводит к её охлаждению.
  \end{itemize}
\item
  \textbf{Молекулярная теплопроводность воздуха}: Это перенос тепла за
  счет хаотического движения молекул воздуха. Из-за малой величины
  коэффициента теплопроводности воздуха, молекулярный теплоперенос
  существенен только в тончайшем приповерхностном слое воздуха.
\item
  \textbf{Турбулентный приток тепла}: Этот процесс связан со случайными,
  беспорядочными колебаниями ветра -- пульсациями или флуктуациями.
  Турбулентное перемешивание является основным механизмом переноса тепла
  (и влаги, и количества движения) между различными слоями атмосферы,
  особенно в пограничном слое. В отличие от молекулярной
  теплопроводности, турбулентный обмен имеет значительно больший масштаб
  и эффективность.
\end{enumerate}

Все эти неадиабатические процессы постоянно изменяют температуру,
плотность и давление воздуха, приводя к возникновению барических
градиентов, которые, в свою очередь, порождают и поддерживают
атмосферные движения. Влияние неадиабатических процессов становится
особенно заметным на сроках прогноза более одних суток. Модели,
учитывающие эти процессы (тепло фазовых переходов воды, турбулентный и
радиационный теплообмены), разрабатываются и внедряются в оперативную
прогностическую практику, дополняя более простые адиабатические модели.

ewpage

Коллега, давайте разберем эти ключевые аспекты динамики атмосферы,
опираясь на наши материалы.

\hypertarget{ux438ux437ux43cux435ux43dux435ux43dux438ux435-ux43fux430ux440ux430ux43cux435ux442ux440ux43eux432-ux432ux43eux437ux434ux443ux448ux43dux43eux439-ux447ux430ux441ux442ux438ux446ux44b-ux43fux440ux438-ux435ux435-ux432ux435ux440ux442ux438ux43aux430ux43bux44cux43dux44bux445-ux43fux435ux440ux435ux43cux435ux449ux435ux43dux438ux44fux445}{%
\subsubsection{Изменение параметров воздушной частицы при ее
вертикальных
перемещениях}\label{ux438ux437ux43cux435ux43dux435ux43dux438ux435-ux43fux430ux440ux430ux43cux435ux442ux440ux43eux432-ux432ux43eux437ux434ux443ux448ux43dux43eux439-ux447ux430ux441ux442ux438ux446ux44b-ux43fux440ux438-ux435ux435-ux432ux435ux440ux442ux438ux43aux430ux43bux44cux43dux44bux445-ux43fux435ux440ux435ux43cux435ux449ux435ux43dux438ux44fux445}}

При вертикальных перемещениях воздушная частица, если не происходит
обмена теплом с окружающей средой (т.е. процесс адиабатический) и
отсутствуют фазовые превращения воды, изменяет свои параметры
предсказуемым образом. Воздух, будучи сжимаемой средой, расширяется при
подъеме и сжимается при опускании. Это приводит к изменению его
удельного объема и, соответственно, температуры.

Для сухой (ненасыщенной) воздушной частицы температура при подъеме
понижается по \textbf{сухоадиабатическому закону}, а при опускании
повышается. Величина этого изменения на единицу расстояния называется
сухоадиабатическим градиентом температуры (\(\gamma_a\)). Согласно
источникам, \(\gamma_a\) равен примерно 1°C на каждые 100 м подъема. В
этом случае потенциальная температура частицы воздуха сохраняется.

Если же воздушная частица насыщена водяным паром, процесс охлаждения при
подъеме замедляется из-за выделения скрытой теплоты конденсации.
Соответствующий градиент температуры называется
\textbf{влажно-адиабатическим градиентом температуры} (\(\gamma_{ва}\)).
Он всегда меньше сухоадиабатического (\(\gamma_{ва} < \gamma_a\)).

Важно отметить, что движение воздуха в атмосфере происходит относительно
медленно, поэтому давление в движущейся порции воздуха на любом уровне
успевает выравниваться с давлением в окружающей атмосфере. Это позволяет
использовать уравнение статики для расчета распределения давления по
высоте с высокой точностью.

\hypertarget{ux43aux440ux438ux432ux430ux44f-ux441ux43eux441ux442ux43eux44fux43dux438ux44f}{%
\subsubsection{Кривая
состояния}\label{ux43aux440ux438ux432ux430ux44f-ux441ux43eux441ux442ux43eux44fux43dux438ux44f}}

\textbf{Кривая состояния} (\(T_i\)) --- это графическое представление
адиабатического изменения температуры поднимающейся порции воздуха с
высотой. На аэрологической диаграмме она проводится от начальной точки,
соответствующей приземным условиям или верхней границе приземной
инверсии/изотермии.

Проведение кривой состояния подчиняется следующим правилам:

\begin{itemize}
\tightlist
\item
  От начальной точки до \textbf{уровня конденсации} она проводится по
  \textbf{сухой адиабате} (поскольку воздух до насыщения ведет себя как
  сухой).
\item
  Выше уровня конденсации она проводится по \textbf{влажной адиабате}
  (поскольку воздух уже насыщен и происходит конденсация).
\end{itemize}

Кривая состояния является одним из ключевых элементов для оценки
вертикальной устойчивости атмосферы, так как ее взаимное расположение с
кривой стратификации (вертикальным распределением температуры в
окружающей атмосфере) определяет знак энергии неустойчивости.

\hypertarget{ux443ux440ux43eux432ux435ux43dux44c-ux43aux43eux43dux434ux435ux43dux441ux430ux446ux438ux438}{%
\subsubsection{Уровень
конденсации}\label{ux443ux440ux43eux432ux435ux43dux44c-ux43aux43eux43dux434ux435ux43dux441ux430ux446ux438ux438}}

\textbf{Уровень конденсации} -- это высота, на которой поднимающийся
воздух достигает состояния насыщения водяным паром. Этот процесс
происходит потому, что при адиабатическом подъеме температура воздушной
частицы понижается. Когда температура частицы достигает температуры
точки росы, упругость водяного пара становится насыщающей, и начинается
конденсация.

Уровень конденсации обычно находится на высоте, близкой к нижней границе
конвективных облаков. До достижения этого уровня удельная влажность и
отношение смеси в поднимающемся воздухе остаются постоянными.
Температура точки росы (\(T_d\)) также изменяется при подъеме, но с
меньшим градиентом (примерно 1.7 °С/км) по сравнению с
сухоадиабатическим. Формула Ферреля (3.5.3) позволяет рассчитать высоту
уровня конденсации, используя начальные значения температуры и точки
росы.

\hypertarget{ux443ux440ux43eux432ux435ux43dux44c-ux43aux43eux43dux432ux435ux43aux446ux438ux438}{%
\subsubsection{Уровень
конвекции}\label{ux443ux440ux43eux432ux435ux43dux44c-ux43aux43eux43dux432ux435ux43aux446ux438ux438}}

\textbf{Уровень конвекции} -- это высота, на которой вертикальное
ускорение поднимающейся перегретой частицы воздуха обращается в нуль.
Это происходит, когда температура поднимающейся частицы воздуха
становится равной температуре окружающей атмосферы (\(T_i = T_e\)). Выше
этого уровня ускоренный подъем прекращается. Обычно уровень конвекции
располагается вблизи верхней границы конвективных облаков.

На аэрологической диаграмме уровень конвекции определяется как точка
пересечения кривой состояния (температуры поднимающейся частицы) с
кривой стратификации (температуры окружающей среды). Если поднимающаяся
частица оказывается на вышележащих уровнях холоднее окружающей
атмосферы, стратификация устойчива; если теплее --- неустойчива; если
температура равна --- безразлична.

\hypertarget{ux44dux43dux435ux440ux433ux438ux44f-ux43dux435ux443ux441ux442ux43eux439ux447ux438ux432ux43eux441ux442ux438}{%
\subsubsection{Энергия
неустойчивости}\label{ux44dux43dux435ux440ux433ux438ux44f-ux43dux435ux443ux441ux442ux43eux439ux447ux438ux432ux43eux441ux442ux438}}

\textbf{Энергия неустойчивости} -- это полная работа, которую совершает
подъемная сила, действующая на единицу массы воздуха, при ее
вертикальном перемещении. Эта энергия пропорциональна площади,
заключенной между кривой состояния (\(T_i\)) и кривой стратификации
(\(T_e\)) на аэрологической диаграмме.

\begin{itemize}
\tightlist
\item
  Если кривая состояния лежит справа от кривой стратификации, энергия
  неустойчивости будет \textbf{положительной}. Это соответствует
  \textbf{неустойчивой стратификации}, при которой поднимающаяся частица
  теплее окружающей среды и продолжает ускоренно подниматься.
\item
  Если кривая состояния лежит слева от кривой стратификации, энергия
  неустойчивости является \textbf{отрицательной}. Это указывает на
  \textbf{устойчивую стратификацию}, когда поднимающаяся частица
  становится холоднее окружающей среды и стремится вернуться на исходный
  уровень.
\end{itemize}

Критерии вертикальной устойчивости атмосферы также могут быть выражены
через градиенты температуры:

\begin{itemize}
\tightlist
\item
  Сухонеустойчивая: \(\gamma > \gamma_a\).
\item
  Сухоустойчивая: \(\gamma < \gamma_a\).
\item
  Сухобезразличная: \(\gamma = \gamma_a\).
\item
  Аналогичные критерии применяются для насыщенного воздуха с
  использованием \(\gamma_{ва}\).
\end{itemize}

Следует отметить, что даже при большой энергии неустойчивости кучевые
облака могут не развиваться, если сохраняется нисходящее движение
воздуха.

\hypertarget{ux430ux44dux440ux43eux43bux43eux433ux438ux447ux435ux441ux43aux430ux44f-ux434ux438ux430ux433ux440ux430ux43cux43cux430}{%
\subsubsection{Аэрологическая
диаграмма}\label{ux430ux44dux440ux43eux43bux43eux433ux438ux447ux435ux441ux43aux430ux44f-ux434ux438ux430ux433ux440ux430ux43cux43cux430}}

\textbf{Аэрологическая диаграмма}, или эмаграмма, является основным
инструментом для анализа термодинамического состояния атмосферы и оценки
ее вертикальной устойчивости. Она представляет собой график, где по оси
ординат отложена величина, пропорциональная \(lnP\) (логарифму
давления), которая уменьшается с высотой и определяет положение точки в
атмосфере, а по оси абсцисс --- температура.

На бланках аэрологической диаграммы нанесены семейства изолиний для
удобства графических построений:

\begin{itemize}
\tightlist
\item
  \textbf{Изобары:} линии равного давления.
\item
  \textbf{Изотермы:} линии равной температуры.
\item
  \textbf{Сухие адиабаты:} линии, показывающие изменение температуры
  сухой воздушной частицы при адиабатическом процессе.
\item
  \textbf{Влажные адиабаты:} линии, показывающие изменение температуры
  насыщенной воздушной частицы при адиабатическом процессе с учетом
  конденсации.
\item
  \textbf{Изограммы:} изолинии максимальной удельной влажности (или
  постоянного отношения смеси).
\end{itemize}

По результатам аэрологического зондирования на диаграмме строятся две
основные кривые:

\begin{itemize}
\tightlist
\item
  \textbf{Кривая стратификации} (\(T_e\)): характеризует фактическое
  вертикальное распределение температуры в атмосфере.
\item
  \textbf{Кривая точек росы (депеграмма):} характеризует изменение точки
  росы с высотой, а также изменение удельной влажности.
\end{itemize}

Затем, отталкиваясь от наблюдаемых приземных условий или условий на
границе приземного слоя, строится \textbf{кривая состояния} (\(T_i\)),
как было описано выше. Анализируя взаимное расположение этих кривых,
метеоролог может качественно и количественно оценить устойчивость
атмосферы, определить уровни конденсации и конвекции, а также рассчитать
энергию неустойчивости. Например, для прогноза максимальной температуры
воздуха аэрологическая диаграмма используется путем перемещения вдоль
сухой адиабаты от кривой стратификации до определенного уровня.

Таким образом, аэрологическая диаграмма предоставляет комплексный
графический инструмент для понимания и прогнозирования вертикальных
движений и связанных с ними явлений погоды.

ewpage

Коллега, в динамической метеорологии термодинамические графики являются
фундаментальным инструментом для анализа и прогнозирования состояния
атмосферы, а также для изучения процессов энергообмена. Они позволяют
наглядно представить сложные термодинамические изменения, происходящие в
движущемся воздухе.

\hypertarget{ux43fux440ux438ux43dux446ux438ux43fux44b-ux43fux43eux441ux442ux440ux43eux435ux43dux438ux44f-ux438-ux438ux441ux43fux43eux43bux44cux437ux43eux432ux430ux43dux438ux435-ux442ux435ux440ux43cux43eux434ux438ux43dux430ux43cux438ux447ux435ux441ux43aux438ux445-ux433ux440ux430ux444ux438ux43aux43eux432}{%
\section{Принципы Построения и Использование Термодинамических
Графиков}\label{ux43fux440ux438ux43dux446ux438ux43fux44b-ux43fux43eux441ux442ux440ux43eux435ux43dux438ux44f-ux438-ux438ux441ux43fux43eux43bux44cux437ux43eux432ux430ux43dux438ux435-ux442ux435ux440ux43cux43eux434ux438ux43dux430ux43cux438ux447ux435ux441ux43aux438ux445-ux433ux440ux430ux444ux438ux43aux43eux432}}

Наиболее широко используемой термодинамической диаграммой в метеорологии
является \textbf{аэрологическая диаграмма}, часто называемая
\textbf{эмаграммой}.

\hypertarget{ux43fux440ux438ux43dux446ux438ux43fux44b-ux43fux43eux441ux442ux440ux43eux435ux43dux438ux44f-ux44dux43cux430ux433ux440ux430ux43cux43cux44b}{%
\subsection{1. Принципы Построения
Эмаграммы}\label{ux43fux440ux438ux43dux446ux438ux43fux44b-ux43fux43eux441ux442ux440ux43eux435ux43dux438ux44f-ux44dux43cux430ux433ux440ux430ux43cux43cux44b}}

Эмаграмма представляет собой систему координат, предназначенную для
графического анализа состояния атмосферного воздуха и происходящих в нём
процессов.

\begin{itemize}
\item
  \textbf{Координатные оси}: По оси абсцисс (горизонтальной)
  откладывается абсолютная температура (\(T\)) воздуха, а по оси ординат
  (вертикальной) -- логарифм давления (\(\ln P\)). При этом давление
  уменьшается с высотой, что соответствует вертикальному разрезу
  атмосферы. Абсолютная температура связана со шкалой Цельсия
  соотношением \(T = t_o + 273.2\), где \(t_o\) --- температура по
  стоградусной шкале.
\item
  \textbf{Семейства изолиний}: На бланках эмаграммы нанесены несколько
  семейств изолиний, каждая из которых отражает определённый
  термодинамический параметр:

  \begin{itemize}
  \tightlist
  \item
    \textbf{Изобары}: Линии постоянного давления. На эмаграмме они
    представляют собой горизонтальные прямые.
  \item
    \textbf{Изотермы}: Линии постоянной температуры. Они изображаются в
    виде вертикальных прямых.
  \item
    \textbf{Сухие адиабаты}: Это линии, характеризующие изменение
    температуры порции сухого воздуха при адиабатическом (без
    теплообмена с окружающей средой) вертикальном перемещении. На
    эмаграмме они имеют определённый наклон, отражающий
    сухоадиабатический градиент температуры
    (\(\gamma_a \approx 1^\circ \text{С}/100 \text{м}\)), который
    является постоянным для сухого воздуха.
  \item
    \textbf{Влажные адиабаты}: Эти линии показывают изменение
    температуры насыщенного влажного воздуха при адиабатическом
    процессе, сопровождающемся конденсацией водяного пара и выделением
    скрытой теплоты парообразования. Их наклон меньше, чем у сухих
    адиабат, и он не является постоянным, завися от температуры и
    давления.
  \item
    \textbf{Изограммы (изолинии максимальной удельной влажности)}: Это
    линии, связывающие точки с одинаковой максимальной удельной
    влажностью насыщенного воздуха при различных температурах и
    давлениях. Они позволяют оценивать содержание водяного пара в
    атмосфере.
  \end{itemize}
\end{itemize}

\hypertarget{ux438ux441ux43fux43eux43bux44cux437ux43eux432ux430ux43dux438ux435-ux44dux43cux430ux433ux440ux430ux43cux43cux44b}{%
\subsection{2. Использование
Эмаграммы}\label{ux438ux441ux43fux43eux43bux44cux437ux43eux432ux430ux43dux438ux435-ux44dux43cux430ux433ux440ux430ux43cux43cux44b}}

Эмаграмма является краеугольным камнем аэрологического анализа и
прогноза.

\begin{itemize}
\item
  \textbf{Нанесение аэрологических данных}:

  \begin{itemize}
  \tightlist
  \item
    \textbf{Кривая стратификации (кривая окружающей атмосферы)}: На
    эмаграмму наносят данные аэрологического зондирования (температуру и
    давление на различных высотах), строя кривую, характеризующую
    фактическое вертикальное распределение температуры воздуха в
    атмосфере (\(eT\)).
  \item
    \textbf{Депеграмма (кривая точки росы)}: Отдельно наносится кривая,
    показывающая вертикальное распределение температуры точки росы
    (\(T_d\)). Эта кривая одновременно характеризует изменение удельной
    влажности с высотой.
  \end{itemize}
\item
  \textbf{Анализ устойчивости атмосферы}:

  \begin{itemize}
  \tightlist
  \item
    \textbf{Кривая состояния (кривая адиабатического изменения
    температуры поднимающейся частицы воздуха)}: Проводится от начальной
    точки на кривой стратификации (обычно от приземного слоя) вверх,
    следуя сначала по сухой адиабате до уровня конденсации, а затем по
    влажной адиабате.
  \item
    \textbf{Определение устойчивости}: Знак энергии неустойчивости
    определяется по взаимному расположению кривых состояния и
    стратификации.

    \begin{itemize}
    \tightlist
    \item
      Если кривая состояния поднимающейся частицы (\(iT\)) лежит справа
      от кривой стратификации (\(eT\)), то энергия неустойчивости будет
      положительной, что указывает на неустойчивую стратификацию.
    \item
      Если кривая состояния лежит слева от кривой стратификации, энергия
      неустойчивости отрицательна, что свидетельствует об устойчивой
      стратификации.
    \item
      Когда кривая состояния параллельна кривой стратификации,
      стратификация нейтральна.
    \end{itemize}
  \item
    \textbf{Энергия неустойчивости}: На эмаграмме энергия неустойчивости
    пропорциональна площади, заключённой между кривыми состояния и
    стратификации. Эта энергия представляет собой работу, которую может
    совершить архимедова сила при вертикальном подъеме единицы массы
    воздуха.
  \end{itemize}
\item
  \textbf{Определение уровня конденсации (УК)}: Уровень конденсации
  (база облака) на эмаграмме находится в точке пересечения сухой
  адиабаты, проходящей через начальное состояние поднимающейся частицы
  воздуха, с изограммой, соответствующей начальной точке росы этой
  частицы. То есть, это высота, на которой ненасыщенная частица воздуха
  становится насыщенной при подъеме.
\item
  \textbf{Прогнозирование облачности и осадков}:

  \begin{itemize}
  \tightlist
  \item
    Эмаграммы активно используются для прогнозирования образования
    конвективной облачности (кучевых форм) и связанных с ней ливневых
    осадков и гроз.
  \item
    По взаимному расположению кривых стратификации и влажных адиабат
    можно определить слои конвективной неустойчивости, где ожидается
    интенсивное развитие облаков. Например, развитие конвекции
    происходит в слое, где прогностическая кривая стратификации
    отклоняется влево от влажной адиабаты, то есть, где
    \(\gamma > \gamma_{\text{ва}}\).
  \item
    При образовании кучевых облаков выделяется скрытая теплота
    парообразования, что приводит к приближению кривой стратификации к
    влажной адиабате.
  \item
    Прогностические кривые стратификации и точки росы строятся на
    диаграмме для оценки будущих условий формирования облаков. Также
    возможно прогнозирование нижней и верхней границ конвективной
    облачности.
  \end{itemize}
\item
  \textbf{Использование потенциальной и псевдопотенциальной температур}:

  \begin{itemize}
  \tightlist
  \item
    \textbf{Потенциальная температура (\(\theta\))}: Определяется как
    температура, которую имела бы порция сухого воздуха, если бы была
    адиабатически приведена к стандартному давлению (1000 гПа). Она
    является консервативной величиной при сухоадиабатических процессах.
  \item
    \textbf{Псевдопотенциальная (или эквивалентно-потенциальная)
    температура}: Это температура, которую имела бы порция влажного
    воздуха, если бы весь водяной пар в ней сконденсировался, а затем
    полученный сухой воздух был бы адиабатически приведен к стандартному
    давлению. Эта величина, наряду с удельной влажностью до начала
    конденсации, является одной из наиболее консервативных характеристик
    воздушной массы, что делает её важной для анализа трансформации
    воздушных масс и их перемещений.
  \item
    Определение этих температур с помощью эмаграммы позволяет
    отслеживать воздушные массы и их трансформацию, так как их изменения
    в процессе движения (например, с учетом притоков тепла и фазовых
    переходов воды) могут быть наглядно проанализированы.
  \end{itemize}
\end{itemize}

\hypertarget{ux434ux440ux443ux433ux438ux435-ux433ux440ux430ux444ux438ux447ux435ux441ux43aux438ux435-ux43cux435ux442ux43eux434ux44b-ux43aux440ux430ux442ux43aux43e}{%
\subsection{3. Другие Графические Методы
(Кратко)}\label{ux434ux440ux443ux433ux438ux435-ux433ux440ux430ux444ux438ux447ux435ux441ux43aux438ux435-ux43cux435ux442ux43eux434ux44b-ux43aux440ux430ux442ux43aux43e}}

Хотя эмаграмма является основным термодинамическим графиком для анализа
вертикального строения атмосферы, в метеорологии используются и другие
типы диаграмм для различных целей:

\begin{itemize}
\tightlist
\item
  \textbf{Радиационные диаграммы}: Применяются для графического
  определения потоков радиации, учитывая сложные процессы поглощения и
  излучения в атмосфере.
\item
  \textbf{Демаркационные графики (диаграммы рассеяния)}: Используются в
  статистическом анализе для выявления связей между различными
  метеорологическими параметрами (предикторами) и прогнозируемыми
  характеристиками (предиктантами), особенно для качественных
  характеристик (например, наличие/отсутствие явления). Они помогают
  визуализировать области преобладания определенных фаз или значений.
\end{itemize}

Таким образом, термодинамические графики, в особенности эмаграмма,
являются незаменимым инструментом для понимания и прогнозирования
атмосферных процессов, позволяя метеорологам интерпретировать сложные
взаимодействия между температурой, давлением, влажностью и вертикальными
движениями воздуха.

ewpage

\hypertarget{ux441ux442ux440ux430ux442ux438ux444ux438ux43aux430ux446ux438ux44f-ux430ux442ux43cux43eux441ux444ux435ux440ux44b}{%
\section{Стратификация
Атмосферы}\label{ux441ux442ux440ux430ux442ux438ux444ux438ux43aux430ux446ux438ux44f-ux430ux442ux43cux43eux441ux444ux435ux440ux44b}}

\textbf{Стратификация атмосферы} -- это характер вертикального
распределения метеорологических величин, в первую очередь температуры,
давления и плотности воздуха. Температурное распределение по вертикали
является ключевым показателем состояния атмосферы и определяет её
устойчивость или неустойчивость.

Традиционно атмосферу делят на ряд слоев по характеру изменения
температуры с высотой:

\begin{itemize}
\tightlist
\item
  \textbf{Тропосфера}: Нижний и наиболее важный для погодообразования
  слой, простирающийся от поверхности Земли до высоты 9-17 км. В этом
  слое температура воздуха, как правило, падает с высотой в среднем на
  0.65°C на 100 м. В тропосфере сосредоточено более 4/5 всей массы
  атмосферного воздуха и практически весь водяной пар, здесь формируются
  все основные виды облаков и воздушные массы.
\item
  \textbf{Стратосфера}: Располагается над тропосферой, до высоты 50-55
  км. В нижнем слое стратосферы температура может быть практически
  постоянной (-45\ldots-75°C), а в верхнем -- расти с высотой (до
  -20\ldots20°C).
\item
  \textbf{Мезосфера}: Над стратосферой, до 80-85 км, с понижением
  температуры с высотой от 0°C до -90°C.
\item
  \textbf{Термосфера}: Выше мезосферы, до 600-1000 км, с быстрым
  повышением температуры.
\end{itemize}

Особое внимание в метеорологии уделяется слоям, где наблюдаются
аномальные вертикальные градиенты температуры:

\begin{itemize}
\tightlist
\item
  \textbf{Изотермия}: Слой, в котором температура с высотой не меняется.
\item
  \textbf{Инверсия}: Слой воздуха, в котором температура возрастает с
  высотой. Инверсии и изотермии являются предельно устойчивыми слоями,
  которые настолько сильно подавляют турбулентность, что проникновение
  через них тепла, влаги или примесей практически прекращается.

  \begin{itemize}
  \tightlist
  \item
    \textbf{Приземные инверсии}: Возникают над сушей или снежным/ледяным
    покровом.

    \begin{itemize}
    \tightlist
    \item
      \emph{Радиационные инверсии}: Типичны для ясных ночей со слабым
      ветром, особенно сильны в котловинах, откуда холодный воздух не
      может вытекать.
    \item
      \emph{Адвективные инверсии}: Образуются при поступлении более
      теплого воздуха на более холодную подстилающую поверхность.
    \end{itemize}
  \item
    \textbf{Приподнятые инверсии}: Чаще возникают в результате оседания
    воздуха при нисходящих вертикальных потоках в антициклонах,
    называемых \textbf{инверсиями оседания}. Также могут формироваться
    при перетекании воздуха через горные перевалы (орографические
    инверсии). Наиболее значимы пассатные инверсии в областях нисходящих
    токов субтропических антициклонов, охватывающие огромные
    пространства. Они препятствуют развитию мощной конвекции и являются
    причиной развития пустынь.
  \end{itemize}
\item
  \textbf{Тропопауза и стратосфера}: Являются своеобразными слоями
  изотермии или инверсии, формирование которых происходит в основном под
  влиянием радиационных процессов, хотя турбулентное перемешивание и
  вертикальные движения также оказывают влияние.
\end{itemize}

\hypertarget{ux43aux440ux438ux442ux435ux440ux438ux438-ux43eux446ux435ux43dux43aux438-ux432ux435ux440ux442ux438ux43aux430ux43bux44cux43dux43eux439-ux442ux435ux440ux43cux438ux447ux435ux441ux43aux43eux439-ux443ux441ux442ux43eux439ux447ux438ux432ux43eux441ux442ux438-ux430ux442ux43cux43eux441ux444ux435ux440ux44b}{%
\section{Критерии Оценки Вертикальной Термической Устойчивости
Атмосферы}\label{ux43aux440ux438ux442ux435ux440ux438ux438-ux43eux446ux435ux43dux43aux438-ux432ux435ux440ux442ux438ux43aux430ux43bux44cux43dux43eux439-ux442ux435ux440ux43cux438ux447ux435ux441ux43aux43eux439-ux443ux441ux442ux43eux439ux447ux438ux432ux43eux441ux442ux438-ux430ux442ux43cux43eux441ux444ux435ux440ux44b}}

Вертикальные движения воздуха в атмосфере играют исключительно большую
роль в процессах формирования погоды, таких как образование облаков и
осадков. Одной из главных причин, вызывающих эти движения, является
разность между температурой движущейся порции воздуха и температурой
окружающей атмосферы. При наличии этой разности возникают архимедовы
силы, сообщающие порции воздуха вертикальное ускорение.

\textbf{Метод частицы} (о нём подробнее ниже) используется для
определения, как будет меняться температура поднимающейся или
опускающейся воздушной частицы. Сравнивая её температуру с температурой
окружающей среды, мы можем определить устойчивость.

Выделяют три состояния стратификации:

\begin{enumerate}
\def\labelenumi{\arabic{enumi}.}
\tightlist
\item
  \textbf{Неустойчивое состояние}: Частица воздуха, начавшая движение по
  вертикали, получает ускорение в том же направлении, стремящееся
  удалить частицу от исходного уровня. Характеризуется развитием
  конвекции.
\item
  \textbf{Устойчивое состояние}: Частица воздуха, начавшая смещение по
  вертикали, получает ускорение в направлении, противоположном её
  движению, стремящееся вернуть частицу на исходный уровень. Конвекция в
  таких слоях невозможна.
\item
  \textbf{Безразличное состояние}: Вертикальное движение массы воздуха
  зависит только от начальной скорости, и вертикальное ускорение равно
  нулю.
\end{enumerate}

Для количественной оценки используются следующие градиенты температуры:

\begin{itemize}
\tightlist
\item
  \textbf{Сухоадиабатический градиент температуры (\(\gamma_a\))}:
  Понижение температуры ненасыщенного (сухого) воздуха при его
  адиабатическом подъёме на 100 м, равный примерно 1°C на 100 м.
  Потенциальная температура (\(\theta\)) при сухоадиабатических
  процессах не изменяется. В среднем, атмосфера стратифицирована
  устойчиво, и потенциальная температура с высотой возрастает.
\item
  \textbf{Влажноадиабатический градиент температуры (\(\gamma_{wa}\))}:
  Понижение температуры насыщенного воздуха при его адиабатическом
  подъёме на 100 м. Этот градиент всегда меньше сухоадиабатического
  (\(\gamma_{wa} < \gamma_a\)) из-за выделения скрытой теплоты
  конденсации при подъёме насыщенного воздуха.
\end{itemize}

\textbf{Критерии вертикальной устойчивости для сухого воздуха}:

\begin{itemize}
\tightlist
\item
  \(\gamma > \gamma_a\) -- Сухонеустойчивая стратификация. Поднимающаяся
  частица оказывается теплее окружающей среды, и вертикальное ускорение
  будет положительным.
\item
  \(\gamma < \gamma_a\) -- Сухоустойчивая стратификация. Поднимающаяся
  частица становится холоднее окружающей среды, и стратификация
  атмосферы будет устойчивой.
\item
  \(\gamma = \gamma_a\) -- Сухобезразличная стратификация. Температура
  частицы и окружающей среды равны.
\end{itemize}

\textbf{Критерии вертикальной устойчивости для насыщенного воздуха}:

\begin{itemize}
\tightlist
\item
  \(\gamma > \gamma_{wa}\) -- Влажнонеустойчивая стратификация. (В
  облаке).
\item
  \(\gamma < \gamma_{wa}\) -- Влажноустойчивая стратификация. (В
  облаке).
\item
  \(\gamma = \gamma_{wa}\) -- Влажнобезразличная стратификация. (В
  облаке).
\end{itemize}

Возникает также понятие \textbf{условной неустойчивости (или влажной
неустойчивости)}, когда атмосфера устойчива по отношению к ненасыщенным
вертикальным перемещениям (\(\gamma < \gamma_a\)), но неустойчива по
отношению к насыщенным перемещениям (\(\gamma > \gamma_{wa}\)). Такая
ситуация (когда \(\gamma_{wa} < \gamma < \gamma_a\)) является наиболее
распространённой в условиях конвекции, когда при наличии достаточной
влажности возможно развитие мощных кучевых облаков.

\textbf{Энергия неустойчивости}: Это работа, которую может совершить
архимедова сила при вертикальном подъеме единицы массы воздуха. Если
поднимающаяся частица оказывается теплее окружающей среды
(\(\text{T}_i > \text{T}_e\)), энергия неустойчивости будет
положительной, и состояние атмосферы будет неустойчивым. Отрицательная
энергия неустойчивости указывает на устойчивое состояние. Графически
энергию неустойчивости можно определить с помощью аэрологической
диаграммы, сопоставляя взаимное расположение кривых стратификации
(температура окружающей среды) и состояния (температура поднимающейся
частицы).

\hypertarget{ux43cux435ux442ux43eux434-ux447ux430ux441ux442ux438ux446ux44b}{%
\section{Метод
Частицы}\label{ux43cux435ux442ux43eux434-ux447ux430ux441ux442ux438ux446ux44b}}

\textbf{Метод частицы} -- это способ оценки вертикальной устойчивости
атмосферы, при котором предполагается, что единица массы воздуха
перемещается в неподвижном окружающем её воздухе.

\textbf{Суть метода}:

\begin{enumerate}
\def\labelenumi{\arabic{enumi}.}
\tightlist
\item
  Предполагается, что рассматриваемая \textbf{частица воздуха
  изолирована} от окружающей среды (т.е., процесс адиабатический).
\item
  В процессе движения (подъёма или опускания) \textbf{давление внутри
  частицы воздуха успевает выравниваться с давлением в окружающей
  атмосфере} на каждом уровне (условие квазистатичности).
\item
  Если частица воздуха \textbf{ненасыщенная}, её температура изменяется
  по сухоадиабатическому закону
  (\(\gamma_a \approx 1^\circ\text{C}/100\text{м}\)).
\item
  Если частица воздуха \textbf{насыщенная} (т.е. достигла уровня
  конденсации и содержит сконденсированную влагу), её температура
  изменяется по влажноадиабатическому закону (\(\gamma_{wa}\)). При этом
  в псевдодиабатическом процессе предполагается, что вся
  сконденсировавшаяся вода \emph{немедленно удаляется} из частицы {[}мои
  предыдущие заметки{]}, а выделяющаяся скрытая теплота идет на нагрев
  воздуха, замедляя падение температуры.
\item
  \textbf{Температура частицы сравнивается с температурой окружающей
  среды} на том же уровне.

  \begin{itemize}
  \tightlist
  \item
    Если частица теплее окружающей среды, она продолжает двигаться в том
    же направлении (неустойчивость).
  \item
    Если частица холоднее окружающей среды, она возвращается к исходному
    уровню (устойчивость).
  \item
    Если температуры равны, это безразличное равновесие.
  \end{itemize}
\end{enumerate}

\textbf{Применение метода}: Метод частицы активно используется при
работе с \textbf{аэрологическими диаграммами} (например, эмаграммами),
где кривая стратификации (фактическое вертикальное распределение
температуры в атмосфере) и кривая состояния (температура поднимающейся
частицы) наносятся для визуализации и оценки устойчивости. На этих
диаграммах также наносятся сухие и влажные адиабаты, а также изограммы
(изолинии максимальной удельной влажности). Кривая состояния проводится
от начальной точки на кривой стратификации до уровня конденсации.

\textbf{Ограничения метода}: Важно понимать, что метод частицы является
упрощенным представлением. В действительности, при развитии конвекции
\textbf{подъем одних частиц воздуха сопровождается компенсирующим
опусканием других частиц и изменением их термодинамического состояния}.
Происходит интенсивное перемешивание (турбулентность) и вовлечение
окружающего воздуха, что метод частицы в его простейшем виде не
учитывает. Например, в кучевом облаке мощные восходящие движения в
центральной части сопровождаются нисходящими движениями на его
периферии. Это делает реальные атмосферные процессы значительно сложнее,
чем предполагает идеализированная ``частица''.

ewpage

\hypertarget{ux431ux43bux43eux43a-1.-ux444ux438ux437ux438ux43aux430-ux430ux442ux43cux43eux441ux444ux435ux440ux44b-2}{%
\section{Блок 1. «Физика
атмосферы»}\label{ux431ux43bux43eux43a-1.-ux444ux438ux437ux438ux43aux430-ux430ux442ux43cux43eux441ux444ux435ux440ux44b-2}}

\hypertarget{ux442ux435ux43cux430-ux43bux443ux447ux438ux441ux442ux430ux44f-ux44dux43dux435ux440ux433ux438ux44f-ux432-ux430ux442ux43cux43eux441ux444ux435ux440ux435}{%
\subsection{1.3. Тема «Лучистая энергия в
атмосфере»}\label{ux442ux435ux43cux430-ux43bux443ux447ux438ux441ux442ux430ux44f-ux44dux43dux435ux440ux433ux438ux44f-ux432-ux430ux442ux43cux43eux441ux444ux435ux440ux435}}

\hypertarget{ux432ux435ux43bux438ux447ux438ux43dux44b-ux445ux430ux440ux430ux43aux442ux435ux440ux438ux437ux443ux44eux449ux438ux435-ux44dux43bux435ux43aux442ux440ux43eux43cux430ux433ux43dux438ux442ux43dux43eux435-ux438ux437ux43bux443ux447ux435ux43dux438ux435-ux43fux43eux442ux43eux43a-ux438ux43dux442ux435ux43dux441ux438ux432ux43dux43eux441ux442ux44c-ux438ux43dux441ux43eux43bux44fux446ux438ux44f}{%
\subsubsection{\texorpdfstring{\textbf{Величины, характеризующие
электромагнитное излучение (поток, интенсивность,
инсоляция)}}{Величины, характеризующие электромагнитное излучение (поток, интенсивность, инсоляция)}}\label{ux432ux435ux43bux438ux447ux438ux43dux44b-ux445ux430ux440ux430ux43aux442ux435ux440ux438ux437ux443ux44eux449ux438ux435-ux44dux43bux435ux43aux442ux440ux43eux43cux430ux433ux43dux438ux442ux43dux43eux435-ux438ux437ux43bux443ux447ux435ux43dux438ux435-ux43fux43eux442ux43eux43a-ux438ux43dux442ux435ux43dux441ux438ux432ux43dux43eux441ux442ux44c-ux438ux43dux441ux43eux43bux44fux446ux438ux44f}}

Перенос лучистой энергии в атмосфере описывается несколькими ключевыми
величинами, каждая из которых имеет важное метеорологическое значение.

\begin{itemize}
\item
  \textbf{Интенсивность излучения (\(I_{\lambda}\)):} Это
  фундаментальная характеристика поля излучения, представляющая собой
  количество энергии, проходящее в единицу времени через единичную
  площадку, перпендикулярную направлению распространения, в единичном
  телесном угле и в единичном интервале длин волн. Размерность:
  Вт·м⁻²·ср⁻¹·мкм⁻¹. Интенсивность является основной переменной в
  \textbf{уравнении переноса излучения}, которое решается в радиационных
  блоках численных моделей погоды и климата для расчета радиационных
  потоков тепла. Кроме того, спутниковые приборы (радиометры) измеряют
  именно интенсивность уходящего излучения в различных направлениях и
  спектральных каналах, что позволяет дистанционно определять
  температуру поверхности, облаков, а также восстанавливать вертикальные
  профили температуры и влажности атмосферы.
\item
  \textbf{Поток излучения (\(F_{\lambda}\)):} Это количество энергии,
  проходящее в единицу времени через единичную площадку (обычно
  горизонтальную) со всех направлений полусферы. Поток связан с
  интенсивностью через интегрирование по телесному углу. Например,
  нисходящий поток через горизонтальную площадку вычисляется как:
  \[F_{\lambda}^{\downarrow} = \int_{0}^{2\pi} \int_{0}^{\pi/2} I_{\lambda}(\theta, \phi) \cos\theta \sin\theta \,d\theta \,d\phi\]
  где \(\theta\) --- зенитный угол, а \(\phi\) --- азимут. Размерность:
  Вт·м⁻²·мкм⁻¹. Для метеорологии ключевое значение имеет вертикальная
  дивергенция суммарного (солнечного и теплового) потока излучения
  (\(\partial F_{net}/\partial z\)), так как она определяет скорость
  радиационного нагрева или охлаждения атмосферы. Этот процесс является
  ключевым неадиабатическим фактором, влияющим на температуру,
  статическую устойчивость и, в конечном счете, на динамику атмосферы.
\item
  \textbf{Инсоляция:} Поток прямой солнечной радиации, приходящий на
  горизонтальную поверхность. Его величина зависит от высоты Солнца над
  горизонтом (\(h\)): \(S' = S_p \sin(h)\), где \(S_p\) --- поток
  радиации на поверхность, перпендикулярную солнечным лучам.
\end{itemize}

\hypertarget{ux441ux43eux43bux43dux435ux447ux43dux430ux44f-ux440ux430ux434ux438ux430ux446ux438ux44f-ux43dux430-ux432ux435ux440ux445ux43dux435ux439-ux433ux440ux430ux43dux438ux446ux435-ux430ux442ux43cux43eux441ux444ux435ux440ux44b.-ux441ux43eux43bux43dux435ux447ux43dux430ux44f-ux43fux43eux441ux442ux43eux44fux43dux43dux430ux44f}{%
\subsubsection{\texorpdfstring{\textbf{Солнечная радиация на верхней
границе атмосферы. Солнечная
постоянная}}{Солнечная радиация на верхней границе атмосферы. Солнечная постоянная}}\label{ux441ux43eux43bux43dux435ux447ux43dux430ux44f-ux440ux430ux434ux438ux430ux446ux438ux44f-ux43dux430-ux432ux435ux440ux445ux43dux435ux439-ux433ux440ux430ux43dux438ux446ux435-ux430ux442ux43cux43eux441ux444ux435ux440ux44b.-ux441ux43eux43bux43dux435ux447ux43dux430ux44f-ux43fux43eux441ux442ux43eux44fux43dux43dux430ux44f}}

\begin{itemize}
\item
  \textbf{Спектр солнечного излучения:} Излучение Солнца близко к
  излучению \textbf{абсолютно чёрного тела} с температурой около 6000 К.
  Максимум энергии приходится на видимую часть спектра
  (\textasciitilde0.48 мкм), что определяется \textbf{законом смещения
  Вина}: \[\lambda_{max} = \frac{b}{T}\] где \(b \approx 2898\) мкм·К.
  Около 99\% энергии солнечной радиации заключено в диапазоне 0.1--4.0
  мкм, поэтому её называют \textbf{коротковолновой радиацией}. Это
  разделение на коротко- и длинноволновую радиацию принципиально важно
  для метеорологии, так как атмосфера по-разному взаимодействует с этими
  двумя типами излучения: она в основном прозрачна для коротковолнового
  и непрозрачна для длинноволнового.
\item
  \textbf{Солнечная постоянная (\(S_0\)):} Суммарный поток солнечной
  радиации на верхней границе атмосферы, падающий на перпендикулярную
  лучам площадку при среднем расстоянии от Земли до Солнца. Современное
  значение составляет \(S_0 \approx 1361\) \textbf{Вт/м²}. Хотя
  колебания самой солнечной постоянной малы, её распределение по широтам
  и сезонам, определяемое астрономическими факторами (циклами
  Миланковича), является главным климатообразующим фактором, который
  управляет сменой ледниковых и межледниковых эпох в геологическом
  прошлом. Для краткосрочной погоды важен не столько абсолютный поток,
  сколько его суточный и годовой ход, который определяет нагрев
  поверхности и, как следствие, развитие суточного хода температуры и
  конвекции.
\end{itemize}

\hypertarget{ux43fux43eux433ux43bux43eux449ux435ux43dux438ux435-ux438-ux440ux430ux441ux441ux435ux44fux43dux438ux435-ux441ux43eux43bux43dux435ux447ux43dux43eux439-ux440ux430ux434ux438ux430ux446ux438ux438.-ux437ux430ux43aux43eux43d-ux43eux441ux43bux430ux431ux43bux435ux43dux438ux44f-ux43fux43eux442ux43eux43aux43eux432-ux440ux430ux434ux438ux430ux446ux438ux438.-ux444ux443ux43dux43aux446ux438ux438-ux43fux440ux43eux43fux443ux441ux43aux430ux43dux438ux44f-ux438-ux43fux43eux433ux43bux43eux449ux435ux43dux438ux44f}{%
\subsubsection{\texorpdfstring{\textbf{Поглощение и рассеяние солнечной
радиации. Закон ослабления потоков радиации. Функции пропускания и
поглощения}}{Поглощение и рассеяние солнечной радиации. Закон ослабления потоков радиации. Функции пропускания и поглощения}}\label{ux43fux43eux433ux43bux43eux449ux435ux43dux438ux435-ux438-ux440ux430ux441ux441ux435ux44fux43dux438ux435-ux441ux43eux43bux43dux435ux447ux43dux43eux439-ux440ux430ux434ux438ux430ux446ux438ux438.-ux437ux430ux43aux43eux43d-ux43eux441ux43bux430ux431ux43bux435ux43dux438ux44f-ux43fux43eux442ux43eux43aux43eux432-ux440ux430ux434ux438ux430ux446ux438ux438.-ux444ux443ux43dux43aux446ux438ux438-ux43fux440ux43eux43fux443ux441ux43aux430ux43dux438ux44f-ux438-ux43fux43eux433ux43bux43eux449ux435ux43dux438ux44f}}

Проходя через атмосферу, поток прямой солнечной радиации ослабляется за
счёт \textbf{поглощения} и \textbf{рассеяния}.

\begin{itemize}
\item
  \textbf{Закон Бугера-Ламберта-Бера:} Описывает ослабление
  монохроматического излучения в среде:
  \[dI_{\lambda} = -I_{\lambda} k_{\lambda} \rho ds = -I_{\lambda} d\tau_{\lambda}\]
  где \(k_{\lambda}\) --- массовый коэффициент ослабления, \(\rho\) ---
  плотность, \(ds\) --- элементарный путь, \(\tau_{\lambda}\) ---
  оптическая толщина. После интегрирования по всей толще атмосферы
  формула принимает вид:
  \[I_{\lambda} = I_{\lambda,0} e^{-\tau_{\lambda} m}\] где
  \(I_{\lambda,0}\) --- интенсивность на верхней границе, а
  \(m = 1/\cos\theta_0\) --- оптическая масса атмосферы (\(\theta_0\)
  --- зенитный угол Солнца).
\item
  \textbf{Поглощение:} Происходит селективно. \textbf{Озон (\(O_3\))}
  поглощает УФ-радиацию, создавая тёплый слой в стратосфере и защищая
  биосферу. \textbf{Водяной пар (\(H_2O\))} и \textbf{углекислый газ
  (\(CO_2\))} имеют полосы поглощения в ближней ИК-области, внося прямой
  вклад в радиационный нагрев тропосферы.
\item
  \textbf{Рассеяние:}

  \begin{itemize}
  \tightlist
  \item
    \textbf{Рэлеевское (молекулярное) рассеяние:} Интенсивность
    рассеяния обратно пропорциональна четвёртой степени длины волны
    (\(I_{расс} \propto \lambda^{-4}\)). Это явление объясняет голубой
    цвет неба, который является визуальным индикатором чистой, свободной
    от крупных аэрозолей атмосферы.
  \item
    \textbf{Рассеяние Ми:} Происходит на аэрозольных частицах и облачных
    каплях. Зависимость от длины волны слабая, что объясняет белый цвет
    облаков и туманов. Белёсый оттенок неба, в отличие от ярко-голубого,
    свидетельствует о высокой концентрации аэрозолей в атмосфере (мгла).
  \end{itemize}
\item
  \textbf{Функция пропускания (\(P_{\Delta\lambda}\)):} Доля радиации в
  конечном спектральном интервале \(\Delta\lambda\), которая проходит
  через атмосферу. Является ключевой величиной в радиационных кодах
  моделей NWP для расчета потоков.
\end{itemize}

\hypertarget{ux445ux430ux440ux430ux43aux442ux435ux440ux438ux441ux442ux438ux43aux438-ux43fux440ux43eux437ux440ux430ux447ux43dux43eux441ux442ux438-ux430ux442ux43cux43eux441ux444ux435ux440ux44b.-ux444ux430ux43aux442ux43eux440-ux43cux443ux442ux43dux43eux441ux442ux438}{%
\subsubsection{\texorpdfstring{\textbf{Характеристики прозрачности
атмосферы. Фактор
мутности}}{Характеристики прозрачности атмосферы. Фактор мутности}}\label{ux445ux430ux440ux430ux43aux442ux435ux440ux438ux441ux442ux438ux43aux438-ux43fux440ux43eux437ux440ux430ux447ux43dux43eux441ux442ux438-ux430ux442ux43cux43eux441ux444ux435ux440ux44b.-ux444ux430ux43aux442ux43eux440-ux43cux443ux442ux43dux43eux441ux442ux438}}

\begin{itemize}
\tightlist
\item
  \textbf{Коэффициент прозрачности (\(p_m\)):} Интегральная
  характеристика ослабления прямой солнечной радиации всей толщей
  атмосферы. Используется в актинометрии для оценки общего содержания
  аэрозолей и водяного пара.
\item
  \textbf{Фактор мутности Линке (\(T_L\)):} Показывает, во сколько раз
  ослабление радиации в реальной атмосфере (с учётом аэрозолей и
  водяного пара) больше, чем в идеальной сухой и чистой атмосфере. Этот
  параметр является важным климатическим индикатором. Например, после
  мощных вулканических извержений (Эль-Чичон, Пинатубо) фактор мутности
  глобально возрастает из-за выброса стратосферного аэрозоля, что
  приводит к уменьшению приходящей солнечной радиации и вызывает
  кратковременное глобальное похолодание (на 0.2-0.5°C).
\end{itemize}

\hypertarget{ux441ux43fux435ux43aux442ux440ux430ux43bux44cux43dux44bux439-ux441ux43eux441ux442ux430ux432-ux441ux43eux43bux43dux435ux447ux43dux43eux439-ux440ux430ux434ux438ux430ux446ux438ux438-ux443-ux437ux435ux43cux43dux43eux439-ux43fux43eux432ux435ux440ux445ux43dux43eux441ux442ux438.-ux43eux441ux43eux431ux435ux43dux43dux43eux441ux442ux438-ux432-ux437ux430ux433ux440ux44fux437ux43dux435ux43dux43dux43eux439-ux430ux442ux43cux43eux441ux444ux435ux440ux435}{%
\subsubsection{\texorpdfstring{\textbf{Спектральный состав солнечной
радиации у земной поверхности. Особенности в загрязненной
атмосфере}}{Спектральный состав солнечной радиации у земной поверхности. Особенности в загрязненной атмосфере}}\label{ux441ux43fux435ux43aux442ux440ux430ux43bux44cux43dux44bux439-ux441ux43eux441ux442ux430ux432-ux441ux43eux43bux43dux435ux447ux43dux43eux439-ux440ux430ux434ux438ux430ux446ux438ux438-ux443-ux437ux435ux43cux43dux43eux439-ux43fux43eux432ux435ux440ux445ux43dux43eux441ux442ux438.-ux43eux441ux43eux431ux435ux43dux43dux43eux441ux442ux438-ux432-ux437ux430ux433ux440ux44fux437ux43dux435ux43dux43dux43eux439-ux430ux442ux43cux43eux441ux444ux435ux440ux435}}

У земной поверхности спектр прямой солнечной радиации сильно изменён:
УФ-часть ``вырезана'' озоном, а в ИК-части видны мощные полосы
поглощения водяного пара и \(CO_2\). Этот изменённый спектр определяет
энергию, доступную для фотосинтеза, и влияет на нагрев различных типов
поверхностей. В \textbf{загрязнённой атмосфере} значительно возрастает
\textbf{аэрозольное ослабление}. Аэрозоли (особенно сажа) могут не
только рассеивать, но и поглощать радиацию, что приводит к
дополнительному нагреву городской атмосферы и способствует формированию
``острова тепла''.

\hypertarget{ux43fux440ux44fux43cux43eux439-ux440ux430ux441ux441ux435ux44fux43dux43dux43eux439-ux438-ux441ux443ux43cux43cux430ux440ux43dux43eux439-ux43fux43eux442ux43eux43aux438-ux441ux43eux43bux43dux435ux447ux43dux43eux439-ux440ux430ux434ux438ux430ux446ux438ux438-ux438-ux432ux43bux438ux44fux44eux449ux438ux435-ux43dux430-ux43dux438ux445-ux444ux430ux43aux442ux43eux440ux44b}{%
\subsubsection{\texorpdfstring{\textbf{Прямой, рассеянной и суммарной
потоки солнечной радиации и влияющие на них
факторы}}{Прямой, рассеянной и суммарной потоки солнечной радиации и влияющие на них факторы}}\label{ux43fux440ux44fux43cux43eux439-ux440ux430ux441ux441ux435ux44fux43dux43dux43eux439-ux438-ux441ux443ux43cux43cux430ux440ux43dux43eux439-ux43fux43eux442ux43eux43aux438-ux441ux43eux43bux43dux435ux447ux43dux43eux439-ux440ux430ux434ux438ux430ux446ux438ux438-ux438-ux432ux43bux438ux44fux44eux449ux438ux435-ux43dux430-ux43dux438ux445-ux444ux430ux43aux442ux43eux440ux44b}}

\begin{itemize}
\tightlist
\item
  \textbf{Прямая радиация (\(S'\)):} Поток, приходящий непосредственно
  от диска Солнца.
\item
  \textbf{Рассеянная радиация (\(D\)):} Поток, приходящий от всего
  небесного свода.
\item
  \textbf{Суммарная радиация (\(Q\)):} Общий поток,
  \(Q = S' + D = S_p \sin(h) + D\). Соотношение между прямой и
  рассеянной радиацией является ключевым фактором для многих
  метеорологических и прикладных задач. Например, для солнечной
  энергетики важна прямая радиация. Для растительности важна суммарная
  радиация, но её рассеянная компонента может проникать глубже в
  растительный покров. \textbf{Облачность} является главным модулятором
  этих потоков. Плотная облачность может уменьшить суммарную радиацию на
  80-90\%, что является основным фактором, определяющим отсутствие
  значительного дневного прогрева и малый суточный ход температуры в
  пасмурный день.
\end{itemize}

\hypertarget{ux43eux442ux440ux430ux436ux435ux43dux438ux435-ux438-ux43fux43eux433ux43bux43eux449ux435ux43dux438ux435-ux441ux43eux43bux43dux435ux447ux43dux43eux439-ux440ux430ux434ux438ux430ux446ux438ux438-ux437ux435ux43cux43dux43eux439-ux43fux43eux432ux435ux440ux445ux43dux43eux441ux442ux44cux44e.-ux430ux43bux44cux431ux435ux434ux43e-ux43aux43eux44dux444ux444ux438ux446ux438ux435ux43dux442-ux43eux442ux440ux430ux436ux435ux43dux438ux44f}{%
\subsubsection{\texorpdfstring{\textbf{Отражение и поглощение солнечной
радиации земной поверхностью. Альбедо (коэффициент
отражения)}}{Отражение и поглощение солнечной радиации земной поверхностью. Альбедо (коэффициент отражения)}}\label{ux43eux442ux440ux430ux436ux435ux43dux438ux435-ux438-ux43fux43eux433ux43bux43eux449ux435ux43dux438ux435-ux441ux43eux43bux43dux435ux447ux43dux43eux439-ux440ux430ux434ux438ux430ux446ux438ux438-ux437ux435ux43cux43dux43eux439-ux43fux43eux432ux435ux440ux445ux43dux43eux441ux442ux44cux44e.-ux430ux43bux44cux431ux435ux434ux43e-ux43aux43eux44dux444ux444ux438ux446ux438ux435ux43dux442-ux43eux442ux440ux430ux436ux435ux43dux438ux44f}}

\textbf{Альбедо (\(\alpha\))} --- это отношение отражённого потока к
приходящему: \(\alpha = Q_{\uparrow}/Q_{\downarrow}\). Это ключевой
параметр, определяющий, какая часть солнечной энергии поглощается
поверхностью и идёт на её нагрев, запуская все последующие процессы в
пограничном слое.

\begin{itemize}
\tightlist
\item
  \textbf{Снег и лёд:} Высокое альбедо (80--95\%) приводит к тому, что
  большая часть энергии отражается. Это основа \textbf{ледово-альбедной
  обратной связи} --- одного из важнейших усиливающих механизмов в
  климатической системе. Потепление приводит к таянию снега/льда,
  уменьшению альбедо, что вызывает большее поглощение радиации и
  дальнейшее, ускоренное потепление.
\item
  \textbf{Вода:} Низкое альбедо (5--10\%) делает океан основным
  поглотителем солнечной энергии на планете, что объясняет его огромную
  роль в глобальном тепловом балансе.
\end{itemize}

\textbf{Поглощённая радиация} \(Q_{погл} = Q(1-\alpha)\) является
основным источником энергии для нагрева подстилающей поверхности и, как
следствие, нижних слоёв атмосферы.

\hypertarget{ux434ux43bux438ux43dux43dux43eux432ux43eux43bux43dux43eux432ux43eux435-ux438ux437ux43bux443ux447ux435ux43dux438ux435-ux437ux435ux43cux43dux43eux439-ux43fux43eux432ux435ux440ux445ux43dux43eux441ux442ux438-ux438-ux430ux442ux43cux43eux441ux444ux435ux440ux44b}{%
\subsubsection{\texorpdfstring{\textbf{Длинноволновое излучение земной
поверхности и
атмосферы}}{Длинноволновое излучение земной поверхности и атмосферы}}\label{ux434ux43bux438ux43dux43dux43eux432ux43eux43bux43dux43eux432ux43eux435-ux438ux437ux43bux443ux447ux435ux43dux438ux435-ux437ux435ux43cux43dux43eux439-ux43fux43eux432ux435ux440ux445ux43dux43eux441ux442ux438-ux438-ux430ux442ux43cux43eux441ux444ux435ux440ux44b}}

Земная поверхность и атмосфера излучают энергию в ИК-диапазоне (пик
около 10 мкм), поэтому это излучение называют \textbf{длинноволновым}.

\begin{itemize}
\tightlist
\item
  \textbf{Излучение земной поверхности (\(E_s^{\uparrow}\)):}
  Описывается \textbf{законом Стефана-Больцмана} для серого тела:
  \[E_s^{\uparrow} = \varepsilon \sigma T_s^4\] где \(\varepsilon\) ---
  излучательная способность, \(\sigma\) --- постоянная
  Стефана-Больцмана, \(T_s\) --- температура поверхности.
\item
  \textbf{Излучение атмосферы:} Атмосферные газы (в основном \(H_2O\),
  \(CO_2\), \(O_3\)) и облака поглощают длинноволновое излучение от
  поверхности и сами излучают в соответствии со своей температурой
  (\textbf{закон Кирхгофа}). Это взаимодействие является физической
  основой \textbf{парникового эффекта}.
\end{itemize}

\hypertarget{ux443ux445ux43eux434ux44fux449ux435ux435-ux438-ux432ux441ux442ux440ux435ux447ux43dux43eux435-ux438ux437ux43bux443ux447ux435ux43dux438ux435-ux430ux442ux43cux43eux441ux444ux435ux440ux44b.-ux44dux444ux444ux435ux43aux442ux438ux432ux43dux43eux435-ux438ux437ux43bux443ux447ux435ux43dux438ux435}{%
\subsubsection{\texorpdfstring{\textbf{Уходящее и встречное излучение
атмосферы. Эффективное
излучение}}{Уходящее и встречное излучение атмосферы. Эффективное излучение}}\label{ux443ux445ux43eux434ux44fux449ux435ux435-ux438-ux432ux441ux442ux440ux435ux447ux43dux43eux435-ux438ux437ux43bux443ux447ux435ux43dux438ux435-ux430ux442ux43cux43eux441ux444ux435ux440ux44b.-ux44dux444ux444ux435ux43aux442ux438ux432ux43dux43eux435-ux438ux437ux43bux443ux447ux435ux43dux438ux435}}

\begin{itemize}
\tightlist
\item
  \textbf{Встречное излучение атмосферы (\(E_a^{\downarrow}\)):} Поток
  длинноволновой радиации, излучаемый атмосферой вниз. Он компенсирует
  до 80-90\% излучения поверхности, поддерживая её температуру в среднем
  на 33°C выше, чем она была бы без атмосферы. Это и есть суть
  \textbf{парникового эффекта}.
\item
  \textbf{Эффективное излучение (\(E_{eff}\)):} Разность
  \(E_{eff} = E_s^{\uparrow} - E_a^{\downarrow}\). Характеризует чистую
  скорость радиационного охлаждения поверхности. Ночью при ясном небе
  \(E_{eff}\) максимально, что приводит к сильному выхолаживанию и
  является главной причиной образования приземных инверсий и
  радиационных туманов.
\item
  \textbf{Уходящее излучение (\(F^{\uparrow}\)):} Поток длинноволновой
  радиации, уходящий в космос. Его измерение со спутников является
  основой дистанционного зондирования и используется для определения
  температуры вершин облаков (и, следовательно, их высоты), температуры
  поверхности океана и содержания водяного пара в атмосфере.
\end{itemize}

\hypertarget{ux440ux430ux434ux438ux430ux446ux438ux43eux43dux43dux44bux439-ux431ux430ux43bux430ux43dux441-ux438-ux435ux433ux43e-ux441ux443ux442ux43eux447ux43dux44bux439ux433ux43eux434ux43eux432ux43eux439-ux445ux43eux434}{%
\subsubsection{\texorpdfstring{\textbf{Радиационный баланс и его
суточный/годовой
ход}}{Радиационный баланс и его суточный/годовой ход}}\label{ux440ux430ux434ux438ux430ux446ux438ux43eux43dux43dux44bux439-ux431ux430ux43bux430ux43dux441-ux438-ux435ux433ux43e-ux441ux443ux442ux43eux447ux43dux44bux439ux433ux43eux434ux43eux432ux43eux439-ux445ux43eux434}}

\textbf{Радиационный баланс (\(R_n\))} --- это алгебраическая сумма всех
потоков лучистой энергии.

\begin{itemize}
\tightlist
\item
  \textbf{Радиационный баланс земной поверхности:}
  \[R_n = Q(1-\alpha) - E_{eff} = (S' + D)(1-\alpha) - (E_s^{\uparrow} - E_a^{\downarrow})\]
  \(R_n\) является основным источником энергии для \textbf{теплового
  баланса} поверхности. Положительный \(R_n\) (днём) расходуется на
  нагрев почвы/воды, испарение и турбулентный нагрев воздуха.
  Отрицательный \(R_n\) (ночью) компенсируется потоками тепла из
  почвы/воды и воздуха к поверхности.
\item
  \textbf{Радиационный баланс системы Земля-атмосфера:}
  \[R_{sys} = S_0(1-\alpha_p) - F^{\uparrow}\] где \(\alpha_p\) ---
  планетарное альбедо (\textasciitilde0.3). В среднем за год для планеты
  в целом \(R_{sys} \approx 0\). Однако его \textbf{широтный дисбаланс}
  (избыток энергии в тропиках, недостаток в полярных широтах) является
  фундаментальной причиной существования \textbf{общей циркуляции
  атмосферы и океана}, которые выполняют меридиональный перенос тепла,
  поддерживая климатическое равновесие.
\end{itemize}

ewpage

\hypertarget{ux43eux43fux440ux435ux434ux435ux43bux435ux43dux438ux44f-ux43fux43eux43dux44fux442ux438ux439-ux438-ux432ux435ux43bux438ux447ux438ux43d-ux445ux430ux440ux430ux43aux442ux435ux440ux438ux437ux443ux44eux449ux438ux445-ux44dux43bux435ux43aux442ux440ux43eux43cux430ux433ux43dux438ux442ux43dux43eux435-ux438ux437ux43bux443ux447ux435ux43dux438ux435}{%
\section{Определения Понятий и Величин, Характеризующих Электромагнитное
Излучение}\label{ux43eux43fux440ux435ux434ux435ux43bux435ux43dux438ux44f-ux43fux43eux43dux44fux442ux438ux439-ux438-ux432ux435ux43bux438ux447ux438ux43d-ux445ux430ux440ux430ux43aux442ux435ux440ux438ux437ux443ux44eux449ux438ux445-ux44dux43bux435ux43aux442ux440ux43eux43cux430ux433ux43dux438ux442ux43dux43eux435-ux438ux437ux43bux443ux447ux435ux43dux438ux435}}

\hypertarget{ux43eux431ux449ux438ux435-ux43fux43eux43dux44fux442ux438ux44f-ux44dux43bux435ux43aux442ux440ux43eux43cux430ux433ux43dux438ux442ux43dux43eux433ux43e-ux438ux437ux43bux443ux447ux435ux43dux438ux44f}{%
\subsection{1. Общие Понятия Электромагнитного
Излучения}\label{ux43eux431ux449ux438ux435-ux43fux43eux43dux44fux442ux438ux44f-ux44dux43bux435ux43aux442ux440ux43eux43cux430ux433ux43dux438ux442ux43dux43eux433ux43e-ux438ux437ux43bux443ux447ux435ux43dux438ux44f}}

Электромагнитное излучение представляет собой энергию, переносимую
электромагнитными волнами (ЭМВ) различной длины. В пространстве
распространяются различные виды электромагнитных колебаний, включая
радиоволны, тепловое излучение тел, видимый свет, ультрафиолетовые и
рентгеновские лучи, а также лучи, испускаемые радиоактивными веществами.

В метеорологии особое значение имеют инфракрасные (тепловое излучение)
лучи с длиной волны от 100 до 0,76 мкм, видимые световые лучи (0,76 до
0,4 мкм) и ультрафиолетовые лучи (0,4 до 0,1 мкм).

\hypertarget{ux445ux430ux440ux430ux43aux442ux435ux440ux438ux441ux442ux438ux43aux438-ux44dux43bux435ux43aux442ux440ux43eux43cux430ux433ux43dux438ux442ux43dux44bux445-ux432ux43eux43bux43d}{%
\subsubsection{1.1. Характеристики Электромагнитных
Волн}\label{ux445ux430ux440ux430ux43aux442ux435ux440ux438ux441ux442ux438ux43aux438-ux44dux43bux435ux43aux442ux440ux43eux43cux430ux433ux43dux438ux442ux43dux44bux445-ux432ux43eux43bux43d}}

ЭМВ характеризуются следующими параметрами:

\begin{itemize}
\tightlist
\item
  \textbf{Скорость волны (C)}: Скорость перемещения отдельных ложбин или
  гребней волны. Для ЭМВ в вакууме C составляет приблизительно 3·10\^{}8
  м/с. В любой среде скорость ЭМВ уменьшается.
\item
  \textbf{Длина волны (λ)}: Расстояние между двумя соседними гребнями
  или ложбинами. Для ЭМВ она может варьироваться от 10\^{}-9 м до
  10\^{}6 м. Самые длинные волны, достигающие нескольких километров, ---
  это радиоволны, а самые короткие, измеряемые в нанометрах и ангстремах
  (1 нм = 10\^{}-6 мкм, 1 Å = 10\^{}-7 мкм), --- радиоактивные лучи.
\item
  \textbf{Период (T)}: Время прохождения двух соседних гребней (ложбин)
  через фиксированную точку наблюдения. Для ЭМВ период T = λ/C и зависит
  только от длины волны.
\end{itemize}

\hypertarget{ux432ux437ux430ux438ux43cux43eux434ux435ux439ux441ux442ux432ux438ux435-ux438ux437ux43bux443ux447ux435ux43dux438ux44f-ux441-ux430ux442ux43cux43eux441ux444ux435ux440ux43eux439}{%
\subsubsection{1.2. Взаимодействие Излучения с
Атмосферой}\label{ux432ux437ux430ux438ux43cux43eux434ux435ux439ux441ux442ux432ux438ux435-ux438ux437ux43bux443ux447ux435ux43dux438ux44f-ux441-ux430ux442ux43cux43eux441ux444ux435ux440ux43eux439}}

При прохождении через атмосферу электромагнитное излучение
взаимодействует с ее компонентами, что приводит к различным эффектам:

\begin{itemize}
\tightlist
\item
  \textbf{Показатель преломления (n)}: Отношение скорости
  распространения ЭМВ в вакууме (c) к скорости распространения в среде
  (v); n = c/v. Для воздуха n = 1,00029, для воды n = 1,333. В одной и
  той же среде скорость распространения может быть разной для разных
  длин ЭМВ, что называется \textbf{дисперсией}.
\item
  \textbf{Рефракция}: Явление отклонения луча света в сторону более
  плотного и холодного воздуха, когда он проходит через слои с
  различными показателями преломления, например, из космоса до земной
  поверхности.
\item
  \textbf{Отражение и альбедо (rλ)}: При прохождении радиации через
  границу слоев с резко различающимися показателями преломления,
  например из воздуха в воду, ЭМВ испытывают отражение. Альбедо
  определяет энергию отраженной радиации. Оно особенно велико для снега,
  льда и облаков.
\item
  \textbf{Рассеяние}: Процесс изменения направления распространения
  излучения, которое воспринимается как несобственное свечение среды.
  Возникает, когда падающая световая волна возбуждает вынужденные
  колебания электрических зарядов в неоднородностях атмосферы, которые
  становятся источниками вторичных ЭМВ. Самые короткие световые волны
  рассеиваются сильнее всего.
\item
  \textbf{Поглощение}: Происходит, когда атомы и молекулы газов,
  составляющих воздух, переходят в другое энергетическое состояние при
  воздействии солнечной радиации. Основными атмосферными газами,
  поглощающими солнечную радиацию, являются O2, O3, CO2, H2O, O, N.
  Молекулярный и атомарный кислород, а также азот и озон почти полностью
  поглощают ультрафиолетовые лучи. H2O, CO2 и O3 поглощают инфракрасную
  радиацию. В видимой области спектра атмосферные газы энергию не
  поглощают. Газы поглощают радиацию селективно, т.е. только на
  определенных длинах волн, свойственных каждому газу.
\end{itemize}

\hypertarget{ux442ux435ux43fux43bux43eux432ux43eux435-ux438ux437ux43bux443ux447ux435ux43dux438ux435-ux438-ux441ux432ux44fux437ux430ux43dux43dux44bux435-ux441-ux43dux438ux43c-ux432ux435ux43bux438ux447ux438ux43dux44b}{%
\subsection{2. Тепловое Излучение и Связанные с ним
Величины}\label{ux442ux435ux43fux43bux43eux432ux43eux435-ux438ux437ux43bux443ux447ux435ux43dux438ux435-ux438-ux441ux432ux44fux437ux430ux43dux43dux44bux435-ux441-ux43dux438ux43c-ux432ux435ux43bux438ux447ux438ux43dux44b}}

Тепловое излучение --- это электромагнитное излучение, испускаемое
веществом за счет его внутренней энергии, возникающее в результате
квантовых переходов атомов из неустойчивых состояний в устойчивые.
Процесс поглощения энергии ЭМВ определенной частоты (длины волн)
вызывает обратный переход из устойчивых состояний в неустойчивые.

\hypertarget{ux445ux430ux440ux430ux43aux442ux435ux440ux438ux441ux442ux438ux43aux438-ux442ux435ux43fux43bux43eux432ux43eux433ux43e-ux438ux437ux43bux443ux447ux435ux43dux438ux44f}{%
\subsubsection{2.1. Характеристики Теплового
Излучения}\label{ux445ux430ux440ux430ux43aux442ux435ux440ux438ux441ux442ux438ux43aux438-ux442ux435ux43fux43bux43eux432ux43eux433ux43e-ux438ux437ux43bux443ux447ux435ux43dux438ux44f}}

\begin{itemize}
\tightlist
\item
  \textbf{Абсолютно черное тело}: Тело, полностью поглощающее всю
  падающую на него радиацию. Излучение абсолютно черного тела является
  верхним пределом излучения для всех тел при данной температуре. Оно
  является изотропным, т.е. не зависит от природы тел и направления
  распространения.
\item
  \textbf{Излучательная способность (Jλ, J)}: Интенсивность радиации,
  испускаемой единицей площади в единицу времени.
\item
  \textbf{Поглощательная способность (aλ)}: Доля падающей радиации,
  которая поглощается телом.
\end{itemize}

\hypertarget{ux437ux430ux43aux43eux43dux44b-ux442ux435ux43fux43bux43eux432ux43eux433ux43e-ux438ux437ux43bux443ux447ux435ux43dux438ux44f}{%
\subsubsection{2.2. Законы Теплового
Излучения}\label{ux437ux430ux43aux43eux43dux44b-ux442ux435ux43fux43bux43eux432ux43eux433ux43e-ux438ux437ux43bux443ux447ux435ux43dux438ux44f}}

\begin{itemize}
\tightlist
\item
  \textbf{Закон Кирхгофа}: Отношение излучательной способности
  поверхности к ее поглощательной способности равно интенсивности
  черного излучения и зависит только от температуры и длины волны. Газы,
  которые поглощают энергию, также должны ее излучать.
\item
  \textbf{Закон Стефана-Больцмана}: Полный поток излучения абсолютно
  черного тела зависит от его абсолютной температуры. Для серых тел
  излучение земной поверхности (Es) с высокой точностью может быть
  вычислено по формуле Es = ε·σ·T\^{}4, где ε --- коэффициент
  ``серости'' (для Земли в среднем ≈ 0,95).
\item
  \textbf{Закон Планка}: Описывает зависимость монохроматического
  излучения абсолютно черного тела Bλ(T) от длины волны (λ) и абсолютной
  температуры (T). Функция Планка Bλ(T) обращается в нуль при λ=0 и λ=∞,
  имея максимум интенсивности излучения в промежутке длин волн.
\item
  \textbf{Закон смещения Вина}: Произведение длины волны, при которой
  черное излучение достигает максимального значения, на абсолютную
  температуру излучающего тела является постоянной величиной. С
  повышением температуры максимум энергии излучения смещается в область
  более коротких волн.
\end{itemize}

\hypertarget{ux440ux430ux434ux438ux430ux446ux438ux43eux43dux43dux44bux435-ux43fux43eux442ux43eux43aux438-ux438-ux431ux430ux43bux430ux43dux441ux44b-ux432-ux430ux442ux43cux43eux441ux444ux435ux440ux435}{%
\subsection{3. Радиационные Потоки и Балансы в
Атмосфере}\label{ux440ux430ux434ux438ux430ux446ux438ux43eux43dux43dux44bux435-ux43fux43eux442ux43eux43aux438-ux438-ux431ux430ux43bux430ux43dux441ux44b-ux432-ux430ux442ux43cux43eux441ux444ux435ux440ux435}}

\hypertarget{ux432ux438ux434ux44b-ux440ux430ux434ux438ux430ux446ux438ux43eux43dux43dux44bux445-ux43fux43eux442ux43eux43aux43eux432}{%
\subsubsection{3.1. Виды Радиационных
Потоков}\label{ux432ux438ux434ux44b-ux440ux430ux434ux438ux430ux446ux438ux43eux43dux43dux44bux445-ux43fux43eux442ux43eux43aux43eux432}}

\begin{itemize}
\tightlist
\item
  \textbf{Коротковолновое излучение}: Солнечное излучение, волны
  которого короче 4 мкм.
\item
  \textbf{Длинноволновое излучение}: Земное излучение, волны которого
  длиннее 4 мкм. Потоки длинноволновой радиации в атмосфере в основном
  состоят из инфракрасного излучения Земли и атмосферы.
\item
  \textbf{Противоизлучение атмосферы (Ea)}: Часть длинноволнового
  излучения атмосферы, которая возвращается к поверхности Земли.
\item
  \textbf{Уходящее излучение}: Часть излучения атмосферы, которая уходит
  в космос.
\end{itemize}

\hypertarget{ux440ux430ux434ux438ux430ux446ux438ux43eux43dux43dux44bux435-ux431ux430ux43bux430ux43dux441ux44b}{%
\subsubsection{3.2. Радиационные
Балансы}\label{ux440ux430ux434ux438ux430ux446ux438ux43eux43dux43dux44bux435-ux431ux430ux43bux430ux43dux441ux44b}}

\begin{itemize}
\tightlist
\item
  \textbf{Эффективное излучение (Eэф)}: Разность между излучением
  подстилающей поверхности (Es) и противоизлучением атмосферы (Ea).
  Характеризует чистую потерю энергии земной поверхностью при излучении
  и обычно является положительной величиной (направлено вверх). Зависит
  от температуры (растет с ростом) и влажности воздуха (уменьшается с
  ростом).
\item
  \textbf{Радиационный баланс (R)}: Разность значения поглощенной
  суммарной солнечной радиации (Qs) и эффективного излучения (Eэф).
  Определяется для различных временных промежутков. Он почти всегда
  положителен днем, поскольку поступление солнечной радиации значительно
  превышает эффективное излучение, и отрицателен ночью.
\item
  \textbf{Радиационный баланс Земли как планеты}: Земля в целом
  находится в состоянии радиационного равновесия, при котором поток
  приходящей от Солнца радиации уравновешен потоком отраженной
  коротковолновой и уходящей длинноволновой радиации.
\item
  \textbf{Радиационный баланс поверхности Земли}: Поверхность Земли не
  находится в состоянии радиационного равновесия; эффективное излучение
  лишь частично компенсирует приток от поглощенной радиации.
\item
  \textbf{Радиационный баланс атмосферы}: Атмосфера также не находится в
  состоянии радиационного равновесия; поток энергии в атмосфере больше,
  чем радиационные потоки, что указывает на высокую скорость
  преобразования энергии.
\end{itemize}

\hypertarget{ux434ux43eux43fux43eux43bux43dux438ux442ux435ux43bux44cux43dux44bux435-ux43aux43eux44dux444ux444ux438ux446ux438ux435ux43dux442ux44b}{%
\subsubsection{3.3. Дополнительные
Коэффициенты}\label{ux434ux43eux43fux43eux43bux43dux438ux442ux435ux43bux44cux43dux44bux435-ux43aux43eux44dux444ux444ux438ux446ux438ux435ux43dux442ux44b}}

\begin{itemize}
\tightlist
\item
  \textbf{Коэффициент поглощения (αλ)}, \textbf{коэффициент пропускания
  (τλ)}, \textbf{коэффициент отражения (rλ)}: Вводятся для
  монохроматического потока радиации.
\end{itemize}

ewpage

Приветствую, коллега. Давайте углубимся в определения этих
фундаментальных величин, которые являются краеугольными в актинометрии и
в целом в динамической метеорологии.

\hypertarget{ux43eux43fux440ux435ux434ux435ux43bux435ux43dux438ux44f-ux43fux43eux43dux44fux442ux438ux439-ux438-ux432ux435ux43bux438ux447ux438ux43d-ux445ux430ux440ux430ux43aux442ux435ux440ux438ux437ux443ux44eux449ux438ux445-ux44dux43bux435ux43aux442ux440ux43eux43cux430ux433ux43dux438ux442ux43dux43eux435-ux438ux437ux43bux443ux447ux435ux43dux438ux435-1}{%
\section{Определения Понятий и Величин, Характеризующих Электромагнитное
Излучение}\label{ux43eux43fux440ux435ux434ux435ux43bux435ux43dux438ux44f-ux43fux43eux43dux44fux442ux438ux439-ux438-ux432ux435ux43bux438ux447ux438ux43d-ux445ux430ux440ux430ux43aux442ux435ux440ux438ux437ux443ux44eux449ux438ux445-ux44dux43bux435ux43aux442ux440ux43eux43cux430ux433ux43dux438ux442ux43dux43eux435-ux438ux437ux43bux443ux447ux435ux43dux438ux435-1}}

\hypertarget{ux43fux43eux442ux43eux43a-ux440ux430ux434ux438ux430ux446ux438ux438}{%
\subsection{1. Поток
Радиации}\label{ux43fux43eux442ux43eux43a-ux440ux430ux434ux438ux430ux446ux438ux438}}

Поток радиации, в контексте метеорологии, чаще всего подразумевает
\textbf{плотность потока радиации}. Эта величина определяет количество
энергии, переносимой за единицу времени (мощность) через единицу площади
(1 м²) рассматриваемой поверхности. Единицей измерения плотности потока
радиации является Ватт на квадратный метр (Вт/м²).

При рассмотрении излучения в определенном интервале длин волн, мы
говорим о \textbf{монохроматической плотности потока радиации}.
Количество лучистой энергии в интервале длин волн от \texttt{λ} до
\texttt{λ+dλ}, поступающей на единичную площадку \texttt{dS} в единицу
времени из телесного угла \texttt{dω} в направлении, определяемом
полярным углом \texttt{ϑ} и азимутальным углом \texttt{ϕ}, выражается
формулой \texttt{dE\_λ\ =\ j\_λ(ϕ,ϑ)\ *\ cos(ϑ)\ *\ dω\ *\ dλ}. Здесь
\texttt{j\_λ(ϕ,ϑ)} -- это интенсивность излучения.

Если излучение является изотропным, то есть его интенсивность
\texttt{j\_λ} не зависит от направления, то монохроматический поток
радиации из полупространства связан с его интенсивностью простой
формулой: \texttt{E\_λ\ =\ π\ *\ j\_λ}.

\textbf{Полный поток радиации} из полупространства получается путем
интегрирования монохроматического потока по всем длинам волн в диапазоне
от 0 до бесконечности: \texttt{E\ =\ ∫(от\ 0\ до\ ∞)\ E\_λ\ *\ dλ}.

В целом, лучистый теплообмен в атмосфере формируется в результате
поглощения и излучения электромагнитных волн слоями воздуха.

\hypertarget{ux438ux43dux442ux435ux43dux441ux438ux432ux43dux43eux441ux442ux44c-ux438ux437ux43bux443ux447ux435ux43dux438ux44f}{%
\subsection{2. Интенсивность
Излучения}\label{ux438ux43dux442ux435ux43dux441ux438ux432ux43dux43eux441ux442ux44c-ux438ux437ux43bux443ux447ux435ux43dux438ux44f}}

\textbf{Интенсивность излучения} (\texttt{j\_λ} или \texttt{J(λ,T)}) --
это характеристика излучения, которая, согласно законам Кирхгофа, для
абсолютно черного тела является универсальной функцией температуры и
длины волны. Она не зависит от индивидуальных свойств вещества.

Для \textbf{абсолютно черного тела} -- гипотетического объекта, который
полностью поглощает всю падающую на него радиацию (\texttt{α\_λ\ =\ 1},
\texttt{r\_λ\ =\ 0}, \texttt{τ\_λ\ =\ 0}) и излучает максимально
возможную энергию при данной температуре -- интенсивность излучения
описывается \textbf{формулой Планка}:
\texttt{J(λ,T)\ =\ (c1\ *\ λ\^{}-5)\ /\ (exp(c2\ /\ (λ\ *\ k\ *\ T))\ -\ 1)},
где \texttt{h} -- постоянная Планка, \texttt{c} -- скорость света,
\texttt{k} -- постоянная Больцмана. Из этой формулы следует, что
интенсивность монохроматического излучения обращается в нуль при очень
малых и очень больших длинах волн, а в промежутке между ними существует
как минимум один максимум интенсивности излучения, зависящий от
температуры.

\textbf{Отношение излучательной способности поверхности к ее
поглощательной способности равно интенсивности черного излучения и
зависит только от температуры и длины волны}.

Сравнивая источники лучистой энергии для атмосферы, следует отметить,
что интенсивность солнечной радиации во всех частях спектра значительно
больше интенсивности излучения Земли и самой атмосферы, особенно в
коротковолновой области. Однако солнечная радиация поступает в атмосферу
из очень малого телесного угла, что позволяет пренебречь потоками
длинноволнового излучения Солнца по сравнению с излучением Земли и
атмосферы.

При прохождении радиации через тонкий горизонтальный слой атмосферы
\texttt{dz} с плотностью поглощающего вещества \texttt{ρ\_n(z)}, ее
интенсивность \texttt{j\_λ} уменьшается за счет поглощения слоем
\texttt{dz} и увеличивается за счет излучения этого слоя. Для
длинноволновой радиации эффект рассеяния очень мал и им обычно
пренебрегают.

\hypertarget{ux438ux43dux441ux43eux43bux44fux446ux438ux44f}{%
\subsection{3.
Инсоляция}\label{ux438ux43dux441ux43eux43bux44fux446ux438ux44f}}

\textbf{Инсоляция} определяется как поступление солнечной радиации на
земную поверхность. Это ключевой фактор, определяющий тепловой режим
подстилающей поверхности.

Распределение солнечной радиации по Земле и, соответственно, величина
инсоляции зависят от нескольких факторов:

\begin{itemize}
\tightlist
\item
  \textbf{Географическая широта:} Определяет полуденную высоту солнца и
  продолжительность облучения. Это является основной причиной
  географической зональности в распределении температуры воздуха и
  других климатических характеристик.
\item
  \textbf{Вращение Земли вокруг своей оси:} Влияет на суточный ход
  инсоляции.
\item
  \textbf{Вращение Земли вокруг Солнца:} Влияет на годовой ход инсоляции
  через склонение Солнца.
\end{itemize}

Основным источником тепла, поступающего на подстилающую поверхность,
является солнечная радиация, которая поглощается в тонком слое толщиной
в несколько миллиметров и полностью превращается в тепловую энергию.

Изменения на Солнце, такие как \textbf{солнечные пятна} -- области с
пониженной температурой на поверхности Солнца, могут приводить к
уменьшению инсоляции примерно на 0.1\% в течение нескольких лет. Такое,
казалось бы, небольшое уменьшение может иметь заметные климатические
последствия.

Инсоляция является частью \textbf{радиационного баланса} подстилающей
поверхности, который представляет собой разность между поглощенной
суммарной солнечной радиацией и эффективным излучением. Радиационный
баланс имеет ярко выраженный суточный и годовой ход.

ewpage

\hypertarget{ux440ux430ux441ux43fux440ux435ux434ux435ux43bux435ux43dux438ux435-ux44dux43dux435ux440ux433ux438ux438-ux43fux43e-ux441ux43fux435ux43aux442ux440ux443-ux438-ux438ux43dux442ux435ux433ux440ux430ux43bux44cux43dux44bux439-ux43fux43eux442ux43eux43a-ux441ux43eux43bux43dux435ux447ux43dux43eux439-ux440ux430ux434ux438ux430ux446ux438ux438-ux43dux430-ux432ux435ux440ux445ux43dux435ux439-ux433ux440ux430ux43dux438ux446ux435-ux430ux442ux43cux43eux441ux444ux435ux440ux44b}{%
\section{Распределение энергии по спектру и интегральный поток солнечной
радиации на верхней границе
атмосферы}\label{ux440ux430ux441ux43fux440ux435ux434ux435ux43bux435ux43dux438ux435-ux44dux43dux435ux440ux433ux438ux438-ux43fux43e-ux441ux43fux435ux43aux442ux440ux443-ux438-ux438ux43dux442ux435ux433ux440ux430ux43bux44cux43dux44bux439-ux43fux43eux442ux43eux43a-ux441ux43eux43bux43dux435ux447ux43dux43eux439-ux440ux430ux434ux438ux430ux446ux438ux438-ux43dux430-ux432ux435ux440ux445ux43dux435ux439-ux433ux440ux430ux43dux438ux446ux435-ux430ux442ux43cux43eux441ux444ux435ux440ux44b}}

\hypertarget{ux440ux430ux441ux43fux440ux435ux434ux435ux43bux435ux43dux438ux435-ux44dux43dux435ux440ux433ux438ux438-ux43fux43e-ux441ux43fux435ux43aux442ux440ux443}{%
\subsection{Распределение энергии по
спектру}\label{ux440ux430ux441ux43fux440ux435ux434ux435ux43bux435ux43dux438ux435-ux44dux43dux435ux440ux433ux438ux438-ux43fux43e-ux441ux43fux435ux43aux442ux440ux443}}

Солнечная радиация представляет собой энергию, переносимую
электромагнитными волнами различной длины. С точки зрения метеорологии,
наибольшее значение имеют инфракрасные (тепловое излучение с длиной
волны от 0.76 до 100 мкм), видимые световые (от 0.4 до 0.76 мкм) и
ультрафиолетовые лучи (от 0.1 до 0.4 мкм).

Спектр солнечной радиации, достигающей верхней границы атмосферы, по
своим характеристикам очень близок к излучению абсолютно черного тела с
температурой поверхности около 5900 К. Это объясняется законом Планка,
который описывает монохроматическое излучение абсолютно черного тела как
функцию длины волны и температуры:
\texttt{Bλ(T)\ =\ C1λ\^{}-5\ *\ (exp(C2/λT)\ -\ 1)\^{}-1}, где
\texttt{C1} и \texttt{C2} --- постоянные. Согласно этому закону, по мере
понижения температуры максимум излучения смещается в сторону более
длинных волн. В связи с высокой температурой Солнца, его излучение
сосредоточено преимущественно в коротковолновой части спектра (длины
волн короче 4 мкм), что существенно отличает его от длинноволнового
излучения Земли и атмосферы.

\hypertarget{ux438ux43dux442ux435ux433ux440ux430ux43bux44cux43dux44bux439-ux43fux43eux442ux43eux43a-ux441ux43eux43bux43dux435ux447ux43dux43eux439-ux440ux430ux434ux438ux430ux446ux438ux438-ux43dux430-ux432ux435ux440ux445ux43dux435ux439-ux433ux440ux430ux43dux438ux446ux435-ux430ux442ux43cux43eux441ux444ux435ux440ux44b}{%
\subsection{Интегральный поток солнечной радиации на верхней границе
атмосферы}\label{ux438ux43dux442ux435ux433ux440ux430ux43bux44cux43dux44bux439-ux43fux43eux442ux43eux43a-ux441ux43eux43bux43dux435ux447ux43dux43eux439-ux440ux430ux434ux438ux430ux446ux438ux438-ux43dux430-ux432ux435ux440ux445ux43dux435ux439-ux433ux440ux430ux43dux438ux446ux435-ux430ux442ux43cux43eux441ux444ux435ux440ux44b}}

Интегральный поток солнечной радиации, поступающей на верхнюю границу
атмосферы, определяется понятием солнечной постоянной (\texttt{Io}).
Солнечная постоянная представляет собой поток солнечной радиации,
приходящей на единичную площадку, перпендикулярную солнечным лучам, за
единицу времени на среднем расстоянии от Земли до Солнца.

Согласно актуальным данным, значение солнечной постоянной \texttt{Io}
составляет примерно 1380 Вт/м². Это значение может варьироваться, но
колебания, обусловленные солнечной активностью или изменением параметров
орбиты Земли, как правило, не превышают одного процента. Данный
интегральный поток энергии является основным источником, обусловливающим
атмосферные движения и, в конечном счете, определяющим погоду и климат
на Земле.

ewpage

\hypertarget{ux441ux43eux43bux43dux435ux447ux43dux430ux44f-ux43fux43eux441ux442ux43eux44fux43dux43dux430ux44f-ux43fux43eux433ux43bux43eux449ux435ux43dux438ux435-ux438-ux440ux430ux441ux441ux435ux44fux43dux438ux435-ux441ux43eux43bux43dux435ux447ux43dux43eux439-ux440ux430ux434ux438ux430ux446ux438ux438-ux432-ux430ux442ux43cux43eux441ux444ux435ux440ux435}{%
\section{Солнечная Постоянная, Поглощение и Рассеяние Солнечной Радиации
в
Атмосфере}\label{ux441ux43eux43bux43dux435ux447ux43dux430ux44f-ux43fux43eux441ux442ux43eux44fux43dux43dux430ux44f-ux43fux43eux433ux43bux43eux449ux435ux43dux438ux435-ux438-ux440ux430ux441ux441ux435ux44fux43dux438ux435-ux441ux43eux43bux43dux435ux447ux43dux43eux439-ux440ux430ux434ux438ux430ux446ux438ux438-ux432-ux430ux442ux43cux43eux441ux444ux435ux440ux435}}

Как коллеги, мы прекрасно понимаем, что лучистый теплообмен является
фундаментальным процессом в атмосфере, определяющим ее энергетический
баланс и, как следствие, динамические и термодинамические процессы,
формирующие погоду и климат. Основным источником этой энергии для земной
атмосферы, безусловно, является солнечное излучение.

\hypertarget{ux441ux43eux43bux43dux435ux447ux43dux430ux44f-ux43fux43eux441ux442ux43eux44fux43dux43dux430ux44f}{%
\subsection{Солнечная
Постоянная}\label{ux441ux43eux43bux43dux435ux447ux43dux430ux44f-ux43fux43eux441ux442ux43eux44fux43dux43dux430ux44f}}

Солнечная радиация представляет собой тепловое излучение, испускаемое
Солнцем за счет его внутренней энергии, распространяющееся в виде
электромагнитных волн (ЭМВ) различных длин.

Солнечная постоянная (\(I_0\)) --- это величина потока солнечной
радиации, поступающего на верхнюю границу атмосферы Земли. Она
определяется как количество солнечной энергии, проходящей за единицу
времени через единичную площадку, расположенную перпендикулярно
солнечным лучам на среднем расстоянии Земли от Солнца.

Расчетное значение солнечной постоянной составляет приблизительно
\(1380 \, \text{Вт/м}^2\), хотя в некоторых источниках указывается
\(1,37 \, \text{кВт/м}^2\). Важно отметить, что эта величина не
абсолютно постоянна и может изменяться. Основные факторы, вызывающие ее
колебания, --- это изменения излучения самого Солнца (солнечная
активность, связанные с солнечными пятнами) и вариации расстояния от
Земли до Солнца, обусловленные изменением параметров земной орбиты
(например, колебания эксцентриситета орбиты с периодом около 100 000
лет). Эти колебания, однако, обычно не превышают одного процента от
номинального значения.

Шарообразность Земли приводит к тому, что солнечная радиация
перехватывается диском Земли, а не всей поверхностью. Приходящая энергия
\(I_0\) перехватывается площадью, равной площади поперечного сечения
Земли, то есть \(\pi R^2 I_0\), где \(R\) --- радиус Земли.

\hypertarget{ux43fux43eux433ux43bux43eux449ux435ux43dux438ux435-ux438-ux440ux430ux441ux441ux435ux44fux43dux438ux435-ux441ux43eux43bux43dux435ux447ux43dux43eux439-ux440ux430ux434ux438ux430ux446ux438ux438-ux432-ux430ux442ux43cux43eux441ux444ux435ux440ux435}{%
\subsection{Поглощение и Рассеяние Солнечной Радиации в
Атмосфере}\label{ux43fux43eux433ux43bux43eux449ux435ux43dux438ux435-ux438-ux440ux430ux441ux441ux435ux44fux43dux438ux435-ux441ux43eux43bux43dux435ux447ux43dux43eux439-ux440ux430ux434ux438ux430ux446ux438ux438-ux432-ux430ux442ux43cux43eux441ux444ux435ux440ux435}}

По пути из космического пространства к земной поверхности солнечная
радиация взаимодействует с атмосферными газами, аэрозолями и облаками.
Атмосфера является ``мутной средой'' для света, содержащей хаотически
расположенные флуктуации плотности воздуха, частицы атмосферного
аэрозоля, капельки воды и снежинки. Эти взаимодействия включают процессы
поглощения, рассеяния и отражения.

\hypertarget{ux43fux43eux433ux43bux43eux449ux435ux43dux438ux435-ux440ux430ux434ux438ux430ux446ux438ux438}{%
\subsubsection{Поглощение
Радиации}\label{ux43fux43eux433ux43bux43eux449ux435ux43dux438ux435-ux440ux430ux434ux438ux430ux446ux438ux438}}

Поглощение радиации происходит, когда атомы и молекулы газов,
составляющих воздух, переходят в другое энергетическое состояние под
воздействием солнечной радиации. При этом лучистая энергия
трансформируется в тепловую.

Ключевые газы, участвующие в поглощении солнечного излучения, включают:

\begin{itemize}
\tightlist
\item
  \textbf{Молекулярный и атомарный кислород (\(O_2, O\)):} Практически
  полностью поглощают ультрафиолетовое излучение, в основном в верхних
  слоях атмосферы.
\item
  \textbf{Озон (\(O_3\)):} Также активно поглощает ультрафиолетовые
  лучи.
\item
  \textbf{Углекислый газ (\(CO_2\)) и водяной пар (\(H_2O\)):} Поглощают
  преимущественно инфракрасную радиацию. Важность водяного пара в
  поглощении лучистой энергии и накоплении ее Землей особенно велика.
\item
  \textbf{Азот (\(N_2\)):} В меньшей степени участвует в поглощении.
\end{itemize}

Важно отметить, что в видимой области спектра солнечной радиации
атмосферные газы практически не поглощают энергию.

\hypertarget{ux440ux430ux441ux441ux435ux44fux43dux438ux435-ux440ux430ux434ux438ux430ux446ux438ux438}{%
\subsubsection{Рассеяние
Радиации}\label{ux440ux430ux441ux441ux435ux44fux43dux438ux435-ux440ux430ux434ux438ux430ux446ux438ux438}}

Рассеяние -- это процесс изменения направления распространения
излучения, в результате которого свет воспринимается как несобственное
свечение среды. Интенсивность и характер рассеяния зависят от
соотношения размера рассеивающих частиц и длины волны падающего
излучения.

Выделяют два основных типа рассеяния:

\begin{enumerate}
\def\labelenumi{\arabic{enumi}.}
\tightlist
\item
  \textbf{Рэлеевское рассеяние:} Возникает, когда размер рассеивающих
  неоднородностей (например, молекул газов) значительно меньше длины
  волны падающего света. В этом случае поток рассеянной энергии сильно
  зависит от длины волны: короткие световые волны (фиолетовые, голубые)
  рассеиваются сильнее всего. Это объясняет голубой цвет неба в дневное
  время. При низком положении Солнца (закат) голубые волны поглощаются
  или рассеиваются за горизонтом, и до наблюдателя доходят
  преимущественно оранжевые и красные лучи, определяющие цвет неба.
\item
  \textbf{Рассеяние Ми:} Происходит, когда размер частиц (например,
  атмосферных аэрозолей, капель облаков) сопоставим или больше длины
  волны. В этом случае зависимость рассеянной радиации от длины волны
  исчезает, и белый свет после рассеяния остается белым. Этот механизм
  объясняет белый цвет облаков, не дающих осадков.
\end{enumerate}

Кроме того, существуют другие явления, влияющие на солнечную радиацию:

\begin{itemize}
\tightlist
\item
  \textbf{Отражение:} При взаимодействии световой волны с веществом
  часть энергии отражается. Альбедо -- коэффициент отражения -- особенно
  велико для снега, льда и облаков. Облачность, например, значительно
  уменьшает приток солнечной радиации к земной поверхности.
\item
  \textbf{Преломление (Рефракция):} Отклонение луча света в сторону
  более плотного воздуха, что связано с увеличением показателя
  преломления в более плотных слоях атмосферы.
\end{itemize}

\hypertarget{ux43eux431ux449ux435ux435-ux43eux441ux43bux430ux431ux43bux435ux43dux438ux435-ux440ux430ux434ux438ux430ux446ux438ux438}{%
\subsubsection{Общее Ослабление
Радиации}\label{ux43eux431ux449ux435ux435-ux43eux441ux43bux430ux431ux43bux435ux43dux438ux435-ux440ux430ux434ux438ux430ux446ux438ux438}}

В результате поглощения, рассеяния и отражения солнечная радиация
ослабляется по пути к земной поверхности. Максимально возможный поток
радиации, достигающий поверхности Земли (так называемая ``подзонная''
солнечная постоянная), уменьшается с \(1380 \, \text{Вт/м}^2\) до
\(1250 \, \text{Вт/м}^2\).

Суммарный поток лучистой энергии в атмосфере включает коротковолновое
солнечное излучение, направленное вниз, а также длинноволновое излучение
Земли и самой атмосферы. При этом солнечная радиация, поступающая от
Солнца, называется прямой радиацией, а та, что поступает от небесной
полусферы в результате рассеяния, называется рассеянной радиацией. Их
сумма образует суммарную радиацию.

Земля как планета в целом находится в состоянии радиационного
равновесия, при котором приходящий от Солнца поток радиации
уравновешивается потоком отраженной коротковолновой и уходящей
длинноволновой радиации. Однако атмосфера и земная поверхность по
отдельности не находятся в состоянии радиационного равновесия;
атмосферные процессы, такие как перенос энергии и влаги, играют ключевую
роль в распределении и трансформации этой энергии.

ewpage

Коллега, возвращаясь к вопросам лучистого обмена в атмосфере, давайте
разберем механизм ослабления как монохроматических, так и интегральных
потоков радиации. Это краеугольный камень в понимании теплового режима
атмосферы и формирования климата.

\hypertarget{ux437ux430ux43aux43eux43d-ux43eux441ux43bux430ux431ux43bux435ux43dux438ux44f-ux43cux43eux43dux43eux445ux440ux43eux43cux430ux442ux438ux447ux435ux441ux43aux43eux433ux43e-ux438-ux438ux43dux442ux435ux433ux440ux430ux43bux44cux43dux43eux433ux43e-ux43fux43eux442ux43eux43aux43eux432-ux440ux430ux434ux438ux430ux446ux438ux438}{%
\section{Закон Ослабления Монохроматического и Интегрального Потоков
Радиации}\label{ux437ux430ux43aux43eux43d-ux43eux441ux43bux430ux431ux43bux435ux43dux438ux44f-ux43cux43eux43dux43eux445ux440ux43eux43cux430ux442ux438ux447ux435ux441ux43aux43eux433ux43e-ux438-ux438ux43dux442ux435ux433ux440ux430ux43bux44cux43dux43eux433ux43e-ux43fux43eux442ux43eux43aux43eux432-ux440ux430ux434ux438ux430ux446ux438ux438}}

Ослабление радиации при прохождении через атмосферу -- это многогранный
процесс, включающий поглощение, рассеяние и излучение самой средой. В
наших источниках этот закон не представлен одной универсальной формулой,
как, например, закон Бугера-Ламберта-Бера в его простейшей форме для
однородной среды, но детализируется через уравнения переноса и описание
физических механизмов взаимодействия излучения с атмосферными газами и
частицами.

\hypertarget{ux43eux441ux43bux430ux431ux43bux435ux43dux438ux435-ux43cux43eux43dux43eux445ux440ux43eux43cux430ux442ux438ux447ux435ux441ux43aux43eux433ux43e-ux43fux43eux442ux43eux43aux430-ux440ux430ux434ux438ux430ux446ux438ux438}{%
\subsection{1. Ослабление Монохроматического Потока
Радиации}\label{ux43eux441ux43bux430ux431ux43bux435ux43dux438ux435-ux43cux43eux43dux43eux445ux440ux43eux43cux430ux442ux438ux447ux435ux441ux43aux43eux433ux43e-ux43fux43eux442ux43eux43aux430-ux440ux430ux434ux438ux430ux446ux438ux438}}

При движении лучистой энергии через атмосферный слой происходит её
частичное поглощение. В общем виде, взаимодействие монохроматической
радиации (излучения с определенной длиной волны \texttt{λ}) со средой
описывается через изменение её интенсивности.

\hypertarget{ux443ux440ux430ux432ux43dux435ux43dux438ux44f-ux43fux435ux440ux435ux43dux43eux441ux430-ux438ux43dux442ux435ux43dux441ux438ux432ux43dux43eux441ux442ux438}{%
\subsubsection{1.1. Уравнения Переноса
Интенсивности}\label{ux443ux440ux430ux432ux43dux435ux43dux438ux44f-ux43fux435ux440ux435ux43dux43eux441ux430-ux438ux43dux442ux435ux43dux441ux438ux432ux43dux43eux441ux442ux438}}

Для нисходящей (\texttt{j\_λ↓}) и восходящей (\texttt{j\_λ↑})
интенсивности радиации в атмосферном столбе используются следующие
дифференциальные уравнения:

\begin{itemize}
\item
  \textbf{Для нисходящей интенсивности:}

\begin{verbatim}
d j_λ↓ / dz = (ρ / cos(ϑ)) * (l_λ * J_λ - a_λ * j_λ↓)
\end{verbatim}
\item
  \textbf{Для восходящей интенсивности:}

\begin{verbatim}
d j_λ↑ / dz = (ρ / cos(ϑ)) * (a_λ * J_λ - l_λ * j_λ↑)
\end{verbatim}
\end{itemize}

Где:

\begin{itemize}
\tightlist
\item
  \texttt{j\_λ} -- монохроматическая интенсивность радиации.
\item
  \texttt{z} -- вертикальная координата, представляющая высоту.
\item
  \texttt{ρ} -- плотность воздуха.
\item
  \texttt{ϑ} -- угол падения лучей к нормали поверхности.
\item
  \texttt{a\_λ} -- коэффициент поглощения для данной длины волны.
\item
  \texttt{l\_λ} -- излучательная способность среды (коэффициент
  излучения).
\item
  \texttt{J\_λ} -- интенсивность излучения абсолютно черного тела при
  температуре \texttt{T} среды.
\end{itemize}

Физический смысл этих уравнений заключается в том, что изменение
интенсивности излучения на данном уровне \texttt{dz} определяется двумя
основными процессами:

\begin{enumerate}
\def\labelenumi{\arabic{enumi}.}
\tightlist
\item
  \textbf{Поглощение:} Уменьшение интенсивности за счет поглощения
  энергией средой (член \texttt{a\_λ\ *\ j\_λ}).
\item
  \textbf{Эмиссия (излучение):} Увеличение интенсивности за счет
  собственного излучения слоя воздуха (член \texttt{l\_λ\ *\ J\_λ}).
\end{enumerate}

Согласно закону Кирхгофа, в условиях локального термодинамического
равновесия, излучательная способность среды (\texttt{l\_λ}) равна её
поглощательной способности (\texttt{a\_λ}). Таким образом, эти уравнения
описывают баланс между поглощением и эмиссией в каждом слое атмосферы.

\hypertarget{ux43eux43fux442ux438ux447ux435ux441ux43aux438ux439-ux43fux443ux442ux44c-ux438-ux444ux443ux43dux43aux446ux438ux44f-ux43fux440ux43eux43fux443ux441ux43aux430ux43dux438ux44f}{%
\subsubsection{1.2. Оптический Путь и Функция
Пропускания}\label{ux43eux43fux442ux438ux447ux435ux441ux43aux438ux439-ux43fux443ux442ux44c-ux438-ux444ux443ux43dux43aux446ux438ux44f-ux43fux440ux43eux43fux443ux441ux43aux430ux43dux438ux44f}}

Спектр поглощения атмосферных газов, особенно водяного пара, имеет
сложную, линейчатую структуру и не может быть легко представлен
аналитически. В связи с этим, вместо прямого использования коэффициента
поглощения \texttt{a\_λ}, в расчетах часто удобнее использовать так
называемую \textbf{функцию пропускания радиации} (\texttt{D(ξ)}).
Функция пропускания зависит от \textbf{оптического пути} (\texttt{ξ}),
пройденного радиацией в атмосфере.

Формулы для расчета потоков длинноволновой радиации могут быть выражены
через эту функцию пропускания.

\hypertarget{ux43eux441ux43bux430ux431ux43bux435ux43dux438ux435-ux438ux43dux442ux435ux433ux440ux430ux43bux44cux43dux43eux433ux43e-ux43fux43eux442ux43eux43aux430-ux440ux430ux434ux438ux430ux446ux438ux438}{%
\subsection{2. Ослабление Интегрального Потока
Радиации}\label{ux43eux441ux43bux430ux431ux43bux435ux43dux438ux435-ux438ux43dux442ux435ux433ux440ux430ux43bux44cux43dux43eux433ux43e-ux43fux43eux442ux43eux43aux430-ux440ux430ux434ux438ux430ux446ux438ux438}}

Интегральный поток радиации получается путем интегрирования
монохроматического потока по всем длинам волн. Ослабление интегрального
потока является результатом совокупного действия всех процессов
взаимодействия излучения со всеми компонентами атмосферы в широком
спектральном диапазоне.

\hypertarget{ux441ux435ux43bux435ux43aux442ux438ux432ux43dux43eux435-ux43fux43eux433ux43bux43eux449ux435ux43dux438ux435-ux430ux442ux43cux43eux441ux444ux435ux440ux43dux44bux43cux438-ux433ux430ux437ux430ux43cux438}{%
\subsubsection{2.1. Селективное Поглощение Атмосферными
Газами}\label{ux441ux435ux43bux435ux43aux442ux438ux432ux43dux43eux435-ux43fux43eux433ux43bux43eux449ux435ux43dux438ux435-ux430ux442ux43cux43eux441ux444ux435ux440ux43dux44bux43cux438-ux433ux430ux437ux430ux43cux438}}

Газы, в отличие от твердых тел и жидкостей, поглощают радиацию
\textbf{селективно}, т.е. только на определенных длинах волн,
характерных для каждого газа.

\begin{itemize}
\tightlist
\item
  \textbf{Поглощение солнечной (коротковолновой) радиации:}

  \begin{itemize}
  \tightlist
  \item
    В верхних слоях атмосферы основной вклад в поглощение
    ультрафиолетового излучения вносят молекулярный и атомарный
    кислород, а также озон и азот, поглощая его практически полностью.
  \item
    Инфракрасная часть солнечного спектра поглощается в основном водяным
    паром, углекислым газом и озоном. В видимой области спектра
    атмосферные газы практически не поглощают энергию.
  \end{itemize}
\item
  \textbf{Поглощение земной (длинноволновой) радиации:}

  \begin{itemize}
  \tightlist
  \item
    Основными поглотителями являются водяной пар, углекислый газ и озон,
    причем водяной пар является наиболее важным из них.
  \item
    Наблюдается ``окно прозрачности'' атмосферы в диапазоне длин волн от
    8.5 до 12 мкм, где излучение практически не поглощается. Это
    критично для выхода земного излучения в космос.
  \end{itemize}
\end{itemize}

\hypertarget{ux440ux43eux43bux44c-ux43eux431ux43bux430ux43aux43eux432-ux438-ux430ux44dux440ux43eux437ux43eux43bux435ux439}{%
\subsubsection{2.2. Роль Облаков и
Аэрозолей}\label{ux440ux43eux43bux44c-ux43eux431ux43bux430ux43aux43eux432-ux438-ux430ux44dux440ux43eux437ux43eux43bux435ux439}}

Помимо газового состава, значительное влияние на ослабление как
солнечной, так и земной радиации оказывают облака и атмосферные
аэрозоли:

\begin{itemize}
\tightlist
\item
  \textbf{Облачность:} Существенно уменьшает приток солнечной радиации к
  земной поверхности днем, а также сокращает эффективное излучение
  ночью. Это происходит за счет высокого альбедо облаков (70-80\%),
  отражающего солнечную радиацию, и их способности поглощать и
  переизлучать длинноволновую радиацию.
\end{itemize}

\hypertarget{ux440ux430ux434ux438ux430ux446ux438ux43eux43dux43dux44bux439-ux431ux430ux43bux430ux43dux441-ux438-ux43fux430ux440ux43dux438ux43aux43eux432ux44bux439-ux44dux444ux444ux435ux43aux442}{%
\subsubsection{2.3. Радиационный Баланс и Парниковый
Эффект}\label{ux440ux430ux434ux438ux430ux446ux438ux43eux43dux43dux44bux439-ux431ux430ux43bux430ux43dux441-ux438-ux43fux430ux440ux43dux438ux43aux43eux432ux44bux439-ux44dux444ux444ux435ux43aux442}}

В результате поглощения радиации атмосферой и её последующего
собственного излучения формируется радиационный баланс и возникает так
называемый парниковый эффект. Атмосфера, имеющая температуру, близкую к
температуре подстилающей поверхности, излучает длинноволновую радиацию
во все стороны. Часть этого излучения возвращается к поверхности Земли
(противоизлучение атмосферы), а другая часть уходит в космос (уходящее
излучение). Этот механизм объясняет, почему температура у поверхности
Земли значительно выше, чем была бы без атмосферы.

Таким образом, закон ослабления радиации в атмосфере описывается
комплексными физическими процессами, где каждый компонент (газы, облака,
аэрозоли) играет свою специфическую роль в поглощении, пропускании и
отражении энергии на различных длинах волн. Понимание этих механизмов
критически важно для численного моделирования атмосферных процессов и
прогнозирования погоды и климата.

ewpage

Коллега, обратимся к ключевым понятиям в актинометрии, которые описывают
взаимодействие электромагнитного излучения с атмосферной средой.

\hypertarget{ux445ux430ux440ux430ux43aux442ux435ux440ux438ux441ux442ux438ux43aux438-ux432ux437ux430ux438ux43cux43eux434ux435ux439ux441ux442ux432ux438ux44f-ux440ux430ux434ux438ux430ux446ux438ux438-ux441-ux430ux442ux43cux43eux441ux444ux435ux440ux43eux439}{%
\section{Характеристики Взаимодействия Радиации с
Атмосферой}\label{ux445ux430ux440ux430ux43aux442ux435ux440ux438ux441ux442ux438ux43aux438-ux432ux437ux430ux438ux43cux43eux434ux435ux439ux441ux442ux432ux438ux44f-ux440ux430ux434ux438ux430ux446ux438ux438-ux441-ux430ux442ux43cux43eux441ux444ux435ux440ux43eux439}}

\hypertarget{ux444ux443ux43dux43aux446ux438ux438-ux43fux440ux43eux43fux443ux441ux43aux430ux43dux438ux44f-ux438-ux43fux43eux433ux43bux43eux449ux435ux43dux438ux44f}{%
\subsection{1. Функции Пропускания и
Поглощения}\label{ux444ux443ux43dux43aux446ux438ux438-ux43fux440ux43eux43fux443ux441ux43aux430ux43dux438ux44f-ux438-ux43fux43eux433ux43bux43eux449ux435ux43dux438ux44f}}

Лучистая энергия, распространяясь в какой-либо среде, частично
поглощается, превращаясь в тепловую энергию, и частично пропускается.
Лучистый теплообмен в атмосфере формируется в результате поглощения и
излучения электромагнитных волн слоями воздуха.

\hypertarget{ux444ux443ux43dux43aux446ux438ux44f-ux43fux440ux43eux43fux443ux441ux43aux430ux43dux438ux44f}{%
\subsubsection{1.1. Функция
Пропускания}\label{ux444ux443ux43dux43aux446ux438ux44f-ux43fux440ux43eux43fux443ux441ux43aux430ux43dux438ux44f}}

Функция пропускания \texttt{D(ξ)} представляет собой отношение потока
черной радиации \texttt{B(ξ)}, прошедшей слой оптической толщины
\texttt{ξ}, к падающему на этот слой потоку радиации \texttt{E\_T}. В
отличие от коэффициента поглощения \texttt{λa}, функция пропускания
\texttt{D(ξ)} имеет более простую структуру и хорошо аппроксимируется
аналитически.

Формально, для потоков радиации, проходящих через атмосферу, интеграл
может быть выражен через функцию \texttt{D(ξ)}. Если температура
\texttt{T} не зависит от оптического пути \texttt{ξ}, то функция
пропускания \texttt{D(ξ)} связана с коэффициентом поглощения \texttt{λa}
следующим образом:
\texttt{D(ξ)\ =\ (∫∫∫(от\ 0\ до\ 2π,\ от\ 0\ до\ π/2,\ от\ 0\ до\ ∞)\ dλ\ dϑ\ dϕ\ j\_λ(T)\ cos(ϑ)\ sin(ϑ)\ exp(-a\_λ(ξ)/cos(ϑ)))\ /\ (π\ ∫(от\ 0\ до\ ∞)\ j\_λ(T)\ dλ)}
или в упрощенном виде: \texttt{D(ξ)\ =\ B(ξ)\ /\ E\_T}.

Для водяного пара (с учетом поглощения также углекислотой) Шехтер на
основе анализа данных получила следующее выражение для \texttt{D(ξ)}:
\texttt{D(ξ)\ =\ 0.471\ *\ exp(-8.94\ *\ ξ)\ +\ 0.529\ *\ exp(-0.696\ *\ ξ)}.
Здесь \texttt{ξ} --- оптический путь длинноволновой радиации в
атмосфере, который определяется как
\texttt{∫(от\ 0\ до\ z)\ ρ\_n(z)\ dP(z)\ /\ 1000}, где \texttt{ρ\_n} --
плотность водяного пара, \texttt{P} -- давление в гектопаскалях.

Свойства функции пропускания \texttt{D(ξ)}:

\begin{itemize}
\tightlist
\item
  \texttt{0\ ≤\ D(ξ)\ ≤\ 1}.
\item
  \texttt{D(0)\ =\ 1} (полное пропускание при нулевой оптической
  толщине).
\item
  \texttt{D(∞)\ =\ 0} (полное поглощение/отсутствие пропускания при
  бесконечной оптической толщине).
\end{itemize}

В формулах для расчета потоков длинноволновой радиации в атмосфере
коэффициент поглощения заменяется функцией пропускания. Например, поток
длинноволновой радиации, направленный вниз на уровне \texttt{u=m},
определяется излучением и поглощением вышележащих слоев атмосферы.
Поток, направленный вверх, складывается из излучения земной поверхности,
достигшего уровня \texttt{m}, потока радиации, обусловленного излучением
и поглощением в слое атмосферы между Землей и уровнем \texttt{m}, и доли
атмосферного излучения, отраженного земной поверхностью и достигшего
уровня \texttt{m}.

\hypertarget{ux43fux43eux433ux43bux43eux449ux435ux43dux438ux435-ux440ux430ux434ux438ux430ux446ux438ux438-1}{%
\subsubsection{1.2. Поглощение
Радиации}\label{ux43fux43eux433ux43bux43eux449ux435ux43dux438ux435-ux440ux430ux434ux438ux430ux446ux438ux438-1}}

Поглощение радиации -- это процесс, при котором атомы и молекулы газов,
составляющих воздух, переходят в другое энергетическое состояние под
воздействием солнечной радиации. Лучистая энергия, проходя через
атмосферу, частично поглощается водяным паром, углекислым газом и
некоторыми другими составными частями воздуха, превращаясь в тепловую
энергию. Каждый слой воздуха излучает лучистую энергию в окружающее
пространство, теряя при этом часть своей внутренней тепловой энергии.

Основные газы-поглотители и диапазоны:

\begin{itemize}
\tightlist
\item
  \textbf{Ультрафиолетовые лучи}: молекулярный и атомарный кислород, а
  также азот и озон поглощают ультрафиолетовые лучи практически
  полностью.
\item
  \textbf{Инфракрасная радиация}: водяной пар, углекислый газ и озон
  являются основными поглотителями земной радиации. Метан также
  прогнозируется к увеличению своей роли в поглощении земной радиации.
  Азот и кислород, главные атмосферные газы, поглощают инфракрасную
  радиацию Солнца в верхних слоях атмосферы.
\item
  \textbf{Видимая область спектра}: атмосферные газы не поглощают
  энергию.
\end{itemize}

Поглощение газами носит селективный характер, то есть они поглощают
только свойственные им длины волн. Спектр поглощения водяного пара имеет
линейчатую структуру, и его зависимость от длины волны не может быть
представлена аналитически. Сложное расположение полос поглощения
атмосферных газов существенно затрудняет расчеты.

\hypertarget{ux441ux43fux435ux43aux442ux440ux430ux43bux44cux43dux44bux435-ux438-ux438ux43dux442ux435ux433ux440ux430ux43bux44cux43dux44bux435-ux445ux430ux440ux430ux43aux442ux435ux440ux438ux441ux442ux438ux43aux438-ux43fux440ux43eux437ux440ux430ux447ux43dux43eux441ux442ux438-ux430ux442ux43cux43eux441ux444ux435ux440ux44b}{%
\subsection{2. Спектральные и Интегральные Характеристики Прозрачности
Атмосферы}\label{ux441ux43fux435ux43aux442ux440ux430ux43bux44cux43dux44bux435-ux438-ux438ux43dux442ux435ux433ux440ux430ux43bux44cux43dux44bux435-ux445ux430ux440ux430ux43aux442ux435ux440ux438ux441ux442ux438ux43aux438-ux43fux440ux43eux437ux440ux430ux447ux43dux43eux441ux442ux438-ux430ux442ux43cux43eux441ux444ux435ux440ux44b}}

\hypertarget{ux441ux43fux435ux43aux442ux440ux430ux43bux44cux43dux44bux435-ux445ux430ux440ux430ux43aux442ux435ux440ux438ux441ux442ux438ux43aux438}{%
\subsubsection{2.1. Спектральные
Характеристики}\label{ux441ux43fux435ux43aux442ux440ux430ux43bux44cux43dux44bux435-ux445ux430ux440ux430ux43aux442ux435ux440ux438ux441ux442ux438ux43aux438}}

Прозрачность атмосферы характеризуется её способностью пропускать
электромагнитное излучение в зависимости от длины волны.

\begin{itemize}
\tightlist
\item
  \textbf{Окно прозрачности}: в диапазоне длин волн от 8.5 до 12 мкм
  атмосфера практически пропускает всё излучение. Эта область называется
  «окном прозрачности» атмосферы. Газы, поглощающие радиацию в этом
  диапазоне (например, метан), при росте их количества могут привести к
  серьёзным изменениям баланса энергии планеты.
\item
  \textbf{Селективность поглощения}: Каждый газ поглощает радиацию
  селективно, т.е. только на определенных длинах волн. Это создает
  сложную линейчатую структуру спектра поглощения.
\item
  \textbf{Видимый спектр}: в видимой области спектра солнечной радиации
  атмосферные газы не поглощают энергию.
\end{itemize}

\hypertarget{ux438ux43dux442ux435ux433ux440ux430ux43bux44cux43dux44bux435-ux445ux430ux440ux430ux43aux442ux435ux440ux438ux441ux442ux438ux43aux438}{%
\subsubsection{2.2. Интегральные
Характеристики}\label{ux438ux43dux442ux435ux433ux440ux430ux43bux44cux43dux44bux435-ux445ux430ux440ux430ux43aux442ux435ux440ux438ux441ux442ux438ux43aux438}}

Интегральные характеристики прозрачности относятся к суммарным потокам
радиации, полученным путём интегрирования монохроматических потоков по
всем длинам волн в диапазоне от 0 до бесконечности. Эти потоки включают:

\begin{itemize}
\tightlist
\item
  Коротковолновое солнечное излучение (вниз).
\item
  Длинноволновое излучение атмосферы (вниз).
\item
  Длинноволновое излучение Земли и атмосферы (вверх).
\end{itemize}

Эти интегральные потоки позволяют оценить общий радиационный баланс,
характеризующий энергетические условия на земной поверхности и в
атмосфере. Например, Земля как планета в целом находится в состоянии
радиационного равновесия, но её поверхность и атмосфера по отдельности
--- нет. Атмосфера, поглощая земное инфракрасное излучение, нагревается.

\hypertarget{ux444ux430ux43aux442ux43eux440-ux43cux443ux442ux43dux43eux441ux442ux438}{%
\subsection{3. Фактор
Мутности}\label{ux444ux430ux43aux442ux43eux440-ux43cux443ux442ux43dux43eux441ux442ux438}}

В контексте метеорологии ``мутность'' обычно связана с уменьшением
прозрачности атмосферы, вызванным наличием взвешенных частиц (аэрозолей,
пыли, пепла) или других примесей. Источники указывают, что извержения
вулканов могут приводить к помутнению атмосферы за счёт выбросов пепла,
пыли и диоксида серы, которые уменьшают прозрачность нижних слоев и
создают мутную пелену в стратосфере, сильно отражающую солнечные лучи.
Антропогенные примеси, особенно в больших городах, также могут
увеличивать мутность тропосферы и стратосферы, что влияет на альбедо и
радиационный режим.

Хотя конкретного количественного ``фактора мутности'' как строго
определённой характеристики в предоставленных источниках не дано, термин
``мутность'' используется для описания общего снижения прозрачности, что
влияет на поступление солнечной радиации.

ewpage

Коллега, обратимся к фундаментальным аспектам радиационного обмена, в
частности, к спектральному составу солнечной радиации, достигающей
земной поверхности. Понимание этого распределения энергии критически
важно для анализа теплового режима и динамики атмосферы.

\hypertarget{ux441ux43fux435ux43aux442ux440ux430ux43bux44cux43dux44bux439-ux441ux43eux441ux442ux430ux432-ux441ux43eux43bux43dux435ux447ux43dux43eux439-ux440ux430ux434ux438ux430ux446ux438ux438-ux443-ux437ux435ux43cux43dux43eux439-ux43fux43eux432ux435ux440ux445ux43dux43eux441ux442ux438}{%
\section{Спектральный Состав Солнечной Радиации у Земной
Поверхности}\label{ux441ux43fux435ux43aux442ux440ux430ux43bux44cux43dux44bux439-ux441ux43eux441ux442ux430ux432-ux441ux43eux43bux43dux435ux447ux43dux43eux439-ux440ux430ux434ux438ux430ux446ux438ux438-ux443-ux437ux435ux43cux43dux43eux439-ux43fux43eux432ux435ux440ux445ux43dux43eux441ux442ux438}}

\hypertarget{ux43fux440ux438ux440ux43eux434ux430-ux441ux43eux43bux43dux435ux447ux43dux43eux439-ux440ux430ux434ux438ux430ux446ux438ux438}{%
\subsection{1. Природа Солнечной
Радиации}\label{ux43fux440ux438ux440ux43eux434ux430-ux441ux43eux43bux43dux435ux447ux43dux43eux439-ux440ux430ux434ux438ux430ux446ux438ux438}}

Солнечная радиация -- это поток энергии, переносимый электромагнитными
волнами, которые генерируются ускоренно движущимися электрическими
зарядами, например, колебательным контуром. Скорость распространения
этих волн в вакууме составляет \(3 \cdot 10^8\) м/с. Электромагнитные
волны различаются по длине, изменяясь от \(1 \cdot 10^{-9}\) м до
\(1 \cdot 10^6\) м. Метеорологический интерес сосредоточен на
ультрафиолетовом, видимом и инфракрасном диапазонах.

\hypertarget{ux441ux43fux435ux43aux442ux440-ux441ux43eux43bux43dux435ux447ux43dux43eux439-ux440ux430ux434ux438ux430ux446ux438ux438-ux43dux430-ux432ux435ux440ux445ux43dux435ux439-ux433ux440ux430ux43dux438ux446ux435-ux430ux442ux43cux43eux441ux444ux435ux440ux44b}{%
\subsection{2. Спектр Солнечной Радиации на Верхней Границе
Атмосферы}\label{ux441ux43fux435ux43aux442ux440-ux441ux43eux43bux43dux435ux447ux43dux43eux439-ux440ux430ux434ux438ux430ux446ux438ux438-ux43dux430-ux432ux435ux440ux445ux43dux435ux439-ux433ux440ux430ux43dux438ux446ux435-ux430ux442ux43cux43eux441ux444ux435ux440ux44b}}

Солнце, как источник излучения, может быть приближенно рассмотрено как
абсолютно черное тело с температурой поверхности около 5900 К. Согласно
закону Вина, максимум излучения абсолютно черного тела при такой
температуре приходится на длину волны около 0.5 мкм, что соответствует
видимой части спектра.

Полный поток солнечной радиации, поступающий на верхнюю границу
атмосферы Земли, называется \textbf{солнечной постоянной} и составляет
приблизительно 1380 Вт/м². Это значение может незначительно колебаться
(менее одного процента) из-за факторов, таких как солнечная активность
(солнечные пятна) и изменение расстояния от Земли до Солнца.

\hypertarget{ux432ux437ux430ux438ux43cux43eux434ux435ux439ux441ux442ux432ux438ux435-ux441ux43eux43bux43dux435ux447ux43dux43eux439-ux440ux430ux434ux438ux430ux446ux438ux438-ux441-ux430ux442ux43cux43eux441ux444ux435ux440ux43eux439}{%
\subsection{3. Взаимодействие Солнечной Радиации с
Атмосферой}\label{ux432ux437ux430ux438ux43cux43eux434ux435ux439ux441ux442ux432ux438ux435-ux441ux43eux43bux43dux435ux447ux43dux43eux439-ux440ux430ux434ux438ux430ux446ux438ux438-ux441-ux430ux442ux43cux43eux441ux444ux435ux440ux43eux439}}

По мере прохождения через атмосферу, солнечная радиация подвергается
ряду процессов, которые изменяют её спектральный состав и интенсивность:

\hypertarget{ux43fux43eux433ux43bux43eux449ux435ux43dux438ux435}{%
\subsubsection{3.1.
Поглощение}\label{ux43fux43eux433ux43bux43eux449ux435ux43dux438ux435}}

Поглощение радиации происходит, когда атомы и молекулы газов воздуха
переходят в другое энергетическое состояние под воздействием солнечного
излучения.

\begin{itemize}
\tightlist
\item
  \textbf{В верхних слоях атмосферы}: Молекулярный и атомарный кислород
  (\(\text{O}_2\), \(\text{O}\)) и озон (\(\text{O}_3\), образующий
  озоносферу на высотах 10-50 км) практически полностью поглощают
  ультрафиолетовые лучи. Это поглощение в озоносфере препятствует
  излишнему поступлению ультрафиолетовой энергии на земную поверхность,
  что создает благоприятные условия для жизни.
\item
  \textbf{В нижних слоях атмосферы}: В видимой области спектра солнечной
  радиации атмосферные газы практически не поглощают энергию. Однако
  водяной пар (\(\text{H}_2\text{O}\)), углекислый газ (\(\text{CO}_2\))
  и озон поглощают инфракрасную радиацию. На рисунке 49 схематически
  показаны полосы поглощения солнечной и земной радиации различными
  газами атмосферы.
\end{itemize}

\hypertarget{ux440ux430ux441ux441ux435ux44fux43dux438ux435}{%
\subsubsection{3.2.
Рассеяние}\label{ux440ux430ux441ux441ux435ux44fux43dux438ux435}}

Рассеяние -- это процесс изменения направления распространения
излучения, который воспринимается нами как собственное свечение среды.

\begin{itemize}
\tightlist
\item
  \textbf{Рассеяние Рэлея}: Происходит на частицах, размер которых
  сопоставим с размерами молекул газов (например, \(10^{-9}\) мкм).
  Длины волн видимого света (около \(10^{-7}\) мкм) значительно больше
  этих частиц, что приводит к сильной зависимости потока рассеянной
  энергии от длины волны: короткие волны рассеиваются сильнее всего. Это
  объясняет голубой цвет неба в ясный день (рассеиваются фиолетовые и
  голубые лучи) и красные/оранжевые закаты (голубые волны поглощаются за
  линией горизонта, до наблюдателя доходят только оранжевые и красные
  лучи).
\item
  \textbf{Рассеяние Ми}: Происходит на частицах атмосферного аэрозоля
  (пыль, дым), размер которых сравним с длиной волны. В этом случае
  зависимость рассеянной радиации от длины волны исчезает, и белый свет
  после рассеяния остается белым. Этот вид рассеяния объясняет белый
  цвет облаков, не дающих дождя. Атмосфера, содержащая хаотически
  расположенные флуктуации плотности воздуха, частицы атмосферного
  аэрозоля, капельки и снежинки, является мутной средой по отношению к
  свету.
\end{itemize}

\hypertarget{ux43eux442ux440ux430ux436ux435ux43dux438ux435-ux430ux43bux44cux431ux435ux434ux43e}{%
\subsubsection{3.3. Отражение
(Альбедо)}\label{ux43eux442ux440ux430ux436ux435ux43dux438ux435-ux430ux43bux44cux431ux435ux434ux43e}}

Энергия отраженной радиации определяется величиной, называемой
\textbf{альбедо}. Альбедо различных подстилающих поверхностей
значительно варьирует (Таблица 6). Особенно велико альбедо снега, льда и
облаков. Среднее альбедо Земли как планеты составляет 0.33.

\hypertarget{ux438ux43dux442ux435ux433ux440ux430ux43bux44cux43dux44bux439-ux43fux43eux442ux43eux43a-ux438-ux441ux43fux435ux43aux442ux440ux430ux43bux44cux43dux44bux439-ux441ux43eux441ux442ux430ux432-ux441ux43eux43bux43dux435ux447ux43dux43eux439-ux440ux430ux434ux438ux430ux446ux438ux438-ux443-ux437ux435ux43cux43dux43eux439-ux43fux43eux432ux435ux440ux445ux43dux43eux441ux442ux438}{%
\subsection{4. Интегральный Поток и Спектральный Состав Солнечной
Радиации у Земной
Поверхности}\label{ux438ux43dux442ux435ux433ux440ux430ux43bux44cux43dux44bux439-ux43fux43eux442ux43eux43a-ux438-ux441ux43fux435ux43aux442ux440ux430ux43bux44cux43dux44bux439-ux441ux43eux441ux442ux430ux432-ux441ux43eux43bux43dux435ux447ux43dux43eux439-ux440ux430ux434ux438ux430ux446ux438ux438-ux443-ux437ux435ux43cux43dux43eux439-ux43fux43eux432ux435ux440ux445ux43dux43eux441ux442ux438}}

В результате взаимодействия с атмосферой, солнечная радиация ослабляется
по пути до земной поверхности. Максимально возможный поток радиации,
достигающий земной поверхности («подзонная» солнечная постоянная),
уменьшается до 1250 Вт/м², что на 130 Вт/м² меньше, чем на верхней
границе атмосферы (1380 Вт/м²).

Суммарный поток солнечной радиации у земной поверхности складывается из
двух основных компонент:

\begin{itemize}
\tightlist
\item
  \textbf{Прямая радиация}: поступает непосредственно от солнечного
  диска на площадку, перпендикулярную солнечным лучам.
\item
  \textbf{Рассеянная радиация}: поступает от небесной полусферы, за
  исключением прямой радиации. \textbf{Суммарная радиация} является
  суммой прямой и рассеянной радиации.
\end{itemize}

Примерное распределение потоков коротковолновой радиации между
различными процессами ослабления в атмосфере выглядит следующим образом:

\begin{itemize}
\tightlist
\item
  \textbf{100\%} - Инсоляция на верхней границе атмосферы.
\item
  \textbf{6\%} - Отражается облаками обратно в космос.
\item
  \textbf{17\%} - Рассеивается атмосферой в космос.
\item
  \textbf{20\%} - Поглощается атмосферой.
\item
  \textbf{10\%} - Отражается подстилающей поверхностью.
\item
  \textbf{47\%} - Поглощается подстилающей поверхностью.
\end{itemize}

Таким образом, спектральный состав солнечной радиации у земной
поверхности значительно отличается от внеатмосферного, будучи обедненным
в ультрафиолетовом и частично в инфракрасном диапазонах за счет
поглощения и рассеяния атмосферными газами и аэрозолями. Видимый свет,
хотя и подвержен рассеянию, достигает поверхности в значительной
степени, формируя наши повседневные оптические явления.

ewpage

\hypertarget{ux43eux441ux43eux431ux435ux43dux43dux43eux441ux442ux438-ux440ux430ux434ux438ux430ux446ux438ux43eux43dux43dux44bux445-ux43fux440ux43eux446ux435ux441ux441ux43eux432-ux432-ux437ux430ux433ux440ux44fux437ux43dux435ux43dux43dux43eux439-ux430ux442ux43cux43eux441ux444ux435ux440ux435}{%
\section{Особенности радиационных процессов в загрязненной
атмосфере}\label{ux43eux441ux43eux431ux435ux43dux43dux43eux441ux442ux438-ux440ux430ux434ux438ux430ux446ux438ux43eux43dux43dux44bux445-ux43fux440ux43eux446ux435ux441ux441ux43eux432-ux432-ux437ux430ux433ux440ux44fux437ux43dux435ux43dux43dux43eux439-ux430ux442ux43cux43eux441ux444ux435ux440ux435}}

\hypertarget{ux432ux432ux435ux434ux435ux43dux438ux435-ux432-ux437ux430ux433ux440ux44fux437ux43dux435ux43dux438ux435-ux430ux442ux43cux43eux441ux444ux435ux440ux44b-ux438-ux435ux433ux43e-ux438ux441ux442ux43eux447ux43dux438ux43aux438}{%
\subsection{Введение в загрязнение атмосферы и его
источники}\label{ux432ux432ux435ux434ux435ux43dux438ux435-ux432-ux437ux430ux433ux440ux44fux437ux43dux435ux43dux438ux435-ux430ux442ux43cux43eux441ux444ux435ux440ux44b-ux438-ux435ux433ux43e-ux438ux441ux442ux43eux447ux43dux438ux43aux438}}

Загрязнение атмосферы, в особенности в крупных городах, таких как те,
где население превышает 500 тысяч человек, представляет собой сложную
совокупность газообразных, жидких и твердых примесей, которые поступают
в атмосферу из различных антропогенных источников. Ключевые источники
включают транспортные средства, отопительные системы и промышленные
предприятия. Эти примеси, известные как атмосферные аэрозоли, а также
изменяющиеся компоненты, такие как водяной пар, существенно влияют на
состав воздуха.

\hypertarget{ux432ux43bux438ux44fux43dux438ux435-ux437ux430ux433ux440ux44fux437ux43dux435ux43dux438ux44f-ux43dux430-ux430ux442ux43cux43eux441ux444ux435ux440ux43dux44bux435-ux441ux432ux43eux439ux441ux442ux432ux430}{%
\subsection{Влияние загрязнения на атмосферные
свойства}\label{ux432ux43bux438ux44fux43dux438ux435-ux437ux430ux433ux440ux44fux437ux43dux435ux43dux438ux44f-ux43dux430-ux430ux442ux43cux43eux441ux444ux435ux440ux43dux44bux435-ux441ux432ux43eux439ux441ux442ux432ux430}}

Антропогенные примеси не просто пассивно присутствуют в атмосфере; они
оптически активны и оказывают обратное влияние на состояние атмосферы. В
отличие от окружающей сельской местности, в больших городах под влиянием
антропогенных примесей и измененной подстилающей поверхности изменяются
практически все метеорологические величины и атмосферные явления. К ним
относятся температура и влажность воздуха, потоки солнечной и земной
радиации, скорость ветра, а также вероятность образования и
интенсивность туманов, дымок и осадков.

Особенно важно отметить, что даже незначительные детали естественного и
искусственного рельефа, а также растительность, могут вызывать
деформацию воздушного потока и существенно отклонять локальные средние
характеристики от общих значений.

\hypertarget{ux432ux43eux437ux434ux435ux439ux441ux442ux432ux438ux435-ux43dux430-ux440ux430ux434ux438ux430ux446ux438ux43eux43dux43dux44bux439-ux440ux435ux436ux438ux43c}{%
\subsection{Воздействие на радиационный
режим}\label{ux432ux43eux437ux434ux435ux439ux441ux442ux432ux438ux435-ux43dux430-ux440ux430ux434ux438ux430ux446ux438ux43eux43dux43dux44bux439-ux440ux435ux436ux438ux43c}}

Основное влияние загрязнения атмосферы на радиационные процессы
проявляется через изменение радиационного режима. Газообразные, жидкие и
твердые примеси, как правило, сильно поглощают радиацию, особенно в
инфракрасном диапазоне длин волн (от 2 до 120 мкм). Это поглощение
приводит к изменению теплового и, как следствие, термического режима
города.

\hypertarget{ux440ux43eux43bux44c-ux432ux43eux434ux44fux43dux43eux433ux43e-ux43fux430ux440ux430}{%
\subsubsection{Роль водяного
пара}\label{ux440ux43eux43bux44c-ux432ux43eux434ux44fux43dux43eux433ux43e-ux43fux430ux440ux430}}

Среди антропогенных примесей особую роль играет дополнительное
количество водяного пара, образующегося при сжигании различных видов
топлива и выбрасываемого в атмосферу. Концентрация (абсолютная
влажность) антропогенного водяного пара может быть в 100-1000 раз больше
содержания оксида углерода (СО) и в 10-100 раз больше концентрации
углекислого газа (CO2). Независимо от поглощательной способности других
примесей, количество поглощенной ими радиации в первую очередь
пропорционально их концентрации. Таким образом, антропогенный водяной
пар вносит значительный вклад в изменение радиационного режима.

\hypertarget{ux43fux430ux440ux43dux438ux43aux43eux432ux44bux439-ux44dux444ux444ux435ux43aux442}{%
\subsubsection{Парниковый
эффект}\label{ux43fux430ux440ux43dux438ux43aux43eux432ux44bux439-ux44dux444ux444ux435ux43aux442}}

Поглощение и излучение радиации в атмосфере оказывает воздействие на
температуру земной поверхности и воздуха, что приводит к так называемому
``парниковому эффекту''. Основными поглотителями земной (длинноволновой)
радиации являются водяной пар, углекислый газ и озон. В последние годы,
в связи с ростом техногенных выбросов, прогнозируется увеличение роли
метана в поглощении земной радиации.

Атмосфера имеет ``окно прозрачности'' в диапазоне длин волн от 8.5 до 12
мкм, где она практически пропускает все излучение. Однако любой газ,
который поглощает радиацию в этом диапазоне (например, метан), при росте
его количества может привести к серьезным изменениям энергетического
баланса планеты.

Увеличение концентрации парниковых газов, таких как CO2 (которая
возросла примерно на 30\% за 150 лет за счет сжигания топлива), ведет к
росту температуры воздуха у поверхности Земли при падении температуры в
верхней тропосфере. Противоизлучение атмосферы, направленное обратно к
земной поверхности, уменьшает потерю тепла и усиливает парниковый
эффект.

Земля как планета в целом находится в состоянии радиационного
равновесия, где приходящий поток солнечной радиации уравновешивается
отраженной коротковолновой и уходящей длинноволновой радиацией. Однако
ни поверхность Земли, ни атмосфера по отдельности не находятся в
состоянии радиационного равновесия; эффективное излучение лишь частично
компенсирует приток от поглощенной радиации. Этот запас мощности
позволяет атмосфере совершать работу по переносу воздушных масс.

\hypertarget{ux441ux43eux43fux443ux442ux441ux442ux432ux443ux44eux449ux438ux435-ux44fux432ux43bux435ux43dux438ux44f}{%
\subsection{Сопутствующие
явления}\label{ux441ux43eux43fux443ux442ux441ux442ux432ux443ux44eux449ux438ux435-ux44fux432ux43bux435ux43dux438ux44f}}

Изменение радиационного и термического режимов в загрязненной атмосфере
города естественным образом ведет к изменению условий образования
туманов и дымок, а также облаков и осадков. Например, рассеяние или
уменьшение интенсивности тумана в городе по сравнению с окрестностями
часто объясняется уменьшением относительной влажности в городе под
влиянием повышения температуры. Конденсационные следы, образующиеся за
самолетами, могут разрастаться, приводя к образованию плотных облаков
верхнего яруса, а на аэродромах и в промышленных центрах могут возникать
туманы и низкая облачность.

Таким образом, антропогенное загрязнение атмосферы через воздействие на
радиационные процессы играет ключевую роль в формировании измененного
метеорологического режима в городских и промышленных зонах.

ewpage

Уважаемый коллега, давайте систематизируем наше понимание компонентов
солнечной радиации и ключевых факторов, влияющих на них в атмосфере.

\hypertarget{ux43fux43eux442ux43eux43aux438-ux441ux43eux43bux43dux435ux447ux43dux43eux439-ux440ux430ux434ux438ux430ux446ux438ux438-ux438-ux432ux43bux438ux44fux44eux449ux438ux435-ux444ux430ux43aux442ux43eux440ux44b}{%
\section{Потоки солнечной радиации и влияющие
факторы}\label{ux43fux43eux442ux43eux43aux438-ux441ux43eux43bux43dux435ux447ux43dux43eux439-ux440ux430ux434ux438ux430ux446ux438ux438-ux438-ux432ux43bux438ux44fux44eux449ux438ux435-ux444ux430ux43aux442ux43eux440ux44b}}

Солнечная радиация, поступающая в атмосферу и достигающая земной
поверхности, подразделяется на три основных потока.

\hypertarget{ux43fux440ux44fux43cux430ux44f-ux441ux43eux43bux43dux435ux447ux43dux430ux44f-ux440ux430ux434ux438ux430ux446ux438ux44f}{%
\subsection{Прямая солнечная
радиация}\label{ux43fux440ux44fux43cux430ux44f-ux441ux43eux43bux43dux435ux447ux43dux430ux44f-ux440ux430ux434ux438ux430ux446ux438ux44f}}

\textbf{Прямая радиация} -- это часть солнечного излучения, которая
достигает подстилающей поверхности непосредственно от солнечного диска,
приходя на площадку, перпендикулярную солнечным лучам. Иными словами,
это непоглощенный и нерассеянный поток.

\hypertarget{ux440ux430ux441ux441ux435ux44fux43dux43dux430ux44f-ux441ux43eux43bux43dux435ux447ux43dux430ux44f-ux440ux430ux434ux438ux430ux446ux438ux44f}{%
\subsection{Рассеянная солнечная
радиация}\label{ux440ux430ux441ux441ux435ux44fux43dux43dux430ux44f-ux441ux43eux43bux43dux435ux447ux43dux430ux44f-ux440ux430ux434ux438ux430ux446ux438ux44f}}

\textbf{Рассеянная радиация} -- это излучение, поступающее на земную
поверхность со всей небесной полусферы, за исключением прямой радиации
от солнечного диска. Этот поток возникает в результате взаимодействия
солнечного излучения с различными компонентами атмосферы. Процесс
\textbf{рассеяния} представляет собой изменение направления
распространения излучения, которое воспринимается наблюдателем как
несобственное свечение среды. В зависимости от соотношения длины волны
излучения и размера рассеивающих частиц выделяют два основных типа
рассеяния:

\begin{itemize}
\tightlist
\item
  \textbf{Рэлеевское рассеяние}: Происходит на молекулярных комплексах
  атмосферных газов (размером порядка 10⁻⁹ мкм), когда длина волны
  видимого света значительно больше этих частиц. Наиболее сильно
  рассеиваются коротковолновые части спектра (фиолетовые и голубые
  лучи), что объясняет голубой цвет безоблачного неба днем.
\item
  \textbf{Ми-рассеяние}: Возникает при взаимодействии света с частицами
  атмосферного аэрозоля, размер которых сопоставим с длиной волны
  (например, капли облаков, пыль). В этом случае поток рассеянной
  радиации слабо зависит от длины волны, поэтому белый свет остается
  белым (например, белые облака).
\end{itemize}

\hypertarget{ux441ux443ux43cux43cux430ux440ux43dux430ux44f-ux441ux43eux43bux43dux435ux447ux43dux430ux44f-ux440ux430ux434ux438ux430ux446ux438ux44f}{%
\subsection{Суммарная солнечная
радиация}\label{ux441ux443ux43cux43cux430ux440ux43dux430ux44f-ux441ux43eux43bux43dux435ux447ux43dux430ux44f-ux440ux430ux434ux438ux430ux446ux438ux44f}}

\textbf{Суммарная радиация} представляет собой общую сумму прямой и
рассеянной солнечной радиации, достигающей земной поверхности. Это
основной источник коротковолновой энергии, формирующий тепловой баланс
поверхности.

\hypertarget{ux444ux430ux43aux442ux43eux440ux44b-ux432ux43bux438ux44fux44eux449ux438ux435-ux43dux430-ux43fux43eux442ux43eux43aux438-ux441ux43eux43bux43dux435ux447ux43dux43eux439-ux440ux430ux434ux438ux430ux446ux438ux438}{%
\subsection{Факторы, влияющие на потоки солнечной
радиации}\label{ux444ux430ux43aux442ux43eux440ux44b-ux432ux43bux438ux44fux44eux449ux438ux435-ux43dux430-ux43fux43eux442ux43eux43aux438-ux441ux43eux43bux43dux435ux447ux43dux43eux439-ux440ux430ux434ux438ux430ux446ux438ux438}}

На величину и распределение прямой, рассеянной и, как следствие,
суммарной солнечной радиации влияют многочисленные факторы, связанные
как с астрономическими условиями, так и с состоянием атмосферы и
подстилающей поверхности.

\hypertarget{ux430ux441ux442ux440ux43eux43dux43eux43cux438ux447ux435ux441ux43aux438ux435-ux444ux430ux43aux442ux43eux440ux44b-ux438-ux433ux435ux43eux43cux435ux442ux440ux438ux44f-ux437ux435ux43cux43bux438}{%
\subsubsection{1. Астрономические факторы и геометрия
Земли}\label{ux430ux441ux442ux440ux43eux43dux43eux43cux438ux447ux435ux441ux43aux438ux435-ux444ux430ux43aux442ux43eux440ux44b-ux438-ux433ux435ux43eux43cux435ux442ux440ux438ux44f-ux437ux435ux43cux43bux438}}

\begin{itemize}
\tightlist
\item
  \textbf{Высота Солнца над горизонтом (Инсоляция)}: Распределение
  солнечной радиации по земной поверхности, называемое инсоляцией,
  определяется сферической формой Земли, ее суточным вращением вокруг
  оси, наклоном этой оси к плоскости эклиптики и движением по орбите
  вокруг Солнца.

  \begin{itemize}
  \tightlist
  \item
    На экваторе солнечные лучи падают перпендикулярно, а на высоких
    широтах -- под острым углом.
  \item
    Суточное вращение Земли определяет смену дня и ночи, а также
    полуденную высоту Солнца. Годовое движение Земли по орбите и наклон
    оси определяют сезонные изменения высоты Солнца и продолжительности
    дня, что приводит к широтному распределению инсоляции, с максимумами
    не всегда на экваторе, а на тропиках соответствующего полушария в
    летний период и значительным притоком радиации в полярных областях
    во время полярного дня.
  \end{itemize}
\item
  \textbf{Солнечная постоянная и параметры орбиты Земли}: Интегральный
  поток солнечной радиации на верхней границе атмосферы, известный как
  солнечная постоянная (\texttt{I₀\ ≈\ 1380\ Вт/м²}), может колебаться.
  Эти колебания (обычно менее одного процента) обусловлены солнечной
  активностью и изменением параметров орбиты Земли. Более крупные
  изменения параметров орбиты Земли (эксцентриситет, наклон оси,
  прецессия) являются внешними факторами изменения климата и могут
  приводить к значительным колебаниям приходящей солнечной энергии (до
  20\% при максимальном эксцентриситете орбиты за 100 000 лет).
\end{itemize}

\hypertarget{ux441ux43eux441ux442ux430ux432-ux438-ux441ux43eux441ux442ux43eux44fux43dux438ux435-ux430ux442ux43cux43eux441ux444ux435ux440ux44b}{%
\subsubsection{2. Состав и состояние
атмосферы}\label{ux441ux43eux441ux442ux430ux432-ux438-ux441ux43eux441ux442ux43eux44fux43dux438ux435-ux430ux442ux43cux43eux441ux444ux435ux440ux44b}}

Проходя через атмосферу, солнечная радиация ослабляется в результате
поглощения, рассеяния и отражения. Максимально возможный поток радиации
на земной поверхности («подзонная» солнечная постоянная) уменьшается до
1250 Вт/м².

\begin{itemize}
\tightlist
\item
  \textbf{Поглощающие газы}:

  \begin{itemize}
  \tightlist
  \item
    Водяной пар (H₂O), углекислый газ (CO₂), озон (O₃) и некоторые
    другие газы (например, метан) являются основными поглотителями
    радиации. Они преобразуют лучистую энергию в тепловую, нагревая
    атмосферу. Кислород (O₂), атомарный кислород (O) и азот (N) также
    поглощают ультрафиолетовые лучи, преимущественно в верхних слоях
    атмосферы. Видимая область спектра солнечной радиации атмосферными
    газами практически не поглощается.
  \item
    Увеличение концентрации CO₂ связывают с потеплением из-за поглощения
    земной инфракрасной радиации.
  \end{itemize}
\item
  \textbf{Аэрозоли и мутность атмосферы}:

  \begin{itemize}
  \tightlist
  \item
    \textbf{Атмосферные аэрозоли} --- это взвешенные в воздухе твердые
    частицы или капельки (дым, сажа, пепел, морская соль, пыльца, споры,
    микроорганизмы) размером от 0.001 до 5 мкм. Они вызывают рассеяние
    солнечной радиации (Ми-рассеяние).
  \item
    \textbf{Антропогенные примеси} (особенно в больших городах)
    оптически активны, сильно поглощая радиацию в инфракрасном диапазоне
    (2-120 мкм) и оказывая обратное влияние на потоки солнечной и земной
    радиации.
  \item
    \textbf{Вулканические извержения}: Мощные извержения могут
    выбрасывать в стратосферу пепел, пыль и диоксид серы, что приводит к
    помутнению атмосферы и значительному ослаблению потока солнечной
    радиации на 1-3 года.
  \end{itemize}
\item
  \textbf{Облачность}:

  \begin{itemize}
  \tightlist
  \item
    Облака оказывают существенное влияние на радиационный режим,
    уменьшая приток солнечной радиации к земной поверхности.
  \item
    Облака имеют очень высокое \textbf{альбедо}, то есть сильно отражают
    солнечную радиацию.
  \item
    Наличие облачности усложняет расчеты радиационных потоков.
  \end{itemize}
\item
  \textbf{Рефракция}: Постепенное увеличение показателя преломления в
  более плотных слоях воздуха приводит к отклонению лучей света в
  сторону более холодного и плотного воздуха. Хотя рефракция не является
  поглощением или рассеянием в прямом смысле, она изменяет путь
  распространения радиации.
\end{itemize}

\hypertarget{ux441ux432ux43eux439ux441ux442ux432ux430-ux43fux43eux434ux441ux442ux438ux43bux430ux44eux449ux435ux439-ux43fux43eux432ux435ux440ux445ux43dux43eux441ux442ux438}{%
\subsubsection{3. Свойства подстилающей
поверхности}\label{ux441ux432ux43eux439ux441ux442ux432ux430-ux43fux43eux434ux441ux442ux438ux43bux430ux44eux449ux435ux439-ux43fux43eux432ux435ux440ux445ux43dux43eux441ux442ux438}}

\begin{itemize}
\tightlist
\item
  \textbf{Альбедо подстилающей поверхности}: Это отношение отраженной
  солнечной радиации к падающей. Различные типы поверхности имеют разное
  альбедо, и оно значительно влияет на поглощение энергии. Например,
  альбедо снега и льда очень велико. Характер подстилающей поверхности
  (ее вид, наличие растительности) и ее альбедо сильно влияют на
  температуру поверхности и, соответственно, на поглощенную радиацию.
  Изменения альбедо поверхности могут быть внутренним фактором
  климатических изменений.
\end{itemize}

Таким образом, оценка и прогнозирование прямой, рассеянной и суммарной
солнечной радиации требуют учета сложного комплекса взаимодействующих
факторов, от макромасштабных астрономических параметров до
микрофизических процессов в атмосфере.

ewpage

Коллега, давайте детально рассмотрим крайне важные для энергетического
баланса системы Земля-атмосфера процессы отражения и поглощения
солнечной радиации земной поверхностью. Эти механизмы определяют
температурный режим и циркуляцию, поэтому их понимание является
фундаментальным для метеорологии.

\hypertarget{ux43eux442ux440ux430ux436ux435ux43dux438ux435-ux438-ux43fux43eux433ux43bux43eux449ux435ux43dux438ux435-ux441ux43eux43bux43dux435ux447ux43dux43eux439-ux440ux430ux434ux438ux430ux446ux438ux438-ux437ux435ux43cux43dux43eux439-ux43fux43eux432ux435ux440ux445ux43dux43eux441ux442ux44cux44e}{%
\section{Отражение и поглощение солнечной радиации земной
поверхностью}\label{ux43eux442ux440ux430ux436ux435ux43dux438ux435-ux438-ux43fux43eux433ux43bux43eux449ux435ux43dux438ux435-ux441ux43eux43bux43dux435ux447ux43dux43eux439-ux440ux430ux434ux438ux430ux446ux438ux438-ux437ux435ux43cux43dux43eux439-ux43fux43eux432ux435ux440ux445ux43dux43eux441ux442ux44cux44e}}

\hypertarget{ux441ux43fux435ux43aux442ux440ux430ux43bux44cux43dux44bux435-ux445ux430ux440ux430ux43aux442ux435ux440ux438ux441ux442ux438ux43aux438-ux438ux437ux43bux443ux447ux435ux43dux438ux44f}{%
\subsection{1. Спектральные характеристики
излучения}\label{ux441ux43fux435ux43aux442ux440ux430ux43bux44cux43dux44bux435-ux445ux430ux440ux430ux43aux442ux435ux440ux438ux441ux442ux438ux43aux438-ux438ux437ux43bux443ux447ux435ux43dux438ux44f}}

Солнечная радиация, достигающая земной поверхности, относится к
коротковолновому диапазону электромагнитного спектра, с длинами волн
короче 4 мкм. В то же время собственное тепловое излучение земной
поверхности, температура которой составляет около 300 К, приходится на
инфракрасный (невидимый) диапазон, преимущественно около 10 мкм, что
относится к длинноволновой радиации. Таким образом, эти два спектральных
интервала четко разделены.

\hypertarget{ux43fux43eux433ux43bux43eux449ux435ux43dux438ux435-ux441ux43eux43bux43dux435ux447ux43dux43eux439-ux440ux430ux434ux438ux430ux446ux438ux438}{%
\subsection{2. Поглощение солнечной
радиации}\label{ux43fux43eux433ux43bux43eux449ux435ux43dux438ux435-ux441ux43eux43bux43dux435ux447ux43dux43eux439-ux440ux430ux434ux438ux430ux446ux438ux438}}

Земная поверхность является основным поглотителем солнечной радиации.
Количество поглощенной солнечной энергии напрямую зависит от конкретного
типа подстилающей поверхности. В результате поглощения радиации земная
поверхность нагревается.

В процессе прохождения через атмосферу солнечная радиация ослабляется.
Максимально возможный поток радиации у земной поверхности (``подзонная''
солнечная постоянная) уменьшается до 1250 Вт/м² из-за поглощения в
атмосфере.

Важные аспекты поглощения в системе Земля-атмосфера включают:

\begin{itemize}
\tightlist
\item
  \textbf{Атмосферное поглощение}: Газы, такие как водяной пар
  (\(\text{H}_2\text{O}\)), углекислый газ (\(\text{CO}_2\)) и озон
  (\(\text{O}_3\)), избирательно поглощают радиацию. Молекулярный и
  атомарный кислород, а также азот и озон практически полностью
  поглощают ультрафиолетовые лучи в верхних слоях атмосферы. Водяной
  пар, углекислый газ и озон поглощают инфракрасную радиацию. Однако в
  видимой области спектра солнечной радиации атмосферные газы
  практически не поглощают энергию.
\item
  \textbf{Поглощение поверхностью}: Поглощенная солнечная радиация и
  суммарная потеря тепла за счет эффективного излучения подстилающей
  поверхности определяют радиационный баланс этой поверхности.
\item
  \textbf{Свойства воды, льда и снега}: Эти среды хорошо пропускают
  солнечную радиацию, но слои толщиной 1-2 мм уже полностью поглощают
  длинноволновое излучение.
\item
  \textbf{Антропогенное влияние}: В больших городах примеси
  антропогенного происхождения (газообразные, жидкие, твердые),
  поступающие от транспорта, отопительных систем и промышленных
  предприятий, являются оптически активными и сильно поглощают радиацию,
  особенно в инфракрасном диапазоне (2-120 мкм). Это оказывает обратное
  влияние на состояние атмосферы и формирует ``остров тепла''.
\end{itemize}

\hypertarget{ux43eux442ux440ux430ux436ux435ux43dux438ux435-ux441ux43eux43bux43dux435ux447ux43dux43eux439-ux440ux430ux434ux438ux430ux446ux438ux438-ux430ux43bux44cux431ux435ux434ux43e}{%
\subsection{3. Отражение солнечной радиации
(Альбедо)}\label{ux43eux442ux440ux430ux436ux435ux43dux438ux435-ux441ux43eux43bux43dux435ux447ux43dux43eux439-ux440ux430ux434ux438ux430ux446ux438ux438-ux430ux43bux44cux431ux435ux434ux43e}}

Отражение -- это процесс взаимодействия световой волны с веществом, в
результате которого волна распространяется в обратном направлении.
Энергия отраженной радиации количественно характеризуется
\textbf{альбедо}.

Ключевые моменты, касающиеся альбедо:

\begin{itemize}
\tightlist
\item
  \textbf{Определение}: Альбедо -- это отношение интенсивности солнечной
  радиации, отраженной поверхностью, к интенсивности этой радиации,
  падающей на данную поверхность.
\item
  \textbf{Зависимость от типа поверхности}: Значения альбедо установлены
  для различных видов подстилающих поверхностей и для всего потока
  солнечной радиации.
\item
  \textbf{Высокие значения альбедо}: Особенно высокое альбедо имеют
  снег, лед и облака. Например, свежий снег отражает 80-90\% радиации,
  старый снег -- 50-70\%, мощные облака -- 70-80\%.
\item
  \textbf{Влияние облачности}: Облачный покров значительно изменяет
  радиационный баланс земной поверхности и, как следствие, термический и
  влажностный режим деятельного слоя почвы и атмосферы. Облачность с
  высоким альбедо (70-80\%) сильно ослабляет приток солнечной радиации к
  водной поверхности.
\item
  \textbf{Различия в увлажненности}: Увлажненность почвы значительно
  влияет на её альбедо.
\end{itemize}

\hypertarget{ux440ux430ux434ux438ux430ux446ux438ux43eux43dux43dux44bux439-ux431ux430ux43bux430ux43dux441-ux438-ux440ux430ux432ux43dux43eux432ux435ux441ux438ux435}{%
\subsection{4. Радиационный баланс и
равновесие}\label{ux440ux430ux434ux438ux430ux446ux438ux43eux43dux43dux44bux439-ux431ux430ux43bux430ux43dux441-ux438-ux440ux430ux432ux43dux43eux432ux435ux441ux438ux435}}

\begin{itemize}
\tightlist
\item
  \textbf{Радиационный баланс планеты}: Земля как планета в целом
  находится в состоянии радиационного равновесия. Это означает, что
  поток приходящей от Солнца радиации уравновешивается потоком
  отраженной коротковолновой и уходящей длинноволновой радиации.
\item
  \textbf{Радиационный баланс поверхности}: Поверхность Земли, однако,
  не находится в состоянии радиационного равновесия. Эффективное
  излучение лишь частично компенсирует приток от поглощенной радиации.
  Избыточная энергия расходуется на испарение и контактную теплопередачу
  в атмосферу.
\item
  \textbf{Радиационный баланс атмосферы}: Атмосфера также не находится в
  состоянии радиационного равновесия. Поток энергии в атмосфере больше,
  чем радиационные потоки. Это указывает на то, что скорость
  преобразования энергии в атмосфере превышает скорость установления
  радиационного равновесия. Этот ``запас мощности'' позволяет атмосфере
  выполнять работу по переносу воздушных масс.
\end{itemize}

Таким образом, взаимодействие солнечной радиации с земной поверхностью
через процессы поглощения и отражения, модулируемое атмосферой и её
составляющими, формирует сложный, но динамически сбалансированный
энергетический обмен, который является движущей силой всех атмосферных
процессов.

ewpage

\hypertarget{ux43aux43eux44dux444ux444ux438ux446ux438ux435ux43dux442ux44b-ux43eux442ux440ux430ux436ux435ux43dux438ux44f-ux430ux43bux44cux431ux435ux434ux43e-ux438-ux43fux43eux433ux43bux43eux449ux435ux43dux438ux44f}{%
\section{Коэффициенты отражения (альбедо) и
поглощения}\label{ux43aux43eux44dux444ux444ux438ux446ux438ux435ux43dux442ux44b-ux43eux442ux440ux430ux436ux435ux43dux438ux44f-ux430ux43bux44cux431ux435ux434ux43e-ux438-ux43fux43eux433ux43bux43eux449ux435ux43dux438ux44f}}

\hypertarget{ux43aux43eux44dux444ux444ux438ux446ux438ux435ux43dux442-ux43eux442ux440ux430ux436ux435ux43dux438ux44f-ux430ux43bux44cux431ux435ux434ux43e}{%
\subsection{Коэффициент отражения
(альбедо)}\label{ux43aux43eux44dux444ux444ux438ux446ux438ux435ux43dux442-ux43eux442ux440ux430ux436ux435ux43dux438ux44f-ux430ux43bux44cux431ux435ux434ux43e}}

Коэффициент отражения, или альбедо (\(\alpha_{\lambda}\) для
монохроматического потока или \(a\) для всего потока солнечной
радиации), представляет собой меру энергии излучения, которая отражается
от поверхности. Для абсолютно черного тела значение альбедо равно нулю,
поскольку такое тело полностью поглощает всю падающую на него радиацию.

\hypertarget{ux444ux430ux43aux442ux43eux440ux44b-ux432ux43bux438ux44fux44eux449ux438ux435-ux43dux430-ux430ux43bux44cux431ux435ux434ux43e}{%
\subsubsection{Факторы, влияющие на
альбедо}\label{ux444ux430ux43aux442ux43eux440ux44b-ux432ux43bux438ux44fux44eux449ux438ux435-ux43dux430-ux430ux43bux44cux431ux435ux434ux43e}}

\begin{itemize}
\tightlist
\item
  \textbf{Тип подстилающей поверхности:} Альбедо существенно зависит от
  вида подстилающей поверхности и наличия растительности. Например,
  снег, лед и облака имеют особенно высокое альбедо. В частности,
  облачность может иметь альбедо в диапазоне 70-80\%.
\item
  \textbf{Антропогенные факторы:} Деятельность человека может изменять
  альбедо территорий. В населенных пунктах это проявляется в изменении
  альбедо поселений, например, за счет искусственного орошения или
  создания парков. Загрязнение атмосферы также способствует увеличению
  мутности тропосферы и стратосферы, что приводит к повышению альбедо.
\item
  \textbf{Влияние на термический режим:} Изменения альбедо подстилающей
  поверхности влияют на амплитуду тепловых волн.
\end{itemize}

\hypertarget{ux43aux43eux44dux444ux444ux438ux446ux438ux435ux43dux442-ux43fux43eux433ux43bux43eux449ux435ux43dux438ux44f}{%
\subsection{Коэффициент
поглощения}\label{ux43aux43eux44dux444ux444ux438ux446ux438ux435ux43dux442-ux43fux43eux433ux43bux43eux449ux435ux43dux438ux44f}}

Коэффициент поглощения (\(\alpha_{\lambda}\)) для монохроматического
потока излучения характеризует долю энергии, которая поглощается
веществом. Для абсолютно черного тела коэффициент поглощения равен
единице.

\hypertarget{ux43cux435ux445ux430ux43dux438ux437ux43cux44b-ux43fux43eux433ux43bux43eux449ux435ux43dux438ux44f-ux432-ux430ux442ux43cux43eux441ux444ux435ux440ux435}{%
\subsubsection{Механизмы поглощения в
атмосфере}\label{ux43cux435ux445ux430ux43dux438ux437ux43cux44b-ux43fux43eux433ux43bux43eux449ux435ux43dux438ux44f-ux432-ux430ux442ux43cux43eux441ux444ux435ux440ux435}}

\begin{itemize}
\tightlist
\item
  \textbf{Молекулярное и атомарное поглощение:} Поглощение радиации
  происходит, когда атомы и молекулы газов, составляющих воздух (таких
  как O₂, O₃, CO₂, H₂O, O, N), переходят в другое энергетическое
  состояние под воздействием солнечной радиации.
\item
  \textbf{Спектральная избирательность:}

  \begin{itemize}
  \tightlist
  \item
    Молекулярный и атомарный кислород, а также озон, практически
    полностью поглощают ультрафиолетовые лучи.
  \item
    Водяной пар (H₂O), углекислый газ (CO₂) и озон (O₃) поглощают
    инфракрасную радиацию.
  \item
    В видимой области спектра солнечной радиации атмосферные газы не
    поглощают энергию.
  \end{itemize}
\item
  \textbf{Закон Кирхгофа:} Газы, поглощающие энергию, также должны ее
  излучать. Атмосфера, имеющая температуру, близкую к температуре
  подстилающей поверхности, излучает длинноволновую радиацию. Часть
  этого излучения возвращается к земной поверхности (противоизлучение
  атмосферы), а другая часть уходит в космос (уходящее излучение).
\end{itemize}

\hypertarget{ux432ux43bux438ux44fux43dux438ux435-ux437ux430ux433ux440ux44fux437ux43dux435ux43dux438ux44f-ux43dux430-ux43fux43eux433ux43bux43eux449ux435ux43dux438ux435}{%
\subsubsection{Влияние загрязнения на
поглощение}\label{ux432ux43bux438ux44fux43dux438ux435-ux437ux430ux433ux440ux44fux437ux43dux435ux43dux438ux44f-ux43dux430-ux43fux43eux433ux43bux43eux449ux435ux43dux438ux435}}

\begin{itemize}
\tightlist
\item
  \textbf{Увеличение поглощения примесями:} Газообразные, жидкие и
  твердые примеси, присутствующие в загрязненной атмосфере, как правило,
  сильно поглощают радиацию, особенно в инфракрасном диапазоне (от 2 до
  120 мкм). Это приводит к изменениям теплового и термического режимов
  города {[}Предыдущий ответ{]}.
\item
  \textbf{Роль водяного пара:} Дополнительное количество водяного пара
  антропогенного происхождения (образующегося при сжигании топлива)
  вносит значительный вклад в изменение радиационного режима за счет
  поглощения {[}Предыдущий ответ{]}.
\item
  \textbf{Парниковый эффект:} Увеличение концентрации парниковых газов,
  таких как CO₂ и метан (который поглощает радиацию в ``окне
  прозрачности'' атмосферы между 8.5 и 12 мкм), приводит к усилению
  парникового эффекта и росту температуры воздуха у поверхности
  {[}Предыдущий ответ{]}.
\end{itemize}

\hypertarget{ux432ux437ux430ux438ux43cux43eux441ux432ux44fux437ux44c-ux43aux43eux44dux444ux444ux438ux446ux438ux435ux43dux442ux43eux432-ux438-ux440ux430ux434ux438ux430ux446ux438ux43eux43dux43dux44bux439-ux431ux430ux43bux430ux43dux441}{%
\subsection{Взаимосвязь коэффициентов и радиационный
баланс}\label{ux432ux437ux430ux438ux43cux43eux441ux432ux44fux437ux44c-ux43aux43eux44dux444ux444ux438ux446ux438ux435ux43dux442ux43eux432-ux438-ux440ux430ux434ux438ux430ux446ux438ux43eux43dux43dux44bux439-ux431ux430ux43bux430ux43dux441}}

В соответствии с законом сохранения энергии, для любой точки
пространства сумма коэффициентов поглощения (\(\alpha_{\lambda}\)),
пропускания (\(T_{\lambda}\)) и отражения (\(r_{\lambda}\)) для
монохроматического излучения равна единице. Это означает, что вся
падающая радиация либо поглощается, либо пропускается, либо отражается.

Радиационный баланс (\(\text{R}_\text{n}\)) определяется как разность
между поглощенным потоком суммарной солнечной радиации
(\(\text{Q}_\text{s}\)) и эффективным излучением. На планетарном уровне
Земля в целом находится в состоянии радиационного равновесия, где
приходящая солнечная радиация уравновешивается отраженной
коротковолновой и уходящей длинноволновой радиацией. Однако поверхность
Земли не находится в состоянии радиационного равновесия; эффективное
излучение лишь частично компенсирует приток от поглощенной радиации, а
избыточная энергия расходуется на испарение и контактную теплопередачу в
атмосферу.

\hypertarget{ux430ux43bux44cux431ux435ux434ux43e-ux440ux430ux437ux43bux438ux447ux43dux44bux445-ux435ux441ux442ux435ux441ux442ux432ux435ux43dux43dux44bux445-ux43fux43eux432ux435ux440ux445ux43dux43eux441ux442ux435ux439-ux43eux431ux43bux430ux43aux43eux432-ux438-ux437ux435ux43cux43bux438-ux43aux430ux43a-ux43fux43bux430ux43dux435ux442ux44b-ux438-ux435ux433ux43e-ux441ux443ux442ux43eux447ux43dux44bux439-ux445ux43eux434}{%
\section{Альбедо различных естественных поверхностей, облаков и Земли
как планеты и его суточный
ход}\label{ux430ux43bux44cux431ux435ux434ux43e-ux440ux430ux437ux43bux438ux447ux43dux44bux445-ux435ux441ux442ux435ux441ux442ux432ux435ux43dux43dux44bux445-ux43fux43eux432ux435ux440ux445ux43dux43eux441ux442ux435ux439-ux43eux431ux43bux430ux43aux43eux432-ux438-ux437ux435ux43cux43bux438-ux43aux430ux43a-ux43fux43bux430ux43dux435ux442ux44b-ux438-ux435ux433ux43e-ux441ux443ux442ux43eux447ux43dux44bux439-ux445ux43eux434}}

\hypertarget{ux43eux431ux449ux438ux435-ux43fux43eux43dux44fux442ux438ux44f-ux43eux431-ux430ux43bux44cux431ux435ux434ux43e}{%
\subsection{Общие понятия об
альбедо}\label{ux43eux431ux449ux438ux435-ux43fux43eux43dux44fux442ux438ux44f-ux43eux431-ux430ux43bux44cux431ux435ux434ux43e}}

Альбедо --- это характеристика, которая определяет энергию отраженной
радиации. Значения альбедо устанавливаются для различных видов
подстилающих поверхностей не только для отдельных длин электромагнитных
волн, но и для всего потока солнечной радиации.

\hypertarget{ux430ux43bux44cux431ux435ux434ux43e-ux435ux441ux442ux435ux441ux442ux432ux435ux43dux43dux44bux445-ux43fux43eux432ux435ux440ux445ux43dux43eux441ux442ux435ux439}{%
\subsection{Альбедо естественных
поверхностей}\label{ux430ux43bux44cux431ux435ux434ux43e-ux435ux441ux442ux435ux441ux442ux432ux435ux43dux43dux44bux445-ux43fux43eux432ux435ux440ux445ux43dux43eux441ux442ux435ux439}}

Альбедо подстилающей поверхности варьируется в зависимости от ее типа и
наличия растительности. Особенно высокие значения альбедо характерны для
снега, льда и облаков. Степень увлажненности почвы также оказывает
значительное влияние на ее альбедо. Например, неравномерное залегание
снега или льда, чередующегося с открытой водой и влажной почвой,
особенно сильно проявляется во влиянии на неоднородности альбедо.
Заснеженные участки, благодаря своему высокому альбедо, имеют тенденцию
дольше сохранять снег.

\hypertarget{ux430ux43bux44cux431ux435ux434ux43e-ux43eux431ux43bux430ux43aux43eux432}{%
\subsection{Альбедо
облаков}\label{ux430ux43bux44cux431ux435ux434ux43e-ux43eux431ux43bux430ux43aux43eux432}}

Как уже отмечалось, альбедо облаков является \textbf{особенно высоким}.
Облачность влияет на радиационный баланс земной поверхности, уменьшая
как приток солнечной радиации к ней, так и ее эффективное излучение.

\hypertarget{ux430ux43bux44cux431ux435ux434ux43e-ux437ux435ux43cux43bux438-ux43aux430ux43a-ux43fux43bux430ux43dux435ux442ux44b}{%
\subsection{Альбедо Земли как
планеты}\label{ux430ux43bux44cux431ux435ux434ux43e-ux437ux435ux43cux43bux438-ux43aux430ux43a-ux43fux43bux430ux43dux435ux442ux44b}}

Земля как планета в целом находится в состоянии радиационного
равновесия. Это означает, что поток приходящей от Солнца радиации
уравновешивается потоком отраженной коротковолновой и уходящей
длинноволновой радиации. Отраженная коротковолновая радиация и
составляет альбедо планеты. Шарообразность Земли приводит к тому, что
поток солнечной радиации перехватывается диском Земли, в то время как
излучение тепловой радиации происходит со всей ее поверхности.

\hypertarget{ux432ux43bux438ux44fux43dux438ux435-ux437ux430ux433ux440ux44fux437ux43dux435ux43dux438ux44f-ux430ux442ux43cux43eux441ux444ux435ux440ux44b-ux43dux430-ux430ux43bux44cux431ux435ux434ux43e}{%
\subsection{Влияние загрязнения атмосферы на
альбедо}\label{ux432ux43bux438ux44fux43dux438ux435-ux437ux430ux433ux440ux44fux437ux43dux435ux43dux438ux44f-ux430ux442ux43cux43eux441ux444ux435ux440ux44b-ux43dux430-ux430ux43bux44cux431ux435ux434ux43e}}

Загрязнение атмосферы, в частности увеличение мутности тропосферы и
стратосферы, может приводить к \textbf{повышению альбедо}.

\hypertarget{ux441ux443ux442ux43eux447ux43dux44bux439-ux445ux43eux434-ux430ux43bux44cux431ux435ux434ux43e}{%
\subsection{Суточный ход
альбедо}\label{ux441ux443ux442ux43eux447ux43dux44bux439-ux445ux43eux434-ux430ux43bux44cux431ux435ux434ux43e}}

В предоставленных источниках нет детального описания суточного хода
\textbf{самого альбедо} для различных поверхностей. Однако, радиационный
баланс, который определяется в том числе и альбедо, имеет выраженный
суточный ход: он почти всегда положителен в дневное время и отрицателен
ночью. Альбедо подстилающей поверхности влияет на амплитуду тепловых
волн.

ewpage

Уважаемый коллега, давайте подробно рассмотрим концепцию длинноволнового
излучения и его фундаментальное значение в метеорологии.

\hypertarget{ux434ux43bux438ux43dux43dux43eux432ux43eux43bux43dux43eux432ux43eux435-ux438ux437ux43bux443ux447ux435ux43dux438ux435}{%
\section{Длинноволновое
излучение}\label{ux434ux43bux438ux43dux43dux43eux432ux43eux43bux43dux43eux432ux43eux435-ux438ux437ux43bux443ux447ux435ux43dux438ux435}}

Длинноволновое излучение представляет собой лучистую энергию,
переносимую электромагнитными волнами, которые имеют большую длину волны
по сравнению с коротковолновым солнечным излучением. В метеорологии оно
относится к инфракрасному излучению с длиной волны от 0,76 до 100 мкм,
которое является частью теплового излучения тел.

\hypertarget{ux438ux441ux442ux43eux447ux43dux438ux43aux438-ux434ux43bux438ux43dux43dux43eux432ux43eux43bux43dux43eux432ux43eux433ux43e-ux438ux437ux43bux443ux447ux435ux43dux438ux44f}{%
\subsection{Источники длинноволнового
излучения}\label{ux438ux441ux442ux43eux447ux43dux438ux43aux438-ux434ux43bux438ux43dux43dux43eux432ux43eux43bux43dux43eux432ux43eux433ux43e-ux438ux437ux43bux443ux447ux435ux43dux438ux44f}}

Основными источниками длинноволновой радиации в атмосфере являются
излучение Земли и самой атмосферы. Интенсивность солнечной радиации во
всех частях спектра превышает интенсивность излучения Земли и атмосферы,
особенно в области коротких волн. Однако солнечная радиация поступает в
атмосферу из очень малого телесного угла, поэтому потоками
длинноволнового излучения Солнца, по сравнению с излучением Земли и
атмосферы, как правило, можно пренебречь.

\hypertarget{ux432ux437ux430ux438ux43cux43eux434ux435ux439ux441ux442ux432ux438ux435-ux441-ux430ux442ux43cux43eux441ux444ux435ux440ux43eux439}{%
\subsection{Взаимодействие с
атмосферой}\label{ux432ux437ux430ux438ux43cux43eux434ux435ux439ux441ux442ux432ux438ux435-ux441-ux430ux442ux43cux43eux441ux444ux435ux440ux43eux439}}

Проходя через атмосферу, лучистая энергия частично поглощается водяным
паром, углекислым газом и некоторыми другими составляющими воздуха,
превращаясь в тепловую энергию. Газы H₂O, CO₂ и O₃ поглощают
инфракрасную радиацию. Наряду с поглощением, каждый слой воздуха
излучает лучистую энергию в окружающее пространство, теряя при этом
часть своей внутренней тепловой энергии. Таким образом, лучистый
теплообмен в атмосфере формируется в результате поглощения и излучения
электромагнитных волн слоями воздуха. Для длинноволновой радиации эффект
рассеяния очень мал, и им можно пренебречь.

\hypertarget{ux437ux430ux43aux43eux43d-ux43aux438ux440ux445ux433ux43eux444ux430}{%
\subsubsection{Закон
Кирхгофа}\label{ux437ux430ux43aux43eux43d-ux43aux438ux440ux445ux433ux43eux444ux430}}

Согласно закону Кирхгофа, в условиях локального термодинамического
равновесия любое тело поглощает такую часть потока энергии, излучаемой
абсолютно черным телом, какую оно излучает. Это означает, что отношение
коэффициента излучения вещества к его коэффициенту поглощения не зависит
от индивидуальных свойств вещества, а является универсальной функцией
температуры и длины волны. Применительно к длинноволновой радиации,
закон Кирхгофа позволяет упростить уравнения переноса, поскольку
излучение в каждом участке спектра близко к равновесному излучению при
температуре соответствующей рассматриваемой части системы.

\hypertarget{ux43fux440ux43eux442ux438ux432ux43eux438ux437ux43bux443ux447ux435ux43dux438ux435-ux430ux442ux43cux43eux441ux444ux435ux440ux44b-ux438-ux43fux430ux440ux43dux438ux43aux43eux432ux44bux439-ux44dux444ux444ux435ux43aux442}{%
\subsubsection{Противоизлучение атмосферы и парниковый
эффект}\label{ux43fux440ux43eux442ux438ux432ux43eux438ux437ux43bux443ux447ux435ux43dux438ux435-ux430ux442ux43cux43eux441ux444ux435ux440ux44b-ux438-ux43fux430ux440ux43dux438ux43aux43eux432ux44bux439-ux44dux444ux444ux435ux43aux442}}

По закону Кирхгофа, газы, поглощающие энергию, также должны ее излучать.
Атмосфера, температура которой близка к температуре подстилающей
поверхности, излучает длинноволновую радиацию. Часть этого излучения
возвращается к поверхности Земли и называется \textbf{противоизлучением
атмосферы} (или облаков), а другая часть уходит в космос (называется
\textbf{уходящим излучением}). Инфракрасное излучение сильно поглощается
атмосферой и нагревает ее. Именно благодаря этому Земля находится в
состоянии лучистого равновесия, и количество поступающей от Солнца
энергии компенсируется энергией, излучаемой атмосферой в космос в другом
диапазоне электромагнитных волн. Этот процесс, при котором газы
(например, водяной пар и углекислый газ) поглощают земное инфракрасное
излучение и переизлучают его обратно к поверхности, известен как
\textbf{парниковый эффект}. Водяной пар является важнейшим парниковым
газом, а также двуокись углерода и метан.

\hypertarget{ux44dux444ux444ux435ux43aux442ux438ux432ux43dux43eux435-ux438ux437ux43bux443ux447ux435ux43dux438ux435}{%
\subsubsection{Эффективное
излучение}\label{ux44dux444ux444ux435ux43aux442ux438ux432ux43dux43eux435-ux438ux437ux43bux443ux447ux435ux43dux438ux435}}

\textbf{Эффективное излучение} -- это разность между излучением
подстилающей поверхности и противоизлучением атмосферы, направленным к
этой поверхности. Его обычно считают положительным, если оно направлено
вверх (поверхность нагревает атмосферу). В редких случаях, при сильных
температурных инверсиях, когда атмосфера теплее подстилающей
поверхности, эффективное излучение может быть отрицательным
(направленным вниз).

\hypertarget{ux440ux43eux43bux44c-ux432-ux44dux43dux435ux440ux433ux435ux442ux438ux447ux435ux441ux43aux43eux43c-ux431ux430ux43bux430ux43dux441ux435}{%
\subsection{Роль в энергетическом
балансе}\label{ux440ux43eux43bux44c-ux432-ux44dux43dux435ux440ux433ux435ux442ux438ux447ux435ux441ux43aux43eux43c-ux431ux430ux43bux430ux43dux441ux435}}

Длинноволновое излучение играет ключевую роль в энергетическом балансе
Земли. Энергетические условия на земной поверхности определяются
совместным действием коротковолновой (солнечной) и длинноволновой
радиации. Радиационный баланс подстилающей поверхности -- это разность
между поглощенной суммарной солнечной радиацией и эффективным
излучением.

\hypertarget{ux442ux435ux43fux43bux43eux432ux43eux439-ux431ux430ux43bux430ux43dux441-ux437ux435ux43cux43dux43eux439-ux43fux43eux432ux435ux440ux445ux43dux43eux441ux442ux438}{%
\subsubsection{Тепловой баланс земной
поверхности}\label{ux442ux435ux43fux43bux43eux432ux43eux439-ux431ux430ux43bux430ux43dux441-ux437ux435ux43cux43dux43eux439-ux43fux43eux432ux435ux440ux445ux43dux43eux441ux442ux438}}

Поглощенная солнечная радиация является основным источником тепла,
поступающего на подстилающую поверхность. Она полностью превращается в
тепловую энергию. Эта тепловая энергия поверхностного слоя передается
вверх (в атмосферу) путем теплового излучения, которое частично
компенсируется противоизлучением атмосферы. Поглощенная солнечная
радиация и суммарная потеря тепла на излучение подстилающей поверхности
(эффективное излучение) в сумме определяют радиационный баланс
подстилающей поверхности. Если бы поверхностный слой находился в
состоянии радиационного равновесия, радиационный баланс был бы равен
нулю, и это условие определяло бы температуру подстилающей поверхности.
Однако радиационное равновесие устанавливается медленно, поэтому в
формировании температуры подстилающей поверхности участвуют и другие
процессы теплопереноса.

Земля как планета в целом находится в состоянии радиационного
равновесия: поток приходящей от Солнца радиации уравновешен потоком
отраженной коротковолновой и уходящей длинноволновой радиации. Однако
поверхность Земли и сама атмосфера не находятся в состоянии
радиационного равновесия; атмосферный поток энергии больше, чем
радиационные потоки, что указывает на высокую скорость преобразования
энергии в атмосфере, позволяющую ей совершать работу по переносу
воздушных масс.

\hypertarget{ux443ux440ux430ux432ux43dux435ux43dux438ux44f-ux43fux435ux440ux435ux43dux43eux441ux430-ux434ux43bux438ux43dux43dux43eux432ux43eux43bux43dux43eux432ux43eux439-ux440ux430ux434ux438ux430ux446ux438ux438}{%
\subsection{Уравнения переноса длинноволновой
радиации}\label{ux443ux440ux430ux432ux43dux435ux43dux438ux44f-ux43fux435ux440ux435ux43dux43eux441ux430-ux434ux43bux438ux43dux43dux43eux432ux43eux43bux43dux43eux432ux43eux439-ux440ux430ux434ux438ux430ux446ux438ux438}}

Потоки длинноволновой радиации в атмосфере в основном складываются из
инфракрасного излучения Земли и атмосферы. Интенсивность радиации,
распространяющейся в тонком горизонтальном слое, уменьшается за счет
поглощения и увеличивается за счет излучения. Уравнения переноса
длинноволновой радиации учитывают эти процессы. Потоки длинноволнового
излучения в атмосфере:

\begin{itemize}
\tightlist
\item
  \textbf{A(m)}: поток длинноволнового излучения атмосферы, направленный
  вниз. Определяется как ∫−∞\^{}m D(m−u) dE(T(u)).
\item
  \textbf{B(m)}: поток длинноволнового излучения Земли и атмосферы,
  направленный вверх. Определяется как E(T₀)D(m) + ∫₀\^{}m D(m−u)
  dE(T(u)) + E(T₀)a(1-D(m)), где E(T₀) --- излучение земной поверхности.
\end{itemize}

В этих формулах:

\begin{itemize}
\tightlist
\item
  T(u) --- температура воздуха на уровне u.
\item
  T₀ --- температура у подстилающей поверхности.
\item
  D(u) --- функция пропускания для водяного пара.
\item
  a --- коэффициент поглощения деятельной поверхности почвы. Если a=1,
  это соответствует излучению земной поверхности как абсолютно черного
  тела.
\item
  u --- оптический путь длинноволновой радиации в атмосфере.
\end{itemize}

Эти формулы и другие подходы, использующие функцию пропускания,
позволяют рассчитать потоки длинноволновой радиации, учитывая линейчатую
структуру спектра поглощения водяного пара.

ewpage

\hypertarget{ux438ux437ux43bux443ux447ux435ux43dux438ux435-ux437ux435ux43cux43dux43eux439-ux43fux43eux432ux435ux440ux445ux43dux43eux441ux442ux438-ux438-ux430ux442ux43cux43eux441ux444ux435ux440ux44b}{%
\subsection{Излучение земной поверхности и
атмосферы}\label{ux438ux437ux43bux443ux447ux435ux43dux438ux435-ux437ux435ux43cux43dux43eux439-ux43fux43eux432ux435ux440ux445ux43dux43eux441ux442ux438-ux438-ux430ux442ux43cux43eux441ux444ux435ux440ux44b}}

Уважаемый коллега, продолжая наш разговор об энергетическом балансе,
крайне важно детально рассмотреть процессы излучения земной поверхности
и атмосферы в длинноволновом диапазоне, поскольку они играют
фундаментальную роль в тепловом режиме планеты и циркуляции атмосферы.

\hypertarget{ux441ux43eux431ux441ux442ux432ux435ux43dux43dux43eux435-ux438ux437ux43bux443ux447ux435ux43dux438ux435-ux437ux435ux43cux43dux43eux439-ux43fux43eux432ux435ux440ux445ux43dux43eux441ux442ux438}{%
\subsubsection{1. Собственное излучение земной
поверхности}\label{ux441ux43eux431ux441ux442ux432ux435ux43dux43dux43eux435-ux438ux437ux43bux443ux447ux435ux43dux438ux435-ux437ux435ux43cux43dux43eux439-ux43fux43eux432ux435ux440ux445ux43dux43eux441ux442ux438}}

Земная поверхность, нагреваясь за счет поглощения солнечной
коротковолновой радиации, сама становится источником теплового
излучения. Это собственное излучение Земли относится к длинноволновому
диапазону электромагнитного спектра, с длинами волн, превышающими 4 мкм,
преимущественно около 10 мкм, что четко отделяет его от коротковолновой
солнечной радиации. Земная поверхность поглощает солнечную радиацию, и в
результате этого поглощения нагревается {[}источник 1{]}.

Идеализированным объектом для изучения излучения является ``абсолютно
черное тело'' -- тело, которое полностью поглощает всю падающую на него
радиацию. Из реальных тел, снег, например, является близким к абсолютно
черному телу в области инфракрасного излучения. Закон Стефана-Больцмана
описывает зависимость полного потока излучения абсолютно черного тела от
его абсолютной температуры. Земля как абсолютно черное тело, получая
солнечную энергию, имела бы равновесную температуру 284 К. Однако с
учетом альбедо Земли как планеты (примерно 0.33, где половина
поверхности покрыта облаками), температура равновесия составляет около
253 К.

\hypertarget{ux438ux437ux43bux443ux447ux435ux43dux438ux435-ux430ux442ux43cux43eux441ux444ux435ux440ux44b-ux438-ux43fux430ux440ux43dux438ux43aux43eux432ux44bux439-ux44dux444ux444ux435ux43aux442}{%
\subsubsection{2. Излучение атмосферы и парниковый
эффект}\label{ux438ux437ux43bux443ux447ux435ux43dux438ux435-ux430ux442ux43cux43eux441ux444ux435ux440ux44b-ux438-ux43fux430ux440ux43dux438ux43aux43eux432ux44bux439-ux44dux444ux444ux435ux43aux442}}

Поглощая энергию, газы атмосферы также обладают способностью излучать
ее, теряя при этом часть своей внутренней тепловой энергии. Согласно
закону Кирхгофа, любое тело в условиях локального термодинамического
равновесия поглощает такую часть падающего потока, какую оно излучает.
Таким образом, газы, поглощающие радиацию, также ее излучают.

Атмосфера, температура которой близка к температуре подстилающей
поверхности, излучает длинноволновую радиацию во всех направлениях.
Часть этого излучения возвращается к поверхности Земли, и это явление
называется \textbf{противоизлучением атмосферы}. Противоизлучение
атмосферы уменьшает потерю тепла земной поверхностью и приводит к
возникновению \textbf{парникового эффекта}. Этот эффект обусловлен
избирательным поглощением длинноволновой радиации атмосферными газами:
водяным паром (\(\text{H}_2\text{O}\)), углекислым газом
(\(\text{CO}_2\)) и озоном (\(\text{O}_3\)). В последние годы отмечается
потенциальный рост роли метана в поглощении земной радиации из-за
увеличения техногенных выбросов, хотя его текущая роль все еще мала.

Сложное расположение полос поглощения атмосферных газов существенно
затрудняет расчеты. Однако существует диапазон длин волн от 8.5 до 12
мкм, в котором атмосфера практически полностью пропускает излучение; эта
область называется \textbf{«окном прозрачности» атмосферы}. Любой газ,
который поглощает радиацию в этом ``окне прозрачности'', может вызвать
серьезные изменения в энергетическом балансе планеты при увеличении его
концентрации.

\hypertarget{ux44dux444ux444ux435ux43aux442ux438ux432ux43dux43eux435-ux438ux437ux43bux443ux447ux435ux43dux438ux435-ux438-ux440ux430ux434ux438ux430ux446ux438ux43eux43dux43dux44bux439-ux431ux430ux43bux430ux43dux441}{%
\subsubsection{3. Эффективное излучение и радиационный
баланс}\label{ux44dux444ux444ux435ux43aux442ux438ux432ux43dux43eux435-ux438ux437ux43bux443ux447ux435ux43dux438ux435-ux438-ux440ux430ux434ux438ux430ux446ux438ux43eux43dux43dux44bux439-ux431ux430ux43bux430ux43dux441}}

\textbf{Эффективное излучение} (\(\text{E}_{\text{эф}}\)) характеризует
чистую потерю энергии земной поверхностью при излучении. Оно
определяется как разность между излучением подстилающей поверхности
(\(\text{E}_{\text{s}}\)) и противоизлучением атмосферы
(\(\text{E}_{\text{а}}\)):
\(\text{E}_{\text{эф}} = \text{E}_{\text{s}} - \text{E}_{\text{а}}\).
Типичное значение эффективного излучения составляет около 25\% от
излучения абсолютно черного тела при температуре подстилающей
поверхности. Оно увеличивается с ростом температуры и уменьшается с
ростом влажности. Обычно эффективное излучение положительно (направлено
вверх), но в редких случаях, при сильных инверсиях температуры, может
быть отрицательным.

\textbf{Радиационный баланс подстилающей поверхности}
(\(\text{R}_{\text{s}}\)) представляет собой скорость изменения энергии
в поверхностном слое, определяемую суммой поглощенной солнечной радиации
и суммарной потерей тепла на излучение (эффективное излучение). Если бы
поверхностный слой находился в состоянии радиационного равновесия,
радиационный баланс был бы равен нулю, и это условие определяло бы
температуру поверхности. Однако радиационное равновесие устанавливается
медленно, и в формировании температуры участвуют другие процессы
теплопереноса.

Земля как планета в целом находится в состоянии радиационного
равновесия, когда приходящий от Солнца поток радиации уравновешивается
потоком отраженной коротковолновой и уходящей длинноволновой радиации.
Однако поверхность Земли и атмосфера по отдельности не находятся в
состоянии радиационного равновесия. Избыток энергии в атмосфере,
превышающий радиационные потоки, позволяет атмосфере совершать работу по
переносу воздушных масс. Именно этот лучистый приток тепла является
основным источником тепла, обусловливающим атмосферные движения.

\hypertarget{ux43cux43eux434ux435ux43bux438ux440ux43eux432ux430ux43dux438ux435-ux43bux443ux447ux438ux441ux442ux43eux433ux43e-ux442ux435ux43fux43bux43eux43eux431ux43cux435ux43dux430}{%
\subsubsection{4. Моделирование лучистого
теплообмена}\label{ux43cux43eux434ux435ux43bux438ux440ux43eux432ux430ux43dux438ux435-ux43bux443ux447ux438ux441ux442ux43eux433ux43e-ux442ux435ux43fux43bux43eux43eux431ux43cux435ux43dux430}}

Для описания и прогнозирования лучистого теплообмена в атмосфере
используются уравнения переноса длинноволновой радиации. Эти уравнения
интегрируются с учетом краевых условий: на большой высоте интенсивностью
длинноволновой радиации пренебрегают (радиация извне не поступает), а на
уровне деятельной поверхности интенсивность радиации, распространяющейся
вверх, равна сумме собственного излучения поверхности и непоглощенной
радиации, поступающей сверху.

Поток длинноволновой радиации, направленный вниз, определяется
излучением и поглощением вышележащих слоев атмосферы. Поток радиации,
направленный вверх, складывается из излучения земной поверхности,
достигшего данного уровня, потока радиации, обусловленного излучением и
поглощением в слое атмосферы между Землей и данным уровнем, и доли
атмосферного излучения, отраженного земной поверхностью и достигшего
этого уровня. Приближенные формулы для вычисления потоков длинноволновой
радиации получены исходя из предположения, что земная поверхность
излучает как абсолютно черное тело.

Все эти радиационные процессы, наряду с тепло- и влагообменом,
представляют собой основные факторы, определяющие погоду и климат, и их
учет критически важен в динамической метеорологии.

ewpage

\hypertarget{ux440ux430ux441ux43fux440ux435ux434ux435ux43bux435ux43dux438ux435-ux44dux43dux435ux440ux433ux438ux438-ux43fux43e-ux441ux43fux435ux43aux442ux440ux443-1}{%
\section{Распределение энергии по
спектру}\label{ux440ux430ux441ux43fux440ux435ux434ux435ux43bux435ux43dux438ux435-ux44dux43dux435ux440ux433ux438ux438-ux43fux43e-ux441ux43fux435ux43aux442ux440ux443-1}}

Лучистая энергия представляет собой энергию, переносимую
электромагнитными волнами различной длины. Спектр электромагнитных волн
охватывает широкий диапазон, от радиоволн длиной в несколько километров
до рентгеновских лучей и гамма-лучей, измеряемых в нанометрах и
ангстремах. В метеорологии особое внимание уделяется ультрафиолетовому,
видимому и инфракрасному диапазонам спектра.

\hypertarget{ux438ux441ux442ux43eux447ux43dux438ux43aux438-ux438ux437ux43bux443ux447ux435ux43dux438ux44f-ux432-ux430ux442ux43cux43eux441ux444ux435ux440ux435}{%
\subsection{Источники излучения в
атмосфере}\label{ux438ux441ux442ux43eux447ux43dux438ux43aux438-ux438ux437ux43bux443ux447ux435ux43dux438ux44f-ux432-ux430ux442ux43cux43eux441ux444ux435ux440ux435}}

Основными источниками лучистой энергии для атмосферы являются излучение
Солнца, излучение Земли и самой атмосферы. Интенсивность солнечной
радиации во всех частях спектра превышает интенсивность излучения Земли
и атмосферы, особенно в области коротких волн. Однако, поскольку
солнечная радиация поступает в атмосферу из очень малого телесного угла,
потоками длинноволнового излучения Солнца, по сравнению с излучением
Земли и атмосферы, как правило, можно пренебречь.

\hypertarget{ux432ux437ux430ux438ux43cux43eux434ux435ux439ux441ux442ux432ux438ux435-ux438ux437ux43bux443ux447ux435ux43dux438ux44f-ux441-ux430ux442ux43cux43eux441ux444ux435ux440ux43eux439-ux43fux43e-ux441ux43fux435ux43aux442ux440ux443}{%
\subsection{Взаимодействие излучения с атмосферой по
спектру}\label{ux432ux437ux430ux438ux43cux43eux434ux435ux439ux441ux442ux432ux438ux435-ux438ux437ux43bux443ux447ux435ux43dux438ux44f-ux441-ux430ux442ux43cux43eux441ux444ux435ux440ux43eux439-ux43fux43e-ux441ux43fux435ux43aux442ux440ux443}}

Проходя через атмосферу, лучистая энергия частично поглощается
различными газовыми составляющими и превращается в тепловую энергию.
Наряду с поглощением, каждый слой воздуха излучает лучистую энергию,
теряя при этом часть своей внутренней тепловой энергии.

Спектры солнечного и земного излучения четко разделяются на два основных
интервала:

\begin{itemize}
\tightlist
\item
  \textbf{Коротковолновое излучение} (длины волн короче 4 мкм) --- это
  преимущественно солнечное излучение.
\item
  \textbf{Длинноволновое излучение} (длины волн длиннее 4 мкм) --- это
  преимущественно земное излучение.
\end{itemize}

\hypertarget{ux441ux43fux435ux446ux438ux444ux438ux43aux430-ux43fux43eux433ux43bux43eux449ux435ux43dux438ux44f-ux433ux430ux437ux430ux43cux438}{%
\subsubsection{Специфика поглощения
газами}\label{ux441ux43fux435ux446ux438ux444ux438ux43aux430-ux43fux43eux433ux43bux43eux449ux435ux43dux438ux44f-ux433ux430ux437ux430ux43cux438}}

\begin{itemize}
\tightlist
\item
  Молекулярный и атомарный кислород (O₂), а также озон (O₃) и азот (N)
  практически полностью поглощают ультрафиолетовые лучи в верхних слоях
  атмосферы.
\item
  Газы, такие как водяной пар (H₂O), углекислый газ (CO₂) и озон (O₃),
  поглощают инфракрасную радиацию.
\item
  В видимой области спектра солнечной радиации атмосферные газы
  практически не поглощают энергию.
\item
  Спектр поглощения водяного пара имеет линейчатую структуру, и его
  зависимость от длины волны не может быть представлена аналитически.
\item
  Длинноволновое излучение сильно поглощается атмосферой и нагревает ее.
  При этом эффект рассеяния для длинноволновой радиации очень мал, и им
  можно пренебречь.
\item
  В результате взаимодействия с атмосферой, максимальный поток солнечной
  радиации, достигающий земной поверхности, уменьшается.
\end{itemize}

\hypertarget{ux442ux435ux43eux440ux435ux442ux438ux447ux435ux441ux43aux438ux435-ux43eux441ux43dux43eux432ux44b-ux441ux43fux435ux43aux442ux440ux430ux43bux44cux43dux43eux433ux43e-ux438ux437ux43bux443ux447ux435ux43dux438ux44f}{%
\subsection{Теоретические основы спектрального
излучения}\label{ux442ux435ux43eux440ux435ux442ux438ux447ux435ux441ux43aux438ux435-ux43eux441ux43dux43eux432ux44b-ux441ux43fux435ux43aux442ux440ux430ux43bux44cux43dux43eux433ux43e-ux438ux437ux43bux443ux447ux435ux43dux438ux44f}}

Интенсивность излучения абсолютно черного тела зависит от его
температуры и длины волны. Концепция абсолютно черного тела представляет
собой физическую абстракцию, относящуюся к веществу, которое полностью
поглощает весь падающий на него поток излучения. Излучение абсолютно
черного тела является верхним пределом излучения всех тел при данной
температуре.

Согласно закону Кирхгофа, в условиях локального термодинамического
равновесия любое тело поглощает такую часть потока энергии, излучаемой
абсолютно черным телом, какую оно излучает. Иными словами, отношение
коэффициента излучения вещества к его коэффициенту поглощения равно
интенсивности черного излучения и зависит только от температуры и длины
волны. Зависимость полного потока излучения абсолютно черного тела от
его абсолютной температуры была выведена Стефаном и Больцманом, а Планк
теоретически вывел выражение для интенсивности черного излучения в
зависимости от длины волны и абсолютной температуры.

ewpage

Уважаемый коллега,

Продолжая обсуждение фундаментальных процессов, формирующих тепловой и
динамический режимы атмосферы, крайне важно углубиться в радиационные
свойства естественных поверхностей. Эти свойства определяют, как энергия
взаимодействует с подстилающей поверхностью, что в конечном итоге влияет
на атмосферные процессы и климат.

\hypertarget{ux444ux443ux43dux434ux430ux43cux435ux43dux442ux430ux43bux44cux43dux44bux435-ux43fux43eux43dux44fux442ux438ux44f-ux440ux430ux434ux438ux430ux446ux438ux438}{%
\subsubsection{1. Фундаментальные понятия
радиации}\label{ux444ux443ux43dux434ux430ux43cux435ux43dux442ux430ux43bux44cux43dux44bux435-ux43fux43eux43dux44fux442ux438ux44f-ux440ux430ux434ux438ux430ux446ux438ux438}}

Всякое тело обладает способностью излучать радиацию в окружающее
пространство, при этом теряя запас внутренней тепловой энергии. Это
процесс, обратный поглощению энергии, когда атомы переходят из
неустойчивых состояний в устойчивые, испуская при этом электромагнитные
волны. Земная поверхность поглощает солнечную радиацию, что приводит к
ее нагреванию {[}источник 1{]}.

Идеализированным объектом для изучения излучения является ``абсолютно
черное тело'' --- физическая абстракция, обозначающая тело, которое
полностью поглощает всю падающую на него радиацию (коэффициент
поглощения αλ = 1) и является максимально возможным излучателем при
данной температуре. Из реальных тел, снег, например, является близким к
абсолютно черному телу в области инфракрасного излучения. Закон
Стефана-Больцмана описывает зависимость полного потока излучения
абсолютно черного тела от его абсолютной температуры.

Закон Кирхгофа устанавливает связь между излучательной способностью и
поглощательной способностью тела, утверждая, что отношение коэффициента
излучения к коэффициенту поглощения не зависит от индивидуальных свойств
вещества, а является универсальной функцией температуры и длины волны.

\hypertarget{ux440ux430ux434ux438ux430ux446ux438ux43eux43dux43dux44bux435-ux441ux432ux43eux439ux441ux442ux432ux430-ux440ux430ux437ux43bux438ux447ux43dux44bux445-ux435ux441ux442ux435ux441ux442ux432ux435ux43dux43dux44bux445-ux43fux43eux432ux435ux440ux445ux43dux43eux441ux442ux435ux439}{%
\subsubsection{2. Радиационные свойства различных естественных
поверхностей}\label{ux440ux430ux434ux438ux430ux446ux438ux43eux43dux43dux44bux435-ux441ux432ux43eux439ux441ux442ux432ux430-ux440ux430ux437ux43bux438ux447ux43dux44bux445-ux435ux441ux442ux435ux441ux442ux432ux435ux43dux43dux44bux445-ux43fux43eux432ux435ux440ux445ux43dux43eux441ux442ux435ux439}}

Радиационные свойства естественных поверхностей варьируются в
зависимости от их типа и состояния:

\hypertarget{ux437ux435ux43cux43dux430ux44f-ux43fux43eux432ux435ux440ux445ux43dux43eux441ux442ux44c-ux432-ux446ux435ux43bux43eux43c}{%
\paragraph{2.1. Земная поверхность (в
целом)}\label{ux437ux435ux43cux43dux430ux44f-ux43fux43eux432ux435ux440ux445ux43dux43eux441ux442ux44c-ux432-ux446ux435ux43bux43eux43c}}

Поглощенная солнечная радиация нагревает поверхность Земли, которая
затем излучает энергию обратно, но уже в инфракрасном диапазоне.
Собственное излучение Земли относится к длинноволновому диапазону
электромагнитного спектра, с длинами волн, превышающими 4 мкм,
преимущественно около 10 мкм {[}источник 1{]}. Земля как планета в целом
находится в состоянии радиационного равновесия, где приходящий от Солнца
поток радиации уравновешивается потоком отраженной коротковолновой и
уходящей длинноволновой радиации. Однако сама поверхность Земли не
находится в состоянии радиационного равновесия; избыточная энергия
расходуется на испарение и контактную теплопередачу в атмосферу.

\hypertarget{ux441ux43dux435ux433-ux438-ux43bux435ux434}{%
\paragraph{2.2. Снег и
лед}\label{ux441ux43dux435ux433-ux438-ux43bux435ux434}}

Снег и лед обладают исключительно высоким альбедо, отражая значительную
часть падающей солнечной радиации. Альбедо свежевыпавшего снега
составляет 70-95\%, старого снега - 40-70\%. Эта высокая отражательная
способность приводит к существенному уменьшению поглощения солнечной
энергии, что, в свою очередь, влияет на тепловой баланс и температуру
подстилающей поверхности.

\hypertarget{ux432ux43eux434ux43dux430ux44f-ux43fux43eux432ux435ux440ux445ux43dux43eux441ux442ux44c}{%
\paragraph{2.3. Водная
поверхность}\label{ux432ux43eux434ux43dux430ux44f-ux43fux43eux432ux435ux440ux445ux43dux43eux441ux442ux44c}}

Альбедо водной поверхности значительно зависит от угла падения солнечных
лучей: оно низкое, когда солнце высоко, и высокое, когда солнце
находится низко над горизонтом. Вода обладает большой удельной
теплоемкостью, что позволяет ей аккумулировать значительное количество
тепла. Теплообмен в водоемах происходит не только за счет молекулярной
теплопроводности, но и за счет турбулентного перемешивания, которое
значительно ускоряет перенос тепла по сравнению с молекулярным.
Температура поверхности воды, особенно океанов, играет ключевую роль в
формировании крупномасштабных атмосферных процессов, таких как циклоны и
антициклоны.

\hypertarget{ux43fux43eux447ux432ux430}{%
\paragraph{2.4. Почва}\label{ux43fux43eux447ux432ux430}}

Характер подстилающей поверхности сильно влияет на ее температуру.
Альбедо почвы зависит от ее типа и наличия растительности. Увлажненность
почвы оказывает существенное влияние на ее температуру, определяя
альбедо, скорость испарения и транспирации, а также изменяя теплоемкость
и температуропроводность. Например, более влажная почва нагревается и
охлаждается медленнее сухой, но суточные колебания температуры
распространяются в ней глубже. Различные типы почв (глинистые, песчаные,
торфяные, болотные) имеют свои особенности в тепловом режиме.

\hypertarget{ux440ux430ux441ux442ux438ux442ux435ux43bux44cux43dux44bux439-ux43fux43eux43aux440ux43eux432}{%
\paragraph{2.5. Растительный
покров}\label{ux440ux430ux441ux442ux438ux442ux435ux43bux44cux43dux44bux439-ux43fux43eux43aux440ux43eux432}}

Растительность способствует охлаждению почвы и уменьшению амплитуды
температурных колебаний.

\hypertarget{ux430ux442ux43cux43eux441ux444ux435ux440ux430-ux438-ux43eux431ux43bux430ux43aux430}{%
\paragraph{2.6. Атмосфера и
облака}\label{ux430ux442ux43cux43eux441ux444ux435ux440ux430-ux438-ux43eux431ux43bux430ux43aux430}}

Атмосферный воздух сам по себе слабо поглощает солнечную
(коротковолновую) радиацию. Однако он активно поглощает длинноволновое
(инфракрасное) излучение, испускаемое Землей, главным образом благодаря
парниковым газам, таким как водяной пар. Это поглощение приводит к
нагреванию атмосферы. Увеличение концентрации парниковых газов, в том
числе антропогенного водяного пара, уменьшает эффективный поток
излучения и потери тепла земной поверхностью, что способствует повышению
температуры воздуха.

Облака оказывают значительное влияние на радиационный баланс Земли. Они
обладают высоким альбедо (например, мощные облака - 70-95\%, тонкие -
20-70\%), отражая солнечную радиацию. При этом облака также поглощают и
излучают длинноволновую радиацию. Облачность существенно изменяет
радиационный, а следовательно, и термический и влажностный режимы
деятельного слоя почвы и атмосферы. В целом, облачность увеличивает
результирующий приток радиации к земной поверхности, когда радиационный
баланс отрицателен (зимой в умеренных и высоких широтах), и уменьшает
его, когда он положителен (летом).

\hypertarget{ux432ux437ux430ux438ux43cux43eux441ux432ux44fux437ux44c-ux440ux430ux434ux438ux430ux446ux438ux43eux43dux43dux44bux445-ux441ux432ux43eux439ux441ux442ux432-ux441-ux434ux438ux43dux430ux43cux438ux43aux43eux439-ux430ux442ux43cux43eux441ux444ux435ux440ux44b}{%
\subsubsection{3. Взаимосвязь радиационных свойств с динамикой
атмосферы}\label{ux432ux437ux430ux438ux43cux43eux441ux432ux44fux437ux44c-ux440ux430ux434ux438ux430ux446ux438ux43eux43dux43dux44bux445-ux441ux432ux43eux439ux441ux442ux432-ux441-ux434ux438ux43dux430ux43cux438ux43aux43eux439-ux430ux442ux43cux43eux441ux444ux435ux440ux44b}}

Неравномерное распределение радиационного баланса по земной поверхности,
обусловленное сферичностью Земли и неодинаковым нагревом различных
участков, является основной причиной существования общей циркуляции
атмосферы. Эти перепады давления и температуры заставляют воздух
перемещаться, создавая ветры и формируя крупномасштабные циркуляционные
системы.

Различия в теплофизических свойствах между сушей и океаном приводят к
сезонным изменениям температуры и адвекции тепла/холода, что, в свою
очередь, способствует зарождению и развитию синоптических вихрей, таких
как циклоны и антициклоны. Например, зимой над океанами преобладает
адвекция холода, способствующая циклогенезу, а над материками --
адвекция тепла, благоприятствующая антициклогенезу.

Таким образом, радиационные свойства естественных поверхностей являются
одним из ключевых факторов, определяющих как локальный тепловой режим,
так и глобальные атмосферные процессы, включая формирование полей
давления, ветра, облачности и осадков.

ewpage

Уважаемый коллега,

Продолжая наш диалог о тепловом режиме атмосферы, целесообразно детально
рассмотреть процесс поглощения земного излучения в атмосфере. Это
ключевой механизм, определяющий вертикальную термическую структуру
атмосферы и, как следствие, ее динамику и климат.

\hypertarget{ux43fux440ux438ux440ux43eux434ux430-ux437ux435ux43cux43dux43eux433ux43e-ux438ux437ux43bux443ux447ux435ux43dux438ux44f}{%
\subsubsection{1. Природа земного
излучения}\label{ux43fux440ux438ux440ux43eux434ux430-ux437ux435ux43cux43dux43eux433ux43e-ux438ux437ux43bux443ux447ux435ux43dux438ux44f}}

Земная поверхность, поглощая солнечную радиацию, нагревается и, в свою
очередь, излучает энергию {[}источник 1{]}. Это собственное излучение
Земли относится к длинноволновому диапазону электромагнитного спектра, с
длинами волн, превышающими 4 мкм, и максимумом интенсивности около 10
мкм {[}источник 1, 306{]}. Всякое тело обладает способностью излучать
радиацию, теряя при этом внутреннюю тепловую энергию {[}источник 1,
70{]}. Излучение абсолютно черного тела служит верхним пределом
излучения всех тел при данной температуре {[}источник 70{]}, а его
полный поток зависит от абсолютной температуры, согласно закону
Стефана-Больцмана {[}источник 75{]}.

\hypertarget{ux43fux43eux433ux43bux43eux449ux435ux43dux438ux435-ux434ux43bux438ux43dux43dux43eux432ux43eux43bux43dux43eux432ux43eux433ux43e-ux438ux437ux43bux443ux447ux435ux43dux438ux44f-ux432-ux430ux442ux43cux43eux441ux444ux435ux440ux435}{%
\subsubsection{2. Поглощение длинноволнового излучения в
атмосфере}\label{ux43fux43eux433ux43bux43eux449ux435ux43dux438ux435-ux434ux43bux438ux43dux43dux43eux432ux43eux43bux43dux43eux432ux43eux433ux43e-ux438ux437ux43bux443ux447ux435ux43dux438ux44f-ux432-ux430ux442ux43cux43eux441ux444ux435ux440ux435}}

Атмосфера активно поглощает длинноволновое (инфракрасное) излучение,
испускаемое Землей {[}источник 383{]}. В отличие от коротковолновой
солнечной радиации, которую атмосферный воздух поглощает слабо
{[}источник 266, 322{]}, длинноволновое излучение Земли и атмосферы
является основным источником длинноволновых потоков в атмосфере
{[}источник 76{]}.

\hypertarget{ux43eux441ux43dux43eux432ux43dux44bux435-ux43fux43eux433ux43bux43eux442ux438ux442ux435ux43bux438-ux43fux430ux440ux43dux438ux43aux43eux432ux44bux435-ux433ux430ux437ux44b}{%
\paragraph{2.1. Основные поглотители (парниковые
газы)}\label{ux43eux441ux43dux43eux432ux43dux44bux435-ux43fux43eux433ux43bux43eux442ux438ux442ux435ux43bux438-ux43fux430ux440ux43dux438ux43aux43eux432ux44bux435-ux433ux430ux437ux44b}}

Поглощение длинноволновой радиации в атмосфере осуществляется
избирательно, то есть лишь отдельными длинами волн, характерными для
каждого газа {[}источник 307{]}. Главными поглотителями земной радиации
являются:

\begin{itemize}
\tightlist
\item
  \textbf{Водяной пар (H2O)}: Является наиболее важным поглотителем
  {[}источник 307{]}. Он играет огромную роль в поглощении и излучении
  лучистой энергии, а также в накоплении тепла Землей, обладая большой
  удельной теплоемкостью {[}источник 384{]}. Естественно, что
  антропогенный водяной пар, выбрасываемый в атмосферу, увеличивает его
  содержание и влияет на радиационный режим {[}источник 211, 212{]}.
\item
  \textbf{Углекислый газ (CO2)}: Занимает второе место по значимости
  {[}источник 307{]}. Его роль в повышении температуры воздуха у
  поверхности Земли через парниковый эффект обсуждалась еще С.
  Аррениусом в конце XIX века {[}источник 214{]}.
\item
  \textbf{Озон (O3)}: Также является важным поглотителем длинноволновой
  радиации {[}источник 307{]}.
\item
  \textbf{Метан (CH4)}: В последние годы ученые прогнозируют увеличение
  его роли в поглощении земной радиации в связи с ростом техногенных
  выбросов, хотя его текущая роль все еще мала {[}источник 307{]}.
  Метан, поглощая радиацию в диапазоне атмосферного ``окна
  прозрачности'' (8.5-12 мкм), может привести к серьезным изменениям
  энергетического баланса планеты {[}источник 308{]}.
\end{itemize}

Полосы поглощения этих газов имеют очень сложное расположение в спектре
{[}источник 308{]}.

\hypertarget{ux43cux435ux445ux430ux43dux438ux437ux43c-ux43fux43eux433ux43bux43eux449ux435ux43dux438ux44f-ux438-ux438ux437ux43bux443ux447ux435ux43dux438ux44f}{%
\paragraph{2.2. Механизм поглощения и
излучения}\label{ux43cux435ux445ux430ux43dux438ux437ux43c-ux43fux43eux433ux43bux43eux449ux435ux43dux438ux44f-ux438-ux438ux437ux43bux443ux447ux435ux43dux438ux44f}}

Поглощение радиации происходит, когда атомы и молекулы газов воздуха
переходят в другое энергетическое состояние под воздействием солнечной
радиации {[}источник 302, 289{]}. Согласно закону Кирхгофа, газы,
поглощающие энергию, также должны ее излучать {[}источник 309, 295{]}.
Интенсивность излучения слоя пропорциональна его массовому коэффициенту
излучения и плотности поглощающего вещества {[}источник 72{]}.

\hypertarget{ux43fux430ux440ux43dux438ux43aux43eux432ux44bux439-ux44dux444ux444ux435ux43aux442-ux438-ux435ux433ux43e-ux432ux43bux438ux44fux43dux438ux435-ux43dux430-ux442ux435ux440ux43cux438ux447ux435ux441ux43aux438ux439-ux440ux435ux436ux438ux43c}{%
\subsubsection{3. Парниковый эффект и его влияние на термический
режим}\label{ux43fux430ux440ux43dux438ux43aux43eux432ux44bux439-ux44dux444ux444ux435ux43aux442-ux438-ux435ux433ux43e-ux432ux43bux438ux44fux43dux438ux435-ux43dux430-ux442ux435ux440ux43cux438ux447ux435ux441ux43aux438ux439-ux440ux435ux436ux438ux43c}}

Атмосфера, температура которой близка к температуре подстилающей
поверхности, излучает длинноволновую радиацию во все стороны. Часть
этого излучения возвращается к поверхности Земли, что называется
\textbf{противоизлучением атмосферы}, а другая часть уходит в космос
{[}источник 309{]}.

Противоизлучение атмосферы уменьшает потерю тепла земной поверхностью,
что приводит к возникновению \textbf{парникового эффекта} {[}источник
310{]}. Этот эффект крайне важен для поддержания жизни на Земле. Без
него температура планеты была бы значительно ниже {[}источник 310{]}.
Концентрация парниковых газов в атмосфере (H2O, CO2, O3) регулирует
значение температуры подстилающей поверхности и обусловливает падение
температуры с высотой {[}источник 312{]}.

Разность между излучением подстилающей поверхности и противоизлучением
атмосферы называется \textbf{эффективным излучением} {[}источник 312{]}.
Оно характеризует чистую потерю энергии земной поверхностью при
излучении и обычно положительно (поверхность Земли нагревает атмосферу)
{[}источник 312, 313{]}.

\hypertarget{ux432ux43bux438ux44fux43dux438ux435-ux43eux431ux43bux430ux43aux43eux432}{%
\subsubsection{4. Влияние
облаков}\label{ux432ux43bux438ux44fux43dux438ux435-ux43eux431ux43bux430ux43aux43eux432}}

Облака оказывают значительное влияние на радиационный баланс и
термический режим {[}источник 168{]}.

\begin{itemize}
\tightlist
\item
  Они обладают высоким альбедо (до 95\% для мощных облаков), отражая
  солнечную радиацию {[}источник 148{]}.
\item
  При этом облака также активно поглощают и излучают длинноволновую
  радиацию {[}источник 82{]}.
\item
  Влияние облачности на результирующий приток радиации к земной
  поверхности зависит от сезона и широты: зимой в умеренных и высоких
  широтах облачность увеличивает приток радиации (снижая радиационные
  потери тепла), тогда как летом она его уменьшает {[}источник 148, 171,
  559{]}.
\item
  Облака и выделяющееся при конденсации тепло существенно влияют на
  поддержание разности температур в циклонах и на развитие вихрей
  {[}источник 158{]}.
\end{itemize}

\hypertarget{ux440ux43eux43bux44c-ux43fux43eux433ux43bux43eux449ux435ux43dux438ux44f-ux437ux435ux43cux43dux43eux433ux43e-ux438ux437ux43bux443ux447ux435ux43dux438ux44f-ux432-ux430ux442ux43cux43eux441ux444ux435ux440ux43dux43eux439-ux434ux438ux43dux430ux43cux438ux43aux435}{%
\subsubsection{5. Роль поглощения земного излучения в атмосферной
динамике}\label{ux440ux43eux43bux44c-ux43fux43eux433ux43bux43eux449ux435ux43dux438ux44f-ux437ux435ux43cux43dux43eux433ux43e-ux438ux437ux43bux443ux447ux435ux43dux438ux44f-ux432-ux430ux442ux43cux43eux441ux444ux435ux440ux43dux43eux439-ux434ux438ux43dux430ux43cux438ux43aux435}}

Неравномерное распределение давления, являющееся основной причиной
движения воздуха, обусловлено процессами теплообмена в атмосфере и на ее
границе с Землей {[}источник 1{]}. Лучистый приток тепла от Солнца,
поглощаемый Землей и затем переизлучаемый в инфракрасном диапазоне и
поглощаемый атмосферой, является основным источником тепла,
обусловливающим атмосферные движения {[}источник 85, 385, 645{]}.

Избыточная энергия, получаемая поверхностью Земли от поглощенной
радиации, частично компенсируется эффективным излучением, но оставшаяся
часть расходуется на испарение и контактную теплопередачу в атмосферу
{[}источник 314{]}. Атмосфера, в свою очередь, не находится в состоянии
радиационного равновесия; запас мощности позволяет ей совершать работу
по переносу воздушных масс {[}источник 315{]}.

Горизонтальные различия в температуре и давлении, вызванные
неравномерным распределением радиации и неоднородностью теплофизических
свойств подстилающей поверхности, порождают вихревые движения и общую
циркуляцию атмосферы {[}источник 99, 157{]}. Таким образом, поглощение
земного излучения играет фундаментальную роль в замыкании
энергетического цикла и поддержании динамики атмосферных процессов.

ewpage

Уважаемый коллега,

Радиационный баланс системы Земля-атмосфера, а также тепловой режим
подстилающей поверхности и самой атмосферы, критически зависят от
взаимодействия длинноволнового излучения. Давайте подробно рассмотрим
механизмы уходящего и противоизлучения атмосферы, являющиеся ключевыми
компонентами этого взаимодействия.

\hypertarget{ux444ux443ux43dux434ux430ux43cux435ux43dux442ux430ux43bux44cux43dux44bux435-ux43fux440ux438ux43dux446ux438ux43fux44b-ux438ux437ux43bux443ux447ux435ux43dux438ux44f}{%
\subsubsection{1. Фундаментальные Принципы
Излучения}\label{ux444ux443ux43dux434ux430ux43cux435ux43dux442ux430ux43bux44cux43dux44bux435-ux43fux440ux438ux43dux446ux438ux43fux44b-ux438ux437ux43bux443ux447ux435ux43dux438ux44f}}

Как нам хорошо известно, любое тело, обладающее внутренней тепловой
энергией, излучает радиацию в окружающее пространство, при этом теряя
часть этой энергии {[}источник 1, источник 78{]}. Этот процесс, по сути,
является переходом атомов из неустойчивых энергетических состояний в
устойчивые, сопровождающимся испусканием электромагнитных волн
{[}источник 279{]}. Идеализированной моделью излучателя является
``абсолютно черное тело'' -- объект, который полностью поглощает всю
падающую на него радиацию (коэффициент поглощения αλ = 1) и является
максимально возможным излучателем при заданной температуре {[}источник
78{]}. В области инфракрасного излучения даже снег является хорошим
приближением к абсолютно черному телу {[}источник 78{]}. Полный поток
излучения абсолютно черного тела определяется законом Стефана-Больцмана,
зависящим от его абсолютной температуры {[}источник 81{]}. Закон
Кирхгофа, в свою очередь, устанавливает фундаментальную связь между
излучательной и поглощательной способностью тела, утверждая, что
отношение коэффициента излучения к коэффициенту поглощения есть
универсальная функция температуры и длины волны {[}источник 80{]}.

\hypertarget{ux438ux437ux43bux443ux447ux435ux43dux438ux435-ux437ux435ux43cux43dux43eux439-ux43fux43eux432ux435ux440ux445ux43dux43eux441ux442ux438}{%
\subsubsection{2. Излучение Земной
Поверхности}\label{ux438ux437ux43bux443ux447ux435ux43dux438ux435-ux437ux435ux43cux43dux43eux439-ux43fux43eux432ux435ux440ux445ux43dux43eux441ux442ux438}}

Земная поверхность, поглощая солнечную (коротковолновую) радиацию,
нагревается и затем сама становится источником излучения. Однако, в
отличие от коротковолнового солнечного излучения (длины волн короче 4
мкм), собственное излучение Земли относится к длинноволновому диапазону
электромагнитного спектра, с преобладающими длинами волн около 10 мкм
(более 4 мкм) {[}источник 1, источник 296, источник 356{]}. Это
излучение Земли, наряду с излучением самой атмосферы, формирует основные
потоки длинноволновой радиации в атмосфере {[}источник 83{]}.

\hypertarget{ux43fux43eux433ux43bux43eux449ux435ux43dux438ux435-ux438-ux438ux437ux43bux443ux447ux435ux43dux438ux435-ux430ux442ux43cux43eux441ux444ux435ux440ux44b}{%
\subsubsection{3. Поглощение и Излучение
Атмосферы}\label{ux43fux43eux433ux43bux43eux449ux435ux43dux438ux435-ux438-ux438ux437ux43bux443ux447ux435ux43dux438ux435-ux430ux442ux43cux43eux441ux444ux435ux440ux44b}}

Проходя через атмосферу, лучистая энергия Земли частично поглощается.
Основными поглотителями в длинноволновом диапазоне являются парниковые
газы: водяной пар (H2O), углекислый газ (CO2) и озон (O3) {[}источник
74, источник 291, источник 297{]}. Важно отметить, что атмосферные газы
поглощают радиацию селективно, то есть только определенные длины волн
{[}источник 297{]}. В видимой области спектра солнечной радиации
атмосферные газы практически не поглощают энергию, но они активно
поглощают инфракрасную радиацию {[}источник 291{]}.

После поглощения энергии, каждый слой воздуха, в свою очередь, излучает
радиацию в окружающее пространство, теряя при этом часть своей
внутренней тепловой энергии {[}источник 74{]}. Таким образом, лучистый
теплообмен в атмосфере является результатом непрерывных процессов
поглощения и излучения электромагнитных волн слоями воздуха {[}источник
75{]}.

\hypertarget{ux43fux440ux43eux442ux438ux432ux43eux438ux437ux43bux443ux447ux435ux43dux438ux435-ux430ux442ux43cux43eux441ux444ux435ux440ux44b}{%
\subsubsection{4. Противоизлучение
Атмосферы}\label{ux43fux440ux43eux442ux438ux432ux43eux438ux437ux43bux443ux447ux435ux43dux438ux435-ux430ux442ux43cux43eux441ux444ux435ux440ux44b}}

Согласно закону Кирхгофа, газы, поглощающие энергию, также должны ее и
излучать {[}источник 299{]}. Поскольку атмосфера, особенно вблизи
подстилающей поверхности, имеет температуру, близкую к температуре
Земли, она также излучает длинноволновую радиацию {[}источник 299{]}.
Это излучение атмосферы направлено во все стороны. Та часть этого
излучения, которая возвращается обратно к земной поверхности, называется
\textbf{противоизлучением атмосферы} {[}источник 299{]}. Оно является
нисходящим потоком длинноволнового излучения атмосферы {[}источник
91{]}.

\hypertarget{ux443ux445ux43eux434ux44fux449ux435ux435-ux438ux437ux43bux443ux447ux435ux43dux438ux435}{%
\subsubsection{5. Уходящее
Излучение}\label{ux443ux445ux43eux434ux44fux449ux435ux435-ux438ux437ux43bux443ux447ux435ux43dux438ux435}}

Другая часть излучения атмосферы, а также часть излучения Земли, которая
не была поглощена атмосферой, направляется вверх и уходит в космическое
пространство. Этот поток энергии называется \textbf{уходящим излучением}
{[}источник 299{]}. Он представляет собой восходящий поток
длинноволнового излучения Земли и атмосферы {[}источник 91{]}.

Существование ``окна прозрачности'' атмосферы в диапазоне длин волн от
8.5 до 12 мкм {[}источник 298{]} означает, что в этом спектральном
интервале атмосфера практически пропускает все излучение. Это позволяет
части длинноволнового излучения Земли напрямую уходить в космос, минуя
поглощение парниковыми газами.

\hypertarget{ux43fux430ux440ux43dux438ux43aux43eux432ux44bux439-ux44dux444ux444ux435ux43aux442-ux438-ux440ux430ux434ux438ux430ux446ux438ux43eux43dux43dux44bux439-ux431ux430ux43bux430ux43dux441}{%
\subsubsection{6. Парниковый Эффект и Радиационный
Баланс}\label{ux43fux430ux440ux43dux438ux43aux43eux432ux44bux439-ux44dux444ux444ux435ux43aux442-ux438-ux440ux430ux434ux438ux430ux446ux438ux43eux43dux43dux44bux439-ux431ux430ux43bux430ux43dux441}}

Взаимодействие между излучением Земли, поглощением и противоизлучением
атмосферы составляет суть \textbf{парникового эффекта}. Именно благодаря
парниковому эффекту средняя температура у поверхности Земли
поддерживается на уровне, пригодном для жизни (около 284 К) {[}источник
299{]}. Концентрация парниковых газов (H2O, CO2, O3) в атмосфере
обусловливает падение температуры с высотой и регулирует значение
температуры подстилающей поверхности в масштабах всей планеты
{[}источник 302{]}.

Земля как планета в целом находится в состоянии радиационного
равновесия, где приходящий от Солнца поток радиации уравновешивается
потоком отраженной коротковолновой и уходящей длинноволновой радиации
{[}источник 305{]}. Однако, сама поверхность Земли и атмосфера по
отдельности не находятся в состоянии радиационного равновесия
{[}источник 305, источник 306{]}. Избыточная энергия поверхности
расходуется на испарение и контактную теплопередачу в атмосферу
{[}источник 305{]}. Атмосфера же, имея поток энергии больше, чем
радиационные потоки, использует этот запас мощности для совершения
работы по переносу воздушных масс {[}источник 306{]}.

Разность между излучением подстилающей поверхности и противоизлучением
атмосферы называется \textbf{эффективным излучением} {[}источник 302{]}.
Оно характеризует чистую потерю энергии земной поверхностью при
излучении и обычно положительно (направлено вверх), хотя в редких
случаях сильных инверсий температуры может быть отрицательным
{[}источник 303{]}. Эффективное излучение зависит от температуры и
влажности воздуха: с ростом температуры оно растет, а с ростом влажности
уменьшается {[}источник 302{]}.

\hypertarget{ux432ux43bux438ux44fux43dux438ux435-ux43eux431ux43bux430ux447ux43dux43eux441ux442ux438}{%
\subsubsection{7. Влияние
Облачности}\label{ux432ux43bux438ux44fux43dux438ux435-ux43eux431ux43bux430ux447ux43dux43eux441ux442ux438}}

Облака играют двойную и очень существенную роль в радиационном балансе.
Они обладают высоким альбедо (до 70-95\% для мощных облаков), отражая
значительную часть приходящей солнечной (коротковолновой) радиации
{[}источник 152, источник 286{]}. Однако, одновременно, облака активно
поглощают и излучают длинноволновую радиацию, подобно парниковым газам
{[}источник 152{]}. Влияние облаков на результирующий приток радиации к
земной поверхности зависит от сезона и широты: оно увеличивается, когда
радиационный баланс отрицателен (зимой в умеренных и высоких широтах), и
уменьшается, когда он положителен (летом) {[}источник 152, источник
169{]}. Облачность существенно изменяет как радиационный, так и
связанный с ним термический и влажностный режимы деятельного слоя почвы
и атмосферы {[}источник 169{]}.

\hypertarget{ux432ux437ux430ux438ux43cux43eux441ux432ux44fux437ux44c-ux441-ux430ux442ux43cux43eux441ux444ux435ux440ux43dux44bux43cux438-ux43fux440ux43eux446ux435ux441ux441ux430ux43cux438}{%
\subsubsection{8. Взаимосвязь с Атмосферными
Процессами}\label{ux432ux437ux430ux438ux43cux43eux441ux432ux44fux437ux44c-ux441-ux430ux442ux43cux43eux441ux444ux435ux440ux43dux44bux43cux438-ux43fux440ux43eux446ux435ux441ux441ux430ux43cux438}}

Неравномерное распределение радиационного баланса по земному шару,
вызванное сферичностью Земли и различиями в радиационных свойствах
подстилающей поверхности, является основной причиной неравномерного
распределения давления и температуры, что, в свою очередь, генерирует
атмосферные движения и общую циркуляцию атмосферы {[}источник 1,
источник 104, источник 358, источник 359{]}. Таким образом, процессы
уходящего и противоизлучения являются фундаментальными механизмами,
определяющими энергетику нашей планеты и динамику ее атмосферы.

ewpage

\hypertarget{ux44dux444ux444ux435ux43aux442ux438ux432ux43dux43eux435-ux438ux437ux43bux443ux447ux435ux43dux438ux435-ux438-ux444ux430ux43aux442ux43eux440ux44b-ux432ux43bux438ux44fux44eux449ux438ux435-ux43dux430-ux43dux435ux433ux43e}{%
\section{Эффективное излучение и факторы, влияющие на
него}\label{ux44dux444ux444ux435ux43aux442ux438ux432ux43dux43eux435-ux438ux437ux43bux443ux447ux435ux43dux438ux435-ux438-ux444ux430ux43aux442ux43eux440ux44b-ux432ux43bux438ux44fux44eux449ux438ux435-ux43dux430-ux43dux435ux433ux43e}}

Уважаемый коллега,

Рассматривая энергетический баланс подстилающей поверхности и атмосферы,
ключевым аспектом является понимание эффективного излучения. Этот
параметр характеризует чистые потери энергии земной поверхностью
посредством радиации и является неотъемлемой частью уравнения теплового
баланса.

\hypertarget{ux43eux43fux440ux435ux434ux435ux43bux435ux43dux438ux435-ux44dux444ux444ux435ux43aux442ux438ux432ux43dux43eux433ux43e-ux438ux437ux43bux443ux447ux435ux43dux438ux44f}{%
\subsection{1. Определение эффективного
излучения}\label{ux43eux43fux440ux435ux434ux435ux43bux435ux43dux438ux435-ux44dux444ux444ux435ux43aux442ux438ux432ux43dux43eux433ux43e-ux438ux437ux43bux443ux447ux435ux43dux438ux44f}}

Эффективное излучение (\(\text{E}_{\text{эф}}\)) представляет собой
разность между собственным излучением подстилающей поверхности
(\(\text{E}_{\text{s}}\)) и противоизлучением атмосферы
(\(\text{E}_{\text{а}}\)). То есть,
\(\text{E}_{\text{эф}} = \text{E}_{\text{s}} - \text{E}_{\text{а}}\).
Это фактически характеризует чистую потерю энергии земной поверхностью
при излучении. Как правило, значение эффективного излучения
положительно, что означает, что подстилающая поверхность передает тепло
в атмосферу. Однако в редких случаях, например, при сильных
температурных инверсиях, когда атмосфера теплее подстилающей
поверхности, эффективное излучение может быть отрицательным.

Собственное излучение Земли относится к длинноволновому диапазону
электромагнитного спектра, с длинами волн, превышающими 4 мкм,
преимущественно около 10 мкм. Подстилающая поверхность излучает энергию,
поскольку поглощенная солнечная радиация нагревает её. Это излучение
Земли и атмосферы в основном состоит из инфракрасного излучения.

Атмосфера, температура которой близка к температуре подстилающей
поверхности, также излучает длинноволновую радиацию. Это излучение
атмосферы направлено во все стороны: одна часть возвращается к
поверхности земли (противоизлучение атмосферы, \(\text{E}_{\text{а}}\)),
а другая уходит в космос (уходящее излучение).

В целом, эффективное излучение составляет примерно 25\% от излучения
абсолютно черного тела при температуре подстилающей поверхности.

\hypertarget{ux444ux430ux43aux442ux43eux440ux44b-ux432ux43bux438ux44fux44eux449ux438ux435-ux43dux430-ux44dux444ux444ux435ux43aux442ux438ux432ux43dux43eux435-ux438ux437ux43bux443ux447ux435ux43dux438ux435}{%
\subsection{2. Факторы, влияющие на эффективное
излучение}\label{ux444ux430ux43aux442ux43eux440ux44b-ux432ux43bux438ux44fux44eux449ux438ux435-ux43dux430-ux44dux444ux444ux435ux43aux442ux438ux432ux43dux43eux435-ux438ux437ux43bux443ux447ux435ux43dux438ux435}}

На величину эффективного излучения влияют как свойства самой
подстилающей поверхности, так и характеристики атмосферы.

\hypertarget{ux442ux435ux43cux43fux435ux440ux430ux442ux443ux440ux430-ux43fux43eux434ux441ux442ux438ux43bux430ux44eux449ux435ux439-ux43fux43eux432ux435ux440ux445ux43dux43eux441ux442ux438-ux438-ux432ux43eux437ux434ux443ux445ux430}{%
\subsubsection{2.1. Температура подстилающей поверхности и
воздуха}\label{ux442ux435ux43cux43fux435ux440ux430ux442ux443ux440ux430-ux43fux43eux434ux441ux442ux438ux43bux430ux44eux449ux435ux439-ux43fux43eux432ux435ux440ux445ux43dux43eux441ux442ux438-ux438-ux432ux43eux437ux434ux443ux445ux430}}

Температура поверхности напрямую влияет на её собственное излучение
(\(\text{E}_{\text{s}}\)). Согласно закону Стефана-Больцмана, полный
поток излучения абсолютно черного тела пропорционален четвертой степени
его абсолютной температуры. Хотя земная поверхность не является идеально
черным телом, её излучательная способность, связанная с температурой,
является основным компонентом эффективного излучения. С ростом
температуры эффективное излучение, как правило, растет.

Температура атмосферы, особенно приземного слоя, определяет
интенсивность противоизлучения атмосферы. Чем выше температура воздуха,
тем больше энергии он может излучать обратно к поверхности.

\hypertarget{ux432ux43bux430ux436ux43dux43eux441ux442ux44c-ux432ux43eux437ux434ux443ux445ux430-ux438-ux43fux430ux440ux43dux438ux43aux43eux432ux44bux435-ux433ux430ux437ux44b}{%
\subsubsection{2.2. Влажность воздуха и парниковые
газы}\label{ux432ux43bux430ux436ux43dux43eux441ux442ux44c-ux432ux43eux437ux434ux443ux445ux430-ux438-ux43fux430ux440ux43dux438ux43aux43eux432ux44bux435-ux433ux430ux437ux44b}}

Водяной пар является одним из основных поглотителей и излучателей
длинноволновой радиации в атмосфере. Увеличение содержания
(концентрации) водяного пара в атмосфере, так же как и любого другого
парникового газа, приводит к уменьшению эффективного потока излучения и,
следовательно, к уменьшению потерь тепла земной поверхностью. Влияние
водяного пара на поле температуры проявляется именно через изменение
эффективного излучения земной поверхности.

Помимо водяного пара, другими основными поглотителями земной радиации
являются углекислый газ (\(\text{CO}_2\)) и озон (\(\text{O}_3\)). В
последние годы отмечается увеличение роли метана (\(\text{CH}_4\)) в
поглощении земной радиации, особенно в диапазоне ``окна прозрачности''
атмосферы (8.5 до 12 мкм), где атмосфера обычно практически пропускает
всё излучение. Увеличение концентрации любого газа, поглощающего
радиацию в этом ``окне'', может привести к серьезным изменениям
энергетического баланса планеты.

\hypertarget{ux43eux431ux43bux430ux447ux43dux43eux441ux442ux44c}{%
\subsubsection{2.3.
Облачность}\label{ux43eux431ux43bux430ux447ux43dux43eux441ux442ux44c}}

Облака оказывают значительное влияние на радиационный баланс земной
поверхности и, как следствие, на термический и влажностный режим
деятельного слоя почвы и атмосферы. Облачность уменьшает как приток
солнечной (коротковолновой) радиации к земной поверхности, так и её
собственное эффективное (длинноволновое) излучение.

Влияние облачности на результирующий приток радиации к земной
поверхности зависит от времени года:

\begin{itemize}
\tightlist
\item
  \textbf{Зимой (в умеренных и высоких широтах):} Когда радиационный
  баланс земной поверхности отрицателен, облачность, как правило,
  увеличивает результирующий приток радиации. Это происходит за счет
  уменьшения радиационных потерь тепла, что способствует повышению
  температуры воздуха.
\item
  \textbf{Летом:} Когда радиационный баланс положителен, облачность
  уменьшает результирующий приток радиации.
\end{itemize}

В целом, облачность играет роль в сохранении тепла у поверхности,
блокируя эффективное излучение Земли в космос.

\hypertarget{ux432ux440ux435ux43cux44f-ux441ux443ux442ux43eux43a-ux438-ux433ux43eux434ux430}{%
\subsubsection{2.4. Время суток и
года}\label{ux432ux440ux435ux43cux44f-ux441ux443ux442ux43eux43a-ux438-ux433ux43eux434ux430}}

Эффективное излучение имеет суточный и годовой ход. Эти колебания
обусловлены изменениями температуры подстилающей поверхности и воздуха,
а также изменением содержания водяного пара и облачности в течение суток
и сезонов. Например, ночью, когда отсутствует солнечная радиация,
эффективное излучение является основным механизмом охлаждения
поверхности.

\hypertarget{ux442ux443ux440ux431ux443ux43bux435ux43dux442ux43dux43eux441ux442ux44c-ux438-ux430ux434ux432ux435ux43aux446ux438ux44f-ux43aux43eux441ux432ux435ux43dux43dux43eux435-ux432ux43bux438ux44fux43dux438ux435}{%
\subsubsection{2.5. Турбулентность и адвекция (косвенное
влияние)}\label{ux442ux443ux440ux431ux443ux43bux435ux43dux442ux43dux43eux441ux442ux44c-ux438-ux430ux434ux432ux435ux43aux446ux438ux44f-ux43aux43eux441ux432ux435ux43dux43dux43eux435-ux432ux43bux438ux44fux43dux438ux435}}

Турбулентный теплообмен и адвекция тепла/влаги влияют на температуру и
влажность приземного слоя атмосферы, которые, в свою очередь, изменяют
величину противоизлучения атмосферы и, соответственно, эффективное
излучение. Для оценки чисто радиационного влияния, в исследованиях
стараются исключить эти факторы, выбирая условия наблюдения (например,
ночные сроки, слабый ветер, отсутствие низких облаков, туманов и
осадков).

Таким образом, эффективное излучение является комплексной
характеристикой, интегрирующей взаимодействие теплового состояния
поверхности и атмосферного состава, что критически важно для понимания
общего теплового баланса климатической системы.

ewpage

\hypertarget{ux440ux430ux434ux438ux430ux446ux438ux43eux43dux43dux44bux439-ux431ux430ux43bux430ux43dux441-ux437ux435ux43cux43dux43eux439-ux43fux43eux432ux435ux440ux445ux43dux43eux441ux442ux438}{%
\section{1. Радиационный баланс земной
поверхности}\label{ux440ux430ux434ux438ux430ux446ux438ux43eux43dux43dux44bux439-ux431ux430ux43bux430ux43dux441-ux437ux435ux43cux43dux43eux439-ux43fux43eux432ux435ux440ux445ux43dux43eux441ux442ux438}}

Радиационный баланс подстилающей поверхности (\(R_s\)) представляет
собой алгебраическую сумму всех потоков лучистой энергии, проходящих
через единицу поверхности за единицу времени. По сути, это разность
между поглощенной радиацией и собственным излучением слоя.

Источниками энергии для земной поверхности являются:

\begin{itemize}
\tightlist
\item
  \textbf{Солнечная (коротковолновая) радиация}: Земная поверхность
  поглощает солнечную радиацию, что приводит к ее нагреванию {[}источник
  1, 298{]}. Количество поглощенной солнечной энергии зависит от
  конкретного вида каждого участка поверхности. Альбедо подстилающей
  поверхности, которое является отношением отраженной солнечной радиации
  к падающей, играет ключевую роль в этом поглощении, сильно завися от
  типа поверхности и наличия растительности.
\item
  \textbf{Длинноволновая радиация атмосферы (противоизлучение)}:
  Атмосфера, нагретая излучением Земли, сама излучает длинноволновую
  радиацию, часть которой возвращается к поверхности Земли. Это
  противоизлучение атмосферы уменьшает потерю тепла земной поверхностью
  и приводит к возникновению парникового эффекта.
\end{itemize}

Потери энергии земной поверхностью происходят за счет:

\begin{itemize}
\tightlist
\item
  \textbf{Собственного длинноволнового (инфракрасного) излучения Земли}:
  Поглощенная солнечная радиация нагревает поверхность Земли, а затем
  эта энергия переизлучается в инфракрасном диапазоне. Это излучение
  Земли относится к длинноволновому диапазону электромагнитного спектра,
  с длинами волн, превышающими 4 мкм, преимущественно около 10 мкм
  {[}источник 1{]}.
\item
  \textbf{Отраженной коротковолновой радиации}: Часть солнечной радиации
  отражается от поверхности Земли, определяемая ее альбедо.
\end{itemize}

Земля как планета в целом находится в состоянии радиационного
равновесия, где приходящий от Солнца поток радиации уравновешивается
потоком отраженной коротковолновой и уходящей длинноволновой радиации.
Однако сама поверхность Земли \emph{не находится} в состоянии
радиационного равновесия; эффективное излучение лишь частично
компенсирует приток от поглощенной радиации. Избыточная энергия, или,
наоборот, дефицит, расходуется на испарение и контактную теплопередачу
от подстилающей поверхности в атмосферу {[}источник 1, 248{]}.

\textbf{Географическое распределение и влияние на погоду:} Наибольшие
значения радиационного баланса приходятся на океаны тропической и
субтропической зон. Избыток энергии образуется в широтном поясе от 40°
ю.ш. до 40° с.ш., и в этой области он расходуется в основном на
испарение. Затем, в форме водяного пара, энергия переносится в умеренные
и полярные широты, где пар конденсируется, отдавая содержащееся в нем
тепло. Именно тропический океан является поставщиком энергии для
атмосферных процессов.

Радиационный баланс компенсируется суммарным действием всех остальных
потоков энергии на подстилающей поверхности: потоком тепла в более
глубокие слои (G), турбулентным потоком тепла (P) и затратами тепла на
испарение (LE). Тип подстилающей поверхности (вода, снег, почва,
растительность) сильно влияет на ее температуру и, соответственно, на
радиационный баланс. Например, снег и лед обладают высоким альбедо,
отражая значительную часть солнечной радиации, тогда как водная
поверхность имеет низкое альбедо при высоком положении Солнца
{[}источник 1{]}.

\hypertarget{ux440ux430ux434ux438ux430ux446ux438ux43eux43dux43dux44bux439-ux431ux430ux43bux430ux43dux441-ux430ux442ux43cux43eux441ux444ux435ux440ux44b}{%
\subsubsection{2. Радиационный баланс
атмосферы}\label{ux440ux430ux434ux438ux430ux446ux438ux43eux43dux43dux44bux439-ux431ux430ux43bux430ux43dux441-ux430ux442ux43cux43eux441ux444ux435ux440ux44b}}

Атмосфера также не находится в состоянии радиационного равновесия. Поток
энергии в атмосфере больше, чем радиационные потоки. Это означает, что
скорость преобразования энергии в атмосфере больше, чем скорость
установления радиационного равновесия. Именно этот запас мощности
позволяет атмосфере совершать работу по переносу воздушных масс.

Взаимодействие атмосферы с радиацией происходит следующим образом:

\begin{itemize}
\tightlist
\item
  \textbf{Коротковолновая радиация}: Атмосферный воздух сам по себе
  слабо поглощает солнечную (коротковолновую) радиацию. Большая часть
  солнечной радиации поглощается земной поверхностью. Однако существуют
  механизмы ослабления солнечной радиации в атмосфере, такие как
  рассеяние и поглощение.
\item
  \textbf{Длинноволновая радиация}: Атмосфера активно поглощает
  длинноволновое (инфракрасное) излучение, испускаемое Землей, благодаря
  парниковым газам, таким как водяной пар. Это поглощение приводит к
  нагреванию атмосферы. Антропогенный водяной пар, например, увеличивает
  содержание водяного пара в воздухе, что, в свою очередь, усиливает
  парниковый эффект и приводит к повышению температуры воздуха.
\end{itemize}

\textbf{Роль облаков в радиационном балансе атмосферы:} Облака оказывают
значительное влияние на радиационный баланс Земли и атмосферы. Они
обладают высоким альбедо, отражая солнечную радиацию (мощные облака -
70-95\%, тонкие - 20-70\%), и при этом поглощают и излучают
длинноволновую радиацию {[}источник 1, 123, 245{]}. Облачность
существенно изменяет радиационный, а следовательно, и термический и
влажностный режимы деятельного слоя почвы и атмосферы, особенно
приземного и пограничного слоев. В целом, облачность увеличивает
результирующий приток радиации к земной поверхности, когда радиационный
баланс отрицателен (зимой в умеренных и высоких широтах), и уменьшает
его, когда он положителен (летом).

\textbf{Взаимосвязь с динамикой атмосферы:} Неравномерное распределение
радиационного баланса по земному шару, обусловленное сферичностью Земли
и неодинаковым нагревом различных участков, является основной причиной
существования общей циркуляции атмосферы. Воздействие лучистой энергии
Солнца и Земли усиливает молекулярное движение, повышает температуру
воздуха и влияет на атмосферное давление. Поскольку разные участки Земли
нагреваются неодинаково, между ними возникают перепады атмосферного
давления, что порождает ветры и формирует крупномасштабные атмосферные
движения. Атмосферная циркуляция осуществляет перенос тепла от экватора
к полюсам.

Таким образом, радиационный баланс земной поверхности и атмосферы не
просто описывает энергетический обмен, но и является фундаментальным
механизмом, управляющим всей динамикой атмосферы и формированием
климата.

ewpage

Уважаемый коллега,

Рад предоставить информацию относительно радиационного баланса Земли как
планеты, основываясь на доступных нам источниках. Это ключевой аспект
для понимания климатической системы.

\hypertarget{ux440ux430ux434ux438ux430ux446ux438ux43eux43dux43dux44bux439-ux431ux430ux43bux430ux43dux441-ux437ux435ux43cux43bux438-ux43aux430ux43a-ux43fux43bux430ux43dux435ux442ux44b}{%
\section{Радиационный баланс Земли как
планеты}\label{ux440ux430ux434ux438ux430ux446ux438ux43eux43dux43dux44bux439-ux431ux430ux43bux430ux43dux441-ux437ux435ux43cux43bux438-ux43aux430ux43a-ux43fux43bux430ux43dux435ux442ux44b}}

Радиационный баланс --- это разность между поглощенной радиацией и
собственным излучением тела или слоя. Для планеты Земля в целом он
представляет собой критический индикатор ее энергетического состояния.

\hypertarget{ux440ux430ux434ux438ux430ux446ux438ux43eux43dux43dux43eux435-ux440ux430ux432ux43dux43eux432ux435ux441ux438ux435-ux43fux43bux430ux43dux435ux442ux44b}{%
\subsection{1. Радиационное равновесие
планеты}\label{ux440ux430ux434ux438ux430ux446ux438ux43eux43dux43dux43eux435-ux440ux430ux432ux43dux43eux432ux435ux441ux438ux435-ux43fux43bux430ux43dux435ux442ux44b}}

Земля как планета в целом находится в состоянии радиационного
равновесия. Это означает, что поток приходящей от Солнца радиации
уравновешен потоком отраженной коротковолновой радиации и уходящей
длинноволновой радиации. Собственное излучение слоя выражается через
эталонное излучение абсолютно черного тела при заданной температуре.

\hypertarget{ux43fux440ux438ux445ux43eux434ux44fux449ux430ux44f-ux441ux43eux43bux43dux435ux447ux43dux430ux44f-ux440ux430ux434ux438ux430ux446ux438ux44f}{%
\subsection{2. Приходящая солнечная
радиация}\label{ux43fux440ux438ux445ux43eux434ux44fux449ux430ux44f-ux441ux43eux43bux43dux435ux447ux43dux430ux44f-ux440ux430ux434ux438ux430ux446ux438ux44f}}

Основным источником лучистой энергии для атмосферы Земли является
солнечная радиация. Солнечная радиация --- это тепловое излучение, т.е.
электромагнитные волны. Солнце излучает энергию во всех частях спектра,
причем интенсивность солнечной радиации во всех частях спектра
значительно больше интенсивности излучения Земли и атмосферы, особенно в
области коротких волн.

На верхней границе атмосферы Земли поступает поток солнечной радиации,
называемый солнечной постоянной (I₀), величина которой составляет
примерно 1380 Вт/м². Из-за шарообразной формы Земли, поток солнечной
радиации, перехватываемый диском Земли, равен πR²I₀, где R --- радиус
Земли.

\hypertarget{ux443ux445ux43eux434ux44fux449ux430ux44f-ux440ux430ux434ux438ux430ux446ux438ux44f}{%
\subsection{3. Уходящая
радиация}\label{ux443ux445ux43eux434ux44fux449ux430ux44f-ux440ux430ux434ux438ux430ux446ux438ux44f}}

Уходящая радиация от Земли состоит из двух основных компонентов:

\hypertarget{ux43eux442ux440ux430ux436ux435ux43dux43dux430ux44f-ux43aux43eux440ux43eux442ux43aux43eux432ux43eux43bux43dux43eux432ux430ux44f-ux440ux430ux434ux438ux430ux446ux438ux44f-ux430ux43bux44cux431ux435ux434ux43e}{%
\subsubsection{3.1. Отраженная коротковолновая радиация
(Альбедо)}\label{ux43eux442ux440ux430ux436ux435ux43dux43dux430ux44f-ux43aux43eux440ux43eux442ux43aux43eux432ux43eux43bux43dux43eux432ux430ux44f-ux440ux430ux434ux438ux430ux446ux438ux44f-ux430ux43bux44cux431ux435ux434ux43e}}

Часть приходящей солнечной радиации отражается обратно в космос. Альбедо
-- это отношение интенсивности солнечной радиации, отраженной
поверхностью, к интенсивности этой радиации, падающей на данную
поверхность. Значения альбедо установлены для различных видов
подстилающих поверхностей, но особенно велико альбедо снега, льда и
облаков. Земля как планета имеет довольно высокое альбедо (α = 0,33),
поскольку около 70\% ее поверхности закрыто облаками. Это означает, что
только часть энергии Солнца (1 - α)S₀ участвует в формировании
радиационного равновесия на верхней границе атмосферы.

\hypertarget{ux441ux43eux431ux441ux442ux432ux435ux43dux43dux43eux435-ux438ux437ux43bux443ux447ux435ux43dux438ux435-ux437ux435ux43cux43bux438-ux438-ux43fux440ux43eux442ux438ux432ux43eux438ux437ux43bux443ux447ux435ux43dux438ux435-ux430ux442ux43cux43eux441ux444ux435ux440ux44b}{%
\subsubsection{3.2. Собственное излучение Земли и противоизлучение
атмосферы}\label{ux441ux43eux431ux441ux442ux432ux435ux43dux43dux43eux435-ux438ux437ux43bux443ux447ux435ux43dux438ux435-ux437ux435ux43cux43bux438-ux438-ux43fux440ux43eux442ux438ux432ux43eux438ux437ux43bux443ux447ux435ux43dux438ux435-ux430ux442ux43cux43eux441ux444ux435ux440ux44b}}

Поглощенная солнечная радиация нагревает поверхность Земли, а затем
Земля снова излучает энергию, но уже в инфракрасном (длинноволновом)
диапазоне. Это излучение относится к длинноволновому диапазону
электромагнитного спектра, с длинами волн, превышающими 4 мкм,
преимущественно около 10 мкм {[}источник 1{]}.

Атмосфера, температура которой близка к температуре подстилающей
поверхности, также излучает длинноволновую радиацию. При этом
атмосферные газы, такие как H₂O (водяной пар), CO₂ и O₃, поглощают
инфракрасную радиацию. Водяной пар является главнейшим из парниковых
газов. Одна часть излучения атмосферы возвращается к поверхности Земли
(противоизлучение атмосферы), а другая -- уходит в космос (уходящее
излучение).

Противоизлучение атмосферы уменьшает потерю тепла земной поверхностью и
приводит к возникновению \textbf{парникового эффекта}. Этот эффект
объясняет, почему температура у поверхности Земли в среднем близка к 284
К, тогда как если бы Земля была абсолютно черным телом без атмосферы, ее
температура равновесия составила бы 253 К (с учетом альбедо).

\hypertarget{ux440ux430ux437ux43bux438ux447ux438ux44f-ux43cux435ux436ux434ux443-ux431ux430ux43bux430ux43dux441ux43eux43c-ux43fux43bux430ux43dux435ux442ux44b-ux438-ux43fux43eux432ux435ux440ux445ux43dux43eux441ux442ux438}{%
\subsection{4. Различия между балансом планеты и
поверхности}\label{ux440ux430ux437ux43bux438ux447ux438ux44f-ux43cux435ux436ux434ux443-ux431ux430ux43bux430ux43dux441ux43eux43c-ux43fux43bux430ux43dux435ux442ux44b-ux438-ux43fux43eux432ux435ux440ux445ux43dux43eux441ux442ux438}}

Важно отметить, что хотя Земля как планета в целом находится в состоянии
радиационного равновесия, \textbf{поверхность Земли не находится в
состоянии радиационного равновесия}. Эффективное излучение поверхности
лишь частично компенсирует приток от поглощенной радиации. Избыточная
энергия, полученная поверхностью, расходуется на испарение и контактную
теплопередачу в атмосферу.

Радиационный баланс поверхности Земли имеет суточный и годовой ход. В
суточном ходе он почти всегда положителен в дневное время, так как
поступление солнечной радиации значительно превышает эффективное
излучение, тогда как ночью он отрицателен. В годовом ходе радиационный
баланс положителен примерно до 40° широты обоих полушарий. В высоких
широтах он положителен только во время полярного лета.

\hypertarget{ux432ux43bux438ux44fux43dux438ux435-ux43dux430-ux446ux438ux440ux43aux443ux43bux44fux446ux438ux44e-ux430ux442ux43cux43eux441ux444ux435ux440ux44b}{%
\subsection{5. Влияние на циркуляцию
атмосферы}\label{ux432ux43bux438ux44fux43dux438ux435-ux43dux430-ux446ux438ux440ux43aux443ux43bux44fux446ux438ux44e-ux430ux442ux43cux43eux441ux444ux435ux440ux44b}}

Неравномерное зональное распределение радиационного баланса, связанное
со сферичностью поверхности Земли (более низкие широты получают больше
солнечной энергии, чем умеренные и полярные), является причиной
существования общей циркуляции атмосферы. Избыток энергии образуется в
широтном поясе от 40° ю.ш. до 40° с.ш., где он расходуется в основном на
испарение. Водяной пар переносится в умеренные и полярные широты, где
конденсируется, отдавая скрытую теплоту. Таким образом, атмосферная
циркуляция осуществляет перенос тепла от экватора к полюсам, сглаживая
различия температур. Наибольшие значения радиационного баланса
приходятся на океаны тропической и субтропической зон, что объясняется
распределением облачности и различием альбедо, и именно тропический
океан является поставщиком энергии для атмосферных процессов.

Взаимодействие лучистой энергии Солнца с Земной поверхностью и
атмосферой, а также ее перераспределение через различные процессы
теплообмена и атмосферную циркуляцию, является фундаментом для понимания
погоды и климата {[}источник 1, 313{]}.

ewpage

Уважаемый коллега,

Радиационный баланс является краеугольным камнем в понимании
термического режима подстилающей поверхности и атмосферы, а также
фундаментальным фактором, определяющим динамику атмосферных процессов.
Давайте подробно рассмотрим факторы, определяющие этот баланс, и его
характерные суточный и годовой ход.

\hypertarget{ux43fux43eux43dux44fux442ux438ux435-ux440ux430ux434ux438ux430ux446ux438ux43eux43dux43dux43eux433ux43e-ux431ux430ux43bux430ux43dux441ux430}{%
\subsubsection{1. Понятие радиационного
баланса}\label{ux43fux43eux43dux44fux442ux438ux435-ux440ux430ux434ux438ux430ux446ux438ux43eux43dux43dux43eux433ux43e-ux431ux430ux43bux430ux43dux441ux430}}

Радиационный баланс (\(R\)) представляет собой разность между
поглощенной суммарной солнечной радиацией (\(Q_s\)) и эффективным
излучением (\(E_{эф}\)). Иными словами, это скорость изменения энергии в
единице площади поверхностного слоя. Любое тело, включая земную
поверхность, излучает радиацию, теряя внутреннюю тепловую энергию, в то
время как поглощение энергии приводит к ее нагреванию {[}источник 1{]}.
Собственное излучение Земли относится к длинноволновому диапазону,
преимущественно около 10 мкм {[}источник 1{]}. Важно отметить, что Земля
как планета в целом находится в состоянии радиационного равновесия, где
приходящий от Солнца поток радиации уравновешивается потоком отраженной
коротковолновой и уходящей длинноволновой радиации. Однако сама
поверхность Земли не находится в состоянии радиационного равновесия;
избыточная энергия расходуется на испарение и контактную теплопередачу в
атмосферу.

\hypertarget{ux444ux430ux43aux442ux43eux440ux44b-ux43eux43fux440ux435ux434ux435ux43bux44fux44eux449ux438ux435-ux440ux430ux434ux438ux430ux446ux438ux43eux43dux43dux44bux439-ux431ux430ux43bux430ux43dux441}{%
\subsubsection{2. Факторы, определяющие радиационный
баланс}\label{ux444ux430ux43aux442ux43eux440ux44b-ux43eux43fux440ux435ux434ux435ux43bux44fux44eux449ux438ux435-ux440ux430ux434ux438ux430ux446ux438ux43eux43dux43dux44bux439-ux431ux430ux43bux430ux43dux441}}

Радиационный баланс определяется сложным взаимодействием множества
факторов, связанных как с приходящей, так и с уходящей радиацией:

\hypertarget{ux43fux440ux438ux445ux43eux434ux44fux449ux430ux44f-ux441ux43eux43bux43dux435ux447ux43dux430ux44f-ux440ux430ux434ux438ux430ux446ux438ux44f-ux43aux43eux440ux43eux442ux43aux43eux432ux43eux43bux43dux43eux432ux430ux44f}{%
\paragraph{2.1. Приходящая солнечная радиация
(коротковолновая)}\label{ux43fux440ux438ux445ux43eux434ux44fux449ux430ux44f-ux441ux43eux43bux43dux435ux447ux43dux430ux44f-ux440ux430ux434ux438ux430ux446ux438ux44f-ux43aux43eux440ux43eux442ux43aux43eux432ux43eux43bux43dux43eux432ux430ux44f}}

Это основной источник энергии для Земли. Ее интенсивность и поглощение
зависят от:

\begin{itemize}
\tightlist
\item
  \textbf{Высоты Солнца над горизонтом:} Определяется временем суток,
  сезоном и географической широтой. Чем выше Солнце, тем больше поток
  энергии.
\item
  \textbf{Прозрачности атмосферы:} Атмосферный воздух слабо поглощает
  коротковолновую радиацию {[}источник 1{]}. Однако наличие облаков,
  водяного пара и аэрозолей существенно влияет на ее прохождение.

  \begin{itemize}
  \tightlist
  \item
    \textbf{Облака:} Обладают высоким альбедо (до 70-95\% для мощных
    облаков, 20-70\% для тонких), отражая значительную часть солнечной
    радиации обратно в космос, тем самым уменьшая приток энергии к
    поверхности.
  \item
    \textbf{Водяной пар:} Хотя и не так сильно, как длинноволновую,
    водяной пар также поглощает некоторую часть коротковолновой
    радиации.
  \end{itemize}
\item
  \textbf{Альбедо подстилающей поверхности:} Доля солнечной радиации,
  отраженная поверхностью, является ключевым параметром. Различные
  поверхности имеют разное альбедо:

  \begin{itemize}
  \tightlist
  \item
    \textbf{Снег и лед:} Имеют исключительно высокое альбедо
    (свежевыпавший снег 70-95\%, старый 40-70\%), что приводит к
    минимальному поглощению солнечной энергии.
  \item
    \textbf{Водная поверхность:} Альбедо сильно зависит от угла падения
    солнечных лучей -- низкое при высоком Солнце и высокое при низком
    Солнце.
  \item
    \textbf{Почва и растительность:} Альбедо варьируется в зависимости
    от типа почвы, ее влажности и наличия растительного покрова.
    Растительность способствует охлаждению почвы и уменьшению амплитуды
    температурных колебаний.
  \end{itemize}
\end{itemize}

\hypertarget{ux443ux445ux43eux434ux44fux449ux430ux44f-ux434ux43bux438ux43dux43dux43eux432ux43eux43bux43dux43eux432ux430ux44f-ux440ux430ux434ux438ux430ux446ux438ux44f-ux44dux444ux444ux435ux43aux442ux438ux432ux43dux43eux435-ux438ux437ux43bux443ux447ux435ux43dux438ux435}{%
\paragraph{2.2. Уходящая длинноволновая радиация (эффективное
излучение)}\label{ux443ux445ux43eux434ux44fux449ux430ux44f-ux434ux43bux438ux43dux43dux43eux432ux43eux43bux43dux43eux432ux430ux44f-ux440ux430ux434ux438ux430ux446ux438ux44f-ux44dux444ux444ux435ux43aux442ux438ux432ux43dux43eux435-ux438ux437ux43bux443ux447ux435ux43dux438ux435}}

Земная поверхность, нагретая солнечной радиацией, излучает энергию в
инфракрасном диапазоне. Величина этого излучения зависит от:

\begin{itemize}
\tightlist
\item
  \textbf{Температуры поверхности:} Согласно закону Стефана-Больцмана,
  поток излучения абсолютно черного тела пропорционален четвертой
  степени его абсолютной температуры {[}источник 1{]}.
\item
  \textbf{Излучательной способности поверхности:} ``Абсолютно черное
  тело'' -- это идеализированный излучатель (коэффициент поглощения αλ =
  1). Снег, например, ведет себя как почти абсолютно черное тело в
  инфракрасном диапазоне {[}источник 1, 233{]}. Закон Кирхгофа связывает
  излучательную и поглощательную способности тела {[}источник 1{]}.
\item
  \textbf{Атмосферной противорадиации (парниковый эффект):} Атмосфера
  активно поглощает длинноволновое излучение, испускаемое Землей, в
  основном благодаря парниковым газам, таким как водяной пар. Это
  поглощение приводит к нагреванию атмосферы и возврату части энергии к
  поверхности {[}источник 1, 107, 223, 274{]}. Увеличение концентрации
  парниковых газов, в том числе водяного пара, уменьшает эффективный
  поток излучения и потери тепла земной поверхностью, способствуя
  повышению температуры воздуха {[}источник 1, 186, 193, 274{]}.
\item
  \textbf{Облака:} Помимо отражения коротковолновой радиации, облака
  также активно поглощают и излучают длинноволновую радиацию. Их влияние
  на радиационный баланс двойственно: они уменьшают приток солнечной
  радиации, но также уменьшают потери тепла Землей за счет эффективного
  излучения, действуя как ``одеяло''.
\end{itemize}

\hypertarget{ux442ux435ux43fux43bux43eux444ux438ux437ux438ux447ux435ux441ux43aux438ux435-ux441ux432ux43eux439ux441ux442ux432ux430-ux43fux43eux434ux441ux442ux438ux43bux430ux44eux449ux435ux439-ux43fux43eux432ux435ux440ux445ux43dux43eux441ux442ux438}{%
\paragraph{2.3. Теплофизические свойства подстилающей
поверхности}\label{ux442ux435ux43fux43bux43eux444ux438ux437ux438ux447ux435ux441ux43aux438ux435-ux441ux432ux43eux439ux441ux442ux432ux430-ux43fux43eux434ux441ux442ux438ux43bux430ux44eux449ux435ux439-ux43fux43eux432ux435ux440ux445ux43dux43eux441ux442ux438}}

Эти свойства влияют на то, как быстро поверхность нагревается или
охлаждается, что, в свою очередь, сказывается на ее температуре и,
следовательно, на длинноволновом излучении и теплообмене с атмосферой:

\begin{itemize}
\tightlist
\item
  \textbf{Плотность, удельная теплоемкость и температуропроводность:}
  Определяют скорость установления радиационного равновесия в слое и
  глубину распространения температурных колебаний. Например, вода имеет
  большую удельную теплоемкость, позволяющую ей аккумулировать
  значительное количество тепла {[}предыдущая реплика{]}.
\item
  \textbf{Влагосодержание почвы:} Влияет на альбедо, теплоемкость,
  температуропроводность и испарение {[}предыдущая реплика{]}.
\end{itemize}

\hypertarget{ux441ux443ux442ux43eux447ux43dux44bux439-ux445ux43eux434-ux440ux430ux434ux438ux430ux446ux438ux43eux43dux43dux43eux433ux43e-ux431ux430ux43bux430ux43dux441ux430}{%
\subsubsection{3. Суточный ход радиационного
баланса}\label{ux441ux443ux442ux43eux447ux43dux44bux439-ux445ux43eux434-ux440ux430ux434ux438ux430ux446ux438ux43eux43dux43dux43eux433ux43e-ux431ux430ux43bux430ux43dux441ux430}}

Радиационный баланс демонстрирует четко выраженный суточный ход:

\begin{itemize}
\tightlist
\item
  \textbf{Дневное время:} Почти всегда положителен, так как поступление
  солнечной радиации значительно превышает эффективное излучение. В это
  время поверхность нагревается.
\item
  \textbf{Ночное время:} Отрицателен, поскольку поступление солнечной
  радиации отсутствует, а поверхность продолжает излучать энергию в
  пространство. Это приводит к радиационному охлаждению поверхности и
  прилегающего слоя воздуха.
\item
  \textbf{Переход через нуль:} Приходится на светлое время суток --
  примерно за час до захода Солнца и через час после его восхода.
\end{itemize}

\hypertarget{ux433ux43eux434ux43eux432ux43eux439-ux445ux43eux434-ux440ux430ux434ux438ux430ux446ux438ux43eux43dux43dux43eux433ux43e-ux431ux430ux43bux430ux43dux441ux430}{%
\subsubsection{4. Годовой ход радиационного
баланса}\label{ux433ux43eux434ux43eux432ux43eux439-ux445ux43eux434-ux440ux430ux434ux438ux430ux446ux438ux43eux43dux43dux43eux433ux43e-ux431ux430ux43bux430ux43dux441ux430}}

Годовой ход радиационного баланса сильно зависит от географической
широты и сезона:

\begin{itemize}
\tightlist
\item
  \textbf{Тропические и субтропические зоны (примерно от 40° ю.ш. до 40°
  с.ш.):} Радиационный баланс здесь всегда положителен благодаря
  высокому углу падения солнечных лучей в течение всего года. Избыток
  энергии в этих областях в основном расходуется на испарение, и в форме
  водяного пара переносится в умеренные и полярные широты, где пар
  конденсируется, отдавая содержащееся в нем тепло. Наибольшие значения
  радиационного баланса приходятся на тропические и субтропические
  океаны, которые являются ключевыми поставщиками энергии для
  атмосферных процессов.
\item
  \textbf{Умеренные широты (например, 50° широты):} Радиационный баланс
  положителен в течение теплого полугодия, но отрицателен в течение трех
  зимних месяцев.
\item
  \textbf{Высокие широты/Полярные области:} Радиационный баланс
  положителен только во время полярного лета, когда Солнце не заходит. В
  период полярной ночи он исключительно отрицателен.
\end{itemize}

\hypertarget{ux432ux43bux438ux44fux43dux438ux435-ux440ux430ux434ux438ux430ux446ux438ux43eux43dux43dux43eux433ux43e-ux431ux430ux43bux430ux43dux441ux430-ux43dux430-ux430ux442ux43cux43eux441ux444ux435ux440ux43dux44bux435-ux43fux440ux43eux446ux435ux441ux441ux44b}{%
\subsubsection{5. Влияние радиационного баланса на атмосферные
процессы}\label{ux432ux43bux438ux44fux43dux438ux435-ux440ux430ux434ux438ux430ux446ux438ux43eux43dux43dux43eux433ux43e-ux431ux430ux43bux430ux43dux441ux430-ux43dux430-ux430ux442ux43cux43eux441ux444ux435ux440ux43dux44bux435-ux43fux440ux43eux446ux435ux441ux441ux44b}}

Неравномерное распределение радиационного баланса по земному шару,
обусловленное сферичностью Земли и неодинаковым нагревом различных
участков, является главной причиной существования общей циркуляции
атмосферы. Эти горизонтальные разности температуры и давления порождают
вихревые движения и крупномасштабные воздушные потоки. Различия в
теплофизических свойствах между сушей и океаном также приводят к
сезонным изменениям температуры и адвекции тепла/холода, что
способствует зарождению и развитию синоптических вихрей, таких как
циклоны и антициклоны. Например, зимой над океанами преобладает адвекция
холода, способствующая циклогенезу, а над материками -- адвекция тепла,
благоприятствующая антициклогенезу. Облачность, регулируя радиационный
баланс, оказывает значительное влияние на термический и влажностный
режим деятельного слоя почвы и атмосферы.

Таким образом, радиационный баланс -- это динамический параметр,
формируемый комплексным воздействием астрономических, физических и
географических факторов, и его изменения в пространстве и времени
являются ключевыми драйверами всех атмосферных процессов, от локального
теплообмена до глобальной циркуляции.

ewpage

\hypertarget{ux431ux43bux43eux43a-1.-ux444ux438ux437ux438ux43aux430-ux430ux442ux43cux43eux441ux444ux435ux440ux44b-3}{%
\section{Блок 1. «Физика
атмосферы»}\label{ux431ux43bux43eux43a-1.-ux444ux438ux437ux438ux43aux430-ux430ux442ux43cux43eux441ux444ux435ux440ux44b-3}}

\hypertarget{ux442ux435ux43cux430-ux442ux435ux43fux43bux43eux432ux43eux439-ux440ux435ux436ux438ux43c-ux43fux43eux447ux432ux44b-ux438-ux430ux442ux43cux43eux441ux444ux435ux440ux44b}{%
\subsection{1.4. Тема «Тепловой режим почвы и
атмосферы»}\label{ux442ux435ux43cux430-ux442ux435ux43fux43bux43eux432ux43eux439-ux440ux435ux436ux438ux43c-ux43fux43eux447ux432ux44b-ux438-ux430ux442ux43cux43eux441ux444ux435ux440ux44b}}

\hypertarget{ux442ux435ux43fux43bux43eux444ux438ux437ux438ux447ux435ux441ux43aux438ux435-ux445ux430ux440ux430ux43aux442ux435ux440ux438ux441ux442ux438ux43aux438-ux43fux43eux447ux432ux44b-ux432ux43eux434ux44b-ux438-ux432ux43eux437ux434ux443ux445ux430}{%
\subsubsection{\texorpdfstring{\textbf{Теплофизические характеристики
почвы, воды и
воздуха}}{Теплофизические характеристики почвы, воды и воздуха}}\label{ux442ux435ux43fux43bux43eux444ux438ux437ux438ux447ux435ux441ux43aux438ux435-ux445ux430ux440ux430ux43aux442ux435ux440ux438ux441ux442ux438ux43aux438-ux43fux43eux447ux432ux44b-ux432ux43eux434ux44b-ux438-ux432ux43eux437ux434ux443ux445ux430}}

Тепловой режим любой среды определяется её способностью поглощать,
хранить и передавать тепло. Ключевыми характеристиками являются:

\begin{itemize}
\tightlist
\item
  \textbf{Удельная теплоёмкость (\(c\)):} Количество тепла, необходимое
  для нагрева единицы массы вещества на 1 К. Размерность: Дж·кг⁻¹·К⁻¹.
\item
  \textbf{Теплопроводность (\(\lambda\)):} Способность вещества
  проводить тепло посредством молекулярного движения. Размерность:
  Вт·м⁻¹·К⁻¹.
\item
  \textbf{Температуропроводность (\(k\)):} Характеризует скорость
  выравнивания температуры в веществе. \(k = \lambda / (\rho c)\).
  Размерность: м²·с⁻¹.
\end{itemize}

\begin{longtable}[]{@{}
  >{\raggedright\arraybackslash}p{(\columnwidth - 6\tabcolsep) * \real{0.2500}}
  >{\raggedright\arraybackslash}p{(\columnwidth - 6\tabcolsep) * \real{0.2500}}
  >{\raggedright\arraybackslash}p{(\columnwidth - 6\tabcolsep) * \real{0.2500}}
  >{\raggedright\arraybackslash}p{(\columnwidth - 6\tabcolsep) * \real{0.2500}}@{}}
\toprule\noalign{}
\begin{minipage}[b]{\linewidth}\raggedright
Среда
\end{minipage} & \begin{minipage}[b]{\linewidth}\raggedright
Удельная теплоёмкость, \(c\) (Дж·кг⁻¹·К⁻¹)
\end{minipage} & \begin{minipage}[b]{\linewidth}\raggedright
Теплопроводность, \(\lambda\) (Вт·м⁻¹·К⁻¹)
\end{minipage} & \begin{minipage}[b]{\linewidth}\raggedright
Температуропроводность, \(k\) (10⁻⁶ м²·с⁻¹)
\end{minipage} \\
\midrule\noalign{}
\endhead
\bottomrule\noalign{}
\endlastfoot
\textbf{Воздух (сухой)} & \(\approx 1005\) (\(c_p\)) & \(\approx 0.025\)
& \(\approx 20\) \\
\textbf{Вода (чистая)} & \(\approx 4200\) & \(\approx 0.6\) &
\(\approx 0.14\) \\
\textbf{Почва (сухая, песок)} & \(\approx 800\) & \(\approx 0.3\) &
\(\approx 0.2\) \\
\textbf{Почва (влажная)} & \(\approx 1500\) & \(\approx 1.5\) &
\(\approx 0.7\) \\
\textbf{Снег (свежий)} & \(\approx 2100\) & \(\approx 0.1\) &
\(\approx 0.3\) \\
\end{longtable}

\textbf{Метеорологическое значение:}

\begin{itemize}
\tightlist
\item
  \textbf{Вода vs.~Суша:} Вода обладает огромной теплоёмкостью (примерно
  в 4-5 раз больше, чем сухая почва). Это означает, что для нагрева 1 кг
  воды на 1°C требуется в 4-5 раз больше энергии. Это фундаментальная
  причина \textbf{континентального} и \textbf{морского} типов климата.
  Океаны медленно нагреваются и медленно остывают, сглаживая годовой ход
  температуры, в то время как континенты испытывают резкие сезонные и
  суточные колебания.
\item
  \textbf{Воздух:} Воздух имеет очень низкую теплопроводность, он
  является хорошим теплоизолятором. Поэтому молекулярный перенос тепла в
  атмосфере играет роль только в очень тонком (миллиметровом) слое у
  самой поверхности.
\item
  \textbf{Снег:} Низкая теплопроводность снега делает его отличным
  изолятором, защищающим почву от глубокого промерзания зимой.
\end{itemize}

\hypertarget{ux440ux430ux441ux43fux440ux43eux441ux442ux440ux430ux43dux435ux43dux438ux435-ux442ux435ux43fux43bux430-ux432-ux43fux43eux447ux432ux435-ux438-ux432ux43eux434ux43eux435ux43cux430ux445.-ux43fux43eux442ux43eux43a-ux442ux435ux43fux43bux430-ux432-ux43fux43eux447ux432ux435}{%
\subsubsection{\texorpdfstring{\textbf{Распространение тепла в почве и
водоемах. Поток тепла в
почве}}{Распространение тепла в почве и водоемах. Поток тепла в почве}}\label{ux440ux430ux441ux43fux440ux43eux441ux442ux440ux430ux43dux435ux43dux438ux435-ux442ux435ux43fux43bux430-ux432-ux43fux43eux447ux432ux435-ux438-ux432ux43eux434ux43eux435ux43cux430ux445.-ux43fux43eux442ux43eux43a-ux442ux435ux43fux43bux430-ux432-ux43fux43eux447ux432ux435}}

\begin{itemize}
\item
  \textbf{В почве:} Основной механизм переноса тепла ---
  \textbf{молекулярная теплопроводность}. Процесс описывается
  \textbf{уравнением теплопроводности}: \[
    \frac{\partial T}{\partial t} = k \frac{\partial^2 T}{\partial z^2}
    \] где \(k\) --- температуропроводность почвы. Суточный и годовой
  ход радиационного баланса на поверхности создают \textbf{температурные
  волны}, которые распространяются вглубь почвы. Решение этого уравнения
  показывает два ключевых эффекта:

  \begin{enumerate}
  \def\labelenumi{\arabic{enumi}.}
  \tightlist
  \item
    \textbf{Затухание амплитуды:} Амплитуда температурных колебаний
    убывает с глубиной экспоненциально. Суточные колебания практически
    исчезают на глубине нескольких десятков сантиметров, а годовые ---
    на глубине нескольких метров (слой постоянных годовых температур).
  \item
    \textbf{Запаздывание фазы:} Время наступления максимума и минимума
    температуры запаздывает с глубиной. Например, максимум температуры
    на глубине 20 см может наблюдаться вечером, а минимум годовой
    температуры на глубине нескольких метров --- в начале лета.
  \end{enumerate}
\item
  \textbf{В водоемах:} Молекулярная теплопроводность играет
  незначительную роль. Основной механизм переноса тепла ---
  \textbf{перемешивание (конвективное и ветровое)}. Турбулентность
  переносит тепло гораздо эффективнее, поэтому суточные и годовые
  колебания температуры проникают на десятки и сотни метров. Это
  приводит к формированию \textbf{верхнего квазиоднородного слоя (ВКС)},
  под которым располагается \textbf{термоклин} --- слой резкого падения
  температуры.
\item
  \textbf{Поток тепла в почве (\(G\)):} Направлен от более тёплых слоёв
  к более холодным и описывается \textbf{законом Фурье}: \[
    G = -\lambda \frac{\partial T}{\partial z}
    \] Днём, когда поверхность теплее, поток направлен вглубь почвы
  (\(G<0\)). Ночью, когда поверхность охлаждается, поток направлен из
  глубины к поверхности (\(G>0\)).
\end{itemize}

\hypertarget{ux443ux440ux430ux432ux43dux435ux43dux438ux435-ux43fux440ux438ux442ux43eux43aux430-ux442ux435ux43fux43bux430-ux432-ux430ux442ux43cux43eux441ux444ux435ux440ux435.-ux442ux443ux440ux431ux443ux43bux435ux43dux442ux43dux44bux439-ux43eux431ux43cux435ux43d}{%
\subsubsection{\texorpdfstring{\textbf{Уравнение притока тепла в
атмосфере. Турбулентный
обмен}}{Уравнение притока тепла в атмосфере. Турбулентный обмен}}\label{ux443ux440ux430ux432ux43dux435ux43dux438ux435-ux43fux440ux438ux442ux43eux43aux430-ux442ux435ux43fux43bux430-ux432-ux430ux442ux43cux43eux441ux444ux435ux440ux435.-ux442ux443ux440ux431ux443ux43bux435ux43dux442ux43dux44bux439-ux43eux431ux43cux435ux43d}}

Полное изменение температуры воздушной частицы (Лагранжев подход)
описывается \textbf{первым началом термодинамики} в форме уравнения
притока тепла: \[
c_p \frac{dT}{dt} - \alpha \frac{dp}{dt} = \dot{Q}
\] где \(\frac{dT}{dt}\) и \(\frac{dp}{dt}\) --- полные производные
температуры и давления для движущейся частицы, \(\alpha = 1/\rho\) ---
удельный объём, а \(\dot{Q}\) --- скорость неадиабатических
(диабатических) притоков тепла на единицу массы (Вт/кг).

Для анализа изменений в фиксированной точке (Эйлеров подход), необходимо
развернуть полные производные. Используя оператор
\(\frac{d}{dt} = \frac{\partial}{\partial t} + \vec{V} \cdot \nabla\),
получаем \textbf{уравнение притока тепла в локальной форме}: \[
\frac{\partial T}{\partial t} = - \vec{V}_h \cdot \nabla_h T - w \left( \frac{\partial T}{\partial z} + \gamma_d \right) + \frac{\dot{Q}}{c_p}
\] Здесь:

\begin{itemize}
\tightlist
\item
  \(\frac{\partial T}{\partial t}\) --- \textbf{локальное изменение
  температуры} в фиксированной точке.
\item
  \(- \vec{V}_h \cdot \nabla_h T = - (u\frac{\partial T}{\partial x} + v\frac{\partial T}{\partial y})\)
  --- \textbf{горизонтальная адвекция температуры}, описывающая
  изменение температуры за счёт переноса воздуха ветром из более тёплых
  или холодных областей.
\item
  \(- w \left( \frac{\partial T}{\partial z} + \gamma_d \right)\) ---
  \textbf{эффект вертикальных движений}, который объединяет два
  процесса:

  \begin{itemize}
  \tightlist
  \item
    \(-w \frac{\partial T}{\partial z}\) --- \textbf{вертикальная
    адвекция температуры} (перенос воздуха в слои с другой
    температурой).
  \item
    \(-w \gamma_d = -w \frac{g}{c_p}\) --- \textbf{адиабатический
    нагрев/охлаждение} из-за сжатия (при \(w<0\)) или расширения (при
    \(w>0\)) воздуха.
  \end{itemize}
\item
  \(\frac{\dot{Q}}{c_p}\) --- \textbf{диабатический нагрев}, включающий
  радиационные потоки, фазовые переходы и \textbf{турбулентный обмен}.
\end{itemize}

\textbf{Турбулентный обмен} --- это хаотический перенос тепла, влаги и
импульса за счёт турбулентных вихрей. Он является доминирующим
механизмом обмена между поверхностью и атмосферой в пограничном слое.

\hypertarget{ux43aux43eux44dux444ux444ux438ux446ux438ux435ux43dux442-ux442ux443ux440ux431ux443ux43bux435ux43dux442ux43dux43eux441ux442ux438-ux43cux435ux442ux43eux434ux44b-ux43eux43fux440ux435ux434ux435ux43bux435ux43dux438ux44f-ux438-ux440ux430ux441ux447ux435ux442ux430-ux442ux443ux440ux431ux443ux43bux435ux43dux442ux43dux43eux433ux43e-ux43fux43eux442ux43eux43aux430-ux442ux435ux43fux43bux430}{%
\subsubsection{\texorpdfstring{\textbf{Коэффициент турбулентности:
методы определения и расчета турбулентного потока
тепла}}{Коэффициент турбулентности: методы определения и расчета турбулентного потока тепла}}\label{ux43aux43eux44dux444ux444ux438ux446ux438ux435ux43dux442-ux442ux443ux440ux431ux443ux43bux435ux43dux442ux43dux43eux441ux442ux438-ux43cux435ux442ux43eux434ux44b-ux43eux43fux440ux435ux434ux435ux43bux435ux43dux438ux44f-ux438-ux440ux430ux441ux447ux435ux442ux430-ux442ux443ux440ux431ux443ux43bux435ux43dux442ux43dux43eux433ux43e-ux43fux43eux442ux43eux43aux430-ux442ux435ux43fux43bux430}}

В классической К-теории (теории градиентной диффузии) турбулентный поток
тепла (\(H\)) по аналогии с законом Фурье представляется как: \[
H = -c_p \rho K_H \frac{\partial \theta}{\partial z}
\] где \(K_H\) --- \textbf{коэффициент турбулентной
температуропроводности}. В отличие от молекулярного коэффициента,
\(K_H\) не является свойством среды, а зависит от самого потока: его
интенсивности, сдвига ветра и, главное, статической устойчивости.

Устойчивость характеризуется безразмерным \textbf{числом Ричардсона
(\(Ri\))}: \[
Ri = \frac{\frac{g}{\theta}\frac{\partial\theta}{\partial z}}{(\frac{\partial u}{\partial z})^2}
\]

\begin{itemize}
\tightlist
\item
  При \(Ri < 0\) (неустойчивая стратификация), турбулентность
  усиливается за счёт плавучести.
\item
  При \(Ri > 0\) (устойчивая стратификация), турбулентность подавляется.
\item
  При \(Ri > Ri_{cr} \approx 0.25\) (критическое число Ричардсона),
  турбулентность полностью затухает.
\end{itemize}

Для практических расчётов потока тепла используются
\textbf{аэродинамические формулы}: \[
H = c_p \rho C_H U (\theta_s - \theta_a)
\] где \(C_H\) --- коэффициент теплообмена, \(U\) --- скорость ветра,
\(\theta_s\) и \(\theta_a\) --- потенциальные температуры поверхности и
воздуха.

\hypertarget{ux441ux443ux442ux43eux447ux43dux44bux439-ux438-ux433ux43eux434ux43eux432ux43eux439-ux445ux43eux434-ux442ux435ux43cux43fux435ux440ux430ux442ux443ux440ux44b-ux432ux43eux437ux434ux443ux445ux430}{%
\subsubsection{\texorpdfstring{\textbf{Суточный и годовой ход
температуры
воздуха}}{Суточный и годовой ход температуры воздуха}}\label{ux441ux443ux442ux43eux447ux43dux44bux439-ux438-ux433ux43eux434ux43eux432ux43eux439-ux445ux43eux434-ux442ux435ux43cux43fux435ux440ux430ux442ux443ux440ux44b-ux432ux43eux437ux434ux443ux445ux430}}

\begin{itemize}
\tightlist
\item
  \textbf{Суточный ход:} Определяется балансом между притоком солнечной
  радиации и потерей тепла за счёт длинноволнового излучения и
  турбулентных потоков. \textbf{Минимум температуры} наблюдается обычно
  сразу после восхода Солнца, когда поверхность всю ночь охлаждалась, а
  приток солнечной энергии ещё не компенсировал потери. \textbf{Максимум
  температуры} наблюдается через 2-3 часа после полудня. Этот
  \textbf{лаг} связан с \textbf{тепловой инерцией} системы: требуется
  время, чтобы избыток радиации прогрел почву и воздух. Амплитуда
  суточного хода максимальна при ясном небе и слабом ветре над сушей и
  минимальна при облачной, ветреной погоде над океаном.
\item
  \textbf{Годовой ход:} Также имеет лаг относительно солнцестояний.
  Самый холодный месяц в умеренных широтах --- январь (а не декабрь), а
  самый тёплый --- июль (а не июнь).
\end{itemize}

\hypertarget{ux438ux437ux43cux435ux43dux435ux43dux438ux435-ux442ux435ux43cux43fux435ux440ux430ux442ux443ux440ux44b-ux432ux43eux437ux434ux443ux445ux430-ux441-ux432ux44bux441ux43eux442ux43eux439.-ux438ux43dux432ux435ux440ux441ux438ux438-ux442ux435ux43cux43fux435ux440ux430ux442ux443ux440ux44b.-ux432ux44bux441ux43eux442ux430-ux438-ux442ux435ux43cux43fux435ux440ux430ux442ux443ux440ux430-ux442ux440ux43eux43fux43eux43fux430ux443ux437ux44b}{%
\subsubsection{\texorpdfstring{\textbf{Изменение температуры воздуха с
высотой. Инверсии температуры. Высота и температура
тропопаузы}}{Изменение температуры воздуха с высотой. Инверсии температуры. Высота и температура тропопаузы}}\label{ux438ux437ux43cux435ux43dux435ux43dux438ux435-ux442ux435ux43cux43fux435ux440ux430ux442ux443ux440ux44b-ux432ux43eux437ux434ux443ux445ux430-ux441-ux432ux44bux441ux43eux442ux43eux439.-ux438ux43dux432ux435ux440ux441ux438ux438-ux442ux435ux43cux43fux435ux440ux430ux442ux443ux440ux44b.-ux432ux44bux441ux43eux442ux430-ux438-ux442ux435ux43cux43fux435ux440ux430ux442ux443ux440ux430-ux442ux440ux43eux43fux43eux43fux430ux443ux437ux44b}}

\begin{itemize}
\tightlist
\item
  \textbf{Инверсии температуры:} Слои, где температура растёт с высотой
  (\(\gamma < 0\)). Это зоны повышенной статической устойчивости,
  которые ``запирают'' турбулентность и загрязнения под собой.

  \begin{itemize}
  \tightlist
  \item
    \textbf{Радиационная инверсия:} Формируется у поверхности ночью при
    ясном небе из-за сильного радиационного охлаждения земли.
  \item
    \textbf{Инверсия оседания:} Формируется в антициклонах из-за
    медленного опускания и адиабатического сжатия воздуха.
    \textbf{Пассатная инверсия} в субтропиках --- яркий пример.
  \item
    \textbf{Фронтальная инверсия:} Наблюдается в зоне тёплого фронта,
    где тёплый воздух натекает на холодный.
  \end{itemize}
\item
  \textbf{Тропопауза:} Верхняя граница тропосферы, часто представляет
  собой инверсионный или изотермический слой. Её \textbf{высота}
  максимальна в тропиках (16-18 км) и минимальна у полюсов (8-10 км).
  При этом \textbf{температура тропопаузы} минимальна в тропиках (до
  -80°C) и выше в полярных широтах (-50\ldots-60°C). Этот парадокс
  объясняется интенсивным конвективным подъёмом и адиабатическим
  охлаждением воздуха до больших высот в тропиках.
\end{itemize}

\hypertarget{ux443ux440ux430ux432ux43dux435ux43dux438ux435-ux442ux435ux43fux43bux43eux432ux43eux433ux43e-ux431ux430ux43bux430ux43dux441ux430-ux437ux435ux43cux43dux43eux439-ux43fux43eux432ux435ux440ux445ux43dux43eux441ux442ux438-ux430ux442ux43cux43eux441ux444ux435ux440ux44b-ux438-ux441ux438ux441ux442ux435ux43cux44b-ux437ux435ux43cux43bux44f-ux430ux442ux43cux43eux441ux444ux435ux440ux430}{%
\subsubsection{\texorpdfstring{\textbf{Уравнение теплового баланса
земной поверхности, атмосферы и системы
Земля-атмосфера}}{Уравнение теплового баланса земной поверхности, атмосферы и системы Земля-атмосфера}}\label{ux443ux440ux430ux432ux43dux435ux43dux438ux435-ux442ux435ux43fux43bux43eux432ux43eux433ux43e-ux431ux430ux43bux430ux43dux441ux430-ux437ux435ux43cux43dux43eux439-ux43fux43eux432ux435ux440ux445ux43dux43eux441ux442ux438-ux430ux442ux43cux43eux441ux444ux435ux440ux44b-ux438-ux441ux438ux441ux442ux435ux43cux44b-ux437ux435ux43cux43bux44f-ux430ux442ux43cux43eux441ux444ux435ux440ux430}}

\begin{itemize}
\tightlist
\item
  \textbf{Тепловой баланс земной поверхности:} Радиационный баланс
  (\(R_n\)) уравновешивается нерадиационными потоками тепла: \[
    R_n = H + LE + G
    \] где \(H\) --- турбулентный поток явного тепла, \(LE\) ---
  турбулентный поток скрытого тепла (затраты на испарение), \(G\) ---
  поток тепла в почву/воду. Это уравнение определяет температуру
  подстилающей поверхности.
\item
  \textbf{Тепловой баланс атмосферы:} Атмосфера нагревается за счёт
  поглощения солнечной и длинноволновой радиации, выделения скрытого
  тепла при конденсации и турбулентного потока тепла от поверхности.
  Охлаждается она в основном за счёт собственного длинноволнового
  излучения в космос и к поверхности.
\item
  \textbf{Тепловой баланс системы Земля-атмосфера:} На верхней границе
  атмосферы приходящая солнечная радиация (за вычетом отражённой)
  уравновешивается уходящим длинноволновым излучением. Широтный
  дисбаланс этой энергии является основной движущей силой климатической
  системы.
\end{itemize}

ewpage

\hypertarget{ux442ux435ux43fux43bux43eux444ux438ux437ux438ux447ux435ux441ux43aux438ux435-ux445ux430ux440ux430ux43aux442ux435ux440ux438ux441ux442ux438ux43aux438-ux43fux43eux447ux432ux44b-ux432ux43eux434ux44b-ux438-ux432ux43eux437ux434ux443ux445ux430-1}{%
\section{Теплофизические характеристики почвы, воды и
воздуха}\label{ux442ux435ux43fux43bux43eux444ux438ux437ux438ux447ux435ux441ux43aux438ux435-ux445ux430ux440ux430ux43aux442ux435ux440ux438ux441ux442ux438ux43aux438-ux43fux43eux447ux432ux44b-ux432ux43eux434ux44b-ux438-ux432ux43eux437ux434ux443ux445ux430-1}}

При рассмотрении атмосферных процессов, определяющих погоду и климат,
фундаментальное значение имеют теплофизические характеристики
подстилающей поверхности, в частности почвы и воды, а также самого
атмосферного воздуха. Эти характеристики определяют процессы тепло- и
влагообмена между различными средами, что, в свою очередь, влияет на
атмосферные движения и термодинамические процессы.

\hypertarget{ux432ux43eux437ux434ux443ux445}{%
\subsection{Воздух}\label{ux432ux43eux437ux434ux443ux445}}

Воздух представляет собой механическую смесь газов, преимущественно
азота (78,084\% по массе в нижних десятках километров сухого чистого
воздуха), кислорода (20,946\%), аргона (0,934\%) и диоксида углерода
(0,033\%). В реальных атмосферных условиях к ним добавляются переменные
компоненты, такие как водяной пар и различные аэрозоли (твердые и жидкие
взвешенные частицы). Для практических расчетов смесь газов, составляющих
сухой чистый воздух, можно рассматривать как один идеальный газ.

\hypertarget{ux443ux440ux430ux432ux43dux435ux43dux438ux435-ux441ux43eux441ux442ux43eux44fux43dux438ux44f}{%
\subsubsection{Уравнение
состояния}\label{ux443ux440ux430ux432ux43dux435ux43dux438ux435-ux441ux43eux441ux442ux43eux44fux43dux438ux44f}}

Состояние атмосферного воздуха как термодинамической системы
характеризуется плотностью (\(\rho\)), давлением (\(P\)), и абсолютной
температурой (\(T\)). Эти параметры связаны уравнением состояния
идеального газа: \(P = \rho RT\) где \(R\) --- удельная газовая
постоянная сухого воздуха, равная 287,1 Дж/(кг·К). Для влажного воздуха
справедливо аналогичное уравнение, однако удельная газовая постоянная
влажного воздуха отличается от постоянной сухого воздуха. Чтобы
сохранить константу \(R\) для сухого воздуха, вводится понятие
виртуальной температуры \(T_v\). \(T_v = T(1 + 0,608q)\) или
\(T_v = T(1 + 0,61q)\), где \(q\) --- удельная влажность. Таким образом,
уравнение состояния для влажного воздуха может быть записано как
\(P = \rho R T_v\). Виртуальная температура влажного воздуха выше
действительной, что означает, что влажный воздух легче сухого при том же
давлении и температуре.

\hypertarget{ux442ux435ux43fux43bux43eux435ux43cux43aux43eux441ux442ux44c}{%
\subsubsection{Теплоемкость}\label{ux442ux435ux43fux43bux43eux435ux43cux43aux43eux441ux442ux44c}}

Внутренняя энергия идеального газа пропорциональна его абсолютной
температуре, и ее изменение выражается как \(dJ = c_v dT\), где \(c_v\)
--- удельная теплоемкость при постоянном объеме. Для сухого воздуха
\(c_v \approx 718\) Дж/(кг·К) или 719 Дж/(кг·К). Удельная теплоемкость
воздуха при постоянном давлении \(c_p = 1007\) Дж/(кг·К), она связана с
\(c_v\) и газовой постоянной воздуха \(R\) соотношением Майера. Эти
теплоемкости являются ключевыми для описания притоков тепла в атмосферу.

\hypertarget{ux430ux434ux438ux430ux431ux430ux442ux438ux447ux435ux441ux43aux438ux435-ux43fux440ux43eux446ux435ux441ux441ux44b-1}{%
\subsubsection{Адиабатические
процессы}\label{ux430ux434ux438ux430ux431ux430ux442ux438ux447ux435ux441ux43aux438ux435-ux43fux440ux43eux446ux435ux441ux441ux44b-1}}

Важными теплофизическими характеристиками являются адиабатические
градиенты температуры.

\begin{itemize}
\tightlist
\item
  \textbf{Сухоадиабатический градиент (\(\gamma_a\))}: изменение
  температуры порции сухого воздуха при адиабатическом перемещении ее по
  вертикали на единицу расстояния. Для сухого воздуха
  \(\gamma_a \approx 0,98\) °С/100 м, или -0,0098 °С/м.
\item
  \textbf{Влажноадиабатический градиент (\(\gamma_{ва}\))}: величина, на
  которую изменяется температура порции влажного насыщенного воздуха при
  адиабатическом перемещении ее по вертикали на единицу расстояния,
  сопровождающемся конденсацией водяного пара и выделением скрытой
  теплоты конденсации. Влажноадиабатический градиент значительно меньше
  сухоадиабатического в теплом воздухе, где насыщающая влажность велика,
  и различия уменьшаются в холодном воздухе.
\end{itemize}

\hypertarget{ux43bux443ux447ux438ux441ux442ux44bux439-ux442ux435ux43fux43bux43eux43eux431ux43cux435ux43d}{%
\subsubsection{Лучистый
теплообмен}\label{ux43bux443ux447ux438ux441ux442ux44bux439-ux442ux435ux43fux43bux43eux43eux431ux43cux435ux43d}}

Атмосферный воздух поглощает и излучает электромагнитные волны. Водяной
пар (Н2О), углекислый газ (СО2) и озон (О3) являются основными
поглотителями инфракрасной радиации. В видимой области спектра солнечной
радиации атмосферные газы не поглощают энергию. Антропогенные примеси,
особенно в больших городах, также сильно поглощают радиацию, в первую
очередь в инфракрасном диапазоне, что оказывает обратное влияние на
состояние атмосферы. Дополнительное количество водяного пара,
образующееся при сжигании различных видов топлива, играет важную роль в
изменении радиационного и термического режима города.

\hypertarget{ux432ux43bux430ux436ux43dux43eux441ux442ux44c-ux432ux43eux437ux434ux443ux445ux430}{%
\subsubsection{Влажность
воздуха}\label{ux432ux43bux430ux436ux43dux43eux441ux442ux44c-ux432ux43eux437ux434ux443ux445ux430}}

Влажность воздуха описывает содержание водяного пара. Основные
характеристики включают парциальное давление водяного пара (\(e\)),
абсолютную влажность (\(\rho_n\)), массовую долю водяного пара (\(q\)),
относительную влажность (\(f\)), температуру точки росы (\(t_d\)), и
дефицит насыщения. Массовая доля водяного пара, она же удельная
влажность, определяется как отношение плотности водяного пара к
плотности влажного воздуха. Точка росы --- это температура, при которой
воздух достигает состояния насыщения при данном содержании водяного пара
и неизменном давлении. Чем больше разность между температурой воздуха и
точкой росы, тем суше воздух.

\hypertarget{ux432ux43eux434ux430}{%
\subsection{Вода}\label{ux432ux43eux434ux430}}

Вода является уникальным веществом с аномальными физическими
характеристиками, что определяется особенностями строения ее молекул и
наличием водородных связей.

\hypertarget{ux444ux438ux437ux438ux447ux435ux441ux43aux438ux435-ux445ux430ux440ux430ux43aux442ux435ux440ux438ux441ux442ux438ux43aux438}{%
\subsubsection{Физические
характеристики}\label{ux444ux438ux437ux438ux447ux435ux441ux43aux438ux435-ux445ux430ux440ux430ux43aux442ux435ux440ux438ux441ux442ux438ux43aux438}}

\begin{itemize}
\tightlist
\item
  \textbf{Плотность}: Плотность льда при 0°C составляет 917 кг/м³, а
  воды --- 1000 кг/м³.
\item
  \textbf{Теплоемкость}: Удельная теплоемкость воды значительно выше,
  чем у воздуха и льда. Для жидкой воды она составляет 4190 Дж/(кг·К)
  или 4218 Дж/(кг·К), для льда (снега) --- 2137 Дж/(кг·К) или 2110
  Дж/(кг·К). Большая теплоемкость воды влияет на тепловой режим
  водоемов, что позволяет суточным температурным колебаниям
  распространяться на большие глубины по сравнению с почвой.
\item
  \textbf{Скрытая теплота фазовых переходов}:

  \begin{itemize}
  \tightlist
  \item
    Плавление льда: 0,33 × 10⁶ Дж/кг или 3,34 × 10⁵ Дж/кг.
  \item
    Испарение/конденсация: 2,5 × 10⁶ Дж/кг или 2,50 × 10⁶ Дж/кг.
    Выделение скрытой теплоты при конденсации существенно замедляет
    понижение температуры поднимающегося влажного воздуха.
  \item
    Сублимация (переход из твердого в газообразное состояние): 2,83 ×
    10⁶ Дж/кг. Эти значительные скрытые теплоты влияют на тепловой
    баланс атмосферы и планеты в целом.
  \end{itemize}
\end{itemize}

\hypertarget{ux432ux43bux430ux433ux43eux43eux431ux43cux435ux43d}{%
\subsubsection{Влагообмен}\label{ux432ux43bux430ux433ux43eux43eux431ux43cux435ux43d}}

Испарение --- это процесс перехода воды из жидкого или твердого
состояния в газообразное. Скорость испарения зависит от температуры
воздуха и испаряющей поверхности, типа поверхности, дефицита влажности,
атмосферного давления, скорости ветра и других факторов. Испарение
охлаждает подстилающую поверхность.

Конденсация происходит, когда водяной пар достигает состояния насыщения
и образует капли жидкой воды на поверхностях атмосферных ядер
конденсации. Размеры этих ядер варьируются от до 0,2 мкм (ядра Айткена)
до более 1 мкм (гигантские ядра).

\hypertarget{ux43bux443ux447ux438ux441ux442ux44bux435-ux441ux432ux43eux439ux441ux442ux432ux430}{%
\subsubsection{Лучистые
свойства}\label{ux43bux443ux447ux438ux441ux442ux44bux435-ux441ux432ux43eux439ux441ux442ux432ux430}}

Вода, лед и снег хорошо пропускают солнечную радиацию, но слои толщиной
1-2 мм уже полностью поглощают длинноволновое излучение.

\hypertarget{ux43fux43eux447ux432ux430-1}{%
\subsection{Почва}\label{ux43fux43eux447ux432ux430-1}}

Почва активно участвует в тепло- и влагообмене с атмосферой, что
является одним из основных климатообразующих процессов.

\hypertarget{ux442ux435ux43fux43bux43eux432ux43eux439-ux440ux435ux436ux438ux43c}{%
\subsubsection{Тепловой
режим}\label{ux442ux435ux43fux43bux43eux432ux43eux439-ux440ux435ux436ux438ux43c}}

Температура почвы подвержена суточным и годовым колебаниям. Суточные
колебания температуры проникают на меньшие глубины (порядка десятков
сантиметров), чем годовые (порядка метров). Глубина проникновения
колебаний зависит от теплофизических свойств почвы, таких как
теплоемкость и теплопроводность, а также от ее влажности: во влажной
почве колебания распространяются на большую глубину, чем в засушливой.

\hypertarget{ux432ux43bux438ux44fux43dux438ux435-ux43dux430-ux43fux440ux438ux437ux435ux43cux43dux44bux439-ux441ux43bux43eux439-ux430ux442ux43cux43eux441ux444ux435ux440ux44b}{%
\subsubsection{Влияние на приземный слой
атмосферы}\label{ux432ux43bux438ux44fux43dux438ux435-ux43dux430-ux43fux440ux438ux437ux435ux43cux43dux44bux439-ux441ux43bux43eux439-ux430ux442ux43cux43eux441ux444ux435ux440ux44b}}

Теплофизические свойства подстилающей поверхности, включая почву,
определяют термическую неоднородность прилегающего слоя атмосферы.
Вследствие различий в теплофизических свойствах и механизмах
распространения тепла, температура поверхности суши зимой может быть
ниже температуры поверхности океана, а летом --- наоборот. Через
турбулентный обмен явное и скрытое тепло переносится от поверхности в
атмосферу, создавая разность температур между воздушными массами.
Изменение характера подстилающей поверхности, например, в больших
городах, влияет на скорость испарения воды и способствует формированию
``острова тепла'' и изменению влажностного режима.

\hypertarget{ux448ux435ux440ux43eux445ux43eux432ux430ux442ux43eux441ux442ux44c}{%
\subsubsection{Шероховатость}\label{ux448ux435ux440ux43eux445ux43eux432ux430ux442ux43eux441ux442ux44c}}

Шероховатость поверхности (например, вода или снег, трава, лес) влияет
на силу трения воздуха и, как следствие, на скорость и направление ветра
в приземном слое. Скорость ветра над водой, как правило, больше, чем над
сушей, а отклонение ветра от изобар слабее.

Теплофизические характеристики почвы, воды и воздуха, а также процессы
их взаимодействия, являются ключевыми для понимания динамики атмосферы,
формирования погоды и климата. Моделирование этих процессов, особенно в
условиях антропогенного воздействия, остается актуальной задачей
метеорологии.

ewpage

\hypertarget{ux442ux435ux43cux43fux435ux440ux430ux442ux443ux440ux430-ux437ux435ux43cux43dux43eux439-ux43fux43eux432ux435ux440ux445ux43dux43eux441ux442ux438}{%
\section{Температура земной
поверхности}\label{ux442ux435ux43cux43fux435ux440ux430ux442ux443ux440ux430-ux437ux435ux43cux43dux43eux439-ux43fux43eux432ux435ux440ux445ux43dux43eux441ux442ux438}}

Температура земной поверхности определяется совокупным воздействием
коротковолновой и длинноволновой радиации. Поверхность Земли обычно не
находится в состоянии радиационного равновесия; поглощенная радиация
лишь частично компенсируется эффективным уходящим излучением, а
избыточная энергия расходуется на испарение и контактный теплообмен с
атмосферой. Радиационный баланс подстилающей поверхности, представляющий
собой сумму поглощенной солнечной радиации и общих потерь тепла на
излучение, определяет скорость изменения энергии в поверхностном слое.
Поскольку радиационное равновесие устанавливается медленно, в
формировании температуры поверхности участвуют и другие процессы
теплопереноса.

Ключевые факторы, влияющие на температуру земной поверхности, включают:

\begin{itemize}
\tightlist
\item
  \textbf{Энергетический баланс}: Приток тепла от Солнца, отраженная
  коротковолновая и уходящая длинноволновая радиация. Максимальная
  температура наблюдается в период максимального радиационного баланса
  (обычно около местного полудня), а минимальная --- в момент
  минимального радиационного баланса (вскоре после восхода Солнца).
\item
  \textbf{Скрытая теплота испарения (LE)}: От поверхностного слоя земли
  в ненасыщенный воздух постоянно испаряется вода, что приводит к
  затратам тепла. Влажные поверхности характеризуются большим
  испарением, что приводит к более низким максимальным температурам по
  сравнению с сухими.
\item
  \textbf{Поток тепла в почву (G)}: От поверхностного слоя непрерывно
  идет поток тепла в более глубокие слои подстилающей поверхности.
\item
  \textbf{Турбулентный поток тепла (P)}: Ветер над подстилающей
  поверхностью создает значительный поток тепла между воздухом и
  поверхностью.
\item
  \textbf{Облачность}: Существенно влияет на радиационный баланс земной
  поверхности, а через него -- на термический и влажностный режим
  деятельного слоя почвы и атмосферы. Увеличение облачности уменьшает
  приток солнечной радиации днем и эффективное излучение ночью.
\item
  \textbf{Свойства подстилающей поверхности}: Характер поверхности,
  такой как плотность, удельная теплоемкость, температуропроводность,
  альбедо, наличие растительности и влажность, значительно влияет на ее
  температуру и амплитуду тепловых волн.
\item
  \textbf{Рельеф местности}: Горизонтальные участки прогреваются
  сильнее, чем склоны (кроме южных). Западные склоны теплее восточных, а
  северные -- самые холодные. Дно низин -- самая холодная форма рельефа
  ночью.
\item
  \textbf{Антропогенные факторы}: В больших городах выбросы тепла от
  зданий и транспорта, уменьшение испарения из-за асфальтирования и
  застройки, а также изменения альбедо влияют на формирование городского
  ``острова тепла''. Антропогенный водяной пар играет существенную роль
  в поглощении радиации и формировании температурного режима, в 10--100
  раз превосходя концентрацию CO2 по влиянию. Загрязнение воздуха также
  может приводить к увеличению мутности тропосферы и стратосферы, что
  влияет на альбедо и, как следствие, на температуру поверхности.
\end{itemize}

\hypertarget{ux437ux430ux43aux43eux43dux44b-ux440ux430ux441ux43fux440ux43eux441ux442ux440ux430ux43dux435ux43dux438ux44f-ux442ux435ux43fux43bux430-ux432-ux43fux43eux447ux432ux435}{%
\section{Законы распространения тепла в
почве}\label{ux437ux430ux43aux43eux43dux44b-ux440ux430ux441ux43fux440ux43eux441ux442ux440ux430ux43dux435ux43dux438ux44f-ux442ux435ux43fux43bux430-ux432-ux43fux43eux447ux432ux435}}

Распространение тепла в деятельном слое суши играет важную роль в
изменении температуры и других метеорологических величин. Когда в
пространстве возникает разность температур, внутренняя энергия
переносится из области с более высокой температурой в область с более
низкой температурой. Для описания этого процесса используется
\textbf{теория теплопроводности}.

Основные законы и концепции:

\begin{itemize}
\tightlist
\item
  \textbf{Механизм теплопроводности}: Передача тепла в твердых телах,
  таких как почва, осуществляется путем колебаний частиц (молекул,
  атомов), возбуждающих соседние частицы.
\item
  \textbf{Уравнение теплопроводности}: Описание процесса
  теплопроводности, впервые предложенное Фурье, основано на концепции
  баланса тепла в слое материала. Оно используется для анализа
  распространения температуры в глубь деятельного слоя.
\item
  \textbf{Температуропроводность (λ)}: Это ключевой коэффициент,
  характеризующий скорость распространения температурных изменений в
  материале. Коэффициенты температуропроводности различных видов
  естественных подстилающих поверхностей, включая песчаную, глинистую и
  торфяную почву, хорошо известны.
\item
  \textbf{Периодический характер температурных изменений}: Вследствие
  периодичности солнечной радиации изменения температуры в деятельном
  слое почвы также носят периодический характер и могут быть
  представлены как сумма гармоник (теорема Фурье).

  \begin{itemize}
  \tightlist
  \item
    \textbf{Тепловые волны}: Суточные и годовые тепловые волны проникают
    в почву.
  \item
    \textbf{Затухание амплитуды}: Амплитуда колебаний температуры
    экспоненциально уменьшается с глубиной (например, в \emph{e} раз на
    глубине затухания \emph{h}).
  \item
    \textbf{Сдвиг фазы}: Фаза температурных колебаний запаздывает с
    ростом глубины проникновения тепловых волн. Это свойство хорошо
    иллюстрируется термоизоплетами.
  \end{itemize}
\item
  \textbf{Поток тепла в почву (G)}: Этот поток может быть определен по
  температурному полю тепловых волн.
\end{itemize}

Понимание этих законов имеет решающее значение для моделирования
взаимодействия атмосферы с подстилающей поверхностью и прогнозирования
приземной температуры воздуха, поскольку теплообмен между почвой и
атмосферой является одним из определяющих факторов.

ewpage

\hypertarget{ux432ux435ux440ux442ux438ux43aux430ux43bux44cux43dux43eux435-ux440ux430ux441ux43fux440ux435ux434ux435ux43bux435ux43dux438ux435-ux442ux435ux43cux43fux435ux440ux430ux442ux443ux440ux44b-ux43fux43eux447ux432ux44b-ux438-ux43fux43eux442ux43eux43a-ux442ux435ux43fux43bux430-ux432-ux43fux43eux447ux432ux435}{%
\section{Вертикальное Распределение Температуры Почвы и Поток Тепла в
Почве}\label{ux432ux435ux440ux442ux438ux43aux430ux43bux44cux43dux43eux435-ux440ux430ux441ux43fux440ux435ux434ux435ux43bux435ux43dux438ux435-ux442ux435ux43cux43fux435ux440ux430ux442ux443ux440ux44b-ux43fux43eux447ux432ux44b-ux438-ux43fux43eux442ux43eux43a-ux442ux435ux43fux43bux430-ux432-ux43fux43eux447ux432ux435}}

\hypertarget{ux432ux435ux440ux442ux438ux43aux430ux43bux44cux43dux43eux435-ux440ux430ux441ux43fux440ux435ux434ux435ux43bux435ux43dux438ux435-ux442ux435ux43cux43fux435ux440ux430ux442ux443ux440ux44b-ux43fux43eux447ux432ux44b}{%
\subsection{Вертикальное Распределение Температуры
Почвы}\label{ux432ux435ux440ux442ux438ux43aux430ux43bux44cux43dux43eux435-ux440ux430ux441ux43fux440ux435ux434ux435ux43bux435ux43dux438ux435-ux442ux435ux43cux43fux435ux440ux430ux442ux443ux440ux44b-ux43fux43eux447ux432ux44b}}

Вертикальное распределение температуры почвы характеризуется сложным
изменением с глубиной и во времени, обусловленным теплообменом с
подстилающей поверхностью и процессами теплопереноса внутри деятельного
слоя. Температура подстилающей поверхности всегда отличается от
температуры на глубине деятельного слоя.

\hypertarget{ux441ux443ux442ux43eux447ux43dux44bux439-ux438-ux433ux43eux434ux43eux432ux43eux439-ux445ux43eux434-ux442ux435ux43cux43fux435ux440ux430ux442ux443ux440ux44b-ux432-ux43fux43eux447ux432ux435}{%
\subsubsection{Суточный и Годовой Ход Температуры в
Почве}\label{ux441ux443ux442ux43eux447ux43dux44bux439-ux438-ux433ux43eux434ux43eux432ux43eux439-ux445ux43eux434-ux442ux435ux43cux43fux435ux440ux430ux442ux443ux440ux44b-ux432-ux43fux43eux447ux432ux435}}

В суточном и годовом ходе температурные колебания проникают в почву в
виде тепловых волн.

\begin{itemize}
\tightlist
\item
  \textbf{Суточный ход}: На поверхности минимум температуры наблюдается
  на восходе солнца, а на глубине около 20 см время минимума смещается к
  полудню. Амплитуда суточных колебаний температуры почвы уменьшается с
  глубиной, и фаза (время наступления максимума или минимума)
  запаздывает по мере увеличения глубины проникновения тепловых волн.
  Например, если температура на глубине 3,8 м остается постоянной в
  течение года, то на глубине около 0,38 м она будет приблизительно
  постоянной в течение суток.
\item
  \textbf{Годовой ход}: Минимум температуры вблизи поверхности
  приходится на начало весны, тогда как на глубине 1-1,5 м самый
  холодный месяц --- июнь. Годовой ход температуры воды в водоемах по
  амплитуде значительно меньше, чем годовой ход температуры почвы на той
  же широте, что связано с высокой теплоемкостью воды. Суточный ход
  температуры воды и вовсе незначителен.
\end{itemize}

Для наглядного представления пространственно-временного распределения
температуры почвы используются поля термоизоплет. Эти поля помогают
выявить запаздывание фазы максимальной температуры с ростом глубины
проникновения тепловых волн.

\hypertarget{ux432ux43bux438ux44fux44eux449ux438ux435-ux444ux430ux43aux442ux43eux440ux44b}{%
\subsubsection{Влияющие
Факторы}\label{ux432ux43bux438ux44fux44eux449ux438ux435-ux444ux430ux43aux442ux43eux440ux44b}}

На распределение температуры почвы существенно влияют ее теплофизические
свойства, которые изменяются по площади.

\begin{itemize}
\tightlist
\item
  \textbf{Состав и увлажненность почвы}: Почвенный покров неоднороден по
  составу и структуре, а увлажненность почвы реагирует даже на небольшие
  особенности рельефа, создавая значительные вариации теплофизических
  свойств. Например, при одинаковой увлажненности глинистые почвы летом
  холоднее, а осенью теплее, чем песчаные. Болотные почвы являются
  самыми холодными.
\item
  \textbf{Растительный покров}: Растительность охлаждает почву и
  уменьшает амплитуду тепловых волн.
\item
  \textbf{Рельеф местности}: Горизонтальные участки прогреваются
  сильнее, чем склоны любой экспозиции, кроме южных. Западные склоны
  немного теплее восточных, а северные --- самые холодные. На вершинах и
  гребнях температура днем выше, а ночью ниже, чем на ровной поверхности
  той же высоты. Дно низин --- самая холодная форма рельефа.
\end{itemize}

\hypertarget{ux43fux43eux442ux43eux43a-ux442ux435ux43fux43bux430-ux432-ux43fux43eux447ux432ux435}{%
\subsection{Поток Тепла в
Почве}\label{ux43fux43eux442ux43eux43a-ux442ux435ux43fux43bux430-ux432-ux43fux43eux447ux432ux435}}

Если в пространстве возникает разность температур, внутренняя энергия
переносится из области с более высокой температурой в область с более
низкой температурой. Этот процесс описывается теорией теплопроводности.
В твердых телах, таких как почва, передача тепла осуществляется за счет
колебаний частиц (молекул, атомов), образующих кристаллическую решетку,
где частицы с более высокой энергией возбуждают колебания соседних
частиц.

\hypertarget{ux43cux435ux445ux430ux43dux438ux437ux43c-ux442ux435ux43fux43bux43eux43fux435ux440ux435ux43dux43eux441ux430}{%
\subsubsection{Механизм
Теплопереноса}\label{ux43cux435ux445ux430ux43dux438ux437ux43c-ux442ux435ux43fux43bux43eux43fux435ux440ux435ux43dux43eux441ux430}}

Для температурного поля тепловых волн поток тепла в почву может быть
определен по следующим формулам:

\begin{itemize}
\tightlist
\item
  \(G = -\lambda \frac{\partial T}{\partial z}\)
\item
  \(G = -\rho c_p k_T \frac{\partial T}{\partial z}\) где \(G\) ---
  поток тепла, \(\lambda\) --- коэффициент теплопроводности, \(\rho\)
  --- плотность, \(c_p\) --- удельная теплоемкость, \(k_T\) ---
  коэффициент температуропроводности (диффузии температуры), \(T\) ---
  температура и \(z\) --- глубина. Это означает, что поток тепла
  пропорционален градиенту температуры и соответствующему коэффициенту
  теплопереноса.
\end{itemize}

В водоемах теплообмен происходит не только за счет молекулярной
теплопроводности, но и за счет беспорядочного перемешивания воды
течениями различного масштаба, что называется турбулентным теплообменом.
Главным свойством турбулентного теплообмена является его огромная
скорость по сравнению с молекулярным. Коэффициент молекулярной
температуропроводности воды составляет около \(1 \cdot 10^{-7}\) м²/с,
тогда как коэффициент турбулентной температуропроводности, сильно
зависящий от внешних условий, обычно составляет 0,01 м²/с. В зависимости
от градиента температуры (стратификации) он может как сильно
увеличиваться (при малых градиентах), так и сильно уменьшаться (при
больших). В контексте почвы, доминирующим механизмом является
молекулярная теплопроводность.

\hypertarget{ux441ux443ux442ux43eux447ux43dux44bux439-ux438-ux433ux43eux434ux43eux432ux43eux439-ux43fux43eux442ux43eux43a-ux442ux435ux43fux43bux430}{%
\subsubsection{Суточный и Годовой Поток
Тепла}\label{ux441ux443ux442ux43eux447ux43dux44bux439-ux438-ux433ux43eux434ux43eux432ux43eux439-ux43fux43eux442ux43eux43a-ux442ux435ux43fux43bux430}}

Радиационный баланс подстилающей поверхности (\(R_s\)), определяемый как
сумма поглощенной солнечной радиации и эффективного излучения, напрямую
влияет на скорость изменения энергии в поверхностном слое. Если бы
поверхностный слой находился в состоянии радиационного равновесия,
радиационный баланс был бы равен нулю, что определяло бы температуру
подстилающей поверхности. Однако радиационное равновесие устанавливается
медленно, и в формировании температуры подстилающей поверхности
участвуют другие процессы теплопереноса, включая теплопроводность в
почве.

Поток тепла в почву тесно связан с суточным и годовым ходом
радиационного баланса.

\begin{itemize}
\tightlist
\item
  \textbf{Суточный ход}: Радиационный баланс почти всегда положителен в
  дневное время, так как поступление солнечной радиации обычно
  значительно превышает эффективное излучение. Ночью радиационный баланс
  отрицателен. Переход через нуль происходит за час до захода солнца и
  через час после восхода. Максимальная температура должна наблюдаться в
  период максимального радиационного баланса, обычно в момент местного
  полдня. Минимальная температура наблюдается в момент минимального
  радиационного баланса, обычно вскоре после восхода солнца. В этот
  период испарение близко к нулю, возможна даже конденсация влаги, и
  подстилающая поверхность получает тепло из атмосферы.
\item
  \textbf{Годовой ход}: Избыток энергии образуется в широтном поясе от
  40° ю.ш. до 40° с.ш. В этой области он расходуется в основном на
  испарение и в форме водяного пара переносится в умеренные и полярные
  широты, где пар конденсируется, отдавая содержащееся в нем тепло.
\end{itemize}

Разность температур между почвой и воздухом колеблется от долей градуса
до пяти градусов и может иметь разный знак. А.И. Воейков выделил три
режима: солнечный (почва в среднем за год теплее воздуха, преобладает в
низких и умеренных широтах), снежный (воздух в среднем за год теплее
почвы, характерен для высоких широт) и режим, где разность температур
воздуха и почвы в среднем за год равна нулю (характерен для приморских
районов с большой влажностью).

Турбулентный поток тепла от подстилающей поверхности в атмосферу
составляет около 5-10\% от потока тепла от солнца в среднем за год. Над
океанами и морями он значительно меньше затрат тепла на испарение, за
исключением областей теплых течений (Гольфстрим, Куросио), где разность
температур вода-воздух велика. Над пустынями, где испарение отсутствует,
турбулентный поток тепла является важнейшим фактором теплообмена.

ewpage

\hypertarget{ux43eux441ux43eux431ux435ux43dux43dux43eux441ux442ux438-ux440ux430ux441ux43fux440ux43eux441ux442ux440ux430ux43dux435ux43dux438ux44f-ux442ux435ux43fux43bux430-ux432-ux432ux43eux434ux43eux435ux43cux430ux445}{%
\section{Особенности Распространения Тепла в
Водоемах}\label{ux43eux441ux43eux431ux435ux43dux43dux43eux441ux442ux438-ux440ux430ux441ux43fux440ux43eux441ux442ux440ux430ux43dux435ux43dux438ux44f-ux442ux435ux43fux43bux430-ux432-ux432ux43eux434ux43eux435ux43cux430ux445}}

Распространение тепла в водоемах, особенно в деятельном слое суши и
океана, играет фундаментальную роль в изменении температуры и других
метеорологических величин, а также в формировании туманов и облаков.
Теплообмен в водоемах отличается от теплообмена на суше, прежде всего,
механизмом переноса энергии, что обуславливает специфические
термофизические свойства и их влияние на атмосферные процессы.

\hypertarget{ux43cux435ux445ux430ux43dux438ux437ux43cux44b-ux442ux435ux43fux43bux43eux43eux431ux43cux435ux43dux430}{%
\subsection{Механизмы
Теплообмена}\label{ux43cux435ux445ux430ux43dux438ux437ux43cux44b-ux442ux435ux43fux43bux43eux43eux431ux43cux435ux43dux430}}

В водоемах теплообмен осуществляется не только за счет молекулярной
теплопроводности, но, что более значимо, за счет беспорядочного
перемешивания воды течениями различного масштаба. Этот доминирующий вид
теплообмена называется турбулентным.

\begin{itemize}
\tightlist
\item
  \textbf{Молекулярная теплопроводность:} Коэффициент молекулярной
  температуропроводности воды составляет порядка 10⁻⁷ м²/с. Это довольно
  низкое значение, и в масштабах водоемов его вклад в общий теплообмен,
  как правило, ничтожен.
\item
  \textbf{Турбулентный теплообмен:} Турбулентная температуропроводность
  значительно превосходит молекулярную, обычно составляя порядка 0.01
  м²/с. Эта величина не является постоянной и сильно зависит от внешних
  условий, а также от градиента температуры (стратификации): при малых
  градиентах она может существенно увеличиваться, а при больших ---
  значительно уменьшаться. Это ключевое отличие от процессов на суше,
  где теплообмен преимущественно носит молекулярный характер.
\end{itemize}

\hypertarget{ux441ux43bux435ux434ux441ux442ux432ux438ux44f-ux442ux443ux440ux431ux443ux43bux435ux43dux442ux43dux43eux433ux43e-ux442ux435ux43fux43bux43eux43eux431ux43cux435ux43dux430}{%
\subsection{Следствия Турбулентного
Теплообмена}\label{ux441ux43bux435ux434ux441ux442ux432ux438ux44f-ux442ux443ux440ux431ux443ux43bux435ux43dux442ux43dux43eux433ux43e-ux442ux435ux43fux43bux43eux43eux431ux43cux435ux43dux430}}

Исключительно высокая скорость турбулентного теплообмена в водоемах
приводит к нескольким важным следствиям:

\begin{itemize}
\tightlist
\item
  \textbf{Глубина проникновения тепловых волн:} Тепловые волны проникают
  в воду на гораздо большие глубины, чем в почву. Например, в океане
  годовой ход температуры может охватывать слой в несколько сотен
  метров.
\item
  \textbf{Вертикальные градиенты температуры:} У поверхности воды
  градиенты температуры обычно малы. С глубиной, по мере ослабевания
  влияния ветра и уменьшения коэффициента турбулентности, градиент
  температуры возрастает.
\item
  \textbf{Квазиоднородный поверхностный слой:} Вблизи поверхности воды
  влияние ветра и зависимость коэффициента турбулентности от
  стратификации приводят к более интенсивному теплообмену. Это
  дополнительно уменьшает градиент температуры, формируя квазиоднородное
  вертикальное распределение температуры в поверхностном слое.
\end{itemize}

\hypertarget{ux432ux43bux438ux44fux43dux438ux435-ux43dux430-ux430ux442ux43cux43eux441ux444ux435ux440ux43dux44bux435-ux43fux440ux43eux446ux435ux441ux441ux44b-1}{%
\subsection{Влияние на Атмосферные
Процессы}\label{ux432ux43bux438ux44fux43dux438ux435-ux43dux430-ux430ux442ux43cux43eux441ux444ux435ux440ux43dux44bux435-ux43fux440ux43eux446ux435ux441ux441ux44b-1}}

Особенности распространения тепла в водоемах оказывают существенное
влияние на динамику атмосферы и формирование погодных условий:

\begin{itemize}
\tightlist
\item
  \textbf{Разность температур между воздушными массами:} Вследствие
  различий в теплофизических свойствах и механизмах распространения
  тепла, температура поверхности океана зимой выше температуры
  поверхности суши, а летом --- наоборот. Посредством турбулентного
  обмена явное и скрытое тепло переносится от поверхности океана в
  атмосферу, создавая разность температур между воздушными массами,
  расположенными над океаном и сушей.
\item
  \textbf{Источники энергии для циклонов:} Тепло (явное и скрытое),
  поступающее из океана, является важнейшим источником энергии для
  тропических циклонов. При натекании более холодного воздуха на теплую
  поверхность воды (с температурой выше 26-27 °С) в тропических циклонах
  наблюдаются потоки тепла от океана в диапазоне 1--14 кВт/м². Это тепло
  расходуется как на повышение температуры воздуха (явное тепло), так и
  на испарение морской воды (скрытое тепло). Затем явное тепло и водяной
  пар распространяются на весь слой, охваченный вихревым движением,
  посредством вертикальных движений и турбулентного обмена, практически
  до тропопаузы. Выделение скрытой теплоты при конденсации водяного пара
  в атмосфере также является значительным процессом, приводящим к
  нагреву воздуха.
\item
  \textbf{Сопоставление потоков:} Турбулентный поток тепла с океанов в
  несколько раз меньше затрат тепла на испарение, за исключением
  областей теплых течений, таких как Гольфстрим и Куросио, где разность
  температур вода-воздух велика, и турбулентный поток тепла значителен.
\item
  \textbf{Высокая теплоемкость воды:} Вода обладает весьма необычными
  физическими свойствами, включая высокую удельную теплоемкость, что
  позволяет ей аккумулировать значительное количество тепла. Это
  способствует тому, что водоемы значительно уменьшают амплитуду
  суточного и годового хода температуры прилегающей местности.
\end{itemize}

Эти особенности распространения тепла в водоемах подчеркивают их
критическую роль в энергетическом балансе планеты и формировании
крупномасштабных атмосферных процессов.

ewpage

\hypertarget{ux43fux43eux442ux43eux43aux438-ux442ux435ux43fux43bux430-ux432-ux430ux442ux43cux43eux441ux444ux435ux440ux435}{%
\section{Потоки Тепла в
Атмосфере}\label{ux43fux43eux442ux43eux43aux438-ux442ux435ux43fux43bux430-ux432-ux430ux442ux43cux43eux441ux444ux435ux440ux435}}

Потоки тепла в атмосфере, как фундаментальный аспект динамической
метеорологии, определяют термодинамическое состояние воздушной оболочки
Земли и являются ключевым фактором, обусловливающим атмосферные
процессы, в частности, движение воздуха и формирование погоды и климата.
Динамическая метеорология исследует эти процессы на основе общих законов
физики, включая закон сохранения энергии.

\hypertarget{ux438ux441ux442ux43eux447ux43dux438ux43aux438-ux442ux435ux43fux43bux430-ux438-ux43cux435ux445ux430ux43dux438ux437ux43cux44b-ux442ux435ux43fux43bux43eux43fux435ux440ux435ux434ux430ux447ux438}{%
\subsection{Источники Тепла и Механизмы
Теплопередачи}\label{ux438ux441ux442ux43eux447ux43dux438ux43aux438-ux442ux435ux43fux43bux430-ux438-ux43cux435ux445ux430ux43dux438ux437ux43cux44b-ux442ux435ux43fux43bux43eux43fux435ux440ux435ux434ux430ux447ux438}}

Основным источником энергии для атмосферы является лучистая энергия
Солнца. При этом земная поверхность играет первостепенную роль в
поглощении солнечной радиации, которая затем преобразуется в тепловую
энергию и излучается в инфракрасном диапазоне. Атмосфера, в свою
очередь, активно поглощает это длинноволновое излучение Земли,
нагреваясь и создавая так называемый парниковый эффект.

Потоки тепла в атмосфере осуществляются различными механизмами:

\begin{itemize}
\tightlist
\item
  \textbf{Лучистый теплообмен (Радиация)}: Перенос энергии
  электромагнитными волнами. Атмосфера поглощает и излучает лучистую
  энергию. Водяной пар, углекислый газ и озон являются основными
  поглотителями длинноволновой радиации. Существует ``окно
  прозрачности'' атмосферы в диапазоне 8.5--12 мкм, через которое
  излучение проходит практически без поглощения. Радиационный приток
  тепла к элементарной воздушной частице (εл) является одной из
  составляющих общего притока тепла.
\item
  \textbf{Конвекция}: Перенос тепла путем перемещения воздушных масс.

  \begin{itemize}
  \tightlist
  \item
    \textbf{Адвекция}: Горизонтальный перенос тепла ветром, то есть
    перемещение воздушной массы со своими термическими свойствами.
    Адвекция тепла или холода играет важнейшую роль в циклогенезе и
    антициклогенезе.
  \item
    \textbf{Вертикальная конвекция}: Перенос тепла вверх или вниз за
    счет вертикальных движений воздуха. Разность температур между
    движущейся порцией воздуха и окружающей атмосферой вызывает
    архимедовы силы, сообщающие ей вертикальное ускорение, что приводит
    к развитию конвекции.
  \end{itemize}
\item
  \textbf{Турбулентный теплообмен}: Перенос тепла за счет беспорядочных
  пульсаций скорости и температуры воздуха. Он определяется пульсациями
  вертикальных скоростей и обратно пропорционален градиенту температуры.
  В приземном слое атмосферы, до высоты 1.0--1.5 км, турбулентный
  теплообмен особенно интенсивен и приводит к распространению суточных
  колебаний температуры. Турбулентный приток тепла (εм) также является
  частью общего притока тепла.
\item
  \textbf{Фазовые превращения воды}: Выделение или поглощение скрытой
  теплоты при конденсации, испарении, плавлении или сублимации воды. Это
  один из наиболее значимых источников тепла в атмосфере. Например, при
  конденсации водяного пара в поднимающемся воздухе выделяется скрытая
  теплота, что замедляет падение температуры и способствует развитию
  конвекции, делая влажный воздух менее устойчивым. Тропические циклоны,
  например, черпают колоссальные количества явного и скрытого тепла из
  океана.
\end{itemize}

\hypertarget{ux443ux440ux430ux432ux43dux435ux43dux438ux44f-ux442ux435ux43fux43bux43eux43fux435ux440ux435ux434ux430ux447ux438}{%
\subsection{Уравнения
Теплопередачи}\label{ux443ux440ux430ux432ux43dux435ux43dux438ux44f-ux442ux435ux43fux43bux43eux43fux435ux440ux435ux434ux430ux447ux438}}

В основе количественного анализа тепловых процессов в атмосфере лежат
уравнения гидротермодинамики, в частности, первое начало термодинамики и
уравнение притока тепла.

\textbf{Первое начало термодинамики} для единицы массы воздуха
выражается как: \texttt{dQ\ =\ dJ\ +\ dE} где:

\begin{itemize}
\tightlist
\item
  \texttt{dQ} -- количество подведенного тепла к единице массы воздуха.
\item
  \texttt{dJ} -- изменение внутренней тепловой энергии
  (\texttt{dJ\ =\ cvdT}, где \texttt{cv} -- удельная теплоемкость при
  постоянном объеме).
\item
  \texttt{dE} -- работа, совершаемая воздухом (\texttt{dE\ =\ Pdv}, где
  \texttt{P} -- давление, \texttt{dv} -- приращение удельного объема).
\end{itemize}

В метеорологии часто используется форма уравнения, где работа расширения
выражена через изменение давления: \texttt{dQ\ =\ cpdT\ -\ (1/ρ)dP} где
\texttt{cp} -- удельная теплоемкость при постоянном давлении, \texttt{ρ}
-- плотность.

Если ввести потенциальную температуру \texttt{θ}, уравнение притока
тепла для идеальной атмосферы принимает вид: \texttt{Q\ =\ cp(dθ/dt)θ}
где \texttt{Q} -- количество тепла, сообщаемое извне единице массы
воздуха в единицу времени. Это \texttt{Q} складывается из лучистого,
фазового и молекулярного притоков тепла.

\textbf{Уравнение притока тепла} в развернутой форме (для турбулентной
атмосферы) включает члены, описывающие турбулентные потоки тепла,
которые связаны с пульсациями скорости и потенциальной температуры.

\hypertarget{ux432ux43bux438ux44fux43dux438ux435-ux442ux435ux43fux43bux43eux432ux44bux445-ux43fux43eux442ux43eux43aux43eux432-ux43dux430-ux430ux442ux43cux43eux441ux444ux435ux440ux43dux44bux435-ux43fux440ux43eux446ux435ux441ux441ux44b}{%
\subsection{Влияние Тепловых Потоков на Атмосферные
Процессы}\label{ux432ux43bux438ux44fux43dux438ux435-ux442ux435ux43fux43bux43eux432ux44bux445-ux43fux43eux442ux43eux43aux43eux432-ux43dux430-ux430ux442ux43cux43eux441ux444ux435ux440ux43dux44bux435-ux43fux440ux43eux446ux435ux441ux441ux44b}}

Неравномерное распределение тепла на Земле и связанные с ним процессы
теплообмена являются первопричиной возникновения атмосферных движений.

\begin{itemize}
\tightlist
\item
  \textbf{Давление и Движение Воздуха}: Различия в нагреве приводят к
  горизонтальным градиентам температуры, которые, в свою очередь,
  формируют перепады давления. Эти перепады давят на воздух, заставляя
  его перемещаться, создавая ветры и всю циркуляцию атмосферы. Например,
  в более теплом столбе воздуха давление с высотой падает медленнее, чем
  в более холодном, создавая горизонтальные градиенты давления на
  высоте.
\item
  \textbf{Общая Циркуляция Атмосферы}: Существует за счет разности
  температур между полярными и экваториальными областями, поддерживаемой
  обменом излучения между Землей и космосом. Она осуществляет перенос
  тепла от экватора к полюсам, сглаживая температурные различия.
\item
  \textbf{Вертикальная Устойчивость и Конвекция}: Температурные различия
  между воздушными массами и окружающей средой приводят к конвективным
  движениям. Атмосфера неустойчива, когда температура движущейся порции
  воздуха выше температуры окружающей среды, что приводит к усилению
  вертикального ускорения и развитию конвекции. Высвобождение скрытой
  теплоты при конденсации усиливает неустойчивость насыщенного воздуха.
\item
  \textbf{Облакообразование и Осадки}: Образование облаков и осадков
  тесно связано с вертикальными движениями воздуха и фазовыми
  превращениями воды. В областях восходящих потоков воздух охлаждается,
  приводя к конденсации и образованию облаков. В областях нисходящих
  потоков воздух нагревается и остается безоблачным. Турбулентный обмен
  и вертикальные движения также играют важную роль в переносе влаги и
  образовании облаков и туманов.
\item
  \textbf{Циклоны и Антициклоны}: Формирование синоптических вихрей,
  таких как циклоны и антициклоны, тесно связано с бароклинностью
  атмосферы (зависимость плотности от давления и температуры) и
  горизонтальными контрастами температуры и влажности. Адвекция холода в
  нижней тропосфере способствует зарождению циклонов.
\item
  \textbf{Местные Ветры}: Дифференциальный нагрев и охлаждение
  подстилающей поверхности (например, суши и водоемов, или склонов гор и
  долин) приводят к образованию местных циркуляций, таких как бризы и
  горно-долинные ветры. Фён -- это теплый сухой ветер, возникающий при
  адиабатическом нагреве воздуха, спускающегося с гор.
\item
  \textbf{Городской Остров Тепла}: Влияние антропогенных выбросов,
  включая водяной пар, и изменение подстилающей поверхности в больших
  городах приводит к изменению радиационного режима и повышению
  температуры воздуха по сравнению с сельской местностью, формируя
  ``остров тепла''.
\end{itemize}

\hypertarget{ux442ux435ux43fux43bux43eux432ux43eux439-ux431ux430ux43bux430ux43dux441-ux441ux438ux441ux442ux435ux43cux44b-ux437ux435ux43cux43bux44f-ux430ux442ux43cux43eux441ux444ux435ux440ux430}{%
\subsection{Тепловой Баланс Системы
Земля-Атмосфера}\label{ux442ux435ux43fux43bux43eux432ux43eux439-ux431ux430ux43bux430ux43dux441-ux441ux438ux441ux442ux435ux43cux44b-ux437ux435ux43cux43bux44f-ux430ux442ux43cux43eux441ux444ux435ux440ux430}}

Атмосфера постоянно обменивается теплом с подстилающей поверхностью
Земли (сушей и океаном). В целом, энергетические условия на земной
поверхности определяются совместным действием коротковолновой
(солнечной) и длинноволновой (земной и атмосферной) радиации.
Радиационный баланс подстилающей поверхности (Rs) учитывает поглощенную
солнечную радиацию и суммарную потерю тепла на эффективное излучение
(разность между излучением поверхности и противоизлучением атмосферы).
Атмосфера не находится в состоянии радиационного равновесия; ее
энергетический запас превышает лучистые потоки, что позволяет ей
совершать работу по переносу воздушных масс. Таким образом, атмосферу
можно рассматривать как гигантскую тепловую машину, приводимую в
действие лучистой энергией Солнца.

\hypertarget{ux43fux440ux438ux43cux435ux440ux44b-ux438-ux43fux440ux438ux43cux435ux447ux430ux43dux438ux44f}{%
\subsection{Примеры и
Примечания}\label{ux43fux440ux438ux43cux435ux440ux44b-ux438-ux43fux440ux438ux43cux435ux447ux430ux43dux438ux44f}}

\begin{itemize}
\tightlist
\item
  \textbf{Тропические циклоны} являются ярким примером трансформации и
  переноса огромных количеств тепла и влаги. Тепло, извлеченное из
  океана, идет на повышение температуры воздуха (явное тепло) и
  испарение воды (скрытое тепло). Скрытое тепло высвобождается при
  конденсации в облаках.
\item
  \textbf{Эль-Ниньо} -- это аномально высокая температура поверхности
  океана, которая влияет на распределение температуры воздуха и, как
  следствие, на атмосферную циркуляцию в планетарном масштабе, приводя к
  изменениям в режиме осадков и штормов.
\item
  \textbf{Антропогенное воздействие}: Увеличение концентрации парниковых
  газов (например, CO2) из-за человеческой деятельности ведет к росту
  температуры воздуха у поверхности Земли, усиливая парниковый эффект и
  потенциально влияя на устойчивость атмосферной циркуляции и увеличивая
  количество опасных явлений.
\item
  \textbf{Моделирование}: При составлении прогнозов погоды используются
  гидродинамические модели, которые, наряду с уравнениями движения и
  неразрывности, включают уравнение притока тепла и уравнения переноса
  водяного пара и лучистой энергии. Для упрощения расчетов в
  краткосрочных прогнозах иногда используется адиабатическая модель,
  пренебрегающая притоком тепла извне.
\end{itemize}

Понимание потоков тепла является неотъемлемой частью прогнозирования и
анализа атмосферных явлений, от локальных туманов и осадков до
крупномасштабной циркуляции атмосферы и изменений климата.

ewpage

\hypertarget{ux443ux440ux430ux432ux43dux435ux43dux438ux435-ux43fux440ux438ux442ux43eux43aux430-ux442ux435ux43fux43bux430-ux432-ux442ux443ux440ux431ux443ux43bux435ux43dux442ux43dux43eux439-ux430ux442ux43cux43eux441ux444ux435ux440ux435}{%
\section{Уравнение Притока Тепла в Турбулентной
Атмосфере}\label{ux443ux440ux430ux432ux43dux435ux43dux438ux435-ux43fux440ux438ux442ux43eux43aux430-ux442ux435ux43fux43bux430-ux432-ux442ux443ux440ux431ux443ux43bux435ux43dux442ux43dux43eux439-ux430ux442ux43cux43eux441ux444ux435ux440ux435}}

Изучение уравнения притока тепла является центральным элементом
термодинамики атмосферы, так как оно выражает закон сохранения энергии
применительно к атмосферным процессам. В турбулентной атмосфере, где
метеорологические величины подвержены беспорядочным колебаниям,
необходимо использовать усредненные уравнения для описания теплообмена.

\hypertarget{ux43eux431ux449ux435ux435-ux443ux440ux430ux432ux43dux435ux43dux438ux435-ux43fux440ux438ux442ux43eux43aux430-ux442ux435ux43fux43bux430}{%
\subsection{Общее Уравнение Притока
Тепла}\label{ux43eux431ux449ux435ux435-ux443ux440ux430ux432ux43dux435ux43dux438ux435-ux43fux440ux438ux442ux43eux43aux430-ux442ux435ux43fux43bux430}}

В общем виде уравнение притока тепла для единицы массы воздуха за
единицу времени можно вывести из первого начала термодинамики. Это
уравнение связывает изменение внутренней энергии с подведенным теплом и
совершенной работой. Наиболее часто в метеорологии используется форма
уравнения:

\[ c_p \frac{dT}{dt} - \frac{RT}{P} \frac{dP}{dt} = Q \quad \]

или, если выразить через удельную теплоемкость при постоянном объеме:

\[ c_v dT + Pdv = \Delta Q \quad \]

где:

\begin{itemize}
\tightlist
\item
  \(c_p\) --- удельная теплоемкость воздуха при постоянном давлении (для
  сухого воздуха \(c_p = 1005 \, \text{Дж/(кг} \cdot \text{К)}\)).
\item
  \(c_v\) --- удельная теплоемкость воздуха при постоянном объеме (в
  большинстве расчетов \(c_v = 718 \, \text{Дж/(кг} \cdot \text{К)}\)).
\item
  \(T\) --- абсолютная температура.
\item
  \(P\) --- давление.
\item
  \(R\) --- удельная газовая постоянная воздуха.
\item
  \(\frac{dT}{dt}\) --- полная производная температуры по времени.
\item
  \(\frac{dP}{dt}\) --- полная производная давления по времени.
\item
  \(Q\) --- количество тепла, сообщаемое извне единице массы воздуха за
  единицу времени.
\end{itemize}

Если ввести потенциальную температуру
\(\theta = T \left( \frac{P_o}{P} \right)^{\frac{R}{c_p}}\), то для
идеальной атмосферы уравнение притока тепла примет вид:

\[ c_p \frac{d\theta}{dt} = \frac{Q}{\theta} \quad \]

или в развернутой форме (с учетом адвекции):

\[ c_p \left( \frac{\partial \theta}{\partial t} + u \frac{\partial \theta}{\partial x} + v \frac{\partial \theta}{\partial y} + w \frac{\partial \theta}{\partial z} \right) = \frac{Q}{\theta} \quad \]

где \(u, v, w\) --- проекции скорости движения воздуха на оси координат
\(x, y, z\) соответственно.

\hypertarget{ux43aux43eux43cux43fux43eux43dux435ux43dux442ux44b-ux43fux440ux438ux442ux43eux43aux430-ux442ux435ux43fux43bux430-q}{%
\subsection{Компоненты Притока Тепла
(Q)}\label{ux43aux43eux43cux43fux43eux43dux435ux43dux442ux44b-ux43fux440ux438ux442ux43eux43aux430-ux442ux435ux43fux43bux430-q}}

Приток тепла \(Q\) к элементарной воздушной частице складывается из
нескольких компонент:

\begin{itemize}
\tightlist
\item
  \textbf{Лучистый приток тепла (\(\varepsilon_л\)):} Обусловлен
  процессами поглощения и излучения лучистой энергии. Он является
  основным источником тепла, обусловливающим атмосферные движения.
\item
  \textbf{Фазовый приток тепла (\(\varepsilon_ф\)):} Обусловлен
  выделением или поглощением скрытой теплоты в результате фазовых
  превращений воды (конденсация, испарение, сублимация). Эти процессы
  существенны в облаках и туманах.
\item
  \textbf{Приток тепла за счет молекулярной теплопроводности воздуха
  (\(\varepsilon_м\)):} Вследствие малости коэффициента молекулярной
  теплопроводности воздуха, этот член, как правило, ничтожен и им можно
  пренебречь.
\end{itemize}

Таким образом, общий приток тепла \(\varepsilon\) к единичному объему
воздуха за единицу времени выражается формулой:

\[\varepsilon = \varepsilon_л + \varepsilon_ф + \varepsilon_м \quad \]

где \(Q = \frac{\varepsilon}{\rho}\).

\hypertarget{ux443ux440ux430ux432ux43dux435ux43dux438ux435-ux43fux440ux438ux442ux43eux43aux430-ux442ux435ux43fux43bux430-ux432-ux442ux443ux440ux431ux443ux43bux435ux43dux442ux43dux43eux439-ux430ux442ux43cux43eux441ux444ux435ux440ux435-1}{%
\subsection{Уравнение Притока Тепла в Турбулентной
Атмосфере}\label{ux443ux440ux430ux432ux43dux435ux43dux438ux435-ux43fux440ux438ux442ux43eux43aux430-ux442ux435ux43fux43bux430-ux432-ux442ux443ux440ux431ux443ux43bux435ux43dux442ux43dux43eux439-ux430ux442ux43cux43eux441ux444ux435ux440ux435-1}}

Атмосферные движения, как правило, носят турбулентный характер. В
турбулентной атмосфере мгновенные значения метеорологических величин
(скорости, давления, плотности, температуры) подвержены резким
беспорядочным колебаниям, называемым пульсациями или флуктуациями. Это
делает использование мгновенных значений практически невозможным для
исследования атмосферных процессов, поэтому обращаются к усредненным
уравнениям.

Для вывода усредненного уравнения притока тепла мгновенные значения
каждой величины представляются в виде суммы их средних значений и
соответствующих пульсаций: \(u = \bar{u} + u'\), \(v = \bar{v} + v'\),
\(w = \bar{w} + w'\), \(\theta = \bar{\theta} + \theta'\). Если
пренебречь пульсациями плотности воздуха (\(\rho' = 0\)), полагая
\(\rho = \bar{\rho}\), то усредненное уравнение притока тепла в
турбулентной атмосфере принимает вид:

\[\bar{\rho} c_p \left( \frac{\partial \bar{\theta}}{\partial t} + \bar{u} \frac{\partial \bar{\theta}}{\partial x} + \bar{v} \frac{\partial \bar{\theta}}{\partial y} + \bar{w} \frac{\partial \bar{\theta}}{\partial z} \right) = \bar{\varepsilon}_л + \bar{\varepsilon}_ф - \left( \frac{\partial \overline{\rho c_p u' \theta'}}{\partial x} + \frac{\partial \overline{\rho c_p v' \theta'}}{\partial y} + \frac{\partial \overline{\rho c_p w' \theta'}}{\partial z} \right) \quad \]

где черта сверху обозначает среднее значение, а штрих --- пульсации.
Члены \(\overline{\rho c_p u' \theta'}\),
\(\overline{\rho c_p v' \theta'}\) и \(\overline{\rho c_p w' \theta'}\)
представляют собой составляющие турбулентного потока тепла
\(P_x, P_y, P_z\). Например, \(\overline{\rho c_p w' \theta'}\) выражает
вертикальный турбулентный поток тепла, который численно равняется
взятому с противоположным знаком вертикальному турбулентному потоку
теплосодержания.

Эти составляющие турбулентного потока тепла могут быть выражены через
градиент среднего значения потенциальной температуры при помощи
горизонтальных и вертикальных коэффициентов турбулентного обмена для
переноса тепла \(k_x, k_y, k_z\):

\[ P_x = \rho c_p \overline{u' \theta'} = -\rho c_p k_x \frac{\partial \bar{\theta}}{\partial x} \]
\[ P_y = \rho c_p \overline{v' \theta'} = -\rho c_p k_y \frac{\partial \bar{\theta}}{\partial y} \]
\[ P_z = \rho c_p \overline{w' \theta'} = -\rho c_p k_z \frac{\partial \bar{\theta}}{\partial z} \quad \]

Таким образом, в уравнении притока тепла для усредненного движения
появляется дополнительный приток тепла \(\varepsilon_T\) как результат
воздействия турбулентного перемешивания воздуха на его среднее
термодинамическое состояние. Усредненное уравнение притока тепла в
турбулентной атмосфере может быть записано в виде:

\[\rho c_p \left( \frac{\partial \bar{\theta}}{\partial t} + \bar{u} \frac{\partial \bar{\theta}}{\partial x} + \bar{v} \frac{\partial \bar{\theta}}{\partial y} + \bar{w} \frac{\partial \bar{\theta}}{\partial z} \right) = \bar{\varepsilon}_л + \bar{\varepsilon}_ф + \varepsilon_T \quad \]

где
\(\varepsilon_T = \frac{\partial}{\partial x} \left( \rho c_p k_x \frac{\partial \bar{\theta}}{\partial x} \right) + \frac{\partial}{\partial y} \left( \rho c_p k_y \frac{\partial \bar{\theta}}{\partial y} \right) + \frac{\partial}{\partial z} \left( \rho c_p k_z \frac{\partial \bar{\theta}}{\partial z} \right)\).

Это уравнение отражает, что изменения температуры в турбулентной
атмосфере определяются не только средним переносом тепла и внешними
источниками, но и турбулентным перемешиванием, которое эффективно
переносит тепло вдоль градиентов средней потенциальной температуры.
Коэффициенты турбулентного обмена (\(k_x, k_y, k_z\)) не являются
постоянными и зависят от состояния атмосферы (например, от
стратификации). В целом, турбулентный поток тепла можно вычислять с
помощью гипотезы Фурье.

ewpage

\hypertarget{ux43aux43eux44dux444ux444ux438ux446ux438ux435ux43dux442-ux442ux443ux440ux431ux443ux43bux435ux43dux442ux43dux43eux433ux43e-ux43eux431ux43cux435ux43dux430-ux438-ux43aux43eux44dux444ux444ux438ux446ux438ux435ux43dux442-ux442ux443ux440ux431ux443ux43bux435ux43dux442ux43dux43eux441ux442ux438}{%
\section{Коэффициент турбулентного обмена и коэффициент
турбулентности}\label{ux43aux43eux44dux444ux444ux438ux446ux438ux435ux43dux442-ux442ux443ux440ux431ux443ux43bux435ux43dux442ux43dux43eux433ux43e-ux43eux431ux43cux435ux43dux430-ux438-ux43aux43eux44dux444ux444ux438ux446ux438ux435ux43dux442-ux442ux443ux440ux431ux443ux43bux435ux43dux442ux43dux43eux441ux442ux438}}

\hypertarget{ux43eux431ux449ux438ux435-ux43fux43eux43dux44fux442ux438ux44f-ux438-ux43eux43fux440ux435ux434ux435ux43bux435ux43dux438ux44f}{%
\subsection{Общие понятия и
определения}\label{ux43eux431ux449ux438ux435-ux43fux43eux43dux44fux442ux438ux44f-ux438-ux43eux43fux440ux435ux434ux435ux43bux435ux43dux438ux44f}}

В условиях реальной атмосферы, движение воздуха, как правило, носит
\textbf{турбулентный характер}, особенно ярко проявляющийся в
порывистости ветра. Это означает, что мгновенные значения
метеорологических величин (скорости, давления, температуры, влажности)
подвержены беспорядочным, резким колебаниям, называемым пульсациями или
флуктуациями. Для изучения таких процессов напрямую использовать
мгновенные значения крайне затруднительно, поэтому в динамической
метеорологии применяют \textbf{усредненные уравнения}, в которые,
однако, при осреднении мгновенных величин вводятся новые неизвестные
члены, связанные с пульсациями. Для замыкания этих систем уравнений, а
также для описания турбулентного переноса различных субстанций, вводятся
\textbf{коэффициенты турбулентного обмена}.

Термины «коэффициент турбулентности» и «коэффициент турбулентного
обмена» часто используются взаимозаменяемо для обозначения одной и той
же физической величины, которая характеризует интенсивность
турбулентного перемешивания. Иногда этот коэффициент более специфично
именуется, например, «коэффициентом турбулентной температуропроводности»
применительно к переносу тепла.

По своей сути, коэффициент турбулентного обмена (\(K\)) является
\textbf{коэффициентом пропорциональности} в формулах, описывающих
турбулентные потоки (или напряжения) определенной величины (импульса,
тепла, влаги) через градиент средней (осредненной) величины этой же
характеристики. Он отражает эффективность переноса субстанции
турбулентными вихрями.

Важно отметить, что значения коэффициента турбулентного обмена в
атмосфере \textbf{многократно превышают} аналогичные коэффициенты
молекулярных процессов. Например, коэффициент турбулентной
температуропроводности может достигать \(1 \text{ м}^2/\text{с}\), что в
десятки тысяч раз больше молекулярного коэффициента для воздуха.

\hypertarget{ux440ux43eux43bux44c-ux432-ux430ux442ux43cux43eux441ux444ux435ux440ux43dux44bux445-ux43fux440ux43eux446ux435ux441ux441ux430ux445}{%
\subsection{Роль в атмосферных
процессах}\label{ux440ux43eux43bux44c-ux432-ux430ux442ux43cux43eux441ux444ux435ux440ux43dux44bux445-ux43fux440ux43eux446ux435ux441ux441ux430ux445}}

Коэффициенты турбулентного обмена играют ключевую роль в параметризации
атмосферных процессов, особенно в \textbf{пограничном слое атмосферы},
где турбулентность наиболее выражена. Они позволяют учесть влияние
турбулентного перемешивания на среднее термодинамическое состояние
воздуха, что выражается в появлении дополнительных членов в усредненных
уравнениях гидротермодинамики, таких как уравнение притока тепла.

Например, вертикальный турбулентный поток количества движения (\(\phi\))
определяется как \(\phi = -\rho K_z \frac{\partial u}{\partial z}\), где
\(K_z\) --- вертикальный коэффициент турбулентности. Аналогично,
турбулентный поток произвольной величины \(\phi\) (при условии ее
консервативности и пассивности) выражается как
\(\Phi = -\rho K_z \frac{\partial \phi}{\partial z}\). Для теплового
обмена, составляющие турбулентного потока тепла (\(P_x, P_y, P_z\))
выражаются через соответствующие коэффициенты турбулентного обмена для
тепла (\(k_x, k_y, k_z\)) и градиенты средней потенциальной температуры
(\(\theta\)), например,
\(P_z = -\rho C_p k_z \frac{\partial \theta}{\partial z}\).

Суммарный турбулентный приток тепла (\(\epsilon_T\)) определяется как
отрицательная дивергенция вектора турбулентного потока тепла, что
позволяет выразить его через коэффициенты турбулентного обмена и
градиенты потенциальной температуры.

\hypertarget{ux43cux435ux442ux43eux434ux44b-ux43eux43fux440ux435ux434ux435ux43bux435ux43dux438ux44f}{%
\subsection{Методы
определения}\label{ux43cux435ux442ux43eux434ux44b-ux43eux43fux440ux435ux434ux435ux43bux435ux43dux438ux44f}}

Определение коэффициентов турбулентного обмена является сложной задачей,
требующей учета закономерностей пульсационного движения. Для этого
используются различные подходы:

\begin{enumerate}
\def\labelenumi{\arabic{enumi}.}
\item
  \textbf{Аналогия с молекулярными процессами}: Приближенное определение
  турбулентных напряжений часто основывается на аналогии между
  молекулярным и турбулентным движениями. Это позволяет выражать
  турбулентное напряжение аналогично молекулярному вязкому напряжению,
  но с использованием коэффициента турбулентности вместо кинематического
  коэффициента молекулярной вязкости.
\item
  \textbf{Статистическая теория турбулентности}: Эта теория является
  основой для глубокого понимания и определения турбулентных величин.
  Значительный вклад в ее развитие внесли такие ученые, как А.А.
  Фридман, Л.В. Келлер, А.Н. Колмогоров, А.М. Обухов и А.С. Монин.

  \begin{itemize}
  \tightlist
  \item
    В рамках статистической теории вертикальный коэффициент
    турбулентности (\(K_z\)) может быть выражен через вертикальную
    скорость пульсаций (\(w'_i\)) и градиент средней скорости
    (\(\frac{\partial u}{\partial z}\)):
    \(K_z = \frac{1}{N} \sum_{i=1}^N \alpha_i^2 (z'_i)^2 \left| \frac{\partial u}{\partial z} \right|\),
    где \(\alpha_i\) --- коэффициент пропорциональности, а \(z'_i\) ---
    разность высот.
  \end{itemize}
\item
  \textbf{Эмпирические методы и теория подобия}:

  \begin{itemize}
  \tightlist
  \item
    На основе эмпирических данных и учета порядка метеорологических
    величин, а также при помощи теории подобия, были разработаны методы
    для определения характеристик турбулентного обмена.
  \item
    Примером является параметризация коэффициента турбулентности (\(k\))
    с использованием функции Ричардсона (\(f(Ri)\)), где \(f\) может
    быть линейной функцией своего аргумента, например,
    \(f(l/XL*) = 1 - l/(ZL*t)\).
  \item
    В приземном слое атмосферы коэффициент турбулентности \(k\) может
    зависеть от высоты, например, по формуле
    \(k(h) = \frac{\chi'P L_t}{\ln(\eta_t/\eta_0) Z_t}\) или
    \(k = k_0 [1 + s \cdot \exp(-mz)]\). При этом наблюдается его
    уменьшение по мере приближения к земной поверхности.
  \item
    В верхней части пограничного слоя (слое Экмана) учитывается влияние
    отклоняющей силы вращения Земли на изменение модуля и направления
    скорости ветра с высотой при определении \(k\).
  \end{itemize}
\item
  \textbf{Численные модели}:

  \begin{itemize}
  \tightlist
  \item
    В современных гидродинамических моделях атмосферы коэффициент
    турбулентности (\(k\)) и вертикальная скорость (\(w\)) могут
    определяться с помощью уравнений движения, неразрывности и теории
    подобия.
  \item
    Комплексные численные гидродинамические модели, например, для
    кучево-дождевых облаков, учитывают турбулентный обмен импульсом,
    теплом и влагой как по вертикали, так и по горизонтали.
  \item
    Для практических расчетов турбулентного потока тепла (\(P\)) от
    земной поверхности, который практически не зависит от высоты в
    пределах приземного слоя, может использоваться формула
    \(P = c_p \overline{U} (T_z - T_0) c_f / \int_0^z \frac{dz}{K}\),
    где \(c_f\) --- безразмерный параметр, называемый турбулентной
    проводимостью.
  \end{itemize}
\end{enumerate}

Полное определение всех турбулентных напряжений и соответствующих
коэффициентов турбулентного обмена требует построения сложных систем
уравнений турбулентного движения, что является предметом активных
исследований в динамической метеорологии.

ewpage

\hypertarget{ux43cux435ux442ux43eux434ux44b-ux440ux430ux441ux447ux435ux442ux430-ux442ux443ux440ux431ux443ux43bux435ux43dux442ux43dux43eux433ux43e-ux43fux43eux442ux43eux43aux430-ux442ux435ux43fux43bux430}{%
\section{Методы Расчета Турбулентного Потока
Тепла}\label{ux43cux435ux442ux43eux434ux44b-ux440ux430ux441ux447ux435ux442ux430-ux442ux443ux440ux431ux443ux43bux435ux43dux442ux43dux43eux433ux43e-ux43fux43eux442ux43eux43aux430-ux442ux435ux43fux43bux430}}

В динамической метеорологии и при анализе термодинамических процессов в
атмосфере турбулентный перенос тепла играет ключевую роль, особенно в
пограничном слое. Поскольку атмосферные движения, как правило, носят
турбулентный характер, мгновенные значения метеорологических величин
подвержены беспорядочным колебаниям, или пульсациям. Это делает
невозможным прямое использование мгновенных уравнений; вместо этого
применяются усредненные уравнения гидротермодинамики.

\hypertarget{ux444ux438ux437ux438ux447ux435ux441ux43aux430ux44f-ux441ux443ux449ux43dux43eux441ux442ux44c-ux442ux443ux440ux431ux443ux43bux435ux43dux442ux43dux43eux433ux43e-ux43fux435ux440ux435ux43dux43eux441ux430-ux442ux435ux43fux43bux430}{%
\subsection{Физическая Сущность Турбулентного Переноса
Тепла}\label{ux444ux438ux437ux438ux447ux435ux441ux43aux430ux44f-ux441ux443ux449ux43dux43eux441ux442ux44c-ux442ux443ux440ux431ux443ux43bux435ux43dux442ux43dux43eux433ux43e-ux43fux435ux440ux435ux43dux43eux441ux430-ux442ux435ux43fux43bux430}}

Турбулентное движение характеризуется резкими беспорядочными колебаниями
скорости, давления и плотности --- так называемыми пульсациями или
флуктуациями. Эти пульсации ответственны за перенос тепла. Например, в
вертикальном направлении поток тепла за счет движения воздуха
определяется пульсациями вертикальных скоростей (\(W'\)) и температуры
(\(T'\)). Воздух, поднимающийся в вихре, как правило, теплее, а
опускающийся --- холоднее, что приводит к потоку тепла, направленному в
сторону понижения температуры.

\hypertarget{ux43cux430ux442ux435ux43cux430ux442ux438ux447ux435ux441ux43aux430ux44f-ux444ux43eux440ux43cux443ux43bux438ux440ux43eux432ux43aux430-ux432-ux443ux441ux440ux435ux434ux43dux435ux43dux43dux44bux445-ux443ux440ux430ux432ux43dux435ux43dux438ux44fux445}{%
\subsection{Математическая Формулировка в Усредненных
Уравнениях}\label{ux43cux430ux442ux435ux43cux430ux442ux438ux447ux435ux441ux43aux430ux44f-ux444ux43eux440ux43cux443ux43bux438ux440ux43eux432ux43aux430-ux432-ux443ux441ux440ux435ux434ux43dux435ux43dux43dux44bux445-ux443ux440ux430ux432ux43dux435ux43dux438ux44fux445}}

В усредненных уравнениях гидротермодинамики, которые используются для
описания крупномасштабных атмосферных процессов, члены, описывающие
турбулентный перенос, возникают из усреднения произведений пульсаций
соответствующих величин. Например, составляющие турбулентного потока
тепла (\(P_x, P_y, P_z\)) выражаются как:
\[ P_x = \overline{\rho c_p u' \theta'} \]
\[ P_y = \overline{\rho c_p v' \theta'} \]
\[ P_z = \overline{\rho c_p w' \theta'} \quad \] где \(\rho\) ---
плотность воздуха, \(c_p\) --- удельная теплоемкость при постоянном
давлении, \(u', v', w'\) --- пульсации компонент скорости, \(\theta'\)
--- пульсация потенциальной температуры.

В уравнении притока тепла (первого начала термодинамики), приток тепла к
единице массы воздуха за единицу времени (\(Q\)) включает в себя, помимо
лучистого (\(\varepsilon_л\)) и фазового (\(\varepsilon_ф\)) притоков,
также турбулентный приток тепла (\(\varepsilon_т\)). При этом член,
обусловленный молекулярной теплопроводностью воздуха
(\(\varepsilon_м\)), обычно пренебрежимо мал. В развернутом виде
усредненное уравнение притока тепла в турбулентной атмосфере,
предполагая \(\rho = \bar{\rho}\), выглядит следующим образом:

\[\bar{\rho} c_p \left( \frac{\partial \bar{\theta}}{\partial t} + \bar{u} \frac{\partial \bar{\theta}}{\partial x} + \bar{v} \frac{\partial \bar{\theta}}{\partial y} + \bar{w} \frac{\partial \bar{\theta}}{\partial z} \right) = \bar{\varepsilon}_л + \bar{\varepsilon}_ф - \left( \frac{\partial \overline{\rho c_p u' \theta'}}{\partial x} + \frac{\partial \overline{\rho c_p v' \theta'}}{\partial y} + \frac{\partial \overline{\rho c_p w' \theta'}}{\partial z} \right) \quad \]
Здесь последний член в скобках представляет дивергенцию турбулентного
потока тепла.

\hypertarget{ux43cux435ux442ux43eux434ux44b-ux440ux430ux441ux447ux435ux442ux430-ux442ux443ux440ux431ux443ux43bux435ux43dux442ux43dux43eux433ux43e-ux43fux43eux442ux43eux43aux430-ux442ux435ux43fux43bux430-1}{%
\subsection{Методы Расчета Турбулентного Потока
Тепла}\label{ux43cux435ux442ux43eux434ux44b-ux440ux430ux441ux447ux435ux442ux430-ux442ux443ux440ux431ux443ux43bux435ux43dux442ux43dux43eux433ux43e-ux43fux43eux442ux43eux43aux430-ux442ux435ux43fux43bux430-1}}

Определение турбулентных напряжений и потоков представляет собой сложную
задачу, требующую знания закономерностей пульсационного движения.
Приближенное определение основывается на следующих подходах:

\hypertarget{ux433ux438ux43fux43eux442ux435ux437ux430-ux433ux440ux430ux434ux438ux435ux43dux442ux43dux43eux439-ux434ux438ux444ux444ux443ux437ux438ux438-k-ux442ux435ux43eux440ux438ux44f}{%
\subsubsection{1. Гипотеза Градиентной Диффузии
(K-теория)}\label{ux433ux438ux43fux43eux442ux435ux437ux430-ux433ux440ux430ux434ux438ux435ux43dux442ux43dux43eux439-ux434ux438ux444ux444ux443ux437ux438ux438-k-ux442ux435ux43eux440ux438ux44f}}

Этот подход основан на аналогии между молекулярным и турбулентным
движениями. Подобно тому, как молекулярная теплопроводность связана с
градиентом температуры, турбулентный поток тепла предполагается
пропорциональным градиенту осредненной температуры (или потенциальной
температуры). Вертикальный турбулентный поток величины \(\Phi\) (включая
тепло) может быть определен по формуле:
\[ P_z = - \rho K_\Phi \frac{\partial \Phi}{\partial z} \quad \] где
\(K_\Phi\) --- коэффициент турбулентного обмена (или коэффициент
турбулентной температуропроводности для тепла), а
\(\frac{\partial \Phi}{\partial z}\) --- вертикальный градиент
осредненной величины \(\Phi\). Для турбулентного потока явного тепла по
вертикали используется выражение:
\[ P_z = - \rho c_p K_z \frac{\partial \bar{\theta}}{\partial z} \quad \]
или, более общо, \(W'T' = -K \frac{dT}{dz}\). Знак ``минус'' указывает,
что турбулентный перенос происходит в направлении, противоположном
градиенту, то есть от более высоких значений к более низким. Коэффициент
турбулентности (\(K\)) зависит от вертикального градиента температуры и
значительно меняется в пределах пограничного слоя.

\hypertarget{ux441ux442ux430ux442ux438ux441ux442ux438ux447ux435ux441ux43aux438ux435-ux442ux435ux43eux440ux438ux438-ux442ux443ux440ux431ux443ux43bux435ux43dux442ux43dux43eux441ux442ux438}{%
\subsubsection{2. Статистические Теории
Турбулентности}\label{ux441ux442ux430ux442ux438ux441ux442ux438ux447ux435ux441ux43aux438ux435-ux442ux435ux43eux440ux438ux438-ux442ux443ux440ux431ux443ux43bux435ux43dux442ux43dux43eux441ux442ux438}}

Более сложные методы опираются на статистическую теорию турбулентности,
которая была основана А.А. Фридманом и Л.В. Келлером, с важными вкладами
А.Н. Колмогорова, А.М. Обухова, А.С. Монина и других авторов. Эти теории
позволяют получить более полное описание турбулентных напряжений и
потоков, не ограничиваясь простой градиентной гипотезой, которая не
всегда адекватно описывает сложные условия в турбулентной атмосфере.
Однако, построение таких моделей требует более сложной системы уравнений
турбулентного движения.

\hypertarget{ux43eux43fux440ux435ux434ux435ux43bux435ux43dux438ux435-ux43fux43e-ux43eux43fux44bux442ux43dux44bux43c-ux434ux430ux43dux43dux44bux43c-ux438-ux447ux438ux441ux43bux435ux43dux43dux43eux435-ux43cux43eux434ux435ux43bux438ux440ux43eux432ux430ux43dux438ux435}{%
\subsubsection{3. Определение по Опытным Данным и Численное
Моделирование}\label{ux43eux43fux440ux435ux434ux435ux43bux435ux43dux438ux435-ux43fux43e-ux43eux43fux44bux442ux43dux44bux43c-ux434ux430ux43dux43dux44bux43c-ux438-ux447ux438ux441ux43bux435ux43dux43dux43eux435-ux43cux43eux434ux435ux43bux438ux440ux43eux432ux430ux43dux438ux435}}

На практике, для определения турбулентных потоков могут использоваться:

\begin{itemize}
\tightlist
\item
  \textbf{Прямые измерения пульсаций}: Хотя мгновенные значения
  использовать невозможно для общих исследований, непосредственные
  измерения пульсаций (\(u', v', w', \theta'\)) в определенных условиях
  позволяют рассчитать ковариации и, следовательно, турбулентные потоки.
\item
  \textbf{Эмпирические и полуэмпирические формулы}: Разработаны
  специальные способы расчета параметров, таких как равновесный градиент
  температуры, по опытным данным.
\item
  \textbf{Численные модели}: В комплексных численных моделях атмосферы
  (например, для моделирования образования облаков) учитывается
  турбулентный обмен импульсом, теплом и влагой по вертикали и
  горизонтали. Эти модели решают системы уравнений, включающие члены для
  турбулентного обмена.
\end{itemize}

Турбулентный теплообмен является важнейшим процессом теплопереноса в
воздухе вблизи подстилающей поверхности, и его физико-математическое
описание до настоящего времени продолжает развиваться.

ewpage

\hypertarget{ux441ux443ux442ux43eux447ux43dux44bux439-ux438-ux433ux43eux434ux43eux432ux43eux439-ux445ux43eux434-ux442ux435ux43cux43fux435ux440ux430ux442ux443ux440ux44b}{%
\section{Суточный и Годовой Ход
Температуры}\label{ux441ux443ux442ux43eux447ux43dux44bux439-ux438-ux433ux43eux434ux43eux432ux43eux439-ux445ux43eux434-ux442ux435ux43cux43fux435ux440ux430ux442ux443ux440ux44b}}

Температура воздуха в атмосфере непрерывно изменяется как в
пространстве, так и во времени. Диапазон этих изменений на Земле весьма
широк: от 60 °С в тропических пустынях днем до -90 °С на высокогорных
плато Антарктиды во время полярной ночи. Эти изменения описываются
суточным и годовым ходом температуры, обусловленными в первую очередь
притоком солнечной радиации и последующими процессами теплообмена.

\hypertarget{ux441ux443ux442ux43eux447ux43dux44bux439-ux445ux43eux434-ux442ux435ux43cux43fux435ux440ux430ux442ux443ux440ux44b}{%
\subsection{Суточный Ход
Температуры}\label{ux441ux443ux442ux43eux447ux43dux44bux439-ux445ux43eux434-ux442ux435ux43cux43fux435ux440ux430ux442ux443ux440ux44b}}

Суточный ход температуры воздуха является результатом непрерывных
процессов нагревания и охлаждения подстилающей поверхности и прилегающих
слоев атмосферы под влиянием солнечной радиации и эффективного
излучения.

\hypertarget{ux43eux441ux43dux43eux432ux43dux44bux435-ux445ux430ux440ux430ux43aux442ux435ux440ux438ux441ux442ux438ux43aux438-ux438-ux43cux435ux445ux430ux43dux438ux437ux43cux44b}{%
\subsubsection{Основные Характеристики и
Механизмы}\label{ux43eux441ux43dux43eux432ux43dux44bux435-ux445ux430ux440ux430ux43aux442ux435ux440ux438ux441ux442ux438ux43aux438-ux438-ux43cux435ux445ux430ux43dux438ux437ux43cux44b}}

\begin{itemize}
\tightlist
\item
  \textbf{Причины:} Основной движущей силой является суточный ход
  инсоляции, которая характеризуется резкими изменениями на восходе и
  заходе солнца. Днем поступление солнечной радиации, как правило,
  значительно превышает эффективное излучение, приводя к положительному
  радиационному балансу и нагреву. Ночью радиационный баланс
  отрицателен, что приводит к охлаждению.
\item
  \textbf{Экстремумы:} Максимум температуры обычно наступает примерно
  через час после местного полудня, а минимум -- перед восходом солнца.
\item
  \textbf{Изменение с Высотой:} Амплитуда суточных колебаний температуры
  уменьшается с высотой в атмосфере экспоненциально, подобно тому, как
  это происходит в почве. Например, на высоте 300 м амплитуда составляет
  примерно 3 °С, а в слое 300-700 м убывает до 1 °С. Максимум
  температуры также запаздывает с высотой: на уровне наземных наблюдений
  (2 м) он наступает примерно через два часа после местного полдня, а в
  слое 100-200 м -- в 16-17 часов местного времени.
\end{itemize}

\hypertarget{ux432ux43bux438ux44fux44eux449ux438ux435-ux444ux430ux43aux442ux43eux440ux44b-1}{%
\subsubsection{Влияющие
Факторы}\label{ux432ux43bux438ux44fux44eux449ux438ux435-ux444ux430ux43aux442ux43eux440ux44b-1}}

\begin{itemize}
\tightlist
\item
  \textbf{Тип Подстилающей Поверхности:} Характер подстилающей
  поверхности оказывает сильное влияние на ее температуру и,
  следовательно, на амплитуду тепловых волн.

  \begin{itemize}
  \tightlist
  \item
    \textbf{Почва:} Суточные колебания температуры почвы
    распространяются на относительно небольшие глубины (десятки
    сантиметров), при этом фаза максимальной температуры запаздывает с
    ростом глубины. Например, на поверхности минимум температуры
    наступает на восходе солнца, а на глубине 20 см --- ближе к полудню.
  \item
    \textbf{Вода:} Благодаря высокой теплоемкости воды и свободному
    перемешиванию, суточный ход температуры воды крайне незначителен и
    распространяется на значительно большие глубины, чем в почве.
  \end{itemize}
\item
  \textbf{Облачность:} При ясной погоде суточный ход температуры
  наиболее выражен. Облака, особенно нижнего яруса, уменьшают
  поступление солнечной радиации днем и эффективное излучение ночью,
  снижая амплитуду суточных колебаний.
\item
  \textbf{Ветер и Турбулентность:} Слабый ветер и ясное небо
  благоприятствуют радиационному выхолаживанию подстилающей поверхности
  и образованию приземных инверсий. Сильный ветер, наоборот, интенсивно
  перемешивает воздушную массу, распределяя тепло/холод по более мощному
  слою и уменьшая суточные колебания. Турбулентное перемешивание может
  привести к повышению температуры зимой среди ночи.
\item
  \textbf{Влажность Воздуха:} Влажность воздуха влияет на эффективное
  излучение и, следовательно, на степень охлаждения поверхности земли и
  прилегающего слоя воздуха. Низкое парциальное давление водяного пара
  (низкая влажность) способствует более значительному понижению
  температуры.
\item
  \textbf{Рельеф Местности:} В котловинах, где холодный воздух может
  скапливаться, радиационное выхолаживание выражено сильнее. Дно низин
  -- самая холодная форма рельефа. Склоны гор также проявляют суточные
  особенности нагрева, что приводит к горно-долинным ветрам.
\item
  \textbf{Урбанизация:} Городские ``острова тепла'' характеризуются
  повышением температуры на 2-3°С по сравнению с окружающей природой.
  Это связано с тепловыделением от зданий и транспорта, уменьшением
  испарения из-за застройки и изменениями в теплофизических свойствах
  городской поверхности. Зимой разность температур между городом и
  окрестностями меньше изменяется в течение суток, а летом суточные
  колебания более выражены. Ночью город теплее окрестностей, что связано
  с антропогенными факторами и ослабленным турбулентным обменом. Днем
  испарение в городе ниже, что может приводить к смене знака разности
  температур.
\end{itemize}

\hypertarget{ux441ux443ux442ux43eux447ux43dux44bux435-ux43aux43eux43bux435ux431ux430ux43dux438ux44f-ux434ux440ux443ux433ux438ux445-ux43cux435ux442ux435ux43eux440ux43eux43bux43eux433ux438ux447ux435ux441ux43aux438ux445-ux432ux435ux43bux438ux447ux438ux43d}{%
\subsubsection{Суточные Колебания Других Метеорологических
Величин}\label{ux441ux443ux442ux43eux447ux43dux44bux435-ux43aux43eux43bux435ux431ux430ux43dux438ux44f-ux434ux440ux443ux433ux438ux445-ux43cux435ux442ux435ux43eux440ux43eux43bux43eux433ux438ux447ux435ux441ux43aux438ux445-ux432ux435ux43bux438ux447ux438ux43d}}

\begin{itemize}
\tightlist
\item
  \textbf{Влажность:} Парциальное давление и массовая доля водяного пара
  имеют небольшой суточный ход, в то время как относительная влажность
  может иметь значительный суточный ход, определяемый ходом температуры.
\item
  \textbf{Облачность:} Кучевообразные облака (кучевые, кучево-дождевые)
  чаще образуются во второй половине суток, что связано с развитием
  конвекции при дневном прогреве. Однако, вопреки распространенным
  представлениям, некоторые группы кучевых облаков могут наблюдаться
  реже днем, чем ночью. Облака конвективного происхождения в подавляющем
  большинстве случаев наблюдаются в циклонической обстановке.
\item
  \textbf{Осадки:} Ливневые осадки, связанные с кучево-дождевыми
  облаками, демонстрируют сезонные и суточные колебания, причем число
  дней с ними максимально летом и существенно больше во второй половине
  суток. В городе продолжительность ливневых осадков максимальна летом.
\item
  \textbf{Туманы и Дымки:} Ночью туманы образуются как под влиянием
  адвекции, так и радиационных потерь тепла, тогда как днем -- только
  под влиянием адвективных и турбулентных притоков тепла и влаги. Летом
  вклад радиации в формирование туманов достигает 25-30\%, дымок --
  15-20\%. В другие сезоны этот вклад значительно меньше. В городе число
  туманов меньше, чем в сельской местности, из-за более высокой
  температуры и пониженной относительной влажности.
\end{itemize}

\hypertarget{ux433ux43eux434ux43eux432ux43eux439-ux445ux43eux434-ux442ux435ux43cux43fux435ux440ux430ux442ux443ux440ux44b}{%
\subsection{Годовой Ход
Температуры}\label{ux433ux43eux434ux43eux432ux43eux439-ux445ux43eux434-ux442ux435ux43cux43fux435ux440ux430ux442ux443ux440ux44b}}

Годовой ход температуры отражает сезонные изменения в распределении
солнечной радиации, обусловленные наклоном земной оси и орбитальным
движением Земли.

\hypertarget{ux43eux441ux43dux43eux432ux43dux44bux435-ux445ux430ux440ux430ux43aux442ux435ux440ux438ux441ux442ux438ux43aux438-ux438-ux43cux435ux445ux430ux43dux438ux437ux43cux44b-1}{%
\subsubsection{Основные Характеристики и
Механизмы}\label{ux43eux441ux43dux43eux432ux43dux44bux435-ux445ux430ux440ux430ux43aux442ux435ux440ux438ux441ux442ux438ux43aux438-ux438-ux43cux435ux445ux430ux43dux438ux437ux43cux44b-1}}

\begin{itemize}
\tightlist
\item
  \textbf{Причины:} Распределение инсоляции по земной поверхности имеет
  годовой ход, с максимальной инсоляцией летом не на экваторе, а на
  широте тропика летнего полушария, а также значительным поступлением
  радиации в полярные районы в период полярного дня. В зимнем полушарии
  инсоляция быстро убывает с широтой, а за полярным кругом наблюдаются
  периоды полярной ночи.
\item
  \textbf{Экстремумы:} Максимум и минимум температуры не совпадают по
  времени с соответствующими экстремумами солнечной радиации. На
  континентах запаздывание составляет примерно месяц, на океанах --
  два-три месяца.
\item
  \textbf{Миграция Термического Экватора:} Область термического
  максимума (термический экватор) мигрирует, оставаясь всегда в летнем
  полушарии.
\item
  \textbf{Амплитуда Колебаний:} Годовые колебания температуры наиболее
  выражены над континентами, а над океанами они сглажены. Глубина
  проникновения годовых колебаний температуры в почву (порядка
  нескольких метров) значительно больше, чем суточных (десятки
  сантиметров). Например, на глубине 1-1.5 м самый холодный месяц может
  быть июнь.
\item
  \textbf{Стратосфера:} Тепловой режим стратосферы определяется
  поглощением озоном коротковолновой солнечной радиации. Зимой в
  полярной области происходит охлаждение за счет длинноволнового
  излучения, а летом -- нагревание. Это приводит к значительным сезонным
  изменениям температуры в стратосфере высоких широт.
\end{itemize}

\hypertarget{ux432ux43bux438ux44fux44eux449ux438ux435-ux444ux430ux43aux442ux43eux440ux44b-2}{%
\subsubsection{Влияющие
Факторы}\label{ux432ux43bux438ux44fux44eux449ux438ux435-ux444ux430ux43aux442ux43eux440ux44b-2}}

\begin{itemize}
\tightlist
\item
  \textbf{Континентальность/Океаничность:} Над материками амплитуда
  годовых колебаний влажности значительно больше, чем над водой,
  особенно в районах с жарким летом и снежной зимой. Разности значений
  между материками и океанами по количеству облаков могут быть
  существенно больше зональных.
\item
  \textbf{Адвекция:} Перенос тепла или холода воздушными массами играет
  огромную роль в формировании годового хода температуры.

  \begin{itemize}
  \tightlist
  \item
    \textbf{Адвекция Холода/Тепла:}

    \begin{itemize}
    \tightlist
    \item
      Зимой материки холоднее океанов, что способствует адвекции тепла и
      антициклогенезу. В это время антициклоны чаще образуются над
      сушей.
    \item
      Летом материки теплее океанов, что благоприятствует адвекции
      холода и циклогенезу. Циклоны чаще образуются над сушей летом.
    \item
      Адвекция холода из высоких широт способствует циклогенезу, а
      адвекция тепла из тропиков -- усилению антициклонов.
    \item
      Перенос теплого воздуха из южного полушария в северное в
      декабре-феврале усиливает субтропический пояс высокого давления в
      северном полушарии.
    \item
      Летняя адвекция холода ослабляет субтропические антициклоны и
      смещает их в более высокие широты.
    \end{itemize}
  \item
    \textbf{Примеры:}

    \begin{itemize}
    \tightlist
    \item
      Над Охотским морем летом часто образуются антициклоны из-за
      адвекции тепла (море холоднее суши), а зимой -- почти не
      образуются из-за адвекции холода (суша холоднее моря).
    \item
      Аномально теплые зимы в Европе (например, 1988/89 и 1989/90 гг.)
      связаны с интенсивным выносом тепла с Атлантики, обусловленным
      усилением адвекции холода у восточного побережья Канады и
      углублением исландского минимума.
    \end{itemize}
  \end{itemize}
\item
  \textbf{Облачность:} Годовой ход повторяемости ясного неба и некоторых
  форм облаков хорошо выражен на материках.

  \begin{itemize}
  \tightlist
  \item
    Зимой безоблачная погода чаще наблюдается, чем летом, в большинстве
    широтных зон, за исключением тропиков, где смещение субтропического
    максимума давления летом приводит к увеличению повторяемости ясного
    неба.
  \item
    Кучевые и кучево-дождевые облака максимальны летом и минимальны
    зимой.
  \item
    Слоистообразные облака (Ns) в умеренных и высоких широтах летом
    встречаются реже, чем зимой.
  \item
    В целом, общее количество облаков на материках летом больше, чем
    зимой.
  \item
    В тропиках и субтропиках сезонные колебания облачности обусловлены
    смещением и изменением интенсивности субтропического максимума
    давления.
  \end{itemize}
\item
  \textbf{Морские Течения:} Холодные океанические течения у западных
  побережий континентов способствуют усилению антициклонов за счет
  адвекции тепла.
\item
  \textbf{Эль-Ниньо:} Изменения в атмосферной циркуляции, связанные с
  Эль-Ниньо, включая смещение зон конвергенции и адвекцию холода, влияют
  на образование и эволюцию циркуляционных систем и, как следствие, на
  температурный режим обширных районов.
\end{itemize}

Таким образом, суточный и годовой ход температуры воздуха в атмосфере
являются сложными явлениями, формирующимися под совокупным воздействием
радиационных, адвективных и турбулентных процессов, тесно связанных с
географическими особенностями подстилающей поверхности и динамикой
атмосферной циркуляции.

ewpage

\hypertarget{ux443ux440ux430ux432ux43dux435ux43dux438ux435-ux43fux440ux438ux442ux43eux43aux430-ux442ux435ux43fux43bux430-ux432-ux442ux443ux440ux431ux443ux43bux435ux43dux442ux43dux43eux439-ux430ux442ux43cux43eux441ux444ux435ux440ux435-2}{%
\section{Уравнение Притока Тепла в Турбулентной
Атмосфере}\label{ux443ux440ux430ux432ux43dux435ux43dux438ux435-ux43fux440ux438ux442ux43eux43aux430-ux442ux435ux43fux43bux430-ux432-ux442ux443ux440ux431ux443ux43bux435ux43dux442ux43dux43eux439-ux430ux442ux43cux43eux441ux444ux435ux440ux435-2}}

Изучение уравнения притока тепла является краеугольным камнем
термодинамики атмосферы, так как оно является выражением закона
сохранения энергии, применительно к воздушной среде. Это уравнение
описывает изменение внутренней энергии частицы воздуха под воздействием
подведенного тепла и совершаемой работы.

\hypertarget{ux43eux441ux43dux43eux432ux43dux44bux435-ux444ux43eux440ux43cux44b-ux443ux440ux430ux432ux43dux435ux43dux438ux44f-ux43fux440ux438ux442ux43eux43aux430-ux442ux435ux43fux43bux430}{%
\subsection{Основные Формы Уравнения Притока
Тепла}\label{ux43eux441ux43dux43eux432ux43dux44bux435-ux444ux43eux440ux43cux44b-ux443ux440ux430ux432ux43dux435ux43dux438ux44f-ux43fux440ux438ux442ux43eux43aux430-ux442ux435ux43fux43bux430}}

Из первого начала термодинамики, которое гласит, что подведенное к
единице массы воздуха тепло (\(dQ\)) расходуется на увеличение
внутренней тепловой энергии (\(dJ\)) и на работу (\(dE\)), которую
совершает воздух при расширении, можно получить базовое уравнение:

\[ dJ + dE = dQ \quad \]

Для сухого и влажного ненасыщенного воздуха, рассматриваемого как
идеальный газ, внутренняя энергия пропорциональна абсолютной температуре
(\(J = c_v T + \text{const}\)), откуда \(dJ = c_v dT\). Работа
расширения (\(dE\)) зависит от приращения удельного объема (\(dv\)) и
внешнего давления (\(P\)): \(dE = Pdv\). Таким образом, уравнение
первого начала термодинамики принимает вид:

\[ c_v dT + Pdv = dQ \quad \]

Это уравнение может быть приведено к более распространенной в
метеорологии форме, не содержащей неудобной для измерения величины ---
объема частицы, с использованием уравнения состояния идеального газа
(\(P = \rho RT\)) и соотношения Майера (\(c_p = c_v + R\)), где \(c_p\)
--- удельная теплоемкость воздуха при постоянном давлении, а \(R\) ---
удельная газовая постоянная воздуха:

\[ c_p \frac{dT}{dt} - \frac{RT}{P} \frac{dP}{dt} = Q \quad \]

Здесь \(Q\) представляет собой количество тепла, сообщаемое извне
единице массы воздуха за единицу времени. Полная производная по времени
\(d/dt\) (индивидуальная производная) выражает изменение величины во
времени внутри движущейся массы воздуха. Она включает локальное
изменение и адвективные/конвективные изменения, обусловленные переносом.

Также уравнение притока тепла можно выразить через потенциальную
температуру \(\theta\), которая сохраняется при адиабатических
процессах:

\[ c_p \frac{d\theta}{dt} = \frac{Lr}{P} + \varepsilon_\theta \quad \]

где \(r\) -- удельная скорость конденсации (масса водяного пара,
переходящего в продукты конденсации, в единице объема за единицу
времени), \(L\) -- скрытая теплота парообразования/конденсации, а
\(\varepsilon_\theta\) -- турбулентный приток тепла к единичному объему
воздуха за единицу времени.

\hypertarget{ux43aux43eux43cux43fux43eux43dux435ux43dux442ux44b-ux43fux440ux438ux442ux43eux43aux430-ux442ux435ux43fux43bux430-q-ux438ux43bux438-varepsilon}{%
\subsection{\texorpdfstring{Компоненты Притока Тепла (Q или
\(\varepsilon\))}{Компоненты Притока Тепла (Q или \textbackslash varepsilon)}}\label{ux43aux43eux43cux43fux43eux43dux435ux43dux442ux44b-ux43fux440ux438ux442ux43eux43aux430-ux442ux435ux43fux43bux430-q-ux438ux43bux438-varepsilon}}

Приток тепла к элементарной воздушной частице складывается из нескольких
компонент:

\begin{itemize}
\tightlist
\item
  \textbf{Лучистый приток тепла (\(\varepsilon_л\)):} Обусловлен
  процессами поглощения и излучения лучистой энергии. Лучистая энергия
  слабо поглощается воздухом, однако различные газообразные, жидкие и
  твердые примеси оптически активны и поглощают радиацию, особенно в
  инфракрасном диапазоне. Основными поглотителями земной радиации
  являются водяной пар, углекислый газ и озон.
\item
  \textbf{Фазовый приток тепла (\(\varepsilon_ф\)):} Обусловлен
  выделением или поглощением скрытой теплоты в результате фазовых
  превращений воды (например, конденсация водяного пара, которая
  сопровождается выделением значительного количества скрытой теплоты,
  повышающей температуру воздуха).
\item
  \textbf{Приток тепла за счет молекулярной теплопроводности воздуха
  (\(\varepsilon_м\)):} Вследствие малости коэффициента молекулярной
  теплопроводности воздуха, этот член, как правило, ничтожен и им можно
  пренебречь в метеорологических расчетах крупномасштабных процессов.
\end{itemize}

Кроме того, в уравнении притока тепла явно или неявно учитываются другие
процессы, приводящие к изменению температуры в точке:

\begin{itemize}
\tightlist
\item
  \textbf{Адвективный приток тепла:} Обусловлен горизонтальным переносом
  воздушных масс.
\item
  \textbf{Конвективный приток тепла:} Связан с вертикальными движениями
  воздуха.
\item
  \textbf{Турбулентный приток тепла:} Возникает из-за случайных
  колебаний ветра.
\end{itemize}

\hypertarget{ux443ux440ux430ux432ux43dux435ux43dux438ux435-ux43fux440ux438ux442ux43eux43aux430-ux442ux435ux43fux43bux430-ux432-ux442ux443ux440ux431ux443ux43bux435ux43dux442ux43dux43eux439-ux430ux442ux43cux43eux441ux444ux435ux440ux435-3}{%
\subsection{Уравнение Притока Тепла в Турбулентной
Атмосфере}\label{ux443ux440ux430ux432ux43dux435ux43dux438ux435-ux43fux440ux438ux442ux43eux43aux430-ux442ux435ux43fux43bux430-ux432-ux442ux443ux440ux431ux443ux43bux435ux43dux442ux43dux43eux439-ux430ux442ux43cux43eux441ux444ux435ux440ux435-3}}

Атмосферные движения, как правило, носят турбулентный характер.
Мгновенные значения метеорологических величин подвержены беспорядочным
колебаниям, что делает необходимым использование усредненных уравнений
для их описания.

При осреднении уравнений гидротермодинамики в турбулентной атмосфере
появляются новые неизвестные величины, так называемые турбулентные
потоки. Для турбулентного потока тепла, возникающего за счет
перемешивания, используется гипотеза Фурье. Согласно этой гипотезе,
турбулентный поток тепла пропорционален градиенту средней температуры
(или потенциальной температуры), а произведение пульсаций температуры и
скорости имеет знак, противоположный градиенту температуры:

\[\overline{w'T'} = -K \frac{dT}{dz} \quad \]

где \(K\) --- коэффициент турбулентного обмена (турбулентной
температуропроводности).

Таким образом, в усредненном уравнении притока тепла появляется
дополнительный член, представляющий собой турбулентный приток тепла
(\(\varepsilon_T\) или \(\varepsilon_\theta\) в терминологии), который
описывает перенос тепла турбулентным перемешиванием. Этот член учитывает
воздействие турбулентности на среднее термодинамическое состояние
атмосферы.

ewpage

\hypertarget{ux438ux437ux43cux435ux43dux435ux43dux438ux44f-ux442ux435ux43cux43fux435ux440ux430ux442ux443ux440ux44b-ux432-ux442ux440ux43eux43fux43eux441ux444ux435ux440ux435-ux438-ux442ux435ux43cux43fux435ux440ux430ux442ux443ux440ux43dux44bux435-ux438ux43dux432ux435ux440ux441ux438ux438}{%
\section{Изменения Температуры в Тропосфере и Температурные
Инверсии}\label{ux438ux437ux43cux435ux43dux435ux43dux438ux44f-ux442ux435ux43cux43fux435ux440ux430ux442ux443ux440ux44b-ux432-ux442ux440ux43eux43fux43eux441ux444ux435ux440ux435-ux438-ux442ux435ux43cux43fux435ux440ux430ux442ux443ux440ux43dux44bux435-ux438ux43dux432ux435ux440ux441ux438ux438}}

\hypertarget{ux43eux431ux449ux438ux435-ux43fux43eux43dux44fux442ux438ux44f-ux43e-ux442ux435ux43cux43fux435ux440ux430ux442ux443ux440ux43dux44bux445-ux43fux43eux43bux44fux445-ux432-ux430ux442ux43cux43eux441ux444ux435ux440ux435}{%
\subsection{Общие Понятия о Температурных Полях в
Атмосфере}\label{ux43eux431ux449ux438ux435-ux43fux43eux43dux44fux442ux438ux44f-ux43e-ux442ux435ux43cux43fux435ux440ux430ux442ux443ux440ux43dux44bux445-ux43fux43eux43bux44fux445-ux432-ux430ux442ux43cux43eux441ux444ux435ux440ux435}}

Температура воздуха в атмосфере является одной из ключевых
метеорологических величин, характер пространственного распределения
которой, наряду с давлением, влажностью и ветром, определяет атмосферные
движения, процессы тепло- и влагообмена, а также связанные с ними
изменения погоды. Поле температуры непрерывно изменяется как в
пространстве, так и во времени. Пределы этих изменений на Земле могут
быть весьма широкими, от 60 °С в тропических пустынях днем до -90 °С на
высокогорных плато Антарктиды полярной ночью. Вертикальное распределение
температуры позволяет разделить атмосферу на основные слои, такие как
тропосфера, стратосфера, мезосфера и термосфера, каждый из которых
характеризуется своим температурным режимом.

В тропосфере, которая является нижней и наиболее динамичной частью
атмосферы (до 9-17 км), сосредоточено более 4/5 всей массы атмосферного
воздуха и практически весь водяной пар. Именно здесь происходят основные
атмосферные процессы, определяющие погоду. В тропосфере температура в
целом падает с высотой, примерно на 0.65 °С на каждые 100 метров. Однако
наблюдаются и более сложные закономерности распределения температуры,
включая периодические и непериодические изменения, а также температурные
инверсии.

\hypertarget{ux43fux435ux440ux438ux43eux434ux438ux447ux435ux441ux43aux438ux435-ux438ux437ux43cux435ux43dux435ux43dux438ux44f-ux442ux435ux43cux43fux435ux440ux430ux442ux443ux440ux44b}{%
\subsection{Периодические Изменения
Температуры}\label{ux43fux435ux440ux438ux43eux434ux438ux447ux435ux441ux43aux438ux435-ux438ux437ux43cux435ux43dux435ux43dux438ux44f-ux442ux435ux43cux43fux435ux440ux430ux442ux443ux440ux44b}}

Периодические изменения температуры воздуха обусловлены регулярными
циклами поступления солнечной радиации и особенностями теплообмена с
подстилающей поверхностью.

\hypertarget{ux441ux443ux442ux43eux447ux43dux44bux439-ux445ux43eux434-ux442ux435ux43cux43fux435ux440ux430ux442ux443ux440ux44b-1}{%
\subsubsection{Суточный Ход
Температуры}\label{ux441ux443ux442ux43eux447ux43dux44bux439-ux445ux43eux434-ux442ux435ux43cux43fux435ux440ux430ux442ux443ux440ux44b-1}}

Температура воздуха имеет выраженный суточный ход. Максимум температуры
обычно наблюдается примерно через час после местного полудня, а минимум
--- перед восходом солнца. Этот ход определяется балансом прихода и
расхода тепла в приземном слое, включая радиационный баланс,
турбулентный теплообмен и фазовые превращения воды.

В ночное время, в результате радиационного выхолаживания деятельного
слоя почвы, приземные слои воздуха охлаждаются сильнее, чем вышележащие.
Это способствует развитию устойчивой стратификации и подавлению
турбулентного теплообмена, что приводит к формированию инверсий
температуры. Самые мощные радиационные инверсии обычно развиваются к
рассвету.

Суточные колебания температуры охватывают весь пограничный слой
атмосферы, который может достигать 500-1500 м, а нижние несколько
десятков метров называют приземным слоем. В этом приземном слое
температурный градиент всегда имеет наибольшие значения и самый
выраженный суточный ход. Влияние облачности существенно: она уменьшает
приток солнечной радиации днем и эффективное излучение ночью, сглаживая
суточную амплитуду температуры. Скорость ветра также влияет на
интенсивность турбулентного перемешивания и теплообмена, модифицируя
суточный ход температуры.

\hypertarget{ux433ux43eux434ux43eux432ux43eux439-ux445ux43eux434-ux442ux435ux43cux43fux435ux440ux430ux442ux443ux440ux44b-1}{%
\subsubsection{Годовой Ход
Температуры}\label{ux433ux43eux434ux43eux432ux43eux439-ux445ux43eux434-ux442ux435ux43cux43fux435ux440ux430ux442ux443ux440ux44b-1}}

Температура также подвержена годовым колебаниям. Максимум температуры не
совпадает по времени с максимумом солнечной радиации, а приходится на
более позднее время: на континентах запаздывание составляет примерно
месяц, а над океанами --- два-три месяца. Минимум температуры
наблюдается с аналогичным запаздыванием относительно минимума солнечной
радиации.

Основной причиной годового хода температуры является сезонная
изменчивость притока солнечной энергии и различия в теплофизических
свойствах суши и океана. Изотермы значительно отклоняются от широтных
кругов из-за влияния размещения континентов и океанов. Самые высокие и
низкие температуры воздуха наблюдаются над континентами. Например, над
материками Северного полушария зимой формируются обширные антициклоны с
высоким давлением и низкими температурами, тогда как летом над ними
образуются циклоны с низким давлением и высокими температурами.
Вследствие этого летом на материках циклонов образуется в 3,5 раза
больше, чем антициклонов, а зимой наоборот --- антициклонов в 6 раз
больше, чем циклонов. Эти сезонные изменения атмосферной циркуляции
проявляются в различной повторяемости и интенсивности циклонов и
антициклонов, смещении их траекторий, а также изменении географического
положения и интенсивности высотных фронтальных зон.

\hypertarget{ux43dux435ux43fux435ux440ux438ux43eux434ux438ux447ux435ux441ux43aux438ux435-ux438ux437ux43cux435ux43dux435ux43dux438ux44f-ux442ux435ux43cux43fux435ux440ux430ux442ux443ux440ux44b}{%
\subsection{Непериодические Изменения
Температуры}\label{ux43dux435ux43fux435ux440ux438ux43eux434ux438ux447ux435ux441ux43aux438ux435-ux438ux437ux43cux435ux43dux435ux43dux438ux44f-ux442ux435ux43cux43fux435ux440ux430ux442ux443ux440ux44b}}

Непериодические изменения температуры в атмосфере обусловлены
динамическими и физическими процессами, которые могут быть как
крупномасштабными, так и мезомасштабными.

\hypertarget{ux430ux434ux432ux435ux43aux442ux438ux432ux43dux44bux435-ux438ux437ux43cux435ux43dux435ux43dux438ux44f}{%
\subsubsection{Адвективные
Изменения}\label{ux430ux434ux432ux435ux43aux442ux438ux432ux43dux44bux435-ux438ux437ux43cux435ux43dux435ux43dux438ux44f}}

Адвекция --- это изменение температуры в фиксированной точке
пространства в результате переноса в нее воздушных масс с другой
температурой. Этот процесс является результатом горизонтального
перемещения воздуха. В метеорологии различают геострофическую и
агеострофическую адвекцию, хотя обычно ограничиваются вычислением
геострофической адвекции. Например, в Северном полушарии области
адвекции тепла соответствует правый поворот ветра с высотой, а области
адвекции холода --- левый.

Адвекция тепла или холода играет важную (часто определяющую) роль в
формировании полей давления и других метеовеличин, особенно в зарождении
и развитии синоптических вихрей (циклонов и антициклонов) и атмосферных
фронтов. Например, адвекция холода и более сухого воздуха способствует
возникновению нового или углублению существующего тропического циклона
(ТЦ). В циклонах наблюдается адвекция тепла в передней части и адвекция
холода в тыловой, что способствует их развитию. Зимой облачность
способствует повышению температуры в циклоне и сохранению адвективного
притока холода, что ведет к более длительному существованию циклонов.

Адвекция не является единственным фактором изменения температуры; она
должна быть исключена при анализе трансформации воздушных масс.

\hypertarget{ux438ux437ux43cux435ux43dux435ux43dux438ux44f-ux441ux432ux44fux437ux430ux43dux43dux44bux435-ux441-ux432ux435ux440ux442ux438ux43aux430ux43bux44cux43dux44bux43cux438-ux434ux432ux438ux436ux435ux43dux438ux44fux43cux438-ux430ux434ux438ux430ux431ux430ux442ux438ux447ux435ux441ux43aux438ux435-ux438-ux43aux43eux43dux432ux435ux43aux442ux438ux432ux43dux44bux435}{%
\subsubsection{Изменения, Связанные с Вертикальными Движениями
(Адиабатические и
Конвективные)}\label{ux438ux437ux43cux435ux43dux435ux43dux438ux44f-ux441ux432ux44fux437ux430ux43dux43dux44bux435-ux441-ux432ux435ux440ux442ux438ux43aux430ux43bux44cux43dux44bux43cux438-ux434ux432ux438ux436ux435ux43dux438ux44fux43cux438-ux430ux434ux438ux430ux431ux430ux442ux438ux447ux435ux441ux43aux438ux435-ux438-ux43aux43eux43dux432ux435ux43aux442ux438ux432ux43dux44bux435}}

Изменение температуры при совершении работы сжатия или расширения
воздуха называется адиабатическим изменением. Это происходит, когда
воздух перемещается вертикально без притока или отдачи тепла. В
атмосфере, при вертикальных перемещениях, работа расширения или сжатия
значительно превосходит внешний приток тепла, поэтому в первом
приближении можно считать, что изменения состояния воздуха происходят
адиабатически.

Сухоадиабатический градиент температуры (\(\gamma_а\)) составляет
примерно 1 °С на каждые 100 м (или 10 °С/км). При адиабатическом подъеме
сухой воздух охлаждается на эту величину, а при опускании ---
нагревается. Для влажного насыщенного воздуха применяется
влажноадиабатический градиент (\(\gamma_ва\)), который меньше
сухоадиабатического из-за выделения скрытой теплоты конденсации при
подъеме.

Конвекция --- это перенос тепла вверх или вниз за счет вертикальных
движений воздуха. В циклонах, где наблюдаются восходящие движения
воздуха (w \textgreater{} 0), температура воздуха на всех уровнях
уменьшается, а массовая доля водяного пара увеличивается. Это ведет к
формированию облачности и осадков. Напротив, нисходящие движения воздуха
(w \textless{} 0), характерные для антициклонов, вызывают адиабатическое
нагревание и осушение воздуха, что способствует рассеиванию облаков.

Вертикальные движения играют большую роль в непериодических изменениях
полей температуры и геопотенциала, а также ветра, особенно в тропосфере
и нижней мезосфере.

\hypertarget{ux442ux440ux430ux43dux441ux444ux43eux440ux43cux430ux446ux438ux44f-ux432ux43eux437ux434ux443ux448ux43dux44bux445-ux43cux430ux441ux441}{%
\subsubsection{Трансформация Воздушных
Масс}\label{ux442ux440ux430ux43dux441ux444ux43eux440ux43cux430ux446ux438ux44f-ux432ux43eux437ux434ux443ux448ux43dux44bux445-ux43cux430ux441ux441}}

Трансформация воздушной массы --- это непрерывное изменение ее свойств
(температуры, влажности, устойчивости, облачности и осадков) под
влиянием взаимодействия с новой подстилающей поверхностью и изменившихся
условий радиационного баланса. Этот процесс продолжается до тех пор,
пока в новом районе не будет достигнута температура равновесия. Скорость
трансформации обычно максимальна в первые сутки после вторжения новой
воздушной массы (например, 4-5 °С в сутки, над океаном до 10-15 °С в
сутки), затем уменьшается.

Трансформация может быть вызвана изменением радиационного баланса
(например, из-за изменения облачности), а также процессами испарения и
конденсации.

\hypertarget{ux440ux430ux434ux438ux430ux446ux438ux43eux43dux43dux44bux435-ux43fux440ux438ux442ux43eux43aux438ux43eux442ux442ux43eux43aux438-ux442ux435ux43fux43bux430}{%
\subsubsection{Радиационные Притоки/Оттоки
Тепла}\label{ux440ux430ux434ux438ux430ux446ux438ux43eux43dux43dux44bux435-ux43fux440ux438ux442ux43eux43aux438ux43eux442ux442ux43eux43aux438-ux442ux435ux43fux43bux430}}

Изменение температуры воздуха происходит также за счет радиационных
процессов, то есть поглощения или излучения тепловой энергии. Воздух
слабо поглощает солнечную радиацию, но радиационный приток тепла в
частицу воздуха происходит в длинноволновом диапазоне. Облачность
уменьшает как приток солнечной радиации к земной поверхности, так и ее
эффективное излучение, влияя на результирующий радиационный баланс.
Водяной пар является одним из главных факторов, влияющих на радиационный
баланс приземного слоя и формирование острова тепла в городах.

\hypertarget{ux444ux430ux437ux43eux432ux44bux435-ux43fux440ux435ux432ux440ux430ux449ux435ux43dux438ux44f-ux432ux43eux434ux44b}{%
\subsubsection{Фазовые Превращения
Воды}\label{ux444ux430ux437ux43eux432ux44bux435-ux43fux440ux435ux432ux440ux430ux449ux435ux43dux438ux44f-ux432ux43eux434ux44b}}

При конденсации водяного пара выделяется скрытая теплота, а при
испарении --- поглощается. Это является значительным источником или
стоком тепла для воздушной массы и существенно влияет на ее температуру.
Этот процесс особенно важен в облаках и туманах. Например, при
образовании осадков в циклоне, выделяющаяся скрытая теплота может
существенно повысить температуру.

\hypertarget{ux442ux443ux440ux431ux443ux43bux435ux43dux442ux43dux44bux439-ux43eux431ux43cux435ux43d-ux442ux435ux43fux43bux430}{%
\subsubsection{Турбулентный Обмен
Тепла}\label{ux442ux443ux440ux431ux443ux43bux435ux43dux442ux43dux44bux439-ux43eux431ux43cux435ux43d-ux442ux435ux43fux43bux430}}

Турбулентный теплообмен --- это важнейший процесс переноса тепла в
воздухе, особенно вблизи подстилающей поверхности. Он возникает из-за
случайных коротких порывов ветра внутри воздушной частицы. В отличие от
молекулярной теплопроводности, которая очень мала, турбулентный
теплообмен имеет огромную скорость. Коэффициент турбулентной
температуропроводности сильно зависит от внешних условий и вертикального
градиента температуры. В приземном подслое (около 30-50 м) коэффициент
турбулентного обмена уменьшается по мере приближения к земной
поверхности.

Мгновенные значения метеорологических величин подвержены беспорядочным
колебаниям (пульсациям) в турбулентной атмосфере, что требует
использования усредненных уравнений гидротермодинамики. Составляющие
турбулентного потока тепла (\(P_x, P_y, P_z\)) выражаются через
усредненные произведения пульсаций вертикальных скоростей и температуры
(например, \(P_z = \rho C_p \overline{w'\theta'}\)). Эти потоки могут
быть параметризованы через градиент среднего значения потенциальной
температуры и коэффициенты турбулентного обмена.

\hypertarget{ux430ux43dux442ux440ux43eux43fux43eux433ux435ux43dux43dux44bux435-ux444ux430ux43aux442ux43eux440ux44b}{%
\subsubsection{Антропогенные
Факторы}\label{ux430ux43dux442ux440ux43eux43fux43eux433ux435ux43dux43dux44bux435-ux444ux430ux43aux442ux43eux440ux44b}}

Человеческая деятельность, особенно в больших городах, оказывает
значительное влияние на метеорологический режим. Города становятся
``островами тепла'', где температура воздуха на 2-3°С выше, чем в
окружающей сельской местности. Это связано с повышением температуры за
счет обогрева зданий и тепловыделения транспорта/предприятий,
уменьшением испарения (асфальтирование, застройка), и изменением
альбедо. Дополнительное количество водяного пара, образующееся при
сжигании топлива, также играет важную роль в изменении радиационного и
термического режима города, а также в формировании туманов. Загрязнение
воздуха способствует ухудшению видимости и образованию туманов и дымок.

\hypertarget{ux438ux43dux432ux435ux440ux441ux438ux438-ux442ux435ux43cux43fux435ux440ux430ux442ux443ux440ux44b}{%
\subsection{Инверсии
Температуры}\label{ux438ux43dux432ux435ux440ux441ux438ux438-ux442ux435ux43cux43fux435ux440ux430ux442ux443ux440ux44b}}

Инверсия температуры --- это слой воздуха, в котором температура
возрастает с высотой, что является противоположностью обычного падения
температуры с высотой в тропосфере. Слой, где температура с высотой не
меняется, называется изотермией. Как изотермия, так и инверсия являются
предельно устойчивыми слоями атмосферы. Инверсии подавляют
турбулентность, препятствуя проникновению тепла, влаги или примесей
через них.

\hypertarget{ux432ux438ux434ux44b-ux438ux43dux432ux435ux440ux441ux438ux439-ux438-ux43cux435ux445ux430ux43dux438ux437ux43cux44b-ux43eux431ux440ux430ux437ux43eux432ux430ux43dux438ux44f}{%
\subsubsection{Виды Инверсий и Механизмы
Образования}\label{ux432ux438ux434ux44b-ux438ux43dux432ux435ux440ux441ux438ux439-ux438-ux43cux435ux445ux430ux43dux438ux437ux43cux44b-ux43eux431ux440ux430ux437ux43eux432ux430ux43dux438ux44f}}

\begin{enumerate}
\def\labelenumi{\arabic{enumi}.}
\tightlist
\item
  \textbf{Приземные (Радиационные) Инверсии}:

  \begin{itemize}
  \tightlist
  \item
    Формируются в результате охлаждения подстилающей поверхности и
    прилегающих слоев воздуха за счет эффективного излучения, особенно
    ночью, при ясном небе, слабом ветре и наличии снежного покрова.
  \item
    Наиболее мощные радиационные инверсии наблюдаются к рассвету, когда
    охлаждение достигает максимума.
  \item
    Играют ключевую роль в образовании радиационных туманов. В условиях
    инверсии ослабленный турбулентный обмен способствует тому, что
    нижний слой воздуха охлаждается сильнее, чем вышележащие.
  \end{itemize}
\item
  \textbf{Приподнятые Инверсии}:

  \begin{itemize}
  \tightlist
  \item
    \textbf{Инверсии оседания (Субсиденции)}: Образуются в антициклонах
    в результате нисходящих движений воздуха. Опускающийся воздух
    адиабатически нагревается, что приводит к формированию устойчивого
    слоя. Эти инверсии способствуют сгущению изэнтропических
    поверхностей и подавлению конвекции. Пассатные инверсии, например,
    препятствуют развитию мощной конвекции над субтропическими океанами,
    что объясняет наличие пустынь вдоль побережий.
  \item
    \textbf{Фронтальные Инверсии}: Возникают на атмосферных фронтах,
    когда теплый воздух натекает на клин холодного воздуха. Фронтальные
    слои, если они становятся практически горизонтальными, также могут
    формировать слои инверсии.
  \item
    \textbf{Динамические Инверсии}: Могут формироваться вблизи уровня
    максимального ветра, в том числе вблизи оси струйного течения, где
    воздух подтекает со всех сторон, увеличивая вертикальный градиент
    потенциальной температуры.
  \end{itemize}
\end{enumerate}

Инверсии существенно влияют на погодные условия, так как ограничивают
вертикальное перемешивание воздуха, способствуя накоплению загрязняющих
веществ и образованию туманов.

ewpage

\hypertarget{ux442ux440ux43eux43fux43eux43fux430ux443ux437ux430-ux432ux44bux441ux43eux442ux430-ux438-ux442ux435ux43cux43fux435ux440ux430ux442ux443ux440ux430}{%
\section{Тропопауза: Высота и
Температура}\label{ux442ux440ux43eux43fux43eux43fux430ux443ux437ux430-ux432ux44bux441ux43eux442ux430-ux438-ux442ux435ux43cux43fux435ux440ux430ux442ux443ux440ux430}}

Тропопауза --- это переходный слой атмосферы, который разделяет
тропосферу и стратосферу. Она является критически важной границей,
поскольку в тропосфере температура, как правило, убывает с высотой,
тогда как в стратосфере наблюдается общее повышение температуры.

\hypertarget{ux432ux44bux441ux43eux442ux430-ux442ux440ux43eux43fux43eux43fux430ux443ux437ux44b}{%
\subsection{Высота
Тропопаузы}\label{ux432ux44bux441ux43eux442ux430-ux442ux440ux43eux43fux43eux43fux430ux443ux437ux44b}}

Высота тропопаузы значительно варьируется в зависимости от
географической широты и времени года.

\begin{itemize}
\tightlist
\item
  \textbf{Над экваториальными и тропическими широтами:} Тропопауза
  располагается на наибольших высотах, примерно от 16 до 17 км, а в
  некоторых случаях достигая 20 км.
\item
  \textbf{Над умеренными и полярными широтами:} Высота тропопаузы
  значительно ниже, колеблясь в диапазоне от 8 до 16 км, при этом над
  полярными областями она может опускаться до 9-10 км.
\end{itemize}

Эта зависимость от широты и времени года обусловлена термическим режимом
нижележащей тропосферы, где интенсивный конвективный перенос тепла в
низких широтах приводит к более высокому положению инверсионного слоя,
образующего тропопаузу.

\hypertarget{ux442ux435ux43cux43fux435ux440ux430ux442ux443ux440ux430-ux442ux440ux43eux43fux43eux43fux430ux443ux437ux44b}{%
\subsection{Температура
Тропопаузы}\label{ux442ux435ux43cux43fux435ux440ux430ux442ux443ux440ux430-ux442ux440ux43eux43fux43eux43fux430ux443ux437ux44b}}

Температура в тропопаузе также демонстрирует существенные широтные и
сезонные вариации.

\begin{itemize}
\tightlist
\item
  \textbf{Над экваториальными и тропическими широтами:} Тропопауза
  является наиболее холодной частью атмосферы по вертикали, с
  температурой, достигающей -80 °С\ldots-85 °С. Несмотря на более
  высокое положение, низкие температуры в экваториальной зоне
  обусловлены продолжающимся понижением температуры воздуха до этого
  уровня.
\item
  \textbf{Над умеренными и полярными широтами:} Температура тропопаузы,
  как правило, выше, чем в тропиках. Она может варьироваться от -45 °С
  до -75 °С, в зависимости от широты и времени года. В высоких широтах
  северного полушария зимой температура в стратосфере выше тропопаузы на
  высоте 30 км достигает -65 °С\ldots-75 °С, что указывает на
  аналогичные или близкие значения температуры самой тропопаузы. В
  низких широтах сезонные различия температуры на уровне тропопаузы
  могут достигать 5-10 °С.
\end{itemize}

\hypertarget{ux441ux442ux440ux443ux43aux442ux443ux440ux43dux44bux435-ux43eux441ux43eux431ux435ux43dux43dux43eux441ux442ux438-ux438-ux434ux438ux43dux430ux43cux438ux43aux430}{%
\subsection{Структурные Особенности и
Динамика}\label{ux441ux442ux440ux443ux43aux442ux443ux440ux43dux44bux435-ux43eux441ux43eux431ux435ux43dux43dux43eux441ux442ux438-ux438-ux434ux438ux43dux430ux43cux438ux43aux430}}

Тропопауза не является непрерывным однородным слоем; ее структура может
быть весьма сложной, особенно в зонах интенсивных атмосферных процессов:

\begin{itemize}
\tightlist
\item
  \textbf{Множественность слоев:} В зонах планетарных высотных
  фронтальных зон (ПВФЗ) часто наблюдается не один, а два слоя
  тропопаузы: низкий, относительно теплый, расположенный на
  циклонической периферии, и более высокий, относительно холодный, на
  антициклонической периферии.
\item
  \textbf{Разрывы и наклоны:} В центральной части струйных течений,
  которые располагаются практически под самой тропопаузой, тропопауза
  имеет очень крутой наклон. Иногда в пределах 400-600 км слои
  тропопаузы могут располагаться один над другим или вовсе иметь
  разрывы.
\item
  \textbf{Связь с динамическими процессами:} Интенсивные конвективные
  движения, характерные для мощных кучево-дождевых облаков, могут
  проникать вплоть до тропопаузы и даже в нижнюю стратосферу. Высота
  тропопаузы также колеблется в связи с адвекцией тепла или холода.
\end{itemize}

ewpage

\hypertarget{ux443ux440ux430ux432ux43dux435ux43dux438ux435-ux442ux435ux43fux43bux43eux432ux43eux433ux43e-ux431ux430ux43bux430ux43dux441ux430-ux437ux435ux43cux43dux43eux439-ux43fux43eux432ux435ux440ux445ux43dux43eux441ux442ux438}{%
\section{Уравнение Теплового Баланса Земной
Поверхности}\label{ux443ux440ux430ux432ux43dux435ux43dux438ux435-ux442ux435ux43fux43bux43eux432ux43eux433ux43e-ux431ux430ux43bux430ux43dux441ux430-ux437ux435ux43cux43dux43eux439-ux43fux43eux432ux435ux440ux445ux43dux43eux441ux442ux438}}

Уравнение теплового баланса земной поверхности представляет собой
фундаментальное выражение закона сохранения энергии применительно к
тонкому поверхностному слою подстилающей поверхности. Оно постулирует,
что алгебраическая сумма всех потоков энергии на этой границе равна
нулю, определяя тем самым ее температуру. Это уравнение критически важно
для понимания формирования термического и влажностного режимов
деятельного слоя почвы и атмосферы, а также процессов, происходящих в
них, включая образование туманов и облаков.

\hypertarget{ux444ux43eux440ux43cux443ux43bux438ux440ux43eux432ux43aux430-ux443ux440ux430ux432ux43dux435ux43dux438ux44f}{%
\subsection{Формулировка
Уравнения}\label{ux444ux43eux440ux43cux443ux43bux438ux440ux43eux432ux43aux430-ux443ux440ux430ux432ux43dux435ux43dux438ux44f}}

В наиболее общей форме уравнение теплового баланса подстилающей
поверхности можно записать как:

\(R_s + LE + P + G = 0\)

где:

\begin{itemize}
\tightlist
\item
  \(R_s\) -- радиационный баланс подстилающей поверхности.
\item
  \(LE\) -- затраты тепла на испарение (скрытое тепло).
\item
  \(P\) -- турбулентный поток явного тепла между поверхностью и
  атмосферой.
\item
  \(G\) -- теплообмен с более глубокими слоями подстилающей поверхности
  (почвой или водой).
\end{itemize}

Все величины выражаются в единицах энергии на единицу площади в единицу
времени, например, в Вт/м². Знаки потоков обычно определяются как
положительные, если энергия поступает к поверхности, и отрицательные,
если она уходит от нее. Однако, в некоторых контекстах, потоки из
поверхности могут быть обозначены как положительные.

\hypertarget{ux43aux43eux43cux43fux43eux43dux435ux43dux442ux44b-ux442ux435ux43fux43bux43eux432ux43eux433ux43e-ux431ux430ux43bux430ux43dux441ux430}{%
\subsection{Компоненты Теплового
Баланса}\label{ux43aux43eux43cux43fux43eux43dux435ux43dux442ux44b-ux442ux435ux43fux43bux43eux432ux43eux433ux43e-ux431ux430ux43bux430ux43dux441ux430}}

\hypertarget{ux440ux430ux434ux438ux430ux446ux438ux43eux43dux43dux44bux439-ux431ux430ux43bux430ux43dux441-r_s}{%
\subsubsection{\texorpdfstring{Радиационный Баланс
(\(R_s\))}{Радиационный Баланс (R\_s)}}\label{ux440ux430ux434ux438ux430ux446ux438ux43eux43dux43dux44bux439-ux431ux430ux43bux430ux43dux441-r_s}}

Радиационный баланс земной поверхности -- это разность между поглощенной
солнечной (коротковолновой) радиацией и суммарной потерей тепла на
собственное излучение подстилающей поверхности (эффективное излучение).
Он определяется совместным действием коротковолновой и длинноволновой
радиации.

\begin{itemize}
\tightlist
\item
  \textbf{Коротковолновая радиация:} Включает приток прямой, рассеянной
  и суммарной солнечной радиации. Ее поглощение зависит от типа
  подстилающей поверхности, ее экспозиции к Солнцу, наклона и альбедо.
  Например, альбедо снега, льда и облаков значительно выше, чем у других
  поверхностей. Облачность уменьшает поступление солнечной радиации к
  земной поверхности.
\item
  \textbf{Длинноволновая радиация:} Земная поверхность, поглощая
  солнечную радиацию, нагревается и затем сама излучает в инфракрасном
  диапазоне. Это излучение сильно поглощается атмосферой, которая в свою
  очередь излучает обратно к поверхности (противоизлучение атмосферы),
  уменьшая потери тепла с поверхности и создавая парниковый эффект.
  Облака, особенно нижнего яруса, также уменьшают эффективное излучение
  ночью. Увеличение содержания водяного пара (парникового газа)
  сопровождается уменьшением эффективного потока излучения и потерь
  тепла земной поверхностью. Эффективное излучение обычно положительно,
  то есть направлено вверх.
\end{itemize}

\hypertarget{ux442ux443ux440ux431ux443ux43bux435ux43dux442ux43dux44bux439-ux442ux435ux43fux43bux43eux43eux431ux43cux435ux43d-p}{%
\subsubsection{\texorpdfstring{Турбулентный Теплообмен
(\(P\))}{Турбулентный Теплообмен (P)}}\label{ux442ux443ux440ux431ux443ux43bux435ux43dux442ux43dux44bux439-ux442ux435ux43fux43bux43eux43eux431ux43cux435ux43d-p}}

Этот компонент характеризует обмен явным (ощутимым) теплом между
подстилающей поверхностью и прилегающим к ней слоем воздуха за счет
турбулентного перемешивания. Его величина зависит от разности температур
между поверхностью и воздухом, а также от интенсивности турбулентного
обмена, который, в свою очередь, связан со скоростью ветра и
стратификацией атмосферы. Например, в пустынях радиационный баланс в
значительной степени уравновешивается турбулентным потоком тепла в
атмосферу.

\hypertarget{ux437ux430ux442ux440ux430ux442ux44b-ux442ux435ux43fux43bux430-ux43dux430-ux438ux441ux43fux430ux440ux435ux43dux438ux435-le}{%
\subsubsection{\texorpdfstring{Затраты Тепла на Испарение
(\(LE\))}{Затраты Тепла на Испарение (LE)}}\label{ux437ux430ux442ux440ux430ux442ux44b-ux442ux435ux43fux43bux430-ux43dux430-ux438ux441ux43fux430ux440ux435ux43dux438ux435-le}}

Этот член отражает энергию, затрачиваемую на фазовые переходы воды
(испарение или сублимацию), которая затем переносится в атмосферу в виде
скрытого тепла. Чем более влажная поверхность, тем больше испарение, и
тем меньше ее максимальная температура. Следовательно, влажность
подстилающей поверхности (например, болота) является ключевым фактором,
определяющим значимость этого компонента. Процессы конденсации,
наоборот, выделяют скрытое тепло, изменяя тепловой баланс.

\hypertarget{ux442ux435ux43fux43bux43eux43eux431ux43cux435ux43d-ux441-ux43fux43eux434ux441ux442ux438ux43bux430ux44eux449ux435ux439-ux43fux43eux432ux435ux440ux445ux43dux43eux441ux442ux44cux44e-g}{%
\subsubsection{\texorpdfstring{Теплообмен с Подстилающей Поверхностью
(\(G\))}{Теплообмен с Подстилающей Поверхностью (G)}}\label{ux442ux435ux43fux43bux43eux43eux431ux43cux435ux43d-ux441-ux43fux43eux434ux441ux442ux438ux43bux430ux44eux449ux435ux439-ux43fux43eux432ux435ux440ux445ux43dux43eux441ux442ux44cux44e-g}}

Этот компонент описывает поток тепла из тонкого поверхностного слоя в
более глубокие слои почвы или воды. Его направление и величина зависят
от термических свойств подстилающей поверхности (теплоемкость,
температуропроводность). Например, более влажная почва нагревается и
охлаждается медленнее сухой, но суточные колебания температуры
распространяются в ней глубже. Время установления радиационного
равновесия в слое в 1 см для различных материалов варьируется от секунд
(воздух) до часов (вода, лед, камень).

\hypertarget{ux432ux43bux438ux44fux43dux438ux435-ux438-ux437ux43dux430ux447ux435ux43dux438ux435}{%
\subsection{Влияние и
Значение}\label{ux432ux43bux438ux44fux43dux438ux435-ux438-ux437ux43dux430ux447ux435ux43dux438ux435}}

Уравнение теплового баланса является ключевым для прогнозирования
температуры подстилающей поверхности. Изменение каждого из его
компонентов влияет на итоговую температуру и, как следствие, на
атмосферные процессы.

\begin{itemize}
\tightlist
\item
  \textbf{Суточный и годовой ход:} Радиационный баланс претерпевает
  значительные суточные и годовые колебания, определяя соответствующий
  ход температуры поверхности. Максимум температуры обычно наблюдается в
  период максимального радиационного баланса (местное полудние), но
  может быть смещен в зависимости от свойств поверхности.
\item
  \textbf{Тип подстилающей поверхности:} Характер подстилающей
  поверхности (вода, суша, растительность, урбанизированные территории)
  сильно влияет на амплитуду тепловых волн и распределение энергии между
  составляющими баланса. Например, городские ``острова тепла''
  характеризуются повышением температуры, что связано с изменениями
  радиационного режима, уменьшением испарения и тепловыделением от
  антропогенных источников.
\item
  \textbf{Взаимодействие с атмосферой:} Потоки тепла и влаги от
  поверхности в атмосферу (явное и скрытое тепло) играют важную роль в
  изменении температуры и других метеорологических величин, а также в
  формировании туманов и облаков. Это взаимодействие, в свою очередь,
  влияет на радиационный баланс через облачность и влажность воздуха.
\end{itemize}

Понимание и точное моделирование каждого компонента уравнения теплового
баланса земной поверхности является сложной задачей, требующей учета
множества взаимосвязанных физических процессов и факторов.

ewpage

\hypertarget{ux443ux440ux430ux432ux43dux435ux43dux438ux435-ux442ux435ux43fux43bux43eux432ux43eux433ux43e-ux431ux430ux43bux430ux43dux441ux430-ux430ux442ux43cux43eux441ux444ux435ux440ux44b-ux438-ux441ux438ux441ux442ux435ux43cux44b-ux437ux435ux43cux43bux44f---ux430ux442ux43cux43eux441ux444ux435ux440ux430}{%
\section{Уравнение теплового баланса атмосферы и системы Земля -
атмосфера}\label{ux443ux440ux430ux432ux43dux435ux43dux438ux435-ux442ux435ux43fux43bux43eux432ux43eux433ux43e-ux431ux430ux43bux430ux43dux441ux430-ux430ux442ux43cux43eux441ux444ux435ux440ux44b-ux438-ux441ux438ux441ux442ux435ux43cux44b-ux437ux435ux43cux43bux44f---ux430ux442ux43cux43eux441ux444ux435ux440ux430}}

\hypertarget{ux43eux431ux449ux438ux435-ux43fux43eux43bux43eux436ux435ux43dux438ux44f}{%
\subsection{Общие
положения}\label{ux43eux431ux449ux438ux435-ux43fux43eux43bux43eux436ux435ux43dux438ux44f}}

Динамическая метеорология изучает атмосферные процессы, опираясь на
общие законы физики, включая гидромеханику и термодинамику. Движение
воздуха в атмосфере обусловлено неравномерным распределением давления,
которое, в свою очередь, является результатом процессов теплообмена в
атмосфере и на ее границе с Землей. Атмосферные движения и процессы
тепло- и влагообмена представляют собой ключевые факторы, определяющие
погоду и климат.

\hypertarget{ux443ux440ux430ux432ux43dux435ux43dux438ux435-ux442ux435ux43fux43bux43eux432ux43eux433ux43e-ux431ux430ux43bux430ux43dux441ux430-ux430ux442ux43cux43eux441ux444ux435ux440ux44b}{%
\subsection{Уравнение теплового баланса
атмосферы}\label{ux443ux440ux430ux432ux43dux435ux43dux438ux435-ux442ux435ux43fux43bux43eux432ux43eux433ux43e-ux431ux430ux43bux430ux43dux441ux430-ux430ux442ux43cux43eux441ux444ux435ux440ux44b}}

В термодинамике атмосферы широко применяются выводы из первого начала
общей термодинамики, выражающие закон сохранения энергии. Изменение
количества тепловой энергии в единице объема воздуха за единицу времени
называется притоком тепла. Уравнение притока тепла, являясь
математическим выражением закона сохранения энергии, может быть
представлено в виде:

\(C_v dT + p dV = \Delta Q\)

Это уравнение также может быть преобразовано в форму, часто используемую
в метеорологии, не содержащую неизмеряемого удельного объема:

\(\rho C_p \frac{dT}{dt} - \frac{dP}{dt} = Q\) (в несколько ином виде
\(\frac{dT}{dt} = \frac{RT}{P} \frac{dP}{dt} + \frac{Q}{C_p}\))

где:

\begin{itemize}
\tightlist
\item
  \(dT\) -- изменение температуры
\item
  \(dV\) -- изменение объема
\item
  \(\Delta Q\) -- приток тепла
\item
  \(C_v\) -- удельная теплоемкость воздуха при постоянном объеме
  (приблизительно 718 Дж/(кг·К) для сухого воздуха)
\item
  \(C_p\) -- удельная теплоемкость воздуха при постоянном давлении
  (приблизительно 1005 Дж/(кг·К) для сухого воздуха)
\item
  \(P\) -- давление
\item
  \(R\) -- удельная газовая постоянная воздуха
\item
  \(\rho\) -- плотность воздуха
\item
  \(Q\) -- количество тепла, сообщаемое извне единице массы воздуха за
  единицу времени
\end{itemize}

Физический смысл членов, входящих в уравнение притока тепла, позволяет
учесть различные процессы:

\begin{enumerate}
\def\labelenumi{\arabic{enumi}.}
\tightlist
\item
  \textbf{Конвекция/Адвекция тепла}: Изменение температуры в точке,
  происходящее в результате переноса атмосферными потоками воздуха с
  другой температурой. Различают горизонтальный перенос (адвективное
  изменение температуры) и вертикальный перенос (конвективное изменение
  температуры).
\item
  \textbf{Адиабатические изменения}: Изменение температуры, вызванное
  совершением работы сжатия или расширения частицы воздуха при изменении
  атмосферного давления. В адиабатическом процессе, когда притоки тепла
  равны нулю, изменение температуры происходит только за счет совершения
  работы. При подъеме частицы ее температура понижается, а при опускании
  --- повышается. В первом приближении, изменение термодинамического
  состояния движущегося воздуха в атмосфере можно считать
  адиабатическим, пренебрегая притоком тепла извне. При адиабатических
  процессах потенциальная температура воздуха не изменяется.
\item
  \textbf{Радиационные процессы}: Приток тепла за счет радиационных
  процессов, происходящий преимущественно в длинноволновом диапазоне,
  так как воздух слабо поглощает солнечную радиацию.
\item
  \textbf{Фазовые переходы воды}: Приток или отток тепла при конденсации
  или испарении водяного пара (скрытая теплота). Эти процессы
  существенны в облаках и туманах.
\item
  \textbf{Молекулярная теплопроводность}: Приток тепла за счет
  молекулярной теплопроводности воздуха, существенный только в тончайшем
  приповерхностном слое.
\item
  \textbf{Турбулентный приток тепла}: Приток тепла из-за случайных
  колебаний ветра, характеризующийся турбулентным перемешиванием.
\end{enumerate}

Для идеальной атмосферы уравнение притока тепла может быть записано
через потенциальную температуру \(\theta\):

\(C_p \frac{d\theta}{dt} = \frac{Q}{\theta}\)

При пренебрежении пульсациями плотности воздуха, осредненное уравнение
притока тепла в турбулентной атмосфере также учитывает турбулентные
потоки тепла.

\hypertarget{ux442ux435ux43fux43bux43eux432ux43eux439-ux431ux430ux43bux430ux43dux441-ux441ux438ux441ux442ux435ux43cux44b-ux437ux435ux43cux43bux44f---ux430ux442ux43cux43eux441ux444ux435ux440ux430}{%
\subsection{Тепловой баланс системы Земля -
атмосфера}\label{ux442ux435ux43fux43bux43eux432ux43eux439-ux431ux430ux43bux430ux43dux441-ux441ux438ux441ux442ux435ux43cux44b-ux437ux435ux43cux43bux44f---ux430ux442ux43cux43eux441ux444ux435ux440ux430}}

Основными источниками лучистой энергии для атмосферы являются излучение
Солнца, излучение Земли и самой атмосферы. Лучистый теплообмен в
атмосфере складывается в результате поглощения и излучения
электромагнитных волн слоями воздуха, при этом часть лучистой энергии
превращается в тепловую. Водяной пар, углекислый газ и озон являются
основными поглотителями радиации в атмосфере.

\textbf{Радиационный баланс Земли как планеты}: Земля в целом находится
в состоянии радиационного равновесия: поток приходящей от Солнца
радиации уравновешен потоком отраженной коротковолновой и уходящей
длинноволновой радиации. Средняя температура Земли как абсолютно черного
тела в состоянии лучистого равновесия может быть оценена.

\textbf{Радиационный баланс подстилающей поверхности}: Поверхность Земли
не находится в состоянии радиационного равновесия; эффективное излучение
лишь частично компенсирует приток от поглощенной радиации. Радиационный
баланс подстилающей поверхности (\(R_s\)) определяется суммой
поглощенной солнечной радиации и суммарной потери тепла на излучение
(эффективное излучение), определяя скорость изменения энергии в
поверхностном слое. Этот баланс расходуется на испарение (затраты тепла
на испарение) и контактную теплопередачу от подстилающей поверхности в
атмосферу (турбулентный поток тепла в атмосферу), а также на поток тепла
в почву.

\textbf{Парниковый эффект}: Атмосфера, имеющая температуру, близкую к
температуре подстилающей поверхности, также излучает длинноволновую
радиацию. Часть этого излучения возвращается к поверхности Земли
(противоизлучение атмосферы), уменьшая потерю тепла Землей и приводя к
возникновению парникового эффекта. Парниковый эффект объясняет, почему
температура у поверхности Земли значительно выше, чем если бы Земля была
абсолютно черным телом без атмосферы.

\textbf{Влияние облачности}: Облака существенно влияют на радиационный
баланс земной поверхности. Они уменьшают приток солнечной радиации к
Земле и ее эффективное излучение, что приводит к увеличению
результирующего притока радиации к поверхности Земли в зимнее время в
умеренных и высоких широтах (когда радиационный баланс отрицателен) и
его уменьшению в летнее время (когда он положителен).

\textbf{Обмен тепла между Землей и атмосферой}: Лучистая энергия,
поступающая от Солнца и от Земли, усиливает молекулярное движение,
повышает температуру воздуха и влияет на атмосферное давление.
Неодинаковый нагрев различных участков Земли создает перепады
атмосферного давления, вызывая движение воздуха (ветер) и общую
циркуляцию атмосферы. Атмосфера действует как гигантская тепловая
машина, приводимая в действие лучистой энергией Солнца, перенося тепло в
явной (контактный теплообмен, турбулентные потоки) и скрытой (перенос
водяного пара и тепла конденсации) формах от низких широт к высоким,
сглаживая температурные различия между экватором и полюсами. Скрытая
теплота, выделяющаяся при конденсации водяного пара, оказывает
существенное влияние на сохранение разности температур и развитие
вихрей.

Таким образом, уравнение теплового баланса атмосферы и анализ теплового
баланса системы Земля-атмосфера являются центральными для понимания
динамики атмосферных процессов, формирования погодных явлений и климата
планеты.

ewpage

\hypertarget{ux443ux441ux43bux43eux432ux438ux44f-ux444ux430ux437ux43eux432ux44bux445-ux43fux435ux440ux435ux445ux43eux434ux43eux432-ux432ux43eux434ux44b-ux432-ux430ux442ux43cux43eux441ux444ux435ux440ux435}{%
\section{Условия Фазовых Переходов Воды в
Атмосфере}\label{ux443ux441ux43bux43eux432ux438ux44f-ux444ux430ux437ux43eux432ux44bux445-ux43fux435ux440ux435ux445ux43eux434ux43eux432-ux432ux43eux434ux44b-ux432-ux430ux442ux43cux43eux441ux444ux435ux440ux435}}

Фазовые переходы воды в атмосфере -- это ключевые процессы, определяющие
погоду и климат, поскольку они связаны с обменом тепла и влаги. Вода в
атмосфере существует в трех основных фазовых состояниях: газообразном
(водяной пар), жидком (капли воды) и твердом (кристаллы льда).

\hypertarget{ux432ux43eux434ux44fux43dux43eux439-ux43fux430ux440-ux438-ux43dux430ux441ux44bux449ux435ux43dux438ux435}{%
\subsection{Водяной Пар и
Насыщение}\label{ux432ux43eux434ux44fux43dux43eux439-ux43fux430ux440-ux438-ux43dux430ux441ux44bux449ux435ux43dux438ux435}}

Водяной пар является переменным компонентом атмосферного воздуха и его
количество может значительно меняться. Атмосферный воздух, содержащий
водяной пар, может рассматриваться как смесь сухого воздуха и водяного
пара. Важнейшей характеристикой влажности воздуха, определяющей
количество водяного пара, является его парциальное давление, также
известное как упругость водяного пара.

\textbf{Насыщение} -- это состояние, при котором воздух при данной
температуре и давлении содержит максимально возможное количество
водяного пара. Парциальное давление насыщенного водяного пара (упругость
насыщения) зависит от температуры. При насыщении скорости испарения и
конденсации равны.

\hypertarget{ux443ux441ux43bux43eux432ux438ux44f-ux43aux43eux43dux434ux435ux43dux441ux430ux446ux438ux438-ux438-ux441ux443ux431ux43bux438ux43cux430ux446ux438ux438}{%
\subsection{Условия Конденсации и
Сублимации}\label{ux443ux441ux43bux43eux432ux438ux44f-ux43aux43eux43dux434ux435ux43dux441ux430ux446ux438ux438-ux438-ux441ux443ux431ux43bux438ux43cux430ux446ux438ux438}}

\textbf{Конденсация} -- это переход водяного пара в жидкое состояние, а
\textbf{сублимация} (или десублимация) -- это прямой переход из
газообразного состояния в твердое (кристаллы льда), минуя жидкую фазу.

Основными условиями для конденсации водяного пара в атмосфере являются:

\begin{itemize}
\tightlist
\item
  \textbf{Охлаждение воздуха до точки росы (или точки инея)}: Точка росы
  -- это температура, при которой воздух достигает состояния насыщения
  при данном содержании водяного пара и неизменном давлении. Если воздух
  охлаждается до этой температуры, пар становится насыщенным, и
  начинается конденсация. Для образования тумана требуется охлаждение
  воздуха до десятых долей градуса ниже точки росы.
\item
  \textbf{Наличие ядер конденсации}: Молекулы водяного пара для перехода
  в упорядоченное состояние (вода, лед) должны преодолеть высокий
  энергетический барьер. В атмосфере этот барьер преодолевается на
  поверхности аэрозольных частиц, которые называются атмосферными ядрами
  конденсации. Наимельчайшие устойчивые капли образуются только на
  гигроскопичных ядрах. На гигроскопичных ядрах конденсация может
  начинаться при относительной влажности ниже 100\%, например, на
  частицах поваренной соли при 78\%.
\item
  \textbf{Относительная влажность, приближающаяся к 100\%}: Конденсация
  начинается, когда относительная влажность воздуха (отношение
  фактического парциального давления водяного пара к давлению
  насыщенного водяного пара при той же температуре) приближается к
  100\%.
\end{itemize}

\textbf{Процессы охлаждения, приводящие к конденсации:}

\begin{itemize}
\tightlist
\item
  \textbf{Адиабатическое охлаждение}: При вертикальных перемещениях
  воздуха в атмосфере (например, при подъеме воздушной массы),
  вследствие больших перепадов давления, изменение термодинамического
  состояния происходит адиабатически. Адиабатический подъем
  сопровождается понижением температуры. Если поднимающийся воздух
  достигает состояния насыщения, это называется уровнем конденсации, и
  при дальнейшем подъеме происходит конденсация избыточного водяного
  пара. При этом выделяется скрытая теплота парообразования
  (конденсации), которая составляет приблизительно 2,50·10⁶ Дж/кг и
  замедляет понижение температуры воздуха.
\item
  \textbf{Радиационное охлаждение}: Например, ночное радиационное
  охлаждение подстилающей поверхности приводит к понижению температуры
  приземного слоя воздуха ниже точки росы, что является основной
  причиной образования радиационных туманов.
\item
  \textbf{Смешение воздушных масс}: Смешение различных по температуре и
  влажности воздушных масс также может приводить к насыщению и
  конденсации, способствуя образованию туманов. Туманы испарения,
  например, возникают над водной поверхностью, температура которой
  значительно выше температуры окружающего воздуха, за счет поступления
  теплого воздуха с поверхности воды и его перемешивания с холодным
  окружающим воздухом.
\end{itemize}

\hypertarget{ux443ux441ux43bux43eux432ux438ux44f-ux438ux441ux43fux430ux440ux435ux43dux438ux44f}{%
\subsection{Условия
Испарения}\label{ux443ux441ux43bux43eux432ux438ux44f-ux438ux441ux43fux430ux440ux435ux43dux438ux44f}}

\textbf{Испарение} -- это процесс перехода жидкой или твердой воды в
газообразное состояние. Для атмосферы важны процессы испарения с
поверхности Земли (океанов, суши) и с поверхности облачных частиц.

\begin{itemize}
\tightlist
\item
  \textbf{Наличие дефицита насыщения}: Испарение происходит, когда
  парциальное давление водяного пара в воздухе ниже давления насыщенного
  пара при данной температуре. Чем больше разность между давлением
  насыщенного водяного пара и фактическим парциальным давлением, тем
  суше воздух и тем интенсивнее испарение.
\item
  \textbf{Температура}: Более высокая температура воды и воздуха
  способствует более интенсивному испарению.
\item
  \textbf{Движение воздуха (ветер)}: Ветер способствует уносу
  насыщенного воздуха от испаряющей поверхности, поддерживая градиент
  влажности и увеличивая скорость испарения.
\end{itemize}

В процессе испарения поглощается скрытая теплота парообразования, что
приводит к охлаждению испаряющей поверхности и окружающего воздуха.

\hypertarget{ux443ux441ux43bux43eux432ux438ux44f-ux43fux43bux430ux432ux43bux435ux43dux438ux44f-ux438-ux437ux430ux43cux435ux440ux437ux430ux43dux438ux44f}{%
\subsection{Условия Плавления и
Замерзания}\label{ux443ux441ux43bux43eux432ux438ux44f-ux43fux43bux430ux432ux43bux435ux43dux438ux44f-ux438-ux437ux430ux43cux435ux440ux437ux430ux43dux438ux44f}}

В атмосфере вода также может переходить между жидким и твердым
состояниями.

\begin{itemize}
\tightlist
\item
  \textbf{Плавление/Таяние}: Переход льда в жидкое состояние. Для этого
  требуется скрытая теплота плавления (таяния), составляющая 3,34·10⁵
  Дж/кг. Это происходит, например, при таянии снега или льда в теплых
  слоях атмосферы или на поверхности Земли.
\item
  \textbf{Замерзание/Кристаллизация}: Переход жидкой воды в лед. Этот
  процесс выделяет скрытую теплоту. Формирование ледяных кристаллов в
  облаках обычно происходит при температурах ниже 0°С.
\end{itemize}

\hypertarget{ux434ux438ux430ux433ux440ux430ux43cux43cux430-ux444ux430ux437ux43eux432ux44bux445-ux441ux43eux441ux442ux43eux44fux43dux438ux439-ux432ux43eux434ux44b-ux432-ux430ux442ux43cux43eux441ux444ux435ux440ux435}{%
\subsection{Диаграмма Фазовых Состояний Воды в
Атмосфере}\label{ux434ux438ux430ux433ux440ux430ux43cux43cux430-ux444ux430ux437ux43eux432ux44bux445-ux441ux43eux441ux442ux43eux44fux43dux438ux439-ux432ux43eux434ux44b-ux432-ux430ux442ux43cux43eux441ux444ux435ux440ux435}}

Хотя в предоставленных материалах не приведена конкретная P-T
(давление-температура) диаграмма фазовых состояний воды, описание
условий переходов позволяет воссоздать ее ключевые элементы:

\begin{itemize}
\tightlist
\item
  \textbf{Линии фазового равновесия}: Диаграмма отражает условия
  (сочетания давления и температуры), при которых две фазы воды могут
  сосуществовать в равновесии.

  \begin{itemize}
  \tightlist
  \item
    \textbf{Линия испарения/конденсации}: Отражает зависимость давления
    насыщенного водяного пара от температуры над плоской поверхностью
    воды. Выше этой линии пар конденсируется в воду, ниже -- вода
    испаряется.
  \item
    \textbf{Линия сублимации/десублимации}: Отражает зависимость
    давления насыщенного водяного пара от температуры над поверхностью
    льда. Выше этой линии пар десублимирует в лед, ниже -- лед
    сублимирует в пар.
  \item
    \textbf{Линия плавления/замерзания}: Отражает зависимость
    температуры плавления от давления. Для атмосферных процессов это
    обычно рассматривается при 0°C.
  \end{itemize}
\end{itemize}

Важно отметить, что в атмосфере, особенно для мелких капель, на давление
насыщенного пара также влияют кривизна поверхности и наличие
растворенных веществ (эффект Кельвина и Рауля). Например, над выпуклыми
поверхностями (маленькими каплями) давление насыщенного пара выше, чем
над плоскими, что объясняет, почему мельчайшие капли чистой воды быстро
испаряются, если они не образовались на гигроскопичных ядрах.

ewpage

\hypertarget{ux440ux43eux43bux44c-ux44fux434ux435ux440-ux43aux43eux43dux434ux435ux43dux441ux430ux446ux438ux438}{%
\section{Роль ядер
конденсации}\label{ux440ux43eux43bux44c-ux44fux434ux435ux440-ux43aux43eux43dux434ux435ux43dux441ux430ux446ux438ux438}}

В атмосферной метеорологии ядра конденсации играют фундаментальную роль
в процессах образования облаков, туманов и осадков, выступая в качестве
поверхности для перехода водяного пара из газообразного состояния в
жидкое или твердое.

\hypertarget{ux43eux43fux440ux435ux434ux435ux43bux435ux43dux438ux435-ux438-ux445ux430ux440ux430ux43aux442ux435ux440ux438ux441ux442ux438ux43aux438}{%
\subsection{Определение и
характеристики}\label{ux43eux43fux440ux435ux434ux435ux43bux435ux43dux438ux435-ux438-ux445ux430ux440ux430ux43aux442ux435ux440ux438ux441ux442ux438ux43aux438}}

Ядра конденсации --- это мельчайшие взвешенные частицы (аэрозоли),
присутствующие в воздухе, такие как дым, сажа, пепел, частицы морской
соли, пыльца, споры и микроорганизмы. Их размеры крайне малы, радиус
ядер конденсации обычно составляет от 10\^{}-3 до 10\^{}-1 мкм. В
зависимости от размера их подразделяют на три основные группы:

\begin{itemize}
\tightlist
\item
  \textbf{Ядра Айткена}: размером до 0,2 мкм.
\item
  \textbf{Крупные ядра}: размером до 1 мкм.
\item
  \textbf{Гигантские ядра}: размером более 1 мкм.
\end{itemize}

Эти частицы постоянно поступают в атмосферу с поверхности Земли.

\hypertarget{ux43cux435ux445ux430ux43dux438ux437ux43c-ux43aux43eux43dux434ux435ux43dux441ux430ux446ux438ux438}{%
\subsection{Механизм
конденсации}\label{ux43cux435ux445ux430ux43dux438ux437ux43c-ux43aux43eux43dux434ux435ux43dux441ux430ux446ux438ux438}}

Присутствие ядер конденсации критически важно для начала конденсации
водяного пара в атмосфере. В объеме, состоящем только из чистого
водяного пара, для перехода в жидкое или твердое состояние молекулам
пара необходимо преодолеть высокий энергетический барьер, связанный с
образованием свободной поверхности. Это требует значительного
перенасыщения (более 1\%).

Однако, в реальной атмосфере, благодаря наличию ядер конденсации, этот
барьер существенно снижается. Конденсация начинается на поверхности этих
частиц, когда воздух охлаждается и относительная влажность приближается
к 100\%. Для гигроскопичных частиц (способных поглощать водяной пар,
например, частицы поваренной соли NaCl), конденсация может начаться даже
при относительной влажности ниже 100\%, порой уже при 60-80\%. На ранних
стадиях, химический состав, электрический заряд и радиус ядра влияют на
давление насыщения водяного пара над его поверхностью. Однако, как
только радиус образовавшейся капли достигает примерно 1 мкм, давление
водяного пара на ее поверхности практически не отличается от давления
насыщения над плоской поверхностью чистой воды. Физически это
объясняется тем, что масса самого ядра ничтожно мала по сравнению с
массой сконденсированной на нем воды.

\hypertarget{ux432ux43bux438ux44fux43dux438ux435-ux43dux430-ux430ux442ux43cux43eux441ux444ux435ux440ux43dux44bux435-ux44fux432ux43bux435ux43dux438ux44f}{%
\subsection{Влияние на атмосферные
явления}\label{ux432ux43bux438ux44fux43dux438ux435-ux43dux430-ux430ux442ux43cux43eux441ux444ux435ux440ux43dux44bux435-ux44fux432ux43bux435ux43dux438ux44f}}

\hypertarget{ux43eux431ux440ux430ux437ux43eux432ux430ux43dux438ux435-ux43eux431ux43bux430ux43aux43eux432}{%
\subsubsection{Образование
облаков}\label{ux43eux431ux440ux430ux437ux43eux432ux430ux43dux438ux435-ux43eux431ux43bux430ux43aux43eux432}}

Ядра конденсации являются абсолютно необходимым условием для образования
водных капель в атмосфере. После того как воздух достигает состояния
насыщения, как правило, в результате адиабатического охлаждения при
восходящих движениях, водяной пар конденсируется на этих ядрах, формируя
облака. Восходящие движения являются основной причиной образования
слоистообразных облаков и, как следствие, пасмурной погоды в циклонах.

\hypertarget{ux43eux431ux440ux430ux437ux43eux432ux430ux43dux438ux435-ux442ux443ux43cux430ux43dux43eux432-ux438-ux434ux44bux43cux43eux43a}{%
\subsubsection{Образование туманов и
дымок}\label{ux43eux431ux440ux430ux437ux43eux432ux430ux43dux438ux435-ux442ux443ux43cux430ux43dux43eux432-ux438-ux434ux44bux43cux43eux43a}}

Туманы и дымки также образуются в процессе конденсации водяного пара на
ядрах конденсации. Например, роса и иней образуются, когда ночное
охлаждение подстилающей поверхности понижает ее температуру ниже точки
росы, вызывая конденсацию на травинках или частицах почвы.

\hypertarget{ux440ux43eux43bux44c-ux432-ux43eux441ux430ux434ux43aux430ux445}{%
\subsubsection{Роль в
осадках}\label{ux440ux43eux43bux44c-ux432-ux43eux441ux430ux434ux43aux430ux445}}

Облачные элементы, образовавшиеся на ядрах конденсации, в дальнейшем
растут за счет дополнительной конденсации, сублимации и гравитационной
коагуляции (слияния капель различного размера). Эти процессы приводят к
образованию более крупных капель или ледяных кристаллов, которые в
конечном итоге выпадают в виде осадков.

\hypertarget{ux440ux430ux441ux43fux440ux43eux441ux442ux440ux430ux43dux435ux43dux43dux43eux441ux442ux44c-ux438-ux432ux43bux438ux44fux43dux438ux435-ux43dux430-ux43fux440ux43eux433ux43dux43eux437}{%
\subsection{Распространенность и влияние на
прогноз}\label{ux440ux430ux441ux43fux440ux43eux441ux442ux440ux430ux43dux435ux43dux43dux43eux441ux442ux44c-ux438-ux432ux43bux438ux44fux43dux438ux435-ux43dux430-ux43fux440ux43eux433ux43dux43eux437}}

Ядра конденсации всегда присутствуют в атмосфере в избыточном
количестве; их более чем достаточно для начала образования капель воды в
любой точке, включая океаны и полярные области, где на каждую каплю
приходятся десятки или сотни таких ядер.

Однако, несмотря на их повсеместную необходимость для конденсации,
высокая концентрация ядер конденсации (например, в городах по сравнению
с сельской местностью) не является определяющим фактором для различий в
частоте образования туманов, дымок или облаков. Фактически, число
туманов в городах зачастую меньше, чем в их окрестностях. Это указывает
на то, что другие динамические и термодинамические процессы играют более
существенную роль в формировании и развитии этих явлений.

ewpage

\hypertarget{ux43eux431ux440ux430ux437ux43eux432ux430ux43dux438ux435-ux438-ux440ux43eux441ux442-ux437ux430ux440ux43eux434ux44bux448ux435ux432ux44bux445-ux43aux430ux43fux435ux43bux44c}{%
\section{Образование и рост зародышевых
капель}\label{ux43eux431ux440ux430ux437ux43eux432ux430ux43dux438ux435-ux438-ux440ux43eux441ux442-ux437ux430ux440ux43eux434ux44bux448ux435ux432ux44bux445-ux43aux430ux43fux435ux43bux44c}}

\hypertarget{ux43eux431ux440ux430ux437ux43eux432ux430ux43dux438ux435-ux437ux430ux440ux43eux434ux44bux448ux435ux432ux44bux445-ux43aux430ux43fux435ux43bux44c}{%
\subsection{1. Образование зародышевых
капель}\label{ux43eux431ux440ux430ux437ux43eux432ux430ux43dux438ux435-ux437ux430ux440ux43eux434ux44bux448ux435ux432ux44bux445-ux43aux430ux43fux435ux43bux44c}}

Для того чтобы конденсация водяного пара произошла в воздухе, необходимо
наличие в нем аэрозольных частиц, которые служат ядрами конденсации. В
атмосферных условиях, особенно высоко над землей вне зоны
непосредственного контакта с подстилающей поверхностью, основным и порой
единственным процессом, приводящим к образованию облаков (и,
следовательно, к инициации образования капель), является адиабатическое
охлаждение воздушной частицы при ее подъеме. При таком подъеме давление
в частице снижается, она расширяется, и ее температура уменьшается,
достигая точки росы и приводя к конденсации.

\hypertarget{ux444ux430ux43aux442ux43eux440ux44b-ux432ux43bux438ux44fux44eux449ux438ux435-ux43dux430-ux440ux43eux441ux442-ux437ux430ux440ux43eux434ux44bux448ux435ux432ux44bux445-ux43aux430ux43fux435ux43bux44c}{%
\subsection{2. Факторы, влияющие на рост зародышевых
капель}\label{ux444ux430ux43aux442ux43eux440ux44b-ux432ux43bux438ux44fux44eux449ux438ux435-ux43dux430-ux440ux43eux441ux442-ux437ux430ux440ux43eux434ux44bux448ux435ux432ux44bux445-ux43aux430ux43fux435ux43bux44c}}

Рост уже образовавшихся капель, а также дальнейшее развитие облачных
элементов до размеров, достаточных для выпадения осадков, определяется
сложным взаимодействием микрофизических и динамических факторов.

\hypertarget{ux43cux438ux43aux440ux43eux444ux438ux437ux438ux447ux435ux441ux43aux438ux435-ux43fux440ux43eux446ux435ux441ux441ux44b}{%
\subsubsection{2.1. Микрофизические
процессы}\label{ux43cux438ux43aux440ux43eux444ux438ux437ux438ux447ux435ux441ux43aux438ux435-ux43fux440ux43eux446ux435ux441ux441ux44b}}

\begin{itemize}
\tightlist
\item
  \textbf{Конденсация и сублимация}: Продолжение конденсации водяного
  пара на поверхности жидких капель или сублимации на ледяных кристаллах
  является одним из механизмов их роста. Этот процесс напрямую связан с
  достижением и поддержанием состояния насыщения воздуха.
\item
  \textbf{Гравитационная коагуляция}: Важнейший процесс, приводящий к
  образованию крупных облачных капель. Он заключается в слиянии облачных
  частиц, имеющих неодинаковую скорость падения в поле тяжести. Для
  выпадения осадков элементы облака должны вырасти до размеров, при
  которых скорость их падения превышает скорость восходящих движений.
\end{itemize}

\hypertarget{ux442ux435ux440ux43cux43eux434ux438ux43dux430ux43cux438ux447ux435ux441ux43aux438ux435-ux443ux441ux43bux43eux432ux438ux44f}{%
\subsubsection{2.2. Термодинамические
условия}\label{ux442ux435ux440ux43cux43eux434ux438ux43dux430ux43cux438ux447ux435ux441ux43aux438ux435-ux443ux441ux43bux43eux432ux438ux44f}}

\begin{itemize}
\tightlist
\item
  \textbf{Насыщение водяным паром}: Для инициации и поддержания роста
  капель критически важно достижение и сохранение состояния насыщения
  воздуха водяным паром, то есть относительная влажность должна
  достигать 100\%. Это достигается двумя основными путями:

  \begin{itemize}
  \tightlist
  \item
    \textbf{Увеличение содержания водяного пара}: Рост фактического
    парциального давления водяного пара (\texttt{e}) способствует
    насыщению.
  \item
    \textbf{Понижение температуры воздуха}: Снижение температуры
    приводит к уменьшению максимального давления насыщенного водяного
    пара (\texttt{E}), что также способствует достижению насыщения при
    существующем содержании пара.
  \end{itemize}
\item
  \textbf{Температура на верхней границе облака}: Этот параметр имеет
  значительное влияние на вероятность появления в облаке твердой фазы
  (кристаллов льда), что является ключевым для развития процесса
  Бержерона-Финдайзена, считающегося основным механизмом формирования
  осадков на Земле.
\end{itemize}

\hypertarget{ux434ux438ux43dux430ux43cux438ux447ux435ux441ux43aux438ux435-ux444ux430ux43aux442ux43eux440ux44b}{%
\subsubsection{2.3. Динамические
факторы}\label{ux434ux438ux43dux430ux43cux438ux447ux435ux441ux43aux438ux435-ux444ux430ux43aux442ux43eux440ux44b}}

\begin{itemize}
\tightlist
\item
  \textbf{Вертикальные движения воздуха}:

  \begin{itemize}
  \tightlist
  \item
    \textbf{Восходящие потоки (w \textgreater{} 0)}: Являются
    определяющим фактором охлаждения воздуха до насыщения и последующей
    конденсации. В циклонах, где преобладают восходящие движения, воздух
    охлаждается, что способствует образованию облаков и осадков.
    Восходящий поток также поддерживает образовавшиеся капли,
    способствуя накоплению водяного пара и увеличению водозапаса облака.
    В углубляющихся циклонах вертикальная скорость \texttt{w} растет с
    высотой, достигая максимума в нижней и средней тропосфере.
  \item
    \textbf{Нисходящие движения (w \textless{} 0)}: Хотя нисходящие
    движения обычно связаны с рассеиванием облаков из-за адиабатического
    нагрева, они могут играть роль в выпадении осадков из уже развитых
    облаков. Исследования показывают, что большая часть осадков выпадает
    преимущественно на второй стадии развития облака, когда восходящее
    движение сменяется нисходящим, и накопленная жидкая вода начинает
    быстро выпадать.
  \end{itemize}
\item
  \textbf{Адвекция и турбулентный обмен}:

  \begin{itemize}
  \tightlist
  \item
    \textbf{Адвективные притоки}: Перемещение воздушных масс (адвекция)
    играет значительную роль. Например, приток более холодного и сухого
    воздуха в циклон, смешиваясь с более теплым воздухом, приводит к
    конденсации водяного пара и образованию облака со значительной
    водностью. Адвекция холода способствует углублению циклонического
    вихря, тогда как адвекция тепла может ослаблять его.
  \item
    \textbf{Турбулентные притоки}: Наряду с адвективными и конвективными
    притоками тепла и водяного пара, турбулентный обмен также влияет на
    изменение относительной влажности воздуха и достижение состояния
    насыщения. Процессы вовлечения (энтрэйнмента) и последующего
    смешения воздушных масс облака с окружающей средой играют
    определяющую роль в увеличении размеров облаков и формировании
    осадков.
  \end{itemize}
\item
  \textbf{Толщина облака}: Большая вертикальная протяженность облака
  благоприятствует образованию осадков, так как это создает условия для
  продолжительной конденсации, сублимации и коагуляции.
\item
  \textbf{Тепло конденсации}: При фазовых превращениях воды
  (конденсации) выделяется скрытая теплота, которая играет роль
  внутреннего источника энергии. Это тепло способствует дальнейшему
  понижению давления в нижних слоях тропосферы и поддерживает развитие
  циклонических систем. В тропических циклонах оно служит важнейшим
  источником энергии, поступая из океана.
\end{itemize}

Таким образом, рост зародышевых капель и их последующая эволюция в
осадки является результатом сложного взаимодействия между
термодинамическими условиями, обеспечивающими насыщение и фазовые
переходы, и динамическими процессами, такими как вертикальные движения,
адвекция, турбулентность и коагуляция, которые способствуют накоплению
влаги и укрупнению частиц.

ewpage

Как профессиональный метеоролог, понимающий сложности фазовых переходов
в атмосфере, я готов предоставить детальную информацию по переохлаждению
капель, опираясь на представленные источники.

\hypertarget{ux43fux435ux440ux435ux43eux445ux43bux430ux436ux434ux435ux43dux438ux435-ux43aux430ux43fux435ux43bux44c}{%
\section{Переохлаждение
Капель}\label{ux43fux435ux440ux435ux43eux445ux43bux430ux436ux434ux435ux43dux438ux435-ux43aux430ux43fux435ux43bux44c}}

\hypertarget{ux43eux43fux440ux435ux434ux435ux43bux435ux43dux438ux435}{%
\subsection{Определение}\label{ux43eux43fux440ux435ux434ux435ux43bux435ux43dux438ux435}}

Переохлаждение -- это процесс охлаждения вещества ниже его обычной точки
замерзания без изменения его фазового состояния. В контексте атмосферы,
это относится к охлаждению жидких водяных капель до температур ниже 0°C,
при этом они остаются в жидком виде.

\hypertarget{ux443ux441ux43bux43eux432ux438ux44f-ux43eux431ux440ux430ux437ux43eux432ux430ux43dux438ux44f-ux438-ux441ux443ux449ux435ux441ux442ux432ux43eux432ux430ux43dux438ux44f}{%
\subsection{Условия Образования и
Существования}\label{ux443ux441ux43bux43eux432ux438ux44f-ux43eux431ux440ux430ux437ux43eux432ux430ux43dux438ux44f-ux438-ux441ux443ux449ux435ux441ux442ux432ux43eux432ux430ux43dux438ux44f}}

\hypertarget{ux442ux435ux43cux43fux435ux440ux430ux442ux443ux440ux43dux44bux435-ux443ux441ux43bux43eux432ux438ux44f}{%
\subsubsection{Температурные
Условия}\label{ux442ux435ux43cux43fux435ux440ux430ux442ux443ux440ux43dux44bux435-ux443ux441ux43bux43eux432ux438ux44f}}

Переохлажденные капли воды могут преобладать в облаках в температурном
диапазоне от 0°C до -10°C. Они также могут присутствовать в диапазоне от
-10°C до -20°C, хотя вероятность их наличия выше -20°C становится
значительно меньше, поскольку в таких условиях обычно преобладают
ледяные облака. Тем не менее, отдельные случаи переохлажденных капель
воды были зафиксированы даже при температурах ниже -40°C, особенно в
облаках Ci-Cs, связанных с Cb.

\hypertarget{ux432ux43bux438ux44fux43dux438ux435-ux440ux430ux437ux43cux435ux440ux430-ux43aux430ux43fux435ux43bux44c}{%
\subsubsection{Влияние Размера
Капель}\label{ux432ux43bux438ux44fux43dux438ux435-ux440ux430ux437ux43cux435ux440ux430-ux43aux430ux43fux435ux43bux44c}}

Размер капель играет критическую роль в их способности к переохлаждению.
Чем меньше капли переохлажденной воды, тем более низкие температуры
требуются для их замерзания. Следовательно, в холодных облаках более
крупные капли замерзают легче, в то время как мелкие капли могут
оставаться в переохлажденном состоянии. Этот аспект важен, так как он
влияет на механизмы образования осадков и обледенения.

\hypertarget{ux440ux43eux43bux44c-ux432-ux430ux442ux43cux43eux441ux444ux435ux440ux43dux44bux445-ux43fux440ux43eux446ux435ux441ux441ux430ux445-1}{%
\subsection{Роль в Атмосферных
Процессах}\label{ux440ux43eux43bux44c-ux432-ux430ux442ux43cux43eux441ux444ux435ux440ux43dux44bux445-ux43fux440ux43eux446ux435ux441ux441ux430ux445-1}}

\hypertarget{ux43eux431ux440ux430ux437ux43eux432ux430ux43dux438ux435-ux43eux441ux430ux434ux43aux43eux432-ux43cux435ux445ux430ux43dux438ux437ux43c-ux431ux435ux440ux436ux435ux440ux43eux43dux430-ux444ux438ux43dux434ux430ux439ux437ux435ux43dux430}{%
\subsubsection{Образование Осадков (Механизм
Бержерона-Финдайзена)}\label{ux43eux431ux440ux430ux437ux43eux432ux430ux43dux438ux435-ux43eux441ux430ux434ux43aux43eux432-ux43cux435ux445ux430ux43dux438ux437ux43c-ux431ux435ux440ux436ux435ux440ux43eux43dux430-ux444ux438ux43dux434ux430ux439ux437ux435ux43dux430}}

Переохлажденные капли являются ключевым элементом в механизме
образования осадков в холодных облаках, известном как процесс
Бержерона-Финдайзена. Поскольку давление насыщенного пара над водой
выше, чем над льдом при той же температуре, ледяные кристаллы в
смешанных облаках служат эффективными ядрами конденсации для
переохлажденной воды. Это приводит к тому, что в зоне низких температур
происходит как дополнительная конденсация на кристаллах, так и испарение
мелких переохлажденных капель воды, с последующим ростом ледяных
кристаллов. Этот процесс ``перегонки пара'' способствует быстрому росту
ледяных элементов, что является основой для формирования снега и, при
прохождении через слой теплых температур, дождя.

\hypertarget{ux43eux43fux430ux441ux43dux44bux435-ux44fux432ux43bux435ux43dux438ux44f-ux43fux43eux433ux43eux434ux44b}{%
\subsubsection{Опасные Явления
Погоды}\label{ux43eux43fux430ux441ux43dux44bux435-ux44fux432ux43bux435ux43dux438ux44f-ux43fux43eux433ux43eux434ux44b}}

Переохлажденные капли являются непосредственной причиной многих опасных
явлений погоды, таких как гололед, изморозь и обледенение различных
объектов.

\hypertarget{ux433ux43eux43bux43eux43bux435ux434-ux438-ux433ux43eux43bux43eux43bux435ux434ux438ux446ux430}{%
\paragraph{Гололед и
Гололедица}\label{ux433ux43eux43bux43eux43bux435ux434-ux438-ux433ux43eux43bux43eux43bux435ux434ux438ux446ux430}}

Гололед (гололедица) образуется, когда выпадающие переохлажденные капли
дождя или мокрого снега замерзают при контакте с сильно охлажденной
поверхностью земли или объектами. Этот процесс может быть обусловлен
прохождением осадков через тонкий приземный слой воздуха с отрицательной
температурой, когда над фронтальной поверхностью температуры
положительны.

\hypertarget{ux438ux437ux43cux43eux440ux43eux437ux44c}{%
\paragraph{Изморозь}\label{ux438ux437ux43cux43eux440ux43eux437ux44c}}

Изморозь образуется, если капли тумана имеют размер менее 20 мкм и
замерзают при соприкосновении с ледяными частицами. Синоптические
условия для образования зернистой изморози схожи с условиями для
гололеда.

\hypertarget{ux43eux431ux43bux435ux434ux435ux43dux435ux43dux438ux435-ux432ux43eux437ux434ux443ux448ux43dux44bux445-ux441ux443ux434ux43eux432}{%
\paragraph{Обледенение Воздушных
Судов}\label{ux43eux431ux43bux435ux434ux435ux43dux435ux43dux438ux435-ux432ux43eux437ux434ux443ux448ux43dux44bux445-ux441ux443ux434ux43eux432}}

Обледенение самолетов представляет собой серьезную угрозу безопасности
полетов и происходит, когда переохлажденные капли воды замерзают на
поверхностях воздушного судна.

\begin{itemize}
\tightlist
\item
  \textbf{Типы обледенения}: Различают профильное (повторяет профиль
  поверхности), желобковое (с выемкой по центру) и неправильное
  (бесформенное) обледенение. Профильное обледенение характерно для
  переохлажденных облаков малой водности при низких температурах (обычно
  ниже -20°C), когда все осевшие капли мгновенно замерзают. Желобковое
  обледенение наблюдается в переохлажденных облаках большой водности при
  более высоких температурах.
\item
  \textbf{Интенсивность и опасность}: Интенсивность обледенения зависит
  от таких факторов, как температура воздуха, скорость полета самолета и
  водность переохлажденных капель. Низкие значения параметра \(H\)
  (который включает переохлаждение капель), приводят к образованию
  плотного, прозрачного, очень прочного льда, что является наиболее
  опасным видом обледенения. При больших значениях \(H\) образуется
  менее плотный и прочный матовый или непрозрачный лед.
\item
  \textbf{Районы и условия обледенения}: Обледенение наиболее вероятно в
  чисто капельных переохлажденных облаках, таких как St, Sc, Cu, а также
  в нижних частях Ns и Cb. В смешанных облаках (например, As, верхние
  части Ns) обледенение менее вероятно. В чисто ледяных облаках (Ci, Cs)
  обледенение, как правило, не наблюдается, за исключением редких
  случаев присутствия переохлажденных капель. Интенсивное обледенение
  также возможно в зоне переохлажденного дождя или мороси под облаками,
  особенно вблизи фронтов.
\end{itemize}

\hypertarget{ux43eux431ux43bux435ux434ux435ux43dux435ux43dux438ux435-ux43cux43eux440ux441ux43aux438ux445-ux441ux443ux434ux43eux432}{%
\paragraph{Обледенение Морских
Судов}\label{ux43eux431ux43bux435ux434ux435ux43dux435ux43dux438ux435-ux43cux43eux440ux441ux43aux438ux445-ux441ux443ux434ux43eux432}}

Обледенение морских судов, представляющее собой серьезную опасность,
чаще всего вызвано кристаллизацией переохлажденных капель брызгового
облака, которые обволакивают надстройки судна. Это значительно ухудшает
остойчивость судна и создает трудности при околке льда. Обледенение
также может происходить из-за заливания палубы забортной водой или в
атмосферных осадках, таких как переохлажденный дождь, морось или туман,
хотя интенсивность в последнем случае, как правило, невелика. Один из
главных синоптических процессов, приводящих к обледенению судов, -- это
адвекция холодного воздуха в тылу циклона при сильных ветрах, при этом
наиболее опасная зона обледенения формируется в области ложбины холода
за холодным фронтом.

\hypertarget{ux43fux440ux43eux433ux43dux43eux441ux442ux438ux447ux435ux441ux43aux438ux435-ux430ux441ux43fux435ux43aux442ux44b}{%
\subsection{Прогностические
Аспекты}\label{ux43fux440ux43eux433ux43dux43eux441ux442ux438ux447ux435ux441ux43aux438ux435-ux430ux441ux43fux435ux43aux442ux44b}}

Прогнозирование условий для переохлаждения капель и связанных с ним
явлений является важной частью синоптической практики. Для этого
используются различные параметры, такие как дефицит точки росы,
вертикальные профили температуры и влажности, а также модели,
учитывающие эффект вовлечения ненасыщенного воздуха в облако, что
приводит к дополнительному понижению температуры и частичному испарению
капель. Расчеты полей обледенения могут производиться на основе
прогностических карт атмосферного давления, температуры воздуха и воды,
а также полей ветра и волнения, с использованием номограмм для
определения ожидаемой интенсивности обледенения.

ewpage

\hypertarget{ux43eux431ux440ux430ux437ux43eux432ux430ux43dux438ux435-ux43bux435ux434ux44fux43dux44bux445-ux43aux440ux438ux441ux442ux430ux43bux43bux43eux432-ux432-ux430ux442ux43cux43eux441ux444ux435ux440ux435}{%
\section{Образование Ледяных Кристаллов в
Атмосфере}\label{ux43eux431ux440ux430ux437ux43eux432ux430ux43dux438ux435-ux43bux435ux434ux44fux43dux44bux445-ux43aux440ux438ux441ux442ux430ux43bux43bux43eux432-ux432-ux430ux442ux43cux43eux441ux444ux435ux440ux435}}

Образование ледяных кристаллов в атмосфере является ключевым
микрофизическим процессом, особенно значимым для формирования осадков в
умеренных и высоких широтах. Этот процесс тесно связан с
термодинамическими условиями и динамическими движениями воздушных масс.

\hypertarget{ux43cux435ux445ux430ux43dux438ux437ux43cux44b-ux43eux431ux440ux430ux437ux43eux432ux430ux43dux438ux44f-ux43bux435ux434ux44fux43dux44bux445-ux43aux440ux438ux441ux442ux430ux43bux43bux43eux432}{%
\subsection{Механизмы Образования Ледяных
Кристаллов}\label{ux43cux435ux445ux430ux43dux438ux437ux43cux44b-ux43eux431ux440ux430ux437ux43eux432ux430ux43dux438ux44f-ux43bux435ux434ux44fux43dux44bux445-ux43aux440ux438ux441ux442ux430ux43bux43bux43eux432}}

Основной механизм образования и роста ледяных кристаллов в облаках,
приводящий к выпадению осадков, известен как \textbf{процесс
Бержерона-Финдайзена}. Суть его заключается в следующем:

\hypertarget{ux43fux435ux440ux435ux441ux44bux449ux435ux43dux438ux435-ux43eux442ux43dux43eux441ux438ux442ux435ux43bux44cux43dux43e-ux43bux44cux434ux430}{%
\subsubsection{1. Пересыщение относительно
льда}\label{ux43fux435ux440ux435ux441ux44bux449ux435ux43dux438ux435-ux43eux442ux43dux43eux441ux438ux442ux435ux43bux44cux43dux43e-ux43bux44cux434ux430}}

В холодных облаках, особенно в их верхних частях, температура понижается
до отрицательных значений. При этом водяной пар, переносимый восходящими
токами, конденсируется, образуя мельчайшие капли жидкой воды, которые
могут оставаться в переохлажденном состоянии (жидкие капли при
температуре ниже 0°C). Важным физическим свойством здесь является то,
что давление насыщенного водяного пара над переохлажденной водой больше,
чем над льдом при той же температуре. Это создает условия, при которых
атмосфера может быть ненасыщенной относительно воды, но при этом
пересыщенной относительно льда.

\hypertarget{ux440ux43eux441ux442-ux43aux440ux438ux441ux442ux430ux43bux43bux43eux432-ux437ux430-ux441ux447ux435ux442-ux43aux430ux43fux435ux43bux44c}{%
\subsubsection{2. Рост кристаллов за счет
капель}\label{ux440ux43eux441ux442-ux43aux440ux438ux441ux442ux430ux43bux43bux43eux432-ux437ux430-ux441ux447ux435ux442-ux43aux430ux43fux435ux43bux44c}}

Ледяные кристаллы служат отличными ядрами конденсации (точнее,
сублимации) для переохлажденной воды и водяного пара. Когда
переохлажденные капли воды попадают в зоны с отрицательными
температурами, часть из них замерзает, а другая продолжает существовать
в виде переохлажденной воды. В атмосфере всегда присутствуют ядра
конденсации, что позволяет начать процесс. Присутствие ледяных
кристаллов в смешанной фазе облака приводит к интенсивному росту этих
кристаллов. Водяной пар, находящийся в контакте как с переохлажденными
каплями, так и с ледяными кристаллами, будет испаряться с поверхности
водяных капель (где парциальное давление насыщения выше) и сублимировать
на поверхность ледяных кристаллов (где парциальное давление насыщения
ниже). Таким образом, ледяные кристаллы растут за счет ``перегонки''
пара с капель на кристаллы, а также за счет непосредственной сублимации
пара.

\hypertarget{ux443ux441ux43bux43eux432ux438ux44f-ux438ux43dux442ux435ux43dux441ux438ux432ux43dux43eux439-ux43aux440ux438ux441ux442ux430ux43bux43bux438ux437ux430ux446ux438ux438}{%
\subsubsection{3. Условия интенсивной
кристаллизации}\label{ux443ux441ux43bux43eux432ux438ux44f-ux438ux43dux442ux435ux43dux441ux438ux432ux43dux43eux439-ux43aux440ux438ux441ux442ux430ux43bux43bux438ux437ux430ux446ux438ux438}}

Процесс интенсивной кристаллизации большинства переохлажденных капель в
ледяные кристаллы наблюдается при достижении высоты, где температура
воздуха составляет около -12°C. При температурах около -10°C, при
наличии восходящих движений воздуха (w \textgreater{} 0), начинается
образование ледяной фазы в облаке. Рост ледяных кристаллов происходит
значительно интенсивнее, чем рост водяных капель.

\hypertarget{ux43aux43eux43dux442ux435ux43aux441ux442-ux444ux43eux440ux43cux438ux440ux43eux432ux430ux43dux438ux44f-ux432-ux43eux431ux43bux430ux43aux430ux445}{%
\subsection{Контекст Формирования в
Облаках}\label{ux43aux43eux43dux442ux435ux43aux441ux442-ux444ux43eux440ux43cux438ux440ux43eux432ux430ux43dux438ux44f-ux432-ux43eux431ux43bux430ux43aux430ux445}}

\hypertarget{ux43eux431ux43bux430ux43aux430-ux441ux43cux435ux448ux430ux43dux43dux43eux439-ux444ux430ux437ux44b}{%
\subsubsection{1. Облака смешанной
фазы}\label{ux43eux431ux43bux430ux43aux430-ux441ux43cux435ux448ux430ux43dux43dux43eux439-ux444ux430ux437ux44b}}

Образование ледяных кристаллов характерно для холодных облаков, особенно
таких как слоисто-дождевые (Ns) и кучево-дождевые (Cb). Эти облака, как
правило, состоят из капельных и ледяных элементов. В умеренных и высоких
широтах наличие трех фаз -- водяного пара, капелек и ледяных элементов
-- является основным условием для выпадения осадков.

\hypertarget{ux432ux43bux438ux44fux43dux438ux435-ux432ux435ux440ux442ux438ux43aux430ux43bux44cux43dux44bux445-ux434ux432ux438ux436ux435ux43dux438ux439}{%
\subsubsection{2. Влияние вертикальных
движений}\label{ux432ux43bux438ux44fux43dux438ux435-ux432ux435ux440ux442ux438ux43aux430ux43bux44cux43dux44bux445-ux434ux432ux438ux436ux435ux43dux438ux439}}

Восходящие движения воздуха являются главной причиной охлаждения
воздушных масс, что приводит к достижению состояния насыщения и
последующей конденсации/сублимации. Чем больше вертикальная
протяженность облака и интенсивность вертикальных движений внутри него,
тем быстрее происходит слияние (коагуляция) облачных частиц и,
соответственно, выпадение осадков. Мощные кучево-дождевые облака,
достигающие тропопаузы, являются примерами систем с интенсивными
вертикальными потоками и образованием льда в верхних слоях.

\hypertarget{ux440ux43eux441ux442-ux438-ux432ux44bux43fux430ux434ux435ux43dux438ux435-ux43eux441ux430ux434ux43aux43eux432}{%
\subsubsection{3. Рост и выпадение
осадков}\label{ux440ux43eux441ux442-ux438-ux432ux44bux43fux430ux434ux435ux43dux438ux435-ux43eux441ux430ux434ux43aux43eux432}}

По мере роста ледяные кристаллы становятся достаточно крупными и
начинают падать. При падении они могут проходить через слои с
положительными температурами, где тают, превращаясь в дождевые капли,
или остаются снежинками, если их размер достаточно велик для
сопротивления воздушным потокам.

\hypertarget{ux434ux440ux443ux433ux438ux435-ux432ux438ux434ux44b-ux43eux431ux440ux430ux437ux43eux432ux430ux43dux438ux44f-ux43bux435ux434ux44fux43dux44bux445-ux43eux442ux43bux43eux436ux435ux43dux438ux439}{%
\subsection{Другие Виды Образования Ледяных
Отложений}\label{ux434ux440ux443ux433ux438ux435-ux432ux438ux434ux44b-ux43eux431ux440ux430ux437ux43eux432ux430ux43dux438ux44f-ux43bux435ux434ux44fux43dux44bux445-ux43eux442ux43bux43eux436ux435ux43dux438ux439}}

Помимо формирования в облаках, ледяные кристаллы могут образовываться и
на поверхностях:

\hypertarget{ux438ux437ux43cux43eux440ux43eux437ux44c-1}{%
\subsubsection{1.
Изморозь}\label{ux438ux437ux43cux43eux440ux43eux437ux44c-1}}

Изморозь --- это отложения льда на различных предметах (ветках деревьев,
проводах, травинках), нарастающие преимущественно с наветренной стороны.
Различают два типа:

\begin{itemize}
\tightlist
\item
  \textbf{Кристаллическая изморозь:} Образуется в результате сублимации
  водяного пара в тихую погоду, чаще всего ночью, при температуре
  воздуха от -5°C до -20°C. Имеет нежную тонкую структуру кристаллов
  льда.
\item
  \textbf{Зернистая изморозь:} Представляет собой снеговидный, рыхлый
  лед аморфного строения. Образуется в туманную, преимущественно
  ветреную погоду за счет намерзания переохлажденных капель тумана при
  температурах воздуха от 0°C до -10°C, иногда и при более низких
  температурах.
\end{itemize}

\hypertarget{ux442ux443ux43cux430ux43d}{%
\subsubsection{2. Туман}\label{ux442ux443ux43cux430ux43d}}

Туман также может состоять из кристаллов льда, что приводит к снижению
метеорологической дальности видимости.

ewpage

\hypertarget{ux443ux440ux430ux432ux43dux435ux43dux438ux435-ux43fux435ux440ux435ux43dux43eux441ux430-ux432ux43eux434ux44fux43dux43eux433ux43e-ux43fux430ux440ux430-ux432-ux442ux443ux440ux431ux443ux43bux435ux43dux442ux43dux43eux439-ux430ux442ux43cux43eux441ux444ux435ux440ux435}{%
\section{Уравнение Переноса Водяного Пара в Турбулентной
Атмосфере}\label{ux443ux440ux430ux432ux43dux435ux43dux438ux435-ux43fux435ux440ux435ux43dux43eux441ux430-ux432ux43eux434ux44fux43dux43eux433ux43e-ux43fux430ux440ux430-ux432-ux442ux443ux440ux431ux443ux43bux435ux43dux442ux43dux43eux439-ux430ux442ux43cux43eux441ux444ux435ux440ux435}}

\hypertarget{ux43eux431ux449ux438ux435-ux43fux43eux43bux43eux436ux435ux43dux438ux44f-ux438-ux43aux43eux43dux442ux435ux43aux441ux442}{%
\subsection{Общие Положения и
Контекст}\label{ux43eux431ux449ux438ux435-ux43fux43eux43bux43eux436ux435ux43dux438ux44f-ux438-ux43aux43eux43dux442ux435ux43aux441ux442}}

Для построения замкнутой системы уравнений гидротермодинамики атмосферы,
описывающей ее состояние и процессы, необходимо привлечение уравнения
переноса водяного пара (\(q\)). В условиях реальной атмосферы, которая,
как правило, характеризуется турбулентным характером движения,
исследования проводятся с использованием осредненных уравнений.
Следовательно, и уравнение переноса водяного пара также представляется в
осредненной форме. При его выводе обычно пренебрегают пульсациями
плотности воздуха, что упрощает расчеты.

Уравнение переноса водяного пара выводится по аналогии с уравнением
неразрывности, которое выражает закон сохранения массы. Удельная
влажность (\(q\)), определяемая как отношение массы водяного пара к
общей массе влажного воздуха в том же объеме, рассматривается как
величина, обладающая свойствами консервативности и пассивности.
Консервативность означает, что значение \(q\) сохраняется в каждой
движущейся воздушной частице, а при смешении двух воздушных частиц общее
количество величины \(q\) просто складывается. Пассивность
подразумевает, что распределение \(q\) не влияет на характер движения
воздушных частиц. Эти свойства позволяют корректно определять
турбулентный поток влаги по аналогии с другими величинами.

\hypertarget{ux444ux43eux440ux43cux430-ux443ux440ux430ux432ux43dux435ux43dux438ux44f}{%
\subsection{Форма
Уравнения}\label{ux444ux43eux440ux43cux430-ux443ux440ux430ux432ux43dux435ux43dux438ux44f}}

Осредненное уравнение переноса удельной влажности водяного пара (\(q\))
в турбулентной атмосфере, с пренебрежением молекулярной диффузией
(которая мала по сравнению с турбулентной диффузией), может быть
записано в следующем виде:

\texttt{∂q/∂t\ +\ u\ ∂q/∂x\ +\ v\ ∂q/∂y\ +\ w\ ∂q/∂z\ =\ ∂/∂x\ (xk\ ∂q/∂x)\ +\ ∂/∂y\ (yk\ ∂q/∂y)\ +\ ∂/∂z\ (zk\ ∂q/∂z)\ +\ W}

\hypertarget{ux438ux43dux442ux435ux440ux43fux440ux435ux442ux430ux446ux438ux44f-ux447ux43bux435ux43dux43eux432-ux443ux440ux430ux432ux43dux435ux43dux438ux44f}{%
\subsection{Интерпретация Членов
Уравнения}\label{ux438ux43dux442ux435ux440ux43fux440ux435ux442ux430ux446ux438ux44f-ux447ux43bux435ux43dux43eux432-ux443ux440ux430ux432ux43dux435ux43dux438ux44f}}

Рассмотрим физический смысл каждого члена в этом уравнении:

\begin{itemize}
\tightlist
\item
  \textbf{\texttt{∂q/∂t}}: Этот член выражает локальное (местное)
  изменение удельной влажности во времени в фиксированной точке
  пространства.
\item
  \textbf{\texttt{u\ ∂q/∂x\ +\ v\ ∂q/∂y\ +\ w\ ∂q/∂z}}: Сумма этих трех
  слагаемых представляет собой адвективный перенос водяного пара. Он
  отражает изменение удельной влажности в данной точке, обусловленное
  перемещением воздушной массы с иной влажностью. Проекции скорости
  \texttt{u}, \texttt{v}, \texttt{w} соответствуют движениям по осям x,
  y и z соответственно.
\item
  \textbf{\texttt{∂/∂x\ (xk\ ∂q/∂x)\ +\ ∂/∂y\ (yk\ ∂q/∂y)\ +\ ∂/∂z\ (zk\ ∂q/∂z)}}:
  Эти члены описывают процессы турбулентной диффузии (перемешивания)
  водяного пара вдоль соответствующих координатных осей. Здесь
  \texttt{xk}, \texttt{yk}, \texttt{zk} --- это горизонтальные и
  вертикальный коэффициенты турбулентного обмена (турбулентной
  диффузии). Влияние турбулентного перемешивания на свойства атмосферы
  является значительным, особенно в пограничном слое.
\item
  \textbf{\texttt{W}}: Этот член обозначает скорость испарения, то есть
  количество водяного пара, которое появилось или исчезло в единицу
  времени в единице массы влажного воздуха. Он учитывает фазовые
  превращения воды, такие как конденсация, сублимация и испарение.
  Конденсация водяного пара сопровождается выделением скрытой теплоты,
  что оказывает существенное влияние на термодинамическое состояние
  воздуха, в то время как испарение поглощает тепло.
\end{itemize}

\hypertarget{ux437ux43dux430ux447ux435ux43dux438ux435-ux434ux43bux44f-ux434ux438ux43dux430ux43cux438ux447ux435ux441ux43aux43eux439-ux43cux435ux442ux435ux43eux440ux43eux43bux43eux433ux438ux438}{%
\subsection{Значение для Динамической
Метеорологии}\label{ux437ux43dux430ux447ux435ux43dux438ux435-ux434ux43bux44f-ux434ux438ux43dux430ux43cux438ux447ux435ux441ux43aux43eux439-ux43cux435ux442ux435ux43eux440ux43eux43bux43eux433ux438ux438}}

Уравнение переноса водяного пара является одним из фундаментальных
соотношений в динамической метеорологии и незаменимо для полного
описания атмосферных процессов. Оно позволяет моделировать и
прогнозировать изменения влажности, которые непосредственно связаны с
формированием облаков, туманов и осадков --- ключевых элементов погоды.

В частности, учет притоков водяного пара критически важен для понимания
образования и эволюции таких синоптических объектов, как циклоны и
фронты, а также для прогнозирования конвективных явлений. В более
сложных неадиабатических моделях атмосферы, используемых для численных
прогнозов, уравнение переноса водяного пара дополняет систему, позволяя
учитывать влияние фазовых переходов воды и других источников тепла,
таких как турбулентный и радиационный теплообмен.

ewpage

Коллега, рассмотрим вопрос испарения с различных земных и водных
поверхностей, исходя из нашего текущего понимания атмосферных процессов.

\hypertarget{ux438ux441ux43fux430ux440ux435ux43dux438ux435-ux441-ux437ux435ux43cux43dux43eux439-ux43fux43eux432ux435ux440ux445ux43dux43eux441ux442ux438-ux438-ux441-ux43fux43eux432ux435ux440ux445ux43dux43eux441ux442ux435ux439-ux432ux43eux434ux43eux435ux43cux43eux432}{%
\section{Испарение с земной поверхности и с поверхностей
водоемов}\label{ux438ux441ux43fux430ux440ux435ux43dux438ux435-ux441-ux437ux435ux43cux43dux43eux439-ux43fux43eux432ux435ux440ux445ux43dux43eux441ux442ux438-ux438-ux441-ux43fux43eux432ux435ux440ux445ux43dux43eux441ux442ux435ux439-ux432ux43eux434ux43eux435ux43cux43eux432}}

Испарение является ключевым процессом влагообмена в атмосфере,
определяющим образование облаков, туманов и осадков. Оно характеризуется
скоростью испарения, то есть массой воды, испаряющейся с единицы площади
за единицу времени. В природе относительная влажность воздуха редко
достигает 100\%, обычно не превышая 98\% даже в тумане или сильный
дождь, что указывает на постоянное преобладание молекул пара, покидающих
открытую поверхность воды, снега или льда, над возвращающимися, тем
самым обуславливая процесс испарения.

\hypertarget{ux444ux430ux43aux442ux43eux440ux44b-ux432ux43bux438ux44fux44eux449ux438ux435-ux43dux430-ux438ux441ux43fux430ux440ux435ux43dux438ux435}{%
\subsection{Факторы, влияющие на
испарение}\label{ux444ux430ux43aux442ux43eux440ux44b-ux432ux43bux438ux44fux44eux449ux438ux435-ux43dux430-ux438ux441ux43fux430ux440ux435ux43dux438ux435}}

Процесс испарения существенно охлаждает подстилающую поверхность,
поскольку при испарении теряется скрытая теплота парообразования,
величина которой очень велика (L = 2,5 × 10\^{}6 Дж/кг), что приводит к
потере LE (Вт/м²) энергии поверхностью.

Скорость испарения с подстилающей поверхности пропорциональна разности
значений содержания вещества (водяного пара) у поверхности земли и на
некоторой высоте, а также скорости ветра.

\hypertarget{ux438ux441ux43fux430ux440ux435ux43dux438ux435-ux441-ux437ux435ux43cux43dux43eux439-ux43fux43eux432ux435ux440ux445ux43dux43eux441ux442ux438-ux441ux443ux448ux438}{%
\subsection{Испарение с земной поверхности
(суши)}\label{ux438ux441ux43fux430ux440ux435ux43dux438ux435-ux441-ux437ux435ux43cux43dux43eux439-ux43fux43eux432ux435ux440ux445ux43dux43eux441ux442ux438-ux441ux443ux448ux438}}

Точное определение количества испаряемой влаги с больших и неоднородных
площадей, таких как города, лесные массивы или поля, представляет собой
сложную задачу, и удовлетворительные методы расчета для таких масштабов
пока отсутствуют, в отличие от небольших метеорологических площадок.

Над сушей годовые колебания содержания водяного пара выражены более
четко, чем над океанами. Термическая неоднородность почвы, включая ее
увлажненность, состав, структуру и рельеф, приводит к значительным
вариациям теплофизических свойств и, как следствие, к неоднородности
температуры почвы по горизонтали. Увлажненность почвы существенно влияет
на ее температуру, альбедо, скорость испарения и теплофизические
характеристики, такие как теплоемкость и температуропроводность. Более
влажная почва медленнее нагревается и охлаждается, но суточные колебания
температуры проникают в нее глубже.

В засушливых регионах, например, над Сахарой, испаряемость (максимально
возможное испарение при заданных условиях, характеризуемое дефицитом
влажности) очень велика из-за значительного дефицита насыщения, однако
фактическое испарение крайне мало, поскольку подстилающая поверхность
практически не увлажнена. Напротив, в наиболее влажных районах Земли,
таких как тропические дождевые леса (например, Амазония), испарение
огромно, что связано не только с наличием открытой воды и более высокой
температурой подстилающей поверхности по сравнению с океаном, но и с
обширной площадью испаряющих влагу листьев растений.

\hypertarget{ux438ux441ux43fux430ux440ux435ux43dux438ux435-ux441-ux43fux43eux432ux435ux440ux445ux43dux43eux441ux442ux435ux439-ux432ux43eux434ux43eux435ux43cux43eux432}{%
\subsection{Испарение с поверхностей
водоемов}\label{ux438ux441ux43fux430ux440ux435ux43dux438ux435-ux441-ux43fux43eux432ux435ux440ux445ux43dux43eux441ux442ux435ux439-ux432ux43eux434ux43eux435ux43cux43eux432}}

Над водной, снежной или ледяной поверхностью относительная влажность
воздуха постоянно держится в диапазоне 80-90\%. Над океанами
относительная влажность колеблется около 80\%. В водной среде теплообмен
происходит не только за счет молекулярной теплопроводности, но и за счет
турбулентного перемешивания воды течениями различного масштаба, что
значительно ускоряет процессы. Благодаря большой теплоемкости водных
масс и свободному вертикальному перемешиванию, температура поверхности
океана со временем изменяется незначительно. Вследствие этого амплитуда
суточного хода температуры поверхности воды в океанах очень мала,
составляя около 0,1 °С.

Процессы испарения над открытыми морскими поверхностями особенно важны.
Например, в тыловой части тропического циклона (ТЦ) суммарная
теплоотдача океана в 2-3 раза меньше, чем в передней части, что
обусловлено значительным выносом тепла и водяного пара из океана в
атмосферу. После прохождения ТЦ на поверхности океана остается след из
более холодной воды, температура которой на несколько градусов ниже, чем
до прохождения циклона. Это свидетельствует о значительном поглощении
скрытой теплоты испарения океаном.

Возникновение туманов испарения (парящих туманов) над водоемами
происходит, когда температура поверхности воды значительно выше
температуры окружающего воздуха, обычно при разнице не менее 10°С и
относительной влажности окружающего воздуха около 70\%. Такие туманы
часто наблюдаются над незамерзающими заливами арктических морей, вблизи
кромки льдов, а также осенью и зимой над быстротекущими незамерзающими
реками. Эти туманы образуются за счет насыщения холодного воздуха
водяным паром с более теплой водной поверхности при конвективных и
турбулентных движениях и смешивании.

Если в воде растворены соли, то давление насыщающего пара над
поверхностью уменьшается по мере увеличения концентрации солей в
растворе.

\hypertarget{ux432ux43bux438ux44fux43dux438ux435-ux443ux440ux431ux430ux43dux438ux437ux430ux446ux438ux438-ux43dux430-ux438ux441ux43fux430ux440ux435ux43dux438ux435}{%
\subsection{Влияние урбанизации на
испарение}\label{ux432ux43bux438ux44fux43dux438ux435-ux443ux440ux431ux430ux43dux438ux437ux430ux446ux438ux438-ux43dux430-ux438ux441ux43fux430ux440ux435ux43dux438ux435}}

В больших городах режим влажности воздуха существенно отличается от
такового в окружающей сельской местности. Антропогенное производство
водяного пара в результате сжигания различных видов топлива (например, 1
кг бензина дает 1,3 кг водяного пара, 1 кг природного газа -- 1 кг)
увеличивает содержание водяного пара в атмосфере города.

Однако скорость испарения в городе меньше, чем в окрестностях. Это
связано с тем, что значительная часть осадков в городе отводится в
канализацию и не участвует в испарении, а также с существенно меньшей
площадью растительного покрова и застроенностью. Следствием сниженного
испарения является то, что разница между парциальным давлением водяного
пара в городе и в окрестностях значительно уменьшается, особенно летом в
дневное и вечернее время.

ewpage

\hypertarget{ux440ux430ux432ux43dux43eux432ux435ux441ux43dux430ux44f-ux43eux442ux43dux43eux441ux438ux442ux435ux43bux44cux43dux430ux44f-ux432ux43bux430ux436ux43dux43eux441ux442ux44c}{%
\section{Равновесная относительная
влажность}\label{ux440ux430ux432ux43dux43eux432ux435ux441ux43dux430ux44f-ux43eux442ux43dux43eux441ux438ux442ux435ux43bux44cux43dux430ux44f-ux432ux43bux430ux436ux43dux43eux441ux442ux44c}}

В представленных источниках термин ``равновесная относительная
влажность'' не определяется явным образом. Однако, концепция равновесия
между водяным паром и жидкой/твердой фазой, особенно в контексте условий
для начала конденсации и роста капель, подробно рассматривается. Суть
заключается в том, что условия для конденсации водяного пара значительно
варьируются в зависимости от свойств доступных ядер конденсации.

\hypertarget{ux443ux441ux43bux43eux432ux438ux44f-ux43dux430ux447ux430ux43bux430-ux43aux43eux43dux434ux435ux43dux441ux430ux446ux438ux438}{%
\subsection{Условия начала
конденсации}\label{ux443ux441ux43bux43eux432ux438ux44f-ux43dux430ux447ux430ux43bux430-ux43aux43eux43dux434ux435ux43dux441ux430ux446ux438ux438}}

Для начала конденсации водяного пара в атмосфере решающую роль играют
аэрозольные частицы, известные как ядра конденсации. В отсутствие таких
ядер, то есть в чистом водяном паре, для перехода в упорядоченное
состояние (жидкую воду или лед) молекулам пара необходимо преодолеть
существенный энергетический барьер, связанный с формированием новой
поверхности. В этих условиях конденсация требует значительного
перенасыщения, которое редко превышает 1\% в реальной атмосфере.

Однако, в присутствии гигроскопичных ядер конденсации, процесс
начинается при более низких значениях относительной влажности.
Гигроскопичные частицы, такие как морская соль (NaCl), начинают
поглощать водяной пар задолго до достижения 100\% насыщения. Например,
на частицах поваренной соли насыщение может быть достигнуто уже при
относительной влажности 78\%. Источники указывают, что водяной пар
начинает осаждаться (конденсироваться) на гигроскопичных ядрах ``задолго
до достижения состояния насыщения'', и уже при относительной влажности
60\% доля воды, осевшей на такой частице, может превышать массу самой
частицы.

\hypertarget{ux432ux43bux438ux44fux43dux438ux435-ux440ux430ux434ux438ux443ux441ux430-ux438-ux445ux438ux43cux438ux447ux435ux441ux43aux43eux433ux43e-ux441ux43eux441ux442ux430ux432ux430}{%
\subsection{Влияние радиуса и химического
состава}\label{ux432ux43bux438ux44fux43dux438ux435-ux440ux430ux434ux438ux443ux441ux430-ux438-ux445ux438ux43cux438ux447ux435ux441ux43aux43eux433ux43e-ux441ux43eux441ux442ux430ux432ux430}}

Давление насыщения водяного пара (E) над поверхностью капли, строго
говоря, зависит не только от температуры, но и от химического состава,
электрического заряда и радиуса ядра конденсации.

\begin{itemize}
\tightlist
\item
  \textbf{Химический состав}: Наличие растворенных солей в капле
  уменьшает насыщающее давление пара, причем тем сильнее, чем выше
  концентрация солей. Это означает, что на гигроскопичных ядрах
  конденсации состояние насыщения достигается при относительной
  влажности ниже 100\%.
\item
  \textbf{Радиус ядра/капли}: Радиус ядра также играет определенную
  роль. При уменьшении радиуса ядра давление насыщения (E)
  увеличивается. Однако, это влияние становится незначительным, когда
  радиус капли достигает 1 мкм. Начиная с этого размера, давление
  водяного пара на поверхности капли практически не отличается от
  давления насыщения (E) над плоской поверхностью чистой воды.
\end{itemize}

Таким образом, ``равновесная относительная влажность'' в контексте
образования зародышевых капель и их начального роста --- это та
относительная влажность, при которой устанавливается динамическое
равновесие между испарением и конденсацией на поверхности частицы. Для
гигроскопичных ядер эта влажность может быть значительно ниже 100\%, что
критически важно для формирования облаков и туманов в условиях, когда
воздух еще не достиг полного насыщения или небольшого перенасыщения,
необходимого для конденсации на негигроскопичных частицах или в
абсолютно чистом воздухе.

ewpage

Коллега, давайте подробно рассмотрим аспекты туманов, их
физико-метеорологические условия образования и классификацию,
основываясь на доступных нам источниках.

\hypertarget{ux442ux443ux43cux430ux43dux44b}{%
\section{Туманы}\label{ux442ux443ux43cux430ux43dux44b}}

\hypertarget{ux43eux43fux440ux435ux434ux435ux43bux435ux43dux438ux435-1}{%
\subsection{Определение}\label{ux43eux43fux440ux435ux434ux435ux43bux435ux43dux438ux435-1}}

Туманом называется помутнение воздуха в приземном слое, вызванное
взвешенными в нем каплями воды, ледяными кристаллами или их смесью, при
горизонтальной видимости менее 1 км хотя бы в одном направлении. Туманы
относятся к атмосферным явлениям, оказывающим существенное влияние на
хозяйственную деятельность человека, особенно на транспорт -- авиацию,
автомобильный, железнодорожный, морской и речной. Они также влияют на
строительство, энергетику, здравоохранение и общее благополучие
человека. При образовании туманов и дымок резко изменяются оптические
свойства атмосферы.

В метеорологии туманы принято делить по дальности видимости:

\begin{itemize}
\tightlist
\item
  \textbf{Сильные туманы}: метеорологическая дальность видимости (МДВ)
  \textless{} 50 м.
\item
  \textbf{Умеренные туманы}: 50 м \textless{} МДВ \textless{} 500 м.
\item
  \textbf{Слабые туманы}: 500 м \textless{} МДВ \textless{} 1 км. Дымки,
  хотя и схожи, отличаются порогом видимости: умеренные (1 км
  \textless{} МДВ \textless{} 5 км) и слабые (5 км \textless{} МДВ
  \textless{} 10 км).
\end{itemize}

\hypertarget{ux444ux438ux437ux438ux43aux43e-ux43cux435ux442ux435ux43eux440ux43eux43bux43eux433ux438ux447ux435ux441ux43aux438ux435-ux443ux441ux43bux43eux432ux438ux44f-ux43eux431ux440ux430ux437ux43eux432ux430ux43dux438ux44f-ux442ux443ux43cux430ux43dux43eux432}{%
\subsection{Физико-метеорологические условия образования
туманов}\label{ux444ux438ux437ux438ux43aux43e-ux43cux435ux442ux435ux43eux440ux43eux43bux43eux433ux438ux447ux435ux441ux43aux438ux435-ux443ux441ux43bux43eux432ux438ux44f-ux43eux431ux440ux430ux437ux43eux432ux430ux43dux438ux44f-ux442ux443ux43cux430ux43dux43eux432}}

Образование тумана -- это процесс конденсации водяного пара в приземном
слое воздуха, приводящий к снижению видимости. Этот процесс требует
сочетания нескольких факторов:

\hypertarget{ux43eux445ux43bux430ux436ux434ux435ux43dux438ux435-ux432ux43eux437ux434ux443ux445ux430-ux434ux43e-ux442ux43eux447ux43aux438-ux440ux43eux441ux44b}{%
\subsubsection{1. Охлаждение воздуха до точки
росы}\label{ux43eux445ux43bux430ux436ux434ux435ux43dux438ux435-ux432ux43eux437ux434ux443ux445ux430-ux434ux43e-ux442ux43eux447ux43aux438-ux440ux43eux441ux44b}}

Для образования тумана воздух должен охладиться до температуры, при
которой он становится насыщенным водяным паром, то есть до точки росы.
Это охлаждение может происходить различными путями:

\begin{itemize}
\tightlist
\item
  \textbf{Радиационное охлаждение}: Ночное выхолаживание земной
  поверхности и прилегающего к ней слоя воздуха за счет эффективного
  излучения при ясном небе. Этот фактор наиболее значим для радиационных
  туманов.
\item
  \textbf{Адвективное охлаждение}: Перемещение теплой и влажной
  воздушной массы над более холодной подстилающей поверхностью
  (например, сушей или холодной водной поверхностью).
\item
  \textbf{Адиабатическое охлаждение}: Подъем влажного воздуха, например,
  вдоль склонов гор (орографические туманы).
\end{itemize}

\hypertarget{ux432ux44bux441ux43eux43aux430ux44f-ux432ux43bux430ux436ux43dux43eux441ux442ux44c-ux432ux43eux437ux434ux443ux445ux430}{%
\subsubsection{2. Высокая влажность
воздуха}\label{ux432ux44bux441ux43eux43aux430ux44f-ux432ux43bux430ux436ux43dux43eux441ux442ux44c-ux432ux43eux437ux434ux443ux445ux430}}

Воздушная масса должна быть достаточно влажной. Если относительная
влажность приближается к 100\%, даже небольшое дополнительное охлаждение
может привести к конденсации. Для гигроскопичных ядер конденсации,
конденсация может начаться даже при относительной влажности ниже 100\%,
иногда при 60-80\% {[}ПРЕДЫДУЩИЙ ОТВЕТ{]}. Туман возможен лишь при
значительной влажности воздушной массы.

\hypertarget{ux43dux430ux43bux438ux447ux438ux435-ux44fux434ux435ux440-ux43aux43eux43dux434ux435ux43dux441ux430ux446ux438ux438}{%
\subsubsection{3. Наличие ядер
конденсации}\label{ux43dux430ux43bux438ux447ux438ux435-ux44fux434ux435ux440-ux43aux43eux43dux434ux435ux43dux441ux430ux446ux438ux438}}

В воздухе всегда присутствуют мельчайшие взвешенные частицы (аэрозоли),
называемые ядрами конденсации (дым, сажа, пепел, частицы морской соли,
пыльца и др.), на которых происходит конденсация водяного пара. Без них
для конденсации потребовалось бы значительное перенасыщение (более 1\%),
что редко наблюдается в естественных условиях {[}ПРЕДЫДУЩИЙ ОТВЕТ{]}.

\hypertarget{ux441ux43bux430ux431ux44bux439-ux432ux435ux442ux435ux440}{%
\subsubsection{4. Слабый
ветер}\label{ux441ux43bux430ux431ux44bux439-ux432ux435ux442ux435ux440}}

Для большинства типов туманов (особенно радиационных) необходим слабый
ветер (порядка 1--3 м/с). Слишком слабый ветер (штиль) может привести к
образованию тонкого приземного тумана, который быстро рассеется при
легком ветре, так как турбулентное перемешивание не будет распространять
влагу на достаточную высоту. При более сильном ветре турбулентность
перемешивает воздух, поднимая точку росы и рассеивая туман. Однако
адвективные туманы могут наблюдаться и при скоростях ветра более 15 м/с.

\hypertarget{ux442ux435ux43cux43fux435ux440ux430ux442ux443ux440ux43dux44bux435-ux438ux43dux432ux435ux440ux441ux438ux438}{%
\subsubsection{5. Температурные
инверсии}\label{ux442ux435ux43cux43fux435ux440ux430ux442ux443ux440ux43dux44bux435-ux438ux43dux432ux435ux440ux441ux438ux438}}

Наличие температурных инверсий (повышение температуры с высотой) в
приземном слое способствует концентрации влаги и загрязнений в
ограниченном объеме, что благоприятствует образованию туманов. Это
характерно, например, для антициклонических высотных инверсий, под
которыми располагается образующаяся облачность.

\hypertarget{ux441ux438ux43dux43eux43fux442ux438ux447ux435ux441ux43aux430ux44f-ux43eux431ux441ux442ux430ux43dux43eux432ux43aux430}{%
\subsubsection{6. Синоптическая
обстановка}\label{ux441ux438ux43dux43eux43fux442ux438ux447ux435ux441ux43aux430ux44f-ux43eux431ux441ux442ux430ux43dux43eux432ux43aux430}}

\begin{itemize}
\tightlist
\item
  \textbf{Антициклонические области}: Часто способствуют образованию
  радиационных туманов из-за нисходящих движений и ясного неба. В
  центральной части антициклона может быть сплошная слоистая или
  слоисто-кучевая облачность даже при значительной влажности.
\item
  \textbf{Теплые сектора циклонов}: При адвекции теплого и влажного
  воздуха над относительно холодной подстилающей поверхностью могут
  образовываться слоистые и слоисто-кучевые облака, которые могут
  опускаться до уровня тумана. Моросящие осадки характерны для теплой
  воздушной массы, особенно в теплом секторе циклона, и иногда являются
  результатом укрупнения частиц тумана.
\item
  \textbf{Фронтальные зоны}: Фронтальные туманы могут быть связаны с
  перемещением атмосферных фронтов, особенно теплых и окклюдированных,
  где происходит увлажнение воздуха выпадающими осадками.
\end{itemize}

\hypertarget{ux43aux43bux430ux441ux441ux438ux444ux438ux43aux430ux446ux438ux44f-ux442ux443ux43cux430ux43dux43eux432}{%
\subsection{Классификация
туманов}\label{ux43aux43bux430ux441ux441ux438ux444ux438ux43aux430ux446ux438ux44f-ux442ux443ux43cux430ux43dux43eux432}}

Туманы классифицируются в зависимости от преобладающих физических
процессов их образования:

\hypertarget{ux442ux443ux43cux430ux43dux44b-ux43eux445ux43bux430ux436ux434ux435ux43dux438ux44f}{%
\subsubsection{1. Туманы
охлаждения}\label{ux442ux443ux43cux430ux43dux44b-ux43eux445ux43bux430ux436ux434ux435ux43dux438ux44f}}

Образуются, когда температура воздуха падает до точки росы.

\begin{itemize}
\tightlist
\item
  \textbf{Радиационные туманы (туманы выхолаживания)}:

  \begin{itemize}
  \tightlist
  \item
    \textbf{Условия образования}: Образуются в результате ночного
    радиационного охлаждения земной поверхности и прилегающего к ней
    слоя воздуха. Для их возникновения необходимы ясное или малооблачное
    небо, слабый ветер (1-3 м/с) и достаточно высокая влажность в
    приземном слое.
  \item
    \textbf{Разновидности}:

    \begin{itemize}
    \tightlist
    \item
      \textbf{Поземные туманы}: Высотой до 2 м, возникающие над
      низинами.
    \item
      \textbf{Низкие туманы}: Высотой 10-100 м, с верхней границей,
      видимой с небольших возвышенностей.
    \item
      \textbf{Высокие туманы}: Высотой до нескольких сотен метров,
      которые могут распространяться на сотни километров.
    \end{itemize}
  \item
    \textbf{Время образования и рассеяния}: Минимальная видимость в
    радиационном тумане обычно отмечается в часы, когда температура
    минимальна: в теплое полугодие --- около восхода солнца, в холодное
    --- через 1-2 часа после восхода солнца. Рассеивание радиационного
    тумана происходит, как правило, через 1-2 часа после восхода солнца
    летом, и 3-5 часов осенью за счет солнечного прогрева. Зимой над
    снежным покровом туман может сохраняться весь день.
  \item
    \textbf{Прогноз}: Прогноз радиационного тумана требует определения
    температуры начала его образования (\(T_T\)) и прогнозируемой
    минимальной температуры воздуха (\(T_{мин}\)). Если
    \(T_T > T_{мин}\), туман возможен.
  \end{itemize}
\item
  \textbf{Адвективные туманы}:

  \begin{itemize}
  \tightlist
  \item
    \textbf{Условия образования}: Возникают при перемещении теплой и
    влажной воздушной массы над более холодной подстилающей
    поверхностью. Часто они являются результатом снижения слоистых (St)
    или слоисто-кучевых (Sc) облаков.
  \item
    \textbf{Особенности}: Могут наблюдаться в любое время суток, но
    усиливаются ночью из-за дополнительного радиационного охлаждения.
    Наиболее часто возникают поздней осенью в прибрежных районах суши.
    Над открытым морем адвективный туман образуется при смещении
    воздушной массы с теплой водной поверхности на холодную, особенно
    при больших горизонтальных градиентах температуры воды. Вблизи
    берега моря их образование связано с разностью температур вода-суша.
  \item
    \textbf{Туманы смещения}: Вариант адвективных туманов, возникающий
    при горизонтальном переносе уже сформировавшейся туманной массы.
    Могут перемещаться на значительные расстояния, особенно над морями.
  \item
    \textbf{Рассеяние}: Усиление ветра, надвигание облачности, выпадение
    осадков способствуют рассеянию тумана.
  \end{itemize}
\item
  \textbf{Адвективно-радиационные туманы}:

  \begin{itemize}
  \tightlist
  \item
    \textbf{Условия образования}: В их формировании участвуют как
    адвективные (перенос тепла и влаги), так и радиационные факторы
    (ночное выхолаживание).
  \item
    \textbf{Особенности}: Обычно адвекция тепла и влаги уменьшает
    дефицит точки росы днем, но сам туман образуется ночью при ясном или
    неполном прояснении.
  \end{itemize}
\item
  \textbf{Орографические туманы (туманы горных склонов,
  адиабатические)}:

  \begin{itemize}
  \tightlist
  \item
    \textbf{Условия образования}: Связаны с адиабатическим охлаждением
    влажного воздуха, поднимающегося вдоль склона горы. Одновременно
    происходит теплообмен с поверхностью склона.
  \end{itemize}
\end{itemize}

\hypertarget{ux442ux443ux43cux430ux43dux44b-ux438ux441ux43fux430ux440ux435ux43dux438ux44f}{%
\subsubsection{2. Туманы
испарения}\label{ux442ux443ux43cux430ux43dux44b-ux438ux441ux43fux430ux440ux435ux43dux438ux44f}}

Образуются, когда водяной пар поступает в холодный воздух и насыщает
его.

\begin{itemize}
\tightlist
\item
  \textbf{Туманы испарения (парения) водоемов (арктическое морское
  парение)}:

  \begin{itemize}
  \tightlist
  \item
    \textbf{Условия образования}: Возникают при поступлении холодной
    воздушной массы над значительно более теплой водной поверхностью.
    Теплая вода испаряется, и пар конденсируется в холодном воздухе
    непосредственно над поверхностью.
  \item
    \textbf{Особенности}: Характерны для незамерзающих морей и водоемов
    в холодный период.
  \end{itemize}
\item
  \textbf{Фронтальные туманы}:

  \begin{itemize}
  \tightlist
  \item
    \textbf{Условия образования}: Образуются в результате увлажнения
    приземного слоя воздуха выпадающими осадками (дождем или снегом),
    которые испаряются в более холодном воздухе под облаками. Это
    приводит к насыщению воздуха и образованию тумана.
  \item
    \textbf{Особенности}: Характерны для теплых фронтов и теплых фронтов
    окклюзии при слабых ветрах и моросящих осадках. Интенсивные осадки,
    напротив, способствуют рассеиванию тумана за счет коагуляции капель
    и переконденсации пара.
  \end{itemize}
\end{itemize}

\hypertarget{ux434ux440ux443ux433ux438ux435-ux432ux438ux434ux44b-ux442ux443ux43cux430ux43dux43eux432-ux43fux43e-ux441ux43fux435ux446ux438ux444ux438ux447ux435ux441ux43aux438ux43c-ux443ux441ux43bux43eux432ux438ux44fux43c}{%
\subsubsection{3. Другие виды туманов (по специфическим
условиям)}\label{ux434ux440ux443ux433ux438ux435-ux432ux438ux434ux44b-ux442ux443ux43cux430ux43dux43eux432-ux43fux43e-ux441ux43fux435ux446ux438ux444ux438ux447ux435ux441ux43aux438ux43c-ux443ux441ux43bux43eux432ux438ux44fux43c}}

\begin{itemize}
\tightlist
\item
  \textbf{Городские туманы}:

  \begin{itemize}
  \tightlist
  \item
    \textbf{Особенности}: Часто наблюдаются в больших индустриальных
    городах даже при отсутствии тумана в окрестностях, особенно зимой.
    Это связано с особенностями метеорологического режима города (остров
    тепла, измененная влажность, ветер) и дополнительными источниками
    водяного пара и дыма (антропогенные выбросы), которые увеличивают
    количество ядер конденсации. Загрязнение воздуха способствует
    значительному ухудшению видимости.
  \end{itemize}
\item
  \textbf{Морозные (поселковые, печные, аэродромные) туманы}:

  \begin{itemize}
  \tightlist
  \item
    \textbf{Условия образования}: Возникают при сильных морозах, если
    появляется дополнительный источник водяного пара (топка печей,
    работа двигателя самолета, паровоза и т.п.).
  \end{itemize}
\item
  \textbf{Искусственные туманы и дымы}:

  \begin{itemize}
  \tightlist
  \item
    \textbf{Особенности}: Могут создаваться целенаправленно, например,
    для защиты садов от заморозков. Их устойчивость зависит от слабого
    ветра и высокой относительной влажности.
  \end{itemize}
\end{itemize}

Понимание этих различных механизмов и условий является основой для
точного прогнозирования туманов в оперативной метеорологической
практике.

ewpage

\hypertarget{ux43eux441ux43dux43eux432ux43dux44bux435-ux445ux430ux440ux430ux43aux442ux435ux440ux438ux441ux442ux438ux43aux438-ux442ux443ux43cux430ux43dux43eux432}{%
\section{Основные Характеристики
Туманов}\label{ux43eux441ux43dux43eux432ux43dux44bux435-ux445ux430ux440ux430ux43aux442ux435ux440ux438ux441ux442ux438ux43aux438-ux442ux443ux43cux430ux43dux43eux432}}

Туман определяется как помутнение воздуха в приземном слое, вызванное
взвешенными в нем мельчайшими каплями воды, ледяными кристаллами или их
смесью, при горизонтальной видимости менее 1 км хотя бы в одном
направлении. В метеорологии туманы, наряду с атмосферным давлением,
температурой воздуха, влажностью, ветром, облачностью и осадками,
относятся к важнейшим метеорологическим элементам, определяющим погоду.

\hypertarget{ux43aux43bux430ux441ux441ux438ux444ux438ux43aux430ux446ux438ux44f-ux438-ux432ux438ux434ux438ux43cux43eux441ux442ux44c}{%
\subsubsection{Классификация и
Видимость}\label{ux43aux43bux430ux441ux441ux438ux444ux438ux43aux430ux446ux438ux44f-ux438-ux432ux438ux434ux438ux43cux43eux441ux442ux44c}}

Туманы классифицируются по нескольким признакам:

\begin{itemize}
\tightlist
\item
  \textbf{По синоптическим условиям образования:} выделяют
  внутримассовые и фронтальные туманы.
\item
  \textbf{По основным физическим процессам образования:}

  \begin{itemize}
  \tightlist
  \item
    \textbf{Туманы охлаждения:}

    \begin{itemize}
    \tightlist
    \item
      Радиационные (приземные, низкие, высокие).
    \item
      Адвективные (связанные с адвекцией теплой воздушной массы,
      снижением облаков и перемещением туманной массы).
    \item
      Адвективно-радиационные.
    \item
      Орографические (горных склонов, адиабатические).
    \end{itemize}
  \item
    \textbf{Туманы испарения:}

    \begin{itemize}
    \tightlist
    \item
      Испарения (парения) водоемов (например, арктические).
    \end{itemize}
  \end{itemize}
\item
  \textbf{По интенсивности видимости:}

  \begin{itemize}
  \tightlist
  \item
    Сильные: метеорологическая дальность видимости (МДВ) \textless{} 50
    м.
  \item
    Умеренные: 50 м \textless{} МДВ \textless{} 500 м.
  \item
    Слабые: 500 м \textless{} МДВ \textless{} 1 км.
  \end{itemize}
\item
  Дальность видимости в тумане (\(L\)) напрямую зависит от размеров
  взвешенных частиц (\(r\)) и их концентрации, то есть от водности
  тумана (\(\delta_т\)), которая выражается в г/м³. Связь описывается
  формулой \(L = 2.3 \cdot 10^{-3} \cdot \frac{r}{\delta_т}\).
\end{itemize}

\hypertarget{ux432ux43eux437ux434ux435ux439ux441ux442ux432ux438ux435-ux43dux430-ux434ux435ux44fux442ux435ux43bux44cux43dux43eux441ux442ux44c-ux438-ux43eux43fux442ux438ux447ux435ux441ux43aux438ux435-ux441ux432ux43eux439ux441ux442ux432ux430}{%
\subsubsection{Воздействие на Деятельность и Оптические
Свойства}\label{ux432ux43eux437ux434ux435ux439ux441ux442ux432ux438ux435-ux43dux430-ux434ux435ux44fux442ux435ux43bux44cux43dux43eux441ux442ux44c-ux438-ux43eux43fux442ux438ux447ux435ux441ux43aux438ux435-ux441ux432ux43eux439ux441ux442ux432ux430}}

Туманы, наряду с дымками, существенно изменяют оптические свойства
атмосферы. Они оказывают значительное влияние на хозяйственную
деятельность человека, затрагивая, в частности, работу различных видов
транспорта (авиационного, автомобильного, железнодорожного, морского и
речного). Их влияние также велико на строительство и эксплуатацию
зданий, сельскохозяйственное производство, а также на здоровье и жизнь
человека в целом. В городских условиях, наряду с общими циркуляционными
факторами, на частоту возникновения туманов может влиять температурный
режим, формируемый городской застройкой.

ewpage

\hypertarget{ux43cux43eux434ux435ux43bux438-ux43eux431ux440ux430ux437ux43eux432ux430ux43dux438ux44f-ux438-ux441ux442ux440ux43eux435ux43dux438ux44f-ux442ux443ux43cux430ux43dux43eux432}{%
\section{Модели Образования и Строения
Туманов}\label{ux43cux43eux434ux435ux43bux438-ux43eux431ux440ux430ux437ux43eux432ux430ux43dux438ux44f-ux438-ux441ux442ux440ux43eux435ux43dux438ux44f-ux442ux443ux43cux430ux43dux43eux432}}

Образование туманов тесно связано с достижением состояния насыщения
воздуха водяным паром и последующей конденсацией или сублимацией. Это
достигается либо за счет увеличения содержания водяного пара (увеличения
парциального давления \(e\)), либо за счет понижения температуры воздуха
(уменьшения давления насыщающего водяного пара \(E\)). Основную роль в
возникновении туманов и дымок играют адвективный и турбулентный притоки
тепла и влаги.

\hypertarget{ux442ux443ux43cux430ux43dux44b-ux43eux445ux43bux430ux436ux434ux435ux43dux438ux44f-1}{%
\subsubsection{Туманы
Охлаждения}\label{ux442ux443ux43cux430ux43dux44b-ux43eux445ux43bux430ux436ux434ux435ux43dux438ux44f-1}}

\begin{enumerate}
\def\labelenumi{\arabic{enumi}.}
\tightlist
\item
  \textbf{Радиационные туманы:} Формируются в результате ночного
  понижения температуры приземного слоя воздуха ниже точки росы,
  вызванного эффективным радиационным охлаждением подстилающей
  поверхности.
\item
  \textbf{Адвективные туманы:} Возникают, когда теплый и влажный воздух
  перемещается над значительно более холодной подстилающей поверхностью.
  Они могут наблюдаться в любое время суток, но часто усиливаются ночью
  из-за дополнительного радиационного охлаждения приземного слоя. Для их
  образования благоприятны теплые секторы циклонов и прилегающие к ним
  окраины антициклонов. Скорости ветра выше 6 м/с обычно не
  благоприятствуют их формированию. Над открытым морем адвективный туман
  образуется при смещении воздушной массы с теплой поверхности моря на
  холодную, а чем больше горизонтальный градиент температуры поверхности
  воды вдоль траектории воздушной массы, тем благоприятнее условия для
  образования тумана.
\item
  \textbf{Адвективно-радиационные туманы:} В их образовании участвуют
  как адвективные, так и радиационные факторы. Адвекция тепла и влаги,
  снижающая дефицит точки росы, может происходить днем, но
  непосредственное образование тумана связано с ночным прояснением.
\item
  \textbf{Орографические туманы:} Являются результатом адиабатического
  охлаждения влажного воздуха, поднимающегося вдоль склона горы. При
  этом происходит теплообмен воздуха с поверхностью склона.
\end{enumerate}

\hypertarget{ux442ux443ux43cux430ux43dux44b-ux438ux441ux43fux430ux440ux435ux43dux438ux44f-1}{%
\subsubsection{Туманы
Испарения}\label{ux442ux443ux43cux430ux43dux44b-ux438ux441ux43fux430ux440ux435ux43dux438ux44f-1}}

\begin{enumerate}
\def\labelenumi{\arabic{enumi}.}
\tightlist
\item
  \textbf{Туманы парения (дымы испарения):} Образуются, когда холодный
  воздух перемещается над теплой и влажной поверхностью (например, над
  водоемами, особенно над полыньями зимой).
\item
  \textbf{Фронтальные туманы:} Связаны с увлажнением воздуха выпадающими
  осадками, понижением температуры и снижением облачности в зоне
  атмосферных фронтов.
\end{enumerate}

\hypertarget{ux432ux43bux438ux44fux43dux438ux435-ux441ux43cux435ux448ux435ux43dux438ux44f-ux438-ux430ux43dux442ux440ux43eux43fux43eux433ux435ux43dux43dux44bux445-ux444ux430ux43aux442ux43eux440ux43eux432}{%
\subsubsection{Влияние Смешения и Антропогенных
Факторов}\label{ux432ux43bux438ux44fux43dux438ux435-ux441ux43cux435ux448ux435ux43dux438ux44f-ux438-ux430ux43dux442ux440ux43eux43fux43eux433ux435ux43dux43dux44bux445-ux444ux430ux43aux442ux43eux440ux43eux432}}

Процессы вовлечения и смешения воздушных масс облака и окружающей среды
играют определяющую роль в формировании туманов. В городских условиях
туманы могут распространяться за пределы города по направлению ветра.
Городские туманы формируются под влиянием общих факторов
туманообразования (радиационных, адвективных, испарения) с учетом
местных особенностей города. Морозные (поселковые, печные, аэродромные)
туманы возникают при сильных морозах, когда имеется дополнительный
источник водяного пара (например, от печей или двигателей самолетов).
Искусственные туманы и дымы, создаваемые, например, для защиты садов от
заморозков, более устойчивы при слабом ветре в приземном слое.

ewpage

\hypertarget{ux43fux440ux43eux433ux43dux43eux437-ux440ux430ux434ux438ux430ux446ux438ux43eux43dux43dux44bux445-ux442ux443ux43cux430ux43dux43eux432}{%
\section{Прогноз Радиационных
Туманов}\label{ux43fux440ux43eux433ux43dux43eux437-ux440ux430ux434ux438ux430ux446ux438ux43eux43dux43dux44bux445-ux442ux443ux43cux430ux43dux43eux432}}

Прогноз радиационных туманов является сложной задачей, требующей учета
множества факторов. Основной причиной радиационного тумана является
ночное понижение температуры в приземном слое ниже начального значения
точки росы, вызванное радиационным охлаждением подстилающей поверхности.

\hypertarget{ux431ux43bux430ux433ux43eux43fux440ux438ux44fux442ux43dux44bux435-ux443ux441ux43bux43eux432ux438ux44f}{%
\subsubsection{Благоприятные
Условия}\label{ux431ux43bux430ux433ux43eux43fux440ux438ux44fux442ux43dux44bux435-ux443ux441ux43bux43eux432ux438ux44f}}

Наиболее благоприятные условия для образования радиационного тумана
включают:

\begin{itemize}
\tightlist
\item
  Ясное или малооблачное небо (менее 4 баллов).
\item
  Слабый ветер (менее 3 м/с).
\item
  Высокая относительная влажность воздуха у поверхности земли (85\% и
  более).
\item
  Наличие инверсии температуры в приземном слое атмосферы или приземного
  слоя с изотермией.
\item
  Центральные части антициклонов и оси барических гребней.
\end{itemize}

\hypertarget{ux43aux43bux44eux447ux435ux432ux44bux435-ux430ux441ux43fux435ux43aux442ux44b-ux43fux440ux43eux433ux43dux43eux437ux430}{%
\subsubsection{Ключевые Аспекты
Прогноза}\label{ux43aux43bux44eux447ux435ux432ux44bux435-ux430ux441ux43fux435ux43aux442ux44b-ux43fux440ux43eux433ux43dux43eux437ux430}}

В идеальном случае прогноз радиационного тумана должен отвечать на
следующие вопросы:

\begin{enumerate}
\def\labelenumi{\arabic{enumi}.}
\tightlist
\item
  Будет ли туман и когда он возникнет?
\item
  Какая минимальная видимость ожидается в тумане?
\item
  Какова будет вертикальная протяженность слоя тумана?
\item
  Какова будет горизонтальная протяженность тумана?
\item
  Когда туман рассеется?
\end{enumerate}

\hypertarget{ux43cux435ux442ux43eux434ux438ux43aux430-ux43fux440ux43eux433ux43dux43eux437ux438ux440ux43eux432ux430ux43dux438ux44f}{%
\subsubsection{Методика
Прогнозирования}\label{ux43cux435ux442ux43eux434ux438ux43aux430-ux43fux440ux43eux433ux43dux43eux437ux438ux440ux43eux432ux430ux43dux438ux44f}}

\begin{enumerate}
\def\labelenumi{\arabic{enumi}.}
\tightlist
\item
  \textbf{Прогноз возникновения:} Для определения возможности
  образования тумана необходимо сравнить ожидаемую температуру начала
  образования тумана (\(T_Т\)) с прогнозируемой минимальной температурой
  воздуха в приземном слое (\(T_{мин}\)). Если \(T_Т > T_{мин}\), туман
  возможен, в противном случае --- маловероятен. \(T_Т\) прогнозируется
  по начальному значению точки росы (\(T_d\)), скорректированному на
  возможное понижение точки росы до начала тумана (\(\delta T_d\)) и
  дополнительное охлаждение (\(\delta T_т\)), необходимое для достижения
  видимости менее 1 км. Для этих расчетов используются графические
  методы, связывающие видимость с водностью тумана и температурой. Также
  прекращение или слабое повышение температуры в ясную ночь может
  служить признаком начавшейся конденсации и возможного появления низкой
  облачности через 2--4 часа.
\item
  \textbf{Сроки прогнозирования:} Предварительный прогноз тумана обычно
  составляется одновременно с общим суточным прогнозом погоды. Уточнения
  вносятся на основании послеполуденных наблюдений (например, в 15 и 18
  часов). Туман прогнозируется при сохранении благоприятных условий или
  если он уже наблюдался в данной воздушной массе, сохраняющей свои
  свойства.
\item
  \textbf{Прогноз рассеяния:} Время рассеяния радиационного тумана
  определяется с учетом его предполагаемой вертикальной и горизонтальной
  протяженности, а также широты места и сезона. Летом радиационный туман
  обычно рассеивается через 1--2 часа после восхода солнца из-за
  быстрого прогрева подстилающей поверхности. Осенью туман может
  сохраняться до 3--5 часов после восхода солнца.
\end{enumerate}

ewpage

\hypertarget{ux43eux431ux43bux430ux43aux430-ux438-ux444ux438ux437ux438ux43aux43e-ux43cux435ux442ux435ux43eux440ux43eux43bux43eux433ux438ux447ux435ux441ux43aux438ux435-ux443ux441ux43bux43eux432ux438ux44f-ux438ux445-ux43eux431ux440ux430ux437ux43eux432ux430ux43dux438ux44f}{%
\section{Облака и Физико-Метеорологические Условия Их
Образования}\label{ux43eux431ux43bux430ux43aux430-ux438-ux444ux438ux437ux438ux43aux43e-ux43cux435ux442ux435ux43eux440ux43eux43bux43eux433ux438ux447ux435ux441ux43aux438ux435-ux443ux441ux43bux43eux432ux438ux44f-ux438ux445-ux43eux431ux440ux430ux437ux43eux432ux430ux43dux438ux44f}}

Понимание процессов облакообразования является краеугольным камнем в
динамической и синоптической метеорологии, поскольку облака и связанные
с ними осадки оказывают определяющее влияние на погодные и климатические
условия, а также на многие аспекты хозяйственной деятельности.
Атмосферные движения во взаимосвязи с тепло- и влагообменом представляют
собой основные факторы, определяющие погоду и климат.

\hypertarget{ux43eux431ux449ux438ux435-ux43fux43eux43dux44fux442ux438ux44f-ux438-ux43aux43bux430ux441ux441ux438ux444ux438ux43aux430ux446ux438ux44f}{%
\subsection{Общие Понятия и
Классификация}\label{ux43eux431ux449ux438ux435-ux43fux43eux43dux44fux442ux438ux44f-ux438-ux43aux43bux430ux441ux441ux438ux444ux438ux43aux430ux446ux438ux44f}}

Облака представляют собой скопления взвешенных в атмосфере мельчайших
капель воды или кристаллов льда, образующихся при конденсации или
сублимации водяного пара. Информация о количестве, высоте границ,
водности, обледенении и других характеристиках облаков критически важна
для различных отраслей, включая авиацию, транспорт, строительство,
энергетику и сельское хозяйство.

В современной международной классификации выделяют 10 основных родов
облаков, подразделяемых на виды по особенностям формы и внутренней
структуры. Эти роды делятся по ярусам в тропосфере:

\begin{itemize}
\tightlist
\item
  \textbf{Высокий ярус} (вершины выше 6 км): Cirrus, Cirrocumulus,
  Cirrostratus.
\item
  \textbf{Средний ярус} (нижняя граница 2-6 км): Altocumulus,
  Altostratus.
\item
  \textbf{Нижний ярус} (нижняя граница менее 2 км): Nimbostratus,
  Stratus, Stratocumulus.
\item
  \textbf{Облака вертикального развития} (охватывают несколько ярусов,
  от 0.5 км до тропопаузы): Cumulus, Cumulonimbus.
\end{itemize}

\hypertarget{ux444ux443ux43dux434ux430ux43cux435ux43dux442ux430ux43bux44cux43dux44bux435-ux444ux438ux437ux438ux447ux435ux441ux43aux438ux435-ux443ux441ux43bux43eux432ux438ux44f-ux43eux431ux440ux430ux437ux43eux432ux430ux43dux438ux44f-ux43eux431ux43bux430ux43aux43eux432}{%
\subsection{Фундаментальные Физические Условия Образования
Облаков}\label{ux444ux443ux43dux434ux430ux43cux435ux43dux442ux430ux43bux44cux43dux44bux435-ux444ux438ux437ux438ux447ux435ux441ux43aux438ux435-ux443ux441ux43bux43eux432ux438ux44f-ux43eux431ux440ux430ux437ux43eux432ux430ux43dux438ux44f-ux43eux431ux43bux430ux43aux43eux432}}

Образование облаков -- это сложный физический процесс, требующий
совпадения нескольких ключевых условий:

\hypertarget{ux43dux430ux43bux438ux447ux438ux435-ux432ux43eux434ux44fux43dux43eux433ux43e-ux43fux430ux440ux430}{%
\subsubsection{1. Наличие Водяного
Пара}\label{ux43dux430ux43bux438ux447ux438ux435-ux432ux43eux434ux44fux43dux43eux433ux43e-ux43fux430ux440ux430}}

Тропосфера содержит практически весь водяной пар атмосферы. Испарение с
подстилающей поверхности (суши и океана) является основным источником
водяного пара в атмосфере.

\hypertarget{ux43eux445ux43bux430ux436ux434ux435ux43dux438ux435-ux432ux43eux437ux434ux443ux445ux430-ux434ux43e-ux442ux43eux447ux43aux438-ux440ux43eux441ux44b-1}{%
\subsubsection{2. Охлаждение Воздуха до Точки
Росы}\label{ux43eux445ux43bux430ux436ux434ux435ux43dux438ux435-ux432ux43eux437ux434ux443ux445ux430-ux434ux43e-ux442ux43eux447ux43aux438-ux440ux43eux441ux44b-1}}

Конденсация или сублимация водяного пара происходит, когда воздух
охлаждается до температуры, при которой он достигает состояния
насыщения, т.е. когда его относительная влажность достигает 100\%.
Основным механизмом охлаждения, приводящим к облакообразованию, является
адиабатическое расширение воздуха при его восходящем движении.

\hypertarget{ux43dux430ux43bux438ux447ux438ux435-ux44fux434ux435ux440-ux43aux43eux43dux434ux435ux43dux441ux430ux446ux438ux438ux441ux443ux431ux43bux438ux43cux430ux446ux438ux438}{%
\subsubsection{3. Наличие Ядер
Конденсации/Сублимации}\label{ux43dux430ux43bux438ux447ux438ux435-ux44fux434ux435ux440-ux43aux43eux43dux434ux435ux43dux441ux430ux446ux438ux438ux441ux443ux431ux43bux438ux43cux430ux446ux438ux438}}

Для начала конденсации в воздухе необходимы центры, вокруг которых могут
образовываться капли или кристаллы -- аэрозольные частицы, такие как
дым, сажа, пыль, морская соль.

\hypertarget{ux434ux438ux43dux430ux43cux438ux447ux435ux441ux43aux438ux435-ux444ux430ux43aux442ux43eux440ux44b-1}{%
\subsection{Динамические
Факторы}\label{ux434ux438ux43dux430ux43cux438ux447ux435ux441ux43aux438ux435-ux444ux430ux43aux442ux43eux440ux44b-1}}

Динамические факторы играют определяющую роль в формировании и эволюции
облаков всех классов.

\hypertarget{ux432ux435ux440ux442ux438ux43aux430ux43bux44cux43dux44bux435-ux434ux432ux438ux436ux435ux43dux438ux44f-ux432ux43eux437ux434ux443ux445ux430-w}{%
\subsubsection{1. Вертикальные Движения Воздуха
(w)}\label{ux432ux435ux440ux442ux438ux43aux430ux43bux44cux43dux44bux435-ux434ux432ux438ux436ux435ux43dux438ux44f-ux432ux43eux437ux434ux443ux445ux430-w}}

Восходящие движения воздуха являются главной причиной образования
облаков. При подъеме воздух адиабатически охлаждается, что ведет к
конденсации водяного пара.

\begin{itemize}
\tightlist
\item
  \textbf{Крупномасштабные упорядоченные движения (синоптического
  масштаба)}: Они связаны с областями пониженного давления (циклонами и
  ложбинами) и атмосферными фронтами. Такие движения приводят к
  образованию обширных слоистообразных облаков (Ns-As-Cs). В этих
  системах восходящее движение воздуха способствует достижению состояния
  насыщения, а затем и конденсации.
\item
  \textbf{Конвективные движения}: Возникают в условиях неустойчивой
  стратификации атмосферы. Они приводят к образованию кучевых и
  кучево-дождевых облаков (Cu, Cb). Мощные кучевые облака могут
  развиваться на всю толщу тропосферы с очень сильными вертикальными
  скоростями (до десятков метров в секунду).

  \begin{itemize}
  \tightlist
  \item
    \textbf{Взрывной циклогенез} -- это явление, при котором в
    формирующемся циклоне давление в центре падает за сутки не менее чем
    на 24 гПа, что указывает на очень интенсивные процессы.
  \item
    \textbf{Зоны внутритропической конвергенции (ВЗК)} являются
    генераторами вертикальных потоков воздуха, способствующих развитию
    мощных кучевых облаков, достигающих тропопаузы и распространяющихся
    в стратосферу, играя важную роль в тепло- и влагообмене.
  \end{itemize}
\end{itemize}

\hypertarget{ux430ux434ux432ux435ux43aux446ux438ux44f-ux43fux435ux440ux435ux43dux43eux441}{%
\subsubsection{2. Адвекция
(Перенос)}\label{ux430ux434ux432ux435ux43aux446ux438ux44f-ux43fux435ux440ux435ux43dux43eux441}}

Адвекция, т.е. горизонтальный перенос воздушных масс, существенно влияет
на влажностный и термический режим атмосферы.

\begin{itemize}
\tightlist
\item
  \textbf{Адвекция холода} (перенос холодного воздуха на теплую
  подстилающую поверхность) или \textbf{адвекция тепла} (перенос
  теплого, влажного воздуха на холодную поверхность) способствуют
  созданию условий для конденсации и облакообразования. Например, зимние
  циклоны на океанах часто зарождаются под влиянием адвекции холода, а
  на материках -- антициклоны под влиянием адвекции тепла.
\item
  \textbf{Фронты} (теплые, холодные, окклюзии) являются областями
  взаимодействия воздушных масс с различными физическими свойствами, где
  наблюдаются значительные градиенты температуры и влажности. В зонах
  фронтов, особенно в барических ложбинах, происходит сходимость
  воздушных потоков (конвергенция), что способствует возникновению
  восходящих движений и образованию мощных систем облачности и осадков.
\end{itemize}

\hypertarget{ux442ux443ux440ux431ux443ux43bux435ux43dux442ux43dux44bux439-ux43eux431ux43cux435ux43d-ux438-ux43fux435ux440ux435ux43cux435ux448ux438ux432ux430ux43dux438ux435}{%
\subsubsection{3. Турбулентный Обмен и
Перемешивание}\label{ux442ux443ux440ux431ux443ux43bux435ux43dux442ux43dux44bux439-ux43eux431ux43cux435ux43d-ux438-ux43fux435ux440ux435ux43cux435ux448ux438ux432ux430ux43dux438ux435}}

Турбулентное движение воздуха, характеризующееся беспорядочными
колебаниями скорости, давления и плотности (пульсациями), играет
значительную роль в переносе тепла, влаги и примесей.

\begin{itemize}
\tightlist
\item
  \textbf{Турбулентная диффузия (перемешивание)} водяного пара по
  горизонтали и вертикали приводит к выравниванию его концентрации и
  может способствовать достижению состояния насыщения, особенно вблизи
  подстилающей поверхности.
\item
  \textbf{Вовлечение и смешение воздушных масс} также играют
  определяющую роль, особенно в увеличении размеров облаков и в
  формировании осадков. Смешение воздушных масс с разными температурами
  и влажностями может привести к конденсации.
\end{itemize}

\hypertarget{ux442ux435ux440ux43cux43eux434ux438ux43dux430ux43cux438ux447ux435ux441ux43aux438ux435-ux444ux430ux43aux442ux43eux440ux44b}{%
\subsection{Термодинамические
Факторы}\label{ux442ux435ux440ux43cux43eux434ux438ux43dux430ux43cux438ux447ux435ux441ux43aux438ux435-ux444ux430ux43aux442ux43eux440ux44b}}

\hypertarget{ux432ux435ux440ux442ux438ux43aux430ux43bux44cux43dux430ux44f-ux443ux441ux442ux43eux439ux447ux438ux432ux43eux441ux442ux44c-ux430ux442ux43cux43eux441ux444ux435ux440ux44b}{%
\subsubsection{1. Вертикальная Устойчивость
Атмосферы}\label{ux432ux435ux440ux442ux438ux43aux430ux43bux44cux43dux430ux44f-ux443ux441ux442ux43eux439ux447ux438ux432ux43eux441ux442ux44c-ux430ux442ux43cux43eux441ux444ux435ux440ux44b}}

Состояние вертикальной устойчивости атмосферы (отношение фактического
вертикального градиента температуры к адиабатическим градиентам)
определяет тип облаков и их развитие.

\begin{itemize}
\tightlist
\item
  \textbf{Устойчивая стратификация} (фактический градиент температуры
  меньше влажноадиабатического) способствует образованию слоистых
  облаков (Stratus).
\item
  \textbf{Неустойчивая стратификация} (фактический градиент температуры
  больше влажноадиабатического) приводит к развитию конвективных облаков
  (Cumulus, Cumulonimbus). Зимой достижение влажно-неустойчивого
  состояния значительно труднее из-за более высокого значения
  влажноадиабатического градиента при низких температурах.
\item
  \textbf{Скрытая теплота конденсации}, выделяющаяся при фазовых
  переходах воды, оказывает существенное влияние на сохранение разности
  температур в облаке и окружающей среде, а также на устойчивость
  атмосферы, способствуя развитию облака.
\end{itemize}

\hypertarget{ux440ux430ux441ux43fux440ux435ux434ux435ux43bux435ux43dux438ux435-ux442ux435ux43cux43fux435ux440ux430ux442ux443ux440ux44b-ux438-ux432ux43bux430ux436ux43dux43eux441ux442ux438-ux43fux43e-ux432ux44bux441ux43eux442ux435}{%
\subsubsection{2. Распределение Температуры и Влажности по
Высоте}\label{ux440ux430ux441ux43fux440ux435ux434ux435ux43bux435ux43dux438ux435-ux442ux435ux43cux43fux435ux440ux430ux442ux443ux440ux44b-ux438-ux432ux43bux430ux436ux43dux43eux441ux442ux438-ux43fux43e-ux432ux44bux441ux43eux442ux435}}

Начальное распределение температуры и влажности по высоте является
критическим для прогноза облакообразования. Кривая точки росы
(депеграмма) характеризует изменение удельной влажности с высотой и,
наряду с кривой стратификации, используется для анализа условий
конденсации. Разность между температурой воздуха и точкой росы (дефицит
точки росы) является важным показателем близости к насыщению.

\hypertarget{ux43fux440ux43eux447ux438ux435-ux432ux43bux438ux44fux44eux449ux438ux435-ux444ux430ux43aux442ux43eux440ux44b}{%
\subsection{Прочие Влияющие
Факторы}\label{ux43fux440ux43eux447ux438ux435-ux432ux43bux438ux44fux44eux449ux438ux435-ux444ux430ux43aux442ux43eux440ux44b}}

\hypertarget{ux440ux430ux434ux438ux430ux446ux438ux43eux43dux43dux44bux439-ux440ux435ux436ux438ux43c}{%
\subsubsection{1. Радиационный
Режим}\label{ux440ux430ux434ux438ux430ux446ux438ux43eux43dux43dux44bux439-ux440ux435ux436ux438ux43c}}

Облачность сильно изменяет радиационный баланс земной поверхности и
термический режим атмосферы, что в свою очередь влияет на температуру и
влажность.

\begin{itemize}
\tightlist
\item
  Облачность уменьшает приток солнечной радиации к земной поверхности и
  ее эффективное излучение. Летом в циклоне, мощная облачность понижает
  температуру воздуха и ослабляет адвекцию холода. Зимой, наоборот,
  облачность способствует повышению температуры в циклоне и сохранению
  адвективного притока холода, что продлевает существование циклонов.
\item
  Суточные колебания облачности частично объясняются
  радиационно-термическим фактором, однако основную роль играют
  динамические факторы.
\end{itemize}

\hypertarget{ux440ux435ux43bux44cux435ux444-ux43eux440ux43eux433ux440ux430ux444ux438ux44f}{%
\subsubsection{2. Рельеф
(Орография)}\label{ux440ux435ux43bux44cux435ux444-ux43eux440ux43eux433ux440ux430ux444ux438ux44f}}

Горный рельеф значительно влияет на ветер и может приводить к
орографическому подъему воздуха, способствуя образованию облачности и
осадков на наветренных склонах. Нисходящие движения на подветренных
склонах, напротив, вызывают размывание облаков.

\hypertarget{ux445ux430ux440ux430ux43aux442ux435ux440-ux43fux43eux434ux441ux442ux438ux43bux430ux44eux449ux435ux439-ux43fux43eux432ux435ux440ux445ux43dux43eux441ux442ux438}{%
\subsubsection{3. Характер Подстилающей
Поверхности}\label{ux445ux430ux440ux430ux43aux442ux435ux440-ux43fux43eux434ux441ux442ux438ux43bux430ux44eux449ux435ux439-ux43fux43eux432ux435ux440ux445ux43dux43eux441ux442ux438}}

Теплофизические свойства подстилающей поверхности (суша, океан, снежный
покров) влияют на температурные контрасты, влаго- и теплообмен, что
сказывается на устойчивости воздушных масс и, как следствие, на
облакообразовании. Перемещение циклонов или антициклонов с одной
подстилающей поверхности на другую может изменить интенсивность их
развития и связанные с ними облачные процессы.

\hypertarget{ux430ux43dux442ux440ux43eux43fux43eux433ux435ux43dux43dux44bux435-ux444ux430ux43aux442ux43eux440ux44b-1}{%
\subsubsection{4. Антропогенные
Факторы}\label{ux430ux43dux442ux440ux43eux43fux43eux433ux435ux43dux43dux44bux435-ux444ux430ux43aux442ux43eux440ux44b-1}}

Деятельность человека, особенно в крупных городах, может влиять на
образование туманов и облаков за счет дополнительных источников водяного
пара (сжигание топлива) и аэрозолей (загрязнение атмосферы).

Таким образом, формирование облаков -- это результат сложного
взаимодействия термодинамических и динамических процессов, причем
ведущая роль принадлежит крупномасштабным вертикальным движениям и
горизонтальным контрастам температуры и влажности, обусловленным
бароклинностью атмосферы.

ewpage

\hypertarget{ux440ux43eux43bux44c-ux432ux435ux440ux442ux438ux43aux430ux43bux44cux43dux44bux445-ux434ux432ux438ux436ux435ux43dux438ux439-ux440ux430ux437ux43bux438ux447ux43dux43eux433ux43e-ux43cux430ux441ux448ux442ux430ux431ux430-ux442ux443ux440ux431ux443ux43bux435ux43dux442ux43dux43eux433ux43e-ux43fux435ux440ux435ux43cux435ux448ux438ux432ux430ux43dux438ux44f-ux438-ux440ux430ux434ux438ux430ux446ux438ux43eux43dux43dux43eux433ux43e-ux432ux44bux445ux43eux43bux430ux436ux438ux432ux430ux43dux438ux44f-ux432-ux43eux431ux440ux430ux437ux43eux432ux430ux43dux438ux438-ux43eux431ux43bux430ux43aux43eux432}{%
\section{Роль Вертикальных Движений Различного Масштаба, Турбулентного
Перемешивания и Радиационного Выхолаживания в Образовании
Облаков}\label{ux440ux43eux43bux44c-ux432ux435ux440ux442ux438ux43aux430ux43bux44cux43dux44bux445-ux434ux432ux438ux436ux435ux43dux438ux439-ux440ux430ux437ux43bux438ux447ux43dux43eux433ux43e-ux43cux430ux441ux448ux442ux430ux431ux430-ux442ux443ux440ux431ux443ux43bux435ux43dux442ux43dux43eux433ux43e-ux43fux435ux440ux435ux43cux435ux448ux438ux432ux430ux43dux438ux44f-ux438-ux440ux430ux434ux438ux430ux446ux438ux43eux43dux43dux43eux433ux43e-ux432ux44bux445ux43eux43bux430ux436ux438ux432ux430ux43dux438ux44f-ux432-ux43eux431ux440ux430ux437ux43eux432ux430ux43dux438ux438-ux43eux431ux43bux430ux43aux43eux432}}

Образование облаков является фундаментальным процессом в атмосфере,
определяющим погоду и климат. Оно представляет собой комплексное
взаимодействие множества факторов, ключевыми из которых являются
вертикальные движения воздуха различного масштаба, турбулентное
перемешивание и радиационное выхолаживание. Эти процессы не действуют
изолированно, а взаимосвязаны и могут усиливать или ослаблять друг
друга.

\hypertarget{ux440ux43eux43bux44c-ux432ux435ux440ux442ux438ux43aux430ux43bux44cux43dux44bux445-ux434ux432ux438ux436ux435ux43dux438ux439-ux440ux430ux437ux43bux438ux447ux43dux43eux433ux43e-ux43cux430ux441ux448ux442ux430ux431ux430}{%
\subsection{Роль Вертикальных Движений Различного
Масштаба}\label{ux440ux43eux43bux44c-ux432ux435ux440ux442ux438ux43aux430ux43bux44cux43dux44bux445-ux434ux432ux438ux436ux435ux43dux438ux439-ux440ux430ux437ux43bux438ux447ux43dux43eux433ux43e-ux43cux430ux441ux448ux442ux430ux431ux430}}

Вертикальные движения воздуха играют исключительно важную роль в
формировании облаков и осадков. Причиной образования облаков является
адиабатическое охлаждение поднимающегося воздуха, приводящее к
достижению состояния насыщения и конденсации водяного пара. Напротив,
нисходящие движения приводят к адиабатическому нагреванию воздуха,
рассеиванию облаков и предотвращению их образования. Скорость
вертикальных движений может значительно варьироваться -- от сантиметров
в секунду для крупномасштабных процессов до десятков метров в секунду
для мощной конвекции.

\hypertarget{ux430ux434ux438ux430ux431ux430ux442ux438ux447ux435ux441ux43aux438ux435-ux43fux440ux43eux446ux435ux441ux441ux44b-2}{%
\subsubsection{Адиабатические
Процессы}\label{ux430ux434ux438ux430ux431ux430ux442ux438ux447ux435ux441ux43aux438ux435-ux43fux440ux43eux446ux435ux441ux441ux44b-2}}

При подъеме воздушной частицы в атмосферу, где давление с высотой
падает, происходит ее расширение. Это расширение совершается за счет
внутренней энергии воздуха, что приводит к его охлаждению -- так
называемому адиабатическому охлаждению. Если влажный воздух охлаждается
до точки росы, водяной пар начинает конденсироваться, образуя капли воды
или кристаллы льда, видимые как облака. Выше уровня конденсации, когда
воздух насыщен водяным паром и продолжает подниматься, его охлаждение
замедляется из-за выделения скрытой теплоты конденсации -- это
влажно-адиабатический процесс. Вертикальный градиент температуры в
облаке (влажноадиабатический) меньше сухоадиабатического, что имеет
решающее значение для вертикальной устойчивости.

\hypertarget{ux43cux430ux441ux448ux442ux430ux431ux44b-ux432ux435ux440ux442ux438ux43aux430ux43bux44cux43dux44bux445-ux434ux432ux438ux436ux435ux43dux438ux439}{%
\subsubsection{Масштабы Вертикальных
Движений}\label{ux43cux430ux441ux448ux442ux430ux431ux44b-ux432ux435ux440ux442ux438ux43aux430ux43bux44cux43dux44bux445-ux434ux432ux438ux436ux435ux43dux438ux439}}

Вертикальные движения в атмосфере различаются по своим масштабам и
механизмам формирования:

\begin{itemize}
\tightlist
\item
  \textbf{Крупномасштабные (синоптические) движения}: Эти движения
  охватывают огромные территории (тысячи километров) и связаны с
  циклонами, антициклонами и атмосферными фронтами. Скорости таких
  движений обычно порядка нескольких сантиметров в секунду.

  \begin{itemize}
  \tightlist
  \item
    \textbf{В циклонах и ложбинах}: Характеризуются восходящими
    движениями. В них формируются обширные системы слоистообразных
    облаков, таких как слоисто-дождевые (Ns), высоко-слоистые (As) и
    перисто-слоистые (Cs). Облачная система циклона может иметь
    клинообразный вид.
  \item
    \textbf{В антициклонах и гребнях}: Преобладают нисходящие движения.
    Это приводит к адиабатическому нагреванию воздуха, уменьшению
    относительной влажности и, как следствие, препятствует образованию
    облаков, формируя преимущественно безоблачную погоду. Однако, в
    антициклонах могут наблюдаться слоисто-кучевые (Sc) и слоистые (St)
    облака переносно-трансформационной природы, а также плоские кучевые
    (Cu hum) и средние (Cu med), связанные с сухонеустойчивой
    стратификацией в приземном слое.
  \item
    \textbf{Атмосферные фронты}: Наклонные поверхности раздела между
    воздушными массами с различными свойствами. Восходящие движения
    стимулируются у фронтов, способствуя образованию облачности и
    осадков. На теплых фронтах восходящие движения теплого воздуха
    формируют облачные системы типа Ci-Cs, As-Ns. На холодных фронтах
    2-го рода мощный подъем перед фронтом приводит к развитию
    кучево-дождевых (Cb) облаков и ливневых осадков.
  \end{itemize}
\item
  \textbf{Мезомасштабные и конвективные движения}: Имеют горизонтальные
  размеры от сотен метров до сотен километров. Скорости вертикальных
  движений могут достигать 10 м/с и более. Эти движения ответственны за
  образование кучевообразных облаков (Cu, Cb), известных как
  конвективные. Мезомасштабные циклоны и зоны конвергенции также могут
  порождать интенсивные конвективные облака и ливневые осадки.
\item
  \textbf{Орографические движения}: Подъем воздуха по склонам гор
  приводит к адиабатическому охлаждению и конденсации, вызывая
  орографические осадки на наветренной стороне. Влияние гор
  распространяется на сотни километров по горизонтали и несколько
  километров по вертикали, вызывая деформацию воздушного потока и
  изменение свойств воздушных масс.
\end{itemize}

\hypertarget{ux441ux445ux43eux434ux438ux43cux43eux441ux442ux44c-ux438-ux440ux430ux441ux445ux43eux434ux438ux43cux43eux441ux442ux44c-ux43fux43eux442ux43eux43aux43eux432}{%
\subsubsection{Сходимость и Расходимость
Потоков}\label{ux441ux445ux43eux434ux438ux43cux43eux441ux442ux44c-ux438-ux440ux430ux441ux445ux43eux434ux438ux43cux43eux441ux442ux44c-ux43fux43eux442ux43eux43aux43eux432}}

Дивергенция (расхождение) вектора скорости, или дивергенция скорости
ветра, характеризует относительное изменение объема воздушной массы за
единицу времени. Отрицательная дивергенция (конвергенция) указывает на
сходимость воздушных потоков, что приводит к восходящим движениям и
способствует образованию облаков, тогда как положительная дивергенция
(расходимость) связана с нисходящими движениями. Конвергенция как
синоптического, так и мезомасштаба создает условия для возникновения
вертикальных движений и увеличения вертикального градиента температуры
во времени.

\hypertarget{ux440ux43eux43bux44c-ux442ux443ux440ux431ux443ux43bux435ux43dux442ux43dux43eux433ux43e-ux43fux435ux440ux435ux43cux435ux448ux438ux432ux430ux43dux438ux44f}{%
\subsection{Роль Турбулентного
Перемешивания}\label{ux440ux43eux43bux44c-ux442ux443ux440ux431ux443ux43bux435ux43dux442ux43dux43eux433ux43e-ux43fux435ux440ux435ux43cux435ux448ux438ux432ux430ux43dux438ux44f}}

Турбулентное движение жидкости или газа -- это сложное, запутанное
движение множества отдельных частиц и вихрей, накладывающееся на
основное осредненное движение. Оно характеризуется резкими
беспорядочными колебаниями метеорологических величин, называемыми
пульсациями. Турбулентность играет ключевую роль в процессах переноса
тепла, влаги, импульса и примесей в атмосфере.

\hypertarget{ux441ux443ux442ux44c-ux442ux443ux440ux431ux443ux43bux435ux43dux442ux43dux43eux441ux442ux438-ux438-ux43fux435ux440ux435ux43dux43eux441ux430}{%
\subsubsection{Суть Турбулентности и
Переноса}\label{ux441ux443ux442ux44c-ux442ux443ux440ux431ux443ux43bux435ux43dux442ux43dux43eux441ux442ux438-ux438-ux43fux435ux440ux435ux43dux43eux441ux430}}

В отличие от молекулярной диффузии, турбулентный обмен значительно более
эффективен и приводит к интенсивному перемешиванию воздушных масс.
Коэффициенты турбулентного обмена (или турбулентной диффузии)
характеризуют этот перенос. Уравнение переноса водяного пара в
турбулентной атмосфере включает члены, описывающие турбулентную диффузию
удельной влажности. Турбулентный поток какой-либо величины (например,
влаги) пропорционален градиенту этой величины, с отрицательным знаком,
указывающим на перенос в сторону уменьшения среднего значения.

\hypertarget{ux432ux43eux432ux43bux435ux447ux435ux43dux438ux435-ux438-ux441ux43cux435ux448ux435ux43dux438ux435-ux432-ux43eux431ux43bux430ux43aux430ux445}{%
\subsubsection{Вовлечение и Смешение в
Облаках}\label{ux432ux43eux432ux43bux435ux447ux435ux43dux438ux435-ux438-ux441ux43cux435ux448ux435ux43dux438ux435-ux432-ux43eux431ux43bux430ux43aux430ux445}}

В развивающихся конвективных облаках (Cu, Cb) и циклонах происходит
вовлечение (entrainment) окружающего воздуха, что является формой
турбулентного смешения. Этот процесс играет определяющую роль в
увеличении размеров облака и его водности. Смешение воздуха с различными
физическими свойствами (например, теплого влажного воздуха с более
холодным сухим воздухом) может приводить к конденсации водяного пара и
образованию облака значительной водности. Важно отметить, что даже при
неустойчивой стратификации, где восходящие потоки преобладают,
конденсация может происходить в нисходящих потоках или при смешении
соседних струй.

\hypertarget{ux442ux443ux440ux431ux443ux43bux435ux43dux442ux43dux43eux441ux442ux44c-ux432-ux43fux43eux433ux440ux430ux43dux438ux447ux43dux43eux43c-ux441ux43bux43eux44e}{%
\subsubsection{Турбулентность в Пограничном
Слою}\label{ux442ux443ux440ux431ux443ux43bux435ux43dux442ux43dux43eux441ux442ux44c-ux432-ux43fux43eux433ux440ux430ux43dux438ux447ux43dux43eux43c-ux441ux43bux43eux44e}}

Наиболее резко турбулентный характер атмосферных движений проявляется в
пограничном слое атмосферы (до 1000-1500 м над землей), где происходит
интенсивный обмен энергией и влагой с подстилающей поверхностью.
Турбулентное перемешивание в приземном слое способствует формированию
сухонеустойчивой стратификации при притоке солнечной радиации к земной
поверхности. Это может привести к образованию слаборазвитых кучевых
облаков.

\hypertarget{ux440ux43eux43bux44c-ux440ux430ux434ux438ux430ux446ux438ux43eux43dux43dux43eux433ux43e-ux432ux44bux445ux43eux43bux430ux436ux438ux432ux430ux43dux438ux44f}{%
\subsection{Роль Радиационного
Выхолаживания}\label{ux440ux43eux43bux44c-ux440ux430ux434ux438ux430ux446ux438ux43eux43dux43dux43eux433ux43e-ux432ux44bux445ux43eux43bux430ux436ux438ux432ux430ux43dux438ux44f}}

Радиационный теплообмен в атмосфере происходит в результате поглощения и
излучения электромагнитных волн слоями воздуха. В контексте образования
облаков особенно важным является длинноволновое (инфракрасное)
излучение, испускаемое Землей и атмосферой.

\hypertarget{ux43fux440ux43eux446ux435ux441ux441ux44b-ux434ux43bux438ux43dux43dux43eux432ux43eux43bux43dux43eux432ux43eux433ux43e-ux438ux437ux43bux443ux447ux435ux43dux438ux44f}{%
\subsubsection{Процессы Длинноволнового
Излучения}\label{ux43fux440ux43eux446ux435ux441ux441ux44b-ux434ux43bux438ux43dux43dux43eux432ux43eux43bux43dux43eux432ux43eux433ux43e-ux438ux437ux43bux443ux447ux435ux43dux438ux44f}}

Водяной пар, углекислый газ и озон являются основными поглотителями
земной радиации в инфракрасном диапазоне. В диапазоне длин волн от 8.5
до 12 мкм существует ``окно прозрачности'' атмосферы, через которое
излучение может уходить в космос. Радиационное выхолаживание, особенно в
ночные часы и при безоблачной погоде, приводит к понижению температуры
воздуха в приземном слое или в свободной атмосфере.

\hypertarget{ux43eux431ux440ux430ux437ux43eux432ux430ux43dux438ux435-ux442ux443ux43cux430ux43dux43eux432-ux438-ux43dux438ux437ux43aux438ux445-ux43eux431ux43bux430ux43aux43eux432}{%
\subsubsection{Образование Туманов и Низких
Облаков}\label{ux43eux431ux440ux430ux437ux43eux432ux430ux43dux438ux435-ux442ux443ux43cux430ux43dux43eux432-ux438-ux43dux438ux437ux43aux438ux445-ux43eux431ux43bux430ux43aux43eux432}}

Радиационное выхолаживание является одной из ключевых причин образования
туманов охлаждения, таких как радиационные туманы. В результате ночного
выхолаживания поверхности земли и прилегающего к ней воздуха,
температура может опуститься ниже точки росы, приводя к конденсации и
образованию тумана. Такие туманы могут быть поземными, низкими, средними
или высокими в зависимости от вертикальной протяженности. Радиационное
охлаждение верхней границы облаков, особенно ночью, может усиливать
моросящие и обложные осадки. В антициклонах, где преобладают нисходящие
движения, радиационное выхолаживание может способствовать образованию
слоистых (St) и слоисто-кучевых (Sc) облаков под слоем инверсии
оседания, а также радиационных туманов.

\hypertarget{ux432ux43bux438ux44fux43dux438ux435-ux43eux431ux43bux430ux447ux43dux43eux441ux442ux438-ux43dux430-ux440ux430ux434ux438ux430ux446ux438ux43eux43dux43dux44bux439-ux431ux430ux43bux430ux43dux441}{%
\subsubsection{Влияние Облачности на Радиационный
Баланс}\label{ux432ux43bux438ux44fux43dux438ux435-ux43eux431ux43bux430ux447ux43dux43eux441ux442ux438-ux43dux430-ux440ux430ux434ux438ux430ux446ux438ux43eux43dux43dux44bux439-ux431ux430ux43bux430ux43dux441}}

Сами облака оказывают существенное влияние на радиационный баланс
атмосферы и земной поверхности. Они уменьшают приток солнечной
(коротковолновой) радиации к земной поверхности, но также снижают
эффективное длинноволновое излучение Земли. Результирующий приток
радиации под влиянием облаков увеличивается зимой (когда баланс
отрицателен) и уменьшается летом (когда он положителен).

\hypertarget{ux432ux437ux430ux438ux43cux43eux434ux435ux439ux441ux442ux432ux438ux435-ux444ux430ux43aux442ux43eux440ux43eux432}{%
\subsection{Взаимодействие
Факторов}\label{ux432ux437ux430ux438ux43cux43eux434ux435ux439ux441ux442ux432ux438ux435-ux444ux430ux43aux442ux43eux440ux43eux432}}

В действительности, облакообразование редко происходит под влиянием
одного из перечисленных факторов. Чаще всего наблюдается их комплексное
взаимодействие:

\begin{itemize}
\tightlist
\item
  \textbf{Термическая адвекция и вертикальные движения}: Адвекция холода
  может способствовать усилению циклонического вихря и восходящих
  движений, что приводит к образованию мощных облачных систем.
\item
  \textbf{Турбулентность и конвекция}: Приток солнечной радиации к
  земной поверхности может приводить к формированию сухонеустойчивой
  стратификации в приземном слое. Если при этом уровень конденсации
  расположен ниже задерживающего слоя, и нисходящие движения ослабевают
  или меняют знак на восходящие, кучевые облака могут начать развиваться
  до стадии мощных и кучево-дождевых.
\item
  \textbf{Радиация и турбулентность}: Радиационное выхолаживание может
  приводить к образованию инверсий, однако турбулентное перемешивание
  может изменять их структуру и влиять на возможность развития
  конвекции.
\item
  \textbf{Вовлечение и тепло конденсации}: Тепло конденсации,
  выделяющееся при образовании облаков, существенно влияет на сохранение
  разности температур между циклоном и окружающей средой, что в свою
  очередь влияет на адвекцию холода, обеспечивающую развитие вихря. Это
  также способствует формированию фронтальных зон и облачных систем.
  Вовлечение и смешение являются определяющими в увеличении размеров
  облака и формировании осадков.
\end{itemize}

Таким образом, образование облаков --- это результат сложной
динамической и термодинамической эволюции воздушных масс, где
вертикальные движения, турбулентное перемешивание и радиационные
процессы постоянно взаимодействуют, формируя наблюдаемое разнообразие
облачных систем и явлений погоды.

ewpage

\hypertarget{ux43cux435ux436ux434ux443ux43dux430ux440ux43eux434ux43dux430ux44f-ux43aux43bux430ux441ux441ux438ux444ux438ux43aux430ux446ux438ux44f-ux43eux431ux43bux430ux43aux43eux432}{%
\section{Международная Классификация
Облаков}\label{ux43cux435ux436ux434ux443ux43dux430ux440ux43eux434ux43dux430ux44f-ux43aux43bux430ux441ux441ux438ux444ux438ux43aux430ux446ux438ux44f-ux43eux431ux43bux430ux43aux43eux432}}

\hypertarget{ux43eux431ux449ux438ux435-ux43fux440ux438ux43dux446ux438ux43fux44b-ux438-ux441ux442ux440ux443ux43aux442ux443ux440ux430}{%
\subsection{Общие Принципы и
Структура}\label{ux43eux431ux449ux438ux435-ux43fux440ux438ux43dux446ux438ux43fux44b-ux438-ux441ux442ux440ux443ux43aux442ux443ux440ux430}}

В метеорологической практике формы облаков определяются на основании
Международной классификации облаков. Эта классификация является
фундаментальным инструментом для стандартизации наблюдений и обмена
метеорологической информацией, обеспечивая единое понимание и описание
атмосферных явлений на глобальном уровне. Она включает в себя десять
основных родов (видов) облачности. Для идентификации форм облачности в
ходе метеорологических наблюдений используется специализированный Атлас
облаков.

Следует отметить, что исторически первая классификация облаков была
разработана английским метеорологом-любителем Л. Говардом.

\hypertarget{ux434ux435ux441ux44fux442ux44c-ux440ux43eux434ux43eux432-ux43eux431ux43bux430ux43aux43eux432-ux438-ux438ux445-ux44fux440ux443ux441ux43dux43eux441ux442ux44c}{%
\subsection{Десять Родов Облаков и их
Ярусность}\label{ux434ux435ux441ux44fux442ux44c-ux440ux43eux434ux43eux432-ux43eux431ux43bux430ux43aux43eux432-ux438-ux438ux445-ux44fux440ux443ux441ux43dux43eux441ux442ux44c}}

Международная классификация подразделяет облака на десять родов,
которые, в свою очередь, группируются по ярусам в зависимости от высоты
их расположения в тропосфере.

\hypertarget{ux44fux440ux443ux441ux43dux43eux441ux442ux44c-ux438-ux432ux44bux441ux43eux442ux43dux44bux435-ux434ux438ux430ux43fux430ux437ux43eux43dux44b}{%
\subsubsection{Ярусность и Высотные
Диапазоны}\label{ux44fux440ux443ux441ux43dux43eux441ux442ux44c-ux438-ux432ux44bux441ux43eux442ux43dux44bux435-ux434ux438ux430ux43fux430ux437ux43eux43dux44b}}

\begin{itemize}
\tightlist
\item
  \textbf{Облака верхнего яруса}: В умеренных широтах их нижняя граница
  располагается выше 6 км. В целом, они могут находиться на высоте от 3
  до 8 км. К этому ярусу относятся:

  \begin{itemize}
  \tightlist
  \item
    \textbf{Перистые (Cirrus, Ci)}
  \item
    \textbf{Перисто-кучевые (Cirrocumulus, Cc)}
  \item
    \textbf{Перисто-слоистые (Cirrostratus, Cs)}
  \end{itemize}
\item
  \textbf{Облака среднего яруса}: Их нижняя граница находится в
  диапазоне от 2 до 6 км, или от 2 до 4 км. В пределах этого яруса
  выделяют:

  \begin{itemize}
  \tightlist
  \item
    \textbf{Высоко-кучевые (Altocumulus, Ac)}
  \item
    \textbf{Высоко-слоистые (Altostratus, As)}
  \end{itemize}
\item
  \textbf{Облака нижнего яруса}: Наблюдаются от малых высот до 2 км. К
  ним относятся:

  \begin{itemize}
  \tightlist
  \item
    \textbf{Слоистые (Stratus, St)}
  \item
    \textbf{Слоисто-кучевые (Stratocumulus, Sc)}
  \item
    \textbf{Слоисто-дождевые (Nimbostratus, Ns)}
  \end{itemize}
\item
  \textbf{Облака вертикального развития}: Эти облака начинаются в нижнем
  ярусе, но могут охватывать всю атмосферу до значительных высот. К ним
  относятся:

  \begin{itemize}
  \tightlist
  \item
    \textbf{Кучевые (Cumulus, Cu)}
  \item
    \textbf{Кучево-дождевые (Cumulonimbus, Cb)}
  \end{itemize}
\end{itemize}

Следует отметить, что, несмотря на упоминание в одном из источников о
Таблице 4, содержащей полный список 10 основных видов облаков, сама
таблица в предоставленном материале отсутствует.

\hypertarget{ux444ux438ux437ux438ux447ux435ux441ux43aux438ux435-ux43eux441ux43dux43eux432ux44b-ux444ux43eux440ux43cux438ux440ux43eux432ux430ux43dux438ux44f-ux43eux431ux43bux430ux43aux43eux432}{%
\subsection{Физические Основы Формирования
Облаков}\label{ux444ux438ux437ux438ux447ux435ux441ux43aux438ux435-ux43eux441ux43dux43eux432ux44b-ux444ux43eux440ux43cux438ux440ux43eux432ux430ux43dux438ux44f-ux43eux431ux43bux430ux43aux43eux432}}

Формирование облаков тесно связано с физическими процессами в атмосфере,
особенно с динамическими факторами и термодинамическими условиями:

\begin{itemize}
\tightlist
\item
  \textbf{Восходящие движения и барические поля}: Крупномасштабные
  вертикальные движения воздуха и синоптические вихри (циклоны и
  ложбины) играют определяющую роль в образовании всех классов облаков.
  В циклонах, под влиянием восходящих движений синоптического масштаба
  (скорость которых может достигать 1--10 см/с, а в глубоких циклонах до
  10 см/с), формируются системы слоистообразных облаков, таких как Ns,
  As, Cs. Эти облака преимущественно образуются в областях пониженного
  давления.
\item
  \textbf{Термическая стратификация}: Образование кучевых облаков,
  особенно в дневное время, значительно обусловлено неустойчивой
  термической стратификацией приземного слоя атмосферы. Кучевые облака
  формируются, когда воздушная частица, достигнув уровня конденсации,
  оказывается в условиях неустойчивости, что приводит к возникновению
  сильных вертикальных скоростей (до десятков метров в секунду),
  способных распространяться на всю тропосферу. Для кучевых облаков
  хорошей погоды (Cu hum) роль термического фактора наиболее
  значительна. В то же время, слоистые облака образуются в устойчивой
  атмосфере при малых вертикальных скоростях (1--2 см/с).
\item
  \textbf{Адвекция тепла и влаги}: Притоки водяного пара являются
  ключевым фактором, определяющим образование облаков. Горизонтальные
  контрасты температуры (геострофическая адвекция виртуальной
  температуры) тесно связаны с синоптическими вихрями и, как следствие,
  с формированием облаков всех классов. Тепло (явное и скрытое),
  поступающее из почвы и воды, также играет важную роль в образовании
  туманов и облаков.
\item
  \textbf{Ядра конденсации}: Ядра конденсации необходимы для образования
  капель (туманов), однако их наличие не является причиной различий в
  количестве облаков, туманов и осадков в разных точках и регионах
  Земли, поскольку их более чем достаточно везде.
\item
  \textbf{Фазовые превращения}: Тепло конденсации, выделяющееся при
  образовании облаков, существенно влияет на поддержание разности
  температур между циклоном и окружающей средой, что, в свою очередь,
  способствует развитию вихря и углублению циклона.
\end{itemize}

\hypertarget{ux43eux441ux43eux431ux435ux43dux43dux43eux441ux442ux438-ux43dux430ux431ux43bux44eux434ux435ux43dux438ux439-ux438-ux445ux430ux440ux430ux43aux442ux435ux440ux438ux441ux442ux438ux43aux438-ux43eux431ux43bux430ux43aux43eux432}{%
\subsection{Особенности Наблюдений и Характеристики
Облаков}\label{ux43eux441ux43eux431ux435ux43dux43dux43eux441ux442ux438-ux43dux430ux431ux43bux44eux434ux435ux43dux438ux439-ux438-ux445ux430ux440ux430ux43aux442ux435ux440ux438ux441ux442ux438ux43aux438-ux43eux431ux43bux430ux43aux43eux432}}

Метеорологические наблюдения за облаками включают оценку их количества
(по 10-балльной шкале), а также определение форм. Информация о
характеристиках облаков, таких как количество, высота границ, водность,
обледенение, дальность видимости, болтанка и грозовые разряды, широко
используется в различных сферах, включая авиацию и транспорт.
Распределение облачности в целом оказывает определяющее влияние на
формирование и колебания климата.

Исследования показывают, что крупномасштабные вертикальные движения и
синоптические вихри являются определяющими факторами в формировании и
эволюции облаков всех классов. Распределение облачности также имеет свои
особенности: например, плотность распределения количества облаков, по
данным наземных наблюдений, чаще всего имеет U-образный вид, тогда как
по спутниковым данным преобладают куполообразные распределения. Вклад
радиационно-термического фактора в формирование облаков, включая кучевые
и кучево-дождевые, не превышает 10-20\%, в то время как основную роль
играют динамические факторы, такие как вертикальные движения
синоптического масштаба, адвекция тепла и влаги, а также турбулентный
обмен.

ewpage

Коллега, при обсуждении генетической классификации облаков важно
понимать, что, хотя единой общепринятой табличной классификации,
охватывающей все формы облаков строго по генетическому признаку, в
источниках не представлено, концепция образования облаков тесно связана
с физическими процессами и синоптическими условиями. По сути,
генетическая классификация подразумевает разделение облаков по
доминирующим механизмам их формирования.

\hypertarget{ux433ux435ux43dux435ux442ux438ux447ux435ux441ux43aux430ux44f-ux43aux43bux430ux441ux441ux438ux444ux438ux43aux430ux446ux438ux44f-ux43eux431ux43bux430ux43aux43eux432}{%
\section{Генетическая Классификация
Облаков}\label{ux433ux435ux43dux435ux442ux438ux447ux435ux441ux43aux430ux44f-ux43aux43bux430ux441ux441ux438ux444ux438ux43aux430ux446ux438ux44f-ux43eux431ux43bux430ux43aux43eux432}}

Образование облаков является многофакторным процессом, определяющимся
сложными взаимодействиями температуры, влажности, ветра и вертикальных
движений воздуха. Определяющую роль в формировании облаков всех классов
играют крупномасштабные вертикальные движения и синоптические вихри,
которые, в свою очередь, тесно связаны с горизонтальными контрастами
температуры (в частности, с геострофической адвекцией виртуальной
температуры). Ниже представлены основные категории облаков,
сгруппированные по преобладающим физико-метеорологическим условиям их
образования.

\hypertarget{i.-ux43eux431ux43bux430ux43aux430-ux43eux431ux440ux430ux437ux43eux432ux430ux43dux43dux44bux435-ux43fux440ux435ux438ux43cux443ux449ux435ux441ux442ux432ux435ux43dux43dux43e-ux43aux440ux443ux43fux43dux43eux43cux430ux441ux448ux442ux430ux431ux43dux44bux43cux438-ux432ux43eux441ux445ux43eux434ux44fux449ux438ux43cux438-ux434ux432ux438ux436ux435ux43dux438ux44fux43cux438}{%
\subsection{I. Облака, Образованные Преимущественно Крупномасштабными
Восходящими
Движениями}\label{i.-ux43eux431ux43bux430ux43aux430-ux43eux431ux440ux430ux437ux43eux432ux430ux43dux43dux44bux435-ux43fux440ux435ux438ux43cux443ux449ux435ux441ux442ux432ux435ux43dux43dux43e-ux43aux440ux443ux43fux43dux43eux43cux430ux441ux448ux442ux430ux431ux43dux44bux43cux438-ux432ux43eux441ux445ux43eux434ux44fux449ux438ux43cux438-ux434ux432ux438ux436ux435ux43dux438ux44fux43cux438}}

Этот класс облаков связан с адиабатическим охлаждением воздуха при его
подъеме в крупномасштабных атмосферных процессах.

\hypertarget{ux430.-ux441ux432ux44fux437ux430ux43dux43dux44bux435-ux441-ux43eux431ux43bux430ux441ux442ux44fux43cux438-ux43fux43eux43dux438ux436ux435ux43dux43dux43eux433ux43e-ux434ux430ux432ux43bux435ux43dux438ux44f-ux446ux438ux43aux43bux43eux43dux44b-ux43bux43eux436ux431ux438ux43dux44b-ux438-ux444ux440ux43eux43dux442ux430ux43bux44cux43dux44bux43cux438-ux437ux43eux43dux430ux43cux438}{%
\subsubsection{А. Связанные с Областями Пониженного Давления (Циклоны,
Ложбины) и Фронтальными
Зонами}\label{ux430.-ux441ux432ux44fux437ux430ux43dux43dux44bux435-ux441-ux43eux431ux43bux430ux441ux442ux44fux43cux438-ux43fux43eux43dux438ux436ux435ux43dux43dux43eux433ux43e-ux434ux430ux432ux43bux435ux43dux438ux44f-ux446ux438ux43aux43bux43eux43dux44b-ux43bux43eux436ux431ux438ux43dux44b-ux438-ux444ux440ux43eux43dux442ux430ux43bux44cux43dux44bux43cux438-ux437ux43eux43dux430ux43cux438}}

В циклонах в целом и особенно в барических ложбинах преобладают
восходящие движения воздуха синоптического масштаба, скорость которых
может достигать 10⁻¹ - 10⁰ см/с, а в очень глубоких циклонах --- до 10¹
см/с. Эти условия способствуют формированию обширных облачных систем.

\begin{enumerate}
\def\labelenumi{\arabic{enumi}.}
\tightlist
\item
  \textbf{Слоистообразные Облака}: В областях пониженного давления
  прежде всего формируется система слоистообразных облаков.

  \begin{itemize}
  \tightlist
  \item
    \textbf{Высоко-слоистые (As), Слоисто-дождевые (Ns),
    Перисто-слоистые (Cs)}: Типичны для теплых фронтов и фронтов
    окклюзии. Они располагаются преимущественно перед приземной линией
    фронта, в клине холодной воздушной массы, что соответствует зоне
    наиболее интенсивных восходящих движений теплого воздуха. Система
    облаков Ns-As-Cs приобретает форму клина, вытянутого вперед на
    теплом фронте и назад на холодном, благодаря вихревому движению в
    вертикальной плоскости, возникающему из-за горизонтальной разности
    температур на фронтах.
  \end{itemize}
\item
  \textbf{Мощные Конвективные Облака}: В областях пониженного давления
  (циклонах и ложбинах) также образуются конвективные облака.

  \begin{itemize}
  \tightlist
  \item
    \textbf{Мощные кучевые (Cu cong) и Кучево-дождевые (Cb)}:
    Формируются в неустойчивых воздушных массах. Часто наблюдаются на
    холодных фронтах и холодных фронтах окклюзии, реже -- на теплых
    фронтах и теплых фронтах окклюзии. До 98,4\% гроз (связанных с Cb)
    приходятся на области пониженного давления. Их образование тесно
    связано с зонами конвергенции скорости ветра (до 80\% случаев),
    особенно мезомасштабными зонами конвергенции, которые возникают под
    влиянием механических неоднородностей и колебаний радиационных
    свойств поверхности.
  \end{itemize}
\end{enumerate}

\hypertarget{ii.-ux43eux431ux43bux430ux43aux430-ux43eux431ux440ux430ux437ux43eux432ux430ux43dux43dux44bux435-ux43fux440ux435ux438ux43cux443ux449ux435ux441ux442ux432ux435ux43dux43dux43e-ux43aux43eux43dux432ux435ux43aux446ux438ux435ux439-ux442ux435ux440ux43cux438ux447ux435ux441ux43aux43eux439-ux438-ux432ux44bux43dux443ux436ux434ux435ux43dux43dux43eux439}{%
\subsection{II. Облака, Образованные Преимущественно Конвекцией
(Термической и
Вынужденной)}\label{ii.-ux43eux431ux43bux430ux43aux430-ux43eux431ux440ux430ux437ux43eux432ux430ux43dux43dux44bux435-ux43fux440ux435ux438ux43cux443ux449ux435ux441ux442ux432ux435ux43dux43dux43e-ux43aux43eux43dux432ux435ux43aux446ux438ux435ux439-ux442ux435ux440ux43cux438ux447ux435ux441ux43aux43eux439-ux438-ux432ux44bux43dux443ux436ux434ux435ux43dux43dux43eux439}}

Этот механизм связан с неустойчивой стратификацией атмосферы, которая
может быть усилена прогревом подстилающей поверхности.

\begin{enumerate}
\def\labelenumi{\arabic{enumi}.}
\tightlist
\item
  \textbf{Кучевые Облака (Cu hum, Cu med, Cu cong, Cb)}: Образуются,
  когда воздушная частица, достигнув уровня конденсации, оказывается в
  условиях неустойчивости.

  \begin{itemize}
  \tightlist
  \item
    \textbf{Слабо развитые кучевые облака (Cu hum, Cu med)}: Могут
    образовываться в антициклонах в результате сухонеустойчивой
    стратификации в приземном слое (γ \textgreater{} γa). Однако и здесь
    существенна роль вертикальных движений синоптического масштаба:
    ослабление или смена знака нисходящих движений может привести к
    развитию кучевых облаков до мощных.
  \item
    \textbf{Влияние термического фактора}: Конвективные облака
    образуются чаще летом и днем, что частично объясняется притоком
    солнечной радиации и обусловленной им сухонеустойчивой
    стратификацией. Однако, синоптические условия играют первостепенную
    роль, так как термическая конвекция не является определяющим
    фактором для повторяемости кучевых облаков (например, летом
    повторяемость кучевых облаков не всегда больше, чем зимой).
  \end{itemize}
\item
  \textbf{Взаимодействие с синоптическими условиями}: Конвективные
  облака могут быть перенесены верхней составляющей скорости ветра в
  области нисходящих движений (например, на периферии циклонов), где они
  начинают рассеиваться и трансформироваться.
\end{enumerate}

\hypertarget{iii.-ux43eux431ux43bux430ux43aux430-ux43eux431ux440ux430ux437ux43eux432ux430ux43dux43dux44bux435-ux43fux440ux435ux438ux43cux443ux449ux435ux441ux442ux432ux435ux43dux43dux43e-ux43eux445ux43bux430ux436ux434ux435ux43dux438ux435ux43c-ux432-ux43fux440ux438ux437ux435ux43cux43dux43eux43c-ux441ux43bux43eux435-ux442ux443ux43cux430ux43dux44b}{%
\subsection{III. Облака, Образованные Преимущественно Охлаждением в
Приземном Слое
(Туманы)}\label{iii.-ux43eux431ux43bux430ux43aux430-ux43eux431ux440ux430ux437ux43eux432ux430ux43dux43dux44bux435-ux43fux440ux435ux438ux43cux443ux449ux435ux441ux442ux432ux435ux43dux43dux43e-ux43eux445ux43bux430ux436ux434ux435ux43dux438ux435ux43c-ux432-ux43fux440ux438ux437ux435ux43cux43dux43eux43c-ux441ux43bux43eux435-ux442ux443ux43cux430ux43dux44b}}

Туманы являются облаками, находящимися непосредственно у земной
поверхности. Их генетическая классификация включает следующие типы:

\hypertarget{ux430.-ux442ux443ux43cux430ux43dux44b-ux43eux445ux43bux430ux436ux434ux435ux43dux438ux44f}{%
\subsubsection{А. Туманы
Охлаждения}\label{ux430.-ux442ux443ux43cux430ux43dux44b-ux43eux445ux43bux430ux436ux434ux435ux43dux438ux44f}}

\begin{enumerate}
\def\labelenumi{\arabic{enumi}.}
\tightlist
\item
  \textbf{Радиационные туманы}: Возникают за счет радиационного
  выхолаживания подстилающей поверхности в ночные часы, особенно в
  центральных частях антициклонов и вдоль осей барических гребней.
  Приземный слой инверсии температуры (до нескольких десятков метров)
  легко разрушается днем, но зимой над материками может сохраняться и
  днем, достигая толщины 1-2 км.

  \begin{itemize}
  \tightlist
  \item
    \textbf{Поземные, низкие и высокие}.
  \end{itemize}
\item
  \textbf{Адвективные туманы}: Образуются при адвекции теплой и влажной
  воздушной массы над относительно холодной подстилающей поверхностью,
  приводящей к ее охлаждению. Наиболее благоприятны для них теплые
  секторы циклонов и прилегающие к ним окраины антициклонов. Скорости
  ветра более 6 м/с обычно неблагоприятны.

  \begin{itemize}
  \tightlist
  \item
    \textbf{Над открытым морем}: Возникают при смещении воздушной массы
    с теплой поверхности моря на холодную, с большим горизонтальным
    градиентом температуры воды вдоль траектории.
  \item
    \textbf{Снижение облаков и перемещение туманной массы}.
  \end{itemize}
\item
  \textbf{Адвективно-радиационные туманы}: Сочетают в себе как
  адвективные, так и радиационные факторы. Часто связаны с дневной
  адвекцией тепла и влаги и последующим ночным прояснением.
\item
  \textbf{Орографические (туманы горных склонов)}: Формируются за счет
  адиабатического охлаждения влажного воздуха, поднимающегося вдоль
  склона горы. Могут также сопровождаться теплообменом с поверхностью
  склона.
\end{enumerate}

\hypertarget{ux432.-ux442ux443ux43cux430ux43dux44b-ux438ux441ux43fux430ux440ux435ux43dux438ux44f}{%
\subsubsection{В. Туманы
Испарения}\label{ux432.-ux442ux443ux43cux430ux43dux44b-ux438ux441ux43fux430ux440ux435ux43dux438ux44f}}

\begin{enumerate}
\def\labelenumi{\arabic{enumi}.}
\tightlist
\item
  \textbf{Туманы испарения (парения) водоемов}: Образуются, когда
  холодный воздух перемещается над относительно теплой водной
  поверхностью, вызывая интенсивное испарение и последующую конденсацию.

  \begin{itemize}
  \tightlist
  \item
    \textbf{Парение арктических морей}.
  \end{itemize}
\end{enumerate}

\hypertarget{iv.-ux43eux431ux43bux430ux43aux430-ux43eux431ux440ux430ux437ux43eux432ux430ux43dux43dux44bux435-ux441ux43cux435ux448ux435ux43dux438ux435ux43c-ux432ux43eux437ux434ux443ux448ux43dux44bux445-ux43cux430ux441ux441}{%
\subsection{IV. Облака, Образованные Смешением Воздушных
Масс}\label{iv.-ux43eux431ux43bux430ux43aux430-ux43eux431ux440ux430ux437ux43eux432ux430ux43dux43dux44bux435-ux441ux43cux435ux448ux435ux43dux438ux435ux43c-ux432ux43eux437ux434ux443ux448ux43dux44bux445-ux43cux430ux441ux441}}

Процессы смешения воздушных масс, естественного или антропогенного
происхождения, также могут приводить к конденсации и образованию
облаков.

\begin{enumerate}
\def\labelenumi{\arabic{enumi}.}
\tightlist
\item
  \textbf{Конденсационные следы (самолетные следы)}: Образуются за
  самолетами (преимущественно реактивными) вследствие смешения выхлопных
  газов с окружающим воздухом при благоприятных условиях. Могут
  разрастаться в достаточно плотные облака верхнего яруса.
\item
  \textbf{В природных условиях}: Роль смешения наиболее значительна во
  фронтальных зонах, при образовании и развитии конвективных облаков и
  тропических циклонов.
\end{enumerate}

\hypertarget{v.-ux442ux440ux430ux43dux441ux444ux43eux440ux43cux430ux446ux438ux44f-ux438-ux44dux432ux43eux43bux44eux446ux438ux44f-ux43eux431ux43bux430ux447ux43dux44bux445-ux441ux438ux441ux442ux435ux43c}{%
\subsection{V. Трансформация и Эволюция Облачных
Систем}\label{v.-ux442ux440ux430ux43dux441ux444ux43eux440ux43cux430ux446ux438ux44f-ux438-ux44dux432ux43eux43bux44eux446ux438ux44f-ux43eux431ux43bux430ux447ux43dux44bux445-ux441ux438ux441ux442ux435ux43c}}

Отдельные формы облаков могут трансформироваться из одного вида в другой
под влиянием изменения синоптических условий. Например, слоисто-кучевые
облака (Sc) могут образоваться при растекании кучево-дождевых (Cb) или
слоисто-дождевых (Ns) облаков. Нисходящие движения на периферии
антициклонов могут приводить к рассеиванию облаков Ns-As и их
трансформации в Ac, Sc, Cu.

Таким образом, генетическая классификация облаков основывается на
доминирующих механизмах охлаждения (адиабатическое расширение,
радиационное выхолаживание, адвекция, смешение) и условиях их протекания
(устойчивость/неустойчивость атмосферы, наличие ядер конденсации,
синоптические барические системы и фронты).

ewpage

Уважаемый коллега, давайте систематизируем информацию о физических
характеристиках облаков, основываясь на имеющихся у нас материалах.

\hypertarget{ux444ux438ux437ux438ux447ux435ux441ux43aux438ux435-ux445ux430ux440ux430ux43aux442ux435ux440ux438ux441ux442ux438ux43aux438-ux43eux431ux43bux430ux43aux43eux432}{%
\section{Физические характеристики
облаков}\label{ux444ux438ux437ux438ux447ux435ux441ux43aux438ux435-ux445ux430ux440ux430ux43aux442ux435ux440ux438ux441ux442ux438ux43aux438-ux43eux431ux43bux430ux43aux43eux432}}

\hypertarget{ux432ux43eux434ux43dux43eux441ux442ux44c-ux438-ux440ux430ux437ux43cux435ux440-ux43aux430ux43fux435ux43bux44c}{%
\subsection{Водность и размер
капель}\label{ux432ux43eux434ux43dux43eux441ux442ux44c-ux438-ux440ux430ux437ux43cux435ux440-ux43aux430ux43fux435ux43bux44c}}

\hypertarget{ux432ux43eux434ux43dux43eux441ux442ux44c-ux441ux43eux434ux435ux440ux436ux430ux43dux438ux435-ux436ux438ux434ux43aux43eux439-ux432ux43eux434ux44b-ux438ux43bux438-ux43bux44cux434ux430}{%
\subsubsection{Водность (содержание жидкой воды или
льда)}\label{ux432ux43eux434ux43dux43eux441ux442ux44c-ux441ux43eux434ux435ux440ux436ux430ux43dux438ux435-ux436ux438ux434ux43aux43eux439-ux432ux43eux434ux44b-ux438ux43bux438-ux43bux44cux434ux430}}

Водность облака, или содержание конденсированной влаги, является
ключевой характеристикой, определяющей его массу и оптические свойства.
Она оказывает существенное влияние на различные виды хозяйственной
деятельности, в частности на авиацию, поскольку связана с обледенением и
дальностью видимости. Профиль водности в облаке имеет характерную
структуру: от нулевого значения на нижней границе облака (обозначаемой
как \texttt{zK}), водность увеличивается с высотой в нижней части
облака, достигая максимального значения (\texttt{q\_m}) на некоторой
высоте (\texttt{z\_m}), а затем убывает с высотой выше \texttt{z\_m} и
вновь становится нулевой на верхней границе облака (\texttt{z\_e}).
Значения \texttt{q\_m} могут существенно варьироваться. Например, для
радиационных туманов и облаков \texttt{S} (водность, г/м³) может
изменяться от 10⁻² - 10⁻¹ г/м³ при низких и умеренных температурах до
10⁻¹ - 10⁰ г/м³ при высоких температурах и разностях температур ΔT=3-10
°С в слое от 3 до 10 км в тропических циклонах. В мощных кучево-дождевых
(грозовых) облаках при разностях температур 10-20 °С и температуре 20-40
°С водность может достигать 5-10 и даже 15-20 г/м³. Результаты
моделирования показывают, что \texttt{q\_m} (г/кг) может быть в
диапазоне 0.38-1.96 г/кг в зависимости от начальной температуры у
поверхности, и 0.02-0.73 г/кг в зависимости от начальной относительной
влажности. Снижение температуры точки росы уменьшает водность облака.

\hypertarget{ux440ux430ux437ux43cux435ux440-ux43aux430ux43fux435ux43bux44c}{%
\subsubsection{Размер
капель}\label{ux440ux430ux437ux43cux435ux440-ux43aux430ux43fux435ux43bux44c}}

Размер взвешенных частиц воды или льда в облаке критически важен для
процессов образования осадков и определения видимости. Скорость падения
капли в воздухе зависит от ее радиуса, что описывается формулой Стокса.
Таблица скоростей падения демонстрирует, что частицы могут иметь радиус
от 0.3 мкм до 5000 мкм (5 мм), а скорость их падения может варьироваться
от 1 мм/ч до 8 м/с. Чем больше вертикальные скорости в облаках, тем
более крупные капли могут в них удерживаться. Характерные размеры ядер
конденсации, облачных и дождевых капель сильно различаются. В частности,
видимость в тумане прямо зависит от размеров взвешенных частиц и их
концентрации (водности тумана). Например, при крупнокапельном дожде
видимость может превышать 4 км, тогда как при слабой мороси она может
ухудшаться до 0.5-1 км.

\hypertarget{ux43aux430ux43fux435ux43bux44cux43dux44bux435-ux43aux440ux438ux441ux442ux430ux43bux43bux438ux447ux435ux441ux43aux438ux435-ux438-ux441ux43cux435ux448ux430ux43dux43dux44bux435-ux43eux431ux43bux430ux43aux430}{%
\subsection{Капельные, кристаллические и смешанные
облака}\label{ux43aux430ux43fux435ux43bux44cux43dux44bux435-ux43aux440ux438ux441ux442ux430ux43bux43bux438ux447ux435ux441ux43aux438ux435-ux438-ux441ux43cux435ux448ux430ux43dux43dux44bux435-ux43eux431ux43bux430ux43aux430}}

Классификация облаков по их фазовому составу напрямую связана с
температурным режимом в атмосфере и динамическими процессами.

\begin{itemize}
\tightlist
\item
  \textbf{Капельные облака:} Состоят преимущественно из мелких капель
  воды. К ним относятся облака нижнего яруса, такие как слоистые (St) и
  слоисто-кучевые (Sc), которые, как правило, состоят из небольших
  капелек воды и имеют ограниченную вертикальную протяженность. В
  тропиках, при очень большой вертикальной протяженности облаков и
  интенсивных вертикальных движениях, осадки могут выпадать даже из
  чисто капельных облаков. Туманы также преимущественно состоят из
  капель воды.
\item
  \textbf{Кристаллические облака:} Состоят из ледяных кристаллов. К ним
  относятся облака верхнего яруса, такие как перистые (Ci),
  перисто-кучевые (Cc) и перисто-слоистые (Cs), чья нижняя граница в
  умеренных широтах находится выше 6 км.
\item
  \textbf{Смешанные облака:} Содержат одновременно все три фазы воды:
  водяной пар, капли воды и ледяные кристаллы. К ним относятся облака
  вертикального развития (кучевые мощные Cu cong и кучево-дождевые Cb) и
  некоторые облака среднего (высоко-кучевые Ac, высоко-слоистые As) и
  нижнего (слоисто-дождевые Ns) ярусов. Для выпадения обложных осадков
  (дождь, снег) из слоисто-дождевых (Ns) и кучево-дождевых (Cb) облаков
  в умеренных и высоких широтах необходимо наличие как капель, так и
  ледяных элементов, так как ледяные кристаллы способствуют
  сублимационному росту и быстрой коагуляции. Туманы также могут быть
  смешанного фазового состава.
\end{itemize}

\hypertarget{ux43dux438ux436ux43dux44fux44f-ux438-ux432ux435ux440ux445ux43dux44fux44f-ux433ux440ux430ux43dux438ux446ux44b-ux43eux431ux43bux430ux43aux43eux432}{%
\subsection{Нижняя и верхняя границы
облаков}\label{ux43dux438ux436ux43dux44fux44f-ux438-ux432ux435ux440ux445ux43dux44fux44f-ux433ux440ux430ux43dux438ux446ux44b-ux43eux431ux43bux430ux43aux43eux432}}

Высота нижней и верхней границ облаков --- важные параметры для
прогнозирования и анализа атмосферных процессов.

\begin{itemize}
\tightlist
\item
  \textbf{Классификация по высоте:}

  \begin{itemize}
  \tightlist
  \item
    \textbf{Верхний ярус:} Нижняя граница выше 6 км (Ci, Cc, Cs).
  \item
    \textbf{Средний ярус:} Нижняя граница 2-6 км (Ac, As).
  \item
    \textbf{Нижний ярус:} Нижняя граница менее 2 км (Ns, St, Sc).
  \item
    \textbf{Вертикальное развитие:} Нижняя граница ниже 2 км, а вершины
    могут достигать границ облаков среднего и верхнего ярусов, иногда
    тропопаузы (Cu, Cb).
  \end{itemize}
\item
  \textbf{Профиль водности и границы:} Водность облака на нижней и
  верхней границах равна нулю. Толщина нижней части облака (от нижней
  границы до уровня максимальной водности \texttt{z\_m}) составляет
  примерно одну треть от общей толщины облака для Ns и Cb, и около двух
  третей для Sc-St и Cu.
\item
  \textbf{Фронтальные системы:} Облачная система теплого фронта,
  включающая Ci-Cs-As-Ns, может простираться на расстояние до 700-900 км
  перед линией фронта, при этом ширина зоны Ns составляет около 300 км,
  а вся система As-Ns имеет ширину 500-600 км. Верхняя граница этих
  облаков As-Ns приблизительно горизонтальна и может простираться до
  тропопаузы, особенно летом.
\item
  \textbf{Прогнозирование:} Прогноз облачности включает ожидаемое
  положение нижней и верхней границ. Численные методы прогноза облаков
  также направлены на определение этих границ. Высота уровня
  конденсации, являющаяся нижней границей облаков, может быть
  предсказана на основе максимальной температуры воздуха и точки росы у
  поверхности земли.
\item
  \textbf{Особые случаи:} Для туманов также выделяют различные высоты
  верхней границы: приземные (\textless2 м), низкие (2-10 м), средние
  (10-100 м) и высокие (\textgreater100 м). Связь между видимостью и
  высотой нижней границы облаков нижнего яруса \texttt{h1} также
  отмечается: чем меньше \texttt{h1}, тем хуже видимость.
\end{itemize}

\hypertarget{ux438ux437ux43cux435ux43dux447ux438ux432ux43eux441ux442ux44c-ux432ux43e-ux432ux440ux435ux43cux435ux43dux438-ux438-ux43fux440ux43eux441ux442ux440ux430ux43dux441ux442ux432ux435}{%
\subsection{Изменчивость во времени и
пространстве}\label{ux438ux437ux43cux435ux43dux447ux438ux432ux43eux441ux442ux44c-ux432ux43e-ux432ux440ux435ux43cux435ux43dux438-ux438-ux43fux440ux43eux441ux442ux440ux430ux43dux441ux442ux432ux435}}

Атмосфера является крайне динамичной средой, и облачность, будучи её
неотъемлемой частью, демонстрирует значительную изменчивость.

\hypertarget{ux438ux437ux43cux435ux43dux447ux438ux432ux43eux441ux442ux44c-ux432ux43e-ux432ux440ux435ux43cux435ux43dux438}{%
\subsubsection{Изменчивость во
времени}\label{ux438ux437ux43cux435ux43dux447ux438ux432ux43eux441ux442ux44c-ux432ux43e-ux432ux440ux435ux43cux435ux43dux438}}

\begin{itemize}
\tightlist
\item
  \textbf{Общая динамика:} Атмосфера -- самая подвижная и изменчивая
  составляющая климатической системы. Изменения погоды, включая
  формирование облаков, происходят в тропосфере.
\item
  \textbf{Сезонные и суточные колебания:} Характеристики облаков,
  включая их средние значения и изменчивость, оказывают определяющее
  влияние на формирование и колебания климата. Сезонные колебания
  повторяемости кучевых облаков максимальны летом и минимальны зимой,
  что частично объясняется термическим фактором и зависимостью
  влажноадиабатического градиента от температуры и давления.
  Слоистообразные облака (Ns) чаще встречаются зимой, чем летом, в
  умеренных и высоких широтах. Суточные колебания температуры воздуха,
  особенно в пограничном слое, оказывают значительное влияние на
  облачность. Например, моросящие и обложные осадки часто усиливаются
  ночью из-за радиационного охлаждения верхней границы облаков, в то
  время как ливневые осадки над сушей преобладают днём и вечером из-за
  интенсивной конвекции.
\item
  \textbf{Синоптические условия:} Синоптические условия играют
  определяющую роль в формировании всех видов облачности, тогда как
  влияние годовых и суточных колебаний радиации не столь существенно,
  как это иногда предполагается. Формирование циклона и облачной системы
  может занимать от нескольких часов до 1-2 суток.
\item
  \textbf{Прогнозирование изменчивости:} Общая схема прогноза облачности
  включает учет перемещения (адвекции), эволюции (трансформации),
  суточного хода и влияния местных факторов.
\end{itemize}

\hypertarget{ux438ux437ux43cux435ux43dux447ux438ux432ux43eux441ux442ux44c-ux432-ux43fux440ux43eux441ux442ux440ux430ux43dux441ux442ux432ux435}{%
\subsubsection{Изменчивость в
пространстве}\label{ux438ux437ux43cux435ux43dux447ux438ux432ux43eux441ux442ux44c-ux432-ux43fux440ux43eux441ux442ux440ux430ux43dux441ux442ux432ux435}}

\begin{itemize}
\tightlist
\item
  \textbf{Неоднородность полей:} Метеорологические величины распределены
  в трехмерном пространстве, формируя скалярные и векторные поля.
\item
  \textbf{Зональные и региональные особенности:} Повторяемость
  облачности имеет выраженную зональную зависимость: два минимума (в
  высоких широтах и субтропиках) и два максимума (в умеренных широтах и
  вблизи экватора) в каждом полушарии. Амплитуда колебаний облачности
  значительно больше над сушей, чем над водой. Повторяемость ясного неба
  значительно выше в Северном (континентальном) полушарии по сравнению с
  Южным (океаническим).
\item
  \textbf{Влияние масштаба наблюдений:} Распределение количества облаков
  может иметь U-образный характер (максимумы на 0-2 и 10 баллов, минимум
  на 4-6 баллов) при оценке по малым квадратам (например, 0.5x0.5°), но
  приближается к куполообразному типу (максимум на 7 баллов) при больших
  размерах квадратов (например, 8x8°, 10x10°). Это указывает на
  зависимость наблюдаемых характеристик от пространственного разрешения.
\item
  \textbf{Местные факторы:} Распределение суши и моря влияет на
  континентальность/океаничность климата, что в свою очередь изменяет
  режим облачности. Топография, экспозиция склонов, альбедо,
  увлажненность подстилающей поверхности, растительность и антропогенные
  факторы создают микроклиматические особенности, влияющие на
  температуру и влажность, и, как следствие, на частоту образования
  туманов и низкой облачности. Близость к водоемам может приводить к
  бризам и усилению осадков за счет конвергенции воздушных потоков.
  Орографические препятствия (горы) также влияют на вертикальные
  движения воздуха, способствуя образованию облаков на наветренных
  склонах и изменяя температурный режим. В крупных городах антропогенные
  примеси и изменение подстилающей поверхности приводят к изменению
  температурно-влажностного режима и, как следствие, условий образования
  туманов, дымок и облаков по сравнению с сельскими районами. Туманы
  могут быть привязаны к специфическим локациям, таким как заболоченные
  низины или наветренные склоны гор.
\end{itemize}

Таким образом, облака представляют собой сложную и динамичную
составляющую атмосферы, чьи характеристики определяются многочисленными
взаимосвязанными физическими процессами, проявляющимися на различных
временных и пространственных масштабах.

ewpage

\hypertarget{ux43eux441ux430ux434ux43aux438-ux43eux431ux449ux438ux435-ux43fux43eux43dux44fux442ux438ux44f-ux438-ux43aux43bux430ux441ux441ux438ux444ux438ux43aux430ux446ux438ux44f}{%
\section{Осадки: Общие Понятия и
Классификация}\label{ux43eux441ux430ux434ux43aux438-ux43eux431ux449ux438ux435-ux43fux43eux43dux44fux442ux438ux44f-ux438-ux43aux43bux430ux441ux441ux438ux444ux438ux43aux430ux446ux438ux44f}}

\hypertarget{ux43eux431ux449ux438ux435-ux43fux43eux43dux44fux442ux438ux44f-ux43eux431-ux43eux441ux430ux434ux43aux430ux445}{%
\subsection{Общие Понятия об
Осадках}\label{ux43eux431ux449ux438ux435-ux43fux43eux43dux44fux442ux438ux44f-ux43eux431-ux43eux441ux430ux434ux43aux430ux445}}

Осадки представляют собой любую форму воды, которая выпадает из облаков
или осаждается на земной поверхности или предметах в результате
конденсации. Эти гидрометеоры оказывают значительное влияние на
экономическую, общественную и культурную деятельность человека, а также
на радиационный, термический и влажностный режимы деятельного слоя почвы
и атмосферы. Информация об осадках критически важна для сельского
хозяйства, строительства, энергетики, здравоохранения и различных видов
транспорта, включая авиацию и морской флот.

\hypertarget{ux444ux43eux440ux43cux438ux440ux43eux432ux430ux43dux438ux435-ux43eux441ux430ux434ux43aux43eux432}{%
\subsection{Формирование
Осадков}\label{ux444ux43eux440ux43cux438ux440ux43eux432ux430ux43dux438ux435-ux43eux441ux430ux434ux43aux43eux432}}

Образование осадков тесно связано с процессами конденсации водяного пара
в атмосфере и последующим ростом гидрометеоров.

\begin{enumerate}
\def\labelenumi{\arabic{enumi}.}
\tightlist
\item
  \textbf{Начальная стадия: Охлаждение и Конденсация.} Воздух в областях
  восходящих потоков охлаждается, что приводит к образованию облаков.
  Для начала конденсации водяного пара в воздухе необходимо наличие
  аэрозольных частиц, известных как ядра конденсации. После достижения
  состояния насыщения (относительная влажность 100\%) на определенном
  уровне, называемом уровнем конденсации (\(z_K\)), дальнейшее понижение
  температуры вызывает конденсацию водяного пара на этих ядрах, формируя
  облака.
\item
  \textbf{Рост элементов облаков.} Рост облачных элементов происходит
  как за счет дальнейшей конденсации (или сублимации) пара, так и за
  счет гравитационной коагуляции. В теплом облаке конденсация происходит
  у нижней границы, а коагуляция --- по мере переноса капель вверх
  восходящими токами. Когда капля достигает достаточно больших размеров,
  она преодолевает сопротивление воздуха и выпадает из облака. Для
  эффективного процесса коагуляции необходимо наличие капель разных
  размеров в облаке. Чем больше значения вертикальных скоростей в
  облаках, тем большие по размеру капли могут в них удерживаться.
\item
  \textbf{Выпадение.} В умеренных и высоких широтах основным условием
  выпадения осадков является наличие в облаке трех фаз: водяного пара,
  капелек и ледяных элементов. Дополнительными условиями являются
  большая вертикальная протяженность облака и высокие скорости
  вертикальных движений внутри облака, которые определяют интенсивность
  выпадающих осадков. В тропиках, благодаря очень большой вертикальной
  протяженности облаков и интенсивным вертикальным движениям, осадки
  могут выпадать и из чисто капельных облаков.
\end{enumerate}

\hypertarget{ux43aux43bux430ux441ux441ux438ux444ux438ux43aux430ux446ux438ux44f-ux43eux441ux430ux434ux43aux43eux432-ux43fux43e-ux43cux435ux445ux430ux43dux438ux437ux43cux443-ux43eux431ux440ux430ux437ux43eux432ux430ux43dux438ux44f}{%
\subsection{Классификация Осадков По Механизму
Образования}\label{ux43aux43bux430ux441ux441ux438ux444ux438ux43aux430ux446ux438ux44f-ux43eux441ux430ux434ux43aux43eux432-ux43fux43e-ux43cux435ux445ux430ux43dux438ux437ux43cux443-ux43eux431ux440ux430ux437ux43eux432ux430ux43dux438ux44f}}

Осадки классифицируются по процессу образования, размерам выпадающих
частиц и длительности выпадения. Основные типы включают моросящие,
обложные и ливневые осадки.

\hypertarget{ux43cux43eux440ux43eux441ux44fux449ux438ux435-ux43eux441ux430ux434ux43aux438}{%
\subsubsection{Моросящие
Осадки}\label{ux43cux43eux440ux43eux441ux44fux449ux438ux435-ux43eux441ux430ux434ux43aux438}}

Моросящие осадки, такие как дождь (морось), переохлажденный дождь
(морось) и снежные зерна (снежная морось), характерны для теплой
воздушной массы, особенно в теплом секторе циклона. Они могут быть
результатом укрупнения частиц тумана или вырождения обложных/ливневых
осадков, выпадая при этом длительное время (несколько часов). Обычно
выпадают из слоистых (St) и слоисто-кучевых (Sc) облаков, но
сравнительно редко.

\hypertarget{ux43eux431ux43bux43eux436ux43dux44bux435-ux43eux441ux430ux434ux43aux438}{%
\subsubsection{Обложные
Осадки}\label{ux43eux431ux43bux43eux436ux43dux44bux435-ux43eux441ux430ux434ux43aux438}}

Обложные осадки (дождь, снег, ледяной дождь, ледяная крупа) типичны для
слоисто-дождевых (Ns) и высоко-слоистых (As) облаков. Они связаны с
фронтальными системами, особенно теплыми фронтами и фронтами окклюзии, и
могут выпадать продолжительное время. Интенсивность обложных осадков
может усиливаться в ночные часы из-за дополнительного радиационного
охлаждения верхней границы облаков. Зона обложных осадков на
синоптических картах обычно закрашивается зеленым цветом.

\hypertarget{ux43bux438ux432ux43dux435ux432ux44bux435-ux43eux441ux430ux434ux43aux438}{%
\subsubsection{Ливневые
Осадки}\label{ux43bux438ux432ux43dux435ux432ux44bux435-ux43eux441ux430ux434ux43aux438}}

Ливневые осадки (дождь, снег, мокрый снег, ледяная крупа, град)
характерны для неустойчивых воздушных масс, холодных фронтов и холодных
фронтов окклюзии. Часто сопровождаются грозами и шквалами. Ливневые
осадки могут быть слабыми, умеренными или сильными (иногда
катастрофическими), но всегда кратковременны и внезапны. Они могут
выпадать повторно через небольшие промежутки времени при надвижении
следующих кучево-дождевых (Cb) облаков. На суше ливневые осадки
преимущественно выпадают в дневные и вечерние часы, когда конвективные
движения воздуха наиболее интенсивны. На Европейской части России за
теплое полугодие доля ливневых осадков в среднем составляет 73\% от их
общего количества.

\hypertarget{ux444ux430ux43aux442ux43eux440ux44b-ux432ux43bux438ux44fux44eux449ux438ux435-ux43dux430-ux43eux431ux440ux430ux437ux43eux432ux430ux43dux438ux435-ux438-ux440ux430ux441ux43fux440ux435ux434ux435ux43bux435ux43dux438ux435-ux43eux441ux430ux434ux43aux43eux432}{%
\subsection{Факторы, Влияющие на Образование и Распределение
Осадков}\label{ux444ux430ux43aux442ux43eux440ux44b-ux432ux43bux438ux44fux44eux449ux438ux435-ux43dux430-ux43eux431ux440ux430ux437ux43eux432ux430ux43dux438ux435-ux438-ux440ux430ux441ux43fux440ux435ux434ux435ux43bux435ux43dux438ux435-ux43eux441ux430ux434ux43aux43eux432}}

\hypertarget{ux432ux435ux440ux442ux438ux43aux430ux43bux44cux43dux44bux435-ux434ux432ux438ux436ux435ux43dux438ux44f-ux438-ux431ux430ux440ux43eux43aux43bux438ux43dux43dux43eux441ux442ux44c}{%
\subsubsection{Вертикальные Движения и
Бароклинность}\label{ux432ux435ux440ux442ux438ux43aux430ux43bux44cux43dux44bux435-ux434ux432ux438ux436ux435ux43dux438ux44f-ux438-ux431ux430ux440ux43eux43aux43bux438ux43dux43dux43eux441ux442ux44c}}

Крупномасштабные вертикальные движения воздуха играют исключительно
важную роль в процессах формирования погоды, включая образование облаков
и осадков. Восходящие движения синоптического масштаба (скоростью
порядка 10⁻¹ - 10⁰ см/с, а в глубоких циклонах до 10¹ см/с) являются
главной причиной образования слоистообразных облаков и пасмурной погоды
в циклонах. Конвергенция воздушных потоков синоптического и
мезомасштабов создает условия для возникновения таких вертикальных
движений, которые, в свою очередь, способствуют увеличению вертикального
градиента температуры во времени, приводя к образованию конвективных
облаков и осадков.

Бароклинность, то есть зависимость плотности воздуха не только от
давления, но и от температуры, существенно влияет на образование и
развитие синоптических вихрей (циклонов и антициклонов), атмосферных
фронтов и, как следствие, на поля облаков и осадков. Адвекция холода, в
частности, играет определяющую роль в углублении циклонов и формировании
осадков. Распределение осадков в тропических циклонах также асимметрично
под влиянием бароклинности, с более значительным количеством и толщиной
облаков в правой части от направления движения циклона.

\hypertarget{ux430ux442ux43cux43eux441ux444ux435ux440ux43dux44bux435-ux444ux440ux43eux43dux442ux44b-ux438-ux446ux438ux43aux43bux43eux43dux44b}{%
\subsubsection{Атмосферные Фронты и
Циклоны}\label{ux430ux442ux43cux43eux441ux444ux435ux440ux43dux44bux435-ux444ux440ux43eux43dux442ux44b-ux438-ux446ux438ux43aux43bux43eux43dux44b}}

Циклоны и атмосферные фронты оказывают определяющее влияние на
формирование полей облаков и осадков.

\begin{itemize}
\tightlist
\item
  \textbf{Циклоны:} В циклонах преобладают восходящие потоки, что
  приводит к охлаждению воздуха и образованию облаков и осадков.
  Типичный циклон на снимках с ИСЗ часто опознается по облачной системе
  в форме запятой.
\item
  \textbf{Теплые фронты:} Характеризуются системой облаков Ci-Cs-As-Ns,
  которая преимущественно располагается перед приземной линией фронта,
  сопровождаясь зоной наиболее интенсивных восходящих движений теплого
  воздуха и обложными осадками.
\item
  \textbf{Холодные фронты:} Чаще всего ассоциируются с мощными
  кучево-дождевыми облаками (Cb) и ливневыми осадками непосредственно
  перед линией фронта. Иногда грозы и шквалы располагаются вдоль линий
  неустойчивости.
\item
  \textbf{Фронты окклюзии:} Сочетают черты теплого и холодного фронтов.
  Облачные системы распространяются по обе стороны от приземного фронта
  окклюзии, особенно у холодного фронта окклюзии. Для лета более типичны
  холодные фронты окклюзии, вдоль которых нередко наблюдаются грозы.
\item
  \textbf{Антициклоны:} В центральных частях антициклонов преобладает
  малооблачная погода из-за нисходящих движений воздуха. В
  субтропических антициклонах, несмотря на высокую влажность,
  практически полностью отсутствуют осадки из-за пассатных инверсий.
  Однако на окраинах антициклонов могут наблюдаться облака и осадки,
  особенно на северных перифериях зимой.
\end{itemize}

\hypertarget{ux440ux435ux43bux44cux435ux444-ux43cux435ux441ux442ux43dux43eux441ux442ux438-ux43eux440ux43eux433ux440ux430ux444ux438ux44f}{%
\subsubsection{Рельеф Местности
(Орография)}\label{ux440ux435ux43bux44cux435ux444-ux43cux435ux441ux442ux43dux43eux441ux442ux438-ux43eux440ux43eux433ux440ux430ux444ux438ux44f}}

Рельеф местности вносит существенный вклад в формирование осадков.
Орографические туманы, например, образуются за счет адиабатического
охлаждения влажного воздуха, поднимающегося вдоль склона горы. Усиление
восходящих движений воздуха с наветренной стороны гор приводит к
усилению фронтальных осадков и расширению зоны осадков. На подветренной
стороне гор нисходящие движения воздуха и фёновый эффект вызывают
размывание фронта и прекращение осадков.

\hypertarget{ux432ux43bux438ux44fux43dux438ux435-ux432ux43eux432ux43bux435ux447ux435ux43dux438ux44f-ux438-ux441ux43cux435ux448ux435ux43dux438ux44f}{%
\subsubsection{Влияние Вовлечения и
Смешения}\label{ux432ux43bux438ux44fux43dux438ux435-ux432ux43eux432ux43bux435ux447ux435ux43dux438ux44f-ux438-ux441ux43cux435ux448ux435ux43dux438ux44f}}

Процессы вовлечения и последующего смешения воздушных масс облака и
окружающей его среды играют определяющую роль в увеличении размеров
облаков и формировании осадков.

\hypertarget{ux432ux43bux438ux44fux43dux438ux435-ux433ux43eux440ux43eux434ux430}{%
\subsubsection{Влияние
Города}\label{ux432ux43bux438ux44fux43dux438ux435-ux433ux43eux440ux43eux434ux430}}

Большие города могут оказывать влияние на поле осадков. Например, в
промышленных центрах существенно изменяются условия формирования дымок,
туманов и облаков.

\hypertarget{ux43fux440ux43eux433ux43dux43eux441ux442ux438ux447ux435ux441ux43aux438ux435-ux430ux441ux43fux435ux43aux442ux44b-ux43eux441ux430ux434ux43aux43eux432}{%
\subsection{Прогностические Аспекты
Осадков}\label{ux43fux440ux43eux433ux43dux43eux441ux442ux438ux447ux435ux441ux43aux438ux435-ux430ux441ux43fux435ux43aux442ux44b-ux43eux441ux430ux434ux43aux43eux432}}

Прогноз осадков является одной из сложнейших задач в метеорологии. Он
требует комплексного учета множества факторов:

\begin{itemize}
\tightlist
\item
  \textbf{Необходимая информация:} Для прогноза осадков необходимо иметь
  сведения об ожидаемой форме (роде) и количестве облаков, толщине слоя
  облаков, интенсивности вертикальных движений внутри облака, а также о
  микрофизическом строении облака и его водности.
\item
  \textbf{Схема прогноза:} Общий прогноз осадков включает предсказание
  появления, перемещения и эволюции облаков, дающих осадки (особенно
  фронтальных систем), а также перемещения существующей зоны осадков, ее
  расширения или сужения в зависимости от эволюции облачной системы,
  углубления или заполнения циклона, обострения или размывания фронта.
\item
  \textbf{Радиолокационные данные:} Для прогноза небольшой
  заблаговременности могут быть использованы радиолокационные
  наблюдения, например, для выявления начальной стадии образования
  града.
\end{itemize}

Таким образом, осадки --- это сложный метеорологический феномен, чье
образование и характеристики зависят от множества взаимодействующих
динамических и термодинамических процессов в атмосфере.

ewpage

\hypertarget{ux43fux440ux43eux446ux435ux441ux441ux44b-ux443ux43aux440ux443ux43fux43dux435ux43dux438ux44f-ux43aux430ux43fux435ux43bux44c-ux438-ux43aux440ux438ux441ux442ux430ux43bux43bux43eux432-ux432-ux43eux431ux43bux430ux43aux430ux445-ux441ux43aux43eux440ux43eux441ux442ux44c-ux440ux43eux441ux442ux430-ux438-ux438ux441ux43fux430ux440ux435ux43dux438ux44f-ux447ux430ux441ux442ux438ux446}{%
\section{Процессы Укрупнения Капель и Кристаллов в Облаках, Скорость
Роста и Испарения
Частиц}\label{ux43fux440ux43eux446ux435ux441ux441ux44b-ux443ux43aux440ux443ux43fux43dux435ux43dux438ux44f-ux43aux430ux43fux435ux43bux44c-ux438-ux43aux440ux438ux441ux442ux430ux43bux43bux43eux432-ux432-ux43eux431ux43bux430ux43aux430ux445-ux441ux43aux43eux440ux43eux441ux442ux44c-ux440ux43eux441ux442ux430-ux438-ux438ux441ux43fux430ux440ux435ux43dux438ux44f-ux447ux430ux441ux442ux438ux446}}

Изучение микрофизических процессов облакообразования и выпадения осадков
является фундаментальным в динамической метеорологии, поскольку именно
они определяют формирование и эволюцию облачных систем, а следовательно,
и погодных условий.

\hypertarget{ux43fux440ux43eux446ux435ux441ux441ux44b-ux443ux43aux440ux443ux43fux43dux435ux43dux438ux44f-ux447ux430ux441ux442ux438ux446-ux432-ux43eux431ux43bux430ux43aux430ux445}{%
\subsection{Процессы Укрупнения Частиц в
Облаках}\label{ux43fux440ux43eux446ux435ux441ux441ux44b-ux443ux43aux440ux443ux43fux43dux435ux43dux438ux44f-ux447ux430ux441ux442ux438ux446-ux432-ux43eux431ux43bux430ux43aux430ux445}}

\hypertarget{ux43eux431ux440ux430ux437ux43eux432ux430ux43dux438ux435-ux438-ux440ux43eux441ux442-ux43aux430ux43fux435ux43bux44c}{%
\subsubsection{Образование и Рост
Капель}\label{ux43eux431ux440ux430ux437ux43eux432ux430ux43dux438ux435-ux438-ux440ux43eux441ux442-ux43aux430ux43fux435ux43bux44c}}

Конденсация водяного пара в воздухе требует наличия аэрозолей в виде
твердых частиц, называемых ядрами конденсации. Водяной пар начинает
оседать на гигроскопичных ядрах еще до достижения состояния насыщения
(например, при относительной влажности 60\%), и при дальнейшем
увеличении влажности капли увеличиваются в размерах. Основным процессом,
приводящим к образованию крупных облачных капель и, в конечном итоге,
осадков, является \textbf{коагуляция} -- процесс слияния мелких капель в
более крупные.

Для того чтобы процесс коагуляции привел к выпадению осадков, необходимо
наличие в облаке капель различных размеров. В ``теплом'' облаке (где
температура выше 0°C), конденсация пара происходит у нижней границы
облака, а коагуляция усиливается по мере переноса капель вверх
восходящими токами. Когда капля достигает достаточно больших размеров,
она преодолевает сопротивление воздуха и выпадает из облака.

\hypertarget{ux43eux431ux440ux430ux437ux43eux432ux430ux43dux438ux435-ux438-ux440ux43eux441ux442-ux43aux440ux438ux441ux442ux430ux43bux43bux43eux432-ux43bux435ux434ux44fux43dux430ux44f-ux444ux430ux437ux430}{%
\subsubsection{Образование и Рост Кристаллов (Ледяная
Фаза)}\label{ux43eux431ux440ux430ux437ux43eux432ux430ux43dux438ux435-ux438-ux440ux43eux441ux442-ux43aux440ux438ux441ux442ux430ux43bux43bux43eux432-ux43bux435ux434ux44fux43dux430ux44f-ux444ux430ux437ux430}}

В умеренных и высоких широтах, ключевым условием выпадения осадков
является наличие в облаке трех фаз: водяного пара, капелек воды и
ледяных элементов. Образование кристаллической изморози происходит через
процесс \textbf{сублимации}, то есть прямой переход водяного пара в лед.

Основной механизм формирования осадков из ``холодных'' облаков
(содержащих ледяные кристаллы) называется \textbf{процессом
Бержерона-Финдайзена}. Этот процесс основан на том, что при одинаковой
температуре упругость насыщения водяного пара над ледяной поверхностью
ниже, чем над водной (переохлажденными каплями). Это означает, что при
смешанном фазовом составе облака (переохлажденные капли воды и ледяные
кристаллы) ледяные элементы растут за счет сублимации пара, который
перегоняется с переохлажденных капель на кристаллы. По мере роста
льдинки увеличиваются в размерах, обводняются и превращаются в дождевые
капли или остаются снежинками, становясь слишком большими, чтобы
удерживаться воздушными потоками.

\hypertarget{ux441ux43aux43eux440ux43eux441ux442ux44c-ux440ux43eux441ux442ux430-ux438-ux438ux441ux43fux430ux440ux435ux43dux438ux44f-ux43aux430ux43fux435ux43bux44c}{%
\subsection{Скорость Роста и Испарения
Капель}\label{ux441ux43aux43eux440ux43eux441ux442ux44c-ux440ux43eux441ux442ux430-ux438-ux438ux441ux43fux430ux440ux435ux43dux438ux44f-ux43aux430ux43fux435ux43bux44c}}

Скорость роста и испарения капель зависит от множества взаимосвязанных
термодинамических и динамических факторов.

\hypertarget{ux444ux430ux43aux442ux43eux440ux44b-ux432ux43bux438ux44fux44eux449ux438ux435-ux43dux430-ux440ux43eux441ux442-ux438-ux438ux441ux43fux430ux440ux435ux43dux438ux435}{%
\subsubsection{Факторы, Влияющие на Рост и
Испарение}\label{ux444ux430ux43aux442ux43eux440ux44b-ux432ux43bux438ux44fux44eux449ux438ux435-ux43dux430-ux440ux43eux441ux442-ux438-ux438ux441ux43fux430ux440ux435ux43dux438ux435}}

\begin{enumerate}
\def\labelenumi{\arabic{enumi}.}
\tightlist
\item
  \textbf{Парциальное давление водяного пара (упругость)}: Дефицит
  насыщения (разность между давлением насыщенного водяного пара и
  фактическим парциальным давлением пара) прямо влияет на интенсивность
  испарения. Чем больше дефицит, тем интенсивнее испарение. Увеличение
  содержания водяного пара (рост его парциального давления) способствует
  достижению состояния насыщения и последующему росту капель.
\item
  \textbf{Температура воздуха}: Понижение температуры воздуха
  (уменьшение упругости насыщения) способствует конденсации и росту
  капель. Конденсация водяного пара выделяет скрытую теплоту
  парообразования, которая замедляет понижение температуры
  поднимающегося воздуха, что может усилить его подъем и способствовать
  развитию облака до больших высот.
\item
  \textbf{Вертикальные движения воздуха (w)}:

  \begin{itemize}
  \tightlist
  \item
    \textbf{Восходящие движения}: Приводят к адиабатическому охлаждению
    воздуха, способствуя конденсации, образованию облаков и росту
    капель. Чем больше вертикальные скорости в облаках, тем большие по
    размеру капли могут в них удерживаться.
  \item
    \textbf{Нисходящие движения}: Вызывают адиабатическое нагревание
    воздуха, приводя к рассеиванию облаков и уменьшению влажности. Под
    влиянием нисходящих движений температура воздуха увеличивается, а
    массовая доля и относительная влажность уменьшаются, что
    препятствует образованию облаков и способствует их рассеиванию.
  \end{itemize}
\item
  \textbf{Устойчивость стратификации}:

  \begin{itemize}
  \tightlist
  \item
    \textbf{Влажно-неустойчивая стратификация (γ \textgreater{} γ'а)}:
    Создает условия для самопроизвольного возникновения очень сильных
    вертикальных скоростей (до десятков метров в секунду), что приводит
    к образованию мощных кучевых облаков (Cu cong) и кучево-дождевых
    (Cb), распространяющихся по всей тропосфере. В таких облаках
    интенсивно происходят процессы укрупнения частиц.
  \item
    \textbf{Устойчивая стратификация (γ \textless{} γ'а)}:
    Характеризуется малыми вертикальными скоростями (1-2 см/с) и
    способствует формированию слоистых облаков (St) и слоисто-кучевых
    (Sc), состоящих из небольших капелек воды, где процессы слияния
    происходят медленно.
  \end{itemize}
\item
  \textbf{Турбулентное перемешивание и вовлечение}:

  \begin{itemize}
  \tightlist
  \item
    \textbf{Вовлечение и смешение}: Процессы вовлечения воздуха из
    окружающей среды в облако и последующего смешения играют
    определяющую роль в увеличении размеров облака и формировании
    осадков. Адвективные и турбулентные притоки тепла и водяного пара
    также важны.
  \item
    \textbf{Турбулентный обмен}: Перенос явного и скрытого тепла от
    подстилающей поверхности в атмосферу через турбулентный обмен играет
    важную роль в изменении метеовеличин и образовании облаков.
    Турбулентность может существенно влиять на тепло- и влагообмен, а
    также на скорость ветра.
  \end{itemize}
\item
  \textbf{Горизонтальная адвекция (перенос)}: Перенос воздушных масс с
  различными характеристиками температуры и влажности существенно влияет
  на формирование облаков и их развитие. Например, адвекция теплого и
  влажного воздуха над холодной поверхностью благоприятствует
  образованию низкой облачности.
\item
  \textbf{Радиационный режим}: Облачность влияет на радиационный баланс
  земной поверхности и атмосферы, что, в свою очередь, изменяет
  температурный и влажностный режим и может влиять на образование и
  рассеивание облаков. Радиационное выхолаживание подстилающей
  поверхности способствует образованию инверсий, под которыми могут
  формироваться облака.
\end{enumerate}

\hypertarget{ux43aux43eux43bux438ux447ux435ux441ux442ux432ux435ux43dux43dux44bux435-ux43eux446ux435ux43dux43aux438-ux441ux43aux43eux440ux43eux441ux442ux435ux439}{%
\subsubsection{Количественные Оценки
Скоростей}\label{ux43aux43eux43bux438ux447ux435ux441ux442ux432ux435ux43dux43dux44bux435-ux43eux446ux435ux43dux43aux438-ux441ux43aux43eux440ux43eux441ux442ux435ux439}}

\begin{itemize}
\tightlist
\item
  \textbf{Рост высоты облаков}: Высота верхней границы облаков в период
  их максимального развития может увеличиваться со скоростью 1-1.5 м/с.
\item
  \textbf{Горизонтальное увеличение размеров облака}: Скорость
  увеличения радиуса облака в среднем для всех исследованных случаев
  составила 92.6 м/мин, что в 30 раз превышает скорость роста облака по
  вертикали. Продолжительность роста площади радиоэха от облака до
  достижения максимума колебалась между 45 и 85 минутами, а скорость
  нарастания площади - между 0.3 и 2.1 км²/мин.
\item
  \textbf{Время формирования облака}: Для образования циклона и облачной
  системы, видимых на синоптической карте, требуется от нескольких часов
  до 1-2 суток.
\item
  \textbf{Вертикальные скорости в слоистых облаках}: Характеризуются
  малыми скоростями, порядка 1-2 см/с.
\item
  \textbf{Вертикальные скорости в кучево-дождевых облаках}: Могут
  достигать десятков метров в секунду. Скорость синоптического масштаба
  составляет порядка 10⁻¹ - 10⁰ см/с, а в очень глубоких циклонах и
  ложбинах --- до 10¹ см/с. Мезомасштабная скорость может быть на
  один-два порядка выше скорости синоптического масштаба.
\item
  \textbf{Выпадение осадков}: Обложные осадки имеют умеренную
  интенсивность (0.01-0.05 мм/мин). Интенсивность осадков в тропических
  циклонах может колебаться между 1.8 и 14.8 мм/ч.
\end{itemize}

Таким образом, процессы укрупнения капель и кристаллов в облаках
являются результатом сложного взаимодействия множества термодинамических
и динамических факторов, а их скорость варьируется в широких пределах в
зависимости от масштаба облачного образования и преобладающих в нем
микрофизических механизмов. Детальное изучение этих процессов часто
требует применения численных гидродинамических моделей, учитывающих
баланс тепла и влаги, уравнения движения, а также скорости роста капель
воды и кристаллов льда.

ewpage

\hypertarget{ux43aux43eux44dux444ux444ux438ux446ux438ux435ux43dux442-ux437ux430ux445ux432ux430ux442ux430-ux43aux430ux43fux435ux43bux44c}{%
\section{Коэффициент Захвата
Капель}\label{ux43aux43eux44dux444ux444ux438ux446ux438ux435ux43dux442-ux437ux430ux445ux432ux430ux442ux430-ux43aux430ux43fux435ux43bux44c}}

Коэффициент захвата капель (также иногда называемый коэффициентом
соударения) является безразмерной величиной, имеющей критическое
значение при анализе процессов обледенения, в частности, для
прогнозирования отложения льда на поверхностях движущихся объектов,
таких как самолеты.

\hypertarget{ux43eux43fux440ux435ux434ux435ux43bux435ux43dux438ux435-ux438-ux440ux43eux43bux44c-ux432-ux43eux431ux43bux435ux434ux435ux43dux435ux43dux438ux438-ux430ux432ux438ux430ux446ux438ux438}{%
\subsection{Определение и Роль в Обледенении
Авиации}\label{ux43eux43fux440ux435ux434ux435ux43bux435ux43dux438ux435-ux438-ux440ux43eux43bux44c-ux432-ux43eux431ux43bux435ux434ux435ux43dux435ux43dux438ux438-ux430ux432ux438ux430ux446ux438ux438}}

В контексте обледенения самолетов, коэффициент захвата \textbf{E}
представляет собой долю переохлажденных капель, которые, находясь в
воздушном потоке, соударяются с поверхностью самолета и, соответственно,
могут привести к образованию льда. Это ключевой параметр в уравнении,
определяющем характеристику \textbf{H} процесса обледенения:
\[ H = f(\Delta T, V, \delta, E) \] где:

\begin{itemize}
\tightlist
\item
  \textbf{\(\Delta T\)} --- переохлаждение капель.
\item
  \textbf{V} --- скорость движения (например, самолета).
\item
  \textbf{\(\delta\)} --- водность тумана или облака.
\item
  \textbf{E} --- безразмерный коэффициент захвата капель.
\end{itemize}

\hypertarget{ux432ux43bux438ux44fux43dux438ux435-ux43dux430-ux445ux430ux440ux430ux43aux442ux435ux440ux438ux441ux442ux438ux43aux438-ux43bux44cux434ux430}{%
\subsection{Влияние на Характеристики
Льда}\label{ux432ux43bux438ux44fux43dux438ux435-ux43dux430-ux445ux430ux440ux430ux43aux442ux435ux440ux438ux441ux442ux438ux43aux438-ux43bux44cux434ux430}}

Значение коэффициента захвата \textbf{E} напрямую влияет на тип и
характеристики образующегося льда.

\begin{itemize}
\tightlist
\item
  При меньших значениях \textbf{H} (что может быть связано с меньшим
  \textbf{E}, если другие параметры неизменны), образуется плотный,
  прозрачный и очень прочный лед. Этот вид обледенения считается
  наиболее опасным, поскольку он менее заметен и обладает высокой
  адгезией.
\item
  При больших значениях \textbf{H} (что может быть связано с большим
  \textbf{E}), образуется матовый, непрозрачный лед, который обладает
  меньшей плотностью и прочностью. Чем выше значение \textbf{H}, тем
  менее плотным и прочным будет образующийся лед.
\end{itemize}

Понимание и прогнозирование этого коэффициента крайне важно для
авиационной метеорологии, так как оно позволяет оценить потенциальный
риск обледенения и его характер, что напрямую влияет на безопасность
полетов и эксплуатационные решения.

ewpage

\hypertarget{ux440ux43eux43bux44c-ux442ux432ux435ux440ux434ux43eux439-ux444ux430ux437ux44b-ux432-ux43eux431ux440ux430ux437ux43eux432ux430ux43dux438ux438-ux43eux441ux430ux434ux43aux43eux432}{%
\section{Роль твердой фазы в образовании
осадков}\label{ux440ux43eux43bux44c-ux442ux432ux435ux440ux434ux43eux439-ux444ux430ux437ux44b-ux432-ux43eux431ux440ux430ux437ux43eux432ux430ux43dux438ux438-ux43eux441ux430ux434ux43aux43eux432}}

В метеорологии роль твердой фазы в процессе образования осадков является
фундаментальной, особенно в средних и высоких широтах, где температуры в
облаках часто опускаются ниже нуля.

\hypertarget{ux43eux431ux449ux438ux435-ux43fux440ux438ux43dux446ux438ux43fux44b-ux43eux431ux440ux430ux437ux43eux432ux430ux43dux438ux44f-ux43eux441ux430ux434ux43aux43eux432}{%
\subsection{Общие принципы образования
осадков}\label{ux43eux431ux449ux438ux435-ux43fux440ux438ux43dux446ux438ux43fux44b-ux43eux431ux440ux430ux437ux43eux432ux430ux43dux438ux44f-ux43eux441ux430ux434ux43aux43eux432}}

Начало конденсации водяного пара в атмосфере всегда требует наличия ядер
конденсации -- мелких гигроскопичных твердых частиц. Конденсация на них
начинается при относительной влажности воздуха, близкой к 100\%. Для
гигроскопичных ядер, таких как частицы поваренной соли (NaCl), насыщение
может быть достигнуто уже при относительной влажности 70-80\%. Важно
отметить, что масса самого ядра конденсации ничтожно мала по сравнению с
массой воды, осажденной на нем в процессе конденсации. При этом, хотя
ядра конденсации необходимы для образования капель, они не могут служить
причиной различий в количестве туманов, облаков или осадков в разных
регионах Земли.

\hypertarget{ux43fux440ux43eux446ux435ux441ux441-ux431ux435ux440ux436ux435ux440ux43eux43dux430-ux444ux438ux43dux434ux430ux439ux437ux435ux43dux430-ux43bux435ux434ux44fux43dux43eux439-ux43cux435ux445ux430ux43dux438ux437ux43c}{%
\subsection{Процесс Бержерона-Финдайзена (Ледяной
механизм)}\label{ux43fux440ux43eux446ux435ux441ux441-ux431ux435ux440ux436ux435ux440ux43eux43dux430-ux444ux438ux43dux434ux430ux439ux437ux435ux43dux430-ux43bux435ux434ux44fux43dux43eux439-ux43cux435ux445ux430ux43dux438ux437ux43c}}

Основным процессом, формирующим осадки на Земле, особенно из
``холодных'' облаков (с температурами ниже 0°С), считается процесс
Бержерона-Финдайзена. Этот механизм основывается на том факте, что
давление насыщенного водяного пара над поверхностью льда ниже, чем над
переохлажденной водой при той же отрицательной температуре. Это приводит
к следующему:

\begin{itemize}
\tightlist
\item
  В смешанных облаках, содержащих как переохлажденные капли воды, так и
  ледяные кристаллы, парциальное давление водяного пара становится
  пересыщенным относительно льда, но ненасыщенным или лишь слабо
  пересыщенным относительно воды.
\item
  Молекулы водяного пара активно сублимируют на поверхность ледяных
  кристаллов, в результате чего кристаллы растут за счет испарения
  переохлажденных капель.
\item
  По мере роста ледяные кристаллы сталкиваются с переохлажденными
  каплями, примерзая к ним и увеличивая свою массу, что ускоряет их
  падение.
\item
  Когда эти укрупнившиеся ледяные частицы (снежинки) достигают
  достаточно больших размеров, они преодолевают сопротивление воздуха и
  выпадают из облака.
\end{itemize}

Если снежинки проходят через слой воздуха с положительными
температурами, они тают и выпадают в виде дождя. Если они не успевают
растаять, они достигают поверхности земли в виде снега. В случае очень
высокой скорости падения снежинки могут не успеть растаять даже в слое
положительных температур, что приводит к выпадению града. В умеренных и
высоких широтах наличие в облаке трех фаз -- водяного пара, капелек и
ледяных элементов -- является основным условием выпадения осадков.

\hypertarget{ux43eux431ux440ux430ux437ux43eux432ux430ux43dux438ux435-ux440ux430ux437ux43bux438ux447ux43dux44bux445-ux432ux438ux434ux43eux432-ux442ux432ux435ux440ux434ux44bux445-ux43eux441ux430ux434ux43aux43eux432}{%
\subsection{Образование различных видов твердых
осадков}\label{ux43eux431ux440ux430ux437ux43eux432ux430ux43dux438ux435-ux440ux430ux437ux43bux438ux447ux43dux44bux445-ux432ux438ux434ux43eux432-ux442ux432ux435ux440ux434ux44bux445-ux43eux441ux430ux434ux43aux43eux432}}

\begin{itemize}
\tightlist
\item
  \textbf{Снег:} Выпадает из облаков Ns или As в виде обычных снежинок и
  снежных хлопьев, как правило, при температурах ниже 0°С по всему
  столбу воздуха.
\item
  \textbf{Град:} Образуется в мощных кучево-дождевых облаках (Cb) при
  наличии сильных восходящих движений и низких температур в верхней
  части облака (ниже -20°С). Градины представляют собой крупные частицы
  льда, которые формируются путем намерзания переохлажденных капель на
  ледяные ядра в результате многократных подъемов и опусканий внутри
  облака.
\item
  \textbf{Снежная крупа:} Вид ливневых осадков, состоящих из мелких,
  обычно непрозрачных ледяных частиц.
\item
  \textbf{Гололед (Freezing Rain):} Отложение льда на поверхностях
  предметов, вызванное осаждением и замерзанием переохлажденного дождя,
  мороси или тумана при отрицательной температуре у поверхности земли.
  Это происходит, когда жидкие осадки проходят через слой воздуха с
  температурой выше 0°C, а затем попадают в приземный слой с
  температурой ниже 0°C, где они переохлаждаются и замерзают при
  контакте с поверхностью.
\item
  \textbf{Изморозь (Hoar Frost):} Отложения льда на ветвях деревьев,
  проводах и травинках. Различают кристаллическую изморозь, образующуюся
  в основном в малоградиентных барических полях при наличии приземной
  инверсии температуры, и зернистую изморозь, условия образования
  которой схожи с условиями для внутримассового гололеда. Толщина
  отложений зависит от размера капель тумана: если капли менее 20 мкм,
  образуется изморозь; если более -- гололед.
\item
  \textbf{Гололедица (Glaze Ice):} Лед на поверхности земли,
  образующийся после оттепели или дождя в результате внезапного
  похолодания, а также вследствие замерзания мокрого снега или капель
  дождя и мороси при соприкосновении с сильно охлажденной поверхностью.
\item
  \textbf{Ледяная корка (Наст):} Слой льда на поверхности почвы или
  снежного покрова, который может образовываться в результате замерзания
  талой воды или мокрого снега.
\end{itemize}

\hypertarget{ux434ux438ux43dux430ux43cux438ux447ux435ux441ux43aux438ux435-ux444ux430ux43aux442ux43eux440ux44b-ux432ux43bux438ux44fux44eux449ux438ux435-ux43dux430-ux442ux432ux435ux440ux434ux44bux435-ux43eux441ux430ux434ux43aux438}{%
\subsection{Динамические факторы, влияющие на твердые
осадки}\label{ux434ux438ux43dux430ux43cux438ux447ux435ux441ux43aux438ux435-ux444ux430ux43aux442ux43eux440ux44b-ux432ux43bux438ux44fux44eux449ux438ux435-ux43dux430-ux442ux432ux435ux440ux434ux44bux435-ux43eux441ux430ux434ux43aux438}}

Помимо микрофизических процессов, динамика атмосферы играет решающую
роль в образовании твердых осадков:

\begin{itemize}
\tightlist
\item
  \textbf{Вертикальные движения:} Восходящие движения воздуха (w
  \textgreater{} 0) приводят к его охлаждению, достижению состояния
  насыщения, и последующему образованию облаков. Для мощных
  кучево-дождевых облаков, из которых выпадают ливневые осадки и град,
  характерны очень сильные (до десятков метров в секунду) вертикальные
  скорости. Однако, большая часть осадков выпадает преимущественно во
  второй стадии развития облака, когда происходит смена восходящего
  движения на нисходящее (w \textless{} 0), позволяя скопившейся
  водности выпасть на поверхность.
\item
  \textbf{Турбулентный обмен:} Перенос тепла и влаги в атмосфере,
  особенно в приземном слое, существенно влияет на формирование облаков
  и, как следствие, осадков.
\item
  \textbf{Адвекция:} Перемещение воздушных масс с разными термическими и
  влажностными свойствами (например, адвекция холода на теплую
  подстилающую поверхность) приводит к возникновению циклонических
  вихрей и восходящих движений, способствующих образованию
  кучево-дождевых облаков и ливневых осадков.
\end{itemize}

\hypertarget{ux43fux440ux43eux433ux43dux43eux437-ux438-ux445ux430ux440ux430ux43aux442ux435ux440ux438ux441ux442ux438ux43aux430-ux442ux432ux435ux440ux434ux44bux445-ux43eux441ux430ux434ux43aux43eux432}{%
\subsection{Прогноз и характеристика твердых
осадков}\label{ux43fux440ux43eux433ux43dux43eux437-ux438-ux445ux430ux440ux430ux43aux442ux435ux440ux438ux441ux442ux438ux43aux430-ux442ux432ux435ux440ux434ux44bux445-ux43eux441ux430ux434ux43aux43eux432}}

Прогноз осадков, включая твердые, тесно связан с прогнозом облачности.
Для успешного прогноза необходимы сведения об ожидаемой форме и
количестве облаков, толщине облачного слоя, интенсивности вертикальных
движений внутри облака, а также о микрофизическом строении облака и его
водности. Вид осадков предсказывается с учетом ожидаемой температуры у
поверхности земли и высоты изотермы 0°С.

Ливневые осадки, в том числе град, являются локальными и прерывистыми
явлениями, что делает их точный прогноз крайне сложным. Над сушей
ливневые осадки выпадают преимущественно в дневные и вечерние часы,
когда конвективные движения воздуха наиболее интенсивны.

ewpage

\hypertarget{ux444ux43eux440ux43cux438ux440ux43eux432ux430ux43dux438ux435-ux43eux441ux430ux434ux43aux43eux432-ux438ux437-ux440ux430ux437ux43bux438ux447ux43dux44bux445-ux442ux438ux43fux43eux432-ux43eux431ux43bux430ux43aux43eux432}{%
\section{Формирование Осадков из Различных Типов
Облаков}\label{ux444ux43eux440ux43cux438ux440ux43eux432ux430ux43dux438ux435-ux43eux441ux430ux434ux43aux43eux432-ux438ux437-ux440ux430ux437ux43bux438ux447ux43dux44bux445-ux442ux438ux43fux43eux432-ux43eux431ux43bux430ux43aux43eux432}}

\hypertarget{ux43eux431ux449ux438ux435-ux43fux440ux438ux43dux446ux438ux43fux44b-ux444ux43eux440ux43cux438ux440ux43eux432ux430ux43dux438ux44f-ux43eux441ux430ux434ux43aux43eux432}{%
\subsection{Общие Принципы Формирования
Осадков}\label{ux43eux431ux449ux438ux435-ux43fux440ux438ux43dux446ux438ux43fux44b-ux444ux43eux440ux43cux438ux440ux43eux432ux430ux43dux438ux44f-ux43eux441ux430ux434ux43aux43eux432}}

Осадки являются одной из важнейших характеристик погоды, непосредственно
связанных с вертикальными движениями воздуха и процессами
облакообразования. В областях восходящих потоков, где воздух охлаждается
адиабатически, формируются облака, тогда как в зонах нисходящих движений
воздух нагревается, что препятствует образованию облачности и
способствует их рассеиванию. Интенсивность и объем осадков в
значительной степени определяются динамическими факторами атмосферы.
Крупномасштабные вертикальные движения (со скоростью порядка 10⁻¹ - 10⁰
см/с) способствуют формированию слоистообразных облаков, тогда как
мезомасштабные движения (со скоростью до 10¹ м/с), характеризующиеся
чередующимися восходящими и нисходящими потоками, приводят к развитию
кучевообразных облаков и ливневых осадков.

\hypertarget{ux43eux441ux430ux434ux43aux438-ux438ux437-ux43aux430ux43fux435ux43bux44cux43dux44bux445-ux442ux435ux43fux43bux44bux445-ux43eux431ux43bux430ux43aux43eux432}{%
\subsection{Осадки из Капельных (Теплых)
Облаков}\label{ux43eux441ux430ux434ux43aux438-ux438ux437-ux43aux430ux43fux435ux43bux44cux43dux44bux445-ux442ux435ux43fux43bux44bux445-ux43eux431ux43bux430ux43aux43eux432}}

\hypertarget{ux43cux435ux445ux430ux43dux438ux437ux43c-ux43aux43eux430ux433ux443ux43bux44fux446ux438ux438-ux43aux430ux43fux435ux43bux44c}{%
\subsubsection{Механизм Коагуляции
Капель}\label{ux43cux435ux445ux430ux43dux438ux437ux43c-ux43aux43eux430ux433ux443ux43bux44fux446ux438ux438-ux43aux430ux43fux435ux43bux44c}}

Теплые облака --- это те, чья верхняя граница располагается около
изотермы 0 °С и которые состоят исключительно из жидких капель воды.
Рост облачных капель до размеров дождевых происходит двумя основными
путями: путем конденсации водяного пара на ядрах конденсации и, что
более существенно для образования осадков, путем коагуляции. Для
эффективной коагуляции необходимо наличие в облаке капель различных
размеров. Более крупные капли, падая с большей скоростью, захватывают и
поглощают более мелкие, увеличиваясь в массе.

\hypertarget{ux443ux441ux43bux43eux432ux438ux44f-ux432ux44bux43fux430ux434ux435ux43dux438ux44f-ux43eux441ux430ux434ux43aux43eux432}{%
\subsubsection{Условия Выпадения
Осадков}\label{ux443ux441ux43bux43eux432ux438ux44f-ux432ux44bux43fux430ux434ux435ux43dux438ux44f-ux43eux441ux430ux434ux43aux43eux432}}

Для того чтобы механизм коагуляции привел к выпадению ощутимых осадков,
требуется соблюдение нескольких условий: облако должно иметь
значительную вертикальную протяженность (не менее 2--3 км) и внутри него
должны наблюдаться сильные и стабильные восходящие потоки. Такие
благоприятные условия наиболее часто встречаются в экваториальной зоне,
где осадки из чисто капельных облаков вносят существенный вклад во
влагооборот.

\hypertarget{ux43eux441ux430ux434ux43aux438-ux438ux437-ux441ux43cux435ux448ux430ux43dux43dux44bux445-ux43eux431ux43bux430ux43aux43eux432}{%
\subsection{Осадки из Смешанных
Облаков}\label{ux43eux441ux430ux434ux43aux438-ux438ux437-ux441ux43cux435ux448ux430ux43dux43dux44bux445-ux43eux431ux43bux430ux43aux43eux432}}

\hypertarget{ux441ux443ux449ux43dux43eux441ux442ux44c-ux43fux440ux43eux446ux435ux441ux441ux430-ux43eux431ux440ux430ux437ux43eux432ux430ux43dux438ux44f-ux43eux441ux430ux434ux43aux43eux432}{%
\subsubsection{Сущность Процесса Образования
Осадков}\label{ux441ux443ux449ux43dux43eux441ux442ux44c-ux43fux440ux43eux446ux435ux441ux441ux430-ux43eux431ux440ux430ux437ux43eux432ux430ux43dux438ux44f-ux43eux441ux430ux434ux43aux43eux432}}

Смешанные облака являются наиболее распространенным источником осадков в
умеренных и высоких широтах. Они характеризуются одновременным
присутствием всех трех фаз воды: водяного пара, переохлажденных капель
воды (при температурах от 0 до -40 °С) и кристаллов льда. Основным
условием выпадения осадков из таких облаков является наличие всех трех
фаз.

Ключевым микрофизическим механизмом в смешанных облаках является эффект
Бержерона-Финдайзена. Он основан на том, что при одной и той же
отрицательной температуре упругость насыщения водяного пара над
поверхностью льда ниже, чем над переохлажденной водой. Это приводит к
тому, что кристаллы льда растут за счет сублимации водяного пара,
который, в свою очередь, испаряется с поверхности переохлажденных капель
воды. Укрупнившиеся ледяные кристаллы, падая, коагулируют с
переохлажденными каплями и другими кристаллами, что обеспечивает быстрый
рост частиц до размеров, способных выпадать в виде осадков. Большая
вертикальная протяженность облака и интенсивные вертикальные движения
внутри него дополнительно способствуют этому процессу. Выделение скрытой
теплоты при конденсации и сублимации водяного пара играет значительную
роль в поддержании конвекции и динамики циклонических систем.

\hypertarget{ux442ux438ux43fux44b-ux43eux431ux43bux430ux43aux43eux432-ux438-ux441ux432ux44fux437ux430ux43dux43dux44bux435-ux43eux441ux430ux434ux43aux438}{%
\subsubsection{Типы Облаков и Связанные
Осадки}\label{ux442ux438ux43fux44b-ux43eux431ux43bux430ux43aux43eux432-ux438-ux441ux432ux44fux437ux430ux43dux43dux44bux435-ux43eux441ux430ux434ux43aux438}}

Различные формы облаков смешанного типа дают различные виды осадков:

\begin{itemize}
\tightlist
\item
  \textbf{Слоисто-дождевые (Ns) и Кучево-дождевые (Cb)}: Эти облака
  обладают значительной вертикальной протяженностью и характеризуются
  интенсивными восходящими движениями. Они состоят как из капельных, так
  и из ледяных элементов. Из них выпадают ливневые осадки (дождь, снег,
  крупа, град), интенсивность которых, как правило, превышает 0,05
  мм/мин. Часто они сопровождаются грозами. Переход обложных осадков в
  ливневые может наблюдаться вблизи линии теплого фронта или фронтов
  окклюзии, особенно при сильной неустойчивости теплой воздушной массы.
  Обратный переход ливневых осадков в обложные может происходить за
  линией холодного фронта первого рода.
\item
  \textbf{Высоко-слоистые (As) и Перисто-слоистые (Cs)}: Эти облака
  часто формируются в системах теплого фронта. Осадки из
  слоисто-дождевых (Ns) и высоко-слоистых (As) облаков обычно носят
  обложной характер. Ледяные кристаллы, выпадающие из верхних слоев
  As-Ns, могут попадать в нижележащие слои слоисто-кучевых (Sc) и
  слоистых (St) облаков, способствуя их укрупнению и расширению зоны
  осадков.
\item
  \textbf{Слоистые (St) и Слоисто-кучевые (Sc)}: Эти облака состоят
  преимущественно из мелких капель воды и имеют небольшую вертикальную
  мощность. Вертикальные движения в них слабы. Коагуляция капель
  протекает медленно и часто компенсируется испарением. Как следствие,
  из этих облаков обычно выпадают лишь моросящие осадки. Их образование
  часто связано с адвекцией теплого и влажного воздуха над относительно
  холодной подстилающей поверхностью или с наличием инверсий
  температуры.
\end{itemize}

\hypertarget{ux43eux441ux430ux434ux43aux438-ux438ux437-ux43aux440ux438ux441ux442ux430ux43bux43bux438ux447ux435ux441ux43aux438ux445-ux43bux435ux434ux44fux43dux44bux445-ux43eux431ux43bux430ux43aux43eux432}{%
\subsection{Осадки из Кристаллических (Ледяных)
Облаков}\label{ux43eux441ux430ux434ux43aux438-ux438ux437-ux43aux440ux438ux441ux442ux430ux43bux43bux438ux447ux435ux441ux43aux438ux445-ux43bux435ux434ux44fux43dux44bux445-ux43eux431ux43bux430ux43aux43eux432}}

Чисто ледяные облака, например, перистые облака верхнего яруса,
полностью расположенные выше изотермы 0°С, как правило, не дают осадков,
способных достичь земной поверхности. Для эффективного роста частиц до
размеров осадков необходимо взаимодействие кристаллов льда с
переохлажденными каплями воды, что характерно для смешанных облаков.

\hypertarget{ux432ux43bux438ux44fux43dux438ux435-ux441ux438ux43dux43eux43fux442ux438ux447ux435ux441ux43aux438ux445-ux444ux430ux43aux442ux43eux440ux43eux432-ux43dux430-ux43eux441ux430ux434ux43aux438}{%
\subsection{Влияние Синоптических Факторов на
Осадки}\label{ux432ux43bux438ux44fux43dux438ux435-ux441ux438ux43dux43eux43fux442ux438ux447ux435ux441ux43aux438ux445-ux444ux430ux43aux442ux43eux440ux43eux432-ux43dux430-ux43eux441ux430ux434ux43aux438}}

\hypertarget{ux446ux438ux43aux43bux43eux43dux44b-ux438-ux444ux440ux43eux43dux442ux44b}{%
\subsubsection{Циклоны и
Фронты}\label{ux446ux438ux43aux43bux43eux43dux44b-ux438-ux444ux440ux43eux43dux442ux44b}}

Области пониженного давления, такие как циклоны и ложбины, а также
связанные с ними атмосферные фронты, играют определяющую роль в
формировании облаков и выпадении осадков.

\begin{itemize}
\tightlist
\item
  В теплом секторе молодого циклона над сушей летом может наблюдаться
  как малооблачная, так и значительная облачность, иногда с грозами и
  развитием кучевых облаков. Зимой в этом же секторе характерны сплошные
  слоистые (St) или слоисто-кучевые (Sc) облака, а также адвективные
  туманы и морось.
\item
  Фронты окклюзии, представляющие собой сомкнувшиеся теплый и холодный
  фронты, также являются важными источниками осадков. Теплые фронты
  окклюзии в холодное время года могут приводить к метелям, тогда как
  холодные фронты окклюзии летом часто сопровождаются грозами.
\end{itemize}

\hypertarget{ux43cux435ux437ux43eux43cux430ux441ux448ux442ux430ux431ux43dux44bux435-ux43fux440ux43eux446ux435ux441ux441ux44b}{%
\subsubsection{Мезомасштабные
Процессы}\label{ux43cux435ux437ux43eux43cux430ux441ux448ux442ux430ux431ux43dux44bux435-ux43fux440ux43eux446ux435ux441ux441ux44b}}

Конвергенция воздушных потоков как синоптического, так и мезомасштаба
создает благоприятные условия для возникновения восходящих вертикальных
движений. Изменение скорости этих движений с высотой способствует
увеличению вертикального градиента температуры, что, в свою очередь,
приводит к формированию конвективных облаков и ливневых осадков в зонах
с влажно-неустойчивой стратификацией. Интенсивные ливневые осадки (более
10-30 мм) требуют большой толщины конвективно-неустойчивого слоя
атмосферы.

\hypertarget{ux442ux440ux43eux43fux438ux447ux435ux441ux43aux438ux435-ux446ux438ux43aux43bux43eux43dux44b}{%
\subsubsection{Тропические
Циклоны}\label{ux442ux440ux43eux43fux438ux447ux435ux441ux43aux438ux435-ux446ux438ux43aux43bux43eux43dux44b}}

Тропические циклоны (тайфуны, ураганы) представляют собой уникальные
природные явления, характеризующиеся экстремальными значениями
интенсивности и общего количества осадков, включая град. Осадки в
тропических циклонах формируются в процессе конденсации водяного пара,
который поступает в атмосферу с поверхности океана. Углубление циклонов
и обильные осадки в них часто обусловлены адвекцией холодного воздуха на
более теплую подстилающую поверхность или взаимодействием воздушных масс
на холодных фронтах. Примером является феномен Эль-Ниньо, когда
углубляющиеся циклоны, формирующиеся при натекании холодного воздуха на
аномально теплую воду в экваториальной части Тихого океана, приводят к
мощной кучево-дождевой облачности и катастрофическим ливневым осадкам.

ewpage

\hypertarget{ux43eux441ux43eux431ux435ux43dux43dux43eux441ux442ux438-ux43eux431ux440ux430ux437ux43eux432ux430ux43dux438ux44f-ux433ux440ux430ux434ux430}{%
\section{Особенности образования
града}\label{ux43eux441ux43eux431ux435ux43dux43dux43eux441ux442ux438-ux43eux431ux440ux430ux437ux43eux432ux430ux43dux438ux44f-ux433ux440ux430ux434ux430}}

Образование града тесно связано с развитием мощной кучево-дождевой
облачности, являющейся необходимым условием для его возникновения. Эти
облака формируются в условиях неустойчивой стратификации атмосферы,
когда поднимающаяся воздушная частица, достигнув уровня конденсации,
оказывается в условиях неустойчивости, что приводит к самопроизвольному
возникновению очень сильных (до десятков метров в секунду) вертикальных
скоростей.

\hypertarget{ux43cux435ux445ux430ux43dux438ux437ux43cux44b-ux440ux43eux441ux442ux430-ux433ux440ux430ux434ux438ux43d}{%
\subsection{Механизмы роста
градин}\label{ux43cux435ux445ux430ux43dux438ux437ux43cux44b-ux440ux43eux441ux442ux430-ux433ux440ux430ux434ux438ux43d}}

Процесс роста градин происходит в двух основных режимах:

\begin{itemize}
\tightlist
\item
  \textbf{Влажный режим}: Часть воды не успевает полностью замерзнуть на
  градине, придавая ей комковатую (крубчатую) структуру. Этот режим
  характерен для наиболее опасного роста градин.
\item
  \textbf{Сухой режим}: Преимущественно образуется снежная крупа и снег.
\end{itemize}

Для выпадения осадков, включая град, необходимо, чтобы часть элементов
облака выросла до таких размеров, при которых скорость их падения
превышает скорость восходящих движений в облаке. Укрупнение частицы
облака радиусом \emph{r} при ее перемещении вдоль оси \emph{z} зависит
от величины \emph{r}.

\hypertarget{ux444ux430ux43aux442ux43eux440ux44b-ux432ux43bux438ux44fux44eux449ux438ux435-ux43dux430-ux43eux431ux440ux430ux437ux43eux432ux430ux43dux438ux435-ux438-ux440ux430ux437ux43cux435ux440-ux433ux440ux430ux434ux430}{%
\subsection{Факторы, влияющие на образование и размер
града}\label{ux444ux430ux43aux442ux43eux440ux44b-ux432ux43bux438ux44fux44eux449ux438ux435-ux43dux430-ux43eux431ux440ux430ux437ux43eux432ux430ux43dux438ux435-ux438-ux440ux430ux437ux43cux435ux440-ux433ux440ux430ux434ux430}}

Размеры выпадающих градин зависят от нескольких ключевых факторов:

\begin{itemize}
\tightlist
\item
  \textbf{Максимальная скорость восходящего потока воздуха (Wmax) внутри
  облака}: Является одним из важнейших параметров.
\item
  \textbf{Температура на том же уровне}: Влияет на режим образования
  крупы (или снега) и града.
\item
  \textbf{Высота изотермы 0°С над поверхностью земли}: Этот фактор важен
  из-за частичного таяния градин при их падении.
\end{itemize}

\hypertarget{ux43fux440ux43eux433ux43dux43eux437ux438ux440ux43eux432ux430ux43dux438ux435-ux433ux440ux430ux434ux430}{%
\subsection{Прогнозирование
града}\label{ux43fux440ux43eux433ux43dux43eux437ux438ux440ux43eux432ux430ux43dux438ux435-ux433ux440ux430ux434ux430}}

Прогноз града имеет большое значение, особенно в районах, где
организованы противоградовые защитные мероприятия. Методика прогноза
града, разработанная Н.И. Глушковой, основывается на следующих шагах:

\begin{enumerate}
\def\labelenumi{\arabic{enumi}.}
\tightlist
\item
  \textbf{Анализ синоптического положения}: Выявляется общая возможность
  выпадения града. Отмечено, что для Северного Кавказа Глушкова выделила
  пять типовых процессов, при которых наиболее вероятно выпадение града.
  Для других регионов также определены благоприятные процессы.
\item
  \textbf{Вычисление Wmax}.
\item
  \textbf{Уточнение возможности образования града}: Производится с
  использованием графиков, которые разграничивают области образования
  крупы (или снега) и града в зависимости от соотношения максимальной
  скорости восходящего потока и температуры на этом уровне.
\item
  \textbf{Определение возможного размера градин}: Осуществляется с
  использованием графиков, связывающих радиус выпадающих градин с Wmax и
  высотой изотермы 0°С.
\end{enumerate}

Град диаметром более 20 мм, а также интенсивный град меньшего диаметра,
причиняющий значительный ущерб народному хозяйству, относится к особо
опасным осадкам.

ewpage

Как профессиональные метеорологи, мы понимаем, что наземная конденсация
и осадки являются ключевыми элементами влагооборота в атмосфере, тесно
связанными с термодинамическими процессами и динамикой воздушных масс.
Давайте рассмотрим эти явления более детально.

\hypertarget{ux43dux430ux437ux435ux43cux43dux430ux44f-ux43aux43eux43dux434ux435ux43dux441ux430ux446ux438ux44f}{%
\section{Наземная
Конденсация}\label{ux43dux430ux437ux435ux43cux43dux430ux44f-ux43aux43eux43dux434ux435ux43dux441ux430ux446ux438ux44f}}

Наземная конденсация включает в себя образование росы, инея, изморози и
туманов, которые формируются непосредственно у подстилающей поверхности.

\hypertarget{ux440ux43eux441ux430-ux438-ux438ux43dux435ux439}{%
\subsection{Роса и
Иней}\label{ux440ux43eux441ux430-ux438-ux438ux43dux435ux439}}

Образование росы и инея происходит, когда температура подстилающей
поверхности понижается ниже температуры точки росы воздуха, находящегося
в контакте с этой поверхностью.

\begin{itemize}
\tightlist
\item
  \textbf{Роса} образуется, когда температура точки росы положительна, и
  водяной пар конденсируется в жидкое состояние на травинках, частицах
  почвы и других предметах.
\item
  \textbf{Иней} формируется, когда температура точки росы отрицательна,
  и водяной пар сублимируется, образуя ледяные кристаллы на
  поверхностях.
\end{itemize}

\hypertarget{ux438ux437ux43cux43eux440ux43eux437ux44c-2}{%
\subsection{Изморозь}\label{ux438ux437ux43cux43eux440ux43eux437ux44c-2}}

\textbf{Изморозь} представляет собой отложения льда, нарастающие на
ветвях деревьев, проводах и других предметах, преимущественно с
наветренной стороны. Различают:

\begin{itemize}
\tightlist
\item
  \textbf{Кристаллическую изморозь}.
\item
  \textbf{Зернистую изморозь}, которая по синоптическим условиям
  образования схожа с внутримассовым гололедом и зависит от размера
  капель тумана: если капли меньше 20 мкм, образуется изморозь; если
  больше 20 мкм, то гололед.
\end{itemize}

\hypertarget{ux442ux443ux43cux430ux43dux44b-1}{%
\subsection{Туманы}\label{ux442ux443ux43cux430ux43dux44b-1}}

\textbf{Туманы} --- это скопления мельчайших капель воды или кристаллов
льда, взвешенных в воздухе непосредственно у земной поверхности,
значительно снижающие горизонтальную видимость. Их образование
определяется притоками водяного пара. В зависимости от основных
физических процессов и синоптических условий туманы классифицируются
следующим образом:

\begin{enumerate}
\def\labelenumi{\arabic{enumi}.}
\tightlist
\item
  \textbf{Туманы охлаждения}:

  \begin{itemize}
  \tightlist
  \item
    \textbf{Радиационные туманы:} Формируются за счет ночного
    радиационного выхолаживания подстилающей поверхности, типичны для
    ясных ночей со слабым ветром, особенно в котловинах. Летом
    радиационный туман обычно рассеивается через 1-2 часа после восхода
    солнца, осенью -- через 3-5 часов.
  \item
    \textbf{Адвективные туманы:} Образуются при перемещении теплой,
    влажной воздушной массы над более холодной подстилающей
    поверхностью. Они могут наблюдаться в любое время суток, усиливаясь
    ночью из-за дополнительного радиационного охлаждения. Скорости ветра
    свыше 6 м/с неблагоприятны для их образования. Прогноз рассеяния
    адвективных туманов связан с прекращением благоприятствующих
    факторов, таких как адвекция тепла.
  \item
    \textbf{Адвективно-радиационные туманы:} Сочетают в себе как
    адвективные, так и радиационные факторы, часто связаны с ночными
    прояснениями.
  \item
    \textbf{Орографические туманы:} Возникают при подъеме воздуха по
    склонам гор.
  \end{itemize}
\item
  \textbf{Туманы испарения}:

  \begin{itemize}
  \tightlist
  \item
    Формируются над водоемами или при арктическом морском парении.
  \end{itemize}
\item
  \textbf{Туманы, связанные с деятельностью человека (антропогенные)}:

  \begin{itemize}
  \tightlist
  \item
    \textbf{Городские туманы:} Учитывают общие факторы туманообразования
    с местными поправками на условия города.
  \item
    \textbf{Морозные (поселковые, печные, аэродромные) туманы:}
    Возникают при сильных морозах в сочетании с дополнительными
    источниками водяного пара (топка печей, работа двигателей).
  \end{itemize}
\end{enumerate}

Для начала конденсации водяного пара в воздухе необходимо наличие ядер
конденсации -- мельчайших взвешенных гигроскопичных частиц. Конденсация
начинается, когда относительная влажность воздуха приближается к 100\%,
однако на гигроскопичных ядрах насыщение может быть достигнуто уже при
70-80\%. Ядра конденсации повсеместны, но их количество не объясняет
различий в частоте туманов в разных регионах. В городах туманов зачастую
меньше, чем в сельской местности, что может быть связано с концентрацией
твердых примесей, которые не способствуют образованию капель.

\hypertarget{ux43eux441ux430ux434ux43aux438}{%
\section{Осадки}\label{ux43eux441ux430ux434ux43aux438}}

Осадки, как и облака, являются определяющим фактором погоды и климата и
формируются под влиянием нескольких факторов.

\hypertarget{ux43eux431ux449ux438ux435-ux443ux441ux43bux43eux432ux438ux44f-ux43eux431ux440ux430ux437ux43eux432ux430ux43dux438ux44f}{%
\subsection{Общие условия
образования}\label{ux43eux431ux449ux438ux435-ux443ux441ux43bux43eux432ux438ux44f-ux43eux431ux440ux430ux437ux43eux432ux430ux43dux438ux44f}}

\begin{itemize}
\tightlist
\item
  \textbf{Охлаждение воздуха:} Пересыщение водяного пара в атмосфере,
  необходимое для конденсации, возникает исключительно за счет
  локального охлаждения воздуха. Величина пересыщения обычно мала и
  редко превышает 1\%. Чем сильнее последующее падение температуры, тем
  больше воды конденсируется.
\item
  \textbf{Ядра конденсации:} Для начала конденсации в воздухе необходимо
  наличие аэрозолей в виде твердых частиц, которые служат ядрами
  конденсации.
\item
  \textbf{Вертикальные движения:} Восходящие движения воздуха охлаждают
  его до точки росы, приводя к конденсации и образованию облаков. В то
  же время нисходящие движения приводят к нагреванию воздуха и, как
  следствие, к отсутствию облаков.
\end{itemize}

\hypertarget{ux43cux435ux445ux430ux43dux438ux437ux43cux44b-ux444ux43eux440ux43cux438ux440ux43eux432ux430ux43dux438ux44f-ux43eux441ux430ux434ux43aux43eux432}{%
\subsection{Механизмы формирования
осадков}\label{ux43cux435ux445ux430ux43dux438ux437ux43cux44b-ux444ux43eux440ux43cux438ux440ux43eux432ux430ux43dux438ux44f-ux43eux441ux430ux434ux43aux43eux432}}

После образования мельчайших капель в облаках для выпадения осадков
необходим их рост до размеров, при которых скорость падения превышает
скорость восходящих движений. Этот рост происходит за счет:

\begin{itemize}
\tightlist
\item
  \textbf{Дальнейшей конденсации (или сублимации):} При этом выделяется
  скрытая теплота конденсации, что замедляет падение температуры
  поднимающегося воздуха и может значительно увеличить скорость его
  подъема.
\item
  \textbf{Гравитационной коагуляции:} Соединение частиц облака при
  неодинаковой скорости их падения в поле тяжести. Этот процесс требует
  наличия в облаке капель разных размеров.
\end{itemize}

\hypertarget{ux442ux438ux43fux44b-ux43eux441ux430ux434ux43aux43eux432}{%
\subsection{Типы
осадков}\label{ux442ux438ux43fux44b-ux43eux441ux430ux434ux43aux43eux432}}

В зависимости от процесса образования, размеров выпадающих частиц и
длительности, осадки делятся на:

\begin{enumerate}
\def\labelenumi{\arabic{enumi}.}
\tightlist
\item
  \textbf{Моросящие осадки:} Мелкие капли дождя (менее 0.5 мм) или
  снежные зерна. Характерны для теплых воздушных масс, могут быть
  результатом укрупнения частиц тумана или вырождения обложных осадков.
  Могут выпадать длительно.
\item
  \textbf{Обложные осадки:} Капли дождя более 0.5 мм или обычные
  снежинки/хлопья. Выпадают из слоисто-дождевых (Ns) или высоко-слоистых
  (As) облаков. Характерны для теплых фронтов и теплых фронтов окклюзии,
  но могут выпадать у фронтов любого типа. Обычно продолжительны,
  непрерывны или с перерывами. Часто усиливаются в ночные часы из-за
  дополнительного радиационного охлаждения верхней границы облаков.
\item
  \textbf{Ливневые осадки:} Обычно крупные капли дождя, хлопья снега,
  снежная крупа или град, выпадающие из кучево-дождевых облаков (Cb).
  Характерны для неустойчивых воздушных масс, холодных фронтов и
  холодных фронтов окклюзии, часто сопровождаются грозами и шквалами.
  Кратковременны и внезапны, но могут повторяться. Летом над сушей
  выпадают преимущественно в дневные и вечерние часы.
\end{enumerate}

\textbf{Переходные формы:}

\begin{itemize}
\tightlist
\item
  Переход обложных осадков в ливневые возможен у теплого фронта (при
  большой неустойчивости), у фронтов окклюзии.
\item
  Переход ливневых осадков в обложные наблюдается за холодными фронтами
  1-го рода или окклюзии.
\end{itemize}

\hypertarget{ux441ux438ux43dux43eux43fux442ux438ux447ux435ux441ux43aux438ux435-ux443ux441ux43bux43eux432ux438ux44f-ux432ux44bux43fux430ux434ux435ux43dux438ux44f-ux43eux441ux430ux434ux43aux43eux432}{%
\subsection{Синоптические условия выпадения
осадков}\label{ux441ux438ux43dux43eux43fux442ux438ux447ux435ux441ux43aux438ux435-ux443ux441ux43bux43eux432ux438ux44f-ux432ux44bux43fux430ux434ux435ux43dux438ux44f-ux43eux441ux430ux434ux43aux43eux432}}

\begin{itemize}
\tightlist
\item
  \textbf{Влагосодержание:} Высокое влагосодержание воздуха у
  поверхности земли и на высотах --- общее благоприятное условие для
  ливневых осадков и гроз.
\item
  \textbf{Вертикальные движения:} Восходящие движения являются основной
  динамической причиной образования облаков и осадков.

  \begin{itemize}
  \tightlist
  \item
    \textbf{Конвергенция:} Сходимость воздушных потоков (как крупного,
    так и мезомасштаба) способствует возникновению восходящих движений и
    увеличению вертикального градиента температуры. При достижении
    насыщения и влажно-неустойчивой стратификации образование осадков
    становится вероятным.
  \item
    \textbf{Тропические циклоны:} Образование осадков в них усиливается
    за счет восходящих движений, высвобождения тепла конденсации, а
    также вовлечения холодного и сухого воздуха окружающей среды с
    последующим смешением.
  \item
    \textbf{Фронты:} Вблизи линий фронтов (как перемещающихся, так и
    квазистационарных) часто наблюдаются восходящие движения, облачность
    и осадки.

    \begin{itemize}
    \tightlist
    \item
      \textbf{Теплые фронты:} Системы облаков Ci-Cs-As-Ns располагаются
      преимущественно перед линией фронта, где происходят интенсивные
      восходящие движения теплого воздуха, приводящие к осадкам.
    \item
      \textbf{Холодные фронты:} Кучево-дождевая облачность характерна
      для холодного фронта, возникая из-за статической неустойчивости и
      сильных конвективных токов перед ним, что приводит к ливневым
      осадкам.
    \item
      \textbf{Фронты окклюзии:} Сочетают черты теплых и холодных
      фронтов. Облачные системы распространяются по обе стороны от линии
      фронта, осадки могут быть как обложными, так и ливневыми.
    \end{itemize}
  \end{itemize}
\item
  \textbf{Термические условия:}

  \begin{itemize}
  \tightlist
  \item
    \textbf{Влажно-неустойчивая стратификация:} Конвективные облака и
    осадки формируются в зонах с влажно-неустойчивой стратификацией под
    влиянием пульсационных вертикальных скоростей, порождаемых силами
    плавучести. Для значительных ливневых осадков требуется толстый
    конвективно-неустойчивый слой и значительное отклонение кривой
    состояния от кривой стратификации.
  \item
    \textbf{Радиационно-термический фактор:} Вклад этого фактора в
    формирование облаков и осадков, включая Cu и Cb, не превышает
    10-20\%; основную роль играют динамические факторы. Тем не менее, он
    может определять сезонные и суточные колебания.
  \end{itemize}
\end{itemize}

\hypertarget{ux432ux43bux438ux44fux43dux438ux435-ux442ux443ux440ux431ux443ux43bux435ux43dux442ux43dux43eux441ux442ux438-ux438-ux430ux434ux432ux435ux43aux446ux438ux438}{%
\subsection{Влияние турбулентности и
адвекции}\label{ux432ux43bux438ux44fux43dux438ux435-ux442ux443ux440ux431ux443ux43bux435ux43dux442ux43dux43eux441ux442ux438-ux438-ux430ux434ux432ux435ux43aux446ux438ux438}}

\begin{itemize}
\tightlist
\item
  \textbf{Турбулентность:} В реальных жидкостях и газах турбулентное
  движение приводит к перемешиванию масс, турбулентной вязкости,
  диффузии и теплопроводности. Оно характеризуется беспорядочными
  колебаниями скорости, давления и плотности (пульсациями). Атмосферные
  движения, как правило, турбулентны. Вклад турбулентного обмена в
  формирование облаков и осадков значителен.
\item
  \textbf{Адвекция:} Перенос воздуха и его свойств в горизонтальном
  направлении.

  \begin{itemize}
  \tightlist
  \item
    \textbf{Адвекция холода:} Способствует циклогенезу и углублению
    циклонов, что связано с образованием мощных облачных систем и
    осадков. Зимой облачность способствует повышению температуры в
    циклоне и сохранению адвекции холода, продлевая существование
    циклонов.
  \item
    \textbf{Адвекция тепла:} В антициклонах адвекция тепла способствует
    их разрушению летом, но зимой сильное радиационное выхолаживание
    помогает сохранять контраст температур и более длительное
    существование вихря.
  \end{itemize}
\end{itemize}

\hypertarget{ux43fux440ux43eux433ux43dux43eux437-ux43eux441ux430ux434ux43aux43eux432}{%
\subsection{Прогноз
осадков}\label{ux43fux440ux43eux433ux43dux43eux437-ux43eux441ux430ux434ux43aux43eux432}}

Общий прогноз осадков базируется на предсказании появления, перемещения
и эволюции облаков, дающих осадки, особенно фронтальных систем.
Надежность прогноза осадков зависит от точности прогноза атмосферного
давления (ветра), температуры воздуха и других метеорологических
величин.

Для прогноза обложных осадков учитывается, что для их выпадения
необходимо наличие трех фаз: водяного пара, жидких капель и ледяных
элементов (в умеренных и высоких широтах). Прогноз конвективной
облачности и ливневых осадков основывается на определении уровня
конденсации, толщины конвективно-неустойчивого слоя и других
эмпирических параметрах.

ewpage

\hypertarget{ux444ux438ux437ux438ux447ux435ux441ux43aux438ux439-ux43cux435ux445ux430ux43dux438ux437ux43c-ux432ux43eux437ux434ux435ux439ux441ux442ux432ux438ux44f-ux43dux430-ux43eux431ux43bux430ux43aux430-ux442ux443ux43cux430ux43dux44b-ux438-ux43eux441ux430ux434ux43aux438}{%
\section{Физический Механизм Воздействия на Облака, Туманы и
Осадки}\label{ux444ux438ux437ux438ux447ux435ux441ux43aux438ux439-ux43cux435ux445ux430ux43dux438ux437ux43c-ux432ux43eux437ux434ux435ux439ux441ux442ux432ux438ux44f-ux43dux430-ux43eux431ux43bux430ux43aux430-ux442ux443ux43cux430ux43dux44b-ux438-ux43eux441ux430ux434ux43aux438}}

Понятие о физическом механизме воздействия на облака, туманы и осадки
неразрывно связано с пониманием естественных процессов их образования и
эволюции. Динамическая метеорология, опираясь на общие законы физики, в
частности гидромеханики и термодинамики, исследует атмосферные процессы,
определяющие погоду и климат, включая образование этих явлений.

\hypertarget{ux43eux431ux449ux438ux435-ux43fux440ux438ux43dux446ux438ux43fux44b-ux43eux431ux440ux430ux437ux43eux432ux430ux43dux438ux44f-ux43eux431ux43bux430ux43aux43eux432-ux442ux443ux43cux430ux43dux43eux432-ux438-ux43eux441ux430ux434ux43aux43eux432}{%
\subsection{Общие Принципы Образования Облаков, Туманов и
Осадков}\label{ux43eux431ux449ux438ux435-ux43fux440ux438ux43dux446ux438ux43fux44b-ux43eux431ux440ux430ux437ux43eux432ux430ux43dux438ux44f-ux43eux431ux43bux430ux43aux43eux432-ux442ux443ux43cux430ux43dux43eux432-ux438-ux43eux441ux430ux434ux43aux43eux432}}

Облака и туманы представляют собой скопления взвешенных в атмосфере
мельчайших капель воды или кристаллов льда. Осадки же являются
результатом роста этих элементов до размеров, при которых они выпадают
из облаков. Для их образования необходимы три основных условия:

\begin{enumerate}
\def\labelenumi{\arabic{enumi}.}
\tightlist
\item
  \textbf{Наличие водяного пара}: Практически весь водяной пар атмосферы
  сосредоточен в тропосфере, поступая туда главным образом через
  испарение с подстилающей поверхности.
\item
  \textbf{Охлаждение воздуха до точки росы (насыщения)}: Это приводит к
  конденсации или сублимации водяного пара.
\item
  \textbf{Наличие ядер конденсации/сублимации}: Мельчайшие аэрозольные
  частицы, такие как пыль, дым, морская соль, служат центрами для
  образования капель или кристаллов.
\end{enumerate}

\hypertarget{ux440ux43eux43bux44c-ux432ux435ux440ux442ux438ux43aux430ux43bux44cux43dux44bux445-ux434ux432ux438ux436ux435ux43dux438ux439-ux440ux430ux437ux43bux438ux447ux43dux43eux433ux43e-ux43cux430ux441ux448ux442ux430ux431ux430-1}{%
\subsection{Роль Вертикальных Движений Различного
Масштаба}\label{ux440ux43eux43bux44c-ux432ux435ux440ux442ux438ux43aux430ux43bux44cux43dux44bux445-ux434ux432ux438ux436ux435ux43dux438ux439-ux440ux430ux437ux43bux438ux447ux43dux43eux433ux43e-ux43cux430ux441ux448ux442ux430ux431ux430-1}}

Вертикальные движения воздуха являются ключевым фактором в процессах
формирования облаков и осадков.

\begin{itemize}
\tightlist
\item
  \textbf{Адиабатическое охлаждение и нагревание}: Подъем воздуха
  сопровождается адиабатическим расширением и охлаждением, что приводит
  к насыщению и конденсации водяного пара. И наоборот, нисходящие
  движения вызывают адиабатическое сжатие и нагревание, что приводит к
  рассеиванию облаков. Сила плавучести Архимеда является основной
  причиной развития мощных вертикальных движений в неустойчивой
  атмосфере.
\item
  \textbf{Масштабы вертикальных движений}:

  \begin{itemize}
  \tightlist
  \item
    \textbf{Синоптический масштаб} (скорости порядка \(10^{-1}\) -
    \(10^0\) см/с): Эти движения обусловливают изменение вертикального
    градиента температуры во времени. Они ответственны за формирование
    крупномасштабных систем облаков, таких как Nimbostratus (Ns),
    Altostratus (As) и Cirrostratus (Cs). В циклонах преобладают
    восходящие движения, способствующие образованию облаков и осадков,
    тогда как в антициклонах -- нисходящие, ведущие к безоблачной погоде
    или рассеиванию облаков.
  \item
    \textbf{Мезомасштабные движения} (скорости порядка \(10^{-1}\) -
    \(10^1\) м/с): Эти движения играют решающую роль в образовании
    кучевообразных (конвективных) облаков, таких как мощные Cumulus (Cu
    cong) и Cumulonimbus (Cb). Эти движения имеют характер струй с
    чередующимися восходящими и нисходящими потоками внутри облака. Они
    часто возникают в мезомасштабных зонах конвергенции воздушных
    потоков.
  \end{itemize}
\end{itemize}

\hypertarget{ux440ux43eux43bux44c-ux442ux443ux440ux431ux443ux43bux435ux43dux442ux43dux43eux433ux43e-ux43fux435ux440ux435ux43cux435ux448ux438ux432ux430ux43dux438ux44f-1}{%
\subsection{Роль Турбулентного
Перемешивания}\label{ux440ux43eux43bux44c-ux442ux443ux440ux431ux443ux43bux435ux43dux442ux43dux43eux433ux43e-ux43fux435ux440ux435ux43cux435ux448ux438ux432ux430ux43dux438ux44f-1}}

Турбулентное перемешивание, характеризующееся беспорядочными колебаниями
скорости, давления и плотности воздуха (пульсациями), играет важную роль
в переносе импульса, тепла, водяного пара, а также капель воды и
кристаллов льда в облаках и туманах.

\begin{itemize}
\tightlist
\item
  \textbf{Диссипация механической энергии}: В турбулентной атмосфере
  происходит диссипация (рассеяние) кинетической энергии движения в
  тепловую, что может влиять на термический режим.
\item
  \textbf{Вовлечение и смешение}: Турбулентное перемешивание вызывает
  вовлечение окружающего, часто ненасыщенного, воздуха в развивающееся
  облако. Это может привести к частичному испарению облачных капель и
  дополнительному понижению температуры поднимающегося воздуха. Однако,
  смешение двух воздушных масс с разными температурами и влажностью
  также может приводить к насыщению и образованию облаков или туманов,
  даже если ни одна из исходных масс не была насыщена. Этот механизм
  особенно значим во фронтальных зонах и при формировании
  конденсационных следов за самолетами.
\item
  \textbf{Вертикальный турбулентный поток тепла}: В приземном слое
  атмосферы, где вертикальная скорость мала, турбулентный обмен является
  основным механизмом переноса тепла, оказывая значительное влияние на
  суточный ход температуры и образование пограничного слоя.
\end{itemize}

\hypertarget{ux440ux43eux43bux44c-ux440ux430ux434ux438ux430ux446ux438ux43eux43dux43dux43eux433ux43e-ux432ux44bux445ux43eux43bux430ux436ux438ux432ux430ux43dux438ux44f-1}{%
\subsection{Роль Радиационного
Выхолаживания}\label{ux440ux43eux43bux44c-ux440ux430ux434ux438ux430ux446ux438ux43eux43dux43dux43eux433ux43e-ux432ux44bux445ux43eux43bux430ux436ux438ux432ux430ux43dux438ux44f-1}}

Лучистый теплообмен в атмосфере, включающий поглощение и излучение
электромагнитных волн водяным паром, углекислым газом и другими
компонентами воздуха, оказывает существенное влияние на температуру
воздуха и, следовательно, на условия образования облаков и туманов.

\begin{itemize}
\tightlist
\item
  \textbf{Охлаждение подстилающей поверхности}: Эффективное излучение
  земной поверхности, особенно в ночные часы и при безоблачном небе,
  приводит к радиационному выхолаживанию приземного слоя воздуха. Это
  является основным механизмом образования радиационных туманов.
\item
  \textbf{Влияние облаков на радиационный баланс}: Облачность уменьшает
  приток солнечной радиации к земной поверхности днем и ее эффективное
  излучение ночью. Это изменяет термический и влажностный режим
  деятельного слоя почвы и атмосферы, влияя на продолжительность
  существования туманов и облаков.
\item
  \textbf{Радиационное выхолаживание верхних слоев}: Радиационное
  охлаждение верхнего слоя воздушной массы может способствовать
  возрастанию неустойчивости и дальнейшему развитию облаков.
\end{itemize}

\hypertarget{ux432ux437ux430ux438ux43cux43eux441ux432ux44fux437ux44c-ux444ux430ux43aux442ux43eux440ux43eux432-ux438-ux434ux43eux43fux43eux43bux43dux438ux442ux435ux43bux44cux43dux44bux435-ux430ux441ux43fux435ux43aux442ux44b}{%
\subsection{Взаимосвязь Факторов и Дополнительные
Аспекты}\label{ux432ux437ux430ux438ux43cux43eux441ux432ux44fux437ux44c-ux444ux430ux43aux442ux43eux440ux43eux432-ux438-ux434ux43eux43fux43eux43bux43dux438ux442ux435ux43bux44cux43dux44bux435-ux430ux441ux43fux435ux43aux442ux44b}}

Все эти процессы не действуют изолированно, а тесно взаимосвязаны и
могут усиливать или ослаблять друг друга.

\begin{itemize}
\tightlist
\item
  \textbf{Бароклинность атмосферы}: В реальной атмосфере, где плотность
  воздуха зависит как от давления, так и от температуры, изопикнические,
  изобарические и изотермические поверхности пересекаются. Это явление,
  называемое бароклинностью, играет определяющую роль в зарождении и
  эволюции синоптических вихрей (циклонов и антициклонов). Эти вихри, в
  свою очередь, обусловливают крупномасштабные вертикальные движения и
  формирование облаков и осадков.
\item
  \textbf{Теплота конденсации}: При образовании облаков выделяется
  скрытая теплота конденсации, которая поступает в атмосферу. Это
  дополнительное тепло существенно влияет на сохранение температурных
  контрастов и усиливает развитие вихрей, особенно в тропических
  циклонах, обеспечивая энергию для их поддержания и углубления.
\item
  \textbf{Антропогенное воздействие}: Деятельность человека, такая как
  выбросы промышленных предприятий и транспорта, изменяет состав
  атмосферы, увеличивая количество водяного пара и аэрозолей (ядер
  конденсации). Это может влиять на температуру, влажность, а также на
  условия образования туманов, дымок и облаков, особенно в крупных
  городах.
\end{itemize}

Таким образом, физический механизм воздействия на облака, туманы и
осадки -- это сложный комплекс взаимодействующих термодинамических и
гидродинамических процессов, где ключевую роль играют вертикальные
движения воздуха, турбулентный обмен и радиационный баланс, а также
бароклинность среды и выделение скрытой теплоты.

ewpage

\hypertarget{ux43fux440ux435ux434ux441ux442ux430ux432ux43bux435ux43dux438ux435-ux43e-ux441ux43fux43eux441ux43eux431ux430ux445-ux430ux43aux442ux438ux432ux43dux43eux433ux43e-ux432ux43eux437ux434ux435ux439ux441ux442ux432ux438ux44f-ux438-ux438ux445-ux44dux444ux444ux435ux43aux442ux438ux432ux43dux43eux441ux442ux438}{%
\section{Представление о Способах Активного Воздействия и их
Эффективности}\label{ux43fux440ux435ux434ux441ux442ux430ux432ux43bux435ux43dux438ux435-ux43e-ux441ux43fux43eux441ux43eux431ux430ux445-ux430ux43aux442ux438ux432ux43dux43eux433ux43e-ux432ux43eux437ux434ux435ux439ux441ux442ux432ux438ux44f-ux438-ux438ux445-ux44dux444ux444ux435ux43aux442ux438ux432ux43dux43eux441ux442ux438}}

В контексте динамической метеорологии, изучение атмосферных процессов,
включая их взаимосвязь с термодинамическими явлениями, охватывает также
\textbf{развитие теории воздействий на погоду и климат}. Это направление
ориентировано на понимание и, потенциально, управление атмосферными
явлениями.

\hypertarget{ux442ux435ux43eux440ux435ux442ux438ux447ux435ux441ux43aux438ux435-ux43eux441ux43dux43eux432ux44b-ux432ux43eux437ux434ux435ux439ux441ux442ux432ux438ux44f}{%
\subsection{Теоретические Основы
Воздействия}\label{ux442ux435ux43eux440ux435ux442ux438ux447ux435ux441ux43aux438ux435-ux43eux441ux43dux43eux432ux44b-ux432ux43eux437ux434ux435ux439ux441ux442ux432ux438ux44f}}

Активное воздействие на атмосферные процессы, в частности на облака и
осадки, основывается на глубоком понимании \textbf{микрофизических
процессов} облакообразования. Ключевые аспекты, являющиеся
потенциальными целями для воздействия, включают:

\begin{itemize}
\tightlist
\item
  \textbf{Фазовое состояние воды в облаках}: В умеренных и высоких
  широтах основным условием выпадения осадков является наличие в облаке
  трех фаз: водяного пара, капелек воды и ледяных элементов. В тропиках,
  при достаточной вертикальной протяженности облаков и интенсивных
  вертикальных движениях, осадки могут выпадать и из чисто капельных
  облаков.
\item
  \textbf{Процессы укрупнения частиц}:

  \begin{itemize}
  \tightlist
  \item
    \textbf{Коагуляция}: Основным процессом, приводящим к образованию
    крупных облачных капель, является коагуляция. Это процесс слияния
    мелких капель в более крупные.
  \item
    \textbf{Сублимация}: Образование изморози, являющейся
    кристаллической формой осадков, происходит через процесс сублимации.
    Этот процесс подразумевает прямой переход водяного пара в лед.
  \end{itemize}
\end{itemize}

Для прогнозирования осадков необходимо учитывать информацию о
микрофизическом строении облака и его водности. Логично предположить,
что целенаправленное изменение этих характеристик является основой для
активных воздействий.

\hypertarget{ux43eux433ux440ux430ux43dux438ux447ux435ux43dux438ux44f-ux438-ux432ux44bux437ux43eux432ux44b-ux432-ux438ux437ux443ux447ux435ux43dux438ux438-ux44dux444ux444ux435ux43aux442ux438ux432ux43dux43eux441ux442ux438}{%
\subsection{Ограничения и Вызовы в Изучении
Эффективности}\label{ux43eux433ux440ux430ux43dux438ux447ux435ux43dux438ux44f-ux438-ux432ux44bux437ux43eux432ux44b-ux432-ux438ux437ux443ux447ux435ux43dux438ux438-ux44dux444ux444ux435ux43aux442ux438ux432ux43dux43eux441ux442ux438}}

Источники подчеркивают существование теории воздействий на погоду и
климат как задачу динамической метеорологии. Однако, представленные
материалы не содержат детализированной информации о конкретных способах
активного воздействия на облака и осадки (например, методах засева
облаков), а также о количественной оценке их эффективности.

Отсутствие в данных источниках подробных сведений об эффективности
воздействий указывает на сложность задачи. Выяснение причин расхождений
между предсказанным и наблюдавшимся состоянием атмосферы, а также
получение статистических характеристик для определения границ
применимости прогностических методов, является фундаментальной проблемой
в метеорологии, что косвенно указывает на трудности в однозначной оценке
успешности любых целенаправленных атмосферных модификаций.

Таким образом, хотя концептуальная основа для изучения активных
воздействий присутствует в рамках динамической метеорологии, детализация
методов и систематизированная оценка их эффективности не представлены в
предложенных источниках.

ewpage

\hypertarget{ux431ux43bux43eux43a-1.-ux444ux438ux437ux438ux43aux430-ux430ux442ux43cux43eux441ux444ux435ux440ux44b-4}{%
\section{Блок 1. «Физика
атмосферы»}\label{ux431ux43bux43eux43a-1.-ux444ux438ux437ux438ux43aux430-ux430ux442ux43cux43eux441ux444ux435ux440ux44b-4}}

\hypertarget{ux442ux435ux43cux430-ux444ux438ux437ux438ux43aux430-ux432ux43eux434ux44b-ux432-ux442ux440ux435ux445-ux444ux430ux437ux43eux432ux44bux445-ux441ux43eux441ux442ux43eux44fux43dux438ux44fux445}{%
\subsection{1.5. Тема «Физика воды в трех фазовых
состояниях»}\label{ux442ux435ux43cux430-ux444ux438ux437ux438ux43aux430-ux432ux43eux434ux44b-ux432-ux442ux440ux435ux445-ux444ux430ux437ux43eux432ux44bux445-ux441ux43eux441ux442ux43eux44fux43dux438ux44fux445}}

\hypertarget{ux443ux441ux43bux43eux432ux438ux44f-ux438-ux434ux438ux430ux433ux440ux430ux43cux43cux430-ux444ux430ux437ux43eux432ux44bux445-ux43fux435ux440ux435ux445ux43eux434ux43eux432-ux432ux43eux434ux44b.-ux440ux43eux43bux44c-ux44fux434ux435ux440-ux43aux43eux43dux434ux435ux43dux441ux430ux446ux438ux438}{%
\subsubsection{\texorpdfstring{\textbf{Условия и диаграмма фазовых
переходов воды. Роль ядер
конденсации}}{Условия и диаграмма фазовых переходов воды. Роль ядер конденсации}}\label{ux443ux441ux43bux43eux432ux438ux44f-ux438-ux434ux438ux430ux433ux440ux430ux43cux43cux430-ux444ux430ux437ux43eux432ux44bux445-ux43fux435ux440ux435ux445ux43eux434ux43eux432-ux432ux43eux434ux44b.-ux440ux43eux43bux44c-ux44fux434ux435ux440-ux43aux43eux43dux434ux435ux43dux441ux430ux446ux438ux438}}

Вода --- единственное вещество в атмосфере, которое существует во всех
трёх фазовых состояниях: газообразном (водяной пар), жидком (капли) и
твёрдом (кристаллы льда). Переходы между этими состояниями (конденсация,
испарение, замерзание, таяние, сублимация, депозиция) играют центральную
роль в термодинамике и динамике атмосферы.

\begin{itemize}
\item
  \textbf{Диаграмма фазовых состояний:} Условия равновесия между фазами
  графически представляются на P-T диаграмме.

  \begin{itemize}
  \item
    \textbf{Кривая испарения/конденсации:} Показывает зависимость
    давления насыщенного пара от температуры над жидкой водой.
    Заканчивается в \textbf{критической точке}.
  \item
    \textbf{Кривая плавления/замерзания:} Показывает зависимость
    температуры замерзания от давления.
  \item
    \textbf{Кривая сублимации/депозиции:} Показывает зависимость
    давления насыщенного пара от температуры надо льдом.
  \item
    \textbf{Тройная точка:} Уникальная точка
    (\(T \approx 0.01^\circ C\), \(p \approx 6.11\) гПа), где все три
    фазы находятся в равновесии. Важно отметить, что при температурах
    ниже 0°C давление насыщенного пара над переохлаждённой водой
    \textbf{выше}, чем надо льдом. Это различие является движущей силой
    \textbf{процесса Бержерона-Финдайзена}.
  \end{itemize}
\item
  \textbf{Роль ядер конденсации:} Образование капель в абсолютно чистом
  воздухе (\textbf{гомогенная нуклеация}) требует огромных пересыщений
  (сотни процентов), которые в реальной атмосфере не достигаются.
  Поэтому конденсация происходит на взвешенных в воздухе аэрозольных
  частицах --- \textbf{ядрах конденсации (CCN)}. Этот процесс называется
  \textbf{гетерогенной нуклеацией}. Ядрами служат частицы морской соли,
  сульфаты, пыль. Их способность притягивать воду (гигроскопичность)
  значительно снижает энергетический барьер для образования капель.
\end{itemize}

\hypertarget{ux43eux431ux440ux430ux437ux43eux432ux430ux43dux438ux435-ux438-ux440ux43eux441ux442-ux437ux430ux440ux43eux434ux44bux448ux435ux432ux44bux445-ux43aux430ux43fux435ux43bux44c.-ux43fux435ux440ux435ux43eux445ux43bux430ux436ux434ux435ux43dux438ux435-ux43aux430ux43fux435ux43bux44c.-ux43eux431ux440ux430ux437ux43eux432ux430ux43dux438ux435-ux43bux435ux434ux44fux43dux44bux445-ux43aux440ux438ux441ux442ux430ux43bux43bux43eux432}{%
\subsubsection{\texorpdfstring{\textbf{Образование и рост зародышевых
капель. Переохлаждение капель. Образование ледяных
кристаллов}}{Образование и рост зародышевых капель. Переохлаждение капель. Образование ледяных кристаллов}}\label{ux43eux431ux440ux430ux437ux43eux432ux430ux43dux438ux435-ux438-ux440ux43eux441ux442-ux437ux430ux440ux43eux434ux44bux448ux435ux432ux44bux445-ux43aux430ux43fux435ux43bux44c.-ux43fux435ux440ux435ux43eux445ux43bux430ux436ux434ux435ux43dux438ux435-ux43aux430ux43fux435ux43bux44c.-ux43eux431ux440ux430ux437ux43eux432ux430ux43dux438ux435-ux43bux435ux434ux44fux43dux44bux445-ux43aux440ux438ux441ux442ux430ux43bux43bux43eux432}}

\begin{itemize}
\item
  \textbf{Рост капель конденсацией:} Рост зародышевой капли на ядре
  конденсации описывается \textbf{теорией Кёлера}. Кривая Кёлера
  показывает равновесное пересыщение над каплей раствора как функцию её
  радиуса и учитывает два противоположных эффекта:

  \begin{enumerate}
  \def\labelenumi{\arabic{enumi}.}
  \item
    \textbf{Эффект раствора (закон Рауля):} Растворённое вещество
    понижает давление насыщенного пара, способствуя росту капли.
  \item
    \textbf{Эффект кривизны (эффект Кельвина):} Давление насыщенного
    пара над выпуклой поверхностью капли выше, чем над плоской, что
    препятствует её росту. Для роста капли необходимо, чтобы пересыщение
    в воздухе превысило критическое значение, определяемое пиком кривой
    Кёлера.
  \end{enumerate}
\item
  \textbf{Переохлаждение и кристаллизация:} Капли чистой воды могут
  оставаться в жидком состоянии при температурах значительно ниже 0°C
  (до -40°C). Это явление называется \textbf{переохлаждением}.
  Замерзание в атмосфере, как и конденсация, происходит на \textbf{ядрах
  замерзания (кристаллизации, IN)}. В качестве IN могут выступать
  частицы глинистых минералов (каолинит), органические частицы.
  Концентрация IN в атмосфере на много порядков меньше, чем CCN, что
  объясняет широкое распространение переохлаждённых облаков.
\end{itemize}

\hypertarget{ux43fux435ux440ux435ux43dux43eux441-ux432ux43eux434ux44fux43dux43eux433ux43e-ux43fux430ux440ux430-ux432-ux442ux443ux440ux431ux443ux43bux435ux43dux442ux43dux43eux439-ux430ux442ux43cux43eux441ux444ux435ux440ux435.-ux438ux441ux43fux430ux440ux435ux43dux438ux435}{%
\subsubsection{\texorpdfstring{\textbf{Перенос водяного пара в
турбулентной атмосфере.
Испарение}}{Перенос водяного пара в турбулентной атмосфере. Испарение}}\label{ux43fux435ux440ux435ux43dux43eux441-ux432ux43eux434ux44fux43dux43eux433ux43e-ux43fux430ux440ux430-ux432-ux442ux443ux440ux431ux443ux43bux435ux43dux442ux43dux43eux439-ux430ux442ux43cux43eux441ux444ux435ux440ux435.-ux438ux441ux43fux430ux440ux435ux43dux438ux435}}

Испарение --- это процесс поступления водяного пара в атмосферу с
подстилающей поверхности. Скорость этого процесса определяется не только
температурой поверхности, но и способностью атмосферы уносить
испарившуюся влагу.

\begin{itemize}
\item
  \textbf{Турбулентный перенос:} Основным механизмом переноса водяного
  пара от поверхности вверх является \textbf{турбулентная диффузия}.
  Поток водяного пара (или \textbf{поток скрытого тепла}, \(LE\))
  описывается по аналогии с другими турбулентными потоками:
  \[LE = -L_v \rho K_q \frac{\partial q}{\partial z}\] где \(L_v\) ---
  скрытая теплота парообразования, \(K_q\) --- коэффициент турбулентной
  диффузии для влаги, \(\frac{\partial q}{\partial z}\) --- вертикальный
  градиент удельной влажности.
\item
  \textbf{Аэродинамическая формула:} Для практических расчётов
  используется формула: \[E = \rho C_E U (q_s - q_a)\] где \(C_E\) ---
  коэффициент влагообмена, \(U\) --- скорость ветра, \((q_s - q_a)\) ---
  разность удельной влажности у поверхности и в воздухе (характеризует
  дефицит насыщения). Формула показывает, что испарение усиливается при
  увеличении скорости ветра и сухости воздуха.
\end{itemize}

\hypertarget{ux442ux443ux43cux430ux43dux44b-ux443ux441ux43bux43eux432ux438ux44f-ux43eux431ux440ux430ux437ux43eux432ux430ux43dux438ux44f-ux43aux43bux430ux441ux441ux438ux444ux438ux43aux430ux446ux438ux44f-ux445ux430ux440ux430ux43aux442ux435ux440ux438ux441ux442ux438ux43aux438-ux438-ux43fux440ux43eux433ux43dux43eux437}{%
\subsubsection{\texorpdfstring{\textbf{Туманы: условия образования,
классификация, характеристики и
прогноз}}{Туманы: условия образования, классификация, характеристики и прогноз}}\label{ux442ux443ux43cux430ux43dux44b-ux443ux441ux43bux43eux432ux438ux44f-ux43eux431ux440ux430ux437ux43eux432ux430ux43dux438ux44f-ux43aux43bux430ux441ux441ux438ux444ux438ux43aux430ux446ux438ux44f-ux445ux430ux440ux430ux43aux442ux435ux440ux438ux441ux442ux438ux43aux438-ux438-ux43fux440ux43eux433ux43dux43eux437}}

\textbf{Туман} --- это облако, основание которого находится у земной
поверхности, снижающее горизонтальную видимость до 1 км и менее.
Образуется либо при охлаждении воздуха до точки росы, либо при его
увлажнении до насыщения.

\begin{itemize}
\item
  \textbf{Классификация по механизму образования:}

  \begin{itemize}
  \item
    \textbf{Радиационные туманы:} Образуются в результате охлаждения
    воздуха от подстилающей поверхности, которая, в свою очередь,
    охлаждается за счёт эффективного длинноволнового излучения.
    Необходимые условия: ясная ночь (для максимальной потери тепла
    поверхностью), высокая относительная влажность вечером и слабый
    ветер (1-3 м/с). Полный штиль препятствует образованию тумана, так
    как охлаждение не распространяется вверх от поверхности; сильный
    ветер разрушает приземную инверсию температуры за счёт турбулентного
    перемешивания. По мере формирования тумана его верхняя граница сама
    становится эффективной излучающей поверхностью, что способствует
    росту тумана вверх. Такие туманы типичны для антициклонов, особенно
    в осенне-зимний период, и часто образуются в низинах и долинах, где
    скапливается холодный воздух.
  \item
    \textbf{Адвективные туманы:} Возникают при адвекции (горизонтальном
    переносе) тёплой, влажной воздушной массы над более холодной
    подстилающей поверхностью. Охлаждение воздуха до точки росы
    происходит контактным путём снизу вверх. Турбулентное перемешивание
    играет ключевую роль, распространяя охлаждение на значительную
    высоту и формируя глубокий и плотный туман. Классические примеры:
    движение тёплого морского воздуха с Гольфстрима над холодными водами
    Лабрадорского течения (туманы Ньюфаундленда) или перенос тёплого
    влажного воздуха с моря на охлаждённую сушу или снежный покров
    зимой. В отличие от радиационных, адвективные туманы могут быть
    очень обширными по площади (сотни тысяч км²), глубокими (до 500 м и
    более) и устойчивыми, существуя несколько суток подряд, так как их
    существование определяется крупномасштабным синоптическим процессом.
  \item
    \textbf{Туманы испарения (морское парение):} Образуются при
    поступлении очень холодной воздушной массы на тёплую водную
    поверхность. Происходят два одновременных процесса: 1) интенсивное
    испарение с поверхности воды, которое резко повышает влагосодержание
    прилегающего воздуха, и 2) быстрое турбулентное перемешивание этого
    тёплого и очень влажного воздуха с холодным сухим воздухом,
    находящимся выше. В результате смешения температура смеси падает, и
    она достигает насыщения. Для их образования необходима большая
    разность температур воды и воздуха (более 10°C). Типичны для
    арктических морей зимой при вторжении континентального воздуха на
    открытую воду, а также над незамерзающими озёрами и реками поздней
    осенью.
  \item
    \textbf{Фронтальные туманы:} Связаны с прохождением атмосферных
    фронтов. Чаще всего это \textbf{предфронтальные туманы},
    образующиеся в зоне тёплого фронта. Дождь, выпадающий из
    вышележащего тёплого воздуха, испаряется, проходя через клин
    холодного воздуха у поверхности. Это приводит к повышению влажности
    в холодном воздухе до 100\% и образованию обширной полосы тумана.
    Такие туманы часто сопровождаются моросящими осадками.
  \item
    \textbf{Склоновые туманы:} Образуются при вынужденном подъёме
    устойчивой влажной воздушной массы по наветренному склону горного
    хребта. Поднимаясь, воздух адиабатически охлаждается. Когда
    температура достигает точки росы, начинается конденсация. Основание
    такого тумана (который по сути является слоистым облаком, касающимся
    склона) находится на высоте уровня конденсации. Эти туманы могут
    быть очень устойчивыми, пока сохраняется восходящий перенос.
  \end{itemize}
\end{itemize}

\hypertarget{ux43eux431ux43bux430ux43aux430-ux443ux441ux43bux43eux432ux438ux44f-ux438-ux43cux435ux445ux430ux43dux438ux437ux43cux44b-ux43eux431ux440ux430ux437ux43eux432ux430ux43dux438ux44f}{%
\subsubsection{\texorpdfstring{\textbf{Облака: условия и механизмы
образования}}{Облака: условия и механизмы образования}}\label{ux43eux431ux43bux430ux43aux430-ux443ux441ux43bux43eux432ux438ux44f-ux438-ux43cux435ux445ux430ux43dux438ux437ux43cux44b-ux43eux431ux440ux430ux437ux43eux432ux430ux43dux438ux44f}}

Основной механизм облакообразования --- \textbf{адиабатическое
охлаждение воздуха при его восходящем движении} до уровня конденсации.
Тип облаков определяется характером вертикальных движений.

\begin{enumerate}
\def\labelenumi{\arabic{enumi}.}
\item
  \textbf{Конвекция:} Неустойчивая стратификация приводит к развитию
  мощных, локальных восходящих потоков, формирующих
  \textbf{кучевообразные облака} (Cumulus, Cumulonimbus).
\item
  \textbf{Крупномасштабные упорядоченные движения:} Медленный и плавный
  подъём воздуха на больших площадях (например, в циклонах и на тёплых
  фронтах) приводит к образованию обширных полей \textbf{слоистообразных
  облаков} (Stratus, Nimbostratus, Altostratus, Cirrostratus).
\item
  \textbf{Волновые движения:} Волны, возникающие на инверсионных слоях
  или при обтекании гор, создают \textbf{волнистообразные облака}
  (Stratocumulus, Altocumulus, Cirrocumulus).
\item
  \textbf{Турбулентное перемешивание:} Интенсивная турбулентность в
  пограничном слое может поднимать влажный воздух до уровня конденсации,
  формируя \textbf{слоисто-кучевые облака}.
\end{enumerate}

\hypertarget{ux43aux43bux430ux441ux441ux438ux444ux438ux43aux430ux446ux438ux44f-ux43eux431ux43bux430ux43aux43eux432-ux43cux435ux436ux434ux443ux43dux430ux440ux43eux434ux43dux430ux44f-ux438-ux433ux435ux43dux435ux442ux438ux447ux435ux441ux43aux430ux44f}{%
\subsubsection{\texorpdfstring{\textbf{Классификация облаков
(международная и
генетическая)}}{Классификация облаков (международная и генетическая)}}\label{ux43aux43bux430ux441ux441ux438ux444ux438ux43aux430ux446ux438ux44f-ux43eux431ux43bux430ux43aux43eux432-ux43cux435ux436ux434ux443ux43dux430ux440ux43eux434ux43dux430ux44f-ux438-ux433ux435ux43dux435ux442ux438ux447ux435ux441ux43aux430ux44f}}

\begin{itemize}
\item
  \textbf{Международная (морфологическая):} Основана на внешнем виде и
  высоте расположения. Выделяют 10 основных родов, которые группируются
  по ярусам:

  \begin{itemize}
  \item
    \textbf{Верхний ярус (выше 6 км):} Перистые (Cirrus),
    перисто-кучевые (Cirrocumulus), перисто-слоистые (Cirrostratus).
    Состоят из ледяных кристаллов.
  \item
    \textbf{Средний ярус (2--6 км):} Высококучевые (Altocumulus),
    высокослоистые (Altostratus). Могут быть капельными,
    кристаллическими или смешанными.
  \item
    \textbf{Нижний ярус (ниже 2 км):} Слоистые (Stratus),
    слоисто-кучевые (Stratocumulus), слоисто-дождевые (Nimbostratus). В
    основном капельные.
  \item
    \textbf{Облака вертикального развития:} Кучевые (Cumulus),
    кучево-дождевые (Cumulonimbus). Простираются через несколько ярусов.
  \end{itemize}
\item
  \textbf{Генетическая:} Основана на механизме образования (см. выше):
  кучевообразные, слоистообразные, волнистообразные.
\end{itemize}

\hypertarget{ux444ux438ux437ux438ux447ux435ux441ux43aux438ux435-ux445ux430ux440ux430ux43aux442ux435ux440ux438ux441ux442ux438ux43aux438-ux43eux431ux43bux430ux43aux43eux432-1}{%
\subsubsection{\texorpdfstring{\textbf{Физические характеристики
облаков}}{Физические характеристики облаков}}\label{ux444ux438ux437ux438ux447ux435ux441ux43aux438ux435-ux445ux430ux440ux430ux43aux442ux435ux440ux438ux441ux442ux438ux43aux438-ux43eux431ux43bux430ux43aux43eux432-1}}

\begin{itemize}
\item
  \textbf{Водность (LWC/IWC):} Масса жидкой воды или льда в единице
  объёма воздуха (г/м³). Варьируется от сотых долей в перистых облаках
  до нескольких г/м³ в мощных кучево-дождевых.
\item
  \textbf{Размер частиц:} Облачные капли имеют радиус \textasciitilde10
  мкм, капли дождя --- \textasciitilde1 мм. Для роста в 100 раз по
  радиусу требуется увеличение массы в 1 миллион раз.
\item
  \textbf{Фазовое состояние:} \textbf{Тёплые} (капельные, T
  \textgreater{} 0°C), \textbf{переохлаждённые} (капельные, T
  \textless{} 0°C), \textbf{смешанные} (капли и кристаллы) и
  \textbf{кристаллические}.
\item
  \textbf{Границы:} Основание облака часто соответствует \textbf{уровню
  конденсации (LCL)}. Вершина --- \textbf{уровню равновесия (EL)} или
  устойчивому слою (инверсии).
\end{itemize}

\hypertarget{ux43eux441ux430ux434ux43aux438-ux43aux43bux430ux441ux441ux438ux444ux438ux43aux430ux446ux438ux44f-ux438-ux43fux440ux43eux446ux435ux441ux441ux44b-ux43eux431ux440ux430ux437ux43eux432ux430ux43dux438ux44f}{%
\subsubsection{\texorpdfstring{\textbf{Осадки: классификация и процессы
образования}}{Осадки: классификация и процессы образования}}\label{ux43eux441ux430ux434ux43aux438-ux43aux43bux430ux441ux441ux438ux444ux438ux43aux430ux446ux438ux44f-ux438-ux43fux440ux43eux446ux435ux441ux441ux44b-ux43eux431ux440ux430ux437ux43eux432ux430ux43dux438ux44f}}

\textbf{Классификация:} Осадки классифицируют по нескольким признакам:

\begin{itemize}
\item
  \textbf{По фазовому состоянию:} Жидкие (дождь, морось), твёрдые (снег,
  снежная и ледяная крупа, ледяные иглы) и смешанные (мокрый снег).
  Ледяной дождь и град представляют собой особые формы.
\item
  \textbf{По характеру выпадения:}

  \begin{itemize}
  \item
    \textbf{Обложные:} Умеренной или слабой интенсивности,
    продолжительные (часы), выпадают из слоисто-дождевых (Ns) и
    высокослоистых (As) облаков на больших площадях. Связаны с
    крупномасштабными упорядоченными восходящими движениями на фронтах и
    в циклонах.
  \item
    \textbf{Ливневые:} Интенсивные, но кратковременные, с резкими
    колебаниями интенсивности. Выпадают из кучево-дождевых (Cb) облаков
    на ограниченной площади. Связаны с конвекцией.
  \item
    \textbf{Моросящие:} Очень мелкие капли (диаметр \textless{} 0.5 мм),
    выпадающие из слоистых (St) или слоисто-кучевых (Sc) облаков.
    Связаны с турбулентными процессами в пограничном слое.
  \end{itemize}
\end{itemize}

\textbf{Процессы образования:} Фундаментальная проблема
осадкообразования --- это необходимость укрупнения облачных частиц в
миллион раз по массе, чтобы превратить облачную каплю (радиус
\textasciitilde10 мкм) в дождевую (радиус \textasciitilde1 мм).
Конденсационный рост слишком медленный и не может объяснить образование
осадков за время жизни облака. Необходимы более эффективные механизмы
укрупнения.

\begin{enumerate}
\def\labelenumi{\arabic{enumi}.}
\item
  \textbf{Процесс ``тёплого дождя'' (Коагуляция и коалесценция):} Этот
  механизм доминирует в тёплых облаках (полностью находящихся при T
  \textgreater{} 0°C), что характерно для тропиков. Процесс начинается,
  когда в облаке формируется широкий спектр размеров капель. Более
  крупные капли имеют большую скорость падения и, двигаясь вниз,
  сталкиваются с мелкими каплями на своём пути (коагуляция) и сливаются
  с ними (коалесценция). Эффективность этого процесса зависит от
  \textbf{коэффициента захвата}, который максимален, когда коллекторная
  капля значительно крупнее собираемых ею капель. Таким образом, для
  запуска процесса необходимо наличие в облаке небольшого числа
  ``гигантских'' ядер конденсации.
\item
  \textbf{Ледяной (Бержерона-Финдайзена) процесс:} Это основной механизм
  образования осадков в умеренных и высоких широтах, а также в верхних
  частях мощных облаков в тропиках. Он реализуется в \textbf{смешанных
  облаках}, где при температурах от 0°C до -40°C одновременно существуют
  переохлаждённые капли воды и ледяные кристаллы.

  \begin{itemize}
  \item
    \textbf{Физическая основа:} Ключевым фактором является различие в
    давлении насыщенного водяного пара над поверхностью переохлаждённой
    воды (\(E_w\)) и льда (\(E_i\)) при одной и той же отрицательной
    температуре. Поскольку молекулы в кристаллической решётке льда
    связаны прочнее, им ``труднее'' её покинуть, поэтому \(E_i < E_w\).
  \item
    \textbf{Механизм:} В смешанном облаке воздух, который может быть
    насыщен или даже недосыщен относительно капель, оказывается
    \textbf{пересыщенным} относительно ледяных кристаллов. Это создаёт
    градиент давления водяного пара, направленный от капель к
    кристаллам. В результате происходит непрерывный процесс:
    \textbf{капли испаряются, а кристаллы растут} за счёт сублимации
    (депозиции) водяного пара.
  \item
    \textbf{Дальнейший рост:} Укрупнившиеся кристаллы начинают падать и
    продолжают расти за счёт двух других механизмов:

    \begin{itemize}
    \item
      \textbf{Аккреция (обледенение):} Столкновение с переохлаждёнными
      каплями и их намерзание на поверхности кристалла. Интенсивная
      аккреция приводит к образованию \textbf{снежной крупы} и
      \textbf{града}.
    \item
      \textbf{Агрегация (смерзание):} Столкновение и слипание ледяных
      кристаллов друг с другом, что приводит к образованию
      \textbf{снежинок}.
    \end{itemize}
  \end{itemize}
\end{enumerate}

\hypertarget{ux43eux441ux43eux431ux435ux43dux43dux43eux441ux442ux438-ux43eux431ux440ux430ux437ux43eux432ux430ux43dux438ux44f-ux433ux440ux430ux434ux430.-ux43dux430ux437ux435ux43cux43dux430ux44f-ux43aux43eux43dux434ux435ux43dux441ux430ux446ux438ux44f-ux438-ux43eux441ux430ux434ux43aux438}{%
\subsubsection{\texorpdfstring{\textbf{Особенности образования града.
Наземная конденсация и
осадки}}{Особенности образования града. Наземная конденсация и осадки}}\label{ux43eux441ux43eux431ux435ux43dux43dux43eux441ux442ux438-ux43eux431ux440ux430ux437ux43eux432ux430ux43dux438ux44f-ux433ux440ux430ux434ux430.-ux43dux430ux437ux435ux43cux43dux430ux44f-ux43aux43eux43dux434ux435ux43dux441ux430ux446ux438ux44f-ux438-ux43eux441ux430ux434ux43aux438}}

\begin{itemize}
\item
  \textbf{Град:} Образуется только в кучево-дождевых облаках с очень
  мощными восходящими потоками (\textgreater{} 10-15 м/с). Зародыш града
  (крупинка или замёрзшая капля) многократно поднимается и опускается в
  облаке, обрастая слоями прозрачного и непрозрачного льда в зонах с
  разной водностью и температурой.
\item
  \textbf{Наземная конденсация и осадки:}

  \begin{itemize}
  \item
    \textbf{Роса:} Конденсация водяного пара на поверхностях,
    охладившихся ниже точки росы.
  \item
    \textbf{Иней:} Сублимация водяного пара на поверхностях,
    охладившихся ниже 0°C.
  \item
    \textbf{Изморозь:} Осаждение и замерзание переохлаждённых капель
    тумана на предметах.
  \end{itemize}
\end{itemize}

\hypertarget{ux430ux43aux442ux438ux432ux43dux43eux435-ux432ux43eux437ux434ux435ux439ux441ux442ux432ux438ux435-ux43dux430-ux43eux431ux43bux430ux43aux430-ux442ux443ux43cux430ux43dux44b-ux43eux441ux430ux434ux43aux438-ux43cux435ux445ux430ux43dux438ux437ux43cux44b-ux438-ux44dux444ux444ux435ux43aux442ux438ux432ux43dux43eux441ux442ux44c}{%
\subsubsection{\texorpdfstring{\textbf{Активное воздействие на облака,
туманы, осадки: механизмы и
эффективность}}{Активное воздействие на облака, туманы, осадки: механизмы и эффективность}}\label{ux430ux43aux442ux438ux432ux43dux43eux435-ux432ux43eux437ux434ux435ux439ux441ux442ux432ux438ux435-ux43dux430-ux43eux431ux43bux430ux43aux430-ux442ux443ux43cux430ux43dux44b-ux43eux441ux430ux434ux43aux438-ux43cux435ux445ux430ux43dux438ux437ux43cux44b-ux438-ux44dux444ux444ux435ux43aux442ux438ux432ux43dux43eux441ux442ux44c}}

\begin{itemize}
\item
  \textbf{Засев облаков:} Основной метод. Включает введение в облако
  реагентов для стимуляции фазовых переходов.

  \begin{itemize}
  \item
    \textbf{Для переохлаждённых облаков:} Используют реагенты,
    вызывающие кристаллизацию (твёрдая углекислота, охлаждающая облако
    до -40°C, или йодистое серебро, служащее искусственным ядром
    кристаллизации).
  \item
    \textbf{Для тёплых облаков:} Используют гигроскопические частицы
    (например, соль) для создания крупных зародышевых капель и запуска
    процесса коагуляции.
  \end{itemize}
\item
  \textbf{Цели и эффективность:}

  \begin{itemize}
  \item
    \textbf{Рассеяние туманов:} Наиболее успешное применение.
    Переохлаждённые туманы эффективно рассеиваются путём их
    кристаллизации.
  \item
    \textbf{Увеличение осадков:} Эффективность остаётся предметом
    дискуссий. Статистически доказать значимое увеличение осадков сложно
    из-за их высокой естественной изменчивости.
  \item
    \textbf{Противоградовая защита:} Наиболее экономически оправданное
    направление. Цель --- ввести в облако избыточное количество ядер
    кристаллизации, чтобы образовалось множество мелких градин, которые
    успеют растаять при падении, вместо небольшого числа крупных и
    разрушительных.
  \end{itemize}
\end{itemize}

ewpage

\hypertarget{ux43eux43fux442ux438ux447ux435ux441ux43aux438ux435-ux44fux432ux43bux435ux43dux438ux44f-ux441ux432ux44fux437ux430ux43dux43dux44bux435-ux441-ux440ux430ux441ux441ux435ux44fux43dux438ux435ux43c-ux441ux432ux435ux442ux430-ux432-ux430ux442ux43cux43eux441ux444ux435ux440ux435}{%
\section{Оптические Явления, Связанные с Рассеянием Света в
Атмосфере}\label{ux43eux43fux442ux438ux447ux435ux441ux43aux438ux435-ux44fux432ux43bux435ux43dux438ux44f-ux441ux432ux44fux437ux430ux43dux43dux44bux435-ux441-ux440ux430ux441ux441ux435ux44fux43dux438ux435ux43c-ux441ux432ux435ux442ux430-ux432-ux430ux442ux43cux43eux441ux444ux435ux440ux435}}

Как профессиональные метеорологи, мы понимаем, что оптические явления в
атмосфере обусловлены взаимодействием электромагнитного излучения, в
частности видимого света, с воздушной средой. Атмосферная оптика как
дисциплина целенаправленно изучает феномены, вызванные рассеянием,
поглощением, преломлением и дифракцией света в воздухе.

Атмосфера, по сути, является сложной мутной средой, содержащей множество
хаотически расположенных неоднородностей. К таким неоднородностям
относятся флуктуации плотности воздуха, частицы атмосферного аэрозоля, а
также капли воды и снежинки. Взаимодействие света с этими элементами
приводит к многообразию наблюдаемых оптических эффектов.

Ключевые процессы взаимодействия включают:

\begin{itemize}
\tightlist
\item
  \textbf{Рассеяние света:} Это процесс, при котором направление
  распространения излучения изменяется, и мы воспринимаем это как
  ``свечение'' самой среды. Характер рассеяния существенно зависит от
  соотношения между размером рассеивающих частиц и длиной волны
  падающего солнечного излучения.

  \begin{itemize}
  \tightlist
  \item
    \textbf{Рассеяние Рэлея:} Происходит, когда размеры рассеивающих
    неоднородностей (например, молекул газов) значительно меньше длины
    волны видимого света (порядка 10⁻⁹ мкм для молекул против 10⁻⁷ мкм
    для видимого света). В этом случае рассеяние обратно пропорционально
    четвертой степени длины волны, что означает гораздо более сильное
    рассеяние коротковолновой части спектра.
  \item
    \textbf{Рассеяние Ми:} Наблюдается, когда световые волны
    взаимодействуют с частицами атмосферного аэрозоля, размеры которых
    сопоставимы с длиной волны видимого света. При таком типе рассеяния
    зависимость потока рассеянной радиации от длины волны практически
    исчезает, и белый свет после рассеяния остается белым.
  \end{itemize}
\item
  \textbf{Поглощение света:} Возникает, когда атомы и молекулы газов,
  входящих в состав воздуха, переходят в другое энергетическое состояние
  под воздействием солнечной радиации. В основном, ультрафиолетовые лучи
  практически полностью поглощаются молекулярным и атомарным кислородом,
  а также озоном и азотом в верхних слоях атмосферы. Водяной пар,
  углекислый газ и озон эффективно поглощают инфракрасную радиацию. В
  видимом диапазоне спектра солнечной радиации атмосферные газы
  поглощают энергию незначительно.
\item
  \textbf{Преломление (рефракция):} Представляет собой искривление луча
  света при прохождении через слои атмосферы, имеющие неодинаковую
  плотность. Луч света при этом отклоняется в сторону более плотного и,
  как следствие, более холодного воздуха.
\item
  \textbf{Отражение света:} Взаимодействие световой волны с веществом, в
  результате которого часть волны распространяется обратно. Энергия
  отраженной радиации количественно определяется альбедо поверхности.
\item
  \textbf{Дифракция света:} Наблюдается, когда световые волны
  распространяются мимо резких краев непрозрачных тел, что приводит к
  нарушению прямолинейности распространения света. В атмосфере роль
  дифракционной решетки чаще всего играют очень мелкие капли тумана,
  размер которых составляет около 1 мкм.
\item
  \textbf{Интерференция света:} В сочетании с дифракцией может приводить
  к окрашиванию изображений.
\end{itemize}

Среди конкретных оптических явлений, которые мы часто наблюдаем и
которые упоминаются в источниках, можно выделить следующие:

\begin{itemize}
\tightlist
\item
  \textbf{Радуга:} Это явление видно, когда Солнце находится позади
  наблюдателя, а дождь -- впереди. Каждая дождевая капля действует как
  призма, разлагая белый свет в спектр за счет комбинации полного
  внутреннего отражения и преломления.
\item
  \textbf{Гало:} Наблюдается, если кристаллическое облако (состоящее из
  ледяных кристаллов призматической формы) находится под определенным
  углом к Солнцу или Луне, позволяя свету пройти через две грани
  кристалла.
\item
  \textbf{Миражи:} Являются следствием рефракции света в слоях воздуха с
  резкими различиями в плотности и температуре.

  \begin{itemize}
  \tightlist
  \item
    \textbf{Верхний мираж:} Изображение объекта поднято над его реальным
    положением, поскольку лучи света отклоняются вниз. Это характерно
    для полярных районов и холодных морей, где воздух у земли холоднее
    (инверсия) и, следовательно, плотнее, чем вышележащие слои.
  \item
    \textbf{Нижний мираж:} Изображение объекта кажется расположенным
    ниже его реального положения, так как лучи света отклоняются вверх.
    Возникает, когда температура воздуха очень быстро падает с высотой
    (вертикальный градиент температуры превышает 3,42 °С/100 м),
    создавая эффект ``зеркала'' у поверхности.
  \end{itemize}
\item
  \textbf{Венцы:} Это световые круги, окрашенные в цвета радуги,
  разделенные круговыми тенями, возникающие за счет дифракции и
  интерференции на мелких каплях облаков и туманов.
\item
  \textbf{Глории (нимбы, брокенские призраки):} Аналогичные круги,
  которые возникают вокруг тени наблюдателя на тумане, облаке или росе.
\item
  \textbf{Дымка:} Помутнение воздуха, при котором метеорологическая
  дальность видимости (МДВ) находится в пределах от 1 до 10 км.
  Обусловлена рассеянием и поглощением солнечной радиации взвешенными
  частицами.
\item
  \textbf{Туман:} Мельчайшие капли воды или кристаллы льда, взвешенные в
  приземном слое воздуха, снижающие горизонтальную видимость до значений
  менее 1 км. Возникает также за счет рассеяния и поглощения.
\item
  \textbf{Сухое опалесцирующее помутнение:} Проявляется в голубоватой
  окраске отдаленных предметов и оранжевых тонах в проходящих лучах
  Солнца. Это происходит, когда рассеивающие частицы очень малы или их
  количество невелико, при этом горизонтальная видимость превышает 10
  км.
\item
  \textbf{Мгла:} Сероватая или беловатая пелена, ухудшающая видимость до
  значений менее 10 км. При мгле относительная влажность воздуха часто
  бывает менее 50\%.
\end{itemize}

\hypertarget{ux446ux432ux435ux442-ux43dux435ux431ux430}{%
\section{Цвет Неба}\label{ux446ux432ux435ux442-ux43dux435ux431ux430}}

Цвет неба является прямым следствием Рэлеевского рассеяния солнечной
радиации в атмосфере. В дневное время, когда Солнце находится высоко,
короткие волны видимого спектра -- фиолетовые и голубые -- рассеиваются
наиболее интенсивно во всех направлениях. Именно это обильное рассеяние
коротких волн придает безоблачному небу его характерный голубой оттенок.

По мере того как Солнце опускается к горизонту, длина пути солнечных
лучей через атмосферу значительно увеличивается. В результате, большая
часть коротковолновой (голубой) компоненты рассеивается в стороны или
поглощается до того, как достигает наблюдателя. Это приводит к тому, что
до глаз наблюдателя доходят преимущественно длинноволновые компоненты
спектра -- оранжевые и красные лучи. Эти лучи также рассеиваются,
определяя цвет неба при закате.

\hypertarget{ux44fux440ux43aux43eux441ux442ux44c-ux43dux435ux431ux435ux441ux43dux43eux433ux43e-ux441ux432ux43eux434ux430}{%
\section{Яркость Небесного
Свода}\label{ux44fux440ux43aux43eux441ux442ux44c-ux43dux435ux431ux435ux441ux43dux43eux433ux43e-ux441ux432ux43eux434ux430}}

Яркость небесного свода объясняется тем же процессом рассеяния солнечной
радиации молекулами воздуха и аэрозольными частицами. Рассеянная
радиация поступает к наблюдателю не только непосредственно от Солнца, но
и со всей небесной полусферы. Таким образом, даже в направлении,
отличном от прямого солнечного света, мы видим свет благодаря рассеянию,
что создает эффект светлого, а не черного, небесного свода.
Интенсивность этой рассеянной радиации определяет воспринимаемую яркость
неба.

\hypertarget{ux43aux430ux436ux443ux449ux430ux44fux441ux44f-ux444ux43eux440ux43cux430-ux43dux435ux431ux435ux441ux43dux43eux433ux43e-ux441ux432ux43eux434ux430-ux44fux432ux43bux435ux43dux438ux44f-ux441-ux44dux442ux438ux43c-ux441ux432ux44fux437ux430ux43dux43dux44bux435}{%
\section{Кажущаяся Форма Небесного Свода, Явления, С Этим
Связанные}\label{ux43aux430ux436ux443ux449ux430ux44fux441ux44f-ux444ux43eux440ux43cux430-ux43dux435ux431ux435ux441ux43dux43eux433ux43e-ux441ux432ux43eux434ux430-ux44fux432ux43bux435ux43dux438ux44f-ux441-ux44dux442ux438ux43c-ux441ux432ux44fux437ux430ux43dux43dux44bux435}}

Относительно кажущейся формы небесного свода и связанных с ней явлений,
следует отметить, что представленные источники не содержат детальной
информации по этому специфическому аспекту атмосферной оптики. Основное
внимание уделяется физическим механизмам взаимодействия света с
атмосферой и конкретным, измеримым оптическим явлениям, возникающим в
результате этих взаимодействий. Восприятие кажущейся формы небесного
свода, скорее, относится к области психофизиологии восприятия, нежели к
метеорологической оптике, рассмотренной в данном контексте.

ewpage

\hypertarget{ux43eux441ux432ux435ux449ux435ux43dux43dux43eux441ux442ux44c-ux437ux435ux43cux43dux43eux439-ux43fux43eux432ux435ux440ux445ux43dux43eux441ux442ux438-ux441ux432ux435ux447ux435ux43dux438ux435-ux43dux435ux431ux430-ux438-ux432ux438ux434ux438ux43cux43eux441ux442ux44c}{%
\section{Освещенность Земной Поверхности, Свечение Неба и
Видимость}\label{ux43eux441ux432ux435ux449ux435ux43dux43dux43eux441ux442ux44c-ux437ux435ux43cux43dux43eux439-ux43fux43eux432ux435ux440ux445ux43dux43eux441ux442ux438-ux441ux432ux435ux447ux435ux43dux438ux435-ux43dux435ux431ux430-ux438-ux432ux438ux434ux438ux43cux43eux441ux442ux44c}}

\hypertarget{ux43eux441ux432ux435ux449ux435ux43dux43dux43eux441ux442ux44c-ux437ux435ux43cux43dux43eux439-ux43fux43eux432ux435ux440ux445ux43dux43eux441ux442ux438-ux432-ux440ux430ux437ux43bux438ux447ux43dux43eux435-ux432ux440ux435ux43cux44f-ux441ux443ux442ux43eux43a}{%
\subsection{Освещенность Земной Поверхности в Различное Время
Суток}\label{ux43eux441ux432ux435ux449ux435ux43dux43dux43eux441ux442ux44c-ux437ux435ux43cux43dux43eux439-ux43fux43eux432ux435ux440ux445ux43dux43eux441ux442ux438-ux432-ux440ux430ux437ux43bux438ux447ux43dux43eux435-ux432ux440ux435ux43cux44f-ux441ux443ux442ux43eux43a}}

Освещенность земной поверхности напрямую связана с инсоляцией ---
распределением солнечной радиации по земной поверхности. Эта величина
определяется рядом астрономических и геофизических факторов:
шарообразной формой Земли, ее вращением вокруг своей оси, наклоном оси к
плоскости эклиптики и вращением планеты по орбите вокруг Солнца.

\hypertarget{ux432ux43bux438ux44fux43dux438ux435-ux444ux43eux440ux43cux44b-ux437ux435ux43cux43bux438-ux438-ux443ux433ux43bux430-ux43fux430ux434ux435ux43dux438ux44f-ux441ux43eux43bux43dux435ux447ux43dux44bux445-ux43bux443ux447ux435ux439}{%
\subsubsection{Влияние Формы Земли и Угла Падения Солнечных
Лучей}\label{ux432ux43bux438ux44fux43dux438ux435-ux444ux43eux440ux43cux44b-ux437ux435ux43cux43bux438-ux438-ux443ux433ux43bux430-ux43fux430ux434ux435ux43dux438ux44f-ux441ux43eux43bux43dux435ux447ux43dux44bux445-ux43bux443ux447ux435ux439}}

Шарообразность Земли приводит к тому, что одно и то же количество
радиации падает на экваторе на площадку, перпендикулярную солнечным
лучам, в то время как на высоких широтах солнечные лучи падают на
поверхность под углом, называемым высотой солнца (\texttt{hc}).
Распределение солнечной радиации по поверхности Земли регулируется
высотой солнца, которая в свою очередь зависит от широты, склонения
Солнца и часового угла. Если высота солнца, рассчитанная по формуле,
оказывается отрицательной, это соответствует ночи (время от захода до
восхода Солнца), и формула для расчета инсоляции в это время не
применяется.

\hypertarget{ux441ux443ux442ux43eux447ux43dux44bux439-ux445ux43eux434-ux438ux43dux441ux43eux43bux44fux446ux438ux438}{%
\subsubsection{Суточный Ход
Инсоляции}\label{ux441ux443ux442ux43eux447ux43dux44bux439-ux445ux43eux434-ux438ux43dux441ux43eux43bux44fux446ux438ux438}}

Суточный ход инсоляции не является гладкой функцией времени, поскольку в
моменты восхода и захода Солнца изменения очень резкие. В течение дня
радиационный баланс почти всегда положителен, так как поступление
солнечной радиации значительно превышает эффективное излучение. Ночью же
радиационный баланс отрицателен. Переход радиационного баланса через
нуль приходится на светлое время суток --- примерно за час до захода
Солнца и через час после восхода.

\hypertarget{ux440ux430ux441ux43fux440ux435ux434ux435ux43bux435ux43dux438ux435-ux43fux43e-ux448ux438ux440ux43eux442ux430ux43c-ux438-ux441ux435ux437ux43eux43dux43dux44bux435-ux43aux43eux43bux435ux431ux430ux43dux438ux44f}{%
\subsubsection{Распределение по Широтам и Сезонные
Колебания}\label{ux440ux430ux441ux43fux440ux435ux434ux435ux43bux435ux43dux438ux435-ux43fux43e-ux448ux438ux440ux43eux442ux430ux43c-ux438-ux441ux435ux437ux43eux43dux43dux44bux435-ux43aux43eux43bux435ux431ux430ux43dux438ux44f}}

На разных широтах суточный ход инсоляции различен. В летнем полушарии
инсоляция в умеренных широтах почти такая же, как на экваторе, а у
полюсов --- лишь немного меньше, при этом продолжительность дня там
больше. В зимнем полушарии инсоляция резко падает, особенно вблизи зоны
полярной ночи.

В летние дни максимум инсоляции наблюдается не на самом экваторе, а на
широте тропика летнего полушария. За счет полярного дня в летний период
в район полюса поступает не меньше радиации, чем на экватор. Максимум
инсоляции в экваториальной зоне бывает дважды в год: в месяцы весеннего
и осеннего равноденствия. В зимнем полушарии инсоляция быстро убывает с
широтой, а за полярным кругом зимнего полушария есть периоды полярной
ночи, когда земная поверхность не получает солнечной энергии.

\hypertarget{ux432ux437ux430ux438ux43cux43eux434ux435ux439ux441ux442ux432ux438ux435-ux441ux43eux43bux43dux435ux447ux43dux43eux439-ux440ux430ux434ux438ux430ux446ux438ux438-ux441-ux430ux442ux43cux43eux441ux444ux435ux440ux43eux439-1}{%
\subsubsection{Взаимодействие Солнечной Радиации с
Атмосферой}\label{ux432ux437ux430ux438ux43cux43eux434ux435ux439ux441ux442ux432ux438ux435-ux441ux43eux43bux43dux435ux447ux43dux43eux439-ux440ux430ux434ux438ux430ux446ux438ux438-ux441-ux430ux442ux43cux43eux441ux444ux435ux440ux43eux439-1}}

По пути из космоса до земной поверхности солнечная радиация
взаимодействует с атмосферой.

\hypertarget{ux440ux435ux444ux440ux430ux43aux446ux438ux44f}{%
\paragraph{Рефракция}\label{ux440ux435ux444ux440ux430ux43aux446ux438ux44f}}

Показатель преломления воздуха постепенно увеличивается (скорость света
замедляется) по мере прохождения солнечной радиации через все более
плотные слои воздуха. Это приводит к явлению рефракции, при котором луч
света отклоняется в сторону более холодного и плотного воздуха.

\hypertarget{ux43eux442ux440ux430ux436ux435ux43dux438ux435-ux430ux43bux44cux431ux435ux434ux43e-1}{%
\paragraph{Отражение
(Альбедо)}\label{ux43eux442ux440ux430ux436ux435ux43dux438ux435-ux430ux43bux44cux431ux435ux434ux43e-1}}

При прохождении радиации через границу слоев с резко различными
показателями преломления, например, из воздуха в воду, электромагнитные
волны (ЭМВ) испытывают отражение. Энергия отраженной радиации
определяется альбедо. Альбедо различных подстилающих поверхностей
существенно различается.

\hypertarget{ux440ux430ux441ux441ux435ux44fux43dux438ux435-1}{%
\paragraph{Рассеяние}\label{ux440ux430ux441ux441ux435ux44fux43dux438ux435-1}}

Рассеяние --- это процесс изменения направления распространения
излучения, которое воспринимается как несобственное свечение среды.

\begin{itemize}
\tightlist
\item
  \textbf{Рассеяние Рэлея:} Происходит, если размер неоднородностей в
  атмосфере (например, молекул газов) сопоставим с размерами молекул
  газов (10⁻⁹ мкм), а длины волн видимого света значительно больше их
  (10⁻⁷ мкм). В этом случае поток рассеянной энергии сильно зависит от
  длины волны падающего света: самые короткие световые волны
  рассеиваются сильнее всего. Этот эффект зависит от высоты Солнца. Если
  Солнце высоко (день), рассеиваются фиолетовые и голубые волны, поэтому
  безоблачное небо днем кажется голубым. Если Солнце низко (закат),
  голубые волны поглощаются за линией горизонта, и до наблюдателя
  доходят только оранжевые и красные лучи, которые рассеиваются и
  определяют цвет неба.
\item
  \textbf{Рассеяние Ми:} Происходит, когда световые волны попадают на
  частицы атмосферного аэрозоля, размер которых сравним с длиной волны.
  Зависимость потока рассеянной радиации от длины волны исчезает, и
  белый свет после рассеяния остается белым. Этим видом рассеяния
  объясняется белый цвет безоблачных облаков.
\end{itemize}

\hypertarget{ux43fux43eux433ux43bux43eux449ux435ux43dux438ux435-1}{%
\paragraph{Поглощение}\label{ux43fux43eux433ux43bux43eux449ux435ux43dux438ux435-1}}

Поглощение радиации происходит, когда атомы и молекулы газов,
составляющих воздух, переходят в другое энергетическое состояние под
воздействием солнечной радиации. Ультрафиолетовые лучи практически
полностью поглощаются молекулярным и атомарным кислородом, а также
азотом и озоном в верхних слоях атмосферы. Газы H₂O, CO₂ и O₃ поглощают
инфракрасную радиацию. В видимой области спектра солнечной радиации
атмосферные газы не поглощают энергию.

Суммарный поток лучистой энергии в атмосфере складывается из
длинноволнового излучения атмосферы, направленного вниз, длинноволнового
излучения Земли и атмосферы, направленного вверх, и коротковолнового
солнечного излучения, направленного вниз.

\hypertarget{ux441ux432ux435ux447ux435ux43dux438ux435-ux43dux43eux447ux43dux43eux433ux43e-ux43dux435ux431ux430}{%
\subsection{Свечение Ночного
Неба}\label{ux441ux432ux435ux447ux435ux43dux438ux435-ux43dux43eux447ux43dux43eux433ux43e-ux43dux435ux431ux430}}

Понятие ``свечение ночного неба'' в контексте видимого света, не
связанного с солнечной радиацией, не детализируется в представленных
источниках. Однако, изложенные принципы рассеяния света дают
представление о том, как атмосфера может ``светиться'' за счет рассеяния
солнечного света в периоды, когда Солнце находится вблизи горизонта
(сумерки, заря). Источники не описывают механизмы свечения ночного неба
в условиях полного отсутствия прямого и рассеянного солнечного света.

Что касается теплового обмена, атмосфера излучает длинноволновую
радиацию во все стороны по закону Кирхгофа. Часть этого излучения
возвращается к поверхности Земли и называется противоизлучением
атмосферы, что уменьшает потерю тепла земной поверхностью и приводит к
парниковому эффекту. Эта длинноволновая радиация находится в
инфракрасном диапазоне, то есть не является видимым свечением.

\hypertarget{ux44fux440ux43aux43eux441ux442ux44c-ux444ux43eux43dux430-ux438-ux440ux430ux437ux43bux438ux447ux43dux44bux445-ux43fux43eux432ux435ux440ux445ux43dux43eux441ux442ux435ux439}{%
\subsection{Яркость Фона и Различных
Поверхностей}\label{ux44fux440ux43aux43eux441ux442ux44c-ux444ux43eux43dux430-ux438-ux440ux430ux437ux43bux438ux447ux43dux44bux445-ux43fux43eux432ux435ux440ux445ux43dux43eux441ux442ux435ux439}}

Яркость различных поверхностей определяется их способностью отражать
падающую на них солнечную радиацию, то есть их альбедо.

\hypertarget{ux430ux43bux44cux431ux435ux434ux43e-ux440ux430ux437ux43bux438ux447ux43dux44bux445-ux43fux43eux432ux435ux440ux445ux43dux43eux441ux442ux435ux439}{%
\subsubsection{Альбедо Различных
Поверхностей}\label{ux430ux43bux44cux431ux435ux434ux43e-ux440ux430ux437ux43bux438ux447ux43dux44bux445-ux43fux43eux432ux435ux440ux445ux43dux43eux441ux442ux435ux439}}

Значения альбедо установлены для разных видов подстилающих поверхностей,
как для отдельных длин ЭМВ, так и для всего потока солнечной радиации.
Особенно велико альбедо снега, льда и облаков.

Примеры значений альбедо (\%):

\begin{itemize}
\tightlist
\item
  Свежий снег: 80-90
\item
  Старый снег: 40-70
\item
  Песок: 20-30
\item
  Трава: 20-25
\item
  Сухая почва: 15-25
\item
  Мокрая почва: 10-15
\item
  Лес: 10-20
\item
  Вода (солнце низко): 10-70
\item
  Вода (солнце высоко): 3-10
\item
  Мощные облака: 60-70
\item
  Тонкие облака: 25-50
\item
  Земля как планета: 33
\end{itemize}

Альбедо облаков, например, имеет высокое значение (70-80\%), что сильно
ослабляет приток солнечной радиации к водной поверхности. Влияние
облачности на результирующий приток радиации к земной поверхности
увеличивается, когда радиационный баланс отрицателен (зимой в умеренных
и высоких широтах), и уменьшается, когда он положителен (летом).

\hypertarget{ux441ux443ux43cux435ux440ux43aux438-ux438-ux437ux430ux440ux44f}{%
\subsection{Сумерки и
Заря}\label{ux441ux443ux43cux435ux440ux43aux438-ux438-ux437ux430ux440ux44f}}

Сумерки и заря представляют собой переходные периоды между днем и ночью,
характеризующиеся резкими изменениями освещенности.

\hypertarget{ux438ux437ux43cux435ux43dux435ux43dux438ux44f-ux438ux43dux441ux43eux43bux44fux446ux438ux438}{%
\subsubsection{Изменения
Инсоляции}\label{ux438ux437ux43cux435ux43dux435ux43dux438ux44f-ux438ux43dux441ux43eux43bux44fux446ux438ux438}}

Как уже упоминалось, суточный ход инсоляции не является гладкой
функцией, и в моменты восхода и захода Солнца ее изменения очень резкие.
Переход радиационного баланса через нуль, то есть момент, когда
поступление солнечной энергии уравновешивается эффективным излучением,
приходится на светлое время суток: за час до захода Солнца и через час
после восхода.

\hypertarget{ux43eux43aux440ux430ux441ux43aux430-ux43dux435ux431ux430-ux437ux430ux440ux44f}{%
\subsubsection{Окраска Неба
(Заря)}\label{ux43eux43aux440ux430ux441ux43aux430-ux43dux435ux431ux430-ux437ux430ux440ux44f}}

Окраска неба во время заката (заря) объясняется эффектом рассеяния
Рэлея. Когда Солнце находится низко над горизонтом, коротковолновые
(голубые) волны солнечного спектра поглощаются и рассеиваются за линией
горизонта, а до наблюдателя доходят преимущественно оранжевые и красные
лучи. Эти лучи рассеиваются в атмосфере, определяя характерный цвет неба
во время заката.

\hypertarget{ux432ux43bux438ux44fux43dux438ux435-ux43dux430-ux432ux438ux434ux438ux43cux43eux441ux442ux44c}{%
\subsubsection{Влияние на
Видимость}\label{ux432ux43bux438ux44fux43dux438ux435-ux43dux430-ux432ux438ux434ux438ux43cux43eux441ux442ux44c}}

В сумерки, при резком уменьшении освещенности, контуры предметов быстро
размываются, что приводит к наибольшей неопределенности в определении
дальности видимости. Суточный ход видимости в приземном слое
определяется преимущественно суточным ходом относительной влажности:
увеличение относительной влажности при ночном охлаждении приземного слоя
воздушной массы вызывает ухудшение видимости, вплоть до образования
дымки и тумана.

ewpage

\hypertarget{ux43cux435ux442ux435ux43eux440ux43eux43bux43eux433ux438ux447ux435ux441ux43aux430ux44f-ux434ux430ux43bux44cux43dux43eux441ux442ux44c-ux432ux438ux434ux438ux43cux43eux441ux442ux438-ux438-ux432ux43bux438ux44fux44eux449ux438ux435-ux43dux430-ux43dux435ux435-ux444ux430ux43aux442ux43eux440ux44b}{%
\section{Метеорологическая дальность видимости и влияющие на нее
факторы}\label{ux43cux435ux442ux435ux43eux440ux43eux43bux43eux433ux438ux447ux435ux441ux43aux430ux44f-ux434ux430ux43bux44cux43dux43eux441ux442ux44c-ux432ux438ux434ux438ux43cux43eux441ux442ux438-ux438-ux432ux43bux438ux44fux44eux449ux438ux435-ux43dux430-ux43dux435ux435-ux444ux430ux43aux442ux43eux440ux44b}}

Метеорологическая дальность видимости определяется как наибольшее
расстояние, на котором абсолютно черный предмет, имеющий угловые размеры
более 20', остается различимым днем на фоне неба у горизонта. В ночное
время дальность видимости определяется по расстоянию до наиболее
удаленного видимого точечного источника света известной силы. Визуальное
определение дальности видимости часто бывает меньше метеорологической,
поскольку используемые объекты не всегда соответствуют требованиям,
например, по цвету.

Точность визуальных наблюдений за дальностью видимости имеет большие
погрешности и не всегда сопоставима, так как она зависит от условий
наблюдения (например, ясное небо или серые облака, тип подстилающей
поверхности), а также от отсутствия объектов наблюдения на необходимых
расстояниях. Наибольшая неопределенность возникает в сумерки, когда
контуры предметов быстро размываются из-за резкого уменьшения
освещенности. Приборные измерения с помощью регистраторов прозрачности
дают более однородные и сравнимые результаты.

Дальность видимости является одним из важнейших метеорологических
элементов, определяющих сложность погодных условий для различных видов
деятельности, в частности, для авиации, где важна не только
горизонтальная, но и наклонная видимость.

\hypertarget{ux444ux430ux43aux442ux43eux440ux44b-ux432ux43bux438ux44fux44eux449ux438ux435-ux43dux430-ux43cux435ux442ux435ux43eux440ux43eux43bux43eux433ux438ux447ux435ux441ux43aux443ux44e-ux434ux430ux43bux44cux43dux43eux441ux442ux44c-ux432ux438ux434ux438ux43cux43eux441ux442ux438}{%
\subsection{Факторы, влияющие на метеорологическую дальность
видимости}\label{ux444ux430ux43aux442ux43eux440ux44b-ux432ux43bux438ux44fux44eux449ux438ux435-ux43dux430-ux43cux435ux442ux435ux43eux440ux43eux43bux43eux433ux438ux447ux435ux441ux43aux443ux44e-ux434ux430ux43bux44cux43dux43eux441ux442ux44c-ux432ux438ux434ux438ux43cux43eux441ux442ux438}}

Видимость подвержена влиянию множества атмосферных явлений и параметров,
которые могут значительно ее ухудшать.

\hypertarget{ux442ux443ux43cux430ux43dux44b-ux438-ux434ux44bux43cux43aux438}{%
\subsubsection{1. Туманы и
дымки}\label{ux442ux443ux43cux430ux43dux44b-ux438-ux434ux44bux43cux43aux438}}

Туман --- это мельчайшие капли воды или кристаллы льда, взвешенные в
воздухе у земной поверхности, снижающие горизонтальную видимость до
значений менее 1 км. Он является особо опасным метеорологическим
явлением при видимости 100 м и менее, когда прекращается движение
автомобильного и железнодорожного транспорта на дорогах. Дымка
представляет собой сероватую или беловатую пелену, ухудшающую видимость
до значений менее 10 км, при этом относительная влажность часто
составляет менее 50\%.

Туманы классифицируются по механизму образования:

\begin{itemize}
\tightlist
\item
  \textbf{Радиационные туманы} формируются в результате ночного
  охлаждения воздуха у поверхности земли за счет эффективного излучения.
  Их образование благоприятствует при антициклонической циркуляции,
  малых барических градиентах, слабом ветре (менее 3 м/с), повышенной
  влажности воздуха в приземном слое, ясном небе или малой облачности
  (1-4 балла) вблизи инверсии, а также при наличии теплой и влажной
  подстилающей поверхности (например, сырой почвы после дождя или
  мокрого снега).
\item
  \textbf{Адвективные туманы} образуются при охлаждении относительно
  теплой и влажной воздушной массы в процессе ее перемещения на более
  холодную подстилающую поверхность. Они могут возникать над открытым
  морем при смещении воздушной массы с теплой поверхности на холодную.
\item
  \textbf{Туманы испарения} возникают, когда испарение с водной
  поверхности (например, после дождя на прогретой почве) приводит к
  насыщению приземного слоя воздуха паром, который затем конденсируется.
\item
  \textbf{Фронтальные туманы} часто связаны с теплыми фронтами и теплыми
  фронтами окклюзии, особенно при выпадении переохлажденного дождя.
\item
  \textbf{Морозные (поселковые) туманы} возникают над водоемами в
  холодное время года при адвекции очень холодного воздуха и
  значительном горизонтальном градиенте температуры между водой и
  воздухом, особенно в городах.
\item
  \textbf{Ледяные туманы} образуются при очень низких отрицательных
  температурах (ниже -10°С), когда водяной пар сублимируется, образуя
  ледяные кристаллы.
\end{itemize}

Дальность видимости в тумане зависит не только от его водности, но и от
размеров частиц.

\hypertarget{ux430ux442ux43cux43eux441ux444ux435ux440ux43dux44bux435-ux43eux441ux430ux434ux43aux438}{%
\subsubsection{2. Атмосферные
осадки}\label{ux430ux442ux43cux43eux441ux444ux435ux440ux43dux44bux435-ux43eux441ux430ux434ux43aux438}}

Различные виды осадков значительно ухудшают видимость:

\begin{itemize}
\tightlist
\item
  \textbf{Морось} (размер капель r \textless{} 0,5 мкм) и \textbf{дождь}
  (r \textgreater{} 0,5 мкм). Интенсивность дождя влияет на видимость,
  но в большинстве районов дожди редко ухудшают видимость до 2 км.
\item
  \textbf{Снег и снегопады}: \textbf{снежные зерна}, \textbf{снежные
  заряды} (кратковременное выпадение снега), \textbf{снегопад}
  (продолжительное выпадение снега), \textbf{ливневый снег} (сильный,
  кратковременный), \textbf{снежная буря} (выпадение снега при сильном
  ветре). Видимость при снегопаде ухудшается с усилением ветра и
  возрастанием интенсивности снегопада.
\end{itemize}

\hypertarget{ux43fux44bux43bux44cux43dux44bux435-ux43fux435ux441ux447ux430ux43dux44bux435-ux431ux443ux440ux438-ux438-ux43cux433ux43bux430}{%
\subsubsection{3. Пыльные (песчаные) бури и
мгла}\label{ux43fux44bux43bux44cux43dux44bux435-ux43fux435ux441ux447ux430ux43dux44bux435-ux431ux443ux440ux438-ux438-ux43cux433ux43bux430}}

Пыльная (песчаная) буря --- это перенос сильным ветром больших масс пыли
или песка, поднятых с поверхности. Ее горизонтальная протяженность может
достигать тысяч километров, а вертикальная --- нескольких километров
(иногда до 6-7 км). Видимость при пыльной буре колеблется от 10-20 м до
4-10 км. Интенсивность пыльной бури зависит от силы ветра, степени
развития турбулентных завихрений и устойчивости воздушной массы. Мгла --
это сероватая или беловатая пелена, ухудшающая видимость до значений
менее 10 км, при этом относительная влажность часто бывает менее 50\%.

\hypertarget{ux43aux43eux43dux446ux435ux43dux442ux440ux430ux446ux438ux44f-ux438-ux440ux430ux437ux43cux435ux440-ux447ux430ux441ux442ux438ux446-ux432-ux432ux43eux437ux434ux443ux445ux435}{%
\subsubsection{4. Концентрация и размер частиц в
воздухе}\label{ux43aux43eux43dux446ux435ux43dux442ux440ux430ux446ux438ux44f-ux438-ux440ux430ux437ux43cux435ux440-ux447ux430ux441ux442ux438ux446-ux432-ux432ux43eux437ux434ux443ux445ux435}}

Молекулы газов, а также различные примеси (газообразные, жидкие и
твердые) оптически активны и могут сильно поглощать и рассеивать
радиацию, особенно в инфракрасном диапазоне. Если размер неоднородностей
сопоставим с размерами молекул газов, короткие световые волны
рассеиваются сильнее всего. Увеличение содержания водяного пара, как и
любого парникового газа, сопровождается уменьшением эффективного потока
излучения и потерь тепла земной поверхностью, что влияет на температуру.

\hypertarget{ux442ux435ux43cux43fux435ux440ux430ux442ux443ux440ux430-ux438-ux432ux43bux430ux436ux43dux43eux441ux442ux44c-ux432ux43eux437ux434ux443ux445ux430}{%
\subsubsection{5. Температура и влажность
воздуха}\label{ux442ux435ux43cux43fux435ux440ux430ux442ux443ux440ux430-ux438-ux432ux43bux430ux436ux43dux43eux441ux442ux44c-ux432ux43eux437ux434ux443ux445ux430}}

Эти параметры тесно связаны и влияют на видимость через процессы
конденсации водяного пара:

\begin{itemize}
\tightlist
\item
  Относительная влажность зависит от абсолютного содержания водяного
  пара и температуры воздуха. Повышение температуры воздуха в городах
  приводит к понижению относительной влажности.
\item
  Увеличение влажности, особенно при ночном охлаждении приземного слоя
  воздуха, вызывает ухудшение видимости, вплоть до образования дымки и
  тумана.
\item
  Вероятность образования конвективных облаков, а значит, и осадков,
  выше при более высоких температуре и относительной влажности.
\item
  Изменения влажности воздуха с высотой: влажность быстро убывает с
  высотой, значительно быстрее, чем давление или плотность воздуха.
\end{itemize}

\hypertarget{ux432ux435ux442ux435ux440}{%
\subsubsection{6. Ветер}\label{ux432ux435ux442ux435ux440}}

Ветер играет двойную роль:

\begin{itemize}
\tightlist
\item
  Может усиливать ухудшение видимости, поднимая снег во время метелей
  или пыль и песок во время пыльных бурь.
\item
  Может способствовать рассеянию туманов, особенно радиационных, за счет
  механического перемешивания воздуха.
\end{itemize}

\hypertarget{ux43eux431ux43bux430ux447ux43dux43eux441ux442ux44c-1}{%
\subsubsection{7.
Облачность}\label{ux43eux431ux43bux430ux447ux43dux43eux441ux442ux44c-1}}

Облачность уменьшает как приток солнечной радиации к земной поверхности,
так и ее эффективное излучение. Это влияет на температуру приземного
слоя и, следовательно, на условия образования туманов. Большая
облачность также может маскировать теплый фронт. Связь высоты облаков с
видимостью при относительной влажности более 90\% показывает, что чем
ниже облака, тем хуже видимость.

\hypertarget{ux432ux440ux435ux43cux44f-ux441ux443ux442ux43eux43a}{%
\subsubsection{8. Время
суток}\label{ux432ux440ux435ux43cux44f-ux441ux443ux442ux43eux43a}}

Суточный ход видимости определяется преимущественно суточным ходом
относительной влажности в приземном слое. Ночное охлаждение приземного
слоя воздуха вызывает ухудшение видимости. Максимум развития
конвективных облаков и гроз смещен на более позднее время по сравнению с
максимумом приземной температуры.

\hypertarget{ux43cux435ux441ux442ux43dux44bux435-ux438-ux433ux435ux43eux433ux440ux430ux444ux438ux447ux435ux441ux43aux438ux435-ux444ux430ux43aux442ux43eux440ux44b}{%
\subsubsection{9. Местные и географические
факторы}\label{ux43cux435ux441ux442ux43dux44bux435-ux438-ux433ux435ux43eux433ux440ux430ux444ux438ux447ux435ux441ux43aux438ux435-ux444ux430ux43aux442ux43eux440ux44b}}

\begin{itemize}
\tightlist
\item
  \textbf{Орография}: Горы могут задерживать фронты, создавать
  извилистую линию фронта, а также вызывать усиление фронтальных осадков
  с наветренной стороны и размывание фронта с подветренной стороны за
  счет фёнового эффекта, что влияет на облачность и видимость.
\item
  \textbf{Тип подстилающей поверхности}: Различие в теплофизических
  свойствах суши и океана, а также неравномерное залегание снега или
  льда, чередующегося с открытой водой и влажной почвой, влияют на
  образование разностей температур и влажности, что сказывается на
  туманах и дымках.
\item
  \textbf{Береговой эффект}: На берегах моря сходимость линий тока при
  ветре с моря на сушу благоприятствует фронтогенезу и образованию
  туманов.
\end{itemize}

Таким образом, метеорологическая дальность видимости является
комплексной характеристикой, зависящей от текущего состояния атмосферы,
включая турбулентность, влажность, температуру, осадки, концентрацию
аэрозолей и локальные физико-географические условия.

ewpage

\hypertarget{ux434ux430ux43bux44cux43dux43eux441ux442ux44c-ux432ux438ux434ux438ux43cux43eux441ux442ux438-ux432-ux43cux435ux442ux435ux43eux440ux43eux43bux43eux433ux438ux438}{%
\section{Дальность Видимости в
Метеорологии}\label{ux434ux430ux43bux44cux43dux43eux441ux442ux44c-ux432ux438ux434ux438ux43cux43eux441ux442ux438-ux432-ux43cux435ux442ux435ux43eux440ux43eux43bux43eux433ux438ux438}}

\hypertarget{ux43eux431ux449ux438ux435-ux43fux43eux43dux44fux442ux438ux44f-ux438-ux43eux43fux440ux435ux434ux435ux43bux435ux43dux438ux435-ux434ux430ux43bux44cux43dux43eux441ux442ux438-ux432ux438ux434ux438ux43cux43eux441ux442ux438}{%
\subsection{1. Общие Понятия и Определение Дальности
Видимости}\label{ux43eux431ux449ux438ux435-ux43fux43eux43dux44fux442ux438ux44f-ux438-ux43eux43fux440ux435ux434ux435ux43bux435ux43dux438ux435-ux434ux430ux43bux44cux43dux43eux441ux442ux438-ux432ux438ux434ux438ux43cux43eux441ux442ux438}}

Дальность видимости, как один из ключевых метеорологических элементов,
играет существенную роль в характеристике погодных условий,
определяющих, например, операционную обстановку для авиации, морского и
речного флота, а также автомобильного транспорта.

\textbf{Метеорологическая дальность видимости (МДВ)} определяется как
наибольшее расстояние, на котором абсолютно черный предмет с угловыми
размерами более 20' остается различимым в дневное время на фоне неба у
горизонта.

\textbf{Ночная видимость} определяется по расстоянию до наиболее
удаленного видимого точечного источника света известной силы.

Важно отметить, что \textbf{визуальная дальность видимости}
(определяемая по реальным объектам) зачастую оказывается меньше
\textbf{метеорологической}, поскольку фактические объекты не всегда
соответствуют идеализированным требованиям абсолютно черного предмета
или оптимальным условиям фона. Инструментальные измерения прозрачности
приземного слоя воздуха, хотя и более однородны, не всегда удовлетворяют
всем практическим запросам, например, в авиации важна не только
горизонтальная, но и наклонная видимость.

\hypertarget{ux444ux430ux43aux442ux43eux440ux44b-ux432ux43bux438ux44fux44eux449ux438ux435-ux43dux430-ux434ux430ux43bux44cux43dux43eux441ux442ux44c-ux432ux438ux434ux438ux43cux43eux441ux442ux438-ux43eux431ux44aux435ux43aux442ux43eux432-ux438-ux43eux433ux43dux435ux439}{%
\subsection{2. Факторы, Влияющие на Дальность Видимости Объектов и
Огней}\label{ux444ux430ux43aux442ux43eux440ux44b-ux432ux43bux438ux44fux44eux449ux438ux435-ux43dux430-ux434ux430ux43bux44cux43dux43eux441ux442ux44c-ux432ux438ux434ux438ux43cux43eux441ux442ux438-ux43eux431ux44aux435ux43aux442ux43eux432-ux438-ux43eux433ux43dux435ux439}}

Дальность видимости определяется комплексом атмосферных условий, включая
характеристики воздушных масс и наличие различных помутнений.

\hypertarget{ux432ux43bux438ux44fux43dux438ux435-ux432ux43eux437ux434ux443ux448ux43dux44bux445-ux43cux430ux441ux441-ux438-ux43fux440ux438ux43cux435ux441ux435ux439}{%
\subsubsection{2.1. Влияние Воздушных Масс и
Примесей}\label{ux432ux43bux438ux44fux43dux438ux435-ux432ux43eux437ux434ux443ux448ux43dux44bux445-ux43cux430ux441ux441-ux438-ux43fux440ux438ux43cux435ux441ux435ux439}}

\begin{itemize}
\tightlist
\item
  \textbf{Происхождение воздушной массы}: Воздушные массы арктического
  происхождения, как правило, обладают наибольшей прозрачностью в
  Северном полушарии, где дальность видимости вершин гор может превышать
  300 км. В то же время запыленные воздушные массы тропического
  происхождения значительно ухудшают видимость.
\item
  \textbf{Атмосферные примеси}: Наличие в приземном слое воздуха
  примесей, рассеивающих свет, таких как частицы пыли, капельки воды
  (или кристаллы льда), а также осадки, существенно снижает видимость.
\end{itemize}

\hypertarget{ux442ux438ux43fux44b-ux43fux43eux43cux443ux442ux43dux435ux43dux438ux439-ux432ux43eux437ux434ux443ux445ux430-ux438-ux438ux445-ux432ux43bux438ux44fux43dux438ux435-ux43dux430-ux432ux438ux434ux438ux43cux43eux441ux442ux44c}{%
\subsubsection{2.2. Типы Помутнений Воздуха и их Влияние на
Видимость}\label{ux442ux438ux43fux44b-ux43fux43eux43cux443ux442ux43dux435ux43dux438ux439-ux432ux43eux437ux434ux443ux445ux430-ux438-ux438ux445-ux432ux43bux438ux44fux43dux438ux435-ux43dux430-ux432ux438ux434ux438ux43cux43eux441ux442ux44c}}

Помутнения воздуха классифицируются по степени ухудшения видимости:

\begin{itemize}
\tightlist
\item
  \textbf{Сухое опалесцирующее помутнение}: Вызывает голубоватую окраску
  отдаленных предметов и оранжевые тона в проходящих лучах солнца.
  Горизонтальная видимость при этом остается более 10 км.
\item
  \textbf{Мгла}: Сероватая или беловатая пелена, ухудшающая видимость до
  значений менее 10 км. Часто наблюдается при относительной влажности
  менее 50\%.

  \begin{itemize}
  \tightlist
  \item
    \textbf{Шкала интенсивности мглы}:

    \begin{itemize}
    \tightlist
    \item
      Слабая мгла: 4---10 км
    \item
      Умеренная мгла: 2---4 км
    \item
      Сильная мгла: 1---2 км
    \item
      Очень сильная мгла (сухой туман): \textless{} 1 км
    \end{itemize}
  \end{itemize}
\item
  \textbf{Дымка}: Помутнение, связанное с взвешенными в воздухе
  капельками воды или кристаллами льда. Дальность видимости при дымке
  заключена между 1 и 10 км.

  \begin{itemize}
  \tightlist
  \item
    \textbf{Шкала интенсивности дымки}:

    \begin{itemize}
    \tightlist
    \item
      Слабая дымка: 4---10 км
    \item
      Умеренная дымка: 2---4 км
    \item
      Сильная дымка: 1---2 км
    \end{itemize}
  \item
    Повторяемость дымок значительно выше, чем повторяемость туманов. В
    естественных условиях дымка образуется в 4-5 раз чаще, чем туман. В
    условиях большого города это отношение может увеличиваться до 7-9,
    поскольку загрязнение воздуха способствует росту числа дымок.
  \end{itemize}
\item
  \textbf{Туман}: Помутнение воздуха в приземном слое, вызванное
  взвешенными каплями воды, ледяными кристаллами или их смесью, при
  горизонтальной видимости менее 1 км хотя бы в одном направлении.

  \begin{itemize}
  \tightlist
  \item
    \textbf{Шкала интенсивности тумана}:

    \begin{itemize}
    \tightlist
    \item
      Слабый туман: 500---1000 м
    \item
      Умеренный туман: 200---500 м
    \item
      Сильный туман: 50---200 м
    \item
      Очень сильный туман: \textless{} 50 м
    \end{itemize}
  \item
    Туман при видимости 100 м и менее относится к особо опасным
    метеорологическим явлениям, способным останавливать движение
    транспорта.
  \item
    \textbf{Дальность видимости в тумане} зависит от размеров взвешенных
    частиц и их числа в единице объема, то есть от водности тумана.
  \item
    Минимальная видимость в тумане в теплое полугодие обычно отмечается
    около восхода солнца, а в холодное полугодие при снежном покрове --
    через 1-2 часа после восхода солнца.
  \item
    \textbf{Городские туманы} часто наблюдаются в больших индустриальных
    городах даже при их отсутствии в окрестностях. Это связано с
    измененным режимом метеоэлементов и наличием дополнительных
    антропогенных источников водяного пара и дыма.
  \end{itemize}
\end{itemize}

\hypertarget{ux432ux43bux438ux44fux43dux438ux435-ux43eux441ux430ux434ux43aux43eux432-ux43dux430-ux432ux438ux434ux438ux43cux43eux441ux442ux44c}{%
\subsubsection{2.3. Влияние Осадков на
Видимость}\label{ux432ux43bux438ux44fux43dux438ux435-ux43eux441ux430ux434ux43aux43eux432-ux43dux430-ux432ux438ux434ux438ux43cux43eux441ux442ux44c}}

\begin{itemize}
\tightlist
\item
  \textbf{Низкая облачность}: При относительной влажности более 90\%
  (или T-Td \textless{} 1°C) и скорости ветра менее 5 м/с наблюдается
  связь между высотой нижней границы облаков и видимостью:

  \begin{itemize}
  \tightlist
  \item
    Высота облаков 100-200 м: видимость \textgreater4 км.
  \item
    Высота облаков 60-100 м: видимость 1.5-4 км.
  \item
    Высота облаков 30-60 м: видимость \textless1.5 км.
  \item
    При снежном покрове видимость более 1 км отмечается при ветре 3-4
    м/с, а более 1.5 км --- при ветре 5-6 м/с. Видимость резко
    ухудшается при приближении к нижней границе облаков.
  \end{itemize}
\item
  \textbf{Дождь}: Видимость зависит от интенсивности дождя. В целом, на
  территории СССР дожди редко ухудшают видимость до 2 км и еще реже до 1
  км.
\item
  \textbf{Снегопад и метель}: Видимость ухудшается с усилением ветра и
  возрастанием интенсивности снегопада. В зависимости от вида метели,
  видимость может значительно варьироваться:

  \begin{itemize}
  \tightlist
  \item
    Поземок: средняя видимость 3-4 км.
  \item
    Низовая метель: средняя видимость 1-2 км.
  \item
    Общая метель: средняя видимость \textless1 км.
  \end{itemize}
\item
  \textbf{Пыльная (песчаная) буря}: Видимость сильно колеблется, зависит
  от состояния почвы и скорости ветра, и имеет наименьшее значение в
  начале явления. В дневные часы видимость может ухудшаться до 500 м и
  менее, обычно не более чем на 1 час.
\end{itemize}

\hypertarget{ux432ux43bux438ux44fux43dux438ux435-ux43cux435ux441ux442ux43dux44bux445-ux443ux441ux43bux43eux432ux438ux439-ux438-ux432ux440ux435ux43cux435ux43dux438-ux441ux443ux442ux43eux43a}{%
\subsubsection{2.4. Влияние Местных Условий и Времени
Суток}\label{ux432ux43bux438ux44fux43dux438ux435-ux43cux435ux441ux442ux43dux44bux445-ux443ux441ux43bux43eux432ux438ux439-ux438-ux432ux440ux435ux43cux435ux43dux438-ux441ux443ux442ux43eux43a}}

\begin{itemize}
\tightlist
\item
  \textbf{Местные особенности}: Топографические условия, такие как
  заболоченные низины или близость промышленных центров, могут
  существенно влиять на дальность видимости, создавая локальные туманы
  или дымки.
\item
  \textbf{Суточный ход}: Определяется преимущественно суточным ходом
  относительной влажности в приземном слое. Увеличение относительной
  влажности при ночном охлаждении приводит к ухудшению видимости, вплоть
  до образования дымки и тумана. В сумерки наблюдается быстрое
  размывание контуров предметов из-за резкого уменьшения освещенности.
\item
  \textbf{Высота обнаружения ВПП}: При посадке самолета, высота
  обнаружения взлетно-посадочной полосы в тумане и осадках связана с
  вертикальной видимостью. В адвективном тумане (плотность которого
  возрастает с высотой), высота обнаружения ВПП примерно вдвое больше
  вертикальной видимости, тогда как в радиационном тумане (плотность
  которого уменьшается с высотой) и при осадках они примерно равны.
\end{itemize}

ewpage

\hypertarget{ux43fux43eux43bux435ux442ux43dux430ux44f-ux438-ux43fux43eux441ux430ux434ux43eux447ux43dux430ux44f-ux432ux438ux434ux438ux43cux43eux441ux442ux44c}{%
\section{Полетная и посадочная
видимость}\label{ux43fux43eux43bux435ux442ux43dux430ux44f-ux438-ux43fux43eux441ux430ux434ux43eux447ux43dux430ux44f-ux432ux438ux434ux438ux43cux43eux441ux442ux44c}}

В метеорологической практике, особенно в контексте авиационного
обслуживания, понятие видимости является одним из ключевых
метеорологических элементов, определяющих сложность погодных условий для
взлета, посадки и полета воздушных судов. Хотя источники не выделяют
термины ``полетная видимость'' и ``посадочная видимость'' как строгие
метеорологические дефиниции, они описывают соответствующие концепции и
их количественные характеристики.

\hypertarget{ux43cux435ux442ux435ux43eux440ux43eux43bux43eux433ux438ux447ux435ux441ux43aux430ux44f-ux434ux430ux43bux44cux43dux43eux441ux442ux44c-ux432ux438ux434ux438ux43cux43eux441ux442ux438-ux43cux434ux432}{%
\subsection{Метеорологическая дальность видимости
(МДВ)}\label{ux43cux435ux442ux435ux43eux440ux43eux43bux43eux433ux438ux447ux435ux441ux43aux430ux44f-ux434ux430ux43bux44cux43dux43eux441ux442ux44c-ux432ux438ux434ux438ux43cux43eux441ux442ux438-ux43cux434ux432}}

В основе всех видов видимости лежит понятие метеорологической дальности
видимости (МДВ). Это наибольшее расстояние, на котором абсолютно черный
предмет, имеющий угловые размеры более 20', еще различим днем на фоне
неба у горизонта. Ночью дальность видимости определяется по расстоянию
до наиболее удаленного видимого точечного источника света, сила света
которого известна. Визуальная дальность видимости, определяемая по
реальным объектам, как правило, меньше метеорологической из-за
несоблюдения идеализированных условий.

Метеорологическая дальность видимости используется для классификации
различных явлений:

\begin{itemize}
\tightlist
\item
  \textbf{Туманы}: МДВ менее 1 км. Могут быть сильными (МДВ \textless{}
  50 м), умеренными (50 м \textless{} МДВ \textless{} 500 м) и слабыми
  (500 м \textless{} МДВ \textless{} 1 км).
\item
  \textbf{Дымки}: МДВ от 1 км до 10 км. Могут быть умеренными (1 км
  \textless{} МДВ \textless{} 5 км) и слабыми (5 км \textless{} МДВ
  \textless{} 10 км).
\item
  \textbf{Мгла}: сероватая или беловатая пелена, ухудшающая видимость до
  значений менее 10 км, при этом относительная влажность часто менее
  50\%.
\item
  \textbf{Сухое опалесцирующее помутнение}: вызывает голубоватую окраску
  отдаленных предметов, оранжевые тона в проходящих лучах солнца, когда
  рассеивающие частицы очень малы или их количество невелико.
  Горизонтальная видимость при этом остается более 10 км.
\end{itemize}

\hypertarget{ux43fux43eux43bux435ux442ux43dux430ux44f-ux432ux438ux434ux438ux43cux43eux441ux442ux44c}{%
\subsection{Полетная
видимость}\label{ux43fux43eux43bux435ux442ux43dux430ux44f-ux432ux438ux434ux438ux43cux43eux441ux442ux44c}}

Полетная видимость, в общем смысле, соответствует горизонтальной
метеорологической дальности видимости. Она характеризует условия
видимости на маршруте полета и на аэродроме в целом. На
метеорологических картах зоны туманов и осадков, влияющие на видимость,
обозначаются специальными цветами или условными знаками.

Основные метеорологические явления, снижающие полетную видимость,
включают:

\begin{itemize}
\tightlist
\item
  \textbf{Туманы и дымки}: Являются следствием охлаждения или испарения
  и приводят к резкому ухудшению видимости.
\item
  \textbf{Осадки}:

  \begin{itemize}
  \tightlist
  \item
    \textbf{Дождь}: обычно не ухудшает видимость до 2 км, а тем более до
    1 км.
  \item
    \textbf{Снегопад}: видимость ухудшается с увеличением интенсивности
    снегопада и усилением ветра.
  \item
    \textbf{Метель, низовая метель, поземок}: значительно снижают
    видимость, особенно общая метель.
  \end{itemize}
\item
  \textbf{Пыльные (песчаные) бури}: вызывают резкие колебания видимости,
  особенно в дневные часы, и могут снижать ее до 500 м и менее.
\item
  \textbf{Низкая облачность}: при определенных условиях (относительная
  влажность более 90\% и слабый ветер) высота основания облаков тесно
  связана с видимостью.
\end{itemize}

Локальные условия, такие как рельеф, антропогенные факторы (загрязнение
воздуха в городах), и наличие водоемов, также могут влиять на полетную
видимость, создавая зоны ухудшенной или улучшенной видимости. Суточный
ход видимости в приземном слое определяется суточным ходом относительной
влажности.

\hypertarget{ux43fux43eux441ux430ux434ux43eux447ux43dux430ux44f-ux432ux438ux434ux438ux43cux43eux441ux442ux44c}{%
\subsection{Посадочная
видимость}\label{ux43fux43eux441ux430ux434ux43eux447ux43dux430ux44f-ux432ux438ux434ux438ux43cux43eux441ux442ux44c}}

Для целей посадки воздушных судов наиболее критичным является понятие
\textbf{наклонной визуальной дальности видимости взлетно-посадочной
полосы (ВПП)}. Это расстояние, на котором пилот, снижающийся по глиссаде
(с углом наклона к горизонту β = 2--3°), может визуально обнаружить ВПП.

Высота обнаружения ВПП (\(H_{обн}\)) определяется по формуле:
\(H_{обн} = L_{накл} \cdot \sin(\beta)\). Например, при
\(\beta = 2°54'\), \(H_{обн} = 0.05 \cdot L_{накл}\).

Наклонная дальность видимости (\(L_{накл}\)) может быть определена по
данным о горизонтальной видимости в приземном слое (\(S_М\)) и высоте
основания облаков (\(h_1\)). В условиях тумана и осадков, посадочная
видимость также связана с \textbf{вертикальной видимостью}
(\(H_{верт}\)), которая может быть вычислена по видимости в приземном
слое. В адвективном тумане, плотность которого обычно возрастает с
высотой, \(H_{обн}\) зависит от типа тумана.

Прогноз видимости для авиации требует учета всех этих факторов и
особенностей, так как дальность видимости является одной из важнейших
характеристик, определяющих степень сложности условий погоды для взлета,
посадки и полета.

ewpage

\hypertarget{ux432ux438ux434ux438ux43cux43eux441ux442ux44c-ux432-ux43eux431ux43bux430ux43aux430ux445-ux442ux443ux43cux430ux43dux430ux445-ux438-ux43eux441ux430ux434ux43aux430ux445}{%
\section{Видимость в облаках, туманах и
осадках}\label{ux432ux438ux434ux438ux43cux43eux441ux442ux44c-ux432-ux43eux431ux43bux430ux43aux430ux445-ux442ux443ux43cux430ux43dux430ux445-ux438-ux43eux441ux430ux434ux43aux430ux445}}

\hypertarget{ux432ux438ux434ux438ux43cux43eux441ux442ux44c-ux432-ux442ux443ux43cux430ux43dux430ux445}{%
\subsection{Видимость в
туманах}\label{ux432ux438ux434ux438ux43cux43eux441ux442ux44c-ux432-ux442ux443ux43cux430ux43dux430ux445}}

Туман представляет собой помутнение воздуха в приземном слое, вызванное
взвешенными в нем каплями воды, ледяными кристаллами или их смесью, при
горизонтальной метеорологической дальности видимости (МДВ) менее 1 км.
Дымка отличается от тумана тем, что ее МДВ находится в диапазоне от 1 до
10 км.

\hypertarget{ux43aux43bux430ux441ux441ux438ux444ux438ux43aux430ux446ux438ux44f-ux442ux443ux43cux430ux43dux43eux432-1}{%
\subsubsection{Классификация
туманов}\label{ux43aux43bux430ux441ux441ux438ux444ux438ux43aux430ux446ux438ux44f-ux442ux443ux43cux430ux43dux43eux432-1}}

Туманы классифицируются по синоптическим условиям образования
(внутримассовые и фронтальные) и по преобладающим физическим процессам.
Основные типы включают:

\begin{itemize}
\tightlist
\item
  \textbf{Туманы охлаждения}:

  \begin{itemize}
  \tightlist
  \item
    Радиационные: образуются за счет ночного понижения температуры
    приземного слоя из-за радиационного выхолаживания подстилающей
    поверхности, характерны для ясных ночей со слабым ветром, особенно в
    низинах. Могут быть поземными (h2 \textless{} 2 м), низкими (2-10
    м), средними (10-100 м) и высокими (h2 \textgreater{} 100 м). Летом
    преобладают поземные, низкие и средние, зимой -- высокие туманы.
  \item
    Адвективные: возникают при охлаждении относительно теплой и влажной
    воздушной массы, перемещающейся на более холодную подстилающую
    поверхность. Часто связаны со снижением облаков St или Sc. Могут
    наблюдаться в любое время суток, усиливаясь ночью. Наиболее
    благоприятные условия для них - теплые секторы циклонов и
    прилегающие окраины антициклонов.
  \item
    Адвективно-радиационные: сочетают адвективные и радиационные
    факторы, при этом ночное прояснение играет решающую роль.
  \item
    Орографические (туманы горных склонов): образуются за счет
    адиабатического охлаждения влажного воздуха, поднимающегося вдоль
    склона горы.
  \end{itemize}
\item
  \textbf{Туманы испарения}:

  \begin{itemize}
  \tightlist
  \item
    Образуются над водоемами, когда температура поверхности воды
    значительно выше температуры окружающего воздуха (обычно разность
    \textgreater10°С, а относительная влажность воздуха около 70\%).
    Примером являются туманы над незамерзающими арктическими морями или
    быстрыми реками осенью и зимой.
  \item
    Смешения (береговые): возникают на границе двух воздушных масс.
  \item
    Водяная пыль: образуется при сильном разбрызгивании воды (например,
    у водопадов, морского прибоя).
  \end{itemize}
\item
  \textbf{Туманы, связанные с деятельностью человека}:

  \begin{itemize}
  \tightlist
  \item
    Городские: характеризуются дополнительными источниками водяного пара
    и дыма, что способствует значительному ухудшению видимости.
  \item
    Морозные (поселковые печные, аэродромные): возникают при сильных
    морозах и наличии дополнительного источника водяного пара.
  \end{itemize}
\end{itemize}

\hypertarget{ux444ux430ux43aux442ux43eux440ux44b-ux432ux43bux438ux44fux44eux449ux438ux435-ux43dux430-ux432ux438ux434ux438ux43cux43eux441ux442ux44c-ux432-ux442ux443ux43cux430ux43dux435}{%
\subsubsection{Факторы, влияющие на видимость в
тумане}\label{ux444ux430ux43aux442ux43eux440ux44b-ux432ux43bux438ux44fux44eux449ux438ux435-ux43dux430-ux432ux438ux434ux438ux43cux43eux441ux442ux44c-ux432-ux442ux443ux43cux430ux43dux435}}

Дальность видимости в тумане зависит от размеров взвешенных частиц и их
числа в единице объема, то есть от водности тумана. Формула для
видимости (\(L\)) в метрах: \(L = 2.3 \cdot 10^4 \frac{r}{\delta_T}\),
где \(r\) - радиус капель (см), \(\delta_T\) - водность тумана (г/м³).
На рисунке 16.1 показана зависимость видимости от водности тумана и
водности тумана от его температуры. Минимальная видимость в тумане
отмечается, когда температура воздуха близка к температуре насыщения
(Td).

\begin{itemize}
\tightlist
\item
  \textbf{Относительная влажность}: Конденсация начинается, когда
  относительная влажность приближается к 100\%. На гигроскопичных ядрах
  конденсации насыщение может быть достигнуто уже при 70-80\%.
\item
  \textbf{Температура}: Понижение температуры воздуха (уменьшение
  \(E(T)\)) способствует достижению насыщения и росту капель.
\item
  \textbf{Ветер}: Слабый ветер благоприятен для образования тумана. При
  штиле охлаждение не распространяется высоко, что может привести к
  образованию росы или поземного тумана. Сильный ветер рассеивает туман.
\item
  \textbf{Ядра конденсации}: Наличие ядер конденсации необходимо для
  образования капель, но их количество, как правило, достаточно
  повсеместно и не является причиной различий в повторяемости туманов. В
  городах их больше, но это не означает, что туманов в городах больше.
\item
  \textbf{Антропогенные факторы}: В крупных городах выбросы тепла и
  водяного пара, а также изменение характера подстилающей поверхности,
  влияют на условия образования туманов. Часто в городах наблюдается
  уменьшение относительной влажности из-за повышения температуры, что
  приводит к сокращению числа туманов по сравнению с окрестностями.
\end{itemize}

\hypertarget{ux438ux43dux442ux435ux43dux441ux438ux432ux43dux43eux441ux442ux44c-ux442ux443ux43cux430ux43dux43eux432-ux438-ux434ux44bux43cux43eux43a}{%
\subsubsection{Интенсивность туманов и
дымок}\label{ux438ux43dux442ux435ux43dux441ux438ux432ux43dux43eux441ux442ux44c-ux442ux443ux43cux430ux43dux43eux432-ux438-ux434ux44bux43cux43eux43a}}

Туманы делятся на сильные (МДВ \textless{} 50 м), умеренные (50 м
\textless{} МДВ \textless{} 500 м) и слабые (500 м \textless{} МДВ
\textless{} 1 км). Дымки подразделяются на умеренные (1 км \textless{}
МДВ \textless{} 5 км) и слабые (5 км \textless{} МДВ \textless{} 10 км).

\hypertarget{ux441ux443ux442ux43eux447ux43dux44bux435-ux43aux43eux43bux435ux431ux430ux43dux438ux44f-ux432ux438ux434ux438ux43cux43eux441ux442ux438}{%
\subsubsection{Суточные колебания
видимости}\label{ux441ux443ux442ux43eux447ux43dux44bux435-ux43aux43eux43bux435ux431ux430ux43dux438ux44f-ux432ux438ux434ux438ux43cux43eux441ux442ux438}}

Суточный ход видимости определяется преимущественно суточным ходом
относительной влажности в приземном слое. Ночное охлаждение приземного
слоя приводит к ухудшению видимости, вплоть до образования дымки и
тумана. Рассеяние радиационного тумана обычно происходит через 1-2 часа
после восхода солнца летом, но может сохраняться дольше осенью и зимой.

\hypertarget{ux432ux438ux434ux438ux43cux43eux441ux442ux44c-ux432-ux43eux431ux43bux430ux43aux430ux445}{%
\subsection{Видимость в
облаках}\label{ux432ux438ux434ux438ux43cux43eux441ux442ux44c-ux432-ux43eux431ux43bux430ux43aux430ux445}}

Облака -- это системы капель воды, кристаллов льда или их смесей,
взвешенные в атмосфере на некоторой высоте. Видимость внутри облаков, по
определению, низкая, так как облака состоят из мельчайших частиц,
рассеивающих свет.

\hypertarget{ux43eux431ux440ux430ux437ux43eux432ux430ux43dux438ux435-ux43eux431ux43bux430ux43aux43eux432-1}{%
\subsubsection{Образование
облаков}\label{ux43eux431ux440ux430ux437ux43eux432ux430ux43dux438ux435-ux43eux431ux43bux430ux43aux43eux432-1}}

Основным процессом, приводящим к образованию облаков, является
охлаждение воздушной частицы при подъеме. Когда воздух поднимается
адиабатически, давление в нем падает, он расширяется и его температура
уменьшается.

\begin{itemize}
\tightlist
\item
  \textbf{Уровень конденсации}: После достижения состояния насыщения на
  некотором уровне, называемом уровнем конденсации, дальнейшее понижение
  температуры приводит к конденсации водяного пара на ядрах, образуя
  облако.
\item
  \textbf{Вертикальные движения}: Восходящие движения воздуха
  (характерные для циклонов и ложбин) вызывают охлаждение, увеличение
  относительной влажности и, как следствие, образование облаков.
  Нисходящие движения (в антициклонах и гребнях) приводят к нагреванию
  воздуха, уменьшению влажности и рассеянию облаков.
\item
  \textbf{Бароклинность и адвекция}: Горизонтальные контрасты
  температуры и влажности, или геострофическая адвекция виртуальной
  температуры, тесно связаны с образованием синоптических вихрей и
  крупномасштабных вертикальных движений, которые, в свою очередь,
  определяют формирование облаков. Адвекция холода и сухого воздуха
  способствует углублению циклонов и образованию облаков значительной
  водности при смешении с теплым воздухом.
\item
  \textbf{Турбулентность}: Турбулентный перенос тепла и влаги от
  поверхности в атмосферу играет важную роль в образовании туманов и
  облаков. Атмосферные движения, как правило, носят турбулентный
  характер.
\item
  \textbf{Тепло конденсации}: Выделяющееся тепло конденсации влияет на
  сохранение разности температур и адвекцию холода, поддерживая развитие
  вихрей и формирование фронтальных зон.
\end{itemize}

\hypertarget{ux442ux438ux43fux44b-ux43eux431ux43bux430ux43aux43eux432-ux438-ux438ux445-ux432ux438ux434ux438ux43cux43eux441ux442ux44cux43eux441ux430ux434ux43aux438}{%
\subsubsection{Типы облаков и их
видимость/осадки}\label{ux442ux438ux43fux44b-ux43eux431ux43bux430ux43aux43eux432-ux438-ux438ux445-ux432ux438ux434ux438ux43cux43eux441ux442ux44cux43eux441ux430ux434ux43aux438}}

\begin{itemize}
\tightlist
\item
  \textbf{Слоистообразные облака (Ns, As, Cs)}: Формируются в областях
  пониженного давления под влиянием восходящих движений синоптического
  масштаба. Из слоисто-дождевых облаков (Ns) выпадают обложные осадки
  умеренной интенсивности, а система Ns-As-Cs может достигать
  горизонтальной протяженности 100-1000 км и вертикальной 10 км. В таких
  облаках видимость сильно ограничена.
\item
  \textbf{Кучевые облака (Cu, Cb)}: Образуются в условиях неустойчивой
  стратификации. Кучево-дождевые облака (Cb) связаны с ливневыми
  осадками, грозами и шквалами. Видимость внутри них крайне низкая.
  Слаборазвитые кучевые облака (Cu hum, Cu med) могут образовываться в
  антициклонах при сухонеустойчивой стратификации в приземном слое.
\item
  \textbf{Облака нижнего яруса (St, Sc)}: Образуются чаще всего при
  адвекции теплого и влажного воздуха над относительно холодной
  подстилающей поверхностью, а также при наличии температурных
  антициклонических высотных инверсий. При относительной влажности более
  90\% и скорости ветра менее 3 м/с высота основания облаков St/Sc
  связана с видимостью.

  \begin{itemize}
  \tightlist
  \item
    Видимость \textgreater{} 4 км: h1 = 100-200 м.
  \item
    Видимость 1.5-4 км: h1 = 60-100 м.
  \item
    Видимость \textless{} 1.5 км: h1 = 30-60 м.
  \end{itemize}
\end{itemize}

\hypertarget{ux432ux43bux438ux44fux43dux438ux435-ux43eux431ux43bux430ux447ux43dux43eux441ux442ux438-ux43dux430-ux43cux435ux442ux435ux43eux440ux43eux43bux43eux433ux438ux447ux435ux441ux43aux438ux435-ux432ux435ux43bux438ux447ux438ux43dux44b}{%
\subsubsection{Влияние облачности на метеорологические
величины}\label{ux432ux43bux438ux44fux43dux438ux435-ux43eux431ux43bux430ux447ux43dux43eux441ux442ux438-ux43dux430-ux43cux435ux442ux435ux43eux440ux43eux43bux43eux433ux438ux447ux435ux441ux43aux438ux435-ux432ux435ux43bux438ux447ux438ux43dux44b}}

Облака оказывают значительное влияние на радиационный баланс земной
поверхности, изменяя термический и влажностный режим почвы и атмосферы.
Они уменьшают приток солнечной радиации и эффективное излучение Земли,
что приводит к увеличению результирующего притока радиации зимой (когда
баланс отрицателен) и уменьшению летом (когда баланс положителен).

\hypertarget{ux432ux438ux434ux438ux43cux43eux441ux442ux44c-ux432-ux43eux441ux430ux434ux43aux430ux445}{%
\subsection{Видимость в
осадках}\label{ux432ux438ux434ux438ux43cux43eux441ux442ux44c-ux432-ux43eux441ux430ux434ux43aux430ux445}}

Осадки - это любая форма воды, выпадающая из облаков или осаждающаяся на
земной поверхности/предметах в результате конденсации. Видимость в
осадках сильно зависит от их типа и интенсивности.

\hypertarget{ux442ux438ux43fux44b-ux43eux441ux430ux434ux43aux43eux432-ux438-ux438ux445-ux432ux43bux438ux44fux43dux438ux435-ux43dux430-ux432ux438ux434ux438ux43cux43eux441ux442ux44c}{%
\subsubsection{Типы осадков и их влияние на
видимость}\label{ux442ux438ux43fux44b-ux43eux441ux430ux434ux43aux43eux432-ux438-ux438ux445-ux432ux43bux438ux44fux43dux438ux435-ux43dux430-ux432ux438ux434ux438ux43cux43eux441ux442ux44c}}

\begin{itemize}
\tightlist
\item
  \textbf{Моросящие осадки}: Мелкие капли дождя или снежные зерна.
  Характерны для теплой воздушной массы и могут быть результатом
  укрупнения частиц тумана. Видимость может быть снижена, но обычно не
  критически.
\item
  \textbf{Обложные осадки}: Капли дождя диаметром более 0.5 мм или
  обычные снежинки/хлопья. Выпадают из облаков Ns, реже As. Могут быть
  слабыми, умеренными и сильными. Длительные и непрерывные, значительно
  ухудшают видимость.
\item
  \textbf{Ливневые осадки}: Крупные капли дождя или хлопья снега,
  снежная крупа или град. Выпадают из облаков Cb, характеризуются
  внезапностью, кратковременностью и высокой интенсивностью.
  Сопровождаются грозами и шквалами. Резко ухудшают видимость.
\item
  \textbf{Снежные бури (Метели)}: Снегопад, сопровождающийся переносом
  снега, поднятого с поверхности. Существенно снижают видимость.
\item
  \textbf{Пыльные (песчаные) бури}: Перенос пыли (песка) ветром, что
  приводит к сильному ухудшению видимости, иногда до 10-20 м.
\end{itemize}

\hypertarget{ux444ux430ux43aux442ux43eux440ux44b-ux432ux43bux438ux44fux44eux449ux438ux435-ux43dux430-ux432ux44bux43fux430ux434ux435ux43dux438ux435-ux43eux441ux430ux434ux43aux43eux432-ux438-ux432ux438ux434ux438ux43cux43eux441ux442ux44c}{%
\subsubsection{Факторы, влияющие на выпадение осадков и
видимость}\label{ux444ux430ux43aux442ux43eux440ux44b-ux432ux43bux438ux44fux44eux449ux438ux435-ux43dux430-ux432ux44bux43fux430ux434ux435ux43dux438ux435-ux43eux441ux430ux434ux43aux43eux432-ux438-ux432ux438ux434ux438ux43cux43eux441ux442ux44c}}

\begin{itemize}
\tightlist
\item
  \textbf{Микрофизическое строение облака}: Для выпадения осадков
  необходимо наличие в облаке трех фаз воды (пар, капли, ледяные
  элементы), особенно в умеренных и высоких широтах.
\item
  \textbf{Вертикальная протяженность облака}: Большая вертикальная
  протяженность и интенсивные вертикальные движения внутри облака
  способствуют быстрому слиянию частиц (коагуляции) и выпадению осадков.
\item
  \textbf{Подоблачный слой}: Если осадки испаряются, не достигая
  поверхности земли (например, из-за сухости нижнего слоя воздуха),
  видимость не ухудшается. Толщина подоблачного слоя и дефицит влажности
  в нем определяют, достигнет ли дождь поверхности земли.
\item
  \textbf{Температура}: Осадки усиливаются в ночные часы в результате
  дополнительного радиационного охлаждения верхней границы облаков.
  Ливневые осадки над сушей выпадают преимущественно днем и вечером,
  когда конвективные движения наиболее интенсивны.
\item
  \textbf{Интенсивность осадков}: Видимость при снегопаде уменьшается с
  увеличением его интенсивности и скорости ветра.
\end{itemize}

\hypertarget{ux43eux431ux449ux438ux435-ux43fux43eux43bux43eux436ux435ux43dux438ux44f-1}{%
\subsubsection{Общие
положения}\label{ux43eux431ux449ux438ux435-ux43fux43eux43bux43eux436ux435ux43dux438ux44f-1}}

При прогнозировании видимости в облаках, туманах и осадках учитываются
следующие факторы: перемещение (адвекция) и эволюция (трансформация)
воздушных масс, их влажность и устойчивость, вертикальные движения, а
также влияние местных факторов (таких как рельеф и подстилающая
поверхность) и суточный ход метеорологических элементов. Информация,
полученная со спутников и радиолокационных станций, играет важную роль в
оперативном прогнозировании.

ewpage

Как профессиональные метеорологи, давайте рассмотрим причины рефракции
света в атмосфере, ее виды и основные оптические явления, которые ею
обусловлены.

\hypertarget{ux43fux440ux438ux447ux438ux43dux44b-ux440ux435ux444ux440ux430ux43aux446ux438ux438-ux441ux432ux435ux442ux430-ux432-ux430ux442ux43cux43eux441ux444ux435ux440ux435}{%
\section{Причины рефракции света в
атмосфере}\label{ux43fux440ux438ux447ux438ux43dux44b-ux440ux435ux444ux440ux430ux43aux446ux438ux438-ux441ux432ux435ux442ux430-ux432-ux430ux442ux43cux43eux441ux444ux435ux440ux435}}

Рефракция, или преломление, света в атмосфере возникает из-за изменения
показателя преломления воздуха, который, в свою очередь, напрямую
зависит от его плотности. Атмосфера не является оптически однородной
средой; плотность воздуха меняется как по вертикали (из-за градиента
давления и температуры), так и по горизонтали (из-за термической и
барической неоднородности).

По мере прохождения солнечной радиации или света от любого источника
через атмосферу, он попадает во все более плотные слои воздуха по мере
приближения к земной поверхности. В таких условиях показатель
преломления воздуха постепенно увеличивается, что приводит к замедлению
скорости света. Согласно законам оптики, луч света в этом случае
отклоняется в сторону более плотного воздуха. Поскольку плотность
воздуха также зависит от температуры, лучи света при рефракции всегда
отклоняются в сторону более холодного воздуха.

\hypertarget{ux432ux438ux434ux44b-ux440ux435ux444ux440ux430ux43aux446ux438ux438-ux430ux441ux442ux440ux43eux43dux43eux43cux438ux447ux435ux441ux43aux430ux44f-ux438-ux437ux435ux43cux43dux430ux44f}{%
\section{Виды рефракции: Астрономическая и
земная}\label{ux432ux438ux434ux44b-ux440ux435ux444ux440ux430ux43aux446ux438ux438-ux430ux441ux442ux440ux43eux43dux43eux43cux438ux447ux435ux441ux43aux430ux44f-ux438-ux437ux435ux43cux43dux430ux44f}}

\hypertarget{ux430ux441ux442ux440ux43eux43dux43eux43cux438ux447ux435ux441ux43aux430ux44f-ux440ux435ux444ux440ux430ux43aux446ux438ux44f}{%
\subsection{Астрономическая
рефракция}\label{ux430ux441ux442ux440ux43eux43dux43eux43cux438ux447ux435ux441ux43aux430ux44f-ux440ux435ux444ux440ux430ux43aux446ux438ux44f}}

Астрономическая рефракция относится к преломлению света, идущего от
внеземных источников (звезд, планет, Солнца, Луны) при их прохождении
через атмосферу Земли. Поскольку плотность воздуха увеличивается с
уменьшением высоты, лучи света от небесных светил изгибаются к
поверхности Земли. Это приводит к тому, что небесные объекты кажутся
наблюдателю расположенными выше своего истинного геометрического
положения на небе. Эффект астрономической рефракции наиболее выражен,
когда светило находится низко над горизонтом.

\hypertarget{ux437ux435ux43cux43dux430ux44f-ux438ux43bux438-ux43dux430ux437ux435ux43cux43dux430ux44f-ux440ux435ux444ux440ux430ux43aux446ux438ux44f}{%
\subsection{Земная (или наземная)
рефракция}\label{ux437ux435ux43cux43dux430ux44f-ux438ux43bux438-ux43dux430ux437ux435ux43cux43dux430ux44f-ux440ux435ux444ux440ux430ux43aux446ux438ux44f}}

Земная рефракция --- это преломление света от объектов, расположенных на
самой поверхности Земли или в ее атмосфере. Она также обусловлена
изменениями плотности воздуха, но в данном случае эти изменения часто
бывают связаны с локальными температурными градиентами, особенно в
приземном слое. Например, над сильно нагретой или охлажденной
поверхностью могут возникать значительные неоднородности плотности,
приводящие к изгибанию световых лучей.

\hypertarget{ux44fux432ux43bux435ux43dux438ux44f-ux43eux431ux443ux441ux43bux43eux432ux43bux435ux43dux43dux44bux435-ux440ux435ux444ux440ux430ux43aux446ux438ux435ux439-ux441ux432ux435ux442ux430}{%
\section{Явления, обусловленные рефракцией
света}\label{ux44fux432ux43bux435ux43dux438ux44f-ux43eux431ux443ux441ux43bux43eux432ux43bux435ux43dux43dux44bux435-ux440ux435ux444ux440ux430ux43aux446ux438ux435ux439-ux441ux432ux435ux442ux430}}

Одним из наиболее ярких и известных явлений, обусловленных рефракцией
света в атмосфере, являются миражи.

\hypertarget{ux43cux438ux440ux430ux436}{%
\subsection{Мираж}\label{ux43cux438ux440ux430ux436}}

Миражи --- это оптические явления, при которых наблюдатель видит
изображение объекта, расположенного не в своем истинном положении, а
смещенным или искаженным. Это происходит из-за значительных вертикальных
градиентов температуры (и, следовательно, плотности и показателя
преломления) в приземном слое воздуха.

\hypertarget{ux432ux435ux440ux445ux43dux438ux439-ux43cux438ux440ux430ux436}{%
\subsubsection{Верхний
мираж}\label{ux432ux435ux440ux445ux43dux438ux439-ux43cux438ux440ux430ux436}}

Верхний мираж возникает, когда воздух у поверхности земли оказывается
холоднее и, следовательно, плотнее, чем вышележащие слои воздуха; это
типично для условий температурной инверсии. В такой ситуации лучи света
от удаленного объекта, проходя через слои воздуха с убывающим
показателем преломления (от холодного к теплому воздуху), отклоняются
вниз. В результате объект кажется поднятым над своим истинным
положением. Верхние миражи часто наблюдаются в полярных районах и над
холодными морями.

Хотя источник явно не описывает ``нижний мираж'', он является
противоположностью верхнего миража. Нижний мираж возникает, когда воздух
у поверхности земли значительно горячее и менее плотный, чем воздух над
ним (например, над сильно нагретым асфальтом или пустыней). В этом
случае лучи света изгибаются вверх от более плотного (холодного) воздуха
к менее плотному (горячему) воздуху, создавая впечатление отражения
объекта на поверхности или его ``затопления'' в кажущейся воде.

Таким образом, рефракция является фундаментальным процессом, который
формирует многие наблюдаемые оптические феномены, изменяя кажущееся
положение и форму объектов в атмосфере.

ewpage

\hypertarget{ux43eux43fux442ux438ux447ux435ux441ux43aux438ux435-ux44fux432ux43bux435ux43dux438ux44f-ux432-ux430ux442ux43cux43eux441ux444ux435ux440ux435}{%
\section{Оптические явления в
атмосфере}\label{ux43eux43fux442ux438ux447ux435ux441ux43aux438ux435-ux44fux432ux43bux435ux43dux438ux44f-ux432-ux430ux442ux43cux43eux441ux444ux435ux440ux435}}

Оптические явления в атмосфере представляют собой результат
взаимодействия солнечного света с атмосферными элементами, такими как
капли воды, кристаллы льда, молекулы газов и аэрозоли. Эти явления
демонстрируют сложные процессы преломления, отражения, рассеяния и
дифракции света.

\hypertarget{ux440ux430ux434ux443ux433ux430}{%
\subsection{Радуга}\label{ux440ux430ux434ux443ux433ux430}}

Радуга является одним из наиболее узнаваемых оптических явлений. Она
наблюдается при определенных условиях: Солнце должно находиться позади
наблюдателя, а дождевые осадки --- впереди него. Механизм образования
радуги связан с разложением белого солнечного света в спектр внутри
каждой отдельной капли дождя. Это происходит за счет полного внутреннего
отражения и дисперсии света внутри капель, что приводит к разделению
световых лучей на составляющие цвета.

\hypertarget{ux433ux430ux43bux43e}{%
\subsection{Гало}\label{ux433ux430ux43bux43e}}

Гало --- это семейство оптических явлений, проявляющихся в виде кругов,
дуг или других световых форм, возникающих вокруг Солнца или Луны.
Образование гало происходит при наличии в атмосфере мельчайших ледяных
кристаллов. Чаще всего такие кристаллы встречаются в облаках верхнего
яруса, таких как перисто-слоистые (Cirrostratus) или перисто-кучевые
(Cirrocumulus) облака. Причиной возникновения гало является преломление
света в этих ледяных кристаллах, при котором белый свет также
разлагается на спектральные цвета.

\hypertarget{ux432ux435ux43dux446ux44b}{%
\subsection{Венцы}\label{ux432ux435ux43dux446ux44b}}

Венцы представляют собой светящиеся круги, окружающие источники света,
такие как Солнце, Луна или даже уличный фонарь, когда наблюдатель
находится в положении, при котором источник света экранирован. Эти круги
состоят из чередующихся темных и светлых полос, часто окрашенных в цвета
радуги, при этом внутреннее кольцо обычно голубоватое, а самое внешнее
--- красноватое. Венцы образуются за счет дифракции света, проходящего
через мелкокапельные области конденсации (например, тонкие облака или
туман). Важно отметить, что дифракция не происходит при прохождении
света через крупные капли, в таком случае формируется радуга.

\hypertarget{ux43cux438ux440ux430ux436ux438}{%
\subsection{Миражи}\label{ux43cux438ux440ux430ux436ux438}}

Мираж --- это оптическое явление, при котором изображение объекта
оказывается смещенным относительно его реального положения. Различают
несколько видов миражей. Например, \textbf{верхний мираж} наблюдается,
когда изображение объекта поднято над его фактическим положением. Это
происходит, когда лучи света, формирующие изображение, отклоняются вниз,
достигая глаз наблюдателя. Такое отклонение света обусловлено
\textbf{инверсией температуры}, при которой воздух у поверхности земли
оказывается холоднее и, следовательно, плотнее, чем вышележащие слои.
Верхние миражи наиболее характерны для полярных районов и холодных
морей. Их форма (сплющенные, вытянутые или инвертированные) зависит от
вертикального профиля плотности воздуха.

\hypertarget{ux434ux440ux443ux433ux438ux435-ux441ux43eux43fux443ux442ux441ux442ux432ux443ux44eux449ux438ux435-ux43eux43fux442ux438ux447ux435ux441ux43aux438ux435-ux44fux432ux43bux435ux43dux438ux44f}{%
\subsection{Другие сопутствующие оптические
явления}\label{ux434ux440ux443ux433ux438ux435-ux441ux43eux43fux443ux442ux441ux442ux432ux443ux44eux449ux438ux435-ux43eux43fux442ux438ux447ux435ux441ux43aux438ux435-ux44fux432ux43bux435ux43dux438ux44f}}

Помимо упомянутых, существуют и другие оптические явления, тесно
связанные с дифракцией света в мелкокапельных средах:

\begin{itemize}
\tightlist
\item
  \textbf{Глории, нимбы, или брокенские призраки} --- это аналогичные
  круги, которые возникают вокруг собственной тени наблюдателя на экране
  (например, в тумане, облаке в горах или на покрытой росой поляне),
  когда наблюдатель находится между источником света и ``экраном''. Как
  и венцы, эти явления также обусловлены прохождением света через
  мелкокапельные области конденсации.
\end{itemize}

Все эти явления демонстрируют, как физические свойства атмосферы и
законы оптики определяют визуальное восприятие света в различных
метеорологических условиях.

ewpage

\hypertarget{ux43fux43eux432ux435ux440ux445ux43dux43eux441ux442ux43dux44bux439-ux437ux430ux440ux44fux434-ux437ux435ux43cux43bux438-ux438-ux438ux43eux43dux438ux437ux430ux446ux438ux44f-ux432ux43eux437ux434ux443ux445ux430}{%
\section{Поверхностный заряд Земли и ионизация
воздуха}\label{ux43fux43eux432ux435ux440ux445ux43dux43eux441ux442ux43dux44bux439-ux437ux430ux440ux44fux434-ux437ux435ux43cux43bux438-ux438-ux438ux43eux43dux438ux437ux430ux446ux438ux44f-ux432ux43eux437ux434ux443ux445ux430}}

Как опытный метеоролог, могу предоставить детальный обзор поверхностного
заряда Земли и процессов ионизации воздуха, опираясь на современные
представления и эмпирические данные, представленные в источниках. Эти
аспекты фундаментальны для понимания атмосферного электричества и его
связи с синоптическими процессами.

\hypertarget{ux43fux43eux432ux435ux440ux445ux43dux43eux441ux442ux43dux44bux439-ux437ux430ux440ux44fux434-ux437ux435ux43cux43bux438}{%
\subsection{Поверхностный заряд
Земли}\label{ux43fux43eux432ux435ux440ux445ux43dux43eux441ux442ux43dux44bux439-ux437ux430ux440ux44fux434-ux437ux435ux43cux43bux438}}

Земная поверхность представляет собой хороший электрический проводник.
Измерения показывают, что вертикальная составляющая напряженности
электрического поля (Ez) в атмосфере почти всегда значительно
превосходит горизонтальные составляющие (Ex,y) и направлена вниз, к
земной поверхности. Это явление трактуется так, как если бы Земля была
заряжена отрицательно. В литературе такая напряженность электрического
поля принято называть положительной.

Плотность поверхностного заряда (σ) в любой точке земной поверхности
непосредственно связана с напряженностью электрического поля (E)
соотношением, вытекающим из закона Гаусса: E = σ/ε₀ε, где ε₀ ---
электрическая постоянная, а ε --- диэлектрическая проницаемость воздуха,
которая для воздуха близка к единице. Таким образом, плотность
поверхностного заряда напрямую зависит от напряженности поля в данной
точке.

Важно отметить, что совокупный отрицательный заряд, переносимый молниями
по всему земному шару, является основным механизмом, заряжающим
поверхность Земли отрицательно. В то же время, расчеты показывают, что
объемные заряды, содержащиеся в столбе атмосферы до высоты 9 км,
компенсируют около 95\% заряда Земли.

Глобальная электрическая цепь (ГЭЦ) представляет собой сложную систему
обмена зарядами между Землей, атмосферой и ионосферой. Земля и
ионосфера, будучи хорошими проводниками, поддерживают практически
постоянный потенциал. Разность потенциалов между ионосферой и землей
является положительной и составляет в среднем около 2.9∙10⁵ В. Замыкание
атмосферных токов происходит через земную кору внизу и через
высокопроводящие слои атмосферы на высоте 50--65 км вверху, где заряд
быстро распределяется.

Обмен зарядами между земной поверхностью и атмосферой осуществляется
различными токами:

\begin{itemize}
\tightlist
\item
  \textbf{Ток проводимости:} Переносится атмосферными ионами и
  составляет около 1 пА/м² в условиях отсутствия активных атмосферных
  процессов в тропосфере. В Кембридже (Англия) его вклад оценивается в
  +60 Кл/км² в год.
\item
  \textbf{Токи осадков:} Заряженные частицы осадков (дождь, снег, град),
  выпадающие из облаков, несут на себе электрический заряд. Несмотря на
  то, что в одном и том же дожде присутствуют капли как положительного,
  так и отрицательного заряда (причем положительно заряженных обычно
  больше), вклад осадков в Кембридже составляет +20 Кл/км² в год.
\item
  \textbf{Токи с острий (коронные разряды):} Эти светящиеся тихие
  разряды возникают на остриях и острых углах предметов при высокой
  напряженности электрического поля (порядка 10⁵ В/м). Они представляют
  собой кистевую форму коронного разряда, где усиление тока и свечение
  являются следствием ионизации и возбуждения газовых молекул. Токи с
  острий обычно составляют несколько микроампер. Примечательно, что
  общий отрицательный заряд, приносимый к земной поверхности токами с
  острий (-100 Кл/км² в год для Кембриджа), значительно (примерно в
  десятки раз) превышает заряды, переносимые молниями на землю.
\item
  \textbf{Токи молний:} Молниевые разряды переносят заряд между
  облаками, частями одного облака или между облаком и землей.
  Большинство молний (80--90\%) переносят отрицательный заряд на землю.
  Средняя величина заряда одной молнии оценивается в 20 Кл, а средняя
  плотность тока молнии --- примерно в -0.67∙10⁻¹² А/м² (что
  соответствует около -20 Кл/км² в год).
\end{itemize}

Таким образом, общий баланс зарядов, переносимых к земной поверхности в
Кембридже, составляет около -40 Кл/км² в год, что подчеркивает
доминирующую роль токов с острий в этом балансе.

\hypertarget{ux438ux43eux43dux438ux437ux430ux446ux438ux44f-ux432ux43eux437ux434ux443ux445ux430}{%
\subsection{Ионизация
воздуха}\label{ux438ux43eux43dux438ux437ux430ux446ux438ux44f-ux432ux43eux437ux434ux443ux445ux430}}

Ионизация воздуха -- это процесс образования атмосферных ионов, которые
представляют собой молекулы воздуха или взвешенные частицы, отличающиеся
от остальных молекул только наличием электрических зарядов. Под
действием электрических сил эти ионы перемещаются вдоль силовых линий
поля со скоростью, называемой подвижностью. Подвижность легких ионов в
нижнем слое атмосферы в естественных условиях составляет порядка
(1--2)∙10⁻⁴ м²/В∙с, причем подвижность отрицательных ионов обычно
несколько больше, чем положительных. Увеличение влажности воздуха
заметно увеличивает подвижность отрицательных ионов.

\textbf{Источники ионизации:}

\begin{itemize}
\tightlist
\item
  \textbf{Радиоактивные вещества земной коры и атмосферы:}
  Радиоактивность горных пород и почвенного воздуха, хоть и мала,
  является источником излучений, проникающих в атмосферу. Над сушей у
  земной поверхности до 80\% ионизации обусловлено радиоактивными
  веществами в атмосфере и частично гамма-излучением земной коры. В слое
  до 500 м этот вклад составляет 75\%. В регионах с повышенной
  естественной радиоактивностью (например, вблизи минеральных
  радиоактивных источников) ионизация атмосферы усиливается.
\item
  \textbf{Космические лучи:} Обладают высокой проникающей способностью,
  достигая даже океанов и земной коры. Интенсивность ионообразования под
  действием космических лучей у поверхности Земли наименьшая у экватора
  и возрастает с увеличением широты до 40°, после чего остается
  постоянной.
\item
  \textbf{Антропогенные радиоактивные вещества:} Значительное значение
  приобрели радиоактивные вещества антропогенного происхождения,
  образующиеся, например, при испытаниях ядерного оружия или авариях на
  АЭС (таких как Чернобыльская или Фукусима).
\item
  \textbf{Другие процессы:} Объемные заряды в атмосфере также могут
  образовываться в результате процессов электризации, таких как трение
  или распыление воды, когда атмосфера наделяется заряженными частицами
  преимущественно одного знака.
\end{itemize}

\textbf{Факторы, влияющие на концентрацию ионов и проводимость:}

\begin{itemize}
\tightlist
\item
  \textbf{Высота:} Значимость рассмотренных ионизаторов меняется с
  высотой. Интенсивность ионообразования сначала убывает до высоты около
  3 км (из-за уменьшения влияния радиоактивности почвы и воздуха), а
  затем начинает возрастать с ростом эффективности действия космических
  лучей. Проводимость атмосферного воздуха, в свою очередь, сначала
  падает до высоты около 3 км (также из-за снижения подвижности ионов в
  нижних слоях, где присутствуют пыль и облака), а затем начинает
  неуклонно расти. В верхней тропосфере проводимость воздуха возрастает
  в десятки раз по сравнению с приземной. Низкая проводимость воздуха в
  нижней тропосфере и ее резкий рост с высотой являются ключевыми
  причинами появления заряда атмосферы.
\item
  \textbf{Аэрозоли и взвешенные частицы:} Ионы могут оседать или
  прилипать к нейтральным более крупным взвешенным частицам, прекращая
  при этом свое существование. Концентрация этих более крупных частиц
  значительно больше концентрации легких ионов. Увеличение содержания
  аэрозолей в воздухе может приводить к появлению вторичных максимумов и
  минимумов в суточном ходе концентрации легких ионов.
\item
  \textbf{Рекомбинация:} Необходимо учитывать воссоединение ионов
  противоположного знака, характеризующихся различной подвижностью
  (например, легких с тяжелыми и средними, или тяжелых между собой).
\item
  \textbf{Влажность:} Влияет на подвижность отрицательных ионов, как уже
  упоминалось.
\item
  \textbf{Электродный эффект:} Непосредственно у земной поверхности
  наблюдается явление электродного эффекта, при котором происходит
  обогащение слоя атмосферы ионами знака, противоположного знаку заряда
  Земли (то есть, обычно положительными ионами). Это приводит к тому,
  что величина униполярности ионов (n+/n-) и проводимости (λ+/λ-)
  становится больше единицы.
\end{itemize}

\textbf{Суточный ход концентрации ионов:} Концентрация легких ионов над
сушей в районах с низким содержанием аэрозольных частиц обычно имеет
максимум в ночные и ранние утренние часы и минимум в предполуденные
часы. Этот утренний максимум связывают с наибольшей чистотой воздуха в
эти часы. На морях суточные изменения концентрации легких ионов весьма
невелики, а для тяжелых ионов суточный ход является обратным.

Таким образом, ионизация воздуха -- это динамичный процесс, зависящий от
комплекса природных и антропогенных факторов, а также от сложных
взаимодействий ионов между собой и с другими атмосферными частицами.

ewpage

\hypertarget{ux438ux43eux43dux44b-ux432-ux430ux442ux43cux43eux441ux444ux435ux440ux435-ux43bux435ux433ux43aux438ux435-ux438-ux442ux44fux436ux435ux43bux44bux435}{%
\section{Ионы в Атмосфере: Легкие и
Тяжелые}\label{ux438ux43eux43dux44b-ux432-ux430ux442ux43cux43eux441ux444ux435ux440ux435-ux43bux435ux433ux43aux438ux435-ux438-ux442ux44fux436ux435ux43bux44bux435}}

Атмосферные ионы играют ключевую роль в электрофизических процессах.
Различают две основные группы ионов: легкие и тяжелые, а также иногда
выделяют промежуточные --- средние ионы.

\hypertarget{ux43bux435ux433ux43aux438ux435-ux438ux43eux43dux44b}{%
\subsection{Легкие
Ионы}\label{ux43bux435ux433ux43aux438ux435-ux438ux43eux43dux44b}}

\hypertarget{ux43eux431ux440ux430ux437ux43eux432ux430ux43dux438ux435}{%
\subsubsection{Образование}\label{ux43eux431ux440ux430ux437ux43eux432ux430ux43dux438ux435}}

Легкие ионы образуются в результате ионизации электрически нейтрального
атома. Когда нейтральный атом теряет один из своих валентных электронов,
он становится положительно заряженным ионом. Выделившийся электрон, в
условиях нормального давления, почти мгновенно (за время менее 10⁻⁶ с)
присоединяется к другому нейтральному атому окружающей среды, формируя
отрицательный ион. Таким образом, легкие ионы всегда образуются попарно
(положительный и отрицательный) и имеют молекулярные размеры.

\hypertarget{ux441ux442ux440ux443ux43aux442ux443ux440ux430}{%
\subsubsection{Структура}\label{ux441ux442ux440ux443ux43aux442ux443ux440ux430}}

Ланжевен отмечал, что положительные и отрицательные легкие ионы обладают
существенно различными структурами.

\begin{itemize}
\tightlist
\item
  \textbf{Положительный легкий ион:} Положительный заряд располагается в
  центре молекулы, принадлежащей иону. Вокруг такого иона может
  расположиться мономолекулярным слоем до 12 нейтральных поляризуемых
  молекул. Таким образом, общее количество молекул для положительных
  ионов составляет максимум тринадцать.
\item
  \textbf{Отрицательный легкий ион:} Максимальное количество молекул для
  отрицательных ионов составляет семь.
\end{itemize}

\hypertarget{ux43fux43eux434ux432ux438ux436ux43dux43eux441ux442ux44c}{%
\subsubsection{Подвижность}\label{ux43fux43eux434ux432ux438ux436ux43dux43eux441ux442ux44c}}

Подвижность ионов является важной характеристикой. Подвижность
положительных ионов, состоящих из большего числа молекул, ниже
подвижности отрицательных ионов. Лабораторные исследования при комнатной
температуре (20°С) в чистом воздухе и при нормальном атмосферном
давлении показывают, что подвижность положительных легких ионов в
среднем равна k₊ = 1,37∙10⁻⁴ м²/В∙с, а отрицательных легких ионов k₋ =
1,89∙10⁻⁴ м²/В∙с. Отношение (k₋ / k₊) зависит от плотности воздуха.

\hypertarget{ux442ux44fux436ux435ux43bux44bux435-ux438ux43eux43dux44b}{%
\subsection{Тяжелые
Ионы}\label{ux442ux44fux436ux435ux43bux44bux435-ux438ux43eux43dux44b}}

\hypertarget{ux43eux431ux440ux430ux437ux43eux432ux430ux43dux438ux435-1}{%
\subsubsection{Образование}\label{ux43eux431ux440ux430ux437ux43eux432ux430ux43dux438ux435-1}}

В атмосфере постоянно присутствуют мельчайшие посторонние частицы
больших размеров, такие как ядра конденсации и другие аэрозольные
частицы. Легкие ионы, присоединяясь к этим частицам, передают им свой
заряд, в результате чего образуются более крупные ионы, называемые
тяжелыми ионами, или ионами Ланжевена.

\hypertarget{ux445ux430ux440ux430ux43aux442ux435ux440ux438ux441ux442ux438ux43aux438}{%
\subsubsection{Характеристики}\label{ux445ux430ux440ux430ux43aux442ux435ux440ux438ux441ux442ux438ux43aux438}}

Тяжелые ионы, как правило, несут один элементарный заряд. Теоретические
расчеты показывают, что они могут иметь более одного элементарного
заряда только в случае, если их радиус превышает 10⁻⁶ см, однако
наблюдения демонстрируют, что число таких ионов ничтожно.

\hypertarget{ux441ux440ux435ux434ux43dux438ux435-ux438ux43eux43dux44b}{%
\subsubsection{Средние
Ионы}\label{ux441ux440ux435ux434ux43dux438ux435-ux438ux43eux43dux44b}}

Иногда в атмосфере обнаруживаются ионы средних размеров, которые
называют средними ионами.

\hypertarget{ux43aux43eux43dux446ux435ux43dux442ux440ux430ux446ux438ux44f-ux438-ux440ux430ux441ux43fux440ux435ux434ux435ux43bux435ux43dux438ux435}{%
\subsection{Концентрация и
Распределение}\label{ux43aux43eux43dux446ux435ux43dux442ux440ux430ux446ux438ux44f-ux438-ux440ux430ux441ux43fux440ux435ux434ux435ux43bux435ux43dux438ux435}}

Концентрация ионов выражается числом ионов, содержащихся в единице
объема (1 м³ или 1 см³). Обычно рассматривают отдельно концентрацию
легких и тяжелых ионов. Концентрация тяжелых ионов (N₊ и N₋) сильно
варьируется, что затрудняет указание среднего значения. У земной
поверхности над сушей концентрация тяжелых ионов значительно больше (в
10-100 раз), чем легких, и может изменяться от нескольких сотен до
нескольких десятков тысяч, а также сильно меняться во времени. В
большинстве мест отношение N₋/N₊ для тяжелых ионов в среднем близко к 2,
увеличиваясь с повышением общего числа частиц и уменьшаясь с понижением
их числа.

Концентрация средних ионов весьма непостоянна. Наблюдения спектра ионов
по подвижностям показывают его чрезвычайную непостоянность и сильную
зависимость от местных условий и состояния атмосферы.

Для легких ионов в большинстве пунктов наблюдений отмечается суточный
ход с основным максимумом в поздние часы. Годовой ход концентрации
легких ионов также достаточно сложен и различен в разных местах: в ряде
пунктов наибольшие значения наблюдаются в теплую половину года, а
наименьшие -- зимой. В целом, годовой ход выражен не особенно резко и в
значительной степени определяется местными условиями и состоянием
атмосферы.

\hypertarget{ux441ux443ux449ux435ux441ux442ux432ux43eux432ux430ux43dux438ux435-ux438-ux432ux43bux438ux44fux43dux438ux435}{%
\subsection{Существование и
Влияние}\label{ux441ux443ux449ux435ux441ux442ux432ux43eux432ux430ux43dux438ux435-ux438-ux432ux43bux438ux44fux43dux438ux435}}

Легкие ионы перестают существовать в результате рекомбинации с тяжелыми
ионами противоположного знака, что приводит к образованию нейтральных
ядер и молекул. Также они прекращают существование при соединении с
незаряженными ядрами, образуя тяжелые ионы. Если существуют многократно
заряженные тяжелые ионы, может происходить процесс соединения с тяжелым
ионом, приводящий к образованию иона с большим зарядом. Тяжелые ионы
прекращают свое существование при соединении с легкими ионами
противоположного знака или при рекомбинации с тяжелыми ионами
противоположного знака, в обоих случаях образуются нейтральные частицы.

Наличие ионов в атмосфере определяет ее проводящую способность или
проводимость. Проводимость атмосферы (λ = e(k₊n₊ + k₋n₋)) зависит не
только от числа ионов, но и в значительно большей степени от их
подвижности. Поэтому периодические изменения проводимости атмосферы
примерно сходны с изменением числа легких ионов, но не вполне идентичны
им. Суточный ход проводимости в большинстве мест средних широт северного
полушария характеризуется максимумом в ранние утренние часы (более
выражен летом) и минимумом в вечерние часы. Утренний максимум
объясняется тем, что в это время атмосфера наиболее чиста от
загрязнений, что способствует увеличению числа легких ионов и их
подвижности.

ewpage

\hypertarget{ux44dux43bux435ux43aux442ux440ux438ux447ux435ux441ux43aux438ux435-ux437ux430ux440ux44fux434ux44b-ux438-ux44dux43bux435ux43aux442ux440ux438ux447ux435ux441ux43aux43eux435-ux43fux43eux43bux435-ux430ux442ux43cux43eux441ux444ux435ux440ux44b}{%
\section{Электрические Заряды и Электрическое Поле
Атмосферы}\label{ux44dux43bux435ux43aux442ux440ux438ux447ux435ux441ux43aux438ux435-ux437ux430ux440ux44fux434ux44b-ux438-ux44dux43bux435ux43aux442ux440ux438ux447ux435ux441ux43aux43eux435-ux43fux43eux43bux435-ux430ux442ux43cux43eux441ux444ux435ux440ux44b}}

Атмосферное электричество как область метеорологии занимается изучением
электрических свойств атмосферы и происходящих в ней электрических
явлений, которые, как известно, имеют существенное значение для многих
метеорологических процессов и являются одними из основных характеристик
атмосферы. Анализ электрического состояния атмосферы имеет непреходящее
практическое значение, поскольку электрические параметры атмосферы
нередко значительно влияют на работу производственных процессов в
современном техническом мире. Кроме того, эволюция электрических
параметров, регистрируемая при непрерывных наблюдениях за значительные
периоды, может служить гибким индикатором антропогенного воздействия на
окружающую среду.

\hypertarget{ux43fux440ux438ux440ux43eux434ux430-ux44dux43bux435ux43aux442ux440ux438ux447ux435ux441ux43aux438ux445-ux437ux430ux440ux44fux434ux43eux432-ux432-ux430ux442ux43cux43eux441ux444ux435ux440ux435}{%
\subsection{Природа Электрических Зарядов в
Атмосфере}\label{ux43fux440ux438ux440ux43eux434ux430-ux44dux43bux435ux43aux442ux440ux438ux447ux435ux441ux43aux438ux445-ux437ux430ux440ux44fux434ux43eux432-ux432-ux430ux442ux43cux43eux441ux444ux435ux440ux435}}

Природа электрических зарядов в атмосфере является одним из ключевых
вопросов, исследуемых в атмосферном электричестве. В атмосфере постоянно
присутствуют ионы -- электрически заряженные частицы, образующиеся в
результате процесса ионизации газов, входящих в состав воздуха.

\hypertarget{ux43eux431ux440ux430ux437ux43eux432ux430ux43dux438ux435-ux438ux43eux43dux43eux432-ux438-ux438ux445-ux441ux432ux43eux439ux441ux442ux432ux430}{%
\subsubsection{Образование Ионов и Их
Свойства}\label{ux43eux431ux440ux430ux437ux43eux432ux430ux43dux438ux435-ux438ux43eux43dux43eux432-ux438-ux438ux445-ux441ux432ux43eux439ux441ux442ux432ux430}}

Ионизация происходит, когда под воздействием внешнего агента --
ионизатора -- молекуле или атому газа сообщается энергия, достаточная
для удаления одного из наружных валентных электронов. Первоначально
электрически нейтральный атом, лишившись электрона, становится
положительно заряженным ионом. Выделившийся электрон почти мгновенно (за
время, меньшее 10⁻⁶ с) присоединяется к нейтральному атому окружающей
среды, образуя отрицательный ион. Таким образом, ионы образуются попарно
(положительный и отрицательный) и имеют молекулярные размеры.

Различают несколько групп ионов по их подвижности: легкие, средние и
тяжелые. Легкие положительные и отрицательные ионы имеют различное
строение. Тяжелые ионы обычно несут один элементарный заряд, и лишь
ничтожное число может иметь больше одного элементарного заряда, если их
радиус превышает 10⁻⁶ см.

Ионы в атмосфере можно рассматривать как примесь, отличающуюся от
остальных молекул воздуха только наличием электрических зарядов. Под
действием электрических сил ионы перемещаются вдоль силовых линий поля
со скоростью, пропорциональной напряженности поля E и зависящей от
природы иона. Скорость дрейфа ионов под действием электрической силы при
напряженности поля, равной единице, называется подвижностью.

Интенсивность ионообразования оценивается числом пар ионов, образующихся
в воздухе за 1 с в 1 м³ при стандартных условиях давления и температуры,
и выражается в единицах I (1/см³·с или 10⁶ /м³·с). В среднем, у земной
поверхности интенсивность ионообразования (I₀) составляет 10I или 10⁷
пар ионов/(м³·с).

\hypertarget{ux43eux441ux43dux43eux432ux43dux44bux435-ux438ux43eux43dux438ux437ux430ux442ux43eux440ux44b-ux432-ux430ux442ux43cux43eux441ux444ux435ux440ux435}{%
\subsubsection{Основные Ионизаторы в
Атмосфере}\label{ux43eux441ux43dux43eux432ux43dux44bux435-ux438ux43eux43dux438ux437ux430ux442ux43eux440ux44b-ux432-ux430ux442ux43cux43eux441ux444ux435ux440ux435}}

Главнейшими ионизаторами для нижних слоев атмосферы являются:

\begin{itemize}
\tightlist
\item
  \textbf{Излучения радиоактивных веществ:} Содержащиеся в земной коре и
  атмосфере. Их активность различается над сушей и океаном, а также
  зависит от антропогенного происхождения (например, аварии на АЭС). Над
  сушей у земной поверхности до 80\% ионизации обусловлено
  радиоактивными излучениями, а в слое до 500 м --- на 75\%.
\item
  \textbf{Космические лучи:} Имеют гораздо большее значение для
  ионизации воздуха во всей толще атмосферы. Они были открыты В. Гессом
  в 1912 г.. Различают галактические, солнечные и метагалактические
  космические лучи. Первичные космические лучи взаимодействуют с атомами
  газов в атмосфере, давая начало вторичным космическим лучам.
  Интенсивность ионообразования под действием космических лучей
  наименьшая у экватора и возрастает с широтой. На высотах более 1,5 км
  над сушей и над океаном, где радиоактивность атмосферы мала,
  космические лучи являются основным и практически единственным
  ионизатором.
\item
  \textbf{Ультрафиолетовые лучи Солнца (с λ \textless{} 0.1 мкм) и
  другие корпускулярные излучения:} Становятся основным ионизатором в
  ионосфере, начиная с тех высот, на которые они проникают.
\end{itemize}

Существуют также локальные источники ионов, но их вклад не является
преобладающим.

\hypertarget{ux43eux431ux44aux435ux43cux43dux44bux435-ux437ux430ux440ux44fux434ux44b}{%
\subsubsection{Объемные
Заряды}\label{ux43eux431ux44aux435ux43cux43dux44bux435-ux437ux430ux440ux44fux434ux44b}}

При неодинаковой концентрации положительных и отрицательных ионов в
данном объеме атмосферы возникает избыточный заряд. Величина этого
избыточного заряда, отнесенного к единице объема, называется плотностью
объемного заряда (ρ). Объемные заряды играют очень большую роль во всех
атмосферно-электрических явлениях. Они могут образовываться в результате
неодинакового перемещения ионов различных знаков под действием
электрического поля, а также при электризации в процессах трения,
распыления воды и других явлений, когда атмосфера заряжается частицами
преимущественно одного знака.

\hypertarget{ux44dux43bux435ux43aux442ux440ux438ux447ux435ux441ux43aux43eux435-ux43fux43eux43bux435-ux430ux442ux43cux43eux441ux444ux435ux440ux44b}{%
\subsection{Электрическое Поле
Атмосферы}\label{ux44dux43bux435ux43aux442ux440ux438ux447ux435ux441ux43aux43eux435-ux43fux43eux43bux435-ux430ux442ux43cux43eux441ux444ux435ux440ux44b}}

В атмосфере всегда существует электрическое поле, являющееся результатом
совокупного действия заряда, находящегося на земной поверхности, и
объемных зарядов, содержащихся в атмосфере. Поверхность Земли является
проводником, поэтому силовые линии электрического поля перпендикулярны
ей. В общем случае напряженность электрического поля может иметь
различное направление и изменяться в широких пределах, но почти всегда
вертикальная составляющая (E\_z) значительно превосходит горизонтальные,
и обычно направлена вниз к земной поверхности, как если бы Земля была
заряжена отрицательно. Такую напряженность поля принято называть
положительной.

\hypertarget{ux44dux43bux435ux43aux442ux440ux438ux447ux435ux441ux43aux43eux435-ux43fux43eux43bux435-ux445ux43eux440ux43eux448ux435ux439-ux43fux43eux433ux43eux434ux44b}{%
\subsubsection{Электрическое Поле ``Хорошей
Погоды''}\label{ux44dux43bux435ux43aux442ux440ux438ux447ux435ux441ux43aux43eux435-ux43fux43eux43bux435-ux445ux43eux440ux43eux448ux435ux439-ux43fux43eux433ux43eux434ux44b}}

Даже при хорошей погоде, когда отсутствуют визуальные проявления
электрической активности, в атмосфере протекают электрические процессы:
текут электрические токи, возникают макрозаряды атмосферы и Земли,
формируются электрические поля. Это поле называется электрическим полем
``хорошей погоды''. Среднее значение градиента потенциала у земной
поверхности составляет порядка 130 В/м. При этом средняя поверхностная
плотность объемного заряда Земли σ равна примерно -1.15×10⁻⁹ Кл/м².
Разработанные расчеты электрических параметров тропосферы в условиях
``хорошей погоды'' показывают, что в толще атмосферы до высоты 9 км
содержатся объемные заряды, более чем на 95\% компенсирующие заряд
Земли. Причина этого явления --- низкая проводимость воздуха в нижней
тропосфере и ее резкий рост с высотой.

\hypertarget{ux441ux432ux44fux437ux44c-ux441-ux43fux440ux43eux432ux43eux434ux438ux43cux43eux441ux442ux44cux44e-ux438-ux442ux43eux43aux43eux43c}{%
\subsubsection{Связь с Проводимостью и
Током}\label{ux441ux432ux44fux437ux44c-ux441-ux43fux440ux43eux432ux43eux434ux438ux43cux43eux441ux442ux44cux44e-ux438-ux442ux43eux43aux43eux43c}}

Напряженность электрического поля (E) атмосферы может быть рассчитана по
измеренному значению плотности вертикального тока проводимости атмосферы
(i) и значению проводимости (λ) по формуле: E = i/λ. Плотность тока
проводимости атмосферы i равна заряду, переносимому атмосферными ионами
через каждый 1 м² поверхности, перпендикулярной направлению поля E, в
единицу времени. При отсутствии активных атмосферных процессов в
тропосфере i составляет около 1 пА/м². Важно отметить, что плотность
тока проводимости атмосферы не меняется с высотой. Следовательно,
изменения напряженности поля соответствуют изменениям проводимости, но
на это накладывается влияние неоднородностей объемных зарядов атмосферы.

\hypertarget{ux432ux430ux440ux438ux430ux446ux438ux438-ux44dux43bux435ux43aux442ux440ux438ux447ux435ux441ux43aux43eux433ux43e-ux43fux43eux43bux44f}{%
\subsubsection{Вариации Электрического
Поля}\label{ux432ux430ux440ux438ux430ux446ux438ux438-ux44dux43bux435ux43aux442ux440ux438ux447ux435ux441ux43aux43eux433ux43e-ux43fux43eux43bux44f}}

Напряженность электрического поля обнаруживает правильные годовые и
суточные колебания.

\begin{itemize}
\tightlist
\item
  \textbf{Годовой ход:} В средних широтах северного полушария годовой
  ход поля простой, с одним максимумом в зимние месяцы и минимумом в
  летние.
\item
  \textbf{Суточный ход:} Над океанами, в полярных районах и в
  континентальных областях, удаленных от источников загрязнения,
  суточная вариация напряженности электрического поля атмосферы (так
  называемая ``унитарная вариация'') представляет собой простую волну с
  максимумом около 19 часов UT и минимумом около 04 UT. Унитарная
  вариация является непосредственным прямым доказательством
  существования единого генератора атмосферного электрического поля. Над
  сушей вблизи земной поверхности суточный ход более сложный, что
  обусловлено влиянием местных факторов.
\end{itemize}

\hypertarget{ux44dux43bux435ux43aux442ux440ux438ux447ux435ux441ux43aux430ux44f-ux43fux440ux43eux432ux43eux434ux438ux43cux43eux441ux442ux44c-ux430ux442ux43cux43eux441ux444ux435ux440ux44b}{%
\subsection{Электрическая Проводимость
Атмосферы}\label{ux44dux43bux435ux43aux442ux440ux438ux447ux435ux441ux43aux430ux44f-ux43fux440ux43eux432ux43eux434ux438ux43cux43eux441ux442ux44c-ux430ux442ux43cux43eux441ux444ux435ux440ux44b}}

Проводимость атмосферы характеризует ее способность проводить
электрический ток и определяется концентрацией и подвижностью
атмосферных ионов. Электрический ток в условиях хорошей погоды
переносится как положительными, так и отрицательными атмосферными
ионами, что приводит к разделению на положительную (λ⁺) и отрицательную
(λ⁻) полярные проводимости. Полная, суммарная проводимость (λ) равна их
сумме. Проводимость атмосферы численно равна плотности тока проводимости
при напряженности электрического поля, равной 1 В/м. В приземном слое
атмосферы при отсутствии активных атмосферных процессов λ составляет
около 10⁻¹⁴ См/м.

Проводимость атмосферы зависит не только от числа ионов, но и в
значительно большей мере от их подвижности. Низкие значения проводимости
и концентрации легких ионов, а также большие значения концентрации
тяжелых ионов наблюдаются во время мглы и туманов. Также существует
тесная связь с дальностью видимости: с ее уменьшением уменьшается
концентрация легких ионов и проводимость, при этом увеличивается число
тяжелых ионов.

\hypertarget{ux44dux43bux435ux43aux442ux440ux438ux447ux435ux441ux442ux432ux43e-ux43eux431ux43bux430ux43aux43eux432}{%
\subsection{Электричество
Облаков}\label{ux44dux43bux435ux43aux442ux440ux438ux447ux435ux441ux442ux432ux43e-ux43eux431ux43bux430ux43aux43eux432}}

Электрические процессы в облаках представляют собой грандиозные и
сложные явления. В понятие ``электрические характеристики облака''
входит плотность объемного заряда, градиент потенциала (напряженность)
электрического поля, электропроводность в облаке и его окрестностях,
спектральная плотность электрических зарядов на частицах облака и
осадков, а также плотность электрического тока, текущего в облаке и
вблизи него.

\hypertarget{ux43dux430ux43fux440ux44fux436ux435ux43dux43dux43eux441ux442ux44c-ux44dux43bux435ux43aux442ux440ux438ux447ux435ux441ux43aux43eux433ux43e-ux43fux43eux43bux44f-ux432-ux43eux431ux43bux430ux43aux430ux445}{%
\subsubsection{Напряженность Электрического Поля в
Облаках}\label{ux43dux430ux43fux440ux44fux436ux435ux43dux43dux43eux441ux442ux44c-ux44dux43bux435ux43aux442ux440ux438ux447ux435ux441ux43aux43eux433ux43e-ux43fux43eux43bux44f-ux432-ux43eux431ux43bux430ux43aux430ux445}}

Напряженность электрического поля в облаках может меняться в очень
широких пределах. Совокупность абсолютных значений напряженности
\textbar E\textbar{} удовлетворительно аппроксимируется логарифмически
нормальным распределением. Результаты существенно различаются для разных
форм облаков. Электрическая активность облаков, характеризуемая средней
величиной абсолютных значений напряженности электрического поля, растет
от одного вида облака к другому примерно в следующей последовательности:
St, Sc, Ac, As, Ns, Cb. Как правило, с увеличением толщины облака
возрастает его электрическая активность. Электрические характеристики
меняются от зимы к лету: зимой средние и максимальные значения градиента
потенциала электрических полей, а также разности потенциалов на границах
облаков меньше, чем летом.

\hypertarget{ux43cux435ux445ux430ux43dux438ux437ux43cux44b-ux44dux43bux435ux43aux442ux440ux438ux437ux430ux446ux438ux438-ux43eux431ux43bux430ux447ux43dux44bux445-ux44dux43bux435ux43cux435ux43dux442ux43eux432-ux43cux438ux43aux440ux43eux44dux43bux435ux43aux442ux440ux438ux437ux430ux446ux438ux44f}{%
\subsubsection{Механизмы Электризации Облачных Элементов
(Микроэлектризация)}\label{ux43cux435ux445ux430ux43dux438ux437ux43cux44b-ux44dux43bux435ux43aux442ux440ux438ux437ux430ux446ux438ux438-ux43eux431ux43bux430ux447ux43dux44bux445-ux44dux43bux435ux43cux435ux43dux442ux43eux432-ux43cux438ux43aux440ux43eux44dux43bux435ux43aux442ux440ux438ux437ux430ux446ux438ux44f}}

Заряд частиц в облаках может меняться в результате взаимодействия с
ионами, взаимодействия частиц между собой или их спонтанного разрушения.

\begin{itemize}
\tightlist
\item
  \textbf{Ионная электризация:} Атмосферные ионы взаимодействуют с
  частицами в облаках, изменяя их заряд. Этот механизм в основном
  сопровождает процесс конденсационного роста частиц и наиболее
  существенен на начальной стадии развития облаков и туманов. Внешнее
  электрическое поле может существенно изменить ход процессов
  электризации частиц в атмосфере. В грозовых облаках эффективность
  ионного заряжения возрастает в 5-10 раз в зависимости от размера
  капель.
\item
  \textbf{Электризация в результате разрушения контакта гидрометеоров:}
  Столкновение и последующее разделение облачных гидрометеоров приводит
  к разделению зарядов. Наиболее мощные из этих процессов включают
  электризацию при замерзании, деформации и раскалывании переохлажденных
  капель, а также при столкновении и разбрызгивании переохлажденных
  капелек на крупной ледяной частице, и при столкновении и отскоке
  ледяных кристалликов от крупной ледяной частицы. Во внешнем
  электрическом поле происходит дополнительная индукционная электризация
  взаимодействующих частиц, когда мелкие частицы при столкновении с
  поляризованной крупной частицей (например, градиной) забирают часть
  поляризованного заряда. Величина разделяющегося заряда пропорциональна
  напряженности внешнего электрического поля.
\item
  \textbf{Электризация при столкновении ледяных кристаллов с крупинкой
  или градиной:} Отскок приводит к их взаимной электризации. Величина и
  знак разделяющегося заряда определяются процессами в квазижидком слое,
  возникающем в зоне контакта. Примеси, меняющие поверхностные свойства
  льда, также влияют на электризацию, изменяя вид зависимости заряжения
  от водности и температуры, а также знак заряда.
\item
  \textbf{Индукционная электризация гидрометеоров:} Во внешнем
  электрическом поле при разрыве контакта взаимодействующих частиц
  происходит разделение заряда, связанное с индукционным механизмом.
\end{itemize}

Наиболее мощные механизмы микроэлектризации ``работают'' только в
облаках, содержащих ледяные гидрометеоры. Чем больше ледяных частиц, тем
интенсивнее электризация.

\hypertarget{ux43eux440ux433ux430ux43dux438ux437ux43eux432ux430ux43dux43dux430ux44f-ux43cux430ux43aux440ux43eux44dux43bux435ux43aux442ux440ux438ux437ux430ux446ux438ux44f-ux43eux431ux43bux430ux43aux430}{%
\subsubsection{Организованная Макроэлектризация
Облака}\label{ux43eux440ux433ux430ux43dux438ux437ux43eux432ux430ux43dux43dux430ux44f-ux43cux430ux43aux440ux43eux44dux43bux435ux43aux442ux440ux438ux437ux430ux446ux438ux44f-ux43eux431ux43bux430ux43aux430}}

Макроэлектризация облака -- это процесс разделения разноименно
заряженных частиц в пространстве, приводящий к преимущественному
накоплению положительных или отрицательных зарядов в больших объемах
облака и формированию электрической структуры облака. Этот процесс
происходит, если в значительном объеме облака накапливаются частицы,
преимущественно несущие заряды того или иного знака. Макроразделение
зарядов может произойти под влиянием силы тяжести, если заряды разных
знаков связаны с гидрометеорами разных размеров и, следовательно, разной
массы. Причиной разделения макрозарядов является преимущественная
положительная или отрицательная микроэлектризация частиц в зависимости
от их размеров и дальнейшее разделение разноименно заряженных частиц в
гравитационном поле.

\hypertarget{ux44dux43bux435ux43aux442ux440ux438ux447ux435ux441ux442ux432ux43e-ux43aux43eux43dux432ux435ux43aux442ux438ux432ux43dux44bux445-ux43eux431ux43bux430ux43aux43eux432}{%
\subsubsection{Электричество Конвективных
Облаков}\label{ux44dux43bux435ux43aux442ux440ux438ux447ux435ux441ux442ux432ux43e-ux43aux43eux43dux432ux435ux43aux442ux438ux432ux43dux44bux445-ux43eux431ux43bux430ux43aux43eux432}}

Грозовое электричество возникает в результате атмосферных процессов,
ведущих к образованию мощных кучево-дождевых облаков (Cb). Электрическая
структура грозового облака определяется его гидродинамической и
микрофизической структурой. На начальной стадии развития конвективных
облаков, когда вершина облака находится выше изотермы -8°C (для СНГ) или
-6°C (для Кубы), и отражаемость по самолетному радиолокатору превышает 0
dBZ, и в переохлажденной части облака начинается процесс кристаллизации,
наблюдается организованная электризация. На этой стадии эквивалентная
зарядовая структура представляет собой диполь с ``минусом'' вверху, и
направление вертикальной компоненты поля над облаком совпадает с полем
``хорошей погоды''. На стадии зрелости и распада облака направление
вертикальной компоненты поля обычно меняется на противоположное. В
грозовых облаках наблюдаются очень высокие значения напряженности
электрического поля: под Cb --- 10⁴ В/м, в основной заряженной части ---
10⁵ В/м, в ячейках избыточного заряда --- 3×10⁵ В/м, а в ``запальных''
микроячейках, где возникают разряды, --- 2×10⁶ В/м. Разряд в облаке
начинается, если напряженность электрического поля в нем составляет
E\_макс ≥ 3 ÷ 4×10⁶ В/м.

\hypertarget{ux43cux43eux43bux43dux438ux438}{%
\subsubsection{Молнии}\label{ux43cux43eux43bux43dux438ux438}}

Молния представляет собой электрический разряд между облаками,
отдельными частями одного облака или между облаком и земной
поверхностью. Линейные молнии, развивающиеся между облаком и землей,
являются основной причиной повреждения наземных объектов. Они переносят
электрические заряды из атмосферы в землю, что в значительной мере
определяет действие глобальной электрической цепи Земля-атмосфера.

\hypertarget{ux433ux43bux43eux431ux430ux43bux44cux43dux430ux44f-ux430ux442ux43cux43eux441ux444ux435ux440ux43dux43e-ux44dux43bux435ux43aux442ux440ux438ux447ux435ux441ux43aux430ux44f-ux446ux435ux43fux44c-ux433ux44dux446}{%
\subsection{Глобальная Атмосферно-Электрическая Цепь
(ГЭЦ)}\label{ux433ux43bux43eux431ux430ux43bux44cux43dux430ux44f-ux430ux442ux43cux43eux441ux444ux435ux440ux43dux43e-ux44dux43bux435ux43aux442ux440ux438ux447ux435ux441ux43aux430ux44f-ux446ux435ux43fux44c-ux433ux44dux446}}

Концепция ГЭЦ является общепризнанной и объясняет происхождение
электрического поля атмосферы. ГЭЦ представляет собой распределенный
токовый контур, образованный проводящими слоями ионосферы, верхнего слоя
океана и земной коры, с грозовыми генераторами в качестве основных
источников электродвижущих сил и невозмущенными областями свободной
атмосферы в качестве зон возвратных токов. Таким образом, ГЭЦ --- это
интегральная система, включающая твердую, жидкую и газоплазменную
геосферные оболочки.

Совокупный заряд, переносимый одновременно молниями всех гроз на земном
шаре, заряжает поверхность Земли отрицательно. Поскольку верхняя часть
большинства грозовых облаков имеет положительный заряд, потенциал
ионосферы также оказывается положительным.

\hypertarget{ux44dux43bux435ux43aux442ux440ux438ux447ux435ux441ux43aux438ux435-ux442ux43eux43aux438-ux432-ux430ux442ux43cux43eux441ux444ux435ux440ux435}{%
\subsubsection{Электрические Токи в
Атмосфере}\label{ux44dux43bux435ux43aux442ux440ux438ux447ux435ux441ux43aux438ux435-ux442ux43eux43aux438-ux432-ux430ux442ux43cux43eux441ux444ux435ux440ux435}}

В атмосфере протекают различные электрические токи:

\begin{enumerate}
\def\labelenumi{\arabic{enumi}.}
\tightlist
\item
  \textbf{Вертикальные электрические токи в атмосфере при отсутствии
  осадков и грозовых явлений:}

  \begin{itemize}
  \tightlist
  \item
    Конвективные токи (i\_конв = wρ, где w --- скорость вертикальных
    конвективных потоков, ρ --- плотность объемных зарядов).
  \item
    Вертикальный ток проводимости (оценивается по формуле i = V/R, где V
    --- разность потенциалов, R --- сопротивление столба атмосферы).
  \end{itemize}
\item
  \textbf{Токи осадков:} Перенос заряженных частиц осадков (дождь, снег,
  град) на земную поверхность.
\item
  \textbf{Токи с острий (тихие разряды):} Возникают, когда напряженность
  электрического поля в атмосфере достигает высоких значений, но
  недостаточных для искрового разряда. Они наблюдаются повсеместно и во
  все сезоны года.
\item
  \textbf{Токи молний:} Разряды молний на земную поверхность и в
  ионосферу.
\item
  \textbf{Токи над грозами:} Диапазон изменения токов над грозами
  составляет 0,5--6,0 А, среднее значение 0,5--1,0 А. Кроме того,
  существуют горизонтальные токи в проводящих слоях (в Земле и в верхних
  слоях атмосферы), а также токи, связанные с переносом объемных зарядов
  воздушными течениями в горизонтальном направлении.
\end{enumerate}

\hypertarget{ux431ux430ux43bux430ux43dux441-ux44dux43bux435ux43aux442ux440ux438ux447ux435ux441ux43aux438ux445-ux442ux43eux43aux43eux432}{%
\subsubsection{Баланс Электрических
Токов}\label{ux431ux430ux43bux430ux43dux441-ux44dux43bux435ux43aux442ux440ux438ux447ux435ux441ux43aux438ux445-ux442ux43eux43aux43eux432}}

Электрическое поле в атмосфере существует непрерывно, что указывает на
процессы, постоянно поддерживающие это поле. Вертикальный ток,
рассчитанный на всю земную поверхность, дает величину порядка 1600 А.
Баланс электрических токов поддерживается взаимодействием различных
процессов: ток проводимости и ток осадков приносят положительный заряд
на Землю в областях ``хорошей погоды'', а токи молний и токи с острий
компенсируют его в областях грозовой активности. Основная роль в
компенсации принадлежит токам с острий. Эти две ветви системы
электрических токов замыкаются внизу токами, текущими в земной коре, а
вверху --- токами, протекающими в высоких хорошо проводящих слоях
атмосферы (на высоте 50--65 км), где проводимость достаточна для
быстрого распределения заряда по всей земной поверхности.

\hypertarget{ux43cux435ux442ux43eux434ux44b-ux438ux437ux43cux435ux440ux435ux43dux438ux44f-ux44dux43bux435ux43aux442ux440ux438ux447ux435ux441ux43aux438ux445-ux445ux430ux440ux430ux43aux442ux435ux440ux438ux441ux442ux438ux43a}{%
\subsection{Методы Измерения Электрических
Характеристик}\label{ux43cux435ux442ux43eux434ux44b-ux438ux437ux43cux435ux440ux435ux43dux438ux44f-ux44dux43bux435ux43aux442ux440ux438ux447ux435ux441ux43aux438ux445-ux445ux430ux440ux430ux43aux442ux435ux440ux438ux441ux442ux438ux43a}}

Для измерения атмосферно-электрических характеристик используются
разнообразные методы и средства.

\begin{itemize}
\tightlist
\item
  \textbf{Электрическая проводимость атмосферы:} Измеряется с помощью
  цилиндрического аспирационного конденсатора.
\item
  \textbf{Потенциал электрического поля:} Принцип измерения основан на
  свойстве металлического тела (коллектора) принимать потенциал той
  точки электрического поля, в которой он находится. Коллектор
  размещается на диэлектрическом стержне на нужной высоте над
  поверхностью земли.
\item
  \textbf{Напряженность электрического поля (ПНП):} Измеряется
  электростатическим флюксметром или генерирующим вольтметром. Прибор
  измеряет индуцированный механическим модулятором заряд, потенциал или
  ток индуцированного заряда на изолированном металлическом электроде.
  Измерение заряда является наиболее привлекательным режимом.
\item
  \textbf{Объемный заряд в атмосфере и облаках:} Измеряется на основе
  измерения напряженности электрического поля, создаваемой этим объемным
  зарядом, например, с помощью цилиндрического ротационного флюксметра.
\item
  \textbf{Заряды на частицах осадков:} Измеряются оптическими методами
  (по ослаблению света, размеру тени) с отслеживанием траектории
  движения частиц в электростатическом поле, либо с помощью индукционных
  колец или коллекторными методами.
\item
  \textbf{Измерения с борта летательных аппаратов:} При использовании
  самолетов и других носителей приборов для измерения электрического
  поля необходимо учитывать искажение поля самим аппаратом, а также его
  собственный электрический заряд. Разработана специальная методика
  определения вектора напряженности поля и заряда самолета на основе
  измерений четырех датчиков, расположенных ортогонально. Коэффициенты
  искажения поля определяются на моделях самолетов или путем численного
  моделирования.
\end{itemize}

Эти методы позволяют получать данные, необходимые для мониторинга и
понимания сложнейших атмосферно-электрических процессов.

ewpage

\hypertarget{ux43fux440ux43eux432ux43eux434ux438ux43cux43eux441ux442ux44c-ux430ux442ux43cux43eux441ux444ux435ux440ux44b}{%
\section{Проводимость
Атмосферы}\label{ux43fux440ux43eux432ux43eux434ux438ux43cux43eux441ux442ux44c-ux430ux442ux43cux43eux441ux444ux435ux440ux44b}}

\hypertarget{ux43eux431ux449ux438ux435-ux43fux43eux43dux44fux442ux438ux44f-ux438-ux438ux441ux442ux43eux440ux438ux447ux435ux441ux43aux438ux439-ux43aux43eux43dux442ux435ux43aux441ux442}{%
\subsection{1. Общие Понятия и Исторический
Контекст}\label{ux43eux431ux449ux438ux435-ux43fux43eux43dux44fux442ux438ux44f-ux438-ux438ux441ux442ux43eux440ux438ux447ux435ux441ux43aux438ux439-ux43aux43eux43dux442ux435ux43aux441ux442}}

Проводимость атмосферы -- это ее способность проводить электрический
ток. Воздух не является идеальным изолятором, и это свойство обусловлено
присутствием в нем ионов. Явление проводимости воздуха и утечки заряда с
наэлектризованного тела впервые было отмечено Кулоном в 1795 году.
Однако правильное объяснение этого явления, связанное с существованием
ионов, было дано лишь в конце XIX столетия.

\hypertarget{ux43cux435ux445ux430ux43dux438ux437ux43c-ux43fux440ux43eux432ux43eux434ux438ux43cux43eux441ux442ux438}{%
\subsection{2. Механизм
Проводимости}\label{ux43cux435ux445ux430ux43dux438ux437ux43c-ux43fux440ux43eux432ux43eux434ux438ux43cux43eux441ux442ux438}}

Проводимость атмосферы определяется концентрацией и подвижностью
атмосферных ионов. Ионы --- это частицы (приблизительно молекулярных
размеров), несущие положительные или отрицательные элементарные заряды.
В электрическом поле напряженностью \(E\) ионы движутся со скоростью
\(kE\), где \(k\) -- подвижность. Поскольку каждый ион несет заряд
\(e\), их упорядоченное движение образует электрический ток.

Плотность тока проводимости атмосферы (\(i\)) выражается как:
\(i = eE(n_+k_+ + n_-k_-)\). Здесь \(n_+\) и \(n_-\) -- концентрации
положительных и отрицательных ионов, а \(k_+\) и \(k_-\) -- их
подвижности.

Введенные полярные проводимости: \(\lambda_+ = n_+k_+e\) и
\(\lambda_- = n_-k_-e\). Суммарная проводимость (\(\lambda\)) равна их
сумме: \(\lambda = \lambda_+ + \lambda_-\). Единица измерения
проводимости -- См/м (Сименс на метр). Численно проводимость равна
плотности тока проводимости при напряженности электрического поля,
равной 1 В/м.

Проводимость атмосферы более чем на 95\% обусловлена легкими ионами,
даже при значительном числе тяжелых ионов, что объясняется их гораздо
большей подвижностью.

\hypertarget{ux438ux441ux442ux43eux447ux43dux438ux43aux438-ux438ux43eux43dux43eux432-ux438-ux444ux430ux43aux442ux43eux440ux44b-ux432ux43bux438ux44fux44eux449ux438ux435-ux43dux430-ux43fux440ux43eux432ux43eux434ux438ux43cux43eux441ux442ux44c}{%
\subsection{3. Источники Ионов и Факторы, Влияющие на
Проводимость}\label{ux438ux441ux442ux43eux447ux43dux438ux43aux438-ux438ux43eux43dux43eux432-ux438-ux444ux430ux43aux442ux43eux440ux44b-ux432ux43bux438ux44fux44eux449ux438ux435-ux43dux430-ux43fux440ux43eux432ux43eux434ux438ux43cux43eux441ux442ux44c}}

Ионы в атмосфере образуются в результате процесса ионизации газов под
воздействием внешних агентов-ионизаторов.

\hypertarget{ux43eux441ux43dux43eux432ux43dux44bux435-ux438ux43eux43dux438ux437ux430ux442ux43eux440ux44b}{%
\subsubsection{3.1. Основные
Ионизаторы}\label{ux43eux441ux43dux43eux432ux43dux44bux435-ux438ux43eux43dux438ux437ux430ux442ux43eux440ux44b}}

\begin{itemize}
\tightlist
\item
  \textbf{Нижние слои атмосферы}:

  \begin{itemize}
  \tightlist
  \item
    \textbf{Излучения радиоактивных веществ}: содержащиеся в земной коре
    и атмосфере. Гамма- и бета-излучения играют заметную роль вблизи
    земной поверхности, но их ионизирующее действие убывает
    экспоненциально с высотой и становится пренебрежимым на высоте
    нескольких сотен метров.
  \item
    \textbf{Космические лучи}: являются гораздо более значимым
    ионизатором для всей толщи атмосферы. Интенсивность космических
    лучей почти неизменна во времени, хотя наблюдаются кратковременные
    вспышки, совпадающие с солнечными вспышками. Космические лучи
    обладают большой проникающей способностью, пронизывая всю атмосферу,
    океаны и земную кору.
  \item
    \textbf{Антропогенные радиоактивные вещества}: образующиеся при
    испытаниях ядерного оружия и авариях на АЭС, также приобрели большое
    значение.
  \end{itemize}
\item
  \textbf{Верхние слои атмосферы (ионосфера)}: Здесь основными
  ионизаторами являются ультрафиолетовое (с длиной волны
  \(\lambda < 0.1\) мкм) и корпускулярное излучения Солнца. На больших
  высотах низкая плотность атмосферы позволяет электронам существовать в
  свободном состоянии длительное время, что обусловливает высокую
  проводимость.
\end{itemize}

\hypertarget{ux438ux43dux442ux435ux43dux441ux438ux432ux43dux43eux441ux442ux44c-ux438ux43eux43dux43eux43eux431ux440ux430ux437ux43eux432ux430ux43dux438ux44f}{%
\subsubsection{3.2. Интенсивность
Ионообразования}\label{ux438ux43dux442ux435ux43dux441ux438ux432ux43dux43eux441ux442ux44c-ux438ux43eux43dux43eux43eux431ux440ux430ux437ux43eux432ux430ux43dux438ux44f}}

Интенсивность ионообразования (\(I\)) оценивается числом пар ионов,
образующихся в 1 м³ воздуха за 1 с при стандартных условиях (1
\(I = 10^6\) пар ионов/(м³·с)). У земной поверхности \(I_0\) варьируется
от 5 до 40 \(I\), в среднем принимают 10 \(I\) (\(10^7\) пар
ионов/(м³·с)). Из них 20\% обусловлено космическими лучами, 35\% ---
радиоактивностью почвы, 45\% --- радиоактивностью воздуха. Над океаном
вдали от берегов ионизация за счет радиоактивных излучений практически
отсутствует.

Значимость ионизаторов меняется с высотой:

\begin{itemize}
\tightlist
\item
  У земной поверхности: 80\% ионизации обусловлено радиоактивными
  веществами в атмосфере и частично \(\gamma\)-излучениями земной коры.
\item
  В слое до 500 м: 75\% ионизации определяется радиоактивными
  излучениями.
\item
  Над сушей выше 1 км (особенно 2-3 км): концентрация радиоактивных
  веществ мала, основным и практически единственным ионизатором являются
  космические лучи.
\end{itemize}

\hypertarget{ux43fux440ux43eux446ux435ux441ux441ux44b-ux438ux441ux447ux435ux437ux43dux43eux432ux435ux43dux438ux44f-ux438ux43eux43dux43eux432}{%
\subsubsection{3.3. Процессы Исчезновения
Ионов}\label{ux43fux440ux43eux446ux435ux441ux441ux44b-ux438ux441ux447ux435ux437ux43dux43eux432ux435ux43dux438ux44f-ux438ux43eux43dux43eux432}}

Концентрация ионов изменяется не только в результате их образования, но
и вследствие следующих процессов:

\begin{enumerate}
\def\labelenumi{\arabic{enumi}.}
\tightlist
\item
  Рекомбинация легких ионов.
\item
  Соединение легких ионов с незаряженными ядрами с последующим
  образованием тяжелых ионов.
\item
  Рекомбинация легких ионов с тяжелыми ионами противоположного знака.
\item
  Соединение тяжелых ионов с легкими ионами того же знака и
  результирующим образованием многократно заряженных ионов.
\item
  Рекомбинация тяжелых ионов противоположного знака.
\end{enumerate}

Продолжительность жизни тяжелых ионов значительно больше, чем легких, и
может достигать часа и более.

\hypertarget{ux432ux43bux438ux44fux43dux438ux435-ux43fux43eux433ux43eux434ux43dux44bux445-ux443ux441ux43bux43eux432ux438ux439}{%
\subsubsection{3.4. Влияние Погодных
Условий}\label{ux432ux43bux438ux44fux43dux438ux435-ux43fux43eux433ux43eux434ux43dux44bux445-ux443ux441ux43bux43eux432ux438ux439}}

Проводимость, концентрация ионов и их подвижность в значительной мере
зависят от условий погоды и испытывают резкие и нерегулярные колебания.
Особенно тесная связь наблюдается со степенью запыленности воздуха, что
в основном определяет связи и с другими метеорологическими элементами.
Низкие значения проводимости и концентрации легких ионов, а также
большие значения концентрации тяжелых ионов наблюдаются во время мглы и
туманов. С уменьшением дальности видимости уменьшается концентрация
легких ионов и проводимость, при этом увеличивается число тяжелых ионов.
В облаках электропроводность воздуха должна быть меньше, чем в свободной
атмосфере, из-за захвата ионов каплями. Измерения показывают уменьшение
проводимости в неплотных слоисто-кучевых и слоистообразных облаках в
3-25 раз по сравнению с чистым воздухом.

\hypertarget{ux438ux437ux43cux435ux43dux435ux43dux438ux435-ux43fux440ux43eux432ux43eux434ux438ux43cux43eux441ux442ux438-ux441-ux432ux44bux441ux43eux442ux43eux439-ux438-ux432-ux433ux43bux43eux431ux430ux43bux44cux43dux43eux439-ux44dux43bux435ux43aux442ux440ux438ux447ux435ux441ux43aux43eux439-ux446ux435ux43fux438}{%
\subsection{4. Изменение Проводимости с Высотой и в Глобальной
Электрической
Цепи}\label{ux438ux437ux43cux435ux43dux435ux43dux438ux435-ux43fux440ux43eux432ux43eux434ux438ux43cux43eux441ux442ux438-ux441-ux432ux44bux441ux43eux442ux43eux439-ux438-ux432-ux433ux43bux43eux431ux430ux43bux44cux43dux43eux439-ux44dux43bux435ux43aux442ux440ux438ux447ux435ux441ux43aux43eux439-ux446ux435ux43fux438}}

Проводимость атмосферы изменяется с высотой в тропосфере. Проводимость
воздуха в верхней тропосфере возрастает в десятки раз по сравнению с
приземной проводимостью. Вертикальный ток проводимости атмосферы, как
правило, не меняется с высотой.

Возникновение глобальной электрической цепи (ГЭЦ) обусловлено низкой
проводимостью воздуха в нижней тропосфере и резким ростом проводимости
воздуха с высотой. Земная поверхность и ионосфера, обладая высокой
проводимостью (на 10-12 порядков выше проводимости воздуха в нижней
тропосфере), играют роль обкладок сферического конденсатора, заряжаемого
грозовыми разрядами.

\hypertarget{ux442ux438ux43fux438ux447ux43dux44bux435-ux437ux43dux430ux447ux435ux43dux438ux44f-ux43fux440ux43eux432ux43eux434ux438ux43cux43eux441ux442ux438-ux43fux43e-ux432ux44bux441ux43eux442ux430ux43c}{%
\subsubsection{4.1. Типичные Значения Проводимости по
Высотам}\label{ux442ux438ux43fux438ux447ux43dux44bux435-ux437ux43dux430ux447ux435ux43dux438ux44f-ux43fux440ux43eux432ux43eux434ux438ux43cux43eux441ux442ux438-ux43fux43e-ux432ux44bux441ux43eux442ux430ux43c}}

\begin{itemize}
\tightlist
\item
  Уровень моря: \textasciitilde10⁻¹⁴ См/м
\item
  Тропопауза: \textasciitilde10⁻¹³ См/м
\item
  Стратопауза: \textasciitilde10⁻¹⁰ См/м
\item
  Ионосфера (педерсеновская проводимость): \textasciitilde10⁻⁴ -- 10⁻⁵
  См/м
\item
  Ионосфера (параллельная проводимость): \textasciitilde10 См/м
\end{itemize}

\hypertarget{ux432ux440ux435ux43cux44f-ux440ux435ux43bux430ux43aux441ux430ux446ux438ux438-ux437ux430ux440ux44fux434ux430}{%
\subsubsection{4.2. Время Релаксации
Заряда}\label{ux432ux440ux435ux43cux44f-ux440ux435ux43bux430ux43aux441ux430ux446ux438ux438-ux437ux430ux440ux44fux434ux430}}

Время релаксации электрического заряда, связанное с проводимостью:

\begin{itemize}
\tightlist
\item
  70 км: 10⁻⁴ с
\item
  18 км: 4 с
\item
  10 м: 5-10 мин
\item
  Проводящая земля: 10⁻⁵ с
\end{itemize}

Если бы электрическое поле атмосферы не поддерживалось непрерывно, оно
бы очень быстро было бы ликвидировано током проводимости. Численно, при
характерной проводимости для хорошей погоды, электрическое поле
уменьшилось бы до 0.01 от начального значения всего за 30 минут.

\hypertarget{ux438ux437ux43cux435ux43dux447ux438ux432ux43eux441ux442ux44c-ux43fux440ux43eux432ux43eux434ux438ux43cux43eux441ux442ux438}{%
\subsection{5. Изменчивость
Проводимости}\label{ux438ux437ux43cux435ux43dux447ux438ux432ux43eux441ux442ux44c-ux43fux440ux43eux432ux43eux434ux438ux43cux43eux441ux442ux438}}

Проводимость атмосферы, как и концентрация ионов, подвержена колебаниям.

\begin{itemize}
\tightlist
\item
  \textbf{Годовой ход}: Выражен не особенно резко и сильно определяется
  местными условиями и состоянием атмосферы. В ряде пунктов наибольшие
  значения наблюдаются в теплую половину года, наименьшие --- зимой.
\item
  \textbf{Суточный ход}: Характерной чертой является утренний максимум,
  который объясняется наименьшей загрязненностью атмосферы в это время,
  что способствует увеличению числа легких ионов и их подвижности. Над
  океанами проводимость в течение суток изменяется весьма мало.
\item
  \textbf{Широтные изменения}: В приземном слое наблюдаются широтные
  изменения \(\lambda\), связанные с зависимостью интенсивности
  космических лучей от широты. Интенсивность ионообразования под
  действием космических лучей у земной поверхности наименьшая у экватора
  (1.5 \(I\)) и возрастает до 1.8 \(I\) к широте 40°, далее оставаясь
  неизменной.
\end{itemize}

\hypertarget{ux43cux435ux442ux43eux434ux44b-ux438ux437ux43cux435ux440ux435ux43dux438ux44f}{%
\subsection{6. Методы
Измерения}\label{ux43cux435ux442ux43eux434ux44b-ux438ux437ux43cux435ux440ux435ux43dux438ux44f}}

Для измерения проводимости атмосферы и числа ионов чаще всего
применяется цилиндрический аспирационный конденсатор или группа из двух
таких конденсаторов. Прибор может работать в режиме измерения
проводимости (ток зависит от напряжения и скорости воздуха) или в режиме
насыщения для измерения концентрации легких ионов (все легкие ионы
осаждаются на обкладки).

Измерения с борта самолета имеют особенности, связанные с зарядом
самолета, который может оказывать сильное воздействие на процесс
измерения. Для минимизации этой ошибки применяются активные компенсаторы
заряда самолета. Также учитывается потеря ионов на входе в прибор из-за
электрического поля самолета и изменения скорости полета, что может
вносить ошибку до 15-20\%.

ewpage

\hypertarget{ux431ux43bux43eux43a-1.-ux444ux438ux437ux438ux43aux430-ux430ux442ux43cux43eux441ux444ux435ux440ux44b-5}{%
\section{Блок 1. «Физика
атмосферы»}\label{ux431ux43bux43eux43a-1.-ux444ux438ux437ux438ux43aux430-ux430ux442ux43cux43eux441ux444ux435ux440ux44b-5}}

\hypertarget{ux442ux435ux43cux430-ux43eux43fux442ux438ux447ux435ux441ux43aux438ux435-ux438-ux44dux43bux435ux43aux442ux440ux438ux447ux435ux441ux43aux438ux435-ux44fux432ux43bux435ux43dux438ux44f-ux432-ux430ux442ux43cux43eux441ux444ux435ux440ux435.-ux430ux43aux443ux441ux442ux438ux43aux430-ux430ux442ux43cux43eux441ux444ux435ux440ux44b}{%
\subsection{1.6. Тема «Оптические и электрические явления в атмосфере.
Акустика
атмосферы»}\label{ux442ux435ux43cux430-ux43eux43fux442ux438ux447ux435ux441ux43aux438ux435-ux438-ux44dux43bux435ux43aux442ux440ux438ux447ux435ux441ux43aux438ux435-ux44fux432ux43bux435ux43dux438ux44f-ux432-ux430ux442ux43cux43eux441ux444ux435ux440ux435.-ux430ux43aux443ux441ux442ux438ux43aux430-ux430ux442ux43cux43eux441ux444ux435ux440ux44b}}

\hypertarget{ux43eux43fux442ux438ux447ux435ux441ux43aux438ux435-ux44fux432ux43bux435ux43dux438ux44f-ux441ux432ux44fux437ux430ux43dux43dux44bux435-ux441-ux440ux430ux441ux441ux435ux44fux43dux438ux435ux43c-ux441ux432ux435ux442ux430-ux446ux432ux435ux442-ux438-ux44fux440ux43aux43eux441ux442ux44c-ux43dux435ux431ux430-ux441ux443ux43cux435ux440ux43aux438-ux437ux430ux440ux44f}{%
\subsubsection{\texorpdfstring{\textbf{Оптические явления, связанные с
рассеянием света (цвет и яркость неба, сумерки,
заря)}}{Оптические явления, связанные с рассеянием света (цвет и яркость неба, сумерки, заря)}}\label{ux43eux43fux442ux438ux447ux435ux441ux43aux438ux435-ux44fux432ux43bux435ux43dux438ux44f-ux441ux432ux44fux437ux430ux43dux43dux44bux435-ux441-ux440ux430ux441ux441ux435ux44fux43dux438ux435ux43c-ux441ux432ux435ux442ux430-ux446ux432ux435ux442-ux438-ux44fux440ux43aux43eux441ux442ux44c-ux43dux435ux431ux430-ux441ux443ux43cux435ux440ux43aux438-ux437ux430ux440ux44f}}

Рассеяние солнечного света частицами в атмосфере является причиной
многих фундаментальных оптических явлений. Тип рассеяния зависит от
соотношения между размером рассеивающей частицы (\(r\)) и длиной волны
света (\(\lambda\)).

\begin{itemize}
\item
  \textbf{Рэлеевское (молекулярное) рассеяние:} Происходит, когда
  \(r \ll \lambda\), то есть на молекулах воздуха. Интенсивность
  рассеяния (\(I_{расс}\)) обратно пропорциональна четвёртой степени
  длины волны: \[I_{расс} \propto \frac{1}{\lambda^4}\] \textbf{Цвет
  неба:} Из-за этой сильной зависимости коротковолновая часть спектра
  (синий и фиолетовый свет, \(\lambda \approx 400-450\) нм) рассеивается
  гораздо эффективнее, чем длинноволновая (красный свет,
  \(\lambda \approx 700\) нм). Днём, когда мы смотрим на небо в стороне
  от Солнца, мы видим этот рассеянный свет, который и придаёт небу
  голубой цвет. \textbf{Цвет Солнца и зари:} Когда Солнце находится
  низко над горизонтом, его лучи проходят значительно больший путь в
  атмосфере. За это время большая часть синего света рассеивается в
  стороны и не доходит до наблюдателя. Прямой солнечный свет, а также
  свет, рассеянный на частицах вблизи горизонта, оказывается обогащённым
  красными и оранжевыми тонами, что и создаёт цвета зари и заката.
\item
  \textbf{Рассеяние Ми:} Происходит, когда \(r \approx \lambda\), то
  есть на аэрозольных частицах (пыль, дымка) и облачных каплях.
  Зависимость интенсивности рассеяния от длины волны слабая. Поэтому при
  рассеянии на таких частицах белый солнечный свет остаётся белым. Этим
  объясняется белый цвет облаков и белёсый, мутный оттенок неба при
  высокой запылённости или влажности.
\item
  \textbf{Сумерки:} Явление постепенного перехода от дня к ночи после
  захода Солнца. Обусловлено рассеянием солнечного света в верхних слоях
  атмосферы, которые всё ещё освещены, в то время как земная поверхность
  уже находится в тени. Различают гражданские, навигационные и
  астрономические сумерки, соответствующие погружению Солнца под
  горизонт на 6°, 12° и 18° соответственно.
\end{itemize}

\hypertarget{ux43cux435ux442ux435ux43eux440ux43eux43bux43eux433ux438ux447ux435ux441ux43aux430ux44f-ux434ux430ux43bux44cux43dux43eux441ux442ux44c-ux432ux438ux434ux438ux43cux43eux441ux442ux438-ux438-ux432ux43bux438ux44fux44eux449ux438ux435-ux43dux430-ux43dux435ux435-ux444ux430ux43aux442ux43eux440ux44b.-ux43fux43eux43bux435ux442ux43dux430ux44f-ux438-ux43fux43eux441ux430ux434ux43eux447ux43dux430ux44f-ux432ux438ux434ux438ux43cux43eux441ux442ux44c}{%
\subsubsection{\texorpdfstring{\textbf{Метеорологическая дальность
видимости и влияющие на нее факторы. Полетная и посадочная
видимость}}{Метеорологическая дальность видимости и влияющие на нее факторы. Полетная и посадочная видимость}}\label{ux43cux435ux442ux435ux43eux440ux43eux43bux43eux433ux438ux447ux435ux441ux43aux430ux44f-ux434ux430ux43bux44cux43dux43eux441ux442ux44c-ux432ux438ux434ux438ux43cux43eux441ux442ux438-ux438-ux432ux43bux438ux44fux44eux449ux438ux435-ux43dux430-ux43dux435ux435-ux444ux430ux43aux442ux43eux440ux44b.-ux43fux43eux43bux435ux442ux43dux430ux44f-ux438-ux43fux43eux441ux430ux434ux43eux447ux43dux430ux44f-ux432ux438ux434ux438ux43cux43eux441ux442ux44c}}

\textbf{Метеорологическая дальность видимости (МДВ)} --- это расстояние,
на котором контраст между объектом и фоном уменьшается до порогового
значения (обычно 2\%). Она характеризует прозрачность атмосферы.

\begin{itemize}
\item
  \textbf{Закон Бугера-Ламберта-Бера:} Ослабление света в атмосфере
  описывается экспоненциальным законом. На его основе получена
  \textbf{формула Кошмидера} для МДВ (\(L_v\)):
  \[L_v = \frac{-\ln(\epsilon)}{\sigma_{ext}} \approx \frac{3.912}{\sigma_{ext}}\]
  где \(\epsilon=0.02\) --- пороговый контраст, а \(\sigma_{ext}\) ---
  \textbf{коэффициент ослабления (экстинкции)}, который является суммой
  коэффициентов поглощения и рассеяния.
\item
  \textbf{Факторы, влияющие на видимость:}

  \begin{itemize}
  \tightlist
  \item
    \textbf{Аэрозоли:} Основной фактор ухудшения видимости. Дымка (haze)
    возникает при высокой концентрации мелких аэрозольных частиц.
  \item
    \textbf{Туман и облака:} Состоят из капель воды или кристаллов льда,
    которые эффективно рассеивают свет, резко снижая видимость до
    десятков или сотен метров.
  \item
    \textbf{Осадки:} Дождь и особенно снег уменьшают прозрачность
    атмосферы.
  \item
    \textbf{Загрязнение воздуха:} Промышленные выбросы и смог
    значительно ухудшают видимость в городах.
  \end{itemize}
\item
  \textbf{Полетная и посадочная видимость (RVR - Runway Visual Range):}
  Специализированные понятия в авиационной метеорологии. RVR --- это
  максимальное расстояние, на котором пилот может видеть разметку
  взлётно-посадочной полосы. Она измеряется инструментально
  (трансмиссометрами) и является критическим параметром для принятия
  решения о возможности взлёта и посадки.
\end{itemize}

\hypertarget{ux44fux432ux43bux435ux43dux438ux44f-ux43eux431ux443ux441ux43bux43eux432ux43bux435ux43dux43dux44bux435-ux440ux435ux444ux440ux430ux43aux446ux438ux435ux439-ux441ux432ux435ux442ux430-ux430ux441ux442ux440ux43eux43dux43eux43cux438ux447ux435ux441ux43aux430ux44f-ux438-ux437ux435ux43cux43dux430ux44f-ux440ux435ux444ux440ux430ux43aux446ux438ux438}{%
\subsubsection{\texorpdfstring{\textbf{Явления, обусловленные рефракцией
света (астрономическая и земная
рефракции)}}{Явления, обусловленные рефракцией света (астрономическая и земная рефракции)}}\label{ux44fux432ux43bux435ux43dux438ux44f-ux43eux431ux443ux441ux43bux43eux432ux43bux435ux43dux43dux44bux435-ux440ux435ux444ux440ux430ux43aux446ux438ux435ux439-ux441ux432ux435ux442ux430-ux430ux441ux442ux440ux43eux43dux43eux43cux438ux447ux435ux441ux43aux430ux44f-ux438-ux437ux435ux43cux43dux430ux44f-ux440ux435ux444ux440ux430ux43aux446ux438ux438}}

\textbf{Атмосферная рефракция} --- это искривление световых лучей при их
прохождении через слои воздуха с разной плотностью. Поскольку плотность
воздуха убывает с высотой, лучи света от внеземных объектов всегда
изгибаются в сторону Земли.

\begin{itemize}
\tightlist
\item
  \textbf{Астрономическая рефракция:} Приводит к тому, что небесные
  светила (Солнце, звёзды) кажутся расположенными выше, чем они есть на
  самом деле. У горизонта этот эффект максимален (около 35 угловых
  минут), что примерно равно видимому диаметру Солнца. Благодаря этому
  мы видим восход Солнца на несколько минут раньше, а закат --- на
  несколько минут позже.
\item
  \textbf{Земная рефракция:} Искривление лучей от земных объектов. При
  стандартном распределении плотности она позволяет видеть объекты
  немного дальше, чем это было бы возможно при прямолинейном
  распространении света.
\item
  \textbf{Миражи:} Аномальные рефракционные явления, возникающие при
  очень больших вертикальных градиентах температуры у поверхности.

  \begin{itemize}
  \tightlist
  \item
    \textbf{Нижний мираж:} Наблюдается над сильно нагретой поверхностью
    (пустыня, асфальт). Воздух у поверхности менее плотный, и лучи света
    изгибаются вверх. В результате удалённые объекты (например, небо)
    кажутся отражёнными от мнимой водной поверхности.
  \item
    \textbf{Верхний мираж:} Наблюдается над сильно охлаждённой
    поверхностью (море, снег). Воздух у поверхности более плотный, лучи
    изгибаются вниз. В результате объекты кажутся поднятыми над
    горизонтом, могут выглядеть перевёрнутыми или искажёнными.
  \end{itemize}
\end{itemize}

\hypertarget{ux433ux430ux43bux43e-ux432ux435ux43dux446ux44b-ux440ux430ux434ux443ux433ux430-ux438-ux434ux440ux443ux433ux438ux435-ux43eux43fux442ux438ux447ux435ux441ux43aux438ux435-ux44fux432ux43bux435ux43dux438ux44f}{%
\subsubsection{\texorpdfstring{\textbf{Гало, венцы, радуга и другие
оптические
явления}}{Гало, венцы, радуга и другие оптические явления}}\label{ux433ux430ux43bux43e-ux432ux435ux43dux446ux44b-ux440ux430ux434ux443ux433ux430-ux438-ux434ux440ux443ux433ux438ux435-ux43eux43fux442ux438ux447ux435ux441ux43aux438ux435-ux44fux432ux43bux435ux43dux438ux44f}}

Эти явления возникают в результате взаимодействия света с каплями воды
или кристаллами льда в облаках.

\begin{itemize}
\tightlist
\item
  \textbf{Радуга:} Возникает при \textbf{дисперсии} (разложении белого
  света в спектр при \textbf{рефракции}) и \textbf{внутреннем отражении}
  солнечного света в дождевых каплях. Наблюдатель видит её, стоя спиной
  к Солнцу. \textbf{Первичная радуга} имеет угловой радиус 40-42°
  (красный цвет снаружи), \textbf{вторичная} --- 50-53° (цвета в
  обратном порядке).
\item
  \textbf{Гало:} Группа явлений, возникающих при \textbf{рефракции и
  отражении} света в гексагональных ледяных кристаллах перистых облаков.
  Наиболее частое явление --- \textbf{22-градусное гало} (светлый круг
  вокруг Солнца или Луны). Более сложные формы включают паргелии
  (``ложные солнца''), столбы и дуги.
\item
  \textbf{Венцы:} Цветные кольца малого радиуса непосредственно вокруг
  диска Солнца или Луны. Возникают в результате \textbf{дифракции} света
  на мельчайших, однородных по размеру облачных каплях (обычно в
  высококучевых или перисто-кучевых облаках).
\end{itemize}

\hypertarget{ux438ux43eux43dux438ux437ux430ux446ux438ux44f-ux432ux43eux437ux434ux443ux445ux430.-ux43bux435ux433ux43aux438ux435-ux438-ux442ux44fux436ux435ux43bux44bux435-ux438ux43eux43dux44b}{%
\subsubsection{\texorpdfstring{\textbf{Ионизация воздуха. Легкие и
тяжелые
ионы}}{Ионизация воздуха. Легкие и тяжелые ионы}}\label{ux438ux43eux43dux438ux437ux430ux446ux438ux44f-ux432ux43eux437ux434ux443ux445ux430.-ux43bux435ux433ux43aux438ux435-ux438-ux442ux44fux436ux435ux43bux44bux435-ux438ux43eux43dux44b}}

Атмосферный воздух всегда содержит некоторое количество электрически
заряженных частиц --- \textbf{ионов}.

\begin{itemize}
\tightlist
\item
  \textbf{Источники ионизации:}

  \begin{enumerate}
  \def\labelenumi{\arabic{enumi}.}
  \tightlist
  \item
    \textbf{Космические лучи:} Высокоэнергетические частицы из космоса,
    основной ионизатор в тропосфере и стратосфере.
  \item
    \textbf{Радиоактивность земной коры:} Излучение от естественных
    радиоактивных элементов в почве и горных породах.
  \item
    \textbf{Коротковолновая солнечная радиация:} Основной ионизатор в
    мезосфере и термосфере (создание ионосферы).
  \end{enumerate}
\item
  \textbf{Типы ионов:}

  \begin{itemize}
  \tightlist
  \item
    \textbf{Лёгкие ионы:} Образуются, когда молекула газа (например,
    \(N_2\) или \(O_2\)) теряет или присоединяет электрон. Они быстро
    окружаются несколькими нейтральными молекулами, образуя кластер.
  \item
    \textbf{Тяжёлые ионы:} Образуются, когда лёгкий ион присоединяется к
    аэрозольной частице. Они менее подвижны, чем лёгкие.
  \end{itemize}
\end{itemize}

\hypertarget{ux44dux43bux435ux43aux442ux440ux438ux447ux435ux441ux43aux438ux435-ux437ux430ux440ux44fux434ux44b-ux44dux43bux435ux43aux442ux440ux438ux447ux435ux441ux43aux43eux435-ux43fux43eux43bux435-ux438-ux43fux440ux43eux432ux43eux434ux438ux43cux43eux441ux442ux44c-ux430ux442ux43cux43eux441ux444ux435ux440ux44b}{%
\subsubsection{\texorpdfstring{\textbf{Электрические заряды,
электрическое поле и проводимость
атмосферы}}{Электрические заряды, электрическое поле и проводимость атмосферы}}\label{ux44dux43bux435ux43aux442ux440ux438ux447ux435ux441ux43aux438ux435-ux437ux430ux440ux44fux434ux44b-ux44dux43bux435ux43aux442ux440ux438ux447ux435ux441ux43aux43eux435-ux43fux43eux43bux435-ux438-ux43fux440ux43eux432ux43eux434ux438ux43cux43eux441ux442ux44c-ux430ux442ux43cux43eux441ux444ux435ux440ux44b}}

\begin{itemize}
\tightlist
\item
  \textbf{Глобальная электрическая цепь:} В среднем Земля несёт
  отрицательный заряд (около -500 000 Кулон), а атмосфера (в частности,
  ионосфера) --- эквивалентный положительный. Эта система действует как
  сферический конденсатор.
\item
  \textbf{Электрическое поле ``хорошей погоды'':} Между положительно
  заряженной ионосферой и отрицательно заряженной Землёй существует
  электрическое поле. У поверхности его напряжённость составляет в
  среднем \textbf{100-130 В/м} и направлена вертикально вниз.
\item
  \textbf{Атмосферная проводимость:} Наличие ионов обеспечивает воздуху
  слабую электрическую проводимость. Под действием поля ``хорошей
  погоды'' положительные ионы движутся вниз, а отрицательные --- вверх,
  создавая \textbf{ток хорошей погоды} (около 1-3 пикоампер на м²). Этот
  ток должен был бы нейтрализовать заряд Земли за несколько минут.
  Однако заряд постоянно восполняется за счёт \textbf{глобальной
  грозовой деятельности}.
\end{itemize}

\hypertarget{ux44dux43bux435ux43aux442ux440ux438ux447ux435ux441ux43aux43eux435-ux43fux43eux43bux435-ux43eux431ux43bux430ux43aux43eux432.-ux43eux441ux43dux43eux432ux44b-ux442ux435ux43eux440ux438ux438-ux433ux440ux43eux437ux43eux432ux43eux433ux43e-ux44dux43bux435ux43aux442ux440ux438ux447ux435ux441ux442ux432ux430.-ux433ux440ux43eux437ux430-ux43aux430ux43a-ux44fux432ux43bux435ux43dux438ux435}{%
\subsubsection{\texorpdfstring{\textbf{Электрическое поле облаков.
Основы теории грозового электричества. Гроза как
явление}}{Электрическое поле облаков. Основы теории грозового электричества. Гроза как явление}}\label{ux44dux43bux435ux43aux442ux440ux438ux447ux435ux441ux43aux43eux435-ux43fux43eux43bux435-ux43eux431ux43bux430ux43aux43eux432.-ux43eux441ux43dux43eux432ux44b-ux442ux435ux43eux440ux438ux438-ux433ux440ux43eux437ux43eux432ux43eux433ux43e-ux44dux43bux435ux43aux442ux440ux438ux447ux435ux441ux442ux432ux430.-ux433ux440ux43eux437ux430-ux43aux430ux43a-ux44fux432ux43bux435ux43dux438ux435}}

\begin{itemize}
\tightlist
\item
  \textbf{Электризация облаков:} Мощные кучево-дождевые облака (Cb)
  действуют как гигантские электростатические генераторы. Основной
  механизм \textbf{разделения зарядов} --- неиндуктивный, связанный со
  столкновениями ледяных кристаллов и частиц ледяной крупы (граупеля) в
  присутствии переохлаждённых капель воды. При столкновениях более
  тяжёлый граупель заряжается отрицательно и опускается в нижнюю часть
  облака, а более лёгкие кристаллы заряжаются положительно и уносятся
  восходящими потоками в верхнюю часть.
\item
  \textbf{Структура заряда грозового облака:} Классическая модель ---
  \textbf{вертикальный диполь} (или триполь): основной положительный
  заряд в верхней части, основной отрицательный --- в средней/нижней
  части (при температурах от -10 до -20°C), и иногда небольшой
  положительный заряд у основания.
\item
  \textbf{Молния:} Когда напряжённость электрического поля в облаке или
  между облаком и землёй превышает диэлектрическую прочность воздуха
  (около 3·10⁶ В/м), происходит гигантский искровой разряд ---
  \textbf{молния}. Большинство разрядов (около 80\%) происходит внутри
  облака или между облаками. Разряды ``облако-земля'' в основном
  переносят отрицательный заряд к земле, тем самым поддерживая её
  отрицательный заряд и замыкая глобальную электрическую цепь.
\item
  \textbf{Гроза как явление:} Это комплексное метеорологическое явление,
  связанное с кучево-дождевыми облаками и включающее молнии, гром,
  сильные ливневые осадки, град и шквалистые ветры.
\end{itemize}

\hypertarget{ux430ux43aux443ux441ux442ux438ux43aux430-ux430ux442ux43cux43eux441ux444ux435ux440ux44b-ux441ux43aux43eux440ux43eux441ux442ux44c-ux438-ux443ux441ux43bux43eux432ux438ux44f-ux440ux430ux441ux43fux440ux43eux441ux442ux440ux430ux43dux435ux43dux438ux44f-ux437ux432ux443ux43aux430}{%
\subsubsection{\texorpdfstring{\textbf{Акустика атмосферы: скорость и
условия распространения
звука}}{Акустика атмосферы: скорость и условия распространения звука}}\label{ux430ux43aux443ux441ux442ux438ux43aux430-ux430ux442ux43cux43eux441ux444ux435ux440ux44b-ux441ux43aux43eux440ux43eux441ux442ux44c-ux438-ux443ux441ux43bux43eux432ux438ux44f-ux440ux430ux441ux43fux440ux43eux441ux442ux440ux430ux43dux435ux43dux438ux44f-ux437ux432ux443ux43aux430}}

\begin{itemize}
\tightlist
\item
  \textbf{Скорость звука (\(c_s\)):} В идеальном газе скорость звука
  зависит только от температуры: \[c_s = \sqrt{\gamma R T}\] где
  \(\gamma = c_p/c_v \approx 1.4\) для воздуха. Для практических
  расчётов: \(c_s \approx 20.05 \sqrt{T}\). При \(T=0^\circ C\) скорость
  звука составляет около 331 м/с. Влажность и давление оказывают
  незначительное влияние.
\item
  \textbf{Рефракция звука:} Звуковые волны, как и световые, искривляются
  (рефрагируют) при прохождении через слои с разной скоростью звука.

  \begin{itemize}
  \tightlist
  \item
    \textbf{Температурная инверсия:} При инверсии температура (и
    скорость звука) растёт с высотой. Звуковые волны изгибаются вниз, к
    земле. Это приводит к \textbf{аномальной слышимости} --- звук слышен
    на очень больших расстояниях.
  \item
    \textbf{Стандартная тропосфера:} При стандартном падении температуры
    с высотой звуковые волны изгибаются вверх, от земли. Это может
    приводить к образованию \textbf{зон молчания}, где звук от источника
    на земле не слышен.
  \end{itemize}
\item
  \textbf{Влияние ветра:} Вертикальный сдвиг ветра также вызывает
  рефракцию. Звук распространяется дальше по направлению ветра (где его
  скорость складывается со скоростью звука) и хуже --- против ветра.
\item
  \textbf{Гром:} Звуковая волна, генерируемая быстрым расширением
  воздуха вдоль канала молнии, нагретого до \textasciitilde30 000 К.
  Раскаты грома возникают из-за того, что звук от разных частей длинного
  канала молнии приходит к наблюдателю в разное время.
\end{itemize}

ewpage

\hypertarget{ux44dux43bux435ux43aux442ux440ux438ux447ux435ux441ux43aux43eux435-ux43fux43eux43bux435-ux43eux431ux43bux430ux43aux43eux432}{%
\section{Электрическое Поле
Облаков}\label{ux44dux43bux435ux43aux442ux440ux438ux447ux435ux441ux43aux43eux435-ux43fux43eux43bux435-ux43eux431ux43bux430ux43aux43eux432}}

Электрические свойства атмосферы и происходящие в ней явления имеют
существенное значение для многих метеорологических процессов и являются
одной из основных характеристик атмосферы. Изучение электричества
облаков является фундаментальной задачей в области атмосферного
электричества.

\hypertarget{ux43eux431ux449ux430ux44f-ux445ux430ux440ux430ux43aux442ux435ux440ux438ux441ux442ux438ux43aux430-ux44dux43bux435ux43aux442ux440ux438ux447ux435ux441ux43aux43eux433ux43e-ux43fux43eux43bux44f-ux432-ux43eux431ux43bux430ux43aux430ux445}{%
\subsection{Общая Характеристика Электрического Поля в
Облаках}\label{ux43eux431ux449ux430ux44f-ux445ux430ux440ux430ux43aux442ux435ux440ux438ux441ux442ux438ux43aux430-ux44dux43bux435ux43aux442ux440ux438ux447ux435ux441ux43aux43eux433ux43e-ux43fux43eux43bux44f-ux432-ux43eux431ux43bux430ux43aux430ux445}}

Под электрическими характеристиками облака понимают плотность объемного
заряда, градиент потенциала (напряженность) электрического поля,
электропроводность в облаке и его окрестностях, спектральную плотность
электрических зарядов на частицах облака и осадков, спектральную
плотность ионов в облачном воздухе, а также плотность электрического
тока, текущего в облаке и вблизи него. Распределение величин градиента
потенциала электрического поля в облаке позволяет установить
распределение плотности объемного заряда, так как градиент потенциала
является более легко измеряемой величиной.

Подавляющее большинство данных о вертикальной составляющей градиента
потенциала электрического поля в облаках было получено в ходе
вертикальных зондирований атмосферы с помощью самолетов. Совокупность
абсолютных значений напряженности поля \textbar Е\textbar{} может быть
удовлетворительно аппроксимирована логарифмически нормальным
распределением.

Электрическая активность облаков, характеризуемая средней величиной
абсолютных значений напряженности электрического поля, возрастает от
одного вида облака к другому в следующей последовательности: St, Sc, Ac,
As, Ns, Cb. Как правило, с увеличением толщины облака возрастает его
электрическая активность, при этом эта зависимость ярче проявляется в
южных регионах. Электрическая активность облаков в среднем растет от
северных широт к южным. Электрические характеристики меняются от зимы к
лету: зимой средние и максимальные значения градиента потенциала
электрических полей, а также разности потенциалов на границах облаков
меньше, чем летом.

\hypertarget{ux43cux435ux445ux430ux43dux438ux437ux43cux44b-ux44dux43bux435ux43aux442ux440ux438ux437ux430ux446ux438ux438-ux43eux431ux43bux430ux447ux43dux44bux445-ux44dux43bux435ux43cux435ux43dux442ux43eux432-ux43cux438ux43aux440ux43eux44dux43bux435ux43aux442ux440ux438ux437ux430ux446ux438ux44f-1}{%
\subsection{Механизмы Электризации Облачных Элементов
(Микроэлектризация)}\label{ux43cux435ux445ux430ux43dux438ux437ux43cux44b-ux44dux43bux435ux43aux442ux440ux438ux437ux430ux446ux438ux438-ux43eux431ux43bux430ux447ux43dux44bux445-ux44dux43bux435ux43cux435ux43dux442ux43eux432-ux43cux438ux43aux440ux43eux44dux43bux435ux43aux442ux440ux438ux437ux430ux446ux438ux44f-1}}

Заряд частиц в облаках может изменяться в результате их взаимодействия с
ионами, взаимодействия частиц между собой или их спонтанного разрушения.

\hypertarget{ux438ux43eux43dux43dux430ux44f-ux44dux43bux435ux43aux442ux440ux438ux437ux430ux446ux438ux44f}{%
\subsubsection{Ионная
Электризация}\label{ux438ux43eux43dux43dux430ux44f-ux44dux43bux435ux43aux442ux440ux438ux437ux430ux446ux438ux44f}}

Механизм ионной электризации обычно сопутствует процессу
конденсационного роста частиц, играя основную роль на начальной стадии
развития облаков и туманов. Внешнее электрическое поле может существенно
изменять ход процессов электризации частиц в атмосфере, поскольку
зависимость скорости движения ионов от напряженности поля значительно
влияет на условия электризации. Лабораторные эксперименты подтвердили,
что в присутствии внешнего электрического поля, характерного для
грозовых облаков, эффективность ионного заряжения увеличивается в 5-10
раз в зависимости от размера капель. Было также обнаружено некоторое
преимущественное отрицательное заряжение капелек.

\hypertarget{ux44dux43bux435ux43aux442ux440ux438ux437ux430ux446ux438ux44f-ux43fux440ux438-ux440ux430ux437ux440ux443ux448ux435ux43dux438ux438-ux43aux43eux43dux442ux430ux43aux442ux430-ux433ux438ux434ux440ux43eux43cux435ux442ux435ux43eux440ux43eux432}{%
\subsubsection{Электризация при Разрушении Контакта
Гидрометеоров}\label{ux44dux43bux435ux43aux442ux440ux438ux437ux430ux446ux438ux44f-ux43fux440ux438-ux440ux430ux437ux440ux443ux448ux435ux43dux438ux438-ux43aux43eux43dux442ux430ux43aux442ux430-ux433ux438ux434ux440ux43eux43cux435ux442ux435ux43eux440ux43eux432}}

Столкновение облачных гидрометеоров (частиц облака и осадков) приводит
либо к их слиянию (коагуляции), либо к разделению после контакта.
Разрушение контакта гидрометеоров (а также спонтанное разрушение
отдельных частиц) сопровождается электризацией фрагментов. Наиболее
мощными механизмами микроэлектризации в облаках являются:

\begin{itemize}
\tightlist
\item
  Электризация при замерзании, деформации и раскалывании переохлажденных
  капель. Исследования показали, что электризация происходит за счет
  кристаллизации воды, согласно механизму Воркмана-Рейнгольдса, особенно
  интенсивно при взрывообразном замерзании.
\item
  Электризация при столкновении и разбрызгивании переохлажденных капелек
  на крупной ледяной частице. Разделение заряда происходит в результате
  дробления капель и кристаллизации воды. Во внешнем электрическом поле
  происходит дополнительная индукционная электризация: мелкие частицы
  забирают часть поляризованного заряда крупной частицы (градины),
  величина разделяющегося заряда пропорциональна напряженности внешнего
  электрического поля.
\item
  Электризация при столкновении ледяных кристалликов с крупинкой или
  градиной. Отскок приводит к их взаимной электризации. Величина и знак
  разделяющегося заряда определяются процессами в квазижидком слое,
  возникающем в зоне контакта. Это может происходить как в присутствии,
  так и в отсутствие внешнего электрического поля. Влияние внешнего
  электрического поля на величину разделяющегося заряда проявляется
  только в сильных полях.
\end{itemize}

Наиболее мощные механизмы микроэлектризации ``работают'' только в
облаках, содержащих ледяные гидрометеоры, причем чем больше ледяных
частиц, тем интенсивнее электризация. Особенности взаимодействия частиц
(вероятности столкновения и слияния) меняются, если частицы заряжены:
для разноименно заряженных частиц эти параметры возрастают, для
одноименно заряженных -- убывают. Это приводит к значительно более
быстрой коагуляции в облаках с большим количеством заряженных частиц.

\hypertarget{ux438ux43dux434ux443ux43aux446ux438ux43eux43dux43dux430ux44f-ux44dux43bux435ux43aux442ux440ux438ux437ux430ux446ux438ux44f-ux433ux438ux434ux440ux43eux43cux435ux442ux435ux43eux440ux43eux432}{%
\subsubsection{Индукционная Электризация
Гидрометеоров}\label{ux438ux43dux434ux443ux43aux446ux438ux43eux43dux43dux430ux44f-ux44dux43bux435ux43aux442ux440ux438ux437ux430ux446ux438ux44f-ux433ux438ux434ux440ux43eux43cux435ux442ux435ux43eux440ux43eux432}}

Во внешнем электрическом поле при разрыве контакта взаимодействующих
частиц происходит разделение заряда, связанное с индукционным
воздействием поля. При столкновении проводящие частицы обмениваются
зарядами. Если крупная ледяная частица поляризована во внешнем
электрическом поле, то при контакте и разрыве мелкие частицы забирают
часть поляризованного заряда крупной частицы. Величина разделяющегося
заряда линейно зависит от напряженности поля в облаке и радиуса мелкой
частицы.

\hypertarget{ux43eux440ux433ux430ux43dux438ux437ux43eux432ux430ux43dux43dux430ux44f-ux43cux430ux43aux440ux43eux44dux43bux435ux43aux442ux440ux438ux437ux430ux446ux438ux44f-ux43eux431ux43bux430ux43aux430-1}{%
\subsection{Организованная Макроэлектризация
Облака}\label{ux43eux440ux433ux430ux43dux438ux437ux43eux432ux430ux43dux43dux430ux44f-ux43cux430ux43aux440ux43eux44dux43bux435ux43aux442ux440ux438ux437ux430ux446ux438ux44f-ux43eux431ux43bux430ux43aux430-1}}

Процессом макроэлектризации облака называется процесс разделения
разноименно заряженных частиц в пространстве, приводящий к
преимущественному накоплению положительных или отрицательных зарядов в
больших объемах облака и формированию электрической структуры.
Устойчивая поляризация облака возможна только в том случае, если
восходящие потоки в облаке превышают скорости падения облачных частиц,
но не превосходят скорость падения частиц осадков. Это условие является
также условием гидродинамической устойчивости облаков. Макроэлектризация
и формирование электрической структуры облаков происходят не только в
грозовых облаках, но и во всех облаках и туманах, если в них происходит
направленная микроэлектризация в зависимости от размеров частиц и
последующее разделение частиц в объеме облака (тумана) в поле силы
тяжести.

\hypertarget{ux44dux43bux435ux43aux442ux440ux438ux447ux435ux441ux442ux432ux43e-ux441ux43bux43eux438ux441ux442ux43eux43eux431ux440ux430ux437ux43dux44bux445-ux43eux431ux43bux430ux43aux43eux432}{%
\subsection{Электричество Слоистообразных
Облаков}\label{ux44dux43bux435ux43aux442ux440ux438ux447ux435ux441ux442ux432ux43e-ux441ux43bux43eux438ux441ux442ux43eux43eux431ux440ux430ux437ux43dux44bux445-ux43eux431ux43bux430ux43aux43eux432}}

В слоисто-дождевых облаках могут возникать значительные объемные заряды.
Распределение напряженности электрического поля \textbar E\textbar{} по
высоте в слоисто-дождевых облаках смешанного фазового строения в
умеренных широтах показывает максимальные значения в зоне между
изотермами 0°C и -10°C, где происходит интенсивное разделение зарядов.
Под облаком или в нижней его части часто существуют области
положительных зарядов, связанные с осадками. Встречаются как
положительно, так и отрицательно поляризованные облака.

По результатам измерений градиента потенциала электрического поля
атмосферы, используя уравнение Пуассона, вычислено, что в среднем
плотность объемного заряда ρ в слоисто-кучевых и слоисто-дождевых
облаках составляет порядка 10⁻¹¹ Кл/м³. Горизонтальные электрические
неоднородности в облаках этих видов сравнительно невелики, но в зонах,
где градиент потенциала меняется более чем на 20-30\% от среднего
значения (протяженностью 200-600 м), плотность объемного заряда может
достигать 10⁻¹⁰ -- 10⁻⁹ Кл/м³.

Электропроводность воздуха в облаках должна быть меньше, чем в свободной
атмосфере, за счет захвата ионов каплями. Самолетные измерения показали,
что электропроводность в неплотных St и Sc облаках уменьшается в 3-25
раз по сравнению с чистым воздухом. Вывод: все наблюдаемые облака имеют
свою электрическую структуру и электрическую активность. Слоистообразные
облака имеют значительно меньшую активность, чем конвективные.

\hypertarget{ux44dux43bux435ux43aux442ux440ux438ux447ux435ux441ux442ux432ux43e-ux43aux43eux43dux432ux435ux43aux442ux438ux432ux43dux44bux445-ux43eux431ux43bux430ux43aux43eux432-ux433ux440ux43eux437ux43eux432ux44bux445-ux43eux431ux43bux430ux43aux43eux432}{%
\subsection{Электричество Конвективных Облаков (Грозовых
Облаков)}\label{ux44dux43bux435ux43aux442ux440ux438ux447ux435ux441ux442ux432ux43e-ux43aux43eux43dux432ux435ux43aux442ux438ux432ux43dux44bux445-ux43eux431ux43bux430ux43aux43eux432-ux433ux440ux43eux437ux43eux432ux44bux445-ux43eux431ux43bux430ux43aux43eux432}}

Грозовое электричество возникает в результате атмосферных процессов,
ведущих к образованию мощных кучево-дождевых облаков (Cb). Энергия,
запасаемая в электрическом поле облаков, составляет лишь малую долю
энергии этих атмосферных процессов, и некоторая часть этой энергии
реализуется при развитии молнии. Количество молний, генерируемых
облаком, непосредственно связано со скоростью воспроизводства заряда,
его величиной и распределением в объеме облака. Электрическая структура
грозового облака определяется его гидродинамической и микрофизической
структурой.

Исследования начала организованной электризации конвективных облаков
проводились с помощью самолетов-лабораторий, контролируя электрическое
состояние по напряженности поля над облаком. Фоновые значения
электрического поля на высотах более 3 км не превышают 10 В/м.

\hypertarget{ux43dux430ux447ux430ux43bux43e-ux43eux440ux433ux430ux43dux438ux437ux43eux432ux430ux43dux43dux43eux439-ux44dux43bux435ux43aux442ux440ux438ux437ux430ux446ux438ux438}{%
\subsubsection{Начало Организованной
Электризации}\label{ux43dux430ux447ux430ux43bux43e-ux43eux440ux433ux430ux43dux438ux437ux43eux432ux430ux43dux43dux43eux439-ux44dux43bux435ux43aux442ux440ux438ux437ux430ux446ux438ux438}}

Момент начала организованной электризации можно надежно фиксировать,
когда электрическое поле над облаком существенно превышает фоновые
значения. Для начала организованной электризации необходимо наличие
значительной переохлажденной части облака и крупных частиц (осадков).
Необходимым условием начала электризации конвективного облака, даже в
тропиках, является наличие крупных частиц в переохлажденной зоне, то
есть наличие крупных ледяных частиц.

На начальной стадии развития процесса электризации (при превышении
вершиной облака изотермы -8°C для СНГ и -6°C для Кубы) направление
вертикальной компоненты электрического поля над облаком совпадает с
направлением поля ``хорошей погоды'', что соответствует диполю с
``минусом'' вверху. На стадии зрелости и распада облака направление
вертикальной компоненты поля обычно меняется на противоположное,
характерное для грозовых облаков, то есть эквивалентный облачный диполь
как бы ``переворачивается''. Напряженность электрического поля порядка
10 000 В/м над вершиной облака свидетельствует о предгрозовом состоянии,
и влетевший в такое облако заряженный самолет может спровоцировать
разряд молнии на себя.

На стадии развития конвективного облака наиболее тесная, близкая к
линейной, статистическая связь наблюдается между напряженностью
электрического поля и интегралом (суммой) отражаемости (ZS) по части
сечения облака, расположенной выше нулевой изотермы и имеющей
отражаемость более 35 dBZ. Этот интеграл пропорционален количеству
крупных частиц в облаке, которые, вероятно, являются ледяными и
определяют мощность эквивалентного электрического генератора облака.

\hypertarget{ux437ux430ux440ux44fux434ux44b-ux438-ux44dux43bux435ux43aux442ux440ux438ux447ux435ux441ux43aux438ux435-ux43fux43eux43bux44f-ux433ux440ux43eux437ux43eux432ux44bux445-ux43eux431ux43bux430ux43aux43eux432}{%
\subsubsection{Заряды и Электрические Поля Грозовых
Облаков}\label{ux437ux430ux440ux44fux434ux44b-ux438-ux44dux43bux435ux43aux442ux440ux438ux447ux435ux441ux43aux438ux435-ux43fux43eux43bux44f-ux433ux440ux43eux437ux43eux432ux44bux445-ux43eux431ux43bux430ux43aux43eux432}}

Молнии и другие электрические явления в грозовых облаках однозначно
указывают на присутствие в них больших электрических зарядов. Заряды в
облаке располагаются определенным образом, причем заряды одного знака
концентрируются над зарядами другого знака. Такое облако называется
``биполярным''.

Структура распределения электрических зарядов зрелых грозовых облаков,
как правило, является биполярной, с основным положительным зарядом
вверху и основным отрицательным зарядом в нижней части. Дополнительный
положительный заряд часто существует вблизи основания облака, на уровне
0°C или несколько ниже.

Избыточный заряд, возникающий на гидрометеорах при их взаимодействии,
зависит от степени поляризации и, следовательно, от напряженности
электрического поля. Еще в большей степени от напряженности зависит
избыточный заряд, появляющийся вследствие коронирования гидрометеоров.
Механизм такого рода выполняет роль положительной обратной связи и
способствует нарастанию электрических зарядов в облаке.

Общая величина накопленного заряда (положительного и отрицательного) в
грозовых облаках приблизительно одинакова и сильно варьирует, среднее
значение в умеренных широтах составляет около 25 Кл. Средняя
напряженность электрического поля в основной заряженной области
грозового облака может быть оценена величиной порядка 10⁵ В/м, которая
наблюдалась при самолетном зондировании наружных частей грозовых
облаков. В действительности, заряд облака не распределен равномерно, и
при самолетном зондировании мощных кучевых облаков фиксировались области
с напряженностью поля до 3·10⁵ В/м, протяженностью 200-600 м.
Возникновение нисходящего лидера в грозовом облаке объясняется наличием
локальных неоднородностей. В ``запальных'' микро-ячейках, где возникают
стриммеры, напряженность поля может достигать 2·10⁶ В/м. Разряд в облаке
начинается, если напряженность электрического поля достигает 3 ÷ 4∙10⁵
В/м.

\hypertarget{ux438ux437ux43cux435ux440ux435ux43dux438ux435-ux44dux43bux435ux43aux442ux440ux438ux447ux435ux441ux43aux43eux433ux43e-ux43fux43eux43bux44f-ux432-ux430ux442ux43cux43eux441ux444ux435ux440ux435-ux438-ux43eux431ux43bux430ux43aux430ux445}{%
\subsection{Измерение Электрического Поля в Атмосфере и
Облаках}\label{ux438ux437ux43cux435ux440ux435ux43dux438ux435-ux44dux43bux435ux43aux442ux440ux438ux447ux435ux441ux43aux43eux433ux43e-ux43fux43eux43bux44f-ux432-ux430ux442ux43cux43eux441ux444ux435ux440ux435-ux438-ux43eux431ux43bux430ux43aux430ux445}}

Для измерения напряженности электрического поля применяются различные
методы, среди которых основными являются метод коллекторов и метод
пластины. В первом случае измеряется потенциал в двух точках на разных
высотах, а затем определяется градиент потенциала. Во втором методе
измеряется плотность поверхностного заряда, по которой непосредственно
определяется напряженность поля.

Прибор для измерения напряженности электрического поля (ПНП), также
называемый электростатическим флюксметром или генерирующим вольтметром,
измеряет напряженность поля путем измерения индуцированного механическим
модулятором заряда, потенциала или тока индуцированного заряда на
изолированном металлическом электроде. Измерительный электрод
периодически экспонируется в электрическом поле или экранируется от него
вращающимся затвором. Индуцированный заряд и ток пропорциональны
напряженности электрического поля, нормальной к электроду.

При измерениях электрических характеристик атмосферы с борта самолета
или других летательных аппаратов следует учитывать, что сам аппарат,
будучи металлическим телом сложной формы, существенно искажает значения
поля в местах расположения приборов, даже в условиях безоблачной
атмосферы. Кроме того, самолеты могут иметь собственный электрический
заряд, создающий в местах установки датчиков электрические поля до 15
000 В/м. Существует специальная методика измерения напряженности
электрического поля с использованием самолетов, основанная на допущениях
об эквипотенциальности поверхности фюзеляжа и однородности измеряемого
поля. В такой методике вектор напряженности поля и заряд самолета
определяются по данным измерений четырех датчиков напряженности.
Коэффициенты локального искажения поля могут быть определены расчетным
путем или по результатам исследования модели самолета, помещенной в
электрическое поле конденсатора.

Заряд самолета при полете вне облаков и осадков создается за счет работы
двигателей (двигательная электризация). Для сброса заряда используются
``штатные пассивные разрядники'' (острия), расположенные в местах
наибольшего усиления электрического поля, с которых течет ток
коронирования при достижении критической напряженности поля. Высокий
уровень заряжения самолета может затруднять измерения.

ewpage

\hypertarget{ux43eux441ux43dux43eux432ux44b-ux442ux435ux43eux440ux438ux438-ux433ux440ux43eux437ux43eux432ux43eux433ux43e-ux44dux43bux435ux43aux442ux440ux438ux447ux435ux441ux442ux432ux430}{%
\section{Основы теории грозового
электричества}\label{ux43eux441ux43dux43eux432ux44b-ux442ux435ux43eux440ux438ux438-ux433ux440ux43eux437ux43eux432ux43eux433ux43e-ux44dux43bux435ux43aux442ux440ux438ux447ux435ux441ux442ux432ux430}}

Теория грозового электричества является фундаментальным аспектом
атмосферной физики, рассматривающим электрические процессы, протекающие
в мощных кучево-дождевых облаках, и связанные с ними явления, такие как
молнии. Изучение электрической активности облаков имеет не только
научное, но и важное практическое значение, в том числе для прогноза
гроз, борьбы с электростатической опасностью для авиации и разработки
средств регулирования электрической активности.

\hypertarget{ux44dux43bux435ux43aux442ux440ux438ux447ux435ux441ux43aux438ux435-ux445ux430ux440ux430ux43aux442ux435ux440ux438ux441ux442ux438ux43aux438-ux43eux431ux43bux430ux43aux43eux432}{%
\subsection{Электрические характеристики
облаков}\label{ux44dux43bux435ux43aux442ux440ux438ux447ux435ux441ux43aux438ux435-ux445ux430ux440ux430ux43aux442ux435ux440ux438ux441ux442ux438ux43aux438-ux43eux431ux43bux430ux43aux43eux432}}

Понятие ``электрические характеристики облака'' включает в себя ряд
параметров, таких как плотность объемного заряда, градиент потенциала
(напряженность) электрического поля, электропроводность в облаке и его
окрестностях, спектральная плотность электрических зарядов на частицах
облака и осадков, а также плотность электрического тока. Эти
характеристики взаимосвязаны; например, по распределению градиента
потенциала электрического поля можно определить плотность объемного
заряда.

Напряженность электрического поля в облаках может существенно
различаться для разных форм облаков, а также меняться от зимы к лету,
причем летом средние и максимальные значения градиента потенциала обычно
выше.

\hypertarget{ux43cux435ux445ux430ux43dux438ux437ux43cux44b-ux44dux43bux435ux43aux442ux440ux438ux437ux430ux446ux438ux438-ux43eux431ux43bux430ux447ux43dux44bux445-ux44dux43bux435ux43cux435ux43dux442ux43eux432-ux43cux438ux43aux440ux43eux44dux43bux435ux43aux442ux440ux438ux437ux430ux446ux438ux438}{%
\subsection{Механизмы электризации облачных элементов
(микроэлектризации)}\label{ux43cux435ux445ux430ux43dux438ux437ux43cux44b-ux44dux43bux435ux43aux442ux440ux438ux437ux430ux446ux438ux438-ux43eux431ux43bux430ux447ux43dux44bux445-ux44dux43bux435ux43cux435ux43dux442ux43eux432-ux43cux438ux43aux440ux43eux44dux43bux435ux43aux442ux440ux438ux437ux430ux446ux438ux438}}

Заряд частиц в облаках может изменяться в результате взаимодействия с
ионами, взаимодействия частиц между собой или их спонтанного разрушения.

\hypertarget{ux438ux43eux43dux43dux430ux44f-ux44dux43bux435ux43aux442ux440ux438ux437ux430ux446ux438ux44f-1}{%
\subsubsection{Ионная
электризация}\label{ux438ux43eux43dux43dux430ux44f-ux44dux43bux435ux43aux442ux440ux438ux437ux430ux446ux438ux44f-1}}

На начальной стадии развития облаков и туманов изменение их
электрических характеристик преимущественно обусловлено заряжением
частиц за счет атмосферных ионов. Атмосферные ионы взаимодействуют с
облачными частицами, изменяя их заряд. Характер взаимодействия
определяется размером частицы (r) и длиной свободного пробега иона (l).
Если r значительно больше l, реализуется диффузионный режим
электризации. Лабораторные эксперименты подтвердили теоретические
описания ионного заряжения аэрозолей, выявив также преимущественное
отрицательное заряжение капелек. Внешнее электрическое поле, особенно
сильное, как в грозовых облаках, может значительно усилить эффективность
ионного заряжения частиц, увеличивая ее в 5-10 раз в зависимости от
размера капель.

\hypertarget{ux44dux43bux435ux43aux442ux440ux438ux437ux430ux446ux438ux44f-ux432-ux440ux435ux437ux443ux43bux44cux442ux430ux442ux435-ux440ux430ux437ux440ux443ux448ux435ux43dux438ux44f-ux43aux43eux43dux442ux430ux43aux442ux430-ux433ux438ux434ux440ux43eux43cux435ux442ux435ux43eux440ux43eux432}{%
\subsubsection{Электризация в результате разрушения контакта
гидрометеоров}\label{ux44dux43bux435ux43aux442ux440ux438ux437ux430ux446ux438ux44f-ux432-ux440ux435ux437ux443ux43bux44cux442ux430ux442ux435-ux440ux430ux437ux440ux443ux448ux435ux43dux438ux44f-ux43aux43eux43dux442ux430ux43aux442ux430-ux433ux438ux434ux440ux43eux43cux435ux442ux435ux43eux440ux43eux432}}

Столкновение облачных гидрометеоров, ведущее к их слиянию (коагуляции)
или разделению, сопровождается электризацией фрагментов контакта.
Лабораторное моделирование позволило выделить и изучить наиболее мощные
процессы микроэлектризации в облаках, к которым относятся:

\begin{itemize}
\tightlist
\item
  \textbf{Электризация при замерзании, деформации и раскалывании
  переохлажденных капель}: Это сопровождается разделением зарядов,
  особенно интенсивно при взрывообразном замерзании, и объясняется
  кристаллизацией воды по механизму Воркмана-Рейнгольдса. Ледяные
  фрагменты, как правило, несут отрицательный заряд. Существуют три
  теории объяснения этого механизма: электризация кристаллизующейся
  воды, контактная и термоэлектрическая электризация.
\item
  \textbf{Электризация при разбрызгивании переохлажденных капель на
  крупной ледяной частице}: Разделение заряда происходит в результате
  дробления капель и кристаллизации воды. Во внешнем электрическом поле
  индукционное заряжение взаимодействующих частиц приводит к
  дополнительной электризации. Мелкие частицы, сталкиваясь с
  поляризованной крупной частицей (например, градиной) и разделяясь с
  ней, уносят часть поляризованного заряда, причем величина
  разделяющегося заряда пропорциональна напряженности внешнего
  электрического поля.
\item
  \textbf{Электризация при столкновении ледяных кристаллов с крупинкой
  или градиной}: Отскок ледяных кристалликов от крупной ледяной частицы
  (крупинки или градины) приводит к их взаимной электризации. Величина и
  знак заряда зависят от процессов в квазижидком слое в зоне контакта.
  Влияние внешнего электрического поля проявляется только в сильных
  полях из-за низкой электропроводности льда. Примеси, изменяющие
  поверхностные свойства льда, также могут влиять на электризацию,
  изменяя не только характер зависимости заряжения от водности и
  температуры, но и знак заряда.
\end{itemize}

Наиболее мощные механизмы микроэлектризации ``работают'' только в
облаках, содержащих ледяные гидрометеоры, причем интенсивность
электризации возрастает с увеличением количества ледяных частиц.

\hypertarget{ux43cux430ux43aux440ux43eux44dux43bux435ux43aux442ux440ux438ux437ux430ux446ux438ux44f-ux43eux431ux43bux430ux43aux430}{%
\subsection{Макроэлектризация
облака}\label{ux43cux430ux43aux440ux43eux44dux43bux435ux43aux442ux440ux438ux437ux430ux446ux438ux44f-ux43eux431ux43bux430ux43aux430}}

Процесс макроэлектризации облака, то есть формирования его электрической
структуры, происходит при направленной микроэлектризации частиц в
зависимости от их размеров и последующего разделения этих частиц в
объеме облака под действием силы тяжести. Облака различных форм имеют
свою характерную электрическую структуру.

\hypertarget{ux44dux43bux435ux43aux442ux440ux438ux447ux435ux441ux442ux432ux43e-ux43aux43eux43dux432ux435ux43aux442ux438ux432ux43dux44bux445-ux43eux431ux43bux430ux43aux43eux432-1}{%
\subsubsection{Электричество конвективных
облаков}\label{ux44dux43bux435ux43aux442ux440ux438ux447ux435ux441ux442ux432ux43e-ux43aux43eux43dux432ux435ux43aux442ux438ux432ux43dux44bux445-ux43eux431ux43bux430ux43aux43eux432-1}}

Грозовое электричество возникает в результате атмосферных процессов,
ведущих к образованию мощных кучево-дождевых облаков (Cb). Энергия,
запасенная в электрическом поле облаков, является лишь малой частью
общей энергии этих атмосферных процессов. Часть этой энергии реализуется
в молниях, количество которых напрямую связано со скоростью
воспроизводства заряда, его величиной и распределением в объеме облака.

\hypertarget{ux441ux442ux440ux43eux435ux43dux438ux435-ux438-ux444ux430ux437ux44b-ux436ux438ux437ux43dux438-ux433ux440ux43eux437ux43eux432ux43eux433ux43e-ux43eux431ux43bux430ux43aux430}{%
\subsubsection{Строение и фазы жизни грозового
облака}\label{ux441ux442ux440ux43eux435ux43dux438ux435-ux438-ux444ux430ux437ux44b-ux436ux438ux437ux43dux438-ux433ux440ux43eux437ux43eux432ux43eux433ux43e-ux43eux431ux43bux430ux43aux430}}

Электрическая структура грозового облака тесно связана с его
гидродинамической и микрофизической структурой. Грозовое облако, как
правило, состоит из одной или нескольких конвективных ячеек (до 5-10),
каждая из которых проходит через три стадии эволюции:

\begin{enumerate}
\def\labelenumi{\arabic{enumi}.}
\tightlist
\item
  \textbf{Начальная стадия (стадия роста)}: Характеризуется
  преобладанием восходящих движений во всей толще ячейки, что
  способствует активным процессам конденсационного роста частиц.
\item
  \textbf{Стадия зрелости}: На этой стадии наблюдаются как восходящие,
  так и нисходящие потоки, происходит образование осадков.
\item
  \textbf{Стадия диссипации}: Преобладают нисходящие потоки, облако
  разрушается.
\end{enumerate}

\hypertarget{ux43dux430ux447ux430ux43bux43e-ux43eux440ux433ux430ux43dux438ux437ux43eux432ux430ux43dux43dux43eux439-ux44dux43bux435ux43aux442ux440ux438ux437ux430ux446ux438ux438-ux43aux43eux43dux432ux435ux43aux442ux438ux432ux43dux44bux445-ux43eux431ux43bux430ux43aux43eux432}{%
\subsubsection{Начало организованной электризации конвективных
облаков}\label{ux43dux430ux447ux430ux43bux43e-ux43eux440ux433ux430ux43dux438ux437ux43eux432ux430ux43dux43dux43eux439-ux44dux43bux435ux43aux442ux440ux438ux437ux430ux446ux438ux438-ux43aux43eux43dux432ux435ux43aux442ux438ux432ux43dux44bux445-ux43eux431ux43bux430ux43aux43eux432}}

Современные методы исследований, использующие самолеты-метеолаборатории
(СМЛ), позволяют комплексно изучать электризацию облаков на всех стадиях
их развития. Момент начала организованной электризации облака надежно
фиксируется по характерным изменениям напряженности электрического поля
(E) над облаком.

Экспериментальные исследования показали, что для начала организованной
электризации необходимо наличие значительной переохлажденной части
облака и крупных частиц (осадков). Напряженность электрического поля
порядка 10 000 В/м над вершиной облака свидетельствует о предгрозовом
состоянии, при котором самолет может спровоцировать разряд молнии на
себя. Анализ результатов экспериментов подтверждает гипотезу о том, что
электризация конвективного облака, по крайней мере на предгрозовой
стадии, определяется параметрами, связанными с количеством и размерами
крупных ледяных частиц.

\hypertarget{ux437ux430ux440ux44fux434ux44b-ux438-ux44dux43bux435ux43aux442ux440ux438ux447ux435ux441ux43aux438ux435-ux43fux43eux43bux44f-ux433ux440ux43eux437ux43eux432ux44bux445-ux43eux431ux43bux430ux43aux43eux432-1}{%
\subsubsection{Заряды и электрические поля грозовых
облаков}\label{ux437ux430ux440ux44fux434ux44b-ux438-ux44dux43bux435ux43aux442ux440ux438ux447ux435ux441ux43aux438ux435-ux43fux43eux43bux44f-ux433ux440ux43eux437ux43eux432ux44bux445-ux43eux431ux43bux430ux43aux43eux432-1}}

Распределение зарядов в грозовом облаке является сложной задачей для
однозначного определения по наземным измерениям. В типичном грозовом
облаке наблюдаются зоны избыточного заряда с различными значениями
напряженности электрического поля:

\begin{itemize}
\tightlist
\item
  Под кучево-дождевым облаком (Cb): до 10\^{}4 В/м на протяжении
  нескольких километров.
\item
  Основная заряженная часть Cb: до 10\^{}5 В/м в зоне 200-600 м.
\item
  Ячейки избыточного заряда: до 3∙10\^{}5 В/м в зонах 200-600 м.
\item
  ``Запальные'' микро-ячейки, где возникают стримерные процессы: до
  2∙10\^{}6 В/м на нескольких сантиметрах.
\end{itemize}

\hypertarget{ux43cux43eux43bux43dux438ux438-1}{%
\subsubsection{Молнии}\label{ux43cux43eux43bux43dux438ux438-1}}

Молния --- это мощный электрический разряд в атмосфере, который обычно
возникает во время грозы. Молнии, особенно шаровые, являются
накопителями энергии неопознанной природы. Идея о существовании
атмосферного электричества возникла из предположения, что гром и молния
--- это гигантские проявления явлевлений, наблюдаемых в лабораторных
опытах со статическим электричеством. Молниевые разряды начинаются с
возникновения избыточного заряда на отдельной частице или в микрообъеме
воздуха, то есть со статической электризации.

\hypertarget{ux432ux44bux441ux43eux442ux43dux44bux435-ux440ux430ux437ux440ux44fux434ux44b-ux432-ux430ux442ux43cux43eux441ux444ux435ux440ux435}{%
\subsubsection{Высотные разряды в
атмосфере}\label{ux432ux44bux441ux43eux442ux43dux44bux435-ux440ux430ux437ux440ux44fux434ux44b-ux432-ux430ux442ux43cux43eux441ux444ux435ux440ux435}}

Изучение высотных разрядов в атмосфере, приводящих к перекачке энергии,
накапливаемой грозовыми облаками, в среднюю атмосферу, стало обширным и
интенсивно развивающимся направлением атмосферного электричества. Хотя
морфология этих явлений еще накапливается, уже можно переходить к
исследованию более тонких особенностей их структуры и динамики.

\hypertarget{ux440ux430ux441ux43fux440ux435ux434ux435ux43bux435ux43dux438ux435-ux433ux440ux43eux437ux43eux432ux43eux439-ux434ux435ux44fux442ux435ux43bux44cux43dux43eux441ux442ux438}{%
\subsection{Распределение грозовой
деятельности}\label{ux440ux430ux441ux43fux440ux435ux434ux435ux43bux435ux43dux438ux435-ux433ux440ux43eux437ux43eux432ux43eux439-ux434ux435ux44fux442ux435ux43bux44cux43dux43eux441ux442ux438}}

Наблюдения за грозами на метеорологических станциях традиционно
ограничиваются фиксацией времени появления грозы, иногда с субъективной
оценкой ее мощности.

\hypertarget{ux430ux442ux43cux43eux441ux444ux435ux440ux438ux43aux438}{%
\subsubsection{Атмосферики}\label{ux430ux442ux43cux43eux441ux444ux435ux440ux438ux43aux438}}

Молнии излучают электромагнитные волны, называемые атмосфериками,
которые распространяются в естественном волноводе, образованном
ионосферой и поверхностью Земли. Атмосферики обладают слабым затуханием
и могут распространяться на значительные расстояния. Они были открыты
А.С. Поповым в 1896 году, который понял их связь с молниями. Изучение
атмосферных токов и атмосферных помех, вызванных молниями, позволяет
судить об атмосферных процессах, с которыми связано их возникновение.
Длительные систематические наблюдения показали тесную связь атмосферных
помех с грозовой деятельностью, развивающейся на фронтах, что позволяет
следить издали за формированием и перемещением таких областей.

\hypertarget{ux433ux43bux43eux431ux430ux43bux44cux43dux430ux44f-ux430ux442ux43cux43eux441ux444ux435ux440ux43dux43e-ux44dux43bux435ux43aux442ux440ux438ux447ux435ux441ux43aux430ux44f-ux446ux435ux43fux44c-ux433ux44dux446-1}{%
\subsection{Глобальная атмосферно-электрическая цепь
(ГЭЦ)}\label{ux433ux43bux43eux431ux430ux43bux44cux43dux430ux44f-ux430ux442ux43cux43eux441ux444ux435ux440ux43dux43e-ux44dux43bux435ux43aux442ux440ux438ux447ux435ux441ux43aux430ux44f-ux446ux435ux43fux44c-ux433ux44dux446-1}}

Грозовые облака играют роль генератора в Глобальной электрической цепи
(ГЭЦ). В соответствии с современной концепцией, ГЭЦ представляет собой
распределенный токовый контур, образованный проводящими слоями
ионосферы, верхнего слоя океана и земной коры, с грозовыми генераторами
в качестве основных источников электродвижущих сил. Совокупный заряд,
переносимый молниями всех гроз на земном шаре, заряжает поверхность
Земли отрицательно, а верхняя часть грозовых облаков имеет положительный
заряд, что приводит к положительному потенциалу ионосферы.

Унитарная вариация напряженности электрического поля атмосферы,
наблюдаемая над океанами и в полярных районах, с максимумом около 19
часов UT, служит прямым доказательством существования единого генератора
атмосферного электрического поля. Период максимальной грозовой
активности согласуется с максимумом ``унитарной вариации''.

Основные параметры ГЭЦ включают число действующих гроз (среднее
\textasciitilde1500-2000), вертикальный ток грозовых облаков
(\textasciitilde0.5-1.0 A), потенциал ионосферы (\textasciitilde250-400
кВ), а также проводимость атмосферы на различных высотах (уровень моря
\textasciitilde10\^{}-14 См/м, ионосфера \textasciitilde10\^{}-4 - 10
См/м). Эти характеристики служат индикатором стационарного состояния и
пространственно-временной динамики атмосферных процессов.

ewpage

\hypertarget{ux44dux43bux435ux43aux442ux440ux438ux447ux435ux441ux43aux438ux435-ux442ux43eux43aux438-ux43eux441ux430ux434ux43aux43eux432}{%
\section{Электрические токи
осадков}\label{ux44dux43bux435ux43aux442ux440ux438ux447ux435ux441ux43aux438ux435-ux442ux43eux43aux438-ux43eux441ux430ux434ux43aux43eux432}}

\hypertarget{ux43eux431ux449ux435ux435-ux43eux43fux440ux435ux434ux435ux43bux435ux43dux438ux435-ux438-ux43cux435ux445ux430ux43dux438ux437ux43c-ux444ux43eux440ux43cux438ux440ux43eux432ux430ux43dux438ux44f}{%
\subsection{Общее определение и механизм
формирования}\label{ux43eux431ux449ux435ux435-ux43eux43fux440ux435ux434ux435ux43bux435ux43dux438ux435-ux438-ux43cux435ux445ux430ux43dux438ux437ux43c-ux444ux43eux440ux43cux438ux440ux43eux432ux430ux43dux438ux44f}}

Ток осадков --- это электрический ток, образующийся в результате
выпадения из облака заряженных частиц осадков, таких как дождь, снег или
град. Процесс его формирования начинается, когда гидрометеоры в облаке
приобретают заряд в результате взаимодействия как друг с другом, так и с
другими облачными частицами. По мере падения сквозь подоблачный слой эти
заряженные частицы могут подвергаться перезарядке, прежде чем достигнуть
земной поверхности, создавая таким образом регистрируемый ток.

\hypertarget{ux438ux437ux43cux435ux440ux435ux43dux438ux435-ux438-ux43fux43bux43eux442ux43dux43eux441ux442ux44c-ux442ux43eux43aux430}{%
\subsection{Измерение и плотность
тока}\label{ux438ux437ux43cux435ux440ux435ux43dux438ux435-ux438-ux43fux43bux43eux442ux43dux43eux441ux442ux44c-ux442ux43eux43aux430}}

Для измерения заряда, переносимого осадками, используется методика
улавливания их в специально установленный изолированный сосуд, который
затем соединяется с электрометром. Это позволяет определить плотность
тока, связанную с переносом заряженных частиц осадков, а также
рассчитать заряд, приходящийся на единицу объема выпавшей воды или на
каждую отдельную каплю.

Средние значения плотности токов осадков, выпадающих из слоисто-дождевых
облаков в умеренных широтах, как правило, находятся в диапазоне от
5×10⁻¹² до 5×10⁻¹¹ А/м². Относительно невысокие значения плотности этого
тока обусловлены тем, что количество положительно заряженных капель
зачастую почти эквивалентно числу отрицательно заряженных. Более того,
важно отметить, что величина и даже знак заряда выпадающей капли могут
изменяться в процессе её падения из облака.

\hypertarget{ux437ux43dux430ux447ux435ux43dux438ux435-ux432-ux430ux442ux43cux43eux441ux444ux435ux440ux43dux43eux439-ux44dux43bux435ux43aux442ux440ux438ux43aux435}{%
\subsection{Значение в атмосферной
электрике}\label{ux437ux43dux430ux447ux435ux43dux438ux435-ux432-ux430ux442ux43cux43eux441ux444ux435ux440ux43dux43eux439-ux44dux43bux435ux43aux442ux440ux438ux43aux435}}

Сведения о величине токов осадков собираются с конца XIX века. Однако
эти измерения, даже в сочетании с данными наземных наблюдений
электрического поля, не всегда позволяли с достаточной полнотой
восстановить детальную электрическую структуру облаков. Следовательно,
осадки в целом не выносят существенный чистый заряд даже из отдельных
частей облака. Тем не менее, в некоторых случаях в негрозовых облаках
ток выпадающих осадков может служить индикатором общей электризации
облаков.

В рамках глобальной атмосферно-электрической цепи (ГЭЦ), ток осадков
является одной из компонент, наряду с вертикальными токами проводимости
в ясную погоду, токами разрядов с острий (тихих разрядов) и токами
молний. В среднем, по данным для Кембриджа, Англия, годовой перенос
заряда током осадков составляет около +20 Кл/км² в год. Это указывает на
то, что токи осадков могут переносить как положительный, так и
отрицательный заряд, но в долгосрочной перспективе они вносят вклад в
общий баланс заряда Земли, который также включает значительный
отрицательный перенос от токов с острий и молний.

В более широком контексте атмосферного электричества, токи частиц
осадков, выпадающих из облака, существуют в тесной взаимосвязи с другими
электрическими процессами и рассматриваются как часть единого процесса
энергообмена, охватывающего широкий спектр масштабов, включая глобальную
атмосферно-электрическую цепь. При разработке нестационарных моделей ГЭЦ
важно учитывать разнообразие всех этих процессов, включая токи осадков.

ewpage

\hypertarget{ux433ux440ux43eux437ux430-ux43aux430ux43a-ux430ux442ux43cux43eux441ux444ux435ux440ux43dux43eux435-ux44fux432ux43bux435ux43dux438ux435-ux438-ux435ux435-ux441ux442ux430ux442ux438ux441ux442ux438ux447ux435ux441ux43aux438ux435-ux445ux430ux440ux430ux43aux442ux435ux440ux438ux441ux442ux438ux43aux438}{%
\section{Гроза как Атмосферное Явление и ее Статистические
Характеристики}\label{ux433ux440ux43eux437ux430-ux43aux430ux43a-ux430ux442ux43cux43eux441ux444ux435ux440ux43dux43eux435-ux44fux432ux43bux435ux43dux438ux435-ux438-ux435ux435-ux441ux442ux430ux442ux438ux441ux442ux438ux447ux435ux441ux43aux438ux435-ux445ux430ux440ux430ux43aux442ux435ux440ux438ux441ux442ux438ux43aux438}}

Изучение электричества облаков является одной из фундаментальных задач в
области атмосферного электричества. Гроза, как наиболее известное и
доступное непосредственному восприятию явление, неизменно вызывает живой
интерес. Электрические проявления в облаках сложны и грандиозны,
сопровождаясь такими необычными явлениями, как молнии.

\hypertarget{ux43eux431ux449ux430ux44f-ux445ux430ux440ux430ux43aux442ux435ux440ux438ux441ux442ux438ux43aux430-ux433ux440ux43eux437ux44b}{%
\subsection{Общая Характеристика
Грозы}\label{ux43eux431ux449ux430ux44f-ux445ux430ux440ux430ux43aux442ux435ux440ux438ux441ux442ux438ux43aux430-ux433ux440ux43eux437ux44b}}

Гроза --- это прохождение мощного кучево-дождевого облака,
сопровождающееся сильным ветром, ливневыми осадками, молнией и громом.
Термин ``грозовое облако'' употребляется как синоним кучево-дождевого
облака (Cb), хотя Cb не всегда обязательно сопровождается грозовыми
явлениями. Грозовое электричество возникает в результате атмосферных
процессов, ведущих к образованию мощных кучево-дождевых облаков.

Электрические явления в облаках являются важнейшей частью генератора
глобальной электрической цепи (ГЭЦ), обеспечивающей существование
электрического поля Земли. Английский исследователь Вильсон выдвинул
гипотезу о грозовом генераторе атмосферного электричества, которая в
настоящее время является общепризнанной.

\hypertarget{ux444ux43eux440ux43cux438ux440ux43eux432ux430ux43dux438ux435-ux438-ux44dux43bux435ux43aux442ux440ux438ux437ux430ux446ux438ux44f-ux43eux431ux43bux430ux43aux43eux432}{%
\subsection{Формирование и Электризация
Облаков}\label{ux444ux43eux440ux43cux438ux440ux43eux432ux430ux43dux438ux435-ux438-ux44dux43bux435ux43aux442ux440ux438ux437ux430ux446ux438ux44f-ux43eux431ux43bux430ux43aux43eux432}}

Электрическая структура грозового облака определяется его
гидродинамической и микрофизической структурой. Грозовые ячейки проходят
через три стадии эволюции. Начальная стадия характеризуется
преобладанием восходящих движений по всей толще ячейки, что активно
способствует процессу микрофизического развития. Для образования облаков
необходимо охлаждение воздуха при подъеме воздушной частицы, что
является основной причиной образования слоистообразных облаков. После
начала конденсации, если начальные скорости подъема велики, частица
становится легче, и скорость ее подъема увеличивается. Конвективные
облака образуются в неустойчивых воздушных массах, где
влажноадиабатический градиент, зависящий от температуры и давления
воздуха, играет определяющую роль.

Процесс электризации облака, то есть существенное превышение фоновых
значений электрического поля над облаком, может начаться, только когда
верхняя граница конвективного облака будет находиться выше изотермы -8°C
для СНГ или -6°C для Кубы, отражаемость по самолетному радиолокатору
превысит 0 dBZ, и в переохлажденной части облака начнется процесс
кристаллизации. Электризация облака обязательно наблюдается, если облако
выше изотермы -22°C или отражаемость выше 50 dBZ, и в переохлажденной
части облака наблюдается интенсивный процесс кристаллизации.

\hypertarget{ux43cux435ux445ux430ux43dux438ux437ux43cux44b-ux44dux43bux435ux43aux442ux440ux438ux437ux430ux446ux438ux438-ux43eux431ux43bux430ux43aux43eux432}{%
\subsubsection{Механизмы Электризации
Облаков}\label{ux43cux435ux445ux430ux43dux438ux437ux43cux44b-ux44dux43bux435ux43aux442ux440ux438ux437ux430ux446ux438ux438-ux43eux431ux43bux430ux43aux43eux432}}

Основной процесс электризации начинается с возникновения избыточного
заряда на отдельной частице или в микрообъеме воздуха, то есть со
статической электризации. Изучаются следующие механизмы электризации:

\begin{enumerate}
\def\labelenumi{\arabic{enumi}.}
\tightlist
\item
  \textbf{Ионная электризация.} Атмосферные ионы взаимодействуют с
  частицами в облаках, в результате чего может меняться заряд частиц.
  При наличии внешнего электрического поля напряженностью, наблюдаемой в
  грозовых облаках, эффективность ионного заряжения возрастает в
  зависимости от размера капель в 5-10 раз. В основном, на начальной
  стадии развития облаков и туманов изменение их электрических
  характеристик обусловлено заряжением частиц за счет ионов атмосферы.
\item
  \textbf{Электризация, возникающая в результате разрушения контакта
  облачных гидрометеоров.} Столкновение облачных гидрометеоров,
  приводящее к слиянию (коагуляции) или разрушению. Разделение зарядов
  при разрушении замерзающих капель объясняется электризацией
  кристаллизующейся воды, контактной и термоэлектрической электризацией.
  Отскок при столкновении кристалликов льда с крупной ледяной частицей
  также приводит к взаимной электризации. Наиболее мощные механизмы
  микроэлектризации ``работают'' только в облаках, содержащих ледяные
  гидрометеоры.
\item
  \textbf{Индукционная электризация гидрометеоров.} Во внешнем
  электрическом поле при разрыве контакта взаимодействующих частиц
  происходит разделение заряда, связанное с индукционным эффектом.
\item
  \textbf{Макроэлектризация облака.} Это процесс разделения разноименно
  заряженных частиц в пространстве, приводящий к преимущественному
  накоплению положительных или отрицательных зарядов в больших объемах
  облака и формированию его электрической структуры. Облака большой
  вертикальной протяженности обычно многозарядные. Многие облака, даже
  недождящие, действуют как генераторы электричества. Электропроводность
  воздуха в облаках меньше, чем в свободной атмосфере, из-за захвата
  ионов каплями; в неплотных St и Sc она уменьшается в 3-25 раз.
\end{enumerate}

\hypertarget{ux445ux430ux440ux430ux43aux442ux435ux440ux438ux441ux442ux438ux43aux438-ux44dux43bux435ux43aux442ux440ux438ux447ux435ux441ux43aux43eux433ux43e-ux43fux43eux43bux44f-ux432-ux43eux431ux43bux430ux43aux430ux445}{%
\subsection{Характеристики Электрического Поля в
Облаках}\label{ux445ux430ux440ux430ux43aux442ux435ux440ux438ux441ux442ux438ux43aux438-ux44dux43bux435ux43aux442ux440ux438ux447ux435ux441ux43aux43eux433ux43e-ux43fux43eux43bux44f-ux432-ux43eux431ux43bux430ux43aux430ux445}}

Подавляющее большинство данных о вертикальной составляющей градиента
потенциала электрического поля (E) в облаках было получено во время
самолетных зондирований. Совокупность абсолютных значений
\textbar E\textbar{} удовлетворительно аппроксимируется логарифмически
нормальным распределением. Максимальные значения \textbar E\textbar{} в
слоисто-дождевых облаках смешанного фазового строения наблюдаются в зоне
между изотермами 0°C и -10°C, где происходит интенсивное разделение
зарядов. Под облаком или в нижней части облака часто существуют области
положительных зарядов, связанные с осадками.

Электрическая активность облаков, характеризуемая средней величиной
абсолютных значений напряженности электрического поля, растет от одного
вида облака к другому примерно в следующей последовательности: St, Sc,
Ac, As, Ns, Cb. Как правило, с увеличением толщины облака возрастает его
электрическая активность, что ярче проявляется в южных районах.
Электрическая активность облаков в среднем растет от северных широт к
южным. Облака слоистообразных форм имеют значительно меньшую активность,
чем конвективные.

\hypertarget{ux43cux43eux43bux43dux438ux435ux432ux44bux435-ux440ux430ux437ux440ux44fux434ux44b}{%
\subsection{Молниевые
Разряды}\label{ux43cux43eux43bux43dux438ux435ux432ux44bux435-ux440ux430ux437ux440ux44fux434ux44b}}

Молния -- это электрический разряд в атмосфере, обычно возникающий во
время грозы и проявляющийся в виде яркой вспышки света. Разряды молний
представляют опасность как для летательных аппаратов, так и для наземных
объектов, как за счет непосредственного воздействия токов, так и мощного
электромагнитного излучения. Скачкообразное изменение электрического
поля во время разряда молнии может быть одной из основных причин
``внезапных ливней'' и сильных порывов ветра (``micro- и macrobursts'').

Молниевые разряды начинаются с лидерного процесса. Основная стадия
молнии характеризуется увеличением яркости свечения канала, мощным
звуковым эффектом (громом) и протеканием по каналу молнии импульса тока
с амплитудой до сотен килоампер и длительностью до сотен микросекунд. С
финальной стадией продолжается перенос заряда к земле по каналу молнии,
в основном за счет разрядных явлений в облаках, с током 10-10\^{}3 А.
Траектория молнии между облаком и землей определяется лидерным
процессом.

В течение последних двух десятилетий активно развивается изучение
высотных разрядов в атмосфере, таких как спрайты и гало. Их физическая
причина заключается в том, что пороговое поле пробоя воздуха падает с
высотой экспоненциально, тогда как возмущения электрического поля от
молниевых вспышек облако-земля уменьшаются медленнее (по степенному
закону) и на высоте около 75 км превышают пробойное значение.

\hypertarget{ux441ux438ux43dux43eux43fux442ux438ux447ux435ux441ux43aux438ux435-ux443ux441ux43bux43eux432ux438ux44f-ux438-ux441ux442ux430ux442ux438ux441ux442ux438ux447ux435ux441ux43aux438ux435-ux445ux430ux440ux430ux43aux442ux435ux440ux438ux441ux442ux438ux43aux438-ux433ux440ux43eux437}{%
\subsection{Синоптические Условия и Статистические Характеристики
Гроз}\label{ux441ux438ux43dux43eux43fux442ux438ux447ux435ux441ux43aux438ux435-ux443ux441ux43bux43eux432ux438ux44f-ux438-ux441ux442ux430ux442ux438ux441ux442ux438ux447ux435ux441ux43aux438ux435-ux445ux430ux440ux430ux43aux442ux435ux440ux438ux441ux442ux438ux43aux438-ux433ux440ux43eux437}}

Особенности гроз определяются многими факторами, в первую очередь
особенностями окружающей среды, то есть атмосферой. В этой связи грозы
различают в зависимости от широты места, подстилающей поверхности,
времени года и суток, а также синоптической ситуации. При
экспериментальном исследовании грозы различают в зависимости от
расположения и величины основных зарядов, скорости заряжения, частоты
разрядов и пр..

\hypertarget{ux440ux430ux441ux43fux440ux435ux434ux435ux43bux435ux43dux438ux435-ux433ux440ux43eux437ux43eux432ux43eux439-ux434ux435ux44fux442ux435ux43bux44cux43dux43eux441ux442ux438-1}{%
\subsubsection{Распределение Грозовой
Деятельности}\label{ux440ux430ux441ux43fux440ux435ux434ux435ux43bux435ux43dux438ux435-ux433ux440ux43eux437ux43eux432ux43eux439-ux434ux435ux44fux442ux435ux43bux44cux43dux43eux441ux442ux438-1}}

Вблизи земной поверхности происходит около 100 разрядов молний в 1
секунду, поэтому в любой точке земного шара практически непрерывно можно
регистрировать электрические сигналы, создаваемые радиоволнами,
излучаемыми молниями, называемые атмосфериками. Анализ суточного
распределения молниевых вспышек по поверхности земного шара показал, что
период максимальной грозовой активности согласуется с максимумом
``унитарной вариации''.

\hypertarget{ux441ux438ux43dux43eux43fux442ux438ux447ux435ux441ux43aux438ux435-ux443ux441ux43bux43eux432ux438ux44f-ux434ux43bux44f-ux432ux43eux437ux43dux438ux43aux43dux43eux432ux435ux43dux438ux44f-ux433ux440ux43eux437}{%
\subsubsection{Синоптические Условия для Возникновения
Гроз}\label{ux441ux438ux43dux43eux43fux442ux438ux447ux435ux441ux43aux438ux435-ux443ux441ux43bux43eux432ux438ux44f-ux434ux43bux44f-ux432ux43eux437ux43dux438ux43aux43dux43eux432ux435ux43dux438ux44f-ux433ux440ux43eux437}}

Ливневые осадки и грозы тесно связаны с неустойчивостью атмосферы. Они
характерны для влажных, неустойчивых воздушных масс. Наиболее
благоприятные условия для возникновения ливневых осадков и гроз --- это
размытые барические поля, слабо выраженные и заполняющиеся циклоны,
западная окраина антициклонов, а иногда и теплый сектор циклона.
Фронтальные ливневые осадки и грозы часто связаны с медленно
перемещающимися холодными фронтами и с размытыми фронтами окклюзии. В
летнее время ливневые осадки и грозы (особенно ночные) довольно часто
связаны с прохождением теплых фронтов при неустойчивой теплой воздушной
массе. Зимой ``снежные'' грозы --- крайне редкое явление.

Бароклинность атмосферы играет важную, нередко определяющую роль в
зарождении и эволюции синоптических вихрей (циклонов и антициклонов).
Циклоны зарождаются на холодных фронтах, с которыми связана адвекция
холода. Адвекция холодного воздуха над теплым течением усиливает
вихреобразование. Бароклинный фактор (геострофическая адвекция
виртуальной температуры) играет определяющую роль в формировании и
эволюции облаков всех классов.

\hypertarget{ux437ux430ux434ux430ux447ux438-ux438ux437ux443ux447ux435ux43dux438ux44f-ux438-ux43fux440ux43eux433ux43dux43eux437ux438ux440ux43eux432ux430ux43dux438ux44f-ux433ux440ux43eux437}{%
\subsection{Задачи Изучения и Прогнозирования
Гроз}\label{ux437ux430ux434ux430ux447ux438-ux438ux437ux443ux447ux435ux43dux438ux44f-ux438-ux43fux440ux43eux433ux43dux43eux437ux438ux440ux43eux432ux430ux43dux438ux44f-ux433ux440ux43eux437}}

Основными задачами, исследуемыми в области электричества облаков,
являются прогноз гроз, борьба с электростатической опасностью,
возникающей при полете самолетов в облаках, и создание средств
регулирования электрической активности облаков.

Прогноз гроз тесно связан с прогнозом ливневых осадков. При прогнозе
этих явлений вполне обосновано применение терминов ``временами'' и
``местами'', так как они являются локальными и неоднократно
повторяющимися, что делает конкретный прогноз времени и места их
появления крайне сложным. Существуют различные методики прогноза гроз, в
том числе по Уайтингу, Фаусту, Н. П. Фатееву и И. А. Славину.

ewpage

\hypertarget{ux43eux431ux449ux438ux435-ux43fux43eux43dux44fux442ux438ux44f-ux430ux43aux443ux441ux442ux438ux43aux438-ux430ux442ux43cux43eux441ux444ux435ux440ux44b}{%
\section{Общие понятия акустики
атмосферы}\label{ux43eux431ux449ux438ux435-ux43fux43eux43dux44fux442ux438ux44f-ux430ux43aux443ux441ux442ux438ux43aux438-ux430ux442ux43cux43eux441ux444ux435ux440ux44b}}

Коллега, рад представить краткую сводку по ключевым аспектам акустики
атмосферы. Это междисциплинарная область, находящаяся на стыке
метеорологии, физики и гидродинамики, изучающая генерацию,
распространение и прием звуковых волн в реальной атмосфере.

\begin{center}\rule{0.5\linewidth}{0.5pt}\end{center}

\hypertarget{ux441ux43aux43eux440ux43eux441ux442ux44c-ux437ux432ux443ux43aux430}{%
\subsection{Скорость
звука}\label{ux441ux43aux43eux440ux43eux441ux442ux44c-ux437ux432ux443ux43aux430}}

Фундаментальной характеристикой является скорость распространения звука.
Для идеального газа она определяется формулой Лапласа:

\[c = \sqrt{\gamma \frac{P}{\rho}}\]

где \(\gamma\) --- показатель адиабаты (\(c_p/c_v\)), \(P\) ---
давление, а \(\rho\) --- плотность.

В метеорологической практике удобнее использовать выражение через
температуру:

\[c = \sqrt{\gamma R T_v} \approx 20.05 \sqrt{T_v}\]

Здесь \(R\) --- удельная газовая постоянная для сухого воздуха, а
\(T_v\) --- \textbf{виртуальная температура}, учитывающая влияние
влажности. Таким образом, \textbf{основным фактором, определяющим
скорость звука, является температура}. В приземном слое часто используют
эмпирическую формулу:

\[c \approx 331.3 + 0.606 \cdot T_C\]

где \(T_C\) --- температура в градусах Цельсия.

Необходимо также учитывать \textbf{ветер}, который векторно складывается
со скоростью звука. Эффективная скорость звука \(\vec{c}_{eff}\)
относительно земли:

\[\vec{c}_{eff} = \vec{c} + \vec{u}\]

где \(\vec{u}\) --- вектор скорости ветра. Вертикальный градиент ветра
является критическим фактором для рефракции.

\begin{center}\rule{0.5\linewidth}{0.5pt}\end{center}

\hypertarget{ux440ux435ux444ux440ux430ux43aux446ux438ux44f-ux437ux432ux443ux43aux43eux432ux44bux445-ux432ux43eux43bux43d}{%
\subsection{Рефракция звуковых
волн}\label{ux440ux435ux444ux440ux430ux43aux446ux438ux44f-ux437ux432ux443ux43aux43eux432ux44bux445-ux432ux43eux43bux43d}}

Рефракция, или искривление траектории звуковых лучей, --- ключевое
явление в атмосферной акустике. Оно обусловлено пространственной
неоднородностью поля эффективной скорости звука. Траектория луча
подчиняется акустическому аналогу закона Снеллиуса.

\hypertarget{ux442ux435ux43cux43fux435ux440ux430ux442ux443ux440ux43dux430ux44f-ux440ux435ux444ux440ux430ux43aux446ux438ux44f}{%
\subsubsection{Температурная
рефракция}\label{ux442ux435ux43cux43fux435ux440ux430ux442ux443ux440ux43dux430ux44f-ux440ux435ux444ux440ux430ux43aux446ux438ux44f}}

\begin{itemize}
\tightlist
\item
  \textbf{Стандартная стратификация:} При стандартном температурном
  градиенте (\(\frac{dT}{dz} < 0\)) скорость звука уменьшается с
  высотой. Звуковые лучи изгибаются \textbf{вверх}, отрываясь от
  поверхности и образуя \textbf{зону акустической тени}.
\item
  \textbf{Температурная инверсия:} При инверсии (\(\frac{dT}{dz} > 0\)),
  часто наблюдаемой ночью или в арктических регионах, скорость звука
  растет с высотой. Лучи изгибаются \textbf{вниз}, к поверхности. Это
  приводит к аномально дальнему распространению звука и формированию
  акустических волноводов.
\end{itemize}

\hypertarget{ux432ux435ux442ux440ux43eux432ux430ux44f-ux440ux435ux444ux440ux430ux43aux446ux438ux44f}{%
\subsubsection{Ветровая
рефракция}\label{ux432ux435ux442ux440ux43eux432ux430ux44f-ux440ux435ux444ux440ux430ux43aux446ux438ux44f}}

Вертикальный сдвиг ветра оказывает не менее значимое влияние.

\begin{itemize}
\tightlist
\item
  \textbf{Распространение по ветру:} Скорость ветра обычно увеличивается
  с высотой. Это добавляет положительный градиент к \(c_{eff}\), вызывая
  рефракцию \textbf{вниз}, аналогично температурной инверсии. Слышимость
  по ветру значительно улучшается.
\item
  \textbf{Распространение против ветра:} Скорость ветра вычитается из
  скорости звука, и ее градиент заставляет лучи изгибаться
  \textbf{вверх}, усиливая эффект акустической тени.
\end{itemize}

Часто температурные и ветровые эффекты действуют совместно, создавая
сложную картину распространения.

\begin{center}\rule{0.5\linewidth}{0.5pt}\end{center}

\hypertarget{ux437ux430ux442ux443ux445ux430ux43dux438ux435-ux437ux432ux443ux43aux430}{%
\subsection{Затухание
звука}\label{ux437ux430ux442ux443ux445ux430ux43dux438ux435-ux437ux432ux443ux43aux430}}

По мере распространения интенсивность звука уменьшается из-за ряда
факторов:

\begin{enumerate}
\def\labelenumi{\arabic{enumi}.}
\tightlist
\item
  \textbf{Геометрическое расхождение:} Расширение волнового фронта
  (закон обратных квадратов для сферической волны).
\item
  \textbf{Классическое поглощение:} Вызвано вязкостью и
  теплопроводностью среды. Его вклад значителен только на очень высоких
  частотах (\textgreater10 кГц) и пропорционален квадрату частоты
  (\(f^2\)).
\item
  \textbf{Молекулярное поглощение:} \textbf{Основной механизм затухания}
  в слышимом диапазоне частот. Связан с релаксационными процессами ---
  возбуждением колебательных уровней молекул кислорода (\(O_2\)) и азота
  (\(N_2\)). Этот процесс крайне чувствителен к \textbf{температуре и
  относительной влажности}.
\item
  \textbf{Рассеяние на турбулентностях:} Турбулентные вихри рассеивают
  акустическую энергию, что приводит к флуктуациям амплитуды и фазы
  сигнала.
\item
  \textbf{Взаимодействие с поверхностью:} Поглощение и отражение звука
  от земной поверхности, которое зависит от ее типа (трава, асфальт,
  снег).
\end{enumerate}

\begin{center}\rule{0.5\linewidth}{0.5pt}\end{center}

\hypertarget{ux438ux43dux444ux440ux430ux437ux432ux443ux43a-ux438-ux434ux430ux43bux44cux43dux435ux435-ux440ux430ux441ux43fux440ux43eux441ux442ux440ux430ux43dux435ux43dux438ux435}{%
\subsection{Инфразвук и дальнее
распространение}\label{ux438ux43dux444ux440ux430ux437ux432ux443ux43a-ux438-ux434ux430ux43bux44cux43dux435ux435-ux440ux430ux441ux43fux440ux43eux441ux442ux440ux430ux43dux435ux43dux438ux435}}

Особый интерес представляет инфразвук (частоты \textless{} 20 Гц).

\begin{itemize}
\tightlist
\item
  \textbf{Низкое затухание:} Благодаря низкой частоте, инфразвук
  испытывает очень слабое молекулярное поглощение и может
  распространяться на тысячи километров.
\item
  \textbf{Атмосферные волноводы:} Профиль температуры атмосферы имеет
  локальные максимумы в стратосфере (\textasciitilde50 км) и термосфере
  (\textgreater90 км). Эти слои действуют как верхние границы гигантских
  акустических волноводов. Звук, претерпевший рефракцию на этих высотах,
  возвращается на землю в \textbf{зонах аномальной слышимости}, которые
  отделены от источника первичной \textbf{зоной молчания}.
\end{itemize}

Это явление позволяет осуществлять глобальный мониторинг мощных
источников инфразвука, таких как извержения вулканов, ядерные испытания
и запуски ракет.

В заключение, адекватное описание распространения звука в атмосфере
требует комплексного учета ее термодинамического и динамического
состояния.

ewpage

\hypertarget{ux441ux43aux43eux440ux43eux441ux442ux44c-ux437ux432ux443ux43aux430-ux432-ux430ux442ux43cux43eux441ux444ux435ux440ux435}{%
\section{Скорость звука в
атмосфере}\label{ux441ux43aux43eux440ux43eux441ux442ux44c-ux437ux432ux443ux43aux430-ux432-ux430ux442ux43cux43eux441ux444ux435ux440ux435}}

Скорость звука является важной характеристикой атмосферы, особенно при
рассмотрении динамических процессов и распространения возмущений.

\hypertarget{ux43eux43fux440ux435ux434ux435ux43bux435ux43dux438ux435-ux438-ux444ux438ux437ux438ux447ux435ux441ux43aux438ux439-ux441ux43cux44bux441ux43b}{%
\subsection{Определение и физический
смысл}\label{ux43eux43fux440ux435ux434ux435ux43bux435ux43dux438ux435-ux438-ux444ux438ux437ux438ux447ux435ux441ux43aux438ux439-ux441ux43cux44bux441ux43b}}

Скорость звука (обозначаемая как \(c\)) характеризует скорость
распространения колебаний плотности в среде. Это фундаментальная
характеристика, показывающая, как быстро могут распространяться
механические возмущения, такие как изменения плотности, в воздушной
среде.

\hypertarget{ux437ux430ux432ux438ux441ux438ux43cux43eux441ux442ux44c-ux43eux442-ux43fux430ux440ux430ux43cux435ux442ux440ux43eux432-ux430ux442ux43cux43eux441ux444ux435ux440ux44b}{%
\subsection{Зависимость от параметров
атмосферы}\label{ux437ux430ux432ux438ux441ux438ux43cux43eux441ux442ux44c-ux43eux442-ux43fux430ux440ux430ux43cux435ux442ux440ux43eux432-ux430ux442ux43cux43eux441ux444ux435ux440ux44b}}

Скорость распространения звука в воздухе связана с отношением давления
(\(P\)) к плотности (\(\rho\)) воздуха. В первом приближении эта
зависимость может быть выражена формулой \(c^2 = \chi \frac{P}{\rho}\),
где \(\chi\) -- это адиабатический показатель (для воздуха это отношение
теплоемкостей при постоянном давлении и постоянном объеме).

\hypertarget{ux442ux438ux43fux438ux447ux43dux44bux435-ux437ux43dux430ux447ux435ux43dux438ux44f}{%
\subsection{Типичные
значения}\label{ux442ux438ux43fux438ux447ux43dux44bux435-ux437ux43dux430ux447ux435ux43dux438ux44f}}

В атмосфере Земли характерное значение скорости звука составляет около
300 м/с.

\hypertarget{ux437ux43dux430ux447ux435ux43dux438ux435-ux447ux438ux441ux43bux430-ux43cux430ux445ux430}{%
\subsection{Значение числа
Маха}\label{ux437ux43dux430ux447ux435ux43dux438ux435-ux447ux438ux441ux43bux430-ux43cux430ux445ux430}}

Отношение скорости движения воздуха (\(V\)) к скорости звука (\(c\))
называется числом Маха (\(Ma = V/c\)). Этот безразмерный критерий
подобия имеет большое значение для оценки влияния сжимаемости воздуха на
атмосферные движения:

\begin{itemize}
\tightlist
\item
  При малых значениях числа Маха (менее 0.2) уравнение неразрывности для
  воздуха лишь незначительно отличается от уравнения для несжимаемой
  жидкости, что позволяет во многих метеорологических задачах
  пренебрегать сжимаемостью воздуха. Это объясняется тем, что плотность
  каждой частицы воздуха практически не меняется в процессе ее движения
  при таких скоростях.
\item
  Однако при скоростях движения воздуха, близких к скорости звука (т.е.
  при больших числах Маха), сжимаемость воздуха начинает оказывать
  существенное влияние на свойства движения, и число Маха становится
  важнейшим критерием подобия.
\end{itemize}

\hypertarget{ux43eux442ux43dux43eux441ux438ux442ux435ux43bux44cux43dux430ux44f-ux441ux43aux43eux440ux43eux441ux442ux44c-ux430ux442ux43cux43eux441ux444ux435ux440ux43dux44bux445-ux434ux432ux438ux436ux435ux43dux438ux439}{%
\subsection{Относительная скорость атмосферных
движений}\label{ux43eux442ux43dux43eux441ux438ux442ux435ux43bux44cux43dux430ux44f-ux441ux43aux43eux440ux43eux441ux442ux44c-ux430ux442ux43cux43eux441ux444ux435ux440ux43dux44bux445-ux434ux432ux438ux436ux435ux43dux438ux439}}

По сравнению со скоростью звука в воздухе, большинство атмосферных
движений являются относительно медленными. Это позволяет во многих
случаях упрощать уравнения динамики атмосферы, используя предположение о
несжимаемости воздуха.

ewpage

\hypertarget{ux443ux441ux43bux43eux432ux438ux44f-ux440ux430ux441ux43fux440ux43eux441ux442ux440ux430ux43dux435ux43dux438ux44f-ux437ux432ux443ux43aux43eux432ux44bux445-ux432ux43eux43bux43d-ux432-ux430ux442ux43cux43eux441ux444ux435ux440ux435}{%
\section{Условия распространения звуковых волн в
атмосфере}\label{ux443ux441ux43bux43eux432ux438ux44f-ux440ux430ux441ux43fux440ux43eux441ux442ux440ux430ux43dux435ux43dux438ux44f-ux437ux432ux443ux43aux43eux432ux44bux445-ux432ux43eux43bux43d-ux432-ux430ux442ux43cux43eux441ux444ux435ux440ux435}}

\hypertarget{ux43eux441ux43dux43eux432ux44b-ux430ux442ux43cux43eux441ux444ux435ux440ux43dux43eux439-ux430ux43aux443ux441ux442ux438ux43aux438}{%
\subsection{Основы атмосферной
акустики}\label{ux43eux441ux43dux43eux432ux44b-ux430ux442ux43cux43eux441ux444ux435ux440ux43dux43eux439-ux430ux43aux443ux441ux442ux438ux43aux438}}

Атмосферная акустика как область метеорологии занимается изучением
звуков атмосферного происхождения и, что более важно, ролью атмосферы в
распространении звука. Скорость распространения звука в атмосфере
составляет в среднем около 300 м/с. Распространение звуковых волн в
атмосфере, по своей сути, представляет собой колебания плотности,
распространяющиеся в пространстве. Эти процессы находятся в тесной
взаимосвязи с динамическими и термодинамическими характеристиками
воздушной среды.

\hypertarget{ux432ux43bux438ux44fux43dux438ux435-ux430ux442ux43cux43eux441ux444ux435ux440ux43dux44bux445-ux43fux430ux440ux430ux43cux435ux442ux440ux43eux432}{%
\subsection{Влияние атмосферных
параметров}\label{ux432ux43bux438ux44fux43dux438ux435-ux430ux442ux43cux43eux441ux444ux435ux440ux43dux44bux445-ux43fux430ux440ux430ux43cux435ux442ux440ux43eux432}}

Распространение звуковых волн в атмосфере подвержено влиянию целого ряда
метеорологических параметров и явлений:

\hypertarget{ux442ux435ux440ux43cux43eux434ux438ux43dux430ux43cux438ux447ux435ux441ux43aux438ux435-ux43fux430ux440ux430ux43cux435ux442ux440ux44b}{%
\subsubsection{Термодинамические
параметры}\label{ux442ux435ux440ux43cux43eux434ux438ux43dux430ux43cux438ux447ux435ux441ux43aux438ux435-ux43fux430ux440ux430ux43cux435ux442ux440ux44b}}

Скорость звука в воздухе напрямую зависит от его температуры и,
косвенно, от плотности. Изменения температуры с высотой, такие как в
тропосфере (где температура падает примерно на 0.65°C на каждые 100 м
высоты) или в стратосфере (где температура может расти с высотой),
приводят к изменению скорости звука. Это вызывает рефракцию звуковых
волн -- их искривление в сторону более холодных, а следовательно, более
плотных слоев воздуха. Сжимаемость воздуха также является важным
фактором, влияющим на свойства движения, близкие к скорости звука.
Влажность воздуха, характеризуемая парциальным давлением водяного пара и
влияющая на виртуальную температуру, изменяет эффективную плотность
воздуха и, как следствие, скорость звука.

\hypertarget{ux434ux438ux43dux430ux43cux438ux447ux435ux441ux43aux438ux435-ux43fux440ux43eux446ux435ux441ux441ux44b}{%
\subsubsection{Динамические
процессы}\label{ux434ux438ux43dux430ux43cux438ux447ux435ux441ux43aux438ux435-ux43fux440ux43eux446ux435ux441ux441ux44b}}

\begin{enumerate}
\def\labelenumi{\arabic{enumi}.}
\tightlist
\item
  \textbf{Поле ветра:} Векторное поле ветра, характеризующее воздушные
  течения, оказывает существенное влияние на распространение звука, так
  как среда, в которой распространяется волна, сама движется. Это
  приводит к смещению звуковой энергии по направлению ветра и, в случае
  изменения скорости или направления ветра с высотой (сдвиг ветра), к
  рефракции звуковых лучей.
\item
  \textbf{Турбулентность:} Атмосферные движения носят преимущественно
  турбулентный характер, проявляясь в резких беспорядочных колебаниях
  скорости. Турбулентность создает неоднородности плотности и скорости,
  которые могут вызывать рассеяние и поглощение звуковых волн,
  аналогично рассеянию и дифракции света на хаотически расположенных
  неоднородностях плотности воздуха и атмосферных частицах.
\item
  \textbf{Аэрозоли и гидрометеоры:} Присутствие в атмосфере взвешенных
  частиц (аэрозолей) или капелек воды/кристаллов льда (гидрометеоров,
  таких как туман, облака, осадки) может также влиять на распространение
  звука. Хотя источники напрямую не описывают влияние этих частиц на
  звук, они отмечают их способность рассеивать и поглощать излучение,
  что по аналогии применимо и к звуковым волнам. Например, подвижность
  ионов уменьшается в слоях пыли и облаков, указывая на влияние частиц
  на свойства среды.
\end{enumerate}

\hypertarget{ux430ux442ux43cux43eux441ux444ux435ux440ux43dux430ux44f-ux441ux442ux440ux430ux442ux438ux444ux438ux43aux430ux446ux438ux44f}{%
\subsection{Атмосферная
стратификация}\label{ux430ux442ux43cux43eux441ux444ux435ux440ux43dux430ux44f-ux441ux442ux440ux430ux442ux438ux444ux438ux43aux430ux446ux438ux44f}}

Вертикальное распределение температуры, давления и влажности
(стратификация) формирует слои с различными акустическими свойствами.
Например, в приземном слое, где происходит интенсивный турбулентный
теплообмен с подстилающей поверхностью, суточные колебания температуры
распространяются до высоты около 1.5 км, образуя так называемый
пограничный слой атмосферы. Выше него, в свободной атмосфере, сила
турбулентного трения значительно меньше. Изменения температуры и ветра в
этих слоях могут приводить к формированию акустических волноводов
(каналов), где звук может распространяться на значительные расстояния с
меньшим затуханием, или, напротив, к зонам ``акустической тени'', куда
звук не проникает из-за сильной рефракции. Глобально, атмосфера
разделена на тропосферу, стратосферу, мезосферу и термосферу, каждая из
которых имеет свой характерный температурный профиль, что определяет
вертикальное изменение скорости звука и, как следствие, траектории
звуковых лучей.

ewpage

\hypertarget{ux431ux43bux43eux43a-1.-ux444ux438ux437ux438ux43aux430-ux430ux442ux43cux43eux441ux444ux435ux440ux44b-6}{%
\section{Блок 1. «Физика
атмосферы»}\label{ux431ux43bux43eux43a-1.-ux444ux438ux437ux438ux43aux430-ux430ux442ux43cux43eux441ux444ux435ux440ux44b-6}}

\hypertarget{ux442ux435ux43cux430-ux438ux43eux43dux43eux441ux444ux435ux440ux430-ux435ux435-ux43eux441ux43dux43eux432ux43dux44bux435-ux445ux430ux440ux430ux43aux442ux435ux440ux438ux441ux442ux438ux43aux438}{%
\subsection{1.7. Тема «Ионосфера, ее основные
характеристики»}\label{ux442ux435ux43cux430-ux438ux43eux43dux43eux441ux444ux435ux440ux430-ux435ux435-ux43eux441ux43dux43eux432ux43dux44bux435-ux445ux430ux440ux430ux43aux442ux435ux440ux438ux441ux442ux438ux43aux438}}

\hypertarget{ux438ux43eux43dux43eux441ux444ux435ux440ux430-ux435ux435-ux43eux441ux43dux43eux432ux43dux44bux435-ux445ux430ux440ux430ux43aux442ux435ux440ux438ux441ux442ux438ux43aux438}{%
\subsubsection{\texorpdfstring{\textbf{Ионосфера, ее основные
характеристики}}{Ионосфера, ее основные характеристики}}\label{ux438ux43eux43dux43eux441ux444ux435ux440ux430-ux435ux435-ux43eux441ux43dux43eux432ux43dux44bux435-ux445ux430ux440ux430ux43aux442ux435ux440ux438ux441ux442ux438ux43aux438}}

\textbf{Ионосфера} --- это ионизированная часть верхней атмосферы Земли,
простирающаяся примерно от 60 км до 1000 км. Она представляет собой
плазму, состоящую из смеси нейтральных атомов и молекул, положительных
ионов и свободных электронов.

\begin{itemize}
\item
  \textbf{Причина образования:} Ионизация происходит под действием
  жёсткого коротковолнового (ультрафиолетового и рентгеновского)
  излучения Солнца, а также космических лучей. Эти излучения обладают
  достаточной энергией, чтобы выбивать электроны из атомов и молекул
  атмосферных газов (\(O_2, N_2, O, N\)).
  \[X + h\nu \rightarrow X^+ + e^-\] где \(X\) --- нейтральная частица,
  \(h\nu\) --- квант излучения, \(X^+\) --- положительный ион, \(e^-\)
  --- свободный электрон.
\item
  \textbf{Структура и слои:} Ионосфера имеет слоистую структуру, которая
  сильно меняется в зависимости от времени суток, сезона и уровня
  солнечной активности. Слои характеризуются максимумами концентрации
  электронов (\(N_e\)).

  \begin{itemize}
  \tightlist
  \item
    \textbf{Слой D (60--90 км):} Самый нижний и наименее плотный слой.
    Существует только днём, так как ночью ионы и электроны быстро
    рекомбинируют. Поглощает длинные и средние радиоволны.
  \item
    \textbf{Слой E (90--150 км):} Также в основном дневное явление.
    Отражает средние радиоволны, обеспечивая радиосвязь на расстояниях в
    сотни километров. Иногда в нём возникают спорадические, тонкие и
    очень плотные слои (\(E_s\)), вызывающие дальнее распространение
    УКВ.
  \item
    \textbf{Слой F:} Самый высокий и плотный слой. Днём он разделяется
    на два подслоя: \textbf{F1 (150--220 км)} и \textbf{F2 (220--500
    км)}. Ночью они сливаются в один слой F. Слой F2 является основным
    слоем, отражающим короткие радиоволны, что обеспечивает дальнюю и
    сверхдальнюю радиосвязь по всему миру.
  \end{itemize}
\item
  \textbf{Основные характеристики:}

  \begin{itemize}
  \tightlist
  \item
    \textbf{Электронная концентрация (\(N_e\)):} Количество свободных
    электронов в единице объёма (эл/м³). Достигает максимума
    (\(10^{11}-10^{12}\) эл/м³) в слое F2 днём.
  \item
    \textbf{Плазменная частота (\(f_p\)):} Собственная частота колебаний
    электронов в плазме. Радиоволны с частотой ниже плазменной
    отражаются от ионосферного слоя. \(f_p \propto \sqrt{N_e}\).
  \item
    \textbf{Ионосферные возмущения:} Состояние ионосферы сильно зависит
    от солнечной активности. Во время солнечных вспышек рентгеновское
    излучение вызывает внезапные ионосферные возмущения, нарушая
    радиосвязь. Потоки заряженных частиц от Солнца вызывают ионосферные
    и магнитные бури, а также полярные сияния.
  \end{itemize}
\end{itemize}

\hypertarget{ux44dux43bux435ux43aux442ux440ux438ux447ux435ux441ux43aux43eux435-ux43fux43eux43bux435-ux43eux431ux43bux430ux43aux43eux432.-ux43eux441ux43dux43eux432ux44b-ux442ux435ux43eux440ux438ux438-ux433ux440ux43eux437ux43eux432ux43eux433ux43e-ux44dux43bux435ux43aux442ux440ux438ux447ux435ux441ux442ux432ux430}{%
\subsubsection{\texorpdfstring{\textbf{Электрическое поле облаков.
Основы теории грозового
электричества}}{Электрическое поле облаков. Основы теории грозового электричества}}\label{ux44dux43bux435ux43aux442ux440ux438ux447ux435ux441ux43aux43eux435-ux43fux43eux43bux435-ux43eux431ux43bux430ux43aux43eux432.-ux43eux441ux43dux43eux432ux44b-ux442ux435ux43eux440ux438ux438-ux433ux440ux43eux437ux43eux432ux43eux433ux43e-ux44dux43bux435ux43aux442ux440ux438ux447ux435ux441ux442ux432ux430}}

Мощные кучево-дождевые облака (Cb) действуют как гигантские
электростатические генераторы, создавая сильные электрические поля.

\begin{itemize}
\tightlist
\item
  \textbf{Механизмы электризации:} Основной механизм разделения зарядов
  в грозовом облаке --- \textbf{неиндуктивный}, связанный со
  столкновениями ледяных кристаллов и частиц ледяной крупы (граупеля) в
  присутствии переохлаждённых капель воды.

  \begin{itemize}
  \tightlist
  \item
    \textbf{Процесс:} В восходящих потоках более лёгкие ледяные
    кристаллы сталкиваются с более тяжёлыми частицами граупеля. При
    столкновении происходит перенос заряда: граупель заряжается
    \textbf{отрицательно}, а ледяные кристаллы ---
    \textbf{положительно}.
  \item
    \textbf{Разделение зарядов:} Сильные восходящие потоки уносят лёгкие
    положительно заряженные кристаллы в верхнюю часть облака
    (наковальню), в то время как тяжёлый отрицательно заряженный
    граупель концентрируется в средней и нижней частях облака (в области
    температур от -10 до -20°C).
  \end{itemize}
\item
  \textbf{Структура заряда грозового облака:} Классическая модель ---
  \textbf{вертикальный диполь} (или триполь).

  \begin{itemize}
  \tightlist
  \item
    Основной \textbf{положительный заряд} находится в верхней части
    облака.
  \item
    Основной \textbf{отрицательный заряд} --- в средней/нижней части.
  \item
    Часто у основания облака, в области положительных температур,
    формируется небольшой \textbf{нижний положительный заряд}.
  \end{itemize}
\item
  \textbf{Электрическое поле:} Разделение зарядов создаёт мощное
  электрическое поле как внутри облака, так и между облаком и землёй.
  Напряжённость поля может достигать сотен тысяч вольт на метр. Когда
  она превышает диэлектрическую прочность воздуха (около 3·10⁶ В/м),
  происходит искровой разряд --- \textbf{молния}.
\end{itemize}

\hypertarget{ux44dux43bux435ux43aux442ux440ux438ux447ux435ux441ux43aux438ux435-ux442ux43eux43aux438-ux43eux441ux430ux434ux43aux43eux432.-ux433ux440ux43eux437ux430-ux43aux430ux43a-ux430ux442ux43cux43eux441ux444ux435ux440ux43dux43eux435-ux44fux432ux43bux435ux43dux438ux435}{%
\subsubsection{\texorpdfstring{\textbf{Электрические токи осадков. Гроза
как атмосферное
явление}}{Электрические токи осадков. Гроза как атмосферное явление}}\label{ux44dux43bux435ux43aux442ux440ux438ux447ux435ux441ux43aux438ux435-ux442ux43eux43aux438-ux43eux441ux430ux434ux43aux43eux432.-ux433ux440ux43eux437ux430-ux43aux430ux43a-ux430ux442ux43cux43eux441ux444ux435ux440ux43dux43eux435-ux44fux432ux43bux435ux43dux438ux435}}

\begin{itemize}
\item
  \textbf{Электрические токи осадков:} Частицы осадков (капли дождя,
  снежинки, градины) несут электрический заряд. Выпадая, они создают
  \textbf{ток осадков}. В большинстве случаев дождь переносит на землю
  отрицательный заряд, что соответствует основной структуре заряда
  облака. Этот ток является одним из компонентов глобальной
  электрической цепи.
\item
  \textbf{Гроза как явление:} Это комплексное метеорологическое явление,
  связанное с кучево-дождевыми облаками и включающее:

  \begin{itemize}
  \tightlist
  \item
    \textbf{Молнию:} Гигантский искровой разряд. Большинство разрядов
    (около 80\%) происходит внутри облака или между облаками. Разряды
    ``облако-земля'' в основном переносят отрицательный заряд к земле,
    тем самым поддерживая её отрицательный заряд и замыкая глобальную
    электрическую цепь.
  \item
    \textbf{Гром:} Звуковая волна, генерируемая быстрым расширением
    воздуха вдоль канала молнии.
  \item
    \textbf{Ливневые осадки, град и шквалистые ветры.}
  \end{itemize}
\item
  \textbf{Статистические характеристики гроз:} Грозовая деятельность
  максимальна в тропических континентальных районах (например,
  Центральная Африка), где наблюдается более 100-150 дней с грозой в
  год. В умеренных широтах --- 10-30 дней в год. В полярных широтах
  грозы --- крайне редкое явление. Одновременно на Земле происходит
  около 1500-2000 гроз.
\end{itemize}

\hypertarget{ux43eux431ux449ux438ux435-ux43fux43eux43dux44fux442ux438ux44f-ux430ux43aux443ux441ux442ux438ux43aux438-ux430ux442ux43cux43eux441ux444ux435ux440ux44b.-ux441ux43aux43eux440ux43eux441ux442ux44c-ux437ux432ux443ux43aux430}{%
\subsubsection{\texorpdfstring{\textbf{Общие понятия акустики атмосферы.
Скорость
звука}}{Общие понятия акустики атмосферы. Скорость звука}}\label{ux43eux431ux449ux438ux435-ux43fux43eux43dux44fux442ux438ux44f-ux430ux43aux443ux441ux442ux438ux43aux438-ux430ux442ux43cux43eux441ux444ux435ux440ux44b.-ux441ux43aux43eux440ux43eux441ux442ux44c-ux437ux432ux443ux43aux430}}

\textbf{Атмосферная акустика} изучает генерацию и распространение
звуковых волн в атмосфере.

\begin{itemize}
\tightlist
\item
  \textbf{Скорость звука (\(c_s\)):} В идеальном газе скорость звука
  зависит только от температуры: \[c_s = \sqrt{\gamma R T}\] где
  \(\gamma = c_p/c_v \approx 1.4\) для воздуха, \(R\) --- газовая
  постоянная, \(T\) --- абсолютная температура. Для практических
  расчётов: \(c_s \approx 20.05 \sqrt{T}\).

  \begin{itemize}
  \tightlist
  \item
    \textbf{Зависимость от температуры:} Скорость звука
    \textbf{увеличивается с ростом температуры}. При \(T=0^\circ C\)
    (273 K) скорость звука составляет около 331 м/с.
  \item
    \textbf{Влияние влажности и давления:} Влияние влажности и давления
    на скорость звука в реальной атмосфере незначительно и обычно им
    пренебрегают.
  \end{itemize}
\end{itemize}

\hypertarget{ux443ux441ux43bux43eux432ux438ux44f-ux440ux430ux441ux43fux440ux43eux441ux442ux440ux430ux43dux435ux43dux438ux44f-ux437ux432ux443ux43aux43eux432ux44bux445-ux432ux43eux43bux43d-ux432-ux430ux442ux43cux43eux441ux444ux435ux440ux435-1}{%
\subsubsection{\texorpdfstring{\textbf{Условия распространения звуковых
волн в
атмосфере}}{Условия распространения звуковых волн в атмосфере}}\label{ux443ux441ux43bux43eux432ux438ux44f-ux440ux430ux441ux43fux440ux43eux441ux442ux440ux430ux43dux435ux43dux438ux44f-ux437ux432ux443ux43aux43eux432ux44bux445-ux432ux43eux43bux43d-ux432-ux430ux442ux43cux43eux441ux444ux435ux440ux435-1}}

Распространение звука в атмосфере отличается от распространения в
однородной среде из-за \textbf{рефракции} (искривления) звуковых лучей,
вызванной неоднородностями полей температуры и ветра.

\begin{itemize}
\tightlist
\item
  \textbf{Температурная рефракция:} Звуковые лучи всегда изгибаются в
  сторону слоя с \textbf{меньшей скоростью звука}, то есть в сторону
  более холодного воздуха.

  \begin{itemize}
  \tightlist
  \item
    \textbf{Стандартная тропосфера:} При стандартном падении температуры
    с высотой звуковые лучи изгибаются \textbf{вверх}, от земли. Это
    может приводить к образованию \textbf{зон молчания} на некотором
    расстоянии от источника.
  \item
    \textbf{Температурная инверсия:} При инверсии температура растёт с
    высотой. Звуковые лучи изгибаются \textbf{вниз}, к земле. Это
    создаёт условия для \textbf{аномальной слышимости}, когда звук от
    мощного источника (например, взрыва) слышен на очень больших
    расстояниях (сотни километров). Звуковые волны могут отражаться от
    земли и снова от инверсионного слоя, создавая эффект ``звукового
    волновода''.
  \end{itemize}
\item
  \textbf{Ветровая рефракция:} Вертикальный сдвиг ветра также вызывает
  искривление звуковых лучей.

  \begin{itemize}
  \tightlist
  \item
    \textbf{По ветру:} Скорость ветра складывается со скоростью звука.
    Если скорость ветра растёт с высотой, лучи изгибаются \textbf{вниз},
    улучшая слышимость.
  \item
    \textbf{Против ветра:} Скорость ветра вычитается из скорости звука.
    Лучи изгибаются \textbf{вверх}, ухудшая слышимость.
  \end{itemize}
\item
  \textbf{Распространение в стратосфере и термосфере:} Звуковые волны от
  мощных приземных источников (извержения вулканов, взрывы) могут
  достигать стратосферы. Там, в тёплом слое (из-за озона), они
  претерпевают рефракцию и возвращаются к земле на больших расстояниях,
  создавая зоны аномальной слышимости. Аналогичный эффект наблюдается и
  в термосфере.
\end{itemize}

ewpage

\hypertarget{ux443ux440ux430ux432ux43dux435ux43dux438ux44f-ux434ux432ux438ux436ux435ux43dux438ux44f-ux430ux442ux43cux43eux441ux444ux435ux440ux44b}{%
\section{Уравнения Движения
Атмосферы}\label{ux443ux440ux430ux432ux43dux435ux43dux438ux44f-ux434ux432ux438ux436ux435ux43dux438ux44f-ux430ux442ux43cux43eux441ux444ux435ux440ux44b}}

\hypertarget{ux43eux431ux449ux438ux435-ux43fux440ux438ux43dux446ux438ux43fux44b}{%
\subsubsection{Общие
Принципы}\label{ux43eux431ux449ux438ux435-ux43fux440ux438ux43dux446ux438ux43fux44b}}

Динамическая метеорология, как важнейшая дисциплина в нашей области,
изучает атмосферные процессы, опираясь на фундаментальные законы физики,
а именно гидромеханики и термодинамики. Основной подход заключается в
преобразовании и решении общих уравнений гидротермодинамики,
адаптированных к специфическим физическим условиям атмосферы. Эти
исходные уравнения напрямую выражают ключевые законы сохранения: закон
сохранения импульса (второй закон Ньютона), закон сохранения энергии и
закон сохранения массы.

\hypertarget{ux441ux438ux43bux44b-ux434ux435ux439ux441ux442ux432ux443ux44eux449ux438ux435-ux432-ux430ux442ux43cux43eux441ux444ux435ux440ux435-1}{%
\subsubsection{Силы, Действующие в
Атмосфере}\label{ux441ux438ux43bux44b-ux434ux435ux439ux441ux442ux432ux443ux44eux449ux438ux435-ux432-ux430ux442ux43cux43eux441ux444ux435ux440ux435-1}}

Для адекватного описания атмосферных движений через уравнения необходимо
тщательно учитывать все действующие на частицы воздуха силы. Эти силы
классифицируются на массовые и поверхностные.

\hypertarget{ux43cux430ux441ux441ux43eux432ux44bux435-ux441ux438ux43bux44b-1}{%
\paragraph{Массовые
Силы}\label{ux43cux430ux441ux441ux43eux432ux44bux435-ux441ux438ux43bux44b-1}}

Массовые силы приложены ко всем точкам сплошной среды, вне зависимости
от их положения. К ним относятся:

\begin{itemize}
\tightlist
\item
  \textbf{Сила тяжести}: Ускорение свободного падения (\(g\)) в первом
  приближении принимается постоянным и направленным вертикально вниз.
\item
  \textbf{Отклоняющая сила вращения Земли (сила Кориолиса)}: Это
  инерционная сила, возникающая вследствие переносного вращательного
  движения Земли. Она действует на движущиеся относительно Земли частицы
  воздуха, вызывая кажущееся отклонение их траектории. В Северном
  полушарии это отклонение происходит вправо, а в Южном --- влево от
  первоначального направления движения. Величина этой силы прямо
  пропорциональна угловой скорости вращения Земли и относительной
  скорости движения частицы воздуха.
\item
  \textbf{Центробежная сила}: В рамках метеорологии учитывается
  центробежное ускорение вращения Земли, которое рассматривается как
  составляющая ускорения силы тяжести. Дополнительно, при движении тела
  по криволинейной траектории относительно вращающейся Земли, возникает
  центробежная сила, связанная с этим относительным движением.
\end{itemize}

\hypertarget{ux43fux43eux432ux435ux440ux445ux43dux43eux441ux442ux43dux44bux435-ux441ux438ux43bux44b-1}{%
\paragraph{Поверхностные
Силы}\label{ux43fux43eux432ux435ux440ux445ux43dux43eux441ux442ux43dux44bux435-ux441ux438ux43bux44b-1}}

Поверхностные силы являются результатом взаимодействия между соседними
слоями и частицами воздуха. К ним относятся:

\begin{itemize}
\tightlist
\item
  \textbf{Силы давления (барического градиента)}: В идеальной (невязкой)
  атмосфере эти силы действуют перпендикулярно к поверхностям,
  ограничивающим частицы воздуха. Именно горизонтальная неоднородность
  поля давления служит единственной причиной, порождающей движение
  ветра.
\item
  \textbf{Силы трения (вязкости)}: В реальной атмосфере возникают силы
  внутреннего трения (вязкости) воздуха, обусловливающие деформацию
  частиц. Сила трения постоянно снижает скорости воздушных течений и
  способствует диссипации кинетической энергии, превращая ее в тепловую.
  Для крупномасштабных процессов молекулярной вязкостью обычно
  пренебрегают ввиду ее малости. Однако турбулентное трение играет
  существенную роль, особенно в пограничном слое атмосферы, где оно
  значительно влияет на движения воздуха.
\end{itemize}

\hypertarget{ux444ux443ux43dux434ux430ux43cux435ux43dux442ux430ux43bux44cux43dux44bux435-ux443ux440ux430ux432ux43dux435ux43dux438ux44f}{%
\subsubsection{Фундаментальные
Уравнения}\label{ux444ux443ux43dux434ux430ux43cux435ux43dux442ux430ux43bux44cux43dux44bux435-ux443ux440ux430ux432ux43dux435ux43dux438ux44f}}

\hypertarget{ux432ux435ux43aux442ux43eux440ux43dux43eux435-ux443ux440ux430ux432ux43dux435ux43dux438ux435-ux434ux432ux438ux436ux435ux43dux438ux44f}{%
\paragraph{Векторное Уравнение
Движения}\label{ux432ux435ux43aux442ux43eux440ux43dux43eux435-ux443ux440ux430ux432ux43dux435ux43dux438ux435-ux434ux432ux438ux436ux435ux43dux438ux44f}}

Базисное векторное уравнение движения атмосферы, представляющее второй
закон Ньютона для сплошной среды, формулируется как:
\$\frac{d\vec{V}}{dt} = \vec{K} + \vec{g} +
\frac{1}{\rho}\left(\frac{\partial \vec{P}_x}{\partial x} +
\frac{\partial \vec{P}_y}{\partial y} +
\frac{\partial \vec{P}_z}{\partial z}\right) \$ где:

\begin{itemize}
\tightlist
\item
  \$\frac{d\vec{V}}{dt} \$ --- индивидуальная производная вектора
  скорости частицы воздуха по времени. Она включает как локальное
  изменение скорости в фиксированной точке, так и изменение,
  обусловленное перемещением частицы в пространстве.
\item
  \$\vec{K} \$ --- сила Кориолиса, отнесенная к единице массы.
\item
  \$\vec{g} \$ --- ускорение силы тяжести.
\item
  \$\frac{1}{\rho}\left(\frac{\partial \vec{P}_x}{\partial x} +
  \frac{\partial \vec{P}_y}{\partial y} +
  \frac{\partial \vec{P}_z}{\partial z}\right) \$ --- сумма
  поверхностных сил (включающих силу градиента давления и силы
  вязкости), отнесенных к единице массы.
\end{itemize}

\hypertarget{ux441ux43aux430ux43bux44fux440ux43dux44bux435-ux443ux440ux430ux432ux43dux435ux43dux438ux44f-ux443ux440ux430ux432ux43dux435ux43dux438ux44f-ux43dux430ux432ux44cux435-ux441ux442ux43eux43aux441ux430}{%
\paragraph{Скалярные Уравнения (Уравнения
Навье-Стокса)}\label{ux441ux43aux430ux43bux44fux440ux43dux44bux435-ux443ux440ux430ux432ux43dux435ux43dux438ux44f-ux443ux440ux430ux432ux43dux435ux43dux438ux44f-ux43dux430ux432ux44cux435-ux441ux442ux43eux43aux441ux430}}

Проецирование общего векторного уравнения на оси декартовой системы
координат (x, y, z) приводит к получению трех скалярных уравнений
движения, известных как уравнения Навье-Стокса. Эти уравнения в явном
виде учитывают вязкие напряжения, представляя собой систему
дифференциальных уравнений, связывающих проекции скорости (\(u, v, w\)),
давление (\(P\)), плотность (\(\rho\)) и коэффициенты вязкости.

\hypertarget{ux443ux440ux430ux432ux43dux435ux43dux438ux44f-ux432-ux441ux444ux435ux440ux438ux447ux435ux441ux43aux43eux439-ux441ux438ux441ux442ux435ux43cux435-ux43aux43eux43eux440ux434ux438ux43dux430ux442}{%
\paragraph{Уравнения в Сферической Системе
Координат}\label{ux443ux440ux430ux432ux43dux435ux43dux438ux44f-ux432-ux441ux444ux435ux440ux438ux447ux435ux441ux43aux43eux439-ux441ux438ux441ux442ux435ux43cux435-ux43aux43eux43eux440ux434ux438ux43dux430ux442}}

Для моделирования крупномасштабных атмосферных процессов, особенно тех,
что охватывают планетарный масштаб, целесообразно использовать уравнения
движения в сферической системе координат, где началом отсчета является
центр Земли. В этой системе положение точки определяется радиус-вектором
(\(\vec{r}\)), географической долготой (\(\lambda\)) и полярным углом
(\(\theta\)). Проекции сил Кориолиса и тяжести в сферических координатах
приобретают специфическую форму, что позволяет более точно учесть
влияние сферичности Земли и ее вращения.

\hypertarget{ux443ux440ux430ux432ux43dux435ux43dux438ux435-ux43dux435ux440ux430ux437ux440ux44bux432ux43dux43eux441ux442ux438-ux437ux430ux43aux43eux43d-ux441ux43eux445ux440ux430ux43dux435ux43dux438ux44f-ux43cux430ux441ux441ux44b}{%
\paragraph{Уравнение Неразрывности (Закон Сохранения
Массы)}\label{ux443ux440ux430ux432ux43dux435ux43dux438ux435-ux43dux435ux440ux430ux437ux440ux44bux432ux43dux43eux441ux442ux438-ux437ux430ux43aux43eux43d-ux441ux43eux445ux440ux430ux43dux435ux43dux438ux44f-ux43cux430ux441ux441ux44b}}

Уравнение неразрывности является неотъемлемой частью системы уравнений
динамики атмосферы, поскольку оно выражает закон сохранения массы. Это
уравнение устанавливает взаимосвязь между изменением плотности воздуха
во времени и распределением поля скорости в пространстве. В общем виде
для сжимаемой атмосферы оно может быть записано как:
\$\frac{\partial \rho}{\partial t} + \nabla \cdot (\rho \vec{V}) = 0 \$
Или в форме индивидуальной производной: \$\frac{d\rho}{dt} +
\rho (\nabla \cdot \vec{V}) = 0 \$ где \$\nabla \cdot \vec{V} \$
представляет собой дивергенцию вектора скорости. В случае несжимаемой
атмосферы, где плотность считается постоянной, дивергенция скорости
обращается в нуль (\$\nabla \cdot \vec{V} = 0 \$).

\hypertarget{ux443ux440ux430ux432ux43dux435ux43dux438ux435-ux43fux440ux438ux442ux43eux43aux430-ux442ux435ux43fux43bux430-ux43fux435ux440ux432ux43eux435-ux43dux430ux447ux430ux43bux43e-ux442ux435ux440ux43cux43eux434ux438ux43dux430ux43cux438ux43aux438}{%
\paragraph{Уравнение Притока Тепла (Первое Начало
Термодинамики)}\label{ux443ux440ux430ux432ux43dux435ux43dux438ux435-ux43fux440ux438ux442ux43eux43aux430-ux442ux435ux43fux43bux430-ux43fux435ux440ux432ux43eux435-ux43dux430ux447ux430ux43bux43e-ux442ux435ux440ux43cux43eux434ux438ux43dux430ux43cux438ux43aux438}}

Уравнение притока тепла является проявлением закона сохранения энергии,
примененного к элементарному объему воздуха. Оно связывает изменение
внутренней энергии частицы (\(dE\)) с притоком тепла (\(dQ\)) к ней и
работой (\(PdV\)), совершаемой ею или над ней: \$ C\_v dT + PdV = dQ \$
где \(C_v\) --- удельная теплоемкость воздуха при постоянном объеме. Это
уравнение часто преобразуется, используя уравнение состояния, для
исключения непрямо измеряемых величин, таких как объем: \$ C\_p dT -
\frac{1}{\rho} dP = dQ \$ (Следует отметить, что в источнике
представлена несколько иная форма, однако общая идея сохранения энергии
сохраняется). Уравнение притока тепла учитывает различные механизмы
теплообмена в атмосфере, включая радиационный приток/отток, скрытую
теплоту фазовых переходов воды (конденсация, испарение) и турбулентный
теплообмен.

\hypertarget{ux443ux440ux430ux432ux43dux435ux43dux438ux44f-ux443ux441ux440ux435ux434ux43dux435ux43dux43dux43eux433ux43e-ux434ux432ux438ux436ux435ux43dux438ux44f-ux443ux440ux430ux432ux43dux435ux43dux438ux44f-ux440ux435ux439ux43dux43eux43bux44cux434ux441ux430}{%
\subsubsection{Уравнения Усредненного Движения (Уравнения
Рейнольдса)}\label{ux443ux440ux430ux432ux43dux435ux43dux438ux44f-ux443ux441ux440ux435ux434ux43dux435ux43dux43dux43eux433ux43e-ux434ux432ux438ux436ux435ux43dux438ux44f-ux443ux440ux430ux432ux43dux435ux43dux438ux44f-ux440ux435ux439ux43dux43eux43bux44cux434ux441ux430}}

Поскольку атмосферные движения преимущественно турбулентны, мгновенные
значения метеорологических величин демонстрируют случайный характер. Для
их анализа применяется метод усреднения уравнений движения. Этот процесс
вводит новые неизвестные термины --- добавочные турбулентные напряжения
(компоненты тензора Рейнольдса), которые эффективно параметризуют
суммарное воздействие хаотического движения вихрей на основное
усредненное движение. В результате, система уравнений гидротермодинамики
с учетом турбулентности становится значительно более сложной и требует
дополнительных гипотез для замыкания.

\hypertarget{ux443ux43fux440ux43eux449ux435ux43dux438ux44f-ux438-ux434ux43eux43fux443ux449ux435ux43dux438ux44f}{%
\subsubsection{Упрощения и
Допущения}\label{ux443ux43fux440ux43eux449ux435ux43dux438ux44f-ux438-ux434ux43eux43fux443ux449ux435ux43dux438ux44f}}

\hypertarget{ux43aux432ux430ux437ux438ux441ux442ux430ux442ux438ux447ux435ux441ux43aux43eux435-ux43fux440ux438ux431ux43bux438ux436ux435ux43dux438ux435-1}{%
\paragraph{Квазистатическое
Приближение}\label{ux43aux432ux430ux437ux438ux441ux442ux430ux442ux438ux447ux435ux441ux43aux43eux435-ux43fux440ux438ux431ux43bux438ux436ux435ux43dux438ux435-1}}

Для анализа крупномасштабных атмосферных движений вертикальная
компонента уравнения движения часто упрощается до так называемого
уравнения квазистатики: \$\rho g + \frac{\partial P}{\partial z} = 0 \$.
Это приближение подразумевает, что сила тяжести практически полностью
уравновешивается вертикальной составляющей силы барического градиента.
Данное допущение является вполне обоснованным для процессов,
охватывающих обширные территории, порядка нескольких тысяч километров.

\hypertarget{ux430ux434ux438ux430ux431ux430ux442ux438ux447ux435ux441ux43aux430ux44f-ux43cux43eux434ux435ux43bux44c}{%
\paragraph{Адиабатическая
Модель}\label{ux430ux434ux438ux430ux431ux430ux442ux438ux447ux435ux441ux43aux430ux44f-ux43cux43eux434ux435ux43bux44c}}

При решении задач краткосрочного прогноза, особенно для процессов в
свободной атмосфере, часто допускается пренебрежение внешним притоком
тепла. В этом случае процесс считается адиабатическим (\(dQ=0\)), что
значительно упрощает уравнения термодинамики, поскольку отсутствует
теплообмен с окружающей средой.

\hypertarget{ux431ux430ux440ux43eux442ux440ux43eux43fux43dux430ux44f-ux438-ux431ux430ux440ux43eux43aux43bux438ux43dux43dux430ux44f-ux43cux43eux434ux435ux43bux438}{%
\paragraph{Баротропная и Бароклинная
Модели}\label{ux431ux430ux440ux43eux442ux440ux43eux43fux43dux430ux44f-ux438-ux431ux430ux440ux43eux43aux43bux438ux43dux43dux430ux44f-ux43cux43eux434ux435ux43bux438}}

\begin{itemize}
\tightlist
\item
  \textbf{Баротропная атмосфера}: Это гипотетическая среда, в которой
  плотность воздуха является функцией исключительно давления (\$\rho =
  \rho(P) \$). В такой модели изопикнические, изобарические и
  изотермические поверхности параллельны друг другу. Принятие
  баротропной модели значительно упрощает вычислительные аспекты
  прогностических расчетов.
\item
  \textbf{Бароклинная атмосфера}: Реальная атмосфера является
  бароклинной, что означает, что плотность зависит как от давления, так
  и от температуры (\$\rho = \rho(P, T) \$). В бароклинной атмосфере
  изопикнические, изобарические и изотермические поверхности
  пересекаются, образуя так называемые термодинамические соленоиды.
  Бароклинность играет критически важную роль в зарождении и развитии
  синоптических вихрей и атмосферных фронтов.
\end{itemize}

\hypertarget{ux432ux430ux436ux43dux43eux441ux442ux44c-ux434ux43bux44f-ux43cux435ux442ux435ux43eux440ux43eux43bux43eux433ux438ux438}{%
\subsubsection{Важность для
Метеорологии}\label{ux432ux430ux436ux43dux43eux441ux442ux44c-ux434ux43bux44f-ux43cux435ux442ux435ux43eux440ux43eux43bux43eux433ux438ux438}}

Представленная система уравнений формирует замкнутую математическую
модель, позволяющую описывать и прогнозировать атмосферные процессы.
Динамическая метеорология, исследуя атмосферные движения во взаимосвязи
с термодинамическими процессами, выявляет основные закономерности погоды
и климата. Эти закономерности затем применяются для разработки
объективных методов прогноза погоды и развития теорий воздействия на
погодные и климатические явления. Упрощенные модели, такие как
квазистатическая, адиабатическая и квазигеострофическая, позволяют
эффективно фильтровать ``метеорологические шумы'' и концентрироваться на
динамике крупномасштабных процессов.

Несмотря на исключительную сложность атмосферных процессов и
многочисленные факторы, влияющие на их развитие, применяемые
схематизации и допущения позволяют успешно решать широкий круг
прогностических задач, обеспечивая ценные инструментарий для
метеорологической практики.

ewpage

\hypertarget{ux443ux441ux442ux430ux43dux43eux432ux438ux432ux448ux435ux435ux441ux44f-ux434ux432ux438ux436ux435ux43dux438ux435-ux432ux43eux437ux434ux443ux445ux430-ux431ux435ux437-ux443ux447ux435ux442ux430-ux441ux438ux43b-ux442ux440ux435ux43dux438ux44f}{%
\section{Установившееся движение воздуха без учета сил
трения}\label{ux443ux441ux442ux430ux43dux43eux432ux438ux432ux448ux435ux435ux441ux44f-ux434ux432ux438ux436ux435ux43dux438ux435-ux432ux43eux437ux434ux443ux445ux430-ux431ux435ux437-ux443ux447ux435ux442ux430-ux441ux438ux43b-ux442ux440ux435ux43dux438ux44f}}

Установившееся (или стационарное) движение воздуха без учета сил трения
представляет собой идеализированную модель атмосферных процессов, при
которой скорость движения в данной точке пространства остается
постоянной во времени, то есть отсутствуют ускорения
(\(\frac{dV}{dt} = 0\)). В этом случае линии тока совпадают с
траекториями движения частиц воздуха.

\hypertarget{ux434ux435ux439ux441ux442ux432ux443ux44eux449ux438ux435-ux441ux438ux43bux44b}{%
\subsection{Действующие
силы}\label{ux434ux435ux439ux441ux442ux432ux443ux44eux449ux438ux435-ux441ux438ux43bux44b}}

В данной модели, когда силы трения (молекулярного и турбулентного)
пренебрегаются, на воздушную частицу действуют следующие основные силы:

\begin{itemize}
\tightlist
\item
  \textbf{Массовые силы}:

  \begin{itemize}
  \tightlist
  \item
    \textbf{Сила тяжести (гравитация)}.
  \item
    \textbf{Отклоняющая сила вращения Земли (сила Кориолиса)}. Эта сила
    возникает за счет суточного вращения Земли и относительного движения
    частиц воздуха. В Северном полушарии она отклоняет движущиеся
    частицы вправо, а в Южном --- влево от направления движения. При
    этом сила Кориолиса изменяет лишь направление воздушных течений, не
    меняя абсолютной величины скорости движения и не совершая никакой
    работы.
  \end{itemize}
\item
  \textbf{Поверхностные силы}:

  \begin{itemize}
  \tightlist
  \item
    \textbf{Сила градиента давления}. Она является единственной
    причиной, порождающей ветер, тогда как силы трения и инерции лишь
    изменяют движение воздуха.
  \end{itemize}
\end{itemize}

\hypertarget{ux431ux430ux43bux430ux43dux441-ux441ux438ux43b-ux432-ux443ux441ux442ux430ux43dux43eux432ux438ux432ux448ux435ux43cux441ux44f-ux434ux432ux438ux436ux435ux43dux438ux438}{%
\subsection{Баланс сил в установившемся
движении}\label{ux431ux430ux43bux430ux43dux441-ux441ux438ux43b-ux432-ux443ux441ux442ux430ux43dux43eux432ux438ux432ux448ux435ux43cux441ux44f-ux434ux432ux438ux436ux435ux43dux438ux438}}

В установившемся движении силы, действующие на частицу, уравновешивают
друг друга, что приводит к нулевому ускорению. Распределение этих сил
следующее:

\begin{itemize}
\tightlist
\item
  \textbf{В вертикальном направлении}: Силы тяжести, плавучести и
  центробежная сила вращения Земли действуют в вертикальном направлении
  и уравновешиваются вертикальной составляющей силы барического
  градиента. Это описывается уравнением статики.
\item
  \textbf{В горизонтальном направлении}: При отсутствии сил трения и в
  случае установившегося движения, горизонтальная сила градиента
  давления уравновешивается силой Кориолиса. Такое движение называется
  \textbf{геострофическим движением}. Скорость такого ветра, называемого
  \textbf{геострофическим ветром}, определяется условием равновесия этих
  двух сил. Геострофический ветер направлен параллельно изобарам (или
  изогипсам на картах барической топографии), при этом низкое давление
  остается слева (в Северном полушарии). Геострофический ветер возможен
  только при прямолинейных изобарах или изогипсах.
\end{itemize}

\hypertarget{ux43eux431ux43bux430ux441ux442ux438-ux43fux440ux43eux44fux432ux43bux435ux43dux438ux44f-ux432-ux430ux442ux43cux43eux441ux444ux435ux440ux435}{%
\subsection{Области проявления в
атмосфере}\label{ux43eux431ux43bux430ux441ux442ux438-ux43fux440ux43eux44fux432ux43bux435ux43dux438ux44f-ux432-ux430ux442ux43cux43eux441ux444ux435ux440ux435}}

Модель установившегося движения без учета сил трения наиболее применима
в \textbf{свободной атмосфере}. Это слой атмосферы, расположенный выше
планетарного пограничного слоя (обычно 1000-1500 м или 1-1.5 км над
земной поверхностью).

\begin{itemize}
\tightlist
\item
  \textbf{Молекулярная вязкость}: Влиянием сил молекулярной вязкости
  воздуха на атмосферные движения можно пренебречь, поскольку числа
  Рейнольдса для атмосферных движений очень велики.
\item
  \textbf{Турбулентное трение}: Выше пограничного слоя сила
  турбулентного трения становится на порядок меньше силы Кориолиса и,
  как правило, ею можно пренебречь.
\end{itemize}

\hypertarget{ux437ux43dux430ux447ux435ux43dux438ux435-ux438-ux43eux433ux440ux430ux43dux438ux447ux435ux43dux438ux44f-ux43cux43eux434ux435ux43bux438}{%
\subsection{Значение и ограничения
модели}\label{ux437ux43dux430ux447ux435ux43dux438ux435-ux438-ux43eux433ux440ux430ux43dux438ux447ux435ux43dux438ux44f-ux43cux43eux434ux435ux43bux438}}

Геострофический ветер, являясь математической абстракцией, очень близок
к реальному ветру в свободной атмосфере по скорости и направлению, что
позволяет использовать поле изобар для оценки реального ветра в районах,
где отсутствуют прямые наблюдения.

Однако, важно отметить, что в действительности атмосферные движения
преимущественно нестационарны, а равновесие действующих сил неустойчиво.
Отклонения от идеализированной геострофической модели наблюдаются:

\begin{itemize}
\tightlist
\item
  \textbf{В пограничном слое}: Здесь сила трения о подстилающую
  поверхность существенно влияет на ветер, уменьшая его скорость
  (например, до 0.7 от геострофической над морем) и отклоняя его влево
  от изобар (в Северном полушарии), что приводит к конвергенции
  (сходимости) воздушных потоков в циклонах и дивергенции (расходимости)
  в антициклонах, вызывая вертикальные движения.
\item
  \textbf{В зонах фронтов и струйных течений}: В этих областях могут
  наблюдаться значительные вертикальные и горизонтальные сдвиги ветра,
  что делает модель идеальной жидкости менее точной.
\item
  \textbf{При криволинейных изобарах}: В этом случае вместо
  геострофического ветра более точно описывает движение градиентный
  ветер, хотя на практике часто продолжают использовать геострофический
  из-за сложности расчетов.
\end{itemize}

Таким образом, установившееся движение воздуха без учета сил трения --
это упрощенная, но фундаментальная концепция в динамической
метеорологии, позволяющая понять базовые закономерности крупномасштабных
атмосферных двичений в свободной атмосфере, где влияние трения
минимально.

ewpage

\hypertarget{ux433ux440ux430ux434ux438ux435ux43dux442ux43dux44bux439-ux432ux435ux442ux435ux440}{%
\section{Градиентный
ветер}\label{ux433ux440ux430ux434ux438ux435ux43dux442ux43dux44bux439-ux432ux435ux442ux435ux440}}

Градиентный ветер -- это теоретическое движение воздуха вдоль
криволинейных изобар, при котором достигается равновесие горизонтальных
составляющих силы барического градиента, силы Кориолиса и центробежной
силы. Важно отметить, что в этой модели не учитывается сила трения.

\hypertarget{ux434ux438ux43dux430ux43cux438ux447ux435ux441ux43aux438ux439-ux431ux430ux43bux430ux43dux441}{%
\subsection{Динамический
Баланс}\label{ux434ux438ux43dux430ux43cux438ux447ux435ux441ux43aux438ux439-ux431ux430ux43bux430ux43dux441}}

В концепции градиентного ветра, три силы находятся в равновесии:

\begin{itemize}
\tightlist
\item
  \textbf{Сила барического градиента (G)}: Направлена от высокого
  давления к низкому, перпендикулярно изобарам.
\item
  \textbf{Сила Кориолиса (A)}: В Северном полушарии направлена вправо от
  вектора скорости движения воздуха, в Южном -- влево.
\item
  \textbf{Центробежная сила (C)}: Возникает при криволинейном движении и
  всегда направлена от центра кривизны траектории.
\end{itemize}

\hypertarget{ux441ux43eux43eux442ux43dux43eux448ux435ux43dux438ux435-ux441-ux433ux435ux43eux441ux442ux440ux43eux444ux438ux447ux435ux441ux43aux438ux43c-ux432ux435ux442ux440ux43eux43c}{%
\subsection{Соотношение с Геострофическим
Ветром}\label{ux441ux43eux43eux442ux43dux43eux448ux435ux43dux438ux435-ux441-ux433ux435ux43eux441ux442ux440ux43eux444ux438ux447ux435ux441ux43aux438ux43c-ux432ux435ux442ux440ux43eux43c}}

Геострофический ветер является частным случаем градиентного ветра. Это
происходит тогда, когда центробежная сила равна нулю, что соответствует
движению вдоль прямолинейных изобар. В свободной атмосфере, на высотах
более 1 км, действительный ветер часто близок к геострофическому по
скорости и направлению.

\hypertarget{ux43eux441ux43eux431ux435ux43dux43dux43eux441ux442ux438-ux432-ux431ux430ux440ux438ux447ux435ux441ux43aux438ux445-ux441ux438ux441ux442ux435ux43cux430ux445}{%
\subsection{Особенности в Барических
Системах}\label{ux43eux441ux43eux431ux435ux43dux43dux43eux441ux442ux438-ux432-ux431ux430ux440ux438ux447ux435ux441ux43aux438ux445-ux441ux438ux441ux442ux435ux43cux430ux445}}

Скорость градиентного ветра в циклонах и антициклонах отличается от
геострофического:

\begin{itemize}
\tightlist
\item
  \textbf{В циклоне}: Сила барического градиента (G) направлена к
  центру, а центробежная сила (C) -- от центра. В этом случае
  градиентный ветер несколько \textbf{слабее} геострофического ветра.
\item
  \textbf{В антициклоне}: Обе силы, G и C, направлены от центра. Это
  приводит к тому, что градиентный ветер \textbf{сильнее}
  геострофического ветра.
\item
  В самом центре циклона или антициклона, где сила барического градиента
  как источник движения равна нулю, скорость градиентного ветра также
  равна нулю.
\end{itemize}

\hypertarget{ux43fux440ux438ux43cux435ux43dux438ux43cux43eux441ux442ux44c-ux438-ux43eux433ux440ux430ux43dux438ux447ux435ux43dux438ux44f}{%
\subsection{Применимость и
Ограничения}\label{ux43fux440ux438ux43cux435ux43dux438ux43cux43eux441ux442ux44c-ux438-ux43eux433ux440ux430ux43dux438ux447ux435ux43dux438ux44f}}

Формулы градиентного ветра строго справедливы для круговых изобар, хотя
иногда их применяют и для любых криволинейных изобар. Однако, при малых
радиусах кривизны изобар (менее 300 км) пренебрегать влиянием кривизны
на ветер нельзя.

В случаях нестационарного движения или при значительном изменении
кривизны изобар, действительный ветер может существенно отклоняться от
градиентного. Более того, в некоторых ситуациях учет кривизны может даже
ухудшать результаты расчетов. Поэтому на практике при криволинейных
изобарах часто вместо вычисления градиентного ветра предпочтение
отдается расчету геострофического ветра. Для учета кривизны в таких
случаях могут применяться специальные градиентные линейки.

\hypertarget{ux43fux440ux43eux433ux43dux43eux441ux442ux438ux447ux435ux441ux43aux43eux435-ux43fux440ux438ux43cux435ux43dux435ux43dux438ux435}{%
\subsection{Прогностическое
Применение}\label{ux43fux440ux43eux433ux43dux43eux441ux442ux438ux447ux435ux441ux43aux43eux435-ux43fux440ux438ux43cux435ux43dux435ux43dux438ux435}}

На высотах, ветер обычно предсказывается как геострофический или
градиентный. При этом на прогностических картах изобарических
поверхностей, рассчитанных с помощью ЭВМ, определяется ожидаемая
скорость геострофического ветра в интересующем районе.

ewpage

\hypertarget{ux433ux435ux43eux441ux442ux440ux43eux444ux438ux447ux435ux441ux43aux438ux439-ux432ux435ux442ux435ux440}{%
\section{Геострофический
Ветер}\label{ux433ux435ux43eux441ux442ux440ux43eux444ux438ux447ux435ux441ux43aux438ux439-ux432ux435ux442ux435ux440}}

Геострофический ветер является фундаментальным понятием в динамической
метеорологии, представляя собой идеализированное горизонтальное движение
воздуха. Это не реальное природное явление, а математическая абстракция,
которая, однако, весьма ценна для метеорологов, поскольку фактический
ветер в свободной атмосфере, как показывают наблюдения, по скорости и
направлению очень близок к геострофическому.

\hypertarget{ux43eux43fux440ux435ux434ux435ux43bux435ux43dux438ux435-ux438-ux443ux441ux43bux43eux432ux438ux44f-ux440ux430ux432ux43dux43eux432ux435ux441ux438ux44f}{%
\subsection{Определение и Условия
Равновесия}\label{ux43eux43fux440ux435ux434ux435ux43bux435ux43dux438ux435-ux438-ux443ux441ux43bux43eux432ux438ux44f-ux440ux430ux432ux43dux43eux432ux435ux441ux438ux44f}}

Геострофический ветер определяется как движение воздуха вдоль
прямолинейных изобар, возникающее в результате равновесия между
горизонтальной составляющей силы барического градиента и силой
Кориолиса. В простейшем случае, когда ускорение равно нулю, а также
отсутствуют силы трения и центробежная сила (что подразумевает
прямолинейные изобары или изогипсы карт абсолютной топографии), скорость
геострофического ветра \textbf{Vg} может быть определена по формуле:
\texttt{Vg\ =\ (1\ /\ (ρ\ *\ I))\ *\ (∂p\ /\ ∂n)} или в терминах
градиента геопотенциальной высоты (Н) изобарической поверхности:
\texttt{Vg\ =\ (g\ /\ I)\ *\ (∂H\ /\ ∂n)} где:

\begin{itemize}
\tightlist
\item
  \texttt{p} --- давление
\item
  \texttt{n} --- нормаль к изобаре (направленная в сторону понижения
  давления)
\item
  \texttt{ρ} --- плотность воздуха
\item
  \texttt{g} --- ускорение свободного падения
\item
  \texttt{I\ =\ 2ω\ sin\ φ} --- параметр Кориолиса, где \texttt{ω} ---
  угловая скорость вращения Земли (7,29∙10⁻⁵ с⁻¹), а \texttt{φ} ---
  географическая широта. Важно отметить, что вблизи экватора, где
  \texttt{I} стремится к нулю, вычисление геострофического ветра теряет
  смысл.
\end{itemize}

\hypertarget{ux43dux430ux43fux440ux430ux432ux43bux435ux43dux438ux435-ux438-ux441ux43aux43eux440ux43eux441ux442ux44c}{%
\subsection{Направление и
Скорость}\label{ux43dux430ux43fux440ux430ux432ux43bux435ux43dux438ux435-ux438-ux441ux43aux43eux440ux43eux441ux442ux44c}}

Направление геострофического ветра всегда параллельно изобарам
(изогипсам). В Северном полушарии он направлен таким образом, что
область низкого давления (пониженного геопотенциала) остается слева, а
область высокого давления (повышенного геопотенциала) --- справа. В
Южном полушарии направление обратное.

Скорость геострофического ветра прямо пропорциональна градиенту давления
(барическому градиенту) или, что эквивалентно, густоте изобар (изогипс).
Чем круче наклон изобарических поверхностей к горизонту, тем больше
скорость геострофического ветра.

\hypertarget{ux43fux440ux438ux43cux435ux43dux438ux43cux43eux441ux442ux44c-ux438-ux43eux433ux440ux430ux43dux438ux447ux435ux43dux438ux44f-1}{%
\subsection{Применимость и
Ограничения}\label{ux43fux440ux438ux43cux435ux43dux438ux43cux43eux441ux442ux44c-ux438-ux43eux433ux440ux430ux43dux438ux447ux435ux43dux438ux44f-1}}

Геострофическое приближение применимо в свободной атмосфере, то есть
выше пограничного слоя трения, который обычно простирается до высоты
около 1 км. В этом слое фактический ветер приближается к
геострофическому. Однако, при ослабленном турбулентном обмене влияние
приземного трения распространяется до меньших высот (0,3-0,4 км), а при
сильном турбулентном обмене --- до больших (1,5-2,0 км).

Даже в свободной атмосфере могут наблюдаться существенные отклонения
действительного ветра от геострофического, обусловленные
нестационарностью барического поля и поля ветра. Геострофический ветер
является частным случаем градиентного ветра (при отсутствии центробежной
силы), и в большинстве случаев, когда кривизна изобар мала, на практике
вместо градиентного ветра вычисляют геострофический.

\hypertarget{ux441ux432ux44fux437ux44c-ux441-ux440ux435ux430ux43bux44cux43dux44bux43c-ux432ux435ux442ux440ux43eux43c-ux438-ux43fux440ux43eux433ux43dux43eux437ux438ux440ux43eux432ux430ux43dux438ux435ux43c}{%
\subsection{Связь с Реальным Ветром и
Прогнозированием}\label{ux441ux432ux44fux437ux44c-ux441-ux440ux435ux430ux43bux44cux43dux44bux43c-ux432ux435ux442ux440ux43eux43c-ux438-ux43fux440ux43eux433ux43dux43eux437ux438ux440ux43eux432ux430ux43dux438ux435ux43c}}

Несмотря на то, что геострофический ветер является математической
абстракцией, его практическая ценность очень высока. Поле изобар дает
представление о направлении и скорости реального ветра даже в тех
районах, где наблюдения за ветром отсутствуют. В гидродинамических
моделях прогноза атмосферный воздух рассматривается как бароклинная
жидкость, в которой плотность зависит от давления и температуры. В таких
моделях часто используется квазигеострофическое приближение, которое
позволяет отфильтровывать ``метеорологические шумы'' (явления меньших
масштабов), предполагая, что реальный ветер близок к геострофическому.

\hypertarget{ux43fux440ux43eux433ux43dux43eux437-ux432ux435ux442ux440ux430}{%
\subsubsection{Прогноз
Ветра}\label{ux43fux440ux43eux433ux43dux43eux437-ux432ux435ux442ux440ux430}}

\begin{itemize}
\tightlist
\item
  На высотах направление ветра предсказывается как направление
  геострофического (градиентного) ветра.
\item
  В приземном слое, где существенна сила трения, ветер отклоняется от
  изобары влево (в Северном полушарии) на некоторый угол: в среднем
  около 30° над сушей и около 15° над морем. Скорость ветра в приземном
  слое также меньше скорости геострофического ветра (например,
  приблизительно 0.7 Vg над морем и 0.4 Vg над сушей в умеренных
  широтах).
\item
  При сильном турбулентном обмене в пограничном слое угол пересечения
  вектором ветра изобары может стать менее 15° даже над сушей, а
  скорость может приблизиться к геострофической и даже превзойти ее при
  малых барических градиентах.
\item
  Ночью, при образовании приземной инверсии, ветер в приземном слое
  может быть в 2-3 раза слабее геострофического или даже наблюдаться
  полный штиль. Днем, при сильном прогреве подстилающей поверхности и
  сверхадиабатических градиентах температуры, скорость ветра может в 2-3
  раза превосходить скорость геострофического ветра.
\end{itemize}

\hypertarget{ux442ux435ux440ux43cux438ux447ux435ux441ux43aux438ux439-ux432ux435ux442ux435ux440}{%
\subsection{Термический
Ветер}\label{ux442ux435ux440ux43cux438ux447ux435ux441ux43aux438ux439-ux432ux435ux442ux435ux440}}

Геострофический ветер изменяется с высотой в термически неоднородной
атмосфере. Это изменение описывается понятием \textbf{термического
ветра}. Термический ветер представляет собой векторное приращение
геострофического ветра при переходе с одного уровня на другой.

\begin{itemize}
\tightlist
\item
  Его направление параллельно изотермам слоя, при этом в Северном
  полушарии область более низких температур располагается слева, а более
  высоких --- справа, если смотреть по направлению термического ветра.
\item
  Модуль термического ветра пропорционален горизонтальному градиенту
  средней температуры слоя.
\end{itemize}

Изменение геострофического ветра с высотой связано с адвекцией
температуры: в Северном полушарии правый поворот ветра с высотой
соответствует адвекции тепла, а левый --- адвекции холода. Уровень
максимального ветра в струйных течениях совпадает с высотой, на которой
угол между барическим и термическим градиентами равен ±90°.

ewpage

\hypertarget{ux438ux437ux43cux435ux43dux435ux43dux438ux435-ux433ux435ux43eux441ux442ux440ux43eux444ux438ux447ux435ux441ux43aux43eux433ux43e-ux432ux435ux442ux440ux430-ux441-ux432ux44bux441ux43eux442ux43eux439}{%
\section{Изменение геострофического ветра с
высотой}\label{ux438ux437ux43cux435ux43dux435ux43dux438ux435-ux433ux435ux43eux441ux442ux440ux43eux444ux438ux447ux435ux441ux43aux43eux433ux43e-ux432ux435ux442ux440ux430-ux441-ux432ux44bux441ux43eux442ux43eux439}}

Изменение геострофического ветра с высотой -- это фундаментальное
явление в динамической метеорологии, напрямую связанное с термической
структурой атмосферы и известное как \textbf{термический ветер}.

\hypertarget{ux433ux435ux43eux441ux442ux440ux43eux444ux438ux447ux435ux441ux43aux438ux439-ux432ux435ux442ux435ux440-1}{%
\subsection{Геострофический
ветер}\label{ux433ux435ux43eux441ux442ux440ux43eux444ux438ux447ux435ux441ux43aux438ux439-ux432ux435ux442ux435ux440-1}}

Геострофический ветер (\(V_g\)) представляет собой идеализированное
горизонтальное воздушное течение, возникающее в результате баланса силы
барического градиента (направленной от высокого давления к низкому) и
отклоняющей силы Кориолиса (направленной перпендикулярно движению). В
Северном полушарии геострофический ветер дует параллельно изобарам (или
изогипсам на картах абсолютной топографии), оставляя область низкого
давления слева от себя. Скорость геострофического ветра прямо
пропорциональна густоте изобар (величине горизонтального градиента
давления или геопотенциала), а его компоненты \(u_g\) и \(v_g\) могут
быть выражены через эти градиенты. Важно отметить, что при расчете
геострофического ветра вертикальные движения воздуха (\(w\)) принимаются
равными нулю. В свободной атмосфере, выше пограничного слоя трения
(около 1-1.5 км), реальный ветер очень близок к геострофическому.

\hypertarget{ux43aux43eux43dux446ux435ux43fux446ux438ux44f-ux442ux435ux440ux43cux438ux447ux435ux441ux43aux43eux433ux43e-ux432ux435ux442ux440ux430}{%
\subsection{Концепция термического
ветра}\label{ux43aux43eux43dux446ux435ux43fux446ux438ux44f-ux442ux435ux440ux43cux438ux447ux435ux441ux43aux43eux433ux43e-ux432ux435ux442ux440ux430}}

Поскольку поле изобар в термически неоднородной атмосфере изменяется с
высотой, то и геострофический ветер также претерпевает изменения с
ростом высоты. Это изменение с высотой описывается понятием термического
ветра (\(V_T\)), который является векторным приращением геострофического
ветра при переходе с одного изобарического уровня на другой:

\(V_{g2} = V_{g1} + V_T\)

где \(V_{g1}\) --- геострофический ветер на нижнем уровне \(z_1\) (или
\(H_1\)), а \(V_{g2}\) --- на верхнем уровне \(z_2\) (или \(H_2\)).

\hypertarget{ux43dux430ux43fux440ux430ux432ux43bux435ux43dux438ux435-ux438-ux432ux435ux43bux438ux447ux438ux43dux430-ux442ux435ux440ux43cux438ux447ux435ux441ux43aux43eux433ux43e-ux432ux435ux442ux440ux430}{%
\subsubsection{Направление и величина термического
ветра}\label{ux43dux430ux43fux440ux430ux432ux43bux435ux43dux438ux435-ux438-ux432ux435ux43bux438ux447ux438ux43dux430-ux442ux435ux440ux43cux438ux447ux435ux441ux43aux43eux433ux43e-ux432ux435ux442ux440ux430}}

Вектор термического ветра \(V_T\) непосредственно связан с
горизонтальным градиентом средней температуры слоя (\(Γ_m\)) между двумя
рассматриваемыми уровнями. В Северном полушарии термический ветер
направлен вдоль изотермы таким образом, что область более низких
температур находится слева, а область более высоких температур ---
справа, если смотреть по направлению термического ветра.

\hypertarget{ux441ux432ux44fux437ux44c-ux441-ux430ux434ux432ux435ux43aux446ux438ux435ux439-ux442ux435ux43cux43fux435ux440ux430ux442ux443ux440ux44b}{%
\subsubsection{Связь с адвекцией
температуры}\label{ux441ux432ux44fux437ux44c-ux441-ux430ux434ux432ux435ux43aux446ux438ux435ux439-ux442ux435ux43cux43fux435ux440ux430ux442ux443ux440ux44b}}

Изменение геострофического ветра с высотой тесно связано с адвекцией
температуры в атмосфере:

\begin{itemize}
\tightlist
\item
  \textbf{Правый поворот} (веерность) вектора геострофического ветра с
  высотой в Северном полушарии указывает на \textbf{адвекцию тепла}.
  Адвекция тепла (перенос воздуха из более теплой области в более
  холодную) или влажного воздуха приводит к положительному адвективному
  изменению виртуальной температуры.
\item
  \textbf{Левый поворот} (закручивание против часовой стрелки, или
  бэкинг) вектора геострофического ветра с высотой в Северном полушарии
  свидетельствует об \textbf{адвекции холода}. Адвекция холода (перенос
  воздуха из более холодной области в более теплую) или сухого воздуха
  вызывает отрицательное адвективное изменение виртуальной температуры.
\end{itemize}

Важно подчеркнуть, что сам по себе термический ветер не переносит тепло
или холод; его вектор лишь указывает на расположение областей тепла и
холода в слое. Однако именно этот векторный сдвиг геострофического ветра
позволяет оценить адвекцию температуры.

\hypertarget{ux43cux435ux442ux435ux43eux440ux43eux43bux43eux433ux438ux447ux435ux441ux43aux43eux435-ux437ux43dux430ux447ux435ux43dux438ux435}{%
\subsection{Метеорологическое
значение}\label{ux43cux435ux442ux435ux43eux440ux43eux43bux43eux433ux438ux447ux435ux441ux43aux43eux435-ux437ux43dux430ux447ux435ux43dux438ux435}}

Изменение геострофического ветра с высотой, или термический ветер,
играет ключевую роль в различных атмосферных процессах:

\begin{itemize}
\tightlist
\item
  Оно является важным показателем для понимания вертикальной структуры
  барических систем и их эволюции, поскольку влияет на изменения
  завихренности и градиентов геопотенциала с высотой.
\item
  Вертикальный сдвиг ветра (веерность/бэкинг) и связанная с ним адвекция
  температуры являются мощными динамическими факторами, способствующими
  образованию и развитию синоптических вихрей, таких как циклоны и
  антициклоны, особенно в бароклинной атмосфере.
\item
  Прогноз ветра на различных высотах, особенно в свободной атмосфере,
  базируется на предвычисленных картах абсолютной топографии и понимании
  изменения геострофического ветра с высотой.
\end{itemize}

Таким образом, анализ изменения геострофического ветра с высотой
необходим для глубокого понимания атмосферных процессов и повышения
точности метеорологических прогнозов.

ewpage

Как коллеге, знакомство со спиралью Экмана, безусловно, является одним
из фундаментальных элементов понимания динамики атмосферного
пограничного слоя.

\hypertarget{ux441ux43fux438ux440ux430ux43bux44c-ux44dux43aux43cux430ux43dux430}{%
\subsubsection{Спираль
Экмана}\label{ux441ux43fux438ux440ux430ux43bux44c-ux44dux43aux43cux430ux43dux430}}

Спираль Экмана описывает изменение вектора скорости ветра с высотой в
пограничном слое атмосферы. Этот феномен обусловлен сложным
взаимодействием трех основных сил: силы барического градиента
(давления), силы Кориолиса и силы трения.

\begin{itemize}
\tightlist
\item
  \textbf{Сила барического градиента} (G) -- направлена от области
  высокого давления к области низкого, являясь непосредственной причиной
  возникновения атмосферных движений крупного масштаба.
\item
  \textbf{Сила Кориолиса} (С) -- является инерционной, фиктивной силой,
  которая отклоняет любое свободно движущееся в горизонтальной плоскости
  тело в Северном полушарии вправо, а в Южном --- влево от начального
  направления движения. Вектор этой силы всегда направлен под прямым
  углом к вектору скорости ветра.
\item
  \textbf{Сила трения} (F или R) -- действует в приземном слое, уменьшая
  скорость ветра. Влияние силы трения особенно велико вблизи земной
  поверхности и ослабевает с высотой. Она составляет тупой угол с
  вектором ветра, приблизительно 140--160°.
\end{itemize}

\textbf{Механизм формирования:} У поверхности Земли, где влияние трения
максимально, ветер отклоняется от изобар под наибольшим углом (например,
около 30° над сушей в Северном полушарии). Это отклонение направлено к
области пониженного давления. По мере увеличения высоты, влияние трения
уменьшается, и относительная роль силы Кориолиса возрастает. Это
приводит к постепенному повороту вектора ветра вправо (в Северном
полушарии), приближая его к геострофическому балансу, где ветер дует
практически параллельно изобарам.

На рисунке 108, который схематически изображает спираль Экмана, можно
наглядно проследить это изменение:

\begin{itemize}
\tightlist
\item
  Буква `а' соответствует самой малой высоте, где ветер имеет наибольшее
  отклонение от изобар из-за сильного влияния трения.
\item
  По мере увеличения высоты, вектор ветра постепенно поворачивается
  (через `б' и `в').
\item
  Буква `г' указывает на наибольшую высоту в пределах пограничного слоя,
  где влияние трения становится незначительным, и ветер приближается к
  геострофическому или градиентному (если учитывать кривизну изобар).
\end{itemize}

Таким образом, спираль Экмана визуализирует вертикальный профиль ветра в
пограничном слое, демонстрируя, как ветер поворачивает и увеличивает
свою скорость с высотой от поверхности, где доминирует трение, до
уровня, где достигается квазигеострофический баланс.

ewpage

Как профессиональному метеорологу, нам хорошо известно, что градиентный
ветер представляет собой важную концепцию в динамике атмосферы,
описывающую движение воздуха вдоль криволинейных изобар. Он
характеризует баланс сил в горизонтальной плоскости, действующих на
воздушную частицу, а именно: силы барического градиента (G), силы
Кориолиса (A) и центробежной силы (C).

\hypertarget{ux433ux440ux430ux434ux438ux435ux43dux442ux43dux44bux439-ux432ux435ux442ux435ux440-ux431ux435ux437-ux443ux447ux435ux442ux430-ux441ux438ux43bux44b-ux442ux440ux435ux43dux438ux44f}{%
\subsubsection{Градиентный Ветер без Учета Силы
Трения}\label{ux433ux440ux430ux434ux438ux435ux43dux442ux43dux44bux439-ux432ux435ux442ux435ux440-ux431ux435ux437-ux443ux447ux435ux442ux430-ux441ux438ux43bux44b-ux442ux440ux435ux43dux438ux44f}}

В идеализированной свободной атмосфере, то есть выше пограничного слоя
(обычно выше 1-1.5 км), где силой трения (турбулентной вязкости) можно
пренебречь, баланс сил для градиентного ветра достигается между силой
барического градиента, силой Кориолиса и центробежной силой.

\hypertarget{ux432-ux446ux438ux43aux43bux43eux43dux435-ux43eux431ux43bux430ux441ux442ux44c-ux43fux43eux43dux438ux436ux435ux43dux43dux43eux433ux43e-ux434ux430ux432ux43bux435ux43dux438ux44f}{%
\paragraph{В Циклоне (Область Пониженного
Давления)}\label{ux432-ux446ux438ux43aux43bux43eux43dux435-ux43eux431ux43bux430ux441ux442ux44c-ux43fux43eux43dux438ux436ux435ux43dux43dux43eux433ux43e-ux434ux430ux432ux43bux435ux43dux438ux44f}}

В циклонической системе сила барического градиента (G) всегда направлена
к центру кривизны изобар, то есть к центру циклона, где давление ниже.
Центробежная сила (C) всегда направлена от центра кривизны траектории.
Сила Кориолиса (A), в Северном полушарии, направлена вправо от
направления движения ветра, и для поддержания баланса сил, она также
должна быть направлена к центру циклона. Таким образом, в циклоне сила
Кориолиса уравновешивает сумму силы барического градиента и центробежной
силы. Поскольку центробежная сила действует от центра, для сохранения
баланса при прочих равных условиях, сила Кориолиса должна быть меньше,
чем сила барического градиента. Это означает, что скорость градиентного
ветра в циклоне будет \textbf{субгеострофической}, то есть несколько
слабее геострофического ветра, который представляет собой частный случай
градиентного ветра при прямолинейных изобарах (когда центробежная сила
равна нулю). В случаях с малой кривизной изобар (радиус более 300 км)
этим различием часто пренебрегают, и расчет градиентного ветра заменяют
геострофическим.

\hypertarget{ux432-ux430ux43dux442ux438ux446ux438ux43aux43bux43eux43dux435-ux43eux431ux43bux430ux441ux442ux44c-ux43fux43eux432ux44bux448ux435ux43dux43dux43eux433ux43e-ux434ux430ux432ux43bux435ux43dux438ux44f}{%
\paragraph{В Антициклоне (Область Повышенного
Давления)}\label{ux432-ux430ux43dux442ux438ux446ux438ux43aux43bux43eux43dux435-ux43eux431ux43bux430ux441ux442ux44c-ux43fux43eux432ux44bux448ux435ux43dux43dux43eux433ux43e-ux434ux430ux432ux43bux435ux43dux438ux44f}}

В антициклонической системе сила барического градиента (G) направлена от
центра кривизны изобар, то есть от центра антициклона, где давление
выше. Центробежная сила (C) также направлена от центра. Чтобы
уравновесить эти силы, сила Кориолиса (A) должна быть направлена к
центру антициклона (то есть вправо от направления ветра в Северном
полушарии) и должна быть больше, чем суммарная центробежная сила и сила
барического градиента. Для поддержания этого баланса, скорость
градиентного ветра в антициклоне должна быть
\textbf{супергеострофической}, то есть сильнее геострофического ветра.
Это означает, что воздушные частицы движутся быстрее, чем требовалось бы
только для баланса давления и Кориолиса.

\hypertarget{ux433ux440ux430ux434ux438ux435ux43dux442ux43dux44bux439-ux432ux435ux442ux435ux440-ux441-ux443ux447ux435ux442ux43eux43c-ux441ux438ux43bux44b-ux442ux440ux435ux43dux438ux44f}{%
\subsubsection{Градиентный Ветер с Учетом Силы
Трения}\label{ux433ux440ux430ux434ux438ux435ux43dux442ux43dux44bux439-ux432ux435ux442ux435ux440-ux441-ux443ux447ux435ux442ux43eux43c-ux441ux438ux43bux44b-ux442ux440ux435ux43dux438ux44f}}

В пограничном слое атмосферы, который простирается от земной поверхности
до высоты около 1-2 км, сила трения (турбулентной вязкости) становится
существенным фактором, влияющим на движение воздуха. Сила трения всегда
направлена противоположно направлению движения воздуха.

\hypertarget{ux432-ux446ux438ux43aux43bux43eux43dux435}{%
\paragraph{В Циклоне}\label{ux432-ux446ux438ux43aux43bux43eux43dux435}}

При учете силы трения, которая замедляет движение воздуха, баланс сил
изменяется. В циклоне, поскольку сила трения уменьшает скорость ветра,
сила Кориолиса, пропорциональная скорости ветра, также уменьшается. Это
нарушает баланс между силой барического градиента (G) и силой Кориолиса
(A), а также центробежной силой (C). Чтобы восстановить равновесие,
воздушная частица отклоняется от изобар в сторону пониженного давления,
то есть к центру циклона. Это отклонение приводит к
\textbf{конвергенции} (сходимости) воздушных потоков у поверхности земли
в циклоне. Поскольку воздух не может накапливаться у земли, возникает
\textbf{восходящее движение} воздуха. Восходящие движения, в свою
очередь, благоприятствуют охлаждению воздуха, конденсации водяного пара,
образованию облаков (включая слоисто-дождевые и кучево-дождевые) и
выпадению осадков. Таким образом, циклоны в пограничном слое, за счет
действия трения, становятся областями активной облачности и осадков.

\hypertarget{ux432-ux430ux43dux442ux438ux446ux438ux43aux43bux43eux43dux435}{%
\paragraph{В
Антициклоне}\label{ux432-ux430ux43dux442ux438ux446ux438ux43aux43bux43eux43dux435}}

В антициклоне сила трения также уменьшает скорость ветра. Это приводит к
уменьшению силы Кориолиса, которая в антициклоне должна быть больше
других сил для поддержания баланса. В результате, для восстановления
равновесия, воздушная частица отклоняется от изобар в сторону
пониженного давления, но поскольку в антициклоне это означает движение
от центра, то воздушная частица отклоняется от центра антициклона. Это
отклонение приводит к \textbf{дивергенции} (расходимости) воздушных
потоков у поверхности земли в антициклоне. Расходящийся воздух у земли
восполняется за счет \textbf{нисходящих потоков} из вышележащих слоев
атмосферы. Нисходящие движения приводят к адиабатическому нагреванию
воздуха, что препятствует облакообразованию и способствует развитию
инверсий температуры и накоплению загрязняющих веществ у поверхности.
Поэтому антициклоны в пограничном слое обычно ассоциируются с ясной,
безоблачной погодой.

Различие в угле отклонения ветра от изобар также наблюдается в
пограничном слое: над сушей этот угол составляет около 30°, тогда как
над морем, где трение меньше, он меньше. Это указывает на более близкое
приближение ветра к градиентному (или геострофическому) над водными
поверхностями.

ewpage

Уважаемый коллега,

При анализе динамических факторов возникновения атмосферной
турбулентности, мы, как опытные специалисты, концентрируемся на
фундаментальных процессах, определяющих хаотическое, но подчиненное
определенным закономерностям движение воздуха. Эти факторы тесно
взаимосвязаны и проявляются на различных пространственно-временных
масштабах.

\hypertarget{ux442ux443ux440ux431ux443ux43bux435ux43dux442ux43dux430ux44f-ux432ux44fux437ux43aux43eux441ux442ux44c-ux442ux440ux435ux43dux438ux435}{%
\subsubsection{1. Турбулентная Вязкость
(Трение)}\label{ux442ux443ux440ux431ux443ux43bux435ux43dux442ux43dux430ux44f-ux432ux44fux437ux43aux43eux441ux442ux44c-ux442ux440ux435ux43dux438ux435}}

Турбулентность в атмосфере является повсеместным явлением, наиболее ярко
проявляющимся в порывистости ветра. В отличие от молекулярной вязкости,
влияние которой на атмосферные движения, как правило, ничтожно и ею
можно пренебречь из-за очень больших чисел Рейнольдса, турбулентная
вязкость играет ключевую роль.

\begin{itemize}
\tightlist
\item
  \textbf{Диссипация Кинетической Энергии}: Действие турбулентного
  трения в пограничном слое атмосферы, который простирается на высоту
  500-1500 м и часто называется слоем трения, приводит к постоянному
  ослаблению скоростей ветра и диссипации кинетической энергии,
  превращая ее в тепловую и другие виды энергии.
\item
  \textbf{Влияние Подстилающей Поверхности}: Различия в шероховатости
  подстилающей поверхности, например, между сушей и морем, приводят к
  различиям в турбулентном трении. Это, в свою очередь, вызывает
  сходимость или расходимость потоков при переходе воздушной массы с
  одной поверхности на другую, оказывая существенное влияние на режим
  ветра. Над водой ветер усиливается и слабее отклоняется от изобар по
  сравнению с сушей.
\end{itemize}

\hypertarget{ux431ux430ux440ux43eux43aux43bux438ux43dux43dux43eux441ux442ux44c-ux430ux442ux43cux43eux441ux444ux435ux440ux44b}{%
\subsubsection{2. Бароклинность
Атмосферы}\label{ux431ux430ux440ux43eux43aux43bux438ux43dux43dux43eux441ux442ux44c-ux430ux442ux43cux43eux441ux444ux435ux440ux44b}}

Бароклинность, характеризующаяся пересечением изобарических и
изотермических поверхностей (наличием термодинамических соленоидов),
является фундаментальным динамическим фактором, поскольку атмосфера
практически всегда бароклинна.

\begin{itemize}
\tightlist
\item
  \textbf{Генерация Вихревых Движений}: Горизонтальная разность
  виртуальных температур, связанная с бароклинностью, одновременно
  порождает вихревые движения (циркуляции) как в горизонтальной, так и в
  вертикальной плоскостях. Это имеет решающее значение для образования и
  развития синоптических вихрей (циклонов и антициклонов), атмосферных
  фронтов и фронтальных зон, а также движений муссонно-бризового типа.
\item
  \textbf{Термическая Адвекция}: Адвективное изменение виртуальной
  температуры (термическая адвекция) является ключевым бароклинным
  фактором. Адвекция холодного воздуха, например, способствует усилению
  циклонического вихря, в то время как адвекция теплого воздуха
  усиливает антициклонический.
\item
  \textbf{Усиление Ветра}: В зонах атмосферных фронтов, где
  бароклинность проявляется наиболее явно, сгущение термодинамических
  соленоидов приводит к значительному ускорению циркуляции и, как
  следствие, к увеличению скорости ветра по сравнению с соседними
  районами.
\end{itemize}

\hypertarget{ux432ux435ux440ux442ux438ux43aux430ux43bux44cux43dux44bux435-ux434ux432ux438ux436ux435ux43dux438ux44f-ux432ux43eux437ux434ux443ux445ux430}{%
\subsubsection{3. Вертикальные Движения
Воздуха}\label{ux432ux435ux440ux442ux438ux43aux430ux43bux44cux43dux44bux435-ux434ux432ux438ux436ux435ux43dux438ux44f-ux432ux43eux437ux434ux443ux445ux430}}

Вертикальные движения играют исключительно важную роль в формировании
погоды и непосредственно связаны с возникновением турбулентности,
особенно конвективной.

\begin{itemize}
\tightlist
\item
  \textbf{Конвекция}: Основной причиной вертикальных движений является
  разность температур между движущейся порцией воздуха и окружающей
  средой, что порождает архимедовы силы (плавучесть). Неустойчивая
  термическая стратификация атмосферы (\(\gamma_a < \gamma\)) приводит к
  развитию мощных восходящих движений --- конвекции. Конвективные
  движения, особенно в кучевых и кучево-дождевых облаках, могут
  достигать значительных скоростей и высот.
\item
  \textbf{Конвергенция/Дивергенция}: Сходимость (конвергенция) воздушных
  потоков у поверхности земли является динамическим фактором, приводящим
  к восходящим движениям, что, в свою очередь, способствует развитию
  облачности и осадков, особенно мощных кучево-дождевых облаков в зоне
  внутритропической конвергенции. Напротив, нисходящие движения в
  антициклонах приводят к дивергенции воздушных потоков у поверхности и
  формированию инверсий оседания.
\item
  \textbf{Влияние на Тропопаузу}: Турбулентное перемешивание и
  вертикальные движения оказывают существенное влияние на формирование
  тропопаузы.
\end{itemize}

\hypertarget{ux432ux438ux445ux440ux435ux432ux44bux435-ux434ux432ux438ux436ux435ux43dux438ux44f-ux438-ux441ux434ux432ux438ux433ux438-ux441ux43aux43eux440ux43eux441ux442ux438}{%
\subsubsection{4. Вихревые Движения и Сдвиги
Скорости}\label{ux432ux438ux445ux440ux435ux432ux44bux435-ux434ux432ux438ux436ux435ux43dux438ux44f-ux438-ux441ux434ux432ux438ux433ux438-ux441ux43aux43eux440ux43eux441ux442ux438}}

Вихрь скорости ветра, представляющий собой ротор вектора скорости,
характеризует вращательное движение частиц воздуха. Атмосферные
движения, как правило, носят вихревой характер.

\begin{itemize}
\tightlist
\item
  \textbf{Трехмерность Вихрей}: Вихревое движение в атмосфере является
  трехмерным, проявляясь как в горизонтальной, так и в вертикальной
  плоскостях.
\item
  \textbf{Струйные Течения и ВВФЗ}: Высотные фронтальные зоны (ВФЗ) и
  струйные течения являются областями концентрации кинетической и
  внутренней энергии, где наблюдаются значительные горизонтальные
  градиенты давления и температуры, а также интенсивные сдвиги скорости
  ветра. Эти сдвиги способствуют развитию турбулентности.
\item
  \textbf{Болтанка Самолетов}: Такие явления, как болтанка самолетов,
  часто возникают в зонах больших горизонтальных и вертикальных сдвигов
  ветра, особенно на циклонической периферии струйных течений.
\end{itemize}

\hypertarget{ux43eux440ux43eux433ux440ux430ux444ux438ux44f}{%
\subsubsection{5.
Орография}\label{ux43eux440ux43eux433ux440ux430ux444ux438ux44f}}

Горные массивы оказывают значительное влияние на атмосферные процессы,
способствуя возникновению турбулентности.

\begin{itemize}
\tightlist
\item
  \textbf{Орографические Вертикальные Движения}: Орографические
  препятствия вынуждают воздушные потоки изменять свое направление и
  высоту, что приводит к орографическим восходящим или нисходящим
  движениям.
\item
  \textbf{Орографический Фронтогенез}: Горы могут способствовать
  обострению существующих фронтов или образованию новых, влияя на поля
  температуры воздушных масс. Например, с подветренной стороны гор часто
  наблюдается циклогенез, а с наветренной --- антициклогенез.
\end{itemize}

\hypertarget{ux43dux435ux441ux442ux430ux446ux438ux43eux43dux430ux440ux43dux43eux441ux442ux44c-ux43aux440ux443ux43fux43dux43eux43cux430ux441ux448ux442ux430ux431ux43dux44bux445-ux432ux43eux437ux434ux443ux448ux43dux44bux445-ux442ux435ux447ux435ux43dux438ux439}{%
\subsubsection{6. Нестационарность Крупномасштабных Воздушных
Течений}\label{ux43dux435ux441ux442ux430ux446ux438ux43eux43dux430ux440ux43dux43eux441ux442ux44c-ux43aux440ux443ux43fux43dux43eux43cux430ux441ux448ux442ux430ux431ux43dux44bux445-ux432ux43eux437ux434ux443ux448ux43dux44bux445-ux442ux435ux447ux435ux43dux438ux439}}

Реальные атмосферные движения характеризуются нестационарностью, что
приводит к отклонениям от идеализированных моделей (например,
градиентного или геострофического ветра). Эта нестационарность, вкупе с
действием приземного трения, способствует развитию вертикальных
движений, особенно вблизи атмосферных фронтов и в центральных частях
циклонов и антициклонов.

Таким образом, атмосферная турбулентность является результатом
комплексного взаимодействия множества динамических факторов, включая
трение (турбулентную вязкость), бароклинность, вертикальные движения,
вихревые процессы (сдвиги скорости), орографические эффекты и
нестационарность крупномасштабных потоков.

ewpage

Как коллеге, работающему в области динамической метеорологии, мне не
нужно напоминать о фундаментальном значении турбулентности для понимания
атмосферных процессов. Турбулентное движение является доминирующим
режимом в большинстве слоев атмосферы, и его корректное описание
критически важно для численного моделирования и прогнозирования погоды.

\hypertarget{ux445ux430ux440ux430ux43aux442ux435ux440ux438ux441ux442ux438ux43aux438-ux442ux443ux440ux431ux443ux43bux435ux43dux442ux43dux43eux433ux43e-ux434ux432ux438ux436ux435ux43dux438ux44f}{%
\subsubsection{Характеристики Турбулентного
Движения}\label{ux445ux430ux440ux430ux43aux442ux435ux440ux438ux441ux442ux438ux43aux438-ux442ux443ux440ux431ux443ux43bux435ux43dux442ux43dux43eux433ux43e-ux434ux432ux438ux436ux435ux43dux438ux44f}}

Турбулентное движение воздуха, в отличие от ламинарного, характеризуется
крайней сложностью и неупорядоченностью. Оно представляет собой
наложение сложного, запутанного движения множества отдельных частиц и
вихрей на основное, осредненное движение.

Основные характеристики турбулентности включают:

\begin{itemize}
\tightlist
\item
  \textbf{Пульсации (флуктуации)}: Для турбулентного движения характерны
  резкие, беспорядочные колебания скорости, давления и плотности. Эти
  флуктуации проявляются, например, в порывистости ветра. Мгновенные
  значения метеорологических величин в турбулентной атмосфере являются
  случайными и практически не поддаются прямому использованию.
\item
  \textbf{Перемешивание (турбулентная диффузия)}: Множество вихрей и
  частиц воздуха постоянно перемешиваются с окружающей средой. Этот
  процесс сопровождается турбулентной вязкостью, диффузией (массы,
  влаги, примесей) и теплопроводностью, что приводит к выравниванию в
  пространстве средних характеристик (например, средней скорости
  основного движения).
\item
  \textbf{Турбулентная вязкость/трение}: В отличие от молекулярной
  вязкости, турбулентная вязкость значительно усиливает перенос
  количества движения и других субстанций. В уравнениях осредненного
  движения турбулентность вводит новые члены, называемые турбулентными
  напряжениями (тензором турбулентных напряжений). Каждое такое
  напряжение определяет турбулентные потоки составляющих количества
  движения, переносимого пульсационными скоростями, что приводит к
  развитию силы внутреннего турбулентного трения.
\end{itemize}

\hypertarget{ux442ux443ux440ux431ux443ux43bux435ux43dux442ux43dux43eux441ux442ux44c-ux432-ux430ux442ux43cux43eux441ux444ux435ux440ux435}{%
\subsubsection{Турбулентность в
Атмосфере}\label{ux442ux443ux440ux431ux443ux43bux435ux43dux442ux43dux43eux441ux442ux44c-ux432-ux430ux442ux43cux43eux441ux444ux435ux440ux435}}

Практически все движения в атмосфере носят турбулентный характер. Это
особенно выражено в пограничном слое, непосредственно прилегающем к
земной поверхности (толщиной до нескольких десятков метров), где
отклоняющая сила вращения Земли становится очень малой по сравнению с
силой турбулентной вязкости. Коэффициент турбулентного обмена при этом
уменьшается по мере приближения к земной поверхности.

Шероховатость подстилающей поверхности оказывает существенное влияние на
силу трения и, как следствие, на характеристики ветра. Например, над
водой или снегом, которые являются более гладкими поверхностями,
скорость ветра выше, а отклонение направления ветра от изобар слабее,
чем над более шероховатыми поверхностями, такими как трава или лес.

\hypertarget{ux43cux435ux442ux43eux434ux44b-ux438ux437ux443ux447ux435ux43dux438ux44f-ux438-ux43cux43eux434ux435ux43bux438ux440ux43eux432ux430ux43dux438ux44f-ux442ux443ux440ux431ux443ux43bux435ux43dux442ux43dux43eux441ux442ux438}{%
\subsubsection{Методы Изучения и Моделирования
Турбулентности}\label{ux43cux435ux442ux43eux434ux44b-ux438ux437ux443ux447ux435ux43dux438ux44f-ux438-ux43cux43eux434ux435ux43bux438ux440ux43eux432ux430ux43dux438ux44f-ux442ux443ux440ux431ux443ux43bux435ux43dux442ux43dux43eux441ux442ux438}}

Из-за случайного и беспорядочного характера турбулентных движений, для
их исследования и моделирования используются статистические методы.
Метеорологические величины усредняются за определенный отрезок времени,
который должен быть достаточно продолжительным для репрезентативности,
но не слишком большим, чтобы не сглаживать существенные изменения.

Вследствие турбулентности воздуха, в динамической метеорологии
приходится использовать осредненные уравнения гидротермодинамики. Однако
при таком осреднении появляются новые неизвестные величины (турбулентные
напряжения, турбулентные потоки тепла, влаги и других субстанций),
которые усложняют систему уравнений и требуют дополнительных моделей или
гипотез для их замыкания.

Вертикальные турбулентные потоки количества движения или других
субстанций (\(\phi\)) могут быть приближенно определены по формуле:
\(\Phi = -\rho K \frac{\partial \phi}{\partial z}\), где \(K\) --
коэффициент турбулентной диффузии, а знак минус указывает на перенос в
сторону уменьшения средней величины.

Например, вертикальный турбулентный поток тепла в приземном слое на
высоте нескольких десятков метров практически не зависит от высоты и
может быть рассчитан как произведение скорости ветра, разности значений
субстанции у земли и на некоторой высоте, а также безразмерного
коэффициента турбулентной проводимости.

ewpage

Как коллеги-метеорологи, мы хорошо осведомлены, что турбулентный поток
является неотъемлемой частью атмосферной динамики, и его влияние на
перенос различных субстанций имеет фундаментальное значение для
понимания и прогнозирования погодных процессов.

\hypertarget{ux442ux443ux440ux431ux443ux43bux435ux43dux442ux43dux44bux439-ux43fux43eux442ux43eux43a}{%
\subsection{Турбулентный
Поток}\label{ux442ux443ux440ux431ux443ux43bux435ux43dux442ux43dux44bux439-ux43fux43eux442ux43eux43a}}

\textbf{Определение и Характеристики} Турбулентное движение жидкости или
газа -- это сложный, беспорядочный процесс, при котором на основное
осредненное движение накладывается хаотичное перемешивание множества
отдельных частиц и вихрей. В отличие от ламинарного движения, где поток
упорядочен, в турбулентном потоке случайные возмущения не затухают, а
развиваются, приводя к интенсивному перемешиванию струй среды. Для
анализа турбулентного движения атмосферные характеристики (например,
скорость, температура, влажность) принято представлять в виде суммы
медленно меняющихся осредненных значений и нерегулярных пульсаций.
Среднее значение пульсаций при этом равно нулю.

\textbf{Уравнения Осредненного Движения} При усреднении общих уравнений
гидротермодинамики для турбулентной атмосферы (уравнения Рейнольдса), в
них появляются новые неизвестные члены, называемые \textbf{добавочными
турбулентными напряжениями}. Эти напряжения представляют собой
произведения осредненных пульсационных скоростей, например,
\(\overline{u'u'}\), \(\overline{v'u'}\), \(\overline{w'u'}\). Они
характеризуют турбулентные потоки составляющих количества движения и
отражают тенденцию к выравниванию средней скорости основного движения, а
также к развитию внутренних сил турбулентного трения. Определение этих
турбулентных напряжений является сложной задачей и требует использования
статистической теории турбулентности или аналогий с молекулярным
движением.

\hypertarget{ux43fux440ux438ux442ux43eux43a-ux441ux443ux431ux441ux442ux430ux43dux446ux438ux438-ux442ux443ux440ux431ux443ux43bux435ux43dux442ux43dux44bux435-ux43fux43eux442ux43eux43aux438}{%
\subsection{Приток Субстанции (Турбулентные
Потоки)}\label{ux43fux440ux438ux442ux43eux43a-ux441ux443ux431ux441ux442ux430ux43dux446ux438ux438-ux442ux443ux440ux431ux443ux43bux435ux43dux442ux43dux44bux435-ux43fux43eux442ux43eux43aux438}}

Турбулентность является ключевым механизмом для переноса различных
субстанций в атмосфере, таких как количество движения (импульс), тепло,
влага и электрический заряд (ионы). Этот перенос часто называют
турбулентным обменом.

\hypertarget{ux442ux443ux440ux431ux443ux43bux435ux43dux442ux43dux44bux439-ux43fux43eux442ux43eux43a-ux43aux43eux43bux438ux447ux435ux441ux442ux432ux430-ux434ux432ux438ux436ux435ux43dux438ux44f-ux438ux43cux43fux443ux43bux44cux441ux430}{%
\subsubsection{1. Турбулентный Поток Количества Движения
(Импульса)}\label{ux442ux443ux440ux431ux443ux43bux435ux43dux442ux43dux44bux439-ux43fux43eux442ux43eux43a-ux43aux43eux43bux438ux447ux435ux441ux442ux432ux430-ux434ux432ux438ux436ux435ux43dux438ux44f-ux438ux43cux43fux443ux43bux44cux441ux430}}

Как уже упоминалось, добавочные турбулентные напряжения в уравнениях
Рейнольдса напрямую определяют турбулентные потоки составляющих
количества движения. Вертикальный турбулентный поток количества движения
(\(\varphi\)) может быть приближенно выражен формулой:
\(\varphi = -\rho K \frac{\partial \overline{u}}{\partial z}\) где
\(\rho\) --- плотность воздуха, \(K\) --- кинематический коэффициент
вязкости (или коэффициент турбулентного обмена),
\(\frac{\partial \overline{u}}{\partial z}\) --- вертикальный градиент
осредненной скорости. Отрицательный знак указывает на то, что количество
движения переносится в сторону уменьшения средней скорости основного
движения.

\hypertarget{ux442ux443ux440ux431ux443ux43bux435ux43dux442ux43dux44bux439-ux43fux43eux442ux43eux43a-ux442ux435ux43fux43bux430}{%
\subsubsection{2. Турбулентный Поток
Тепла}\label{ux442ux443ux440ux431ux443ux43bux435ux43dux442ux43dux44bux439-ux43fux43eux442ux43eux43a-ux442ux435ux43fux43bux430}}

Турбулентные потоки тепла (\(P_x, P_y, P_z\)) определяются как
произведения осредненных пульсаций скорости и потенциальной температуры,
умноженные на плотность и удельную теплоемкость воздуха при постоянном
давлении (\(\rho c_p\)). Например:
\(P_x = \rho c_p \overline{u'\theta'}\) Эти потоки могут быть выражены
через коэффициенты турбулентного обмена для переноса тепла
(\(k_x, k_y, k_z\)) и градиенты осредненной потенциальной температуры:
\(P_z = -\rho c_p k_z \frac{\partial \overline{\theta}}{\partial z}\)
Вертикальный турбулентный поток тепла (\(W \overline{T}\)) является
наиболее важным для нагревания или охлаждения атмосферы и определяется
пульсациями вертикальных скоростей и температуры (\(W'T'\)).
Турбулентный обмен явным теплом из почвы и воды играет важную роль в
изменении температуры и других метеорологических величин.

\hypertarget{ux442ux443ux440ux431ux443ux43bux435ux43dux442ux43dux44bux439-ux43fux43eux442ux43eux43a-ux432ux43bux430ux433ux438}{%
\subsubsection{3. Турбулентный Поток
Влаги}\label{ux442ux443ux440ux431ux443ux43bux435ux43dux442ux43dux44bux439-ux43fux43eux442ux43eux43a-ux432ux43bux430ux433ux438}}

Турбулентный обмен скрытым теплом, связанный с переносом водяного пара,
также играет существенную роль. Источники указывают, что горизонтальный
и вертикальный турбулентный обмен являются основными динамическими
факторами в формировании облаков и туманов. При образовании слоистых и
слоисто-кучевых облаков значительную роль играет поступление водяного
пара в приземный слой воздуха с подстилающей поверхности за счет
турбулентного теплообмена. В облаках наблюдаются значительные
турбулентные пульсации влажности, что влияет на возникновение
неустойчивой стратификации и выпадение осадков. Интенсивные вертикальные
движения (часто турбулентные) внутри мощных облаков обеспечивают быстрое
слияние (коагуляцию) облачных частиц, приводя к выпадению осадков.

\hypertarget{ux442ux443ux440ux431ux443ux43bux435ux43dux442ux43dux44bux439-ux43fux43eux442ux43eux43a-ux44dux43bux435ux43aux442ux440ux438ux447ux435ux441ux43aux43eux433ux43e-ux437ux430ux440ux44fux434ux430-ux442ux43eux43aux438}{%
\subsubsection{4. Турбулентный Поток Электрического Заряда
(Токи)}\label{ux442ux443ux440ux431ux443ux43bux435ux43dux442ux43dux44bux439-ux43fux43eux442ux43eux43a-ux44dux43bux435ux43aux442ux440ux438ux447ux435ux441ux43aux43eux433ux43e-ux437ux430ux440ux44fux434ux430-ux442ux43eux43aux438}}

В контексте атмосферного электричества, перенос заряда также может
рассматриваться как приток субстанции. Вертикальные электрические токи в
атмосфере при отсутствии осадков и грозовых явлений включают
конвективные токи, которые рассчитываются как произведение скорости
вертикальных конвективных потоков (\(w\)) на плотность объемных зарядов
(\(\rho\)) в исследуемой области (\(i_{\text{конв}} = w\rho\)).
Горизонтальные токи, связанные с переносом объемных зарядов воздушными
течениями в горизонтальном направлении, определяются аналогично
(\(i_г = v\rho\)), где \(v\) --- горизонтальная скорость воздушных
течений. Проводимость атмосферы, характеризующая ее проводящие свойства,
определяется концентрацией и подвижностью атмосферных ионов, что
напрямую связано с их турбулентным переносом.

\hypertarget{ux43eux431ux449ux438ux439-ux43fux43eux434ux445ux43eux434-ux43a-ux43aux43eux43dux441ux435ux440ux432ux430ux442ux438ux432ux43dux44bux43c-ux438-ux43fux430ux441ux441ux438ux432ux43dux44bux43c-ux441ux443ux431ux441ux442ux430ux43dux446ux438ux44fux43c}{%
\subsubsection{5. Общий Подход к Консервативным и Пассивным
Субстанциям}\label{ux43eux431ux449ux438ux439-ux43fux43eux434ux445ux43eux434-ux43a-ux43aux43eux43dux441ux435ux440ux432ux430ux442ux438ux432ux43dux44bux43c-ux438-ux43fux430ux441ux441ux438ux432ux43dux44bux43c-ux441ux443ux431ux441ux442ux430ux43dux446ux438ux44fux43c}}

В общем случае, для любой консервативной и пассивной величины \(\phi\),
вертикальный турбулентный поток \(\Phi_\phi\) определяется формулой:
\(\Phi_\phi = -\rho K \frac{\partial \overline{\phi}}{\partial z}\) где
\(\overline{\phi}\) --- среднее значение этой величины.

\hypertarget{ux441ux43bux435ux434ux441ux442ux432ux438ux44f-ux438-ux437ux43dux430ux447ux435ux43dux438ux435}{%
\subsection{Следствия и
Значение}\label{ux441ux43bux435ux434ux441ux442ux432ux438ux44f-ux438-ux437ux43dux430ux447ux435ux43dux438ux435}}

Турбулентность и связанные с ней потоки субстанций играют определяющую
роль в формировании погоды и климата. Они обеспечивают эффективный
вертикальный и горизонтальный перенос тепла, влаги и импульса, что
существенно влияет на температуру воздуха, формирование и эволюцию
облаков, осадков, а также на распределение электрического заряда в
атмосфере. Влияние орографии на атмосферные процессы, такие как усиление
вертикальных движений, также неразрывно связано с турбулентным
перемешиванием. Однако, учет турбулентности значительно усложняет
математическое моделирование атмосферных процессов, так как при
усреднении уравнений возникают новые неизвестные величины, что делает
систему уравнений незамкнутой.

ewpage

\hypertarget{ux43fux43eux43dux44fux442ux438ux435-ux43e-ux43fux440ux438ux437ux435ux43cux43dux43eux43c-ux438-ux43fux43eux433ux440ux430ux43dux438ux447ux43dux43eux43c-ux441ux43bux43eux44fux445-ux430ux442ux43cux43eux441ux444ux435ux440ux44b}{%
\section{Понятие о Приземном и Пограничном Слоях
Атмосферы}\label{ux43fux43eux43dux44fux442ux438ux435-ux43e-ux43fux440ux438ux437ux435ux43cux43dux43eux43c-ux438-ux43fux43eux433ux440ux430ux43dux438ux447ux43dux43eux43c-ux441ux43bux43eux44fux445-ux430ux442ux43cux43eux441ux444ux435ux440ux44b}}

Как профессионалам в области метеорологии, нам важно четко различать и
понимать динамические и термодинамические особенности различных слоев
атмосферы, особенно тех, что находятся непосредственно над земной
поверхностью и активно взаимодействуют с ней. Эти слои, приземный и
пограничный, играют ключевую роль в формировании погоды и климата.

\hypertarget{ux43fux43eux433ux440ux430ux43dux438ux447ux43dux44bux439-ux441ux43bux43eux439-ux430ux442ux43cux43eux441ux444ux435ux440ux44b-ux441ux43bux43eux439-ux442ux440ux435ux43dux438ux44f}{%
\subsection{Пограничный Слой Атмосферы (Слой
Трения)}\label{ux43fux43eux433ux440ux430ux43dux438ux447ux43dux44bux439-ux441ux43bux43eux439-ux430ux442ux43cux43eux441ux444ux435ux440ux44b-ux441ux43bux43eux439-ux442ux440ux435ux43dux438ux44f}}

Пограничный слой атмосферы, также известный как планетарный пограничный
слой или слой трения, представляет собой нижнюю часть тропосферы,
простирающуюся от земной поверхности до высоты 500-1500 метров. Его
типичная толщина составляет 1000-1500 метров.

\hypertarget{ux43aux43bux44eux447ux435ux432ux44bux435-ux445ux430ux440ux430ux43aux442ux435ux440ux438ux441ux442ux438ux43aux438-ux438-ux43fux440ux43eux446ux435ux441ux441ux44b}{%
\subsubsection{Ключевые Характеристики и
Процессы}\label{ux43aux43bux44eux447ux435ux432ux44bux435-ux445ux430ux440ux430ux43aux442ux435ux440ux438ux441ux442ux438ux43aux438-ux438-ux43fux440ux43eux446ux435ux441ux441ux44b}}

\begin{itemize}
\tightlist
\item
  \textbf{Баланс Сил:} В пределах этого слоя сила турбулентного трения и
  сила Кориолиса имеют сопоставимый порядок величины. Выше пограничного
  слоя, в свободной атмосфере, силой турбулентного трения, по сравнению
  с отклоняющей силой вращения Земли, можно пренебречь.
\item
  \textbf{Взаимодействие с Подстилающей Поверхностью:} Именно в
  пограничном слое наиболее ярко проявляется взаимодействие атмосферы с
  подстилающей поверхностью. Суточные колебания температуры, например,
  распространяются на высоту до 1.5 км в этом слое. Вертикальное
  изменение температуры в этом слое значительно интенсивнее, чем в
  средней тропосфере.
\item
  \textbf{Влияние Трения на Ветер:} Трение воздуха о земную поверхность
  постоянно уменьшает скорость воздушных течений и изменяет их
  направление. В пограничном слое это приводит к отклонению ветра от
  изобар в сторону пониженного давления.

  \begin{itemize}
  \tightlist
  \item
    \textbf{В Циклонах:} Такое отклонение способствует конвергенции
    (сходимости) воздушных потоков у поверхности земли, что приводит к
    возникновению восходящих движений воздуха и формированию облаков.
  \item
    \textbf{В Антициклонах:} Напротив, наблюдается дивергенция
    (расходимость) воздушных потоков, восполняемая нисходящими
    движениями из вышележащих слоев, что способствует развитию инверсий
    температуры и накоплению загрязняющих веществ у поверхности.
  \end{itemize}
\item
  \textbf{Турбулентность:} Турбулентный обмен и перемешивание масс
  воздуха являются характерными чертами процессов, происходящих в
  пограничном слое. Для описания турбулентных движений используются
  осредненные уравнения движения, а для определения турбулентных
  напряжений применяются приближенные методы, основанные на аналогии
  между молекулярным и турбулентным движениями. Теории подобия и
  размерности, в частности работы А.С. Монина и А.М. Обухова, имеют
  основополагающее значение для исследования турбулентности в приземном
  слое.
\item
  \textbf{Слой Экмана:} Верхняя, большая часть пограничного слоя иногда
  называется слоем Экмана. В этом слое учитывается влияние силы
  Кориолиса на изменение модуля и направления скорости ветра с высотой.
\item
  \textbf{Фронтальные Поверхности:} В пограничном слое атмосферные
  фронтальные поверхности могут быть сильно размыты из-за интенсивного
  турбулентного перемешивания, особенно при сильных ветрах, что
  затрудняет их точное определение.
\end{itemize}

\hypertarget{ux43fux440ux438ux437ux435ux43cux43dux44bux439-ux441ux43bux43eux439-ux430ux442ux43cux43eux441ux444ux435ux440ux44b}{%
\subsection{Приземный Слой
Атмосферы}\label{ux43fux440ux438ux437ux435ux43cux43dux44bux439-ux441ux43bux43eux439-ux430ux442ux43cux43eux441ux444ux435ux440ux44b}}

Приземный слой атмосферы является самым нижним подслоем пограничного
слоя. Он простирается на несколько десятков метров от земной
поверхности, обычно на 30-50 метров.

\hypertarget{ux43eux442ux43bux438ux447ux438ux442ux435ux43bux44cux43dux44bux435-ux43eux441ux43eux431ux435ux43dux43dux43eux441ux442ux438}{%
\subsubsection{Отличительные
Особенности}\label{ux43eux442ux43bux438ux447ux438ux442ux435ux43bux44cux43dux44bux435-ux43eux441ux43eux431ux435ux43dux43dux43eux441ux442ux438}}

\begin{itemize}
\tightlist
\item
  \textbf{Баланс Сил:} В отличие от остальной части пограничного слоя, в
  приземном слое отклоняющая сила вращения Земли (сила Кориолиса) очень
  мала по сравнению с силой турбулентной вязкости.
\item
  \textbf{Толщина:} Его толщина не является постоянной и может
  варьироваться от 20-30 м при сильно устойчивой (инверсионной)
  термической стратификации до 200-300 м при неустойчивой термической
  стратификации.
\item
  \textbf{Турбулентный Обмен:} Коэффициент турбулентного обмена в
  приземном подслое уменьшается по мере приближения к земной
  поверхности.
\item
  \textbf{Градиенты Температуры:} Вертикальные градиенты температуры в
  приземном слое могут быть крайне высокими. Например, в ясный летний
  день они могут превышать 0.2 °C/м, а при ночных инверсиях --- быть
  меньше -0.1 °C/м.
\item
  \textbf{Отклонение Ветра:} Угол отклонения ветра от изобары в
  приземном слое зависит от безразмерных параметров, связанных с
  характеристиками турбулентности и подстилающей поверхности. Например,
  скорость ветра над водной поверхностью, как правило, больше, чем над
  сушей, а направление ветра отклоняется от изобар слабее, поскольку
  водная поверхность обладает меньшей шероховатостью.
\item
  \textbf{Упрощение Уравнений:} Для приземного слоя уравнения движения и
  притока тепла могут быть значительно упрощены.
\end{itemize}

Взаимосвязь между этими слоями неоспорима: приземный слой является
фундаментом, на котором развиваются более масштабные процессы
пограничного слоя, а вместе они формируют ключевое звено в сложной
системе циркуляции атмосферы и определяют локальные погодные условия.

ewpage

Уважаемые коллеги, изменение скорости ветра с высотой представляет собой
один из фундаментальных аспектов атмосферной динамики, который, как мы
знаем, подчиняется сложным физическим законам и значительно варьируется
в зависимости от ряда факторов. Понимание этого вертикального профиля
ветра (или сдвига ветра) критически важно для широкого спектра
метеорологических приложений, от прогнозирования погоды до обеспечения
безопасности полетов. Давайте рассмотрим это явление, разделяя атмосферу
на характерные слои.

\hypertarget{ux432ux43bux438ux44fux43dux438ux435-ux432ux44bux441ux43eux442ux44b-ux43dux430-ux441ux43aux43eux440ux43eux441ux442ux44c-ux432ux435ux442ux440ux430}{%
\subsubsection{Влияние Высоты на Скорость
Ветра}\label{ux432ux43bux438ux44fux43dux438ux435-ux432ux44bux441ux43eux442ux44b-ux43dux430-ux441ux43aux43eux440ux43eux441ux442ux44c-ux432ux435ux442ux440ux430}}

Скорость ветра, определяемая как горизонтальная составляющая движения
воздуха относительно земной поверхности, характеризуется направлением и
модулем. Она измеряется в метрах в секунду (м/с), километрах в час
(км/ч) или узлах.

\hypertarget{ux432-ux43fux43eux433ux440ux430ux43dux438ux447ux43dux43eux43c-ux441ux43bux43eux435-ux430ux442ux43cux43eux441ux444ux435ux440ux44b}{%
\paragraph{В Пограничном Слое
Атмосферы}\label{ux432-ux43fux43eux433ux440ux430ux43dux438ux447ux43dux43eux43c-ux441ux43bux43eux435-ux430ux442ux43cux43eux441ux444ux435ux440ux44b}}

Пограничный слой атмосферы -- это нижний слой, простирающийся от земной
поверхности до высоты около 1--2 км, где турбулентное трение о
подстилающую поверхность играет доминирующую роль.

\begin{enumerate}
\def\labelenumi{\arabic{enumi}.}
\tightlist
\item
  \textbf{Приземный подслой (до 30-50 метров):}

  \begin{itemize}
  \tightlist
  \item
    В этом слое сила трения оказывает наиболее значительное влияние на
    скорость ветра.
  \item
    Скорость ветра у подстилающей поверхности практически равна нулю и
    резко возрастает с высотой. Это возрастание описывается
    логарифмическим или степенным законом.
  \item
    Направление ветра в этом подслое практически не меняется с высотой.
  \item
    Влияние отклоняющей силы вращения Земли (силы Кориолиса) в приземном
    подслое очень мало по сравнению с силой турбулентной вязкости.
  \item
    Угол отклонения ветра от изобары у поверхности Земли является
    наибольшим: около 30° над сушей и примерно 15° над морем, где трение
    меньше.
  \item
    Коэффициент турбулентного обмена возрастает с высотой в этом слое.
  \end{itemize}
\item
  \textbf{Слой Экмана (верхняя часть пограничного слоя, до 1-2 км):}

  \begin{itemize}
  \tightlist
  \item
    Выше приземного подслоя, в так называемом слое Экмана, скорость
    ветра продолжает возрастать с высотой.
  \item
    При этом, в Северном полушарии, ветер постепенно поворачивает вправо
    (по часовой стрелке), приближаясь к направлению градиентного или
    геострофического ветра. Это явление известно как \textbf{спираль
    Экмана}.
  \item
    Высота, с которой ветер приблизительно можно считать
    геострофическим, составляет около 1 км. При ослабленном турбулентном
    обмене влияние приземного трения распространяется до меньшей высоты
    (0.3-0.4 км), а при сильном турбулентном обмене -- до большей высоты
    (1.5-2.0 км).
  \item
    Коэффициент турбулентного обмена в этой части пограничного слоя
    меняется незначительно с высотой.
  \item
    Влияние термической стратификации существенно:

    \begin{itemize}
    \tightlist
    \item
      При устойчивой стратификации (например, при ночных инверсиях
      температуры) турбулентный теплообмен ослаблен, что приводит к
      значительному уменьшению скорости ветра у поверхности земли, часто
      до полного штиля. Ветер может быть в 2-3 раза слабее
      геострофического.
    \item
      Наоборот, днем при сильном прогреве подстилающей поверхности и
      сверхадиабатических градиентах температуры скорость ветра может
      превосходить скорость геострофического ветра в 2-3 раза.
    \end{itemize}
  \end{itemize}
\end{enumerate}

\hypertarget{ux432-ux441ux432ux43eux431ux43eux434ux43dux43eux439-ux430ux442ux43cux43eux441ux444ux435ux440ux435}{%
\paragraph{В Свободной
Атмосфере}\label{ux432-ux441ux432ux43eux431ux43eux434ux43dux43eux439-ux430ux442ux43cux43eux441ux444ux435ux440ux435}}

Выше пограничного слоя (обычно выше 1-2 км) атмосфера считается
``свободной'', так как влияние турбулентного трения о земную поверхность
становится пренебрежимо малым. Здесь ветер в среднем близок к
\textbf{геострофическому}.

\begin{enumerate}
\def\labelenumi{\arabic{enumi}.}
\tightlist
\item
  \textbf{Геострофический ветер:}

  \begin{itemize}
  \tightlist
  \item
    Его направление близко к направлению изогипс на картах абсолютной
    топографии (АТ), а скорость прямо пропорциональна густоте изогипс
    (градиенту геопотенциала).
  \item
    На картах АТ, изогипсы проводятся параллельно вектору ветра, так как
    ветер выше слоя приземного трения обычно направлен по касательной к
    изогипсе.
  \item
    Геострофический ветер -- это идеализированная концепция, где сила
    барического градиента уравновешивается силой Кориолиса. Реальный
    ветер в свободной атмосфере очень близок к геострофическому по
    скорости и направлению.
  \end{itemize}
\item
  \textbf{Термический ветер:}

  \begin{itemize}
  \tightlist
  \item
    Изменения ветра с высотой в свободной атмосфере описываются
    концепцией \textbf{термического ветра}.
  \item
    Термический ветер -- это вектор, представляющий поправку к
    геострофическому ветру на одном уровне, переводящую его в ветер на
    другом уровне.
  \item
    Он напрямую связан с горизонтальными градиентами температуры. В
    верхних слоях атмосферы сила барического градиента все более заметно
    зависит от термической неоднородности и направлена перпендикулярно
    изотермам от более теплого воздуха в сторону более холодного.
  \item
    В Северном полушарии:

    \begin{itemize}
    \tightlist
    \item
      \textbf{Правый поворот ветра с высотой} (по часовой стрелке)
      соответствует \textbf{адвекции тепла}.
    \item
      \textbf{Левый поворот ветра с высотой} (против часовой стрелки)
      соответствует \textbf{адвекции холода}.
    \end{itemize}
  \item
    Термический ветер позволяет определить, какой воздух (теплый или
    холодный) поступает в пункт наблюдения.
  \end{itemize}
\item
  \textbf{Струйные течения:}

  \begin{itemize}
  \tightlist
  \item
    Это узкие полосы быстро движущегося воздуха, опоясывающие земной
    шар, обычно расположенные под тропопаузой или вблизи нее.
  \item
    Скорости в струйных течениях значительно выше, чем в окружающем
    воздушном потоке, и могут достигать 50-70 м/с, а в зимний период
    даже до 600 км/ч (над Тихим океаном).
  \item
    Скорости струйных течений зимой больше, чем летом, из-за более резко
    выраженного различия температур между экваториальными и полярными
    широтами.
  \item
    Они характеризуются значительным вертикальным сдвигом ветра
    (например, средний вертикальный градиент скорости может составлять
    8.8 м/с на 1 км).
  \item
    Уровень максимального ветра в струйном течении совпадает с высотой,
    на которой угол между барическим и термическим градиентами равен
    90°.
  \item
    Большой сдвиг ветра в струйных течениях способствует развитию
    сильной турбулентности, что проявляется как ``болтанка'' самолетов.
  \end{itemize}
\end{enumerate}

\hypertarget{ux434ux440ux443ux433ux438ux435-ux444ux430ux43aux442ux43eux440ux44b-ux432ux43bux438ux44fux44eux449ux438ux435-ux43dux430-ux438ux437ux43cux435ux43dux435ux43dux438ux435-ux432ux435ux442ux440ux430-ux441-ux432ux44bux441ux43eux442ux43eux439}{%
\paragraph{Другие Факторы, Влияющие на Изменение Ветра с
Высотой}\label{ux434ux440ux443ux433ux438ux435-ux444ux430ux43aux442ux43eux440ux44b-ux432ux43bux438ux44fux44eux449ux438ux435-ux43dux430-ux438ux437ux43cux435ux43dux435ux43dux438ux435-ux432ux435ux442ux440ux430-ux441-ux432ux44bux441ux43eux442ux43eux439}}

\begin{itemize}
\tightlist
\item
  \textbf{Орография:} Перепады высот и конфигурация изогипс рельефа
  значительно влияют на ветер, особенно в горных районах. Горы могут
  изменять направление перемещения воздушных масс и создавать зоны
  повышенных скоростей ветра (например, над вершинами гор или в узких
  перевалах). Они также вызывают локальные циркуляции, такие как
  горно-долинные ветры и бризы, которые имеют свои вертикальные профили.
\item
  \textbf{Атмосферные фронты:} В узкой зоне фронта метеорологические
  величины, включая направление ветра, изменяются значительно сильнее,
  чем внутри воздушной массы. Пересечение фронта часто сопровождается
  резким изменением направления ветра, например, правым поворотом при
  прохождении теплого фронта. Сходимость ветров к линии фронта в
  приземном слое стимулирует восходящие движения воздуха.
\item
  \textbf{Турбулентность:} Интенсивность турбулентного перемешивания в
  пограничном слое сильно зависит от стратификации воздушной массы. При
  сильном турбулентном обмене ветер приближается к градиентному и даже
  может превзойти его при малых барических градиентах.
\end{itemize}

Таким образом, вертикальный профиль ветра представляет собой результат
сложного взаимодействия силы трения, силы Кориолиса, силы барического
градиента и термических факторов. В нижних слоях доминирует трение, что
приводит к быстрому увеличению скорости и повороту ветра с высотой. В
свободной атмосфере, где трение незначительно, изменения скорости и
направления ветра с высотой в основном определяются горизонтальными
градиентами температуры (термическим ветром), что проявляется, в
частности, в существовании струйных течений.

ewpage

Уважаемый коллега,

Анализируя информацию о суточном ходе ветра, мы, как опытные
метеорологи, понимаем, что этот феномен глубоко укоренен в термодинамике
и динамике атмосферного пограничного слоя, определяясь суточным циклом
радиационного баланса и его влиянием на устойчивость и турбулентность
атмосферы.

\hypertarget{ux43eux431ux449ux438ux435-ux43fux440ux438ux43dux446ux438ux43fux44b-ux441ux443ux442ux43eux447ux43dux43eux433ux43e-ux445ux43eux434ux430-ux432ux435ux442ux440ux430}{%
\subsubsection{Общие Принципы Суточного Хода
Ветра}\label{ux43eux431ux449ux438ux435-ux43fux440ux438ux43dux446ux438ux43fux44b-ux441ux443ux442ux43eux447ux43dux43eux433ux43e-ux445ux43eux434ux430-ux432ux435ux442ux440ux430}}

Суточный ход ветра, особенно выраженный в приземном слое, является
прямым следствием меняющихся условий теплообмена между подстилающей
поверхностью и атмосферой.

\begin{enumerate}
\def\labelenumi{\arabic{enumi}.}
\tightlist
\item
  \textbf{Дневные Часы:}

  \begin{itemize}
  \tightlist
  \item
    В течение дня солнечная радиация интенсивно прогревает земную
    поверхность, что приводит к развитию конвекции и термической
    неустойчивости в приземном слое.
  \item
    Эта неустойчивость способствует активному турбулентному
    перемешиванию, которое эффективно передает импульс от вышележащих
    слоев (где ветер сильнее и ближе к геострофическому) к земной
    поверхности.
  \item
    В результате, скорость ветра у поверхности Земли обычно
    увеличивается днем, достигая максимума в послеполуденные часы.
  \item
    Высота слоя трения (пограничного слоя), где ощущается влияние
    поверхности, возрастает и может достигать 1.5--2.0 км при сильном
    турбулентном обмене. В условиях малых барических градиентов скорость
    ветра в приземном слое днем может в 2--3 раза превышать
    геострофическую скорость.
  \end{itemize}
\item
  \textbf{Ночные Часы:}

  \begin{itemize}
  \tightlist
  \item
    С заходом солнца начинается радиационное выхолаживание подстилающей
    поверхности, что приводит к формированию приземных инверсий
    температуры (слоев, где температура возрастает с высотой).
  \item
    Инверсии подавляют вертикальное турбулентное перемешивание,
    ``отцепляя'' приземный слой от вышележащих слоев атмосферы.
  \item
    Как следствие, скорость ветра у поверхности Земли резко ослабевает,
    часто приводя к штилю или очень слабым ветрам, особенно при малых
    барических градиентах.
  \item
    Высота слоя трения при ослабленном турбулентном обмене уменьшается
    до 0.3--0.4 км.
  \item
    Угол отклонения ветра от изобары у поверхности Земли ночью
    увеличивается по сравнению с дневным временем, составляя около 30°
    над сушей и 15° над морем.
  \end{itemize}
\end{enumerate}

\hypertarget{ux43cux435ux441ux442ux43dux44bux435-ux432ux435ux442ux440ux44b-ux438-ux438ux445-ux441ux443ux442ux43eux447ux43dux44bux439-ux445ux43eux434}{%
\subsubsection{Местные Ветры и Их Суточный
Ход}\label{ux43cux435ux441ux442ux43dux44bux435-ux432ux435ux442ux440ux44b-ux438-ux438ux445-ux441ux443ux442ux43eux447ux43dux44bux439-ux445ux43eux434}}

Наряду с общими изменениями в пограничном слое, существуют специфические
местные циркуляции, которые проявляют ярко выраженный суточный ход:

\begin{enumerate}
\def\labelenumi{\arabic{enumi}.}
\tightlist
\item
  \textbf{Бризы (Land-Sea Breezes):}

  \begin{itemize}
  \tightlist
  \item
    Эти ветры возникают на границах суши и водоемов в ясную или
    малооблачную погоду.
  \item
    \textbf{Днем:} Суша нагревается быстрее воды, что приводит к
    понижению давления над сушей. Воздух перемещается от более холодного
    водоема к более теплой суше у поверхности, формируя морской бриз. На
    небольшой высоте формируется компенсационный поток в обратном
    направлении.
  \item
    \textbf{Ночью:} Суша охлаждается быстрее воды, и направление ветра
    меняется на противоположное: от более холодной суши к более теплому
    водоему у поверхности (сухопутный бриз).
  \end{itemize}
\item
  \textbf{Горно-Долинные Ветры (Mountain-Valley Winds):}

  \begin{itemize}
  \tightlist
  \item
    Подобны бризам, но формируются между горными хребтами и долинами.
  \item
    \textbf{Ночью:} Склоны охлаждаются, и более холодный, плотный воздух
    стекает вниз в долину (горный ветер). Это может приводить к
    образованию ``озер холода'' в низинах.
  \item
    \textbf{Днем:} Склоны, особенно подверженные солнечному нагреву,
    прогреваются быстрее долин. Возникают восходящие потоки теплого
    воздуха вдоль склонов и долинные ветры, дующие вверх по долине,
    достигая максимальных скоростей к 12--14 часам местного времени.
  \item
    \textbf{Переходные Периоды:} В моменты восхода и захода солнца,
    когда градиенты температур меняют знак, наблюдается затишье.
  \item
    \textbf{Сезонные Отличия:} Зимой в горах чаще преобладают горные
    ветры из-за длительного отрицательного радиационного баланса.
  \end{itemize}
\item
  \textbf{Мезоструи (Low-Level Jets):}

  \begin{itemize}
  \tightlist
  \item
    Это тонкие слои сильного ветра на высоте нескольких сотен метров над
    землей.
  \item
    Они \textbf{особенно хорошо выражены ночью}, что связано с
    формированием приземной инверсии, которая снижает влияние трения и
    позволяет ветру на этих высотах усиливаться.
  \end{itemize}
\end{enumerate}

\hypertarget{ux432ux43bux438ux44fux43dux438ux435-ux441ux443ux442ux43eux447ux43dux43eux433ux43e-ux445ux43eux434ux430-ux432ux435ux442ux440ux430-ux43dux430-ux43fux43eux433ux43eux434ux43dux44bux435-ux44fux432ux43bux435ux43dux438ux44f}{%
\subsubsection{Влияние Суточного Хода Ветра на Погодные
Явления}\label{ux432ux43bux438ux44fux43dux438ux435-ux441ux443ux442ux43eux447ux43dux43eux433ux43e-ux445ux43eux434ux430-ux432ux435ux442ux440ux430-ux43dux430-ux43fux43eux433ux43eux434ux43dux44bux435-ux44fux432ux43bux435ux43dux438ux44f}}

Суточный ход ветра и связанные с ним изменения в пограничном слое
оказывают значительное влияние на развитие различных погодных явлений:

\begin{itemize}
\tightlist
\item
  \textbf{Конвективные Явления (Шквалы, Смерчи, Пыльные Бури):}

  \begin{itemize}
  \tightlist
  \item
    Возникновение мощных конвективных процессов, таких как шквалы,
    смерчи и пыльные бури, часто имеет суточную привязку, достигая
    максимума в послеполуденные и вечерние часы из-за максимального
    развития неустойчивости и вертикальных движений, вызванных дневным
    прогревом.
  \item
    Турбулентность, вызывающая ``болтанку'' самолетов, также наиболее
    интенсивна в часы, благоприятные для развития конвекции.
  \end{itemize}
\item
  \textbf{Туманы:}

  \begin{itemize}
  \tightlist
  \item
    Радиационные туманы обычно образуются ночью и в ранние утренние
    часы, когда поверхность Земли выхолаживается, а слабый ветер
    способствует накоплению влаги в приземном слое. Сильный ветер
    препятствует образованию туманов, перемешивая воздух. Минимальная
    видимость в тумане часто наблюдается около восхода солнца.
  \end{itemize}
\item
  \textbf{Воздушные Массы:}

  \begin{itemize}
  \tightlist
  \item
    В холодных неустойчивых воздушных массах суточный ход
    метеорологических величин, включая ветер, особенно велик. Днем может
    наблюдаться усиление ветра и увеличение облачности.
  \item
    Трансформация воздушных масс над сушей также подвержена суточным
    изменениям температуры в приземном слое, что косвенно влияет на
    ветровой режим.
  \end{itemize}
\end{itemize}

Таким образом, суточный ход ветра --- это не просто колебания скорости и
направления, а сложное взаимодействие турбулентных процессов, барических
градиентов и термически обусловленных циркуляций, которые регулируются
ежедневным циклом солнечной радиации. Понимание этих механизмов
критически важно для точного метеорологического диагноза и прогноза.

ewpage

Как известно, для адекватного описания атмосферных процессов в
динамической метеорологии мы используем систему фундаментальных
гидротермодинамических уравнений, выражающих основные законы физики:
закон сохранения импульса (движения), закон сохранения массы и закон
сохранения энергии (уравнение притока тепла). В локальных декартовых
координатах эти уравнения принимают следующий вид.

\hypertarget{ux443ux440ux430ux432ux43dux435ux43dux438ux44f-ux434ux432ux438ux436ux435ux43dux438ux44f}{%
\subsubsection{Уравнения
Движения}\label{ux443ux440ux430ux432ux43dux435ux43dux438ux44f-ux434ux432ux438ux436ux435ux43dux438ux44f}}

Уравнения движения атмосферы являются математическим выражением второго
закона Ньютона, применительно к единице массы воздуха, на которую
действуют различные силы. В декартовой системе координат (\(x, y, z\))
проекции векторного уравнения движения на оси могут быть записаны в
напряжениях.

Общее векторное уравнение движения для единицы массы воздуха,
учитывающее силу тяжести (\(\vec{g}\)), силу Кориолиса (\(\vec{K}\)), и
силы вязкости (\(\vec{N}\)), а также градиент давления, имеет вид:

\$\frac{d\vec{V}}{dt} = -\frac{1}{\rho}\nabla P + \vec{g} + \vec{K} +
\vec{N} \$

где \(\frac{d}{dt}\) - индивидуальная производная по времени, \(\rho\) -
плотность воздуха, \(P\) - давление.

Проектируя это уравнение на оси координат \(x, y, z\) (где \(u, v, w\)
--- проекции скорости \(\vec{V}\) на эти оси), и раскрывая
индивидуальные производные, получаем три скалярных уравнения движения
атмосферы в напряжениях:

\$\frac{\partial u}{\partial t} + u \frac{\partial u}{\partial x} + v
\frac{\partial u}{\partial y} + w \frac{\partial u}{\partial z} =
-\frac{1}{\rho}\frac{\partial P}{\partial x} + 2\omega\_z v - 2\omega\_y
w + \frac{1}{\rho} \left( \frac{\partial \sigma_{xx}}{\partial x} +
\frac{\partial \sigma_{yx}}{\partial y} +
\frac{\partial \sigma_{zx}}{\partial z} \right) \$
\$\frac{\partial v}{\partial t} + u \frac{\partial v}{\partial x} + v
\frac{\partial v}{\partial y} + w \frac{\partial v}{\partial z} =
-\frac{1}{\rho}\frac{\partial P}{\partial y} - 2\omega\_z u + 2\omega\_x
w + \frac{1}{\rho} \left( \frac{\partial \sigma_{xy}}{\partial x} +
\frac{\partial \sigma_{yy}}{\partial y} +
\frac{\partial \sigma_{zy}}{\partial z} \right) \$
\$\frac{\partial w}{\partial t} + u \frac{\partial w}{\partial x} + v
\frac{\partial w}{\partial y} + w \frac{\partial w}{\partial z} =
-\frac{1}{\rho}\frac{\partial P}{\partial z} - g - 2\omega\_x v +
2\omega\_y u + \frac{1}{\rho} \left(
\frac{\partial \sigma_{xz}}{\partial x} +
\frac{\partial \sigma_{yz}}{\partial y} +
\frac{\partial \sigma_{zz}}{\partial z} \right) \$

Здесь:

\begin{itemize}
\tightlist
\item
  \(\frac{d}{dt} = \frac{\partial}{\partial t} + u \frac{\partial}{\partial x} + v \frac{\partial}{\partial y} + w \frac{\partial}{\partial z}\)
  --- индивидуальная производная, характеризующая изменение величины во
  времени внутри движущейся массы.
\item
  \(-\frac{1}{\rho}\nabla P\) (или его проекции
  \(-\frac{1}{\rho}\frac{\partial P}{\partial x}\), etc.) --- сила
  барического градиента, направленная в сторону наискорейшего падения
  давления.
\item
  \(\vec{g}\) (или \(-g\) в проекции на \(z\)) --- сила тяжести,
  направленная вертикально вниз.
\item
  \(\vec{K}\) (или \(2\omega_z v - 2\omega_y w\), etc.) --- отклоняющая
  сила вращения Земли (сила Кориолиса), которая действует на движущиеся
  относительно Земли частицы воздуха. Она пропорциональна удвоенному
  векторному произведению угловой скорости вращения Земли на вектор
  скорости относительного движения. Проекции угловой скорости вращения
  Земли на оси \(x, y, z\) соответственно обозначаются
  \(\omega_x, \omega_y, \omega_z\).
\item
  \(\frac{1}{\rho} \left( \frac{\partial \sigma_{ij}}{\partial x_j} \right)\)
  --- проекции сил вязкости. \(\sigma_{ij}\) --- компоненты тензора
  вязких напряжений, которые характеризуют силу взаимодействия между
  частицами воздуха, возникающую из-за деформации объема воздуха.
\end{itemize}

Ввиду повсеместной турбулентности воздуха, в динамической метеорологии
приходится использовать осредненные уравнения движения. При осреднении в
уравнениях появляются новые неизвестные величины --- турбулентные
напряжения, которые выражают перенос количества движения пульсационными
скоростями. Эти члены описывают внутреннее турбулентное трение, которое
стремится выровнять среднюю скорость основного движения в пространстве.
В упрощенной форме, пренебрегая пульсациями плотности и молекулярной
вязкостью, система уравнений может быть представлена так:

\$\frac{\partial u}{\partial t} + u \frac{\partial u}{\partial x} + v
\frac{\partial u}{\partial y} + w \frac{\partial u}{\partial z} =
-\frac{1}{\rho}\frac{\partial P}{\partial x} + 2\omega\_z v - 2\omega\_y
w + \frac{1}{\rho} \left(
\frac{\partial k_x \frac{\partial u}{\partial x}}{\partial x} +
\frac{\partial k_y \frac{\partial u}{\partial y}}{\partial y} +
\frac{\partial k_z \frac{\partial u}{\partial z}}{\partial z} \right) \$
\$\frac{\partial v}{\partial t} + u \frac{\partial v}{\partial x} + v
\frac{\partial v}{\partial y} + w \frac{\partial v}{\partial z} =
-\frac{1}{\rho}\frac{\partial P}{\partial y} - 2\omega\_z u + 2\omega\_x
w + \frac{1}{\rho} \left(
\frac{\partial k_x \frac{\partial v}{\partial x}}{\partial x} +
\frac{\partial k_y \frac{\partial v}{\partial y}}{\partial y} +
\frac{\partial k_z \frac{\partial v}{\partial z}}{\partial z} \right) \$
\$\frac{\partial w}{\partial t} + u \frac{\partial w}{\partial x} + v
\frac{\partial w}{\partial y} + w \frac{\partial w}{\partial z} =
-\frac{1}{\rho}\frac{\partial P}{\partial z} - g - 2\omega\_x v +
2\omega\_y u + \frac{1}{\rho} \left(
\frac{\partial k_x \frac{\partial w}{\partial x}}{\partial x} +
\frac{\partial k_y \frac{\partial w}{\partial y}}{\partial y} +
\frac{\partial k_z \frac{\partial w}{\partial z}}{\partial z} \right) \$

Здесь \(k_x, k_y, k_z\) --- коэффициенты турбулентного обмена в
соответствующих направлениях, которые параметризуют турбулентные
напряжения.

\hypertarget{ux443ux440ux430ux432ux43dux435ux43dux438ux435-ux441ux43eux445ux440ux430ux43dux435ux43dux438ux44f-ux43cux430ux441ux441ux44b-ux443ux440ux430ux432ux43dux435ux43dux438ux435-ux43dux435ux440ux430ux437ux440ux44bux432ux43dux43eux441ux442ux438}{%
\subsubsection{Уравнение Сохранения Массы (Уравнение
Неразрывности)}\label{ux443ux440ux430ux432ux43dux435ux43dux438ux435-ux441ux43eux445ux440ux430ux43dux435ux43dux438ux44f-ux43cux430ux441ux441ux44b-ux443ux440ux430ux432ux43dux435ux43dux438ux435-ux43dux435ux440ux430ux437ux440ux44bux432ux43dux43eux441ux442ux438}}

Уравнение неразрывности выражает закон сохранения массы и связывает
изменение плотности воздуха во времени с распределением скорости
движения в пространстве. Для сжимаемой среды, такой как воздух, оно
может быть представлено в декартовых координатах следующим образом:

\$\frac{\partial \rho}{\partial t} +
\frac{\partial (\rho u)}{\partial x} +
\frac{\partial (\rho v)}{\partial y} +
\frac{\partial (\rho w)}{\partial z} = 0 \$

Или в более компактной форме, используя оператор дивергенции:

\$\frac{\partial \rho}{\partial t} + \nabla \cdot (\rho \vec{V}) = 0 \$

Где
\(\nabla \cdot \vec{V} = \frac{\partial u}{\partial x} + \frac{\partial v}{\partial y} + \frac{\partial w}{\partial z}\)
--- дивергенция вектора скорости.

Это уравнение также может быть представлено через индивидуальную
производную плотности:

\$\frac{d \rho}{d t} + \rho (\nabla \cdot \vec{V}) = 0 \$

Для случая несжимаемой атмосферы (где \(\rho = \text{const}\)),
уравнение неразрывности значительно упрощается, принимая вид:

\$\nabla \cdot \vec{V} = 0 \$ или \$\frac{\partial u}{\partial x} +
\frac{\partial v}{\partial y} + \frac{\partial w}{\partial z} = 0 \$

Важно отметить, что атмосферный воздух является сжимаемой средой,
поэтому наиболее общие формы уравнения неразрывности учитывают изменение
плотности.

\hypertarget{ux443ux440ux430ux432ux43dux435ux43dux438ux435-ux43fux440ux438ux442ux43eux43aux430-ux442ux435ux43fux43bux430-ux43fux435ux440ux432ux43eux435-ux43dux430ux447ux430ux43bux43e-ux442ux435ux440ux43cux43eux434ux438ux43dux430ux43cux438ux43aux438-1}{%
\subsubsection{Уравнение Притока Тепла (Первое Начало
Термодинамики)}\label{ux443ux440ux430ux432ux43dux435ux43dux438ux435-ux43fux440ux438ux442ux43eux43aux430-ux442ux435ux43fux43bux430-ux43fux435ux440ux432ux43eux435-ux43dux430ux447ux430ux43bux43e-ux442ux435ux440ux43cux43eux434ux438ux43dux430ux43cux438ux43aux438-1}}

Уравнение притока тепла является выражением закона сохранения энергии,
применительно к каждой частице атмосферы. Оно связывает изменение
внутренней энергии воздуха с подведенным теплом и работой, совершаемой
или над частицей.

В общем виде, для единицы массы воздуха, уравнение притока тепла может
быть выведено из первого начала термодинамики и представлено как:

\$\rho \frac{dJ}{dt} = \rho Q + P (\nabla \cdot \vec{V}) - \frac{3}{2}
\rho \gamma (\frac{\partial u}{\partial x})\^{}2 - \dots \$ (прочие
члены, связанные с деформацией и вязкостью)

Где \(J\) --- внутренняя энергия, отнесенная к единице массы, \(Q\) ---
количество тепла, сообщаемое извне единице массы в единицу времени.

Более распространенная в метеорологии форма, не содержащая неизмеряемой
величины объема, получается путем преобразований с использованием
уравнения состояния идеального газа и соотношения Майера
(\(C_p = C_v + R\)):

\$ C\_p \frac{dT}{dt} - \frac{RT}{P} \frac{dP}{dt} = Q \$

Здесь:

\begin{itemize}
\tightlist
\item
  \(T\) --- температура воздуха.
\item
  \(C_p\) --- удельная теплоемкость при постоянном давлении (для сухого
  воздуха \(C_p = 1005 \text{ Дж}/(\text{кг} \cdot \text{К})\)).
\item
  \(R\) --- удельная газовая постоянная воздуха.
\item
  \(Q\) --- суммарный приток тепла к единице массы воздуха за единицу
  времени. Этот член \(Q\) включает различные факторы:

  \begin{itemize}
  \tightlist
  \item
    \textbf{Радиационный приток тепла:} поглощение солнечной и земной
    радиации.
  \item
    \textbf{Приток/отток тепла при фазовых переходах:} конденсация или
    испарение водяного пара (скрытая теплота).
  \item
    \textbf{Приток тепла за счет молекулярной теплопроводности:} обычно
    пренебрегается в крупномасштабных процессах из-за малости
    коэффициента теплопроводности.
  \item
    \textbf{Турбулентный приток тепла:} перенос тепла за счет случайных
    колебаний ветра.
  \end{itemize}
\item
  \(\frac{dT}{dt}\) и \(\frac{dP}{dt}\) --- индивидуальные производные
  температуры и давления, соответственно.
\end{itemize}

При адиабатических процессах, когда приток тепла \(Q\) равен нулю,
уравнение принимает вид:

\$ C\_p \frac{dT}{dt} - \frac{RT}{P} \frac{dP}{dt} = 0 \$

В турбулентной атмосфере, как и в уравнениях движения, при осреднении
уравнения притока тепла также появляются новые члены, представляющие
турбулентные потоки тепла, которые определяются через градиент средней
потенциальной температуры и коэффициенты турбулентного обмена.

Эти уравнения являются краеугольным камнем для численного моделирования
атмосферных процессов, несмотря на их сложность и необходимость
параметризации турбулентности.

ewpage

Как специалисты в области динамической метеорологии, мы постоянно
сталкиваемся с необходимостью упрощения сложных систем уравнений для
описания атмосферных процессов. Одним из мощных инструментов в этом
направлении является теория подобия, позволяющая переносить результаты
исследований с одной системы на другую, физически подобную ей.

\hypertarget{ux442ux435ux43eux440ux438ux44f-ux43fux43eux434ux43eux431ux438ux44f-ux438-ux431ux435ux437ux440ux430ux437ux43cux435ux440ux43dux44bux435-ux432ux435ux43bux438ux447ux438ux43dux44b}{%
\subsection{Теория Подобия и Безразмерные
Величины}\label{ux442ux435ux43eux440ux438ux44f-ux43fux43eux434ux43eux431ux438ux44f-ux438-ux431ux435ux437ux440ux430ux437ux43cux435ux440ux43dux44bux435-ux432ux435ux43bux438ux447ux438ux43dux44b}}

Теория подобия является одним из методов упрощения уравнений движения
атмосферы. Согласно этой теории, вместо изучения одного движения, можно
рассматривать другое, которое пространственно, и при этом, возможно,
временно, отличается от первого, но является ему подобным.

Для применения теории подобия все физические величины выражаются не в
абсолютных единицах измерения, а в виде отношений к их характерным
значениям или масштабам. Такие масштабы, как характерная длина (\(L\)),
время (\(T\)), скорость (\(V\)), плотность (\(\rho\)), и разность
давления (\(\Delta P\)), позволяют сделать все величины безразмерными.
Например, безразмерные координаты будут \(x/L\), \(y/L\), \(z/L\), а
безразмерное время \(t/T\). Любая функция безразмерных координат и
времени, составленная для подобных движений, должна быть совершенно
одинаковой. Следовательно, дифференциальные уравнения и краевые условия,
которым удовлетворяют функции от безразмерных величин, для подобных
движений также должны совпадать между собой.

\hypertarget{ux43aux440ux438ux442ux435ux440ux438ux438-ux43fux43eux434ux43eux431ux438ux44f}{%
\subsection{Критерии
Подобия}\label{ux43aux440ux438ux442ux435ux440ux438ux438-ux43fux43eux434ux43eux431ux438ux44f}}

В соответствии с безразмерными коэффициентами, появляющимися в
преобразованных уравнениях движения атмосферы, вводятся специальные
критерии подобия. Для того чтобы два движения были подобны друг другу,
необходимо, чтобы они имели одинаковые значения этих критериев.

Основные критерии подобия, применяемые в динамической метеорологии,
включают:

\begin{itemize}
\tightlist
\item
  \textbf{Число Гомохронности (Ho)}: Определяется как
  \(L / (V \cdot T)\). Оно характеризует отношение характерного времени
  движения к характерному времени процесса.
\item
  \textbf{Число Фруда (Fr)}: Определяется как \(V^2 / (g \cdot L)\). Это
  безразмерное число выражает отношение сил инерции к силам гравитации,
  что критически важно для процессов, где существенна роль плавучести
  или стратификации.
\item
  \textbf{Число Эйлера (Eu)}: Определяется как
  \(\Delta P / (\rho \cdot V^2)\). Оно представляет собой отношение сил
  давления к силам инерции.
\item
  \textbf{Число Рейнольдса (Re)}: Определяется как
  \(\rho \cdot V \cdot L / \nu\), где \(\nu\) -- кинематическая
  вязкость. Это один из наиболее известных критериев, характеризующий
  отношение сил инерции к силам вязкости. Большие числа Рейнольдса
  указывают на преобладание турбулентного режима движения, в то время
  как малые -- на ламинарный. Источники также отмечают, что чем больше
  масштаб протяженности движения и чем больше скорость, тем меньшую роль
  играет вязкость.
\item
  \textbf{Безразмерная Характеристика Отклонения Ветра от
  Геострофического (De)}: Определяется как \(V / (\omega \cdot L)\), где
  \(\omega\) -- угловая скорость вращения. Этот критерий отражает
  влияние силы Кориолиса на движение.
\end{itemize}

\hypertarget{ux43aux43bux430ux441ux441ux438ux444ux438ux43aux430ux446ux438ux44f-ux43aux440ux438ux442ux435ux440ux438ux435ux432}{%
\subsubsection{Классификация
Критериев}\label{ux43aux43bux430ux441ux441ux438ux444ux438ux43aux430ux446ux438ux44f-ux43aux440ux438ux442ux435ux440ux438ux435ux432}}

Критерии подобия делятся на две группы:

\begin{enumerate}
\def\labelenumi{\arabic{enumi}.}
\tightlist
\item
  \textbf{Определяющие критерии}: Это критерии подобия, которые содержат
  внешне обусловленные величины и константы, характеризующие физические
  свойства жидкости. К внешне обусловленным величинам относятся те
  элементы движения, значения которых заданы извне или являются
  фундаментальными константами, например, геометрия системы, физические
  свойства среды, граничные условия.
\item
  \textbf{Неопределяющие критерии}: Эта группа не описана в
  предоставленных источниках, но по аналогии с определяющими критериями,
  это будут те, которые выводятся из уравнений, но не содержат внешних
  заданных величин.
\end{enumerate}

Анализ безразмерных уравнений движения с помощью этих критериев
позволяет выявлять главные факторы, определяющие свойства изучаемого
движения, и пренебрегать теми, которые в данном случае не оказывают
существенного влияния. Это значительно упрощает уравнения динамики и
позволяет получать решения, описывающие реальные закономерности
конкретного вида движения.

ewpage

Уважаемые коллеги, обсуждение систем упрощенных уравнений и видов
стационарных течений, таких как геострофический поток, является ключевым
для нашего понимания атмосферной динамики. Как мы знаем, сложность общих
уравнений гидротермодинамики требует применения различных упрощений для
их практического решения и анализа атмосферных процессов.

\hypertarget{ux441ux438ux441ux442ux435ux43cux44b-ux443ux43fux440ux43eux449ux435ux43dux43dux44bux445-ux443ux440ux430ux432ux43dux435ux43dux438ux439}{%
\subsubsection{Системы Упрощенных
Уравнений}\label{ux441ux438ux441ux442ux435ux43cux44b-ux443ux43fux440ux43eux449ux435ux43dux43dux44bux445-ux443ux440ux430ux432ux43dux435ux43dux438ux439}}

Упрощение уравнений динамики атмосферы является неотъемлемой частью
динамической метеорологии, позволяющей выделить доминирующие факторы для
конкретных видов движения и пренебречь второстепенными влияниями. Это
обеспечивает возможность получения решений, описывающих реальные
закономерности. Методы упрощения основаны на оценке степени влияния
различных сил и факторов, действующих в атмосфере.

\begin{enumerate}
\def\labelenumi{\arabic{enumi}.}
\tightlist
\item
  \textbf{Квазистатическое приближение}:

  \begin{itemize}
  \tightlist
  \item
    Для крупномасштабных атмосферных движений уравнение движения,
    спроектированное на вертикальную ось, значительно упрощается,
    превращаясь в уравнение квазистатики:
    \(\rho gzP −=\partial\partial\).
  \item
    Это означает, что уравнение статики, хотя и полностью справедливо
    для неподвижной атмосферы, с высокой степенью точности выполняется и
    в движущейся атмосфере.
  \item
    Большинство современных схем прогноза крупномасштабных процессов
    основываются на квазистатическом приближении. При этом
    пренебрегаются ускорениями, обусловленными нарушением равновесия
    между силой тяжести и вертикальной составляющей силы барического
    градиента.
  \end{itemize}
\item
  \textbf{Адиабатическое приближение}:

  \begin{itemize}
  \tightlist
  \item
    В свободной атмосфере, особенно для сравнительно коротких периодов
    времени (порядка одних суток), можно пренебречь влиянием притока
    тепла, рассматривая движение воздуха как адиабатическое.
  \item
    Принятие условия адиабатичности процессов значительно упрощает
    решение прогностической задачи, не внося существенных ошибок в
    краткосрочный прогноз. Однако, наряду с адиабатическими,
    разрабатываются модели, учитывающие тепло фазовых переходов воды,
    турбулентный и радиационный теплообмены.
  \end{itemize}
\item
  \textbf{Пренебрежение турбулентностью и молекулярной вязкостью}:

  \begin{itemize}
  \tightlist
  \item
    В свободной атмосфере, выше пограничного слоя (обычно 1-2 км), в
    первом приближении можно пренебречь турбулентностью.
  \item
    Влиянием сил молекулярной вязкости воздуха на атмосферные движения
    всегда можно пренебречь, поскольку числа Рейнольдса для атмосферных
    движений очень велики.
  \end{itemize}
\item
  \textbf{Замыкание системы уравнений}:

  \begin{itemize}
  \tightlist
  \item
    При учете бароклинности атмосферы (когда плотность зависит не только
    от давления, но и от температуры), появляется новая неизвестная
    функция --- температура воздуха. Это приводит к необходимости
    добавления в систему уравнений гидротермодинамики уравнения притока
    тепла и связанных с ним уравнений переноса водяного пара и лучистой
    энергии.
  \item
    Для замыкания системы уравнений при численном прогнозе используются
    дополнительные предположения, такие как близость реального ветра к
    соленоидальному полю (уравнение баланса) или к геострофическому полю
    (геострофические соотношения). Это позволяет ``отфильтровать''
    метеорологические шумы --- мелкомасштабные возмущения, искажающие
    крупномасштабные процессы.
  \item
    \textbf{Баротропные и бароклинные модели}:

    \begin{itemize}
    \tightlist
    \item
      Баротропная жидкость --- это среда, в которой плотность зависит
      только от давления (\(\rho = \rho(P)\)). В такой среде
      изобарические и изотермические поверхности совпадают.
    \item
      Бароклинная жидкость --- это среда, в которой плотность является
      функцией не только давления, но и других параметров, например
      температуры (\(\rho = \rho(P, T)\)). В бароклинной атмосфере
      изопикнические, изобарические и изотермические поверхности
      пересекаются, образуя термодинамические соленоиды.
    \item
      Реальная атмосфера является бароклинной. Бароклинные модели
      атмосферы, учитывающие вертикальную структуру, дают более точное
      описание крупномасштабных процессов по сравнению с баротропными,
      которые представляют осредненное по высоте состояние атмосферы.
    \end{itemize}
  \end{itemize}
\item
  \textbf{Негеострофические модели}:

  \begin{itemize}
  \tightlist
  \item
    Наряду с квазигеострофическими моделями разрабатываются и
    используются различные негеострофические модели, в которых уравнения
    горизонтального движения сохраняются неизменными без привлечения
    геострофических соотношений.
  \end{itemize}
\end{enumerate}

\hypertarget{ux43dux435ux43aux43eux442ux43eux440ux44bux435-ux432ux438ux434ux44b-ux441ux442ux430ux446ux438ux43eux43dux430ux440ux43dux44bux445-ux442ux435ux447ux435ux43dux438ux439}{%
\subsubsection{Некоторые Виды Стационарных
Течений}\label{ux43dux435ux43aux43eux442ux43eux440ux44bux435-ux432ux438ux434ux44b-ux441ux442ux430ux446ux438ux43eux43dux430ux440ux43dux44bux445-ux442ux435ux447ux435ux43dux438ux439}}

Понимание стационарных течений, то есть течений, где скорость в каждой
точке поля не меняется со временем, является основой для анализа более
сложных нестационарных атмосферных движений.

\hypertarget{ux433ux435ux43eux441ux442ux440ux43eux444ux438ux447ux435ux441ux43aux438ux439-ux43fux43eux442ux43eux43a}{%
\paragraph{Геострофический
Поток}\label{ux433ux435ux43eux441ux442ux440ux43eux444ux438ux447ux435ux441ux43aux438ux439-ux43fux43eux442ux43eux43a}}

Геострофический поток является фундаментальным концептом в динамической
метеорологии, описывающим равновесие между двумя ключевыми силами в
свободной атмосфере.

\begin{enumerate}
\def\labelenumi{\arabic{enumi}.}
\tightlist
\item
  \textbf{Определение и баланс сил}:

  \begin{itemize}
  \tightlist
  \item
    Геострофический ветер (обозначаемый как \(V_g\)) возникает, когда
    сила барического градиента (сила горизонтального перепада давления,
    направленная от высокого к низкому давлению) уравновешивается силой
    Кориолиса.
  \item
    Математически это выражается как \(A = -G\), где \(A\) --- ускорение
    Кориолиса, \(G\) --- сила барического градиента. При этом
    пренебрегается силами трения, центробежной силой и локальными
    ускорениями.
  \item
    Формула для определения скорости геострофического ветра:
    \(V_g = \frac{1}{\rho l} \frac{\partial P}{\partial n}\), где
    \(l = 2\omega \sin \phi\) (параметр Кориолиса), \(\rho\) ---
    плотность воздуха, \(P\) --- давление, \(n\) --- нормаль к изобаре.
  \item
    В условиях квазистатичности (где \(w=0\)), геострофический ветер
    также может быть выражен через градиент высоты изобарической
    поверхности (изогипсы карт абсолютной топографии, АТ):
    \(V_g = \frac{g}{l} \frac{\partial H}{\partial n}\).
  \end{itemize}
\item
  \textbf{Направление относительно изобар/изогипс}:

  \begin{itemize}
  \tightlist
  \item
    Геострофический ветер направлен параллельно изобарам (или
    изогипсам), так что низкое давление (или меньшие высоты
    изобарической поверхности) остается слева в Северном полушарии, и
    справа в Южном полушарии.
  \end{itemize}
\item
  \textbf{Область применимости}:

  \begin{itemize}
  \tightlist
  \item
    Геострофический ветер --- это идеализированная математическая
    абстракция. Однако в свободной атмосфере (выше пограничного слоя,
    где влияние трения о подстилающую поверхность пренебрежимо мало),
    реальный ветер очень близок к геострофическому по скорости и
    направлению.
  \item
    Это свойство очень ценно для метеорологов, поскольку позволяет
    получить представление о направлении и скорости реального ветра даже
    в районах, где отсутствуют прямые наблюдения за ветром.
  \end{itemize}
\item
  \textbf{Ограничения и отклонения}:

  \begin{itemize}
  \tightlist
  \item
    \textbf{Влияние трения}: В пограничном слое атмосферы (приземный
    подслой до 30-50 м и слой Экмана до 1-2 км) сила трения существенно
    влияет на скорость и направление ветра. Скорость ветра у
    подстилающей поверхности уменьшается (например, \(V = 0.7 V_g\) над
    морем и \(V = 0.4 V_g\) над сушей в умеренных широтах). Направление
    ветра отклоняется от изобары в сторону низкого давления (в Северном
    полушарии).
  \item
    \textbf{Кривизна изобар/изогипс}: Строго говоря, геострофический
    ветер возможен только при прямолинейных изобарах или изогипсах. При
    наличии кривизны возникает центробежная сила, и тогда корректнее
    рассматривать градиентный ветер. Однако на практике, из-за сложности
    измерения радиуса кривизны, даже при криволинейных изобарах часто
    продолжают использовать геострофическое приближение, поскольку учет
    кривизны не всегда улучшает результаты.
  \end{itemize}
\end{enumerate}

\hypertarget{ux43fux43eux442ux43eux43aux438-ux43aux443ux44dux442ux442ux430-ux438-ux43fux443ux430ux437ux435ux439ux43bux44f}{%
\paragraph{Потоки Куэтта и
Пуазейля}\label{ux43fux43eux442ux43eux43aux438-ux43aux443ux44dux442ux442ux430-ux438-ux43fux443ux430ux437ux435ux439ux43bux44f}}

На основе анализа представленных источников, информация о потоках Куэтта
и Пуазейля не обнаружена. Вероятно, данные учебные материалы не
затрагивают эти специфические виды течений, характерные скорее для
фундаментальной гидродинамики, чем для динамики атмосферы, где обычно
акцент делается на крупномасштабные и мезомасштабные процессы, где
доминируют другие физические механизмы.

ewpage

\hypertarget{ux443ux440ux430ux432ux43dux435ux43dux438ux44f-ux433ux438ux434ux440ux43eux442ux435ux440ux43cux43eux434ux438ux43dux430ux43cux438ux43aux438-ux432-ux441ux444ux435ux440ux438ux447ux435ux441ux43aux438ux445-ux43aux43eux43eux440ux434ux438ux43dux430ux442ux430ux445}{%
\section{Уравнения Гидротермодинамики в Сферических
Координатах}\label{ux443ux440ux430ux432ux43dux435ux43dux438ux44f-ux433ux438ux434ux440ux43eux442ux435ux440ux43cux43eux434ux438ux43dux430ux43cux438ux43aux438-ux432-ux441ux444ux435ux440ux438ux447ux435ux441ux43aux438ux445-ux43aux43eux43eux440ux434ux438ux43dux430ux442ux430ux445}}

При изучении крупномасштабных атмосферных движений, особенно
планетарного масштаба, традиционно используются уравнения
гидротермодинамики, представленные в сферической системе координат. Это
позволяет учесть кривизну Земли и вращение планеты, что критически важно
для адекватного описания динамики атмосферы. Начало этой системы
координат располагается в центре Земли. Координаты точки определяются
радиус-вектором \(r\) (расстояние от центра Земли), углом географической
долготы \(\lambda\) (отсчитывается к востоку от начального меридиана) и
полярным углом \(\theta\) (дополнение географической широты).

\hypertarget{ux443ux440ux430ux432ux43dux435ux43dux438ux44f-ux434ux432ux438ux436ux435ux43dux438ux44f-ux437ux430ux43aux43eux43d-ux441ux43eux445ux440ux430ux43dux435ux43dux438ux44f-ux438ux43cux43fux443ux43bux44cux441ux430}{%
\subsection{Уравнения Движения (Закон Сохранения
Импульса)}\label{ux443ux440ux430ux432ux43dux435ux43dux438ux44f-ux434ux432ux438ux436ux435ux43dux438ux44f-ux437ux430ux43aux43eux43d-ux441ux43eux445ux440ux430ux43dux435ux43dux438ux44f-ux438ux43cux43fux443ux43bux44cux441ux430}}

Для атмосферных движений в сферических координатах, уравнения движения,
выражающие второй закон Ньютона (закон сохранения импульса), могут быть
записаны для каждой компоненты скорости --- радиальной (\(V_r\)),
меридиональной (\(V_{\theta}\)) и зональной (\(V_{\lambda}\)):

\begin{enumerate}
\def\labelenumi{\arabic{enumi}.}
\item
  \textbf{Для радиальной компоненты (\(r\))}: \[
   \frac{dV_r}{dt} = -\frac{1}{\rho}\frac{\partial P}{\partial r} - g + \frac{V_{\theta}^2 + V_{\lambda}^2}{r} - 2\omega(V_{\theta}\cos\theta + V_{\lambda}\sin\theta\sin\lambda) \quad \text{}
   \]
\item
  \textbf{Для меридиональной компоненты (\(\theta\))}: \[
   \frac{dV_{\theta}}{dt} = -\frac{1}{\rho r}\frac{\partial P}{\partial \theta} + \frac{V_{\lambda}^2 \operatorname{ctg}\theta}{r} - \frac{V_r V_{\theta}}{r} + 2\omega(V_r\cos\theta - V_{\lambda}\sin\theta\cos\lambda) \quad \text{}
   \]
\item
  \textbf{Для зональной компоненты (\(\lambda\))}: \[
   \frac{dV_{\lambda}}{dt} = -\frac{1}{\rho r \sin\theta}\frac{\partial P}{\partial \lambda} - \frac{V_r V_{\lambda}}{r} - \frac{V_{\theta} V_{\lambda} \operatorname{ctg}\theta}{r} + 2\omega(V_r\sin\theta + V_{\theta}\sin\theta\cos\lambda) \quad \text{}
   \]
\end{enumerate}

Где:

\begin{itemize}
\tightlist
\item
  \(P\) --- давление атмосферы
\item
  \(\rho\) --- плотность воздуха
\item
  \(g\) --- ускорение свободного падения (сила тяжести)
\item
  \(\omega\) --- угловая скорость вращения Земли
\item
  \(V_r, V_{\theta}, V_{\lambda}\) --- проекции вектора скорости на оси
  координат
\end{itemize}

Важно отметить, что такие члены, как
\(\frac{V_{\theta}^2 + V_{\lambda}^2}{r}\),
\(\frac{V_{\lambda}^2 \operatorname{ctg}\theta}{r}\),
\(-\frac{V_r V_{\theta}}{r}\), \(-\frac{V_r V_{\lambda}}{r}\), и
\(-\frac{V_{\theta} V_{\lambda} \operatorname{ctg}\theta}{r}\) в левых
частях уравнений представляют собой дополнительные ускорения,
возникающие из-за кривизны параллелей и меридианов в сферической системе
координат. Эти квадратичные члены, обусловленные кривизной координатных
линий, отражают кинематические особенности движения в неинерциальной
вращающейся системе координат.

\hypertarget{ux434ux440ux443ux433ux438ux435-ux444ux443ux43dux434ux430ux43cux435ux43dux442ux430ux43bux44cux43dux44bux435-ux443ux440ux430ux432ux43dux435ux43dux438ux44f}{%
\subsection{Другие Фундаментальные
Уравнения}\label{ux434ux440ux443ux433ux438ux435-ux444ux443ux43dux434ux430ux43cux435ux43dux442ux430ux43bux44cux43dux44bux435-ux443ux440ux430ux432ux43dux435ux43dux438ux44f}}

Полная система уравнений гидротермодинамики также включает уравнения,
выражающие законы сохранения массы и энергии, а также уравнение
состояния:

\hypertarget{ux443ux440ux430ux432ux43dux435ux43dux438ux435-ux43dux435ux440ux430ux437ux440ux44bux432ux43dux43eux441ux442ux438-ux437ux430ux43aux43eux43d-ux441ux43eux445ux440ux430ux43dux435ux43dux438ux44f-ux43cux430ux441ux441ux44b-1}{%
\subsubsection{Уравнение Неразрывности (Закон Сохранения
Массы)}\label{ux443ux440ux430ux432ux43dux435ux43dux438ux435-ux43dux435ux440ux430ux437ux440ux44bux432ux43dux43eux441ux442ux438-ux437ux430ux43aux43eux43d-ux441ux43eux445ux440ux430ux43dux435ux43dux438ux44f-ux43cux430ux441ux441ux44b-1}}

Это уравнение связывает изменение плотности воздуха во времени с
распределением скорости движения в пространстве. В общей форме оно
записывается как: \[
\frac{d\rho}{dt} + \rho(\nabla \cdot \vec{V}) = 0 \quad \text{}
\] где \(\nabla \cdot \vec{V}\) представляет дивергенцию вектора
скорости. В контексте сферических координат дивергенция также имеет свою
форму, хотя явное ее выражение в данном источнике не приведено.

\hypertarget{ux443ux440ux430ux432ux43dux435ux43dux438ux435-ux43fux440ux438ux442ux43eux43aux430-ux442ux435ux43fux43bux430-ux43fux435ux440ux432ux43eux435-ux43dux430ux447ux430ux43bux43e-ux442ux435ux440ux43cux43eux434ux438ux43dux430ux43cux438ux43aux438-2}{%
\subsubsection{Уравнение Притока Тепла (Первое Начало
Термодинамики)}\label{ux443ux440ux430ux432ux43dux435ux43dux438ux435-ux43fux440ux438ux442ux43eux43aux430-ux442ux435ux43fux43bux430-ux43fux435ux440ux432ux43eux435-ux43dux430ux447ux430ux43bux43e-ux442ux435ux440ux43cux43eux434ux438ux43dux430ux43cux438ux43aux438-2}}

Уравнение притока тепла является выражением закона сохранения энергии и
в общей форме для единицы массы воздуха может быть представлено как: \[
dQ = c_p dT - \frac{RT}{P} dP \quad \text{}
\] Где:

\begin{itemize}
\tightlist
\item
  \(dQ\) --- приток тепла к единице массы воздуха
\item
  \(c_p\) --- удельная теплоемкость при постоянном давлении
\item
  \(dT\) --- изменение температуры
\item
  \(R\) --- газовая постоянная воздуха
\item
  \(T\) --- абсолютная температура
\item
  \(dP\) --- изменение давления
\end{itemize}

\hypertarget{ux443ux440ux430ux432ux43dux435ux43dux438ux435-ux441ux43eux441ux442ux43eux44fux43dux438ux44f-1}{%
\subsubsection{Уравнение
Состояния}\label{ux443ux440ux430ux432ux43dux435ux43dux438ux435-ux441ux43eux441ux442ux43eux44fux43dux438ux44f-1}}

Это алгебраическое соотношение связывает основные термодинамические
параметры атмосферного воздуха -- давление, плотность и температуру: \[
P = \rho RT \quad \text{}
\] Где:

\begin{itemize}
\tightlist
\item
  \(P\) --- давление
\item
  \(\rho\) --- плотность
\item
  \(R\) --- удельная газовая постоянная сухого воздуха
\item
  \(T\) --- абсолютная температура
\end{itemize}

Хотя уравнения неразрывности и притока тепла представлены здесь в общих
формах, их конкретные проекции на оси сферической системы координат
необходимы для полного численного моделирования атмосферных процессов на
планетарном масштабе. Однако приведенные уравнения движения в
сферических координатах являются краеугольным камнем для динамической
метеорологии.

ewpage

В рамках динамической метеорологии, при анализе крупномасштабных
атмосферных процессов, одной из наиболее удобных и широко используемых
систем координат является изобарическая система, где вместо
геометрической высоты в качестве вертикальной координаты используется
давление. Этот переход обусловлен рядом существенных преимуществ, хотя и
сопряжен с определенными упрощениями и ограничениями.

\hypertarget{ux43fux440ux435ux43eux431ux440ux430ux437ux43eux432ux430ux43dux438ux435-ux43aux43eux43eux440ux434ux438ux43dux430ux442}{%
\subsubsection{Преобразование
Координат}\label{ux43fux440ux435ux43eux431ux440ux430ux437ux43eux432ux430ux43dux438ux435-ux43aux43eux43eux440ux434ux438ux43dux430ux442}}

В изобарической системе координат \((x, y, p)\), где \(p\) ---
атмосферное давление, вертикальная координата \(z\) заменяется на \(p\).
Это преобразование осуществляется посредством использования основного
уравнения статики атмосферы, что является фундаментом для так
называемого квазистатического приближения. Потенциальная энергия в такой
системе выражается через геопотенциал \(H\), который является зависимой
переменной, тогда как давление \(p\) становится независимой вертикальной
координатой.

Индивидуальная производная по времени \(d/dt\) в декартовых координатах,
которая включала \(w \frac{\partial}{\partial z}\) (где \(w\) ---
вертикальная скорость), преобразуется в изобарической системе. Здесь
вертикальная скорость \(\omega\) определяется как индивидуальная
производная давления по времени (\$\omega = dp/dt \$). Таким образом,
общая форма индивидуальной производной выглядит как:

\$\frac{d}{dt} = \frac{\partial}{\partial t} + u
\frac{\partial}{\partial x} + v \frac{\partial}{\partial y} +
\omega \frac{\partial}{\partial p} \$

где \(u, v\) --- горизонтальные компоненты скорости, а \(x, y\) ---
горизонтальные координаты.

\hypertarget{ux443ux440ux430ux432ux43dux435ux43dux438ux44f-ux434ux432ux438ux436ux435ux43dux438ux44f-1}{%
\subsubsection{Уравнения
Движения}\label{ux443ux440ux430ux432ux43dux435ux43dux438ux44f-ux434ux432ux438ux436ux435ux43dux438ux44f-1}}

В изобарической системе координат, уравнения движения, выражающие второй
закон Ньютона, приобретают более простую форму, особенно для
горизонтальных компонент, так как силы барического градиента
преобразуются в градиенты геопотенциала.

\hypertarget{ux433ux43eux440ux438ux437ux43eux43dux442ux430ux43bux44cux43dux44bux435-ux443ux440ux430ux432ux43dux435ux43dux438ux44f}{%
\paragraph{Горизонтальные
Уравнения}\label{ux433ux43eux440ux438ux437ux43eux43dux442ux430ux43bux44cux43dux44bux435-ux443ux440ux430ux432ux43dux435ux43dux438ux44f}}

Горизонтальные уравнения движения для единицы массы воздуха в
изобарических координатах, с учетом силы Кориолиса и сил трения
(\(F_x, F_y\)), могут быть записаны следующим образом:

\$\frac{du}{dt} - lv = -\frac{\partial H}{\partial x} + F\_x \quad (1)
\$ \$\frac{dv}{dt} + lu = -\frac{\partial H}{\partial y} + F\_y
\quad (2) \$

где \(l = 2\Omega \sin\varphi\) --- параметр Кориолиса (обозначение
\(\Omega\) используется для угловой скорости вращения Земли, \(\varphi\)
--- широта), \(H\) --- геопотенциал, а \(F_x, F_y\) --- проекции силы
трения на оси \(x\) и \(y\) соответственно. В этих уравнениях исчезает
член, связанный с плотностью, что упрощает расчеты, так как сила
горизонтального градиента давления прямо пропорциональна градиенту
геопотенциала на изобарической поверхности.

\hypertarget{ux432ux435ux440ux442ux438ux43aux430ux43bux44cux43dux43eux435-ux443ux440ux430ux432ux43dux435ux43dux438ux435-ux433ux438ux434ux440ux43eux441ux442ux430ux442ux438ux447ux435ux441ux43aux43eux435}{%
\paragraph{Вертикальное Уравнение
(Гидростатическое)}\label{ux432ux435ux440ux442ux438ux43aux430ux43bux44cux43dux43eux435-ux443ux440ux430ux432ux43dux435ux43dux438ux435-ux433ux438ux434ux440ux43eux441ux442ux430ux442ux438ux447ux435ux441ux43aux43eux435}}

Вертикальное уравнение движения в изобарической системе координат
редуцируется до уравнения гидростатики. Это одно из ключевых допущений
квазистатического приближения, применимого к крупномасштабным
атмосферным процессам, где вертикальные ускорения значительно меньше
силы тяжести и вертикальной составляющей силы барического градиента.
Уравнение имеет вид:

\$\frac{\partial H}{\partial p} = -\frac{RT}{p} \quad (3) \$

где \(R\) --- удельная газовая постоянная для воздуха, \(T\) ---
абсолютная температура. Это уравнение связывает изменение геопотенциала
с изменением давления по вертикали и температурой.

\hypertarget{ux443ux440ux430ux432ux43dux435ux43dux438ux435-ux43dux435ux440ux430ux437ux440ux44bux432ux43dux43eux441ux442ux438-ux441ux43eux445ux440ux430ux43dux435ux43dux438ux44f-ux43cux430ux441ux441ux44b}{%
\subsubsection{Уравнение Неразрывности (Сохранения
Массы)}\label{ux443ux440ux430ux432ux43dux435ux43dux438ux435-ux43dux435ux440ux430ux437ux440ux44bux432ux43dux43eux441ux442ux438-ux441ux43eux445ux440ux430ux43dux435ux43dux438ux44f-ux43cux430ux441ux441ux44b}}

Закон сохранения массы, выражаемый уравнением неразрывности, также
претерпевает значительное упрощение в изобарической системе координат. В
этой системе уравнение приобретает вид:

\$\frac{\partial u}{\partial x} + \frac{\partial v}{\partial y} +
\frac{\partial \omega}{\partial p} = 0 \quad (4) \$

Как отмечают источники, в координатах \(x, y, p\) атмосферный воздух
может быть описан как несжимаемая жидкость, что существенно упрощает
расчеты, так как член, характеризующий изменение плотности, пропадает.

\hypertarget{ux443ux440ux430ux432ux43dux435ux43dux438ux435-ux43fux440ux438ux442ux43eux43aux430-ux442ux435ux43fux43bux430}{%
\subsubsection{Уравнение Притока
Тепла}\label{ux443ux440ux430ux432ux43dux435ux43dux438ux435-ux43fux440ux438ux442ux43eux43aux430-ux442ux435ux43fux43bux430}}

Уравнение притока тепла является математическим выражением первого
начала термодинамики. В изобарической системе координат оно связывает
изменение температуры (или потенциальной температуры) с притоком тепла и
вертикальными движениями. В общем виде, с учетом притока тепла \(Q\) на
единицу массы, оно может быть записано как:

\$\frac{\partial T}{\partial t} + u \frac{\partial T}{\partial x} + v
\frac{\partial T}{\partial y} + \omega \left(
\frac{\partial T}{\partial p} - \frac{RT}{p C_p} \right) = \frac{Q}{C_p}
\quad (5) \$

где \(C_p\) --- удельная теплоемкость воздуха при постоянном давлении.
Член \(\left( \frac{\partial T}{\partial p} - \frac{RT}{p C_p} \right)\)
характеризует вертикальную устойчивость атмосферы. Для не слишком
больших промежутков времени (порядка суток) притоком тепла извне иногда
пренебрегают, что приводит к адиабатической модели.

\hypertarget{ux443ux440ux430ux432ux43dux435ux43dux438ux435-ux441ux43eux441ux442ux43eux44fux43dux438ux44f-2}{%
\subsubsection{Уравнение
Состояния}\label{ux443ux440ux430ux432ux43dux435ux43dux438ux435-ux441ux43eux441ux442ux43eux44fux43dux438ux44f-2}}

Система уравнений гидротермодинамики замыкается уравнением состояния
идеального газа, которое связывает давление, плотность и температуру
воздуха. Оно сохраняет свой вид независимо от выбранной системы
координат:

\$ P = \rho R T \quad (6) \$

где \(\rho\) --- плотность воздуха.

\hypertarget{ux43fux440ux435ux438ux43cux443ux449ux435ux441ux442ux432ux430-ux438-ux43dux435ux434ux43eux441ux442ux430ux442ux43aux438-ux438ux437ux43eux431ux430ux440ux438ux447ux435ux441ux43aux43eux439-ux441ux438ux441ux442ux435ux43cux44b}{%
\subsubsection{Преимущества и Недостатки Изобарической
Системы}\label{ux43fux440ux435ux438ux43cux443ux449ux435ux441ux442ux432ux430-ux438-ux43dux435ux434ux43eux441ux442ux430ux442ux43aux438-ux438ux437ux43eux431ux430ux440ux438ux447ux435ux441ux43aux43eux439-ux441ux438ux441ux442ux435ux43cux44b}}

\textbf{Преимущества:}

\begin{itemize}
\tightlist
\item
  \textbf{Упрощение уравнений}: В изобарической системе, горизонтальные
  уравнения движения и уравнение неразрывности значительно упрощаются за
  счет исключения плотности из явного вида уравнений, что описывается
  как свойство ``несжимаемости'' в данной системе.
\item
  \textbf{Удобство для барического поля}: Поскольку барическое поле
  (поле давления) является одной из ключевых характеристик атмосферы,
  использование давления в качестве вертикальной координаты естественным
  образом интегрирует его в уравнения. Карты барической топографии
  (карты высот изобарических поверхностей) являются основным
  инструментом синоптического анализа.
\end{itemize}

\textbf{Недостатки:}

\begin{itemize}
\tightlist
\item
  \textbf{Потеря бароклинных членов}: Главный недостаток этой системы
  заключается в потере бароклинных членов в уравнениях для компонент
  вихря скорости ветра, а также в выражении для дивергенции
  геострофического ветра. Бароклинность (зависимость плотности не только
  от давления, но и от температуры) является ключевым фактором в
  динамике атмосферы, и ее упрощенное представление может приводить к
  потере важной физической информации.
\item
  \textbf{Ограничения на вихревые движения в вертикальных плоскостях}: В
  изобарической системе невозможно получить уравнения для вихревых
  движений (циркуляций) в вертикальных плоскостях, поскольку третье
  уравнение движения (по вертикали) сведено к уравнению статики и
  использовано для определения новой координаты. Это ограничивает
  описание таких важных циркуляций, как муссонные, бризовые,
  горно-долинные и фронтальные.
\item
  \textbf{Зависимость от квазистатического приближения}: Использование
  изобарических координат тесно связано с квазистатическим приближением,
  которое не всегда полностью справедливо для всех атмосферных
  процессов, особенно для тех, где вертикальные ускорения значительны
  (например, в конвективных облаках или вихрях малого масштаба).
\end{itemize}

Несмотря на эти ограничения, изобарическая система координат остается
фундаментальной для моделирования и прогнозирования крупномасштабных
атмосферных процессов благодаря своей простоте и эффективности в
определенных контекстах.

ewpage

Как профессиональные метеорологи, мы понимаем, что термин
``орографические координаты'' не обозначает отдельную формальную систему
координат в том же смысле, что декартовы, сферические или изобарические
системы. Скорее, это концептуальный подход к адаптации и применению
общих уравнений гидротермодинамики для учета влияния рельефа земной
поверхности. При анализе атмосферных процессов в горной местности мы
сталкиваемся с уникальными динамическими и термодинамическими эффектами,
которые требуют особого внимания при формулировке и решении уравнений.

\hypertarget{ux43eux431ux449ux438ux435-ux443ux440ux430ux432ux43dux435ux43dux438ux44f-ux433ux438ux434ux440ux43eux442ux435ux440ux43cux43eux434ux438ux43dux430ux43cux438ux43aux438}{%
\subsection{Общие Уравнения
Гидротермодинамики}\label{ux43eux431ux449ux438ux435-ux443ux440ux430ux432ux43dux435ux43dux438ux44f-ux433ux438ux434ux440ux43eux442ux435ux440ux43cux43eux434ux438ux43dux430ux43cux438ux43aux438}}

В основе динамической метеорологии лежат общие уравнения
гидротермодинамики, которые представляют собой математическое выражение
фундаментальных законов физики: закона сохранения импульса движения
(второго закона Ньютона), закона сохранения энергии и закона сохранения
массы. Эти уравнения описывают атмосферные движения и процессы тепло- и
влагообмена, определяющие погоду и климат.

Система этих уравнений включает:

\begin{itemize}
\tightlist
\item
  \textbf{Уравнения движения:} Три скалярных уравнения, обычно проекции
  векторного уравнения движения на оси координат, учитывающие силы
  давления, Кориолиса, тяжести и вязкости.
\item
  \textbf{Уравнение неразрывности (сохранения массы):} Связывает
  изменение плотности воздуха во времени с распределением скорости
  движения в пространстве.
\item
  \textbf{Уравнение притока тепла (первое начало термодинамики):}
  Описывает изменение количества тепловой энергии в единице объема
  воздуха.
\item
  \textbf{Уравнение состояния атмосферного воздуха:} Связывает давление,
  плотность и температуру.
\item
  \textbf{Уравнение переноса водяного пара:} Учитывает изменения
  влажности за счет адвекции, турбулентной диффузии и фазовых переходов
  воды.
\end{itemize}

Эти уравнения могут быть сформулированы в различных системах координат:

\begin{itemize}
\tightlist
\item
  \textbf{Декартова система координат (x, y, z):} Часто используется для
  общих формулировок и анализа проекций векторов.
\item
  \textbf{Сферическая система координат (r, λ, θ):} Применяется для
  изучения атмосферных движений планетарного масштаба.
\item
  \textbf{Натуральная система координат (s, n):} Используется для
  кинематического анализа двумерного поля скорости, где координатными
  линиями являются линии тока и нормали к ним.
\end{itemize}

\hypertarget{ux443ux447ux435ux442-ux43eux440ux43eux433ux440ux430ux444ux438ux438-ux432-ux443ux440ux430ux432ux43dux435ux43dux438ux44fux445-ux433ux438ux434ux440ux43eux442ux435ux440ux43cux43eux434ux438ux43dux430ux43cux438ux43aux438}{%
\subsection{Учет Орографии в Уравнениях
Гидротермодинамики}\label{ux443ux447ux435ux442-ux43eux440ux43eux433ux440ux430ux444ux438ux438-ux432-ux443ux440ux430ux432ux43dux435ux43dux438ux44fux445-ux433ux438ux434ux440ux43eux442ux435ux440ux43cux43eux434ux438ux43dux430ux43cux438ux43aux438}}

Когда речь идет об ``орографических координатах'' или учете орографии,
фактически это означает, что уравнения движения и термодинамики
атмосферы применяются к условиям, где рельеф поверхности играет
существенную роль. Это достигается не изменением самой системы координат
на качественно новую ``орографическую'', а скорее:

\hypertarget{ux43cux43eux434ux438ux444ux438ux43aux430ux446ux438ux44f-ux433ux440ux430ux43dux438ux447ux43dux44bux445-ux443ux441ux43bux43eux432ux438ux439}{%
\subsubsection{1. Модификация Граничных
Условий}\label{ux43cux43eux434ux438ux444ux438ux43aux430ux446ux438ux44f-ux433ux440ux430ux43dux438ux447ux43dux44bux445-ux443ux441ux43bux43eux432ux438ux439}}

На нижней границе расчетной области (т.е. на поверхности Земли)
граничные условия должны отражать фактический рельеф. Например,
вертикальная скорость у поверхности земли \(w\) не всегда равна нулю, а
зависит от наклона поверхности и горизонтальной скорости ветра. Это
приводит к так называемому ``орографическому подъему'' воздуха, который
является важным фактором в формировании облаков и осадков.

\hypertarget{ux432ux43bux438ux44fux43dux438ux435-ux43dux430-ux447ux43bux435ux43dux44b-ux443ux440ux430ux432ux43dux435ux43dux438ux439}{%
\subsubsection{2. Влияние на Члены
Уравнений}\label{ux432ux43bux438ux44fux43dux438ux435-ux43dux430-ux447ux43bux435ux43dux44b-ux443ux440ux430ux432ux43dux435ux43dux438ux439}}

Орография оказывает прямое влияние на различные члены в уравнениях:

\begin{itemize}
\tightlist
\item
  \textbf{Вертикальные движения:} Горная местность значительно влияет на
  вертикальную составляющую скорости. В то время как для гладкой
  поверхности \(w=0\) при \(z=0\), в условиях орографии возникают
  вынужденные восходящие или нисходящие движения. Например, в численных
  моделях вертикальная скорость \(w(z)\) может быть параметризована как
  функция высоты, учитывающая дивергенцию горизонтальной скорости.
\item
  \textbf{Бароклинность:} Атмосфера в горных районах часто
  характеризуется сильной бароклинностью, то есть изобарические и
  изотермические поверхности пересекаются. Источники указывают, что
  бароклинность играет важную роль в зарождении и эволюции синоптических
  вихрей, атмосферных фронтов и фронтальных зон. Горная местность может
  способствовать образованию или обострению фронтов, что напрямую
  связано с бароклинными эффектами.
\item
  \textbf{Турбулентный обмен:} Над сложным рельефом значительно
  усиливается турбулентное перемешивание, что влияет на коэффициенты
  турбулентного обмена тепла, влаги и импульса, которые входят в
  турбулентные члены уравнений.
\end{itemize}

\hypertarget{ux43eux441ux43eux431ux435ux43dux43dux43eux441ux442ux438-ux43fux440ux438ux43cux435ux43dux435ux43dux438ux44f-ux43aux43eux43eux440ux434ux438ux43dux430ux442ux43dux44bux445-ux441ux438ux441ux442ux435ux43c}{%
\subsubsection{3. Особенности Применения Координатных
Систем}\label{ux43eux441ux43eux431ux435ux43dux43dux43eux441ux442ux438-ux43fux440ux438ux43cux435ux43dux435ux43dux438ux44f-ux43aux43eux43eux440ux434ux438ux43dux430ux442ux43dux44bux445-ux441ux438ux441ux442ux435ux43c}}

Выбор координатной системы имеет важное значение. Источники критикуют
использование \textbf{изобарической системы координат} (где вертикальной
координатой является давление) для процессов, где бароклинность играет
ключевую роль. В такой системе теряются бароклинные члены в уравнениях
для составляющих вихря скорости ветра, и становится невозможным
получение уравнений для вихревых движений в вертикальных плоскостях,
которые являются основой муссонной, бризовой, горно-долинной циркуляций.
Это означает, что для адекватного описания атмосферных процессов, сильно
подверженных орографическому влиянию, предпочтительнее использовать
системы координат, явно сохраняющие вертикальное измерение (например,
декартовы или σ-координаты, хотя последние не упомянуты в
предоставленных источниках, но подразумеваются в контексте современных
моделей).

\hypertarget{ux43eux440ux43eux433ux440ux430ux444ux438ux447ux435ux441ux43aux438ux435-ux44dux444ux444ux435ux43aux442ux44b-ux432-ux434ux438ux43dux430ux43cux438ux43aux435-ux430ux442ux43cux43eux441ux444ux435ux440ux44b}{%
\subsection{Орографические Эффекты в Динамике
Атмосферы}\label{ux43eux440ux43eux433ux440ux430ux444ux438ux447ux435ux441ux43aux438ux435-ux44dux444ux444ux435ux43aux442ux44b-ux432-ux434ux438ux43dux430ux43cux438ux43aux435-ux430ux442ux43cux43eux441ux444ux435ux440ux44b}}

Учет орографии в уравнениях позволяет описывать такие явления, как:

\begin{itemize}
\tightlist
\item
  \textbf{Изменение фронтов и циклонов:} Горные хребты могут замедлять
  перемещение барических систем, задерживать фронты, вызывать
  орографическую окклюзию. Наблюдается антициклогенез с наветренной
  стороны гор и циклогенез --- с подветренной.
\item
  \textbf{Орографические осадки:} Выпадение осадков на наветренных
  склонах гор вследствие вынужденного подъема и охлаждения воздуха.
\item
  \textbf{Местные ветры:} Формирование бризов, горно-долинных ветров,
  фёнов и бор, обусловленных рельефом.
\item
  \textbf{Инверсии температуры:} Горные перевалы могут приводить к
  образованию орографических инверсий оседания.
\end{itemize}

В заключение, хотя нет отдельной ``орографической системы координат'',
учет сложного рельефа является критически важным аспектом динамической
метеорологии. Он достигается через соответствующую формулировку
граничных условий, учет специфических физических процессов (таких как
вынужденные вертикальные движения и усиленная бароклинность) в рамках
традиционных координатных систем и, при необходимости, через выбор
координатной системы, которая наилучшим образом сохраняет эти важные
физические эффекты.

ewpage

Уважаемый коллега,

Рассмотрим концепцию инерционных волн в баротропной атмосфере, в
частности, волн Россби -- ключевого элемента динамики атмосферы,
определяющего крупномасштабные синоптические процессы.

\hypertarget{ux431ux430ux440ux43eux442ux440ux43eux43fux43dux430ux44f-ux430ux442ux43cux43eux441ux444ux435ux440ux430-ux438-ux438ux43dux435ux440ux446ux438ux43eux43dux43dux44bux435-ux432ux43eux43bux43dux44b}{%
\subsubsection{Баротропная Атмосфера и Инерционные
Волны}\label{ux431ux430ux440ux43eux442ux440ux43eux43fux43dux430ux44f-ux430ux442ux43cux43eux441ux444ux435ux440ux430-ux438-ux438ux43dux435ux440ux446ux438ux43eux43dux43dux44bux435-ux432ux43eux43bux43dux44b}}

Для начала, напомним, что \textbf{баротропной средой} в динамической
метеорологии называется жидкость (в нашем случае, атмосфера), в которой
плотность является функцией только давления (ρ = ρ(P)). Это означает,
что изобарические (P=const), изопикнические (ρ=const) и изотермические
(T=const) поверхности в такой среде параллельны друг другу. Хотя
реальная атмосфера, особенно в крупномасштабных процессах, является
\textbf{бароклинной} (плотность зависит как от давления, так и от
температуры, и изобарические и изотермические поверхности пересекаются),
баротропная модель служит важным упрощением для понимания
фундаментальных закономерностей, представляя собой усредненное по высоте
состояние атмосферы.

В контексте атмосферных движений, \textbf{инерционные волны} относятся к
классу быстрых, короткопериодных движений, при которых колебания
некоторых инвариантов (например, потенциальной температуры или
потенциального вихря) отсутствуют, по крайней мере, в линейном
приближении. Движение воздушных масс в атмосфере подвержено влиянию
различных сил, среди которых для крупномасштабных процессов особое
значение имеет \textbf{отклоняющая сила вращения Земли}, или
\textbf{сила Кориолиса}. Для движений большого масштаба, таких как
устойчивые течения общей циркуляции, сила Кориолиса значительно
превосходит инерционные силы в относительном движении, что является
предпосылкой для возникновения инерционных явлений.

\hypertarget{ux432ux43eux43bux43dux44b-ux440ux43eux441ux441ux431ux438}{%
\subsubsection{Волны
Россби}\label{ux432ux43eux43bux43dux44b-ux440ux43eux441ux441ux431ux438}}

\textbf{Волны Россби} представляют собой особый тип волновых движений,
характерных для крупномасштабных атмосферных процессов. Они являются
\textbf{планетарными волнами}, которые возникают в баротропной (или
квазибаротропной) атмосфере на вращающейся планете, где параметр
Кориолиса (2ωsinφ) изменяется с широтой.

Ключевые аспекты волн Россби включают:

\begin{enumerate}
\def\labelenumi{\arabic{enumi}.}
\tightlist
\item
  \textbf{Природа и Масштабы:} Волны Россби --- это крупномасштабные
  волнообразные сдвиги воздушных потоков, преимущественно в
  горизонтальной плоскости, связанные с особенностями барического поля
  (гребнями и ложбинами). Чарльз Россби различал длинные волны (с длиной
  волны более 5000 км), которым соответствуют семейства циклонов, и
  короткие волны (с длиной волной менее 5000 км), часто связанные с
  отдельными циклонами и антициклонами. Это указывает на их
  синоптический и планетарный масштабы.
\item
  \textbf{Теорема Россби (Сохранение Абсолютного Вихря):}
  Фундаментальным принципом, лежащим в основе существования волн Россби,
  является \textbf{сохранение абсолютного вихря} (Qz + 2ωz = const) в
  движущейся воздушной массе, при условии, что слагаемые в правой части
  уравнения движения обращаются в нуль. Абсолютный вихрь -- это сумма
  относительного вихря скорости ветра (Qz) и переносного вихря (2ωz),
  обусловленного суточным вращением Земли. Этот инвариант играет
  критическую роль в динамике атмосферы.
\item
  \textbf{Формула для Скорости Смещения Волны (Фазовая Скорость):}
  Россби вывел формулу для скорости смещения длинных волн в атмосфере:
  \texttt{c\ =\ u\ -\ (βL\^{}2)/(4π\^{}2)} Где:

  \begin{itemize}
  \tightlist
  \item
    \texttt{c} -- скорость смещения волны (фазовая скорость).
  \item
    \texttt{u} -- скорость зонального потока (средний западный поток, в
    котором распространяется волна).
  \item
    \texttt{L} -- длина волны.
  \item
    \texttt{β} -- изменение параметра Кориолиса (2ωsinφ) с широтой
    (dL/dφ), т.е. изменение эффекта Кориолиса при движении по меридиану.
    Эта формула показывает, что волны Россби движутся относительно
    основного потока на запад (так как член \texttt{(βL\^{}2)/(4π\^{}2)}
    всегда положителен), а абсолютная скорость зависит от скорости
    зонального переноса.
  \end{itemize}
\item
  \textbf{Режим Движения:} В отличие от гравитационных волн, где
  колебания частиц воздуха происходят преимущественно в вертикальной
  плоскости, волны Россби характеризуются волнообразными сдвигами
  воздушных потоков преимущественно в горизонтальной плоскости.
\item
  \textbf{Значение для Синоптики:} Волновая теория связывает зарождение
  внетропических циклонов с волновыми колебаниями фронтальной
  поверхности. Неустойчивые фронтальные волны (длиной от 800 до 2800 км)
  могут развиваться в циклоны. Таким образом, понимание динамики волн
  Россби критически важно для прогнозирования крупномасштабных
  барических систем и связанных с ними погодных явлений.
\end{enumerate}

Необходимо отметить, что, хотя формально ``волны давления'' могут
рассматриваться как волновой процесс, они не имеют физического
толкования, в отличие от волн Россби, чья динамика хорошо описана и
подтверждена наблюдениями.

Таким образом, волны Россби представляют собой краеугольный камень в
изучении динамики крупномасштабных атмосферных процессов, объясняя
распространение и эволюцию планетарных барических систем через призму
сохранения абсолютного вихря и изменения силы Кориолиса с широтой.

ewpage

Коллега,

Рассмотрим природу гравитационно-инерционных волн в контексте
геострофического потока, опираясь на фундаментальные принципы динамики
атмосферы. Хотя специфические названия волн Пуанкаре и Кельвина прямо не
упоминаются в представленных материалах, мы можем обсудить их общие
категории --- гравитационные и инерционные волны, а также их
взаимодействие с геострофическим приближением.

\hypertarget{ux433ux440ux430ux432ux438ux442ux430ux446ux438ux43eux43dux43dux43e-ux438ux43dux435ux440ux446ux438ux43eux43dux43dux44bux435-ux432ux43eux43bux43dux44b-ux432-ux433ux435ux43eux441ux442ux440ux43eux444ux438ux447ux435ux441ux43aux43eux43c-ux43fux43eux442ux43eux43aux435}{%
\section{Гравитационно-Инерционные Волны в Геострофическом
Потоке}\label{ux433ux440ux430ux432ux438ux442ux430ux446ux438ux43eux43dux43dux43e-ux438ux43dux435ux440ux446ux438ux43eux43dux43dux44bux435-ux432ux43eux43bux43dux44b-ux432-ux433ux435ux43eux441ux442ux440ux43eux444ux438ux447ux435ux441ux43aux43eux43c-ux43fux43eux442ux43eux43aux435}}

\hypertarget{ux434ux438ux43dux430ux43cux438ux447ux435ux441ux43aux438ux439-ux43aux43eux43dux442ux435ux43aux441ux442}{%
\subsection{Динамический
Контекст}\label{ux434ux438ux43dux430ux43cux438ux447ux435ux441ux43aux438ux439-ux43aux43eux43dux442ux435ux43aux441ux442}}

Движение воздуха в атмосфере определяется сложным взаимодействием
массовых и поверхностных сил. К массовым силам относятся сила тяжести и
сила Кориолиса. Поверхностные силы включают силу барического градиента и
силу трения. Сила Кориолиса, обусловленная вращением Земли, является
отклоняющей силой, действующей перпендикулярно вектору скорости движения
частицы --- вправо в Северном полушарии и влево в Южном.

\hypertarget{ux433ux435ux43eux441ux442ux440ux43eux444ux438ux447ux435ux441ux43aux43eux435-ux43fux440ux438ux431ux43bux438ux436ux435ux43dux438ux435}{%
\subsection{Геострофическое
Приближение}\label{ux433ux435ux43eux441ux442ux440ux43eux444ux438ux447ux435ux441ux43aux43eux435-ux43fux440ux438ux431ux43bux438ux436ux435ux43dux438ux435}}

В крупномасштабных атмосферных процессах, особенно в свободной
атмосфере, движение воздуха часто аппроксимируется геострофическим
потоком. Геострофический ветер --- это идеализированное движение
воздуха, при котором горизонтальные составляющие силы барического
градиента уравновешиваются силой Кориолиса. Такое равновесие
предполагает прямолинейные изобары или изогипсы. Наблюдения показывают,
что фактический ветер в свободной атмосфере весьма близок к
геострофическому по скорости и направлению.

\hypertarget{ux43fux440ux438ux440ux43eux434ux430-ux433ux440ux430ux432ux438ux442ux430ux446ux438ux43eux43dux43dux44bux445-ux438-ux438ux43dux435ux440ux446ux438ux43eux43dux43dux44bux445-ux432ux43eux43bux43d}{%
\subsection{Природа Гравитационных и Инерционных
Волн}\label{ux43fux440ux438ux440ux43eux434ux430-ux433ux440ux430ux432ux438ux442ux430ux446ux438ux43eux43dux43dux44bux445-ux438-ux438ux43dux435ux440ux446ux438ux43eux43dux43dux44bux445-ux432ux43eux43bux43d}}

Система уравнений, описывающих атмосферные процессы, позволяет выделить
два основных класса движений: медленные (длиннопериодные) и быстрые
(короткопериодные). Гравитационные и инерционные волны относятся именно
к классу \emph{быстрых (короткопериодных) движений}. Важно отметить, что
в рамках линейного приближения при таких движениях колебания полей
геопотенциала (\(\varphi\)) и потенциального вихря (\(\Pi\))
отсутствуют.

\hypertarget{ux438ux43dux435ux440ux446ux438ux43eux43dux43dux44bux435-ux432ux43eux43bux43dux44b}{%
\subsubsection{Инерционные
Волны}\label{ux438ux43dux435ux440ux446ux438ux43eux43dux43dux44bux435-ux432ux43eux43bux43dux44b}}

Инерционные волны, как следует из их названия, связаны с проявлением
инерции в движущемся потоке под действием силы Кориолиса. Если частица
воздуха выведена из состояния геострофического равновесия, сила
Кориолиса будет стремиться вернуть ее к этому равновесию, вызывая
осцилляции. Частота этих осцилляций связана с параметром Кориолиса
\(f = 2\omega\sin\varphi\).

\hypertarget{ux433ux440ux430ux432ux438ux442ux430ux446ux438ux43eux43dux43dux44bux435-ux432ux43eux43bux43dux44b}{%
\subsubsection{Гравитационные
Волны}\label{ux433ux440ux430ux432ux438ux442ux430ux446ux438ux43eux43dux43dux44bux435-ux432ux43eux43bux43dux44b}}

Гравитационные волны, или внутренние гравитационные волны, возникают в
стратифицированной атмосфере, где выталкивающая сила (архимедова сила)
выступает в качестве возвращающей силы при вертикальных смещениях
воздушных масс. Устойчивость атмосферы, определяемая вертикальным
градиентом температуры или потенциальной температуры, играет здесь
ключевую роль. Восходящие и нисходящие движения воздуха, вызванные
различными факторами, включая орографические препятствия или
турбулентность, могут приводить к генерации этих волн.

\hypertarget{ux43cux435ux442ux435ux43eux440ux43eux43bux43eux433ux438ux447ux435ux441ux43aux438ux435-ux448ux443ux43cux44b-ux438-ux438ux445-ux444ux438ux43bux44cux442ux440ux430ux446ux438ux44f}{%
\subsection{``Метеорологические Шумы'' и Их
Фильтрация}\label{ux43cux435ux442ux435ux43eux440ux43eux43bux43eux433ux438ux447ux435ux441ux43aux438ux435-ux448ux443ux43cux44b-ux438-ux438ux445-ux444ux438ux43bux44cux442ux440ux430ux446ux438ux44f}}

Эти быстрые, короткопериодные колебания, к которым относятся
гравитационные и инерционные волны, часто рассматриваются в численном
прогнозировании как ``метеорологические шумы''. Они представляют собой
мелкомасштабные возмущения, способные искажать крупномасштабные
процессы. В квазигеострофических прогностических моделях атмосферы эти
шумы намеренно фильтруются для обеспечения устойчивости расчетов и
акцентирования внимания на медленных, синоптических процессах.

Таким образом, в рамках геострофического потока, где доминирует
равновесие между силой барического градиента и силой Кориолиса,
гравитационные и инерционные волны представляют собой более быстрые
колебания, возникающие при нарушении этого равновесия.

ewpage

Уважаемый коллега,

Мы с Вами, безусловно, сталкиваемся с глубоким взаимопроникновением
различных атмосферных процессов, и вопросы адаптации полей ветра и
давления, а также роль внутренних гравитационных волн, являются
ключевыми для понимания динамики атмосферы. Рассмотрим эти аспекты с
опорой на имеющиеся данные.

\hypertarget{ux430ux434ux430ux43fux442ux430ux446ux438ux44f-ux43fux43eux43bux435ux439-ux432ux435ux442ux440ux430-ux438-ux434ux430ux432ux43bux435ux43dux438ux44f}{%
\subsection{Адаптация Полей Ветра и
Давления}\label{ux430ux434ux430ux43fux442ux430ux446ux438ux44f-ux43fux43eux43bux435ux439-ux432ux435ux442ux440ux430-ux438-ux434ux430ux432ux43bux435ux43dux438ux44f}}

Взаимная адаптация полей ветра и давления -- это фундаментальный процесс
в динамике атмосферы, обеспечивающий восстановление равновесия при его
нарушении.

\hypertarget{ux441ux443ux449ux43dux43eux441ux442ux44c-ux430ux434ux430ux43fux442ux430ux446ux438ux438}{%
\subsubsection{Сущность
Адаптации}\label{ux441ux443ux449ux43dux43eux441ux442ux44c-ux430ux434ux430ux43fux442ux430ux446ux438ux438}}

\begin{itemize}
\tightlist
\item
  При несоответствии полей давления и ветра, вызванном различными
  возмущениями барического поля (например, местными), происходит их
  взаимная адаптация.
\item
  В процессе этой адаптации поля изменяются таким образом, чтобы
  соответствие между ними вновь было восстановлено.
\item
  Важно отметить, что поле давления приспосабливается к полю ветра
  особенно быстро.
\item
  Благодаря подобной адаптации, в среднем в свободной атмосфере (т.е.
  выше пограничного слоя трения) действительный ветер оказывается
  близким к геострофическому. Это означает, что основные горизонтальные
  составляющие сил --- барического градиента и Кориолиса --- приходят в
  квазиравновесие.
\end{itemize}

\hypertarget{ux43fux440ux438ux447ux438ux43dux44b-ux43dux430ux440ux443ux448ux435ux43dux438ux44f-ux438-ux43fux440ux43eux44fux432ux43bux435ux43dux438ux44f-ux430ux434ux430ux43fux442ux430ux446ux438ux438}{%
\subsubsection{Причины Нарушения и Проявления
Адаптации}\label{ux43fux440ux438ux447ux438ux43dux44b-ux43dux430ux440ux443ux448ux435ux43dux438ux44f-ux438-ux43fux440ux43eux44fux432ux43bux435ux43dux438ux44f-ux430ux434ux430ux43fux442ux430ux446ux438ux438}}

\begin{itemize}
\tightlist
\item
  Неоднородность подстилающей поверхности является основной причиной
  возникновения мелких, локальных возмущений.
\item
  Поле ветра, особенно в слое трения, значительно сильнее подвержено
  влиянию сравнительно мелкомасштабных возмущений, чем поле давления.
\item
  Эти возмущения могут формировать так называемые местные системы ветра,
  такие как бризы и горно-долинные ветры.

  \begin{itemize}
  \tightlist
  \item
    \textbf{Бризы:} Возникают между сушей и водоемом в ясную погоду
    из-за разности температур, при этом днем ветер дует с водоема на
    сушу, а ночью -- с суши на водоем.
  \item
    \textbf{Горно-долинные ветры:} Подобны бризам, но развиваются между
    горными хребтами и долинами, с дневными долинными ветрами (вверх по
    склону) и ночными горными ветрами (вниз по склону).
  \end{itemize}
\item
  Поле давления и другие метеорологические величины (температура,
  влажность, ветер) определяются характером их пространственного
  распределения.
\item
  Синоптический анализ и объективный анализ, выполняемый с помощью ЭВМ,
  обязательно включают согласование полей давления и ветра, поскольку
  это является критически важным элементом.
\item
  Термический ветер, который является поправкой к геострофическому ветру
  и направлен вдоль изотерм, показывает расположение областей тепла и
  холода, а его изменение с высотой связано с адвекцией температуры. Это
  также часть сложного взаимодействия полей.
\item
  Бароклинность атмосферы, где плотность зависит не только от давления,
  но и от температуры, приводит к пересечению изобарических и
  изотермических поверхностей и, как следствие, к термической адвекции.
  Это также вызывает вихревые движения в горизонтальных и вертикальных
  плоскостях.
\end{itemize}

\hypertarget{ux432ux43bux438ux44fux43dux438ux435-ux43dux430-ux43fux440ux43eux433ux43dux43eux437}{%
\subsubsection{Влияние на
Прогноз}\label{ux432ux43bux438ux44fux43dux438ux435-ux43dux430-ux43fux440ux43eux433ux43dux43eux437}}

\begin{itemize}
\tightlist
\item
  При прогнозе ветра в приземном слое, его направление предсказывается с
  учетом отклонения от изобар (около 30° над сушей и 15° над морем), а
  скорость -- с учетом отклонения от геострофической скорости.
\item
  Графики зависимости скорости ветра от горизонтального градиента
  давления могут быть составлены для различных условий (секторы
  циклонов/антициклонов, стратификация).
\item
  Ветер над морем ближе к геострофическому, чем над сушей, из-за
  меньшего трения.
\end{itemize}

\hypertarget{ux432ux43dux443ux442ux440ux435ux43dux43dux438ux435-ux433ux440ux430ux432ux438ux442ux430ux446ux438ux43eux43dux43dux44bux435-ux432ux43eux43bux43dux44b}{%
\subsection{Внутренние Гравитационные
Волны}\label{ux432ux43dux443ux442ux440ux435ux43dux43dux438ux435-ux433ux440ux430ux432ux438ux442ux430ux446ux438ux43eux43dux43dux44bux435-ux432ux43eux43bux43dux44b}}

Хотя термин ``внутренние гравитационные волны'' напрямую не упоминается
в источниках в контексте их физики, концепция ``волновых движений''
(waves) и их влияния на атмосферные процессы широко обсуждается,
особенно в аспекте орографии и турбулентности, а также развития
синоптических систем.

\hypertarget{ux432ux43eux43bux43dux43eux432ux44bux435-ux434ux432ux438ux436ux435ux43dux438ux44f-ux432-ux430ux442ux43cux43eux441ux444ux435ux440ux435}{%
\subsubsection{Волновые Движения в
Атмосфере}\label{ux432ux43eux43bux43dux43eux432ux44bux435-ux434ux432ux438ux436ux435ux43dux438ux44f-ux432-ux430ux442ux43cux43eux441ux444ux435ux440ux435}}

\begin{itemize}
\tightlist
\item
  \textbf{Орографические волны:} Горы, будучи препятствием для воздушных
  течений, могут вызывать вертикальные движения воздуха. В источнике
  упоминается ``орографическая болтанка'', которая характеризуется
  периодичностью бросков самолета, что косвенно указывает на волновой
  характер возмущений, возникающих над горами. Эти волны могут
  распространяться на значительные высоты.
\item
  \textbf{Фронтальные волны:} Волновые движения играют ключевую роль в
  циклогенезе, где фронтальные волны на малоподвижных или холодных
  фронтах могут развиваться в циклоны. Устойчивость таких волн зависит
  от их длины (волны длиной 800-2800 км чаще неустойчивы).
\item
  \textbf{Турбулентность и болтанка:} Волновые движения могут вызывать
  турбулентность, приводящую к болтанке самолетов. Это особенно
  проявляется в облаках (например, кучевых и кучево-дождевых) и в зонах
  струйных течений. Волновая болтанка характеризуется периодичностью
  бросков.
\item
  \textbf{Вертикальные движения:} Волновые движения, наряду с
  турбулентностью, конвекцией, приземным трением и орографическими
  препятствиями, являются одним из типов вертикальных движений в
  атмосфере. Эти вертикальные движения определяют локальное изменение
  барического поля и другие атмосферные процессы.
\item
  \textbf{Энергия неустойчивости:} Состояние атмосферы (устойчивое,
  неустойчивое, безразличное) определяется вертикальным градиентом
  температуры. Когда вертикальное движение массы воздуха зависит только
  от начальной скорости, и вертикальное ускорение равно нулю, состояние
  атмосферы называется безразличным. Эти условия играют роль в развитии
  конвекции и, опосредованно, волновых процессов.
\item
  \textbf{Колебания погоды и климата:} Облака и вихри являются основой
  колебаний погоды и климата, и волновые движения, включая те, что
  связаны с циклонами, являются частью этого.
\end{itemize}

Хотя источники не углубляются в детальное описание механики внутренних
гравитационных волн, они признают существование волновых процессов как
важных факторов, влияющих на динамику атмосферы, от формирования
синоптических систем до локальных турбулентных явлений. Понимание
адаптации полей и волновых процессов позволяет нам более точно
диагностировать и прогнозировать атмосферные условия.

ewpage

Уважаемый коллега,

Рассматривая гидродинамическую неустойчивость зонального потока, мы
сталкиваемся с фундаментальными механизмами, определяющими
крупномасштабную динамику атмосферы. Этот вопрос распадается на два
ключевых случая: баротропный и бароклинный, каждый из которых обладает
своими специфическими особенностями и вносит вклад в формирование
погодных явлений.

\hypertarget{ux441ux443ux449ux43dux43eux441ux442ux44c-ux433ux438ux434ux440ux43eux434ux438ux43dux430ux43cux438ux447ux435ux441ux43aux43eux439-ux43dux435ux443ux441ux442ux43eux439ux447ux438ux432ux43eux441ux442ux438-ux437ux43eux43dux430ux43bux44cux43dux43eux433ux43e-ux43fux43eux442ux43eux43aux430}{%
\subsubsection{Сущность Гидродинамической Неустойчивости Зонального
Потока}\label{ux441ux443ux449ux43dux43eux441ux442ux44c-ux433ux438ux434ux440ux43eux434ux438ux43dux430ux43cux438ux447ux435ux441ux43aux43eux439-ux43dux435ux443ux441ux442ux43eux439ux447ux438ux432ux43eux441ux442ux438-ux437ux43eux43dux430ux43bux44cux43dux43eux433ux43e-ux43fux43eux442ux43eux43aux430}}

Гидродинамическая неустойчивость в контексте зонального потока --- это
процесс, при котором относительно устойчивое горизонтальное течение
(зональный поток) теряет свою стабильность под влиянием различных
факторов, что приводит к развитию волновых возмущений и, в конечном
итоге, к формированию крупномасштабных вихрей, таких как циклоны и
антициклоны. Этот процесс тесно связан с изменением барических и
ветровых полей.

\hypertarget{ux431ux430ux440ux43eux442ux440ux43eux43fux43dux44bux439-ux441ux43bux443ux447ux430ux439}{%
\subsubsection{Баротропный
Случай}\label{ux431ux430ux440ux43eux442ux440ux43eux43fux43dux44bux439-ux441ux43bux443ux447ux430ux439}}

В \textbf{баротропной модели} атмосферы предполагается, что плотность
воздуха является функцией только давления (ρ = ρ(P)), либо только
температуры (ρ = ρ(T)). Это означает, что изопикнические (постоянной
плотности), изобарические (постоянного давления) и изотермические
(постоянной температуры) поверхности параллельны друг другу и совпадают.
Закономерности движения баротропной среды полностью определяются
уравнениями движения и неразрывности.

Хотя реальная атмосфера практически всегда бароклинна, концепции,
близкие к баротропной неустойчивости, проявляются в некоторых аспектах:

\begin{itemize}
\tightlist
\item
  \textbf{Волны Россби}: Взаимодействие зонального потока с эффектом β
  (изменение параметра Кориолиса с широтой) может приводить к генерации
  и развитию длинных волн Россби. В упрощенных моделях, когда не
  учитывается вертикальная структура или бароклинные эффекты, рост
  амплитуды этих волн за счет энергии зонального потока представляет
  собой проявление баротропной неустойчивости. Эта неустойчивость, как
  правило, связана с наличием горизонтальных сдвигов скорости в
  зональном потоке.
\item
  \textbf{Применимость}: В оперативной практике баротропные модели ранее
  широко применялись для прогнозов абсолютной топографии на средних
  уровнях (например, АТ700 и АТ500), поскольку на этих высотах (3-5 км)
  термическая адвекция часто близка к нулю, и атмосферу можно в первом
  приближении считать баротропной. Однако в настоящее время они в
  основном заменены более совершенными бароклинными моделями.
\end{itemize}

\hypertarget{ux431ux430ux440ux43eux43aux43bux438ux43dux43dux44bux439-ux441ux43bux443ux447ux430ux439}{%
\subsubsection{Бароклинный
Случай}\label{ux431ux430ux440ux43eux43aux43bux438ux43dux43dux44bux439-ux441ux43bux443ux447ux430ux439}}

\textbf{Бароклинность} --- это более реалистичное свойство атмосферы,
при котором плотность воздуха зависит не только от давления, но и от
температуры. Это приводит к тому, что изобарические и изотермические
поверхности пересекаются. Именно бароклинность играет определяющую роль
в зарождении и эволюции синоптических вихрей и атмосферных фронтов.

Механизмы бароклинной неустойчивости зонального потока значительно
сложнее и включают следующее:

\begin{itemize}
\tightlist
\item
  \textbf{Горизонтальные Градиенты Температуры}: Неравномерное
  распределение радиационного баланса по Земле и неоднородность
  теплофизических свойств подстилающей поверхности приводят к
  формированию горизонтальных градиентов температуры между низкими и
  высокими широтами. Эти градиенты являются первопричиной
  крупномасштабных атмосферных движений и бароклинной неустойчивости.
\item
  \textbf{Термическая Адвекция}: В бароклинной атмосфере пересечение
  изобарических и изотермических поверхностей в сочетании с движением
  воздуха приводит к термической адвекции.

  \begin{itemize}
  \tightlist
  \item
    \textbf{Адвекция холода} (перенос холодного воздуха) способствует
    циклогенезу, т.е. углублению циклонов и их зарождению, особенно в
    районах восточных побережий материков зимой. Интенсивная адвекция
    холода в тылу циклона приводит к увеличению контраста температур и
    его углублению.
  \item
    \textbf{Адвекция тепла} (перенос теплого воздуха) способствует
    антициклогенезу.
  \end{itemize}
\item
  \textbf{Высотные Фронтальные Зоны (ВФЗ) и Струйные Течения}: ВФЗ
  являются областями с большими горизонтальными градиентами температуры
  и давления. Струйные течения, расположенные в пределах ВФЗ, также
  тесно связаны с процессами цикло- и антициклогенеза. Развивающийся
  циклон часто зарождается на антициклонической стороне струйного
  течения. Волны в планетарной ВФЗ, возникающие из-за изменения
  параметра Кориолиса с широтой, могут приводить к сходимости воздушных
  течений, обострению горизонтальных градиентов и формированию ВФЗ.
\item
  \textbf{Наклон Осей Барических Образований}: Отличительной чертой
  бароклинной неустойчивости является наклон высотной оси циклона в
  сторону очага холода, а антициклона --- в сторону очага тепла. Этот
  наклон позволяет реализовать высвобождение потенциальной энергии в
  кинетическую.
\item
  \textbf{Фронтогенез}: Бароклинность приводит к формированию и
  обострению атмосферных фронтов --- узких переходных зон между
  воздушными массами с резкими изменениями метеорологических величин.
  Именно во фронтальных зонах синоптические вихри зарождаются и
  претерпевают наибольшие изменения.
\item
  \textbf{Влажно-неустойчивая Стратификация}: При влажно-неустойчивой
  стратификации атмосферы (где вертикальный градиент температуры
  превышает влажно-адиабатический градиент, γ \textgreater{} γwa)
  создаются благоприятные условия для развития циклонов. Этот фактор
  также играет роль в формировании ливневых осадков и гроз.
\end{itemize}

\hypertarget{ux432ux437ux430ux438ux43cux43eux441ux432ux44fux437ux44c-ux438-ux437ux43dux430ux447ux435ux43dux438ux435}{%
\subsubsection{Взаимосвязь и
Значение}\label{ux432ux437ux430ux438ux43cux43eux441ux432ux44fux437ux44c-ux438-ux437ux43dux430ux447ux435ux43dux438ux435}}

В современной синоптической метеорологии именно \textbf{бароклинная
модель} атмосферы является основной для описания крупномасштабных
процессов. Она позволяет учитывать сложную вертикальную структуру
атмосферы, взаимодействие полей температуры и давления, что дает более
точное описание реальных атмосферных процессов по сравнению с
баротропной моделью. Понимание бароклинной неустойчивости критически
важно для прогнозирования зарождения, развития и перемещения циклонов,
антициклонов и фронтальных систем, которые являются основными объектами,
определяющими погоду на значительных территориях.

ewpage

Уважаемый коллега,

Рассмотрим подробно уравнение энергии и механизмы перехода одних видов
энергии в другие, что является фундаментальным аспектом динамики
атмосферы.

\hypertarget{ux437ux430ux43aux43eux43d-ux441ux43eux445ux440ux430ux43dux435ux43dux438ux44f-ux44dux43dux435ux440ux433ux438ux438-ux438-ux43fux435ux440ux432ux43eux435-ux43dux430ux447ux430ux43bux43e-ux442ux435ux440ux43cux43eux434ux438ux43dux430ux43cux438ux43aux438}{%
\subsubsection{Закон Сохранения Энергии и Первое Начало
Термодинамики}\label{ux437ux430ux43aux43eux43d-ux441ux43eux445ux440ux430ux43dux435ux43dux438ux44f-ux44dux43dux435ux440ux433ux438ux438-ux438-ux43fux435ux440ux432ux43eux435-ux43dux430ux447ux430ux43bux43e-ux442ux435ux440ux43cux43eux434ux438ux43dux430ux43cux438ux43aux438}}

В основе всех энергетических процессов в атмосфере лежит общий закон
сохранения энергии, который утверждает, что энергия не исчезает и не
возникает из ничего, а лишь трансформируется из одной формы в другую.
Первое начало термодинамики является экспериментально установленным
выражением этого закона применительно к термодинамическим процессам.

Согласно первому началу термодинамики, тепло, подведенное к единице
массы воздуха (\texttt{dQ}), расходуется на увеличение внутренней
тепловой энергии (\texttt{dJ}) и на работу (\texttt{dE}), которую
совершает воздух, преодолевая давление. В математической форме это
выражается как \texttt{dQ\ =\ dJ\ +\ dE}. Если рассматривать единицу
массы воздуха, занимающую удельный объем \texttt{v} и имеющую
температуру \texttt{T}, то после сообщения ей тепла \texttt{dQ},
температура повысится на \texttt{dT}, внутренняя энергия увеличится на
\texttt{dJ}, и воздух, расширяясь, совершит работу \texttt{Pdv}. Таким
образом, одно из выражений первого начала термодинамики:
\texttt{dQ\ =\ C\_v\ dT\ +\ Pdv}.

В метеорологии часто используется форма, выражающая приток тепла через
измеряемые величины: \texttt{dQ\ =\ C\_p\ dT\ -\ (RT/P)dP}. Здесь
\texttt{C\_p} - удельная теплоемкость при постоянном давлении, а
\texttt{R} - удельная газовая постоянная.

\hypertarget{ux443ux440ux430ux432ux43dux435ux43dux438ux435-ux43fux440ux438ux442ux43eux43aux430-ux442ux435ux43fux43bux430-1}{%
\subsubsection{Уравнение Притока
Тепла}\label{ux443ux440ux430ux432ux43dux435ux43dux438ux435-ux43fux440ux438ux442ux43eux43aux430-ux442ux435ux43fux43bux430-1}}

Уравнение притока тепла является расширенным выражением закона
сохранения энергии, описывающим изменение количества тепловой энергии в
единице объема воздуха за единицу времени при его движении. В общем
виде, оно показывает, что изменение внутренней и кинетической энергии
воздуха равно работе всех массовых и поверхностных сил, сложенной с
притоком тепла.

Исключая работу вязкости и пренебрегая пульсациями плотности воздуха,
уравнение притока тепла (или первое начало термодинамики для единицы
массы) может быть записано как:
\texttt{d(C\_v\ T)/dt\ =\ Q\ -\ V\ *\ div(P)} или
\texttt{C\_v\ *\ dT/dt\ =\ Q\ -\ (P/ρ)\ *\ (dρ/dt)}. Здесь \texttt{Q} --
количество тепла, сообщаемое извне единице массы воздуха в единицу
времени. Второй член правой части \texttt{(P/ρ)\ *\ (dρ/dt)} отражает
тепло, идущее на работу расширения или выделяющееся при сжатии.

Для турбулентной атмосферы уравнение притока тепла включает турбулентные
потоки тепла, выраженные через осредненные пульсации скорости и
потенциальной температуры.

\hypertarget{ux43eux441ux43dux43eux432ux43dux44bux435-ux432ux438ux434ux44b-ux44dux43dux435ux440ux433ux438ux438-ux432-ux430ux442ux43cux43eux441ux444ux435ux440ux435}{%
\subsubsection{Основные Виды Энергии в
Атмосфере}\label{ux43eux441ux43dux43eux432ux43dux44bux435-ux432ux438ux434ux44b-ux44dux43dux435ux440ux433ux438ux438-ux432-ux430ux442ux43cux43eux441ux444ux435ux440ux435}}

В атмосфере присутствуют и преобразуются следующие основные виды
энергии:

\begin{itemize}
\tightlist
\item
  \textbf{Внутренняя тепловая энергия:} Связана с температурой воздуха.
\item
  \textbf{Потенциальная энергия:} Определяется высотой столба атмосферы.
\item
  \textbf{Кинетическая энергия:} Связана с движением воздушных масс
  (ветрами).
\end{itemize}

Количественные оценки показывают, что запасы внутренней энергии и
потенциальной энергии значительно превышают запасы кинетической энергии
в атмосфере. Например, над северным полушарием в январе потенциальная
энергия составляет около 1.88·10\^{}23 Дж, внутренняя --- 4.69·10\^{}23
Дж, тогда как кинетическая --- всего 4.36·10\^{}20 Дж.

\hypertarget{ux43fux440ux435ux43eux431ux440ux430ux437ux43eux432ux430ux43dux438ux44f-ux44dux43dux435ux440ux433ux438ux438-ux432-ux430ux442ux43cux43eux441ux444ux435ux440ux435}{%
\subsubsection{Преобразования Энергии в
Атмосфере}\label{ux43fux440ux435ux43eux431ux440ux430ux437ux43eux432ux430ux43dux438ux44f-ux44dux43dux435ux440ux433ux438ux438-ux432-ux430ux442ux43cux43eux441ux444ux435ux440ux435}}

Атмосфера функционирует как гигантская тепловая машина, приводимая в
действие лучистой энергией Солнца. В ней постоянно происходят
преобразования одного вида энергии в другой.

\begin{enumerate}
\def\labelenumi{\arabic{enumi}.}
\tightlist
\item
  \textbf{Лучистая энергия в тепловую:}

  \begin{itemize}
  \tightlist
  \item
    Основным источником лучистой энергии для атмосферы является Солнце,
    а также излучение Земли и самой атмосферы.
  \item
    Солнечная радиация, проходя через атмосферу, частично поглощается
    водяным паром, углекислым газом и другими компонентами воздуха,
    превращаясь в тепловую энергию.
  \item
    Поглощенная солнечная радиация нагревает земную поверхность, которая
    затем излучает в инфракрасном диапазоне. Инфракрасное излучение
    сильно поглощается атмосферой, нагревая ее.
  \item
    Процесс формирования теплового излучения на молекулярном уровне
    связан с квантовыми переходами атомов из неустойчивых состояний в
    устойчивые.
  \end{itemize}
\item
  \textbf{Потенциальная энергия в кинетическую и наоборот:}

  \begin{itemize}
  \tightlist
  \item
    При подъеме воздуха в столбе атмосферы увеличивается его
    потенциальная энергия. Часть этой энергии впоследствии расходуется
    на поддержание атмосферных течений, то есть ветров.
  \item
    Нисходящие движения воздуха в антициклонах связаны с переходом
    потенциальной энергии в кинетическую, что способствует формированию
    инверсий температуры.
  \end{itemize}
\item
  \textbf{Кинетическая энергия в тепловую (диссипация):}

  \begin{itemize}
  \tightlist
  \item
    Трение воздуха о земную поверхность всегда уменьшает скорость
    воздушных течений и изменяет их направление. Этот процесс происходит
    в пограничном слое атмосферы (до высоты 1.0--1.5 км) и приводит к
    диссипации (рассеянию) кинетической энергии, превращению ее в
    тепловую и другие виды энергии. Диссипация не приводит к прекращению
    циркуляции атмосферы, так как существуют постоянные источники,
    стимулирующие атмосферные движения.
  \end{itemize}
\item
  \textbf{Скрытая теплота конденсации в явную:}

  \begin{itemize}
  \tightlist
  \item
    Водяной пар переносится воздушными потоками и конденсируется на
    различных высотах и в различных широтных зонах. При этом выделяется
    огромное количество скрытой теплоты, которая преобразуется в явное
    тепло, повышая температуру воздуха. Это является одним из важнейших
    факторов, влияющих на атмосферное давление и циркуляцию.
  \end{itemize}
\end{enumerate}

\hypertarget{ux44dux43dux435ux440ux433ux435ux442ux438ux447ux435ux441ux43aux438ux439-ux431ux430ux43bux430ux43dux441-ux430ux442ux43cux43eux441ux444ux435ux440ux43dux43eux439-ux441ux438ux441ux442ux435ux43cux44b}{%
\subsubsection{Энергетический Баланс Атмосферной
Системы}\label{ux44dux43dux435ux440ux433ux435ux442ux438ux447ux435ux441ux43aux438ux439-ux431ux430ux43bux430ux43dux441-ux430ux442ux43cux43eux441ux444ux435ux440ux43dux43eux439-ux441ux438ux441ux442ux435ux43cux44b}}

Энергетический баланс охватывает взаимодействие между Землей, ее
поверхностью и атмосферой.

\begin{enumerate}
\def\labelenumi{\arabic{enumi}.}
\tightlist
\item
  \textbf{Земля как планета:} В целом находится в состоянии
  радиационного равновесия: поток приходящей солнечной радиации
  уравновешивается потоком отраженной коротковолновой и уходящей
  длинноволновой радиации.
\item
  \textbf{Поверхность Земли:} Не находится в состоянии радиационного
  равновесия. Эффективное излучение лишь частично компенсирует приток
  поглощенной радиации. Избыточная энергия расходуется на испарение и
  контактную теплопередачу в атмосферу.
\item
  \textbf{Атмосфера:} Также не находится в состоянии радиационного
  равновесия. Поток энергии в атмосфере больше, чем радиационные потоки,
  что означает наличие запаса ``мощности'', позволяющего атмосфере
  совершать работу по переносу воздушных масс.
\end{enumerate}

Радиационный баланс имеет суточный и годовой ход. Днем он обычно
положителен за счет притока солнечной радиации, ночью --- отрицателен
из-за эффективного излучения. Наибольшие значения радиационного баланса
приходятся на океаны тропической и субтропической зон, что объясняется
распределением облачности и различием альбедо, делая тропический океан
ключевым поставщиком энергии для атмосферных процессов.

Сила градиента давления приводит воздух в движение, порождая ветер. На
движущийся воздух также действуют сила Кориолиса, центробежная сила, а в
нижних слоях --- сила трения. Сочетание этих сил приводит к вихревым
движениям (циклоническим и антициклоническим), а также к перемещению
воздушных масс, перераспределяющих полученное от Солнца тепло.

Понимание этих энергетических преобразований и балансов является
краеугольным камнем для диагностики и прогнозирования атмосферных
процессов, включая формирование циклонов, антициклонов, фронтов и
развитие погодных явлений.

ewpage

Коллега,

Рассмотрим энергетический аспект общей циркуляции атмосферы, уделяя
внимание кинетической и доступной потенциальной энергии.

\hypertarget{ux44dux43dux435ux440ux433ux435ux442ux438ux43aux430-ux43eux431ux449ux435ux439-ux446ux438ux440ux43aux443ux43bux44fux446ux438ux438-ux430ux442ux43cux43eux441ux444ux435ux440ux44b}{%
\section{Энергетика Общей Циркуляции
Атмосферы}\label{ux44dux43dux435ux440ux433ux435ux442ux438ux43aux430-ux43eux431ux449ux435ux439-ux446ux438ux440ux43aux443ux43bux44fux446ux438ux438-ux430ux442ux43cux43eux441ux444ux435ux440ux44b}}

\hypertarget{ux43eux431ux449ux430ux44f-ux446ux438ux440ux43aux443ux43bux44fux446ux438ux44f-ux438-ux438ux441ux442ux43eux447ux43dux438ux43aux438-ux44dux43dux435ux440ux433ux438ux438}{%
\subsection{Общая Циркуляция и Источники
Энергии}\label{ux43eux431ux449ux430ux44f-ux446ux438ux440ux43aux443ux43bux44fux446ux438ux44f-ux438-ux438ux441ux442ux43eux447ux43dux438ux43aux438-ux44dux43dux435ux440ux433ux438ux438}}

Общая циркуляция атмосферы представляет собой крупномасштабную систему
воздушных течений, охватывающую весь земной шар. Ее существование
обусловлено комплексом факторов, среди которых ключевую роль играют
лучистая энергия Солнца, вращение Земли вокруг своей оси, неоднородность
подстилающей поверхности и трение.

Первопричиной атмосферных движений является неравномерное зональное
распределение радиационного баланса, возникающее из-за сферичности
Земли. Следствием этого является то, что низкие широты получают
значительно больше солнечной энергии, чем умеренные и полярные. Эта
разница в нагреве приводит к возникновению барических градиентов,
которые, в свою очередь, стимулируют атмосферные движения. Атмосферная
циркуляция выполняет функцию переноса тепла от экватора к полюсам,
сглаживая температурные различия. В этом смысле атмосфера может быть
рассмотрена как гигантская тепловая машина, приводимая в действие
лучистой энергией Солнца.

\hypertarget{ux43fux43eux442ux435ux43dux446ux438ux430ux43bux44cux43dux430ux44f-ux44dux43dux435ux440ux433ux438ux44f-ux438-ux435ux435-ux43dux430ux43aux43eux43fux43bux435ux43dux438ux435}{%
\subsection{Потенциальная Энергия и Ее
Накопление}\label{ux43fux43eux442ux435ux43dux446ux438ux430ux43bux44cux43dux430ux44f-ux44dux43dux435ux440ux433ux438ux44f-ux438-ux435ux435-ux43dux430ux43aux43eux43fux43bux435ux43dux438ux435}}

В процессе нагревания атмосферного столба происходит увеличение его
потенциальной энергии за счет работы расширения, совершаемой силой
давления. Атмосфера в целом не находится в состоянии радиационного
равновесия; приток энергии в ней превышает радиационные потери, и этот
``запас мощности'' обеспечивает возможность выполнения работы по
переносу воздушных масс.

\hypertarget{ux43fux440ux435ux43eux431ux440ux430ux437ux43eux432ux430ux43dux438ux435-ux432-ux43aux438ux43dux435ux442ux438ux447ux435ux441ux43aux443ux44e-ux44dux43dux435ux440ux433ux438ux44e}{%
\subsection{Преобразование в Кинетическую
Энергию}\label{ux43fux440ux435ux43eux431ux440ux430ux437ux43eux432ux430ux43dux438ux435-ux432-ux43aux438ux43dux435ux442ux438ux447ux435ux441ux43aux443ux44e-ux44dux43dux435ux440ux433ux438ux44e}}

Накопленная потенциальная энергия впоследствии частично трансформируется
в кинетическую энергию, поддерживая атмосферные течения, то есть ветры.
Это преобразование является фундаментальным аспектом общей циркуляции,
где тепловая энергия превращается в механическую работу движения
воздушных масс.

\hypertarget{ux43aux438ux43dux435ux442ux438ux447ux435ux441ux43aux430ux44f-ux44dux43dux435ux440ux433ux438ux44f-ux438-ux435ux435-ux440ux430ux441ux43fux440ux435ux434ux435ux43bux435ux43dux438ux435}{%
\subsection{Кинетическая Энергия и Ее
Распределение}\label{ux43aux438ux43dux435ux442ux438ux447ux435ux441ux43aux430ux44f-ux44dux43dux435ux440ux433ux438ux44f-ux438-ux435ux435-ux440ux430ux441ux43fux440ux435ux434ux435ux43bux435ux43dux438ux435}}

Крупномасштабные атмосферные движения, такие как струйные течения и
циркуляция в циклонах и антициклонах, обладают значительными запасами
кинетической энергии. Высотные фронтальные зоны (ВФЗ) представляют собой
области концентрации огромных объемов энергии, характеризующиеся
большими горизонтальными градиентами давления и температуры. ВФЗ
обладают значительными запасами как кинетической, так и внутренней
энергии, и именно в них происходят интенсивные преобразования одного
вида энергии в другой. Распределение кинетической энергии в общей
циркуляции атмосферы неравномерно как по горизонтали, так и по
вертикали.

\hypertarget{ux434ux438ux441ux441ux438ux43fux430ux446ux438ux44f-ux44dux43dux435ux440ux433ux438ux438}{%
\subsection{Диссипация
Энергии}\label{ux434ux438ux441ux441ux438ux43fux430ux446ux438ux44f-ux44dux43dux435ux440ux433ux438ux438}}

Несмотря на постоянное поступление энергии, происходит непрерывное
рассеяние кинетической энергии. Действие приземного трения приводит к
диссипации кинетической энергии, превращая ее в тепловую и другие формы
энергии. Турбулентность, проявляющаяся в беспорядочном движении вихрей,
также способствует рассеянию энергии через турбулентную вязкость,
диффузию и теплопроводность. Однако это рассеяние не ведет к прекращению
атмосферной циркуляции, поскольку постоянно действуют факторы,
стимулирующие новые атмосферные движения.

ewpage

\hypertarget{ux447ux438ux441ux43bux435ux43dux43dux44bux439-ux430ux43dux430ux43bux438ux437-ux441ux438ux43dux445ux440ux43eux43dux43dux44bux445-ux43cux435ux442ux435ux43eux440ux43eux43bux43eux433ux438ux447ux435ux441ux43aux438ux445-ux43fux43eux43bux435ux439}{%
\section{Численный Анализ Синхронных Метеорологических
Полей}\label{ux447ux438ux441ux43bux435ux43dux43dux44bux439-ux430ux43dux430ux43bux438ux437-ux441ux438ux43dux445ux440ux43eux43dux43dux44bux445-ux43cux435ux442ux435ux43eux440ux43eux43bux43eux433ux438ux447ux435ux441ux43aux438ux445-ux43fux43eux43bux435ux439}}

Численный анализ метеорологических полей представляет собой
фундаментальный этап в оперативной работе прогностических центров и
неотъемлемую часть комплекса автоматизированной обработки
метеорологической информации. Основная задача этого этапа ---
преобразование дискретных данных наблюдений, полученных с
метеорологических и аэрологических станций, в значения метеорологических
величин в узлах регулярной расчетной сетки. Это необходимо для
последующего использования данных в численных прогностических схемах.

В настоящее время объективный анализ полей различных метеорологических
величин, включая высоты основных изобарических поверхностей,
составляющие ветра, точку росы, давление воздуха на уровне моря и
барическую тенденцию, осуществляется с использованием нескольких
распространенных методов.

\hypertarget{ux43cux435ux442ux43eux434ux44b-ux43eux431ux44aux435ux43aux442ux438ux432ux43dux43eux433ux43e-ux430ux43dux430ux43bux438ux437ux430}{%
\subsection{Методы Объективного
Анализа}\label{ux43cux435ux442ux43eux434ux44b-ux43eux431ux44aux435ux43aux442ux438ux432ux43dux43eux433ux43e-ux430ux43dux430ux43bux438ux437ux430}}

\hypertarget{ux43cux435ux442ux43eux434-ux43fux43eux43bux438ux43dux43eux43cux438ux430ux43bux44cux43dux43eux439-ux438ux43dux442ux435ux440ux43fux43eux43bux44fux446ux438ux438}{%
\subsubsection{1. Метод Полиномиальной
Интерполяции}\label{ux43cux435ux442ux43eux434-ux43fux43eux43bux438ux43dux43eux43cux438ux430ux43bux44cux43dux43eux439-ux438ux43dux442ux435ux440ux43fux43eux43bux44fux446ux438ux438}}

В методе полиномиальной интерполяции поле метеорологической величины в
районе узла сетки аппроксимируется алгебраическим полиномом. Обычно для
этой цели используется полином второй, реже третьей степени.
Коэффициенты аппроксимирующего полинома вычисляются по значениям
метеорологической величины на станциях, окружающих данный узел. Расчет
коэффициентов производится с помощью метода наименьших квадратов, что
обеспечивает наилучшее приближение аппроксимированного поля к
фактическим значениям на станциях. Цикл этих операций повторяется для
всех узлов сетки.

\hypertarget{ux43cux435ux442ux43eux434-ux43fux43eux441ux43bux435ux434ux43eux432ux430ux442ux435ux43bux44cux43dux44bux445-ux443ux442ux43eux447ux43dux435ux43dux438ux439-ux43fux43eux441ux43bux435ux434ux43eux432ux430ux442ux435ux43bux44cux43dux44bux445-ux43aux43eux440ux440ux435ux43aux446ux438ux439}{%
\subsubsection{2. Метод Последовательных Уточнений (Последовательных
Коррекций)}\label{ux43cux435ux442ux43eux434-ux43fux43eux441ux43bux435ux434ux43eux432ux430ux442ux435ux43bux44cux43dux44bux445-ux443ux442ux43eux447ux43dux435ux43dux438ux439-ux43fux43eux441ux43bux435ux434ux43eux432ux430ux442ux435ux43bux44cux43dux44bux445-ux43aux43eux440ux440ux435ux43aux446ux438ux439}}

Метод последовательных уточнений предусматривает итерационный порядок
расчетов.

\hypertarget{ux43fux440ux435ux434ux432ux430ux440ux438ux442ux435ux43bux44cux43dux43eux435-ux43fux43eux43bux435}{%
\paragraph{Предварительное
Поле}\label{ux43fux440ux435ux434ux432ux430ux440ux438ux442ux435ux43bux44cux43dux43eux435-ux43fux43eux43bux435}}

Изначально в узлы регулярной сетки записываются ``предварительные''
значения метеорологической величины. Это может быть:

\begin{itemize}
\tightlist
\item
  Ожидаемые по прогнозу значения.
\item
  Климатические нормы.
\item
  Сочетания климатических и прогностических значений.
\end{itemize}

\hypertarget{ux438ux442ux435ux440ux430ux446ux438ux43eux43dux43dux44bux439-ux43fux440ux43eux446ux435ux441ux441}{%
\paragraph{Итерационный
Процесс}\label{ux438ux442ux435ux440ux430ux446ux438ux43eux43dux43dux44bux439-ux43fux440ux43eux446ux435ux441ux441}}

Созданное таким образом предварительное поле затем последовательно
исправляется (корректируется) фактическими данными на станциях. Процесс
включает следующие шаги:

\begin{enumerate}
\def\labelenumi{\arabic{enumi}.}
\tightlist
\item
  Значения метеорологической величины, записанные в узлах сетки,
  интерполируются обратно на станции.
\item
  Находятся ``невязки'' --- разности между интерполированными и
  фактическими значениями на станциях.
\item
  По рассчитанным невязкам определяются поправки, которые вносятся в
  предварительное поле для его согласования с полем фактических
  значений.
\item
  Подправленные значения в узлах сетки снова интерполируются на станции,
  находятся новые невязки, и процесс коррекции повторяется.
\end{enumerate}

Этот итерационный цикл повторяется многократно, пока поле в узлах
регулярной сетки не приводится в соответствие с полем фактических данных
на станциях.

\hypertarget{ux43cux435ux442ux43eux434-ux43eux43fux442ux438ux43cux430ux43bux44cux43dux43eux439-ux438ux43dux442ux435ux440ux43fux43eux43bux44fux446ux438ux438}{%
\subsubsection{3. Метод Оптимальной
Интерполяции}\label{ux43cux435ux442ux43eux434-ux43eux43fux442ux438ux43cux430ux43bux44cux43dux43eux439-ux438ux43dux442ux435ux440ux43fux43eux43bux44fux446ux438ux438}}

Метод оптимальной интерполяции является статистически наиболее
совершенным.

\hypertarget{ux43fux440ux438ux43dux446ux438ux43f-ux440ux430ux441ux447ux435ux442ux430}{%
\paragraph{Принцип
Расчета}\label{ux43fux440ux438ux43dux446ux438ux43f-ux440ux430ux441ux447ux435ux442ux430}}

Значение метеорологической величины в каждом узле регулярной сетки
получается путем сложения климатической нормы анализируемой
метеорологической величины в этом узле с аномалией (отклонением от
нормы).

\hypertarget{ux440ux430ux441ux447ux435ux442-ux430ux43dux43eux43cux430ux43bux438ux438}{%
\paragraph{Расчет
Аномалии}\label{ux440ux430ux441ux447ux435ux442-ux430ux43dux43eux43cux430ux43bux438ux438}}

Аномалия в узле рассчитывается по фактическим значениям аномалий на
окружающих станциях. Для этого используются определенные ``весовые
множители''.

\hypertarget{ux432ux435ux441ux43eux432ux44bux435-ux43cux43dux43eux436ux438ux442ux435ux43bux438}{%
\paragraph{Весовые
Множители}\label{ux432ux435ux441ux43eux432ux44bux435-ux43cux43dux43eux436ux438ux442ux435ux43bux438}}

Весовые множители зависят от взаимного расположения станций и узла:

\begin{itemize}
\tightlist
\item
  Они уменьшаются с удалением станции от узла.
\item
  Они также уменьшаются при возрастании плотности близлежащих станций,
  привлекаемых к интерполяции.
\item
  Эти множители подбираются таким образом, чтобы минимизировать ошибку
  интерполяции в статистическом смысле. Для их нахождения используется
  пространственная автокорреляционная функция анализируемой
  метеорологической величины, которая характеризует статистическую связь
  между значениями аномалии в двух точках в зависимости от расстояния
  между ними.
\end{itemize}

\hypertarget{ux434ux430ux43bux44cux43dux435ux439ux448ux435ux435-ux440ux430ux437ux432ux438ux442ux438ux435-ux43cux435ux442ux43eux434ux43eux432}{%
\subsection{Дальнейшее Развитие
Методов}\label{ux434ux430ux43bux44cux43dux435ux439ux448ux435ux435-ux440ux430ux437ux432ux438ux442ux438ux435-ux43cux435ux442ux43eux434ux43eux432}}

С развитием спутниковой метеорологии активно разрабатываются и
внедряются в оперативную работу методы численного анализа и усвоения
\emph{несинхронной} метеорологической информации. В отличие от
классических методов, которые преимущественно работают с синхронными
наземными и аэрологическими данными, эти новые подходы призваны
проводить совместный четырехмерный анализ, учитывающий различия по трем
пространственным координатам и времени наблюдения.

ewpage

Коллега,

Вопрос о согласовании начальных данных для прогностических моделей и
концепции четырехмерного усвоения данных затрагивает фундаментальные
аспекты численного метеорологического прогнозирования. Хотя термин
``четырехмерное усвоение данных'' в явном виде не встречается в
предоставленных источниках, в них подробно обсуждаются методы и подходы
к подготовке начальных полей, их согласованию, а также задачи, которые
решаются современными системами, что, по сути, является частью более
широкой концепции усвоения данных.

\hypertarget{ux441ux43eux433ux43bux430ux441ux43eux432ux430ux43dux438ux435-ux43dux430ux447ux430ux43bux44cux43dux44bux445-ux434ux430ux43dux43dux44bux445-ux434ux43bux44f-ux43fux440ux43eux433ux43dux43eux441ux442ux438ux447ux435ux441ux43aux438ux445-ux43cux43eux434ux435ux43bux435ux439}{%
\section{Согласование Начальных Данных для Прогностических
Моделей}\label{ux441ux43eux433ux43bux430ux441ux43eux432ux430ux43dux438ux435-ux43dux430ux447ux430ux43bux44cux43dux44bux445-ux434ux430ux43dux43dux44bux445-ux434ux43bux44f-ux43fux440ux43eux433ux43dux43eux441ux442ux438ux447ux435ux441ux43aux438ux445-ux43cux43eux434ux435ux43bux435ux439}}

\hypertarget{ux43eux441ux43dux43eux432ux44b-ux43fux43eux441ux442ux430ux43dux43eux432ux43aux438-ux437ux430ux434ux430ux447ux438}{%
\subsection{Основы Постановки
Задачи}\label{ux43eux441ux43dux43eux432ux44b-ux43fux43eux441ux442ux430ux43dux43eux432ux43aux438-ux437ux430ux434ux430ux447ux438}}

Для решения задач динамической метеорологии, будь то определение
пространственного распределения метеорологических величин в данный
момент времени или прогноз погоды, необходимо построить и решить
замкнутую систему уравнений гидротермодинамики атмосферы. Эта система,
как правило, представляет собой нелинейные дифференциальные уравнения в
частных производных, которые не имеют точных аналитических решений и
поэтому решаются численными методами на высокопроизводительных
компьютерах.

Ключевым требованием для запуска такой системы является задание
\textbf{начальных и граничных условий}. Начальные условия подразумевают,
что для определенного исходного момента времени (\(t=0\)) должны быть
известны значения всех искомых функций (например, трех проекций скорости
\(u, v, w\), давления \(P\) и плотности воздуха \(\rho\)) во всех точках
рассматриваемой части пространства. Граничные условия, например, на
поверхности Земли, предполагают обращение в нуль составляющих скорости
движения воздуха из-за влияния вязкости.

\hypertarget{ux43fux440ux43eux431ux43bux435ux43cux44b-ux438-ux437ux430ux434ux430ux447ux438-ux43fux440ux438-ux43fux43eux434ux433ux43eux442ux43eux432ux43aux435-ux434ux430ux43dux43dux44bux445}{%
\subsection{Проблемы и Задачи При Подготовке
Данных}\label{ux43fux440ux43eux431ux43bux435ux43cux44b-ux438-ux437ux430ux434ux430ux447ux438-ux43fux440ux438-ux43fux43eux434ux433ux43eux442ux43eux432ux43aux435-ux434ux430ux43dux43dux44bux445}}

Подготовка и ввод необходимой исходной информации в ЭВМ -- это
трудоемкий процесс. Ручная обработка может приводить к дополнительным
ошибкам, что снижает эффективность прогноза. Поэтому первоочередной
задачей является автоматизация подготовки исходных данных для численных
расчетов.

Комплекс автоматизированной обработки метеорологической информации
включает несколько ключевых этапов:

\begin{enumerate}
\def\labelenumi{\arabic{enumi}.}
\item
  \textbf{Сбор и первичная обработка поступающих сводок:} На этом этапе
  значения основных метеорологических величин со станций вводятся в
  память ЭВМ. Проводится \textbf{контроль правильности принятых
  сообщений}, включающий проверку согласованности отдельных данных в
  телеграмме между собой с использованием физических и статистических
  критериальных соотношений. Это позволяет выявить и забраковать
  противоречивые данные (например, точка росы не может быть выше
  температуры).
\item
  \textbf{Численный анализ метеорологических полей:} Цель этого этапа
  --- получить значения метеорологических величин в узлах регулярной
  сетки, используя данные наблюдений со станций. Это включает:

  \begin{itemize}
  \tightlist
  \item
    \textbf{Горизонтальный контроль:} Сравнение значений
    метеорологического элемента на одной станции со значениями на
    окружающих станциях. Если расхождения допустимы, данные сохраняются;
    в противном случае станция бракуется.
  \item
    \textbf{Объективный анализ:} Интерполяция значений метеорологической
    величины со станций на узлы регулярной сетки. Среди распространенных
    методов выделяют:

    \begin{itemize}
    \tightlist
    \item
      \textbf{Метод полиномиальной интерполяции}.
    \item
      \textbf{Метод последовательных уточнений:} Предварительное поле
      (ожидаемые по прогнозу значения или климатические нормы)
      корректируется итеративно фактическими данными станций путем
      расчета невязок и внесения поправок.
    \item
      \textbf{Метод оптимальной интерполяции:} Значение в узле
      рассчитывается как сумма климатической нормы и аномалии, где
      аномалия определяется суммированием аномалий на окружающих
      станциях с оптимальными весовыми множителями, зависящими от
      взаимного расположения станций и узла и статистической связи
      (автокорреляционной функции).
    \end{itemize}
  \end{itemize}
\end{enumerate}

\hypertarget{ux441ux43eux433ux43bux430ux441ux43eux432ux430ux43dux438ux435-ux43fux43eux43bux435ux439}{%
\subsection{Согласование
Полей}\label{ux441ux43eux433ux43bux430ux441ux43eux432ux430ux43dux438ux435-ux43fux43eux43bux435ux439}}

Особое внимание в процессе подготовки начальных данных уделяется
\textbf{согласованию полей давления и ветра}. Это является обязательным
элементом объективного анализа, поскольку поле ветра, особенно в
приземном слое, значительно больше подвержено влиянию мелкомасштабных
возмущений, вызванных неоднородностью подстилающей поверхности
(например, бризы, горно-долинные ветры). В реальной атмосфере происходит
взаимная адаптация полей давления и ветра, в результате которой
действительный ветер в свободной атмосфере приближается к
геострофическому.

В негеострофических прогностических моделях, которые не используют
геострофическое приближение для фильтрации быстрых волновых движений,
возникает проблема \textbf{``метеорологических шумов''} ---
мелкомасштабных возмущений, накапливающихся в процессе счета и
искажающих истинную картину крупномасштабных процессов. Для преодоления
этих трудностей применяется \textbf{``введение различных видов взаимного
согласования исходных и рассчитываемых полей давления и ветра''}. Это
позволяет подавить шумы и обеспечить вычислительную устойчивость
решения. После каждого шага по времени рассчитанные поля геопотенциала
могут подвергаться сглаживанию для ослабления роли этих возмущений.

Таким образом, процесс согласования начальных данных --- это комплексная
задача, включающая контроль качества, интерполяцию наблюдений на
регулярную сетку и динамическую коррекцию полей для их взаимной
совместимости в рамках прогностической модели.

\hypertarget{ux447ux435ux442ux44bux440ux435ux445ux43cux435ux440ux43dux43eux435-ux443ux441ux432ux43eux435ux43dux438ux435-ux434ux430ux43dux43dux44bux445}{%
\section{Четырехмерное Усвоение
Данных}\label{ux447ux435ux442ux44bux440ux435ux445ux43cux435ux440ux43dux43eux435-ux443ux441ux432ux43eux435ux43dux438ux435-ux434ux430ux43dux43dux44bux445}}

Как уже было отмечено, термин ``четырехмерное усвоение данных''
(Four-Dimensional Data Assimilation, 4DDA) не встречается в источниках.
Однако, концепции, описанные в них, являются фундаментальными для
понимания и развития 4DDA, а именно:

\begin{itemize}
\item
  \textbf{Использование наблюдений, распределенных во времени и
  пространстве:} Источники подчеркивают важность получения
  метеорологической информации от различных систем (наземные,
  аэрологические, автоматические станции, метеорологические
  радиолокаторы, искусственные спутники Земли, авиационная разведка
  погоды). Всемирная метеорологическая организация (ВМО) координирует
  глобальный мониторинг атмосферных процессов, обеспечивая сбор данных
  восемь раз в сутки через каждые три часа по единому (универсальному)
  времени. Эта ``трехмерность'' (пространство) и
  ``синхронность/регулярность'' (время) информации формируют основу для
  ``четырехмерности''.
\item
  \textbf{Необходимость учета эволюции во времени:} При решении
  прогностических задач, особенно в динамико-статистических методах,
  учитывается эволюция атмосферных процессов. Например, обсуждаются
  ``последовательности'' данных и их статистические связи.
  Экстраполяционные схемы, которые непосредственно вытекают из
  функциональных, гидродинамических моделей, также требуют учета
  результатов статистической обработки архивного материала. Это
  означает, что модель должна не просто использовать данные в один
  момент времени, но и учитывать их временную изменчивость.
\item
  \textbf{Согласование полей в динамической модели:} Введение различных
  видов взаимного согласования исходных и рассчитываемых полей давления
  и ветра для подавления метеорологических шумов является ключевым шагом
  в динамическом усвоении данных. По сути, это означает, что модель
  пытается найти такое начальное состояние, которое не только хорошо
  соответствует наблюдениям, но и динамически согласуется с уравнениями
  модели, предотвращая генерацию несвязанных с крупномасштабными
  процессами быстрых волн.
\item
  \textbf{Цикл ``анализ-прогноз'':} Оперативная работа
  метеорологического вычислительного центра включает непрерывный цикл:
  сбор и первичная обработка данных, численный анализ метеорологических
  полей, расчет прогнозов, и выдача результатов. Весь цикл проводится
  два раза в сутки, и расчеты могут повторяться для увеличенной области
  по мере поступления информации. Этот итерационный характер, где
  прогноз на короткий срок используется как ``первое приближение'' для
  последующего анализа, является одним из подходов в усвоении данных.
  Например, в методе последовательных уточнений ``предварительное''
  поле, созданное из ожидаемых значений по прогнозу, затем исправляется
  фактическими данными.
\end{itemize}

В контексте описанных подходов, четырехмерное усвоение данных можно
рассматривать как более продвинутую форму согласования начальных данных,
которая интегрирует наблюдения, распределенные во времени, напрямую в
динамическую модель. Это позволяет не только определить оптимальное
начальное состояние для прогноза, но и учесть динамику атмосферы в
течение определенного временного окна усвоения. Методы, описанные в
источниках, такие как взаимное согласование полей давления и ветра,
объективный анализ с использованием автокорреляционных функций и
итерационных схем, являются логическим базисом для развития сложных
систем 4DDA.

ewpage

Коллега,

Рассмотрим ключевые аспекты постановки задачи численного прогноза погоды
и неизбежно связанную с ней проблему предсказуемости атмосферных
процессов.

\hypertarget{ux43fux43eux441ux442ux430ux43dux43eux432ux43aux430-ux437ux430ux434ux430ux447ux438-ux447ux438ux441ux43bux435ux43dux43dux43eux433ux43e-ux43fux440ux43eux433ux43dux43eux437ux430-ux43fux43eux433ux43eux434ux44b}{%
\section{Постановка Задачи Численного Прогноза
Погоды}\label{ux43fux43eux441ux442ux430ux43dux43eux432ux43aux430-ux437ux430ux434ux430ux447ux438-ux447ux438ux441ux43bux435ux43dux43dux43eux433ux43e-ux43fux440ux43eux433ux43dux43eux437ux430-ux43fux43eux433ux43eux434ux44b}}

Постановка задачи численного прогноза погоды требует формирования и
решения замкнутой системы уравнений, описывающих процессы, происходящие
в атмосфере. Эти уравнения представляют собой математическое выражение
фундаментальных законов физики: закона сохранения количества движения
(второго закона Ньютона), закона сохранения энергии и закона сохранения
массы.

\hypertarget{ux43eux441ux43dux43eux432ux43dux44bux435-ux443ux440ux430ux432ux43dux435ux43dux438ux44f}{%
\subsection{Основные
Уравнения}\label{ux43eux441ux43dux43eux432ux43dux44bux435-ux443ux440ux430ux432ux43dux435ux43dux438ux44f}}

Изначально система включает:

\begin{itemize}
\tightlist
\item
  \textbf{Три уравнения движения} (в проекциях на оси координат),
  содержащие пять функций, зависящих от времени и координат: три
  проекции скорости (\(u, v, w\)), давление (\(P\)) и плотность воздуха
  (\(\rho\)).
\item
  \textbf{Уравнение неразрывности} (закон сохранения массы).
\item
  \textbf{Уравнение состояния} (например, для идеального газа), которое
  связывает плотность с давлением и температурой.
\item
  \textbf{Уравнение притока тепла} (первое начало термодинамики),
  учитывающее изменение внутренней энергии воздуха под воздействием
  притока тепла.
\end{itemize}

\hypertarget{ux43fux440ux43eux431ux43bux435ux43cux44b-ux438-ux443ux43fux440ux43eux449ux435ux43dux438ux44f}{%
\subsection{Проблемы и
Упрощения}\label{ux43fux440ux43eux431ux43bux435ux43cux44b-ux438-ux443ux43fux440ux43eux449ux435ux43dux438ux44f}}

Несмотря на наличие полной системы уравнений, их точное аналитическое
решение для нестационарных атмосферных процессов невозможно из-за их
нелинейности и сложности. Кроме того, исходная метеорологическая
информация поступает дискретно, что требует использования численных
методов дифференцирования и интерполяции.

\hypertarget{ux442ux443ux440ux431ux443ux43bux435ux43dux442ux43dux43eux441ux442ux44c}{%
\subsubsection{Турбулентность}\label{ux442ux443ux440ux431ux443ux43bux435ux43dux442ux43dux43eux441ux442ux44c}}

Движение воздуха в атмосфере является турбулентным, что вынуждает
использовать усредненные уравнения, при этом в системе появляются новые
неизвестные величины. Определение турбулентных напряжений является
сложной задачей, требующей знания закономерностей пульсационного
движения.

\hypertarget{ux440ux430ux434ux438ux430ux446ux438ux44f}{%
\subsubsection{Радиация}\label{ux440ux430ux434ux438ux430ux446ux438ux44f}}

Распределение радиации в атмосфере зависит от облачности, а также от
излучения и поглощения атмосферными газами, имеющими очень сложный
спектр. Точный учет этих факторов связан со значительными трудностями.

\hypertarget{ux43aux432ux430ux437ux438ux441ux442ux430ux442ux438ux447ux435ux441ux43aux43eux435-ux43fux440ux438ux431ux43bux438ux436ux435ux43dux438ux435-2}{%
\subsubsection{Квазистатическое
Приближение}\label{ux43aux432ux430ux437ux438ux441ux442ux430ux442ux438ux447ux435ux441ux43aux43eux435-ux43fux440ux438ux431ux43bux438ux436ux435ux43dux438ux435-2}}

Для крупномасштабных движений в качестве уравнения движения,
спроектированного на вертикальную ось, используется уравнение
квазистатики. Это приближение предполагает, что вертикальные ускорения
пренебрежимо малы по сравнению с силой тяжести.

\hypertarget{ux431ux430ux440ux43eux442ux440ux43eux43fux43dux44bux435-ux438-ux431ux430ux440ux43eux43aux43bux438ux43dux43dux44bux435-ux43cux43eux434ux435ux43bux438}{%
\subsubsection{Баротропные и Бароклинные
Модели}\label{ux431ux430ux440ux43eux442ux440ux43eux43fux43dux44bux435-ux438-ux431ux430ux440ux43eux43aux43bux438ux43dux43dux44bux435-ux43cux43eux434ux435ux43bux438}}

\begin{itemize}
\tightlist
\item
  \textbf{Баротропная жидкость:} Плотность зависит только от давления. В
  таком случае изобарические и изотермические поверхности должны
  совпадать. Это простейший случай, позволяющий решить задачу с четырьмя
  неизвестными функциями (\(P, u, v, w\)).
\item
  \textbf{Бароклинная жидкость:} Плотность является функцией не только
  давления, но и других параметров, таких как температура. В этом случае
  изобарические и изотермические поверхности могут пересекаться. Задача
  прогноза барического поля для бароклинной модели атмосферы в
  квазигеострофическом и адиабатическом приближениях была впервые решена
  в СССР Н. И. Булеевым и Г. И. Марчуком в 1951 году.
\end{itemize}

\hypertarget{ux43aux432ux430ux437ux438ux433ux435ux43eux441ux442ux440ux43eux444ux438ux447ux435ux441ux43aux43eux435-ux43fux440ux438ux431ux43bux438ux436ux435ux43dux438ux435-ux438-ux444ux438ux43bux44cux442ux440ux430ux446ux438ux44f-ux448ux443ux43cux43eux432}{%
\subsubsection{Квазигеострофическое Приближение и Фильтрация
Шумов}\label{ux43aux432ux430ux437ux438ux433ux435ux43eux441ux442ux440ux43eux444ux438ux447ux435ux441ux43aux43eux435-ux43fux440ux438ux431ux43bux438ux436ux435ux43dux438ux435-ux438-ux444ux438ux43bux44cux442ux440ux430ux446ux438ux44f-ux448ux443ux43cux43eux432}}

Для замыкания системы уравнений и ее решения часто привлекаются
дополнительные предположения, например, о близости реального ветра к
соленоидальному или геострофическому. Это позволяет исключить из
рассмотрения явления меньших масштабов, то есть \emph{отфильтровать так
называемые ``метеорологические шумы''}. Эти ``шумы'' представляют собой
мелкомасштабные возмущения, которые, накладываясь на крупномасштабные
процессы, искажают их и могут привести к вычислительной неустойчивости
решения.

\hypertarget{ux43dux430ux447ux430ux43bux44cux43dux44bux435-ux438-ux433ux440ux430ux43dux438ux447ux43dux44bux435-ux443ux441ux43bux43eux432ux438ux44f}{%
\subsection{Начальные и Граничные
Условия}\label{ux43dux430ux447ux430ux43bux44cux43dux44bux435-ux438-ux433ux440ux430ux43dux438ux447ux43dux44bux435-ux443ux441ux43bux43eux432ux438ux44f}}

Для решения системы дифференциальных уравнений в частных производных
должны быть заданы начальные и граничные условия.

\begin{itemize}
\tightlist
\item
  \textbf{Начальные условия:} Значения всех искомых функций (скорости,
  давления, температуры) должны быть известны во всех точках
  рассматриваемой части пространства для определенного исходного момента
  времени.
\item
  \textbf{Граничные условия:} На поверхности Земли (или на ее границе)
  составляющие скорости движения воздуха обращаются в нуль вследствие
  влияния вязкости. Для решения уравнения также могут задаваться
  граничные условия по вертикальной переменной.
\end{itemize}

\hypertarget{ux447ux438ux441ux43bux435ux43dux43dux44bux435-ux43cux435ux442ux43eux434ux44b-ux438-ux43eux43fux435ux440ux430ux442ux438ux432ux43dux430ux44f-ux440ux430ux431ux43eux442ux430}{%
\subsection{Численные Методы и Оперативная
Работа}\label{ux447ux438ux441ux43bux435ux43dux43dux44bux435-ux43cux435ux442ux43eux434ux44b-ux438-ux43eux43fux435ux440ux430ux442ux438ux432ux43dux430ux44f-ux440ux430ux431ux43eux442ux430}}

Уравнения гидротермодинамики решаются численными методами на
высокопроизводительных компьютерах. Современная оперативная работа
метеорологического вычислительного центра включает комплекс
автоматизированной обработки информации. Этот комплекс включает:

\begin{enumerate}
\def\labelenumi{\arabic{enumi}.}
\tightlist
\item
  \textbf{Сбор и первичную обработку} аэрологических и синоптических
  сводок.
\item
  \textbf{Контроль правильности принятых сообщений}, в том числе
  горизонтальный контроль и контроль согласованности данных на основе
  физических и статистических связей.
\item
  \textbf{Объективный анализ} (интерполяция значений метеорологических
  величин со станций на узлы регулярной сетки).
\item
  \textbf{Численное решение} прогностических уравнений, часто с
  использованием итерационных методов или метода функций влияния. Для
  поддержания устойчивости расчетов поля могут подвергаться сглаживанию.
\end{enumerate}

\hypertarget{ux43fux440ux43eux431ux43bux435ux43cux430-ux43fux440ux435ux434ux441ux43aux430ux437ux443ux435ux43cux43eux441ux442ux438}{%
\section{Проблема
Предсказуемости}\label{ux43fux440ux43eux431ux43bux435ux43cux430-ux43fux440ux435ux434ux441ux43aux430ux437ux443ux435ux43cux43eux441ux442ux438}}

Несмотря на все достижения в области численного моделирования, проблема
предсказуемости погоды остается одной из самых сложных в метеорологии.

\hypertarget{ux441ux443ux449ux43dux43eux441ux442ux44c-ux43dux435ux43fux440ux435ux434ux441ux43aux430ux437ux443ux435ux43cux43eux441ux442ux438}{%
\subsection{Сущность
Непредсказуемости}\label{ux441ux443ux449ux43dux43eux441ux442ux44c-ux43dux435ux43fux440ux435ux434ux441ux43aux430ux437ux443ux435ux43cux43eux441ux442ux438}}

Прогноз погоды, особенно на длительные сроки, пока не достигает желаемой
точности. Это связано с тем, что атмосферные процессы обладают
свойством, которое часто называют ``хаотичностью'' или чувствительностью
к начальным условиям.

\hypertarget{ux447ux443ux432ux441ux442ux432ux438ux442ux435ux43bux44cux43dux43eux441ux442ux44c-ux43a-ux43dux430ux447ux430ux43bux44cux43dux44bux43c-ux443ux441ux43bux43eux432ux438ux44fux43c}{%
\subsection{Чувствительность к Начальным
Условиям}\label{ux447ux443ux432ux441ux442ux432ux438ux442ux435ux43bux44cux43dux43eux441ux442ux44c-ux43a-ux43dux430ux447ux430ux43bux44cux43dux44bux43c-ux443ux441ux43bux43eux432ux438ux44fux43c}}

Атмосфера почти всегда находится в состоянии ``Геркулеса на распутье''.
Это означает, что даже весьма малые случайные взаимодействия или
неточности в начальных данных могут дать значительный эффект. В любой
момент достаточно какого-либо воздействия, чтобы перевести атмосферу из
устойчивого состояния в неустойчивое, как только параметры ее состояния
достигнут критических значений. Это принципиально ограничивает
возможность точного и обоснованного предсказания погоды.

\hypertarget{ux43cux435ux442ux435ux43eux440ux43eux43bux43eux433ux438ux447ux435ux441ux43aux438ux435-ux448ux443ux43cux44b-ux438-ux438ux445-ux432ux43bux438ux44fux43dux438ux435}{%
\subsubsection{``Метеорологические Шумы'' и Их
Влияние}\label{ux43cux435ux442ux435ux43eux440ux43eux43bux43eux433ux438ux447ux435ux441ux43aux438ux435-ux448ux443ux43cux44b-ux438-ux438ux445-ux432ux43bux438ux44fux43dux438ux435}}

Как уже упоминалось, квазигеострофические прогностические модели
намеренно фильтруют ``метеорологические шумы'' --- мелкомасштабные
возмущения, такие как гравитационные и инерционные волны. Эти быстрые,
короткопериодные колебания, если их не фильтровать, могут искажать
крупномасштабные процессы и приводить к вычислительной неустойчивости
решения. Однако такая фильтрация, улучшая устойчивость прогноза крупных
масштабов, естественно, уменьшает детализацию и точность предсказания на
мелких масштабах.

\hypertarget{ux432ux43bux438ux44fux43dux438ux435-ux43cux430ux441ux448ux442ux430ux431ux43eux432-ux43fux440ux43eux446ux435ux441ux441ux43eux432}{%
\subsection{Влияние Масштабов
Процессов}\label{ux432ux43bux438ux44fux43dux438ux435-ux43cux430ux441ux448ux442ux430ux431ux43eux432-ux43fux440ux43eux446ux435ux441ux441ux43eux432}}

Формирование облака и его системы, а также развитие циклона, требует
определенного времени (от нескольких часов до 1-2 суток) и проходит
через стадии возмущения, которые могут быть не отражены на стандартных
синоптических картах. Это часто приводит к ``внезапным''
(непредсказуемым численным моделированием) изменениям синоптической
обстановки и погоды в целом.

\hypertarget{ux441ux443ux431ux44aux435ux43aux442ux438ux432ux438ux437ux43c-ux438-ux432ux435ux440ux43eux44fux442ux43dux43eux441ux442ux43dux44bux439-ux445ux430ux440ux430ux43aux442ux435ux440}{%
\subsection{Субъективизм и Вероятностный
Характер}\label{ux441ux443ux431ux44aux435ux43aux442ux438ux432ux438ux437ux43c-ux438-ux432ux435ux440ux43eux44fux442ux43dux43eux441ux442ux43dux44bux439-ux445ux430ux440ux430ux43aux442ux435ux440}}

Многие прогностические синоптические методики содержат элементы
субъективизма, и успешность прогнозов может зависеть от опыта работы
синоптика и знания им географических и климатических особенностей
района. В конечном итоге, все прогнозы являются вероятностными.
Потребителю сообщаются наиболее вероятные значения или фазы, но важно,
чтобы ему были известны и вероятности осуществления различных явлений.
Это подчеркивает фундаментальную неопределенность в детальном прогнозе
будущего состояния атмосферы.

Несмотря на эти вызовы, постоянное совершенствование прогностических
методов, включая переход от эмпирических правил к более строгим
гидродинамическим и статистическим моделям, продолжает улучшать точность
и заблаговременность прогнозов.

ewpage

\hypertarget{ux447ux438ux441ux43bux435ux43dux43dux44bux439-ux430ux43dux430ux43bux438ux437-ux441ux438ux43dux445ux440ux43eux43dux43dux44bux445-ux43cux435ux442ux435ux43eux440ux43eux43bux43eux433ux438ux447ux435ux441ux43aux438ux445-ux43fux43eux43bux435ux439-1}{%
\section{Численный Анализ Синхронных Метеорологических
Полей}\label{ux447ux438ux441ux43bux435ux43dux43dux44bux439-ux430ux43dux430ux43bux438ux437-ux441ux438ux43dux445ux440ux43eux43dux43dux44bux445-ux43cux435ux442ux435ux43eux440ux43eux43bux43eux433ux438ux447ux435ux441ux43aux438ux445-ux43fux43eux43bux435ux439-1}}

Численный анализ метеорологических полей является краеугольным камнем в
комплексе автоматизированной обработки метеорологической информации,
необходимой для оперативного прогноза погоды. Его главная функция
заключается в преобразовании дискретных данных наблюдений, поступающих с
метеорологических и аэрологических станций, в значения метеорологических
величин, таких как высота изобарических поверхностей, компоненты ветра,
точка росы, приземное давление и барическая тенденция, в узлах
регулярной расчетной сетки. Это обеспечивает подготовку исходных данных
для последующего применения в численных прогностических схемах.

Следует отметить, что уравнения гидротермодинамики, описывающие
нестационарные атмосферные процессы, являются нелинейными
дифференциальными уравнениями в частных производных, для которых
аналитическое решение в большинстве случаев невозможно. Следовательно,
их решение осуществляется численными методами с использованием
высокопроизводительных вычислительных систем.

\hypertarget{ux43aux43eux43dux435ux447ux43dux43eux440ux430ux437ux43dux43eux441ux442ux43dux44bux435-ux43fux43eux434ux445ux43eux434ux44b}{%
\subsection{Конечноразностные
Подходы}\label{ux43aux43eux43dux435ux447ux43dux43eux440ux430ux437ux43dux43eux441ux442ux43dux44bux435-ux43fux43eux434ux445ux43eux434ux44b}}

Конечноразностные методы являются одними из основных подходов в
численном решении уравнений гидротермодинамики. Суть их заключается в
замене бесконечно малых приращений метеорологической величины её
конечными разностями. Это позволяет аппроксимировать производные по
координатам и времени, которые встречаются в уравнениях движения,
энергии и неразрывности. Одним из наиболее часто применяемых приемов
является метод центральных разностей.

Пример применения конечноразностных подходов можно увидеть в решении
прогностических задач, таких как прогноз барического поля. Так, в
методике, предложенной Булеевым и Марчуком, конечноразностные уравнения
используются для численного решения дифференциального уравнения второго
порядка для геопотенциала.

\hypertarget{ux43cux435ux442ux43eux434ux44b-ux43eux431ux44aux435ux43aux442ux438ux432ux43dux43eux433ux43e-ux430ux43dux430ux43bux438ux437ux430-1}{%
\subsection{Методы Объективного
Анализа}\label{ux43cux435ux442ux43eux434ux44b-ux43eux431ux44aux435ux43aux442ux438ux432ux43dux43eux433ux43e-ux430ux43dux430ux43bux438ux437ux430-1}}

В рамках численного анализа метеорологических полей со станций на узлы
регулярной сетки наибольшее распространение получили следующие методы:

\hypertarget{ux43cux435ux442ux43eux434-ux43fux43eux43bux438ux43dux43eux43cux438ux430ux43bux44cux43dux43eux439-ux438ux43dux442ux435ux440ux43fux43eux43bux44fux446ux438ux438-1}{%
\subsubsection{1. Метод Полиномиальной
Интерполяции}\label{ux43cux435ux442ux43eux434-ux43fux43eux43bux438ux43dux43eux43cux438ux430ux43bux44cux43dux43eux439-ux438ux43dux442ux435ux440ux43fux43eux43bux44fux446ux438ux438-1}}

Этот метод предполагает аппроксимацию поля метеорологической величины в
районе узла сетки алгебраическим полиномом, обычно второй, иногда
третьей степени. Коэффициенты полинома определяются по значениям
метеорологической величины на окружающих станциях с использованием
метода наименьших квадратов. Это обеспечивает наилучшее приближение
аппроксимированного поля к фактическим данным наблюдений. После
вычисления коэффициентов полинома, интерполированное значение для узла
сетки находится, и процесс повторяется для всех узлов.

\hypertarget{ux43cux435ux442ux43eux434-ux43fux43eux441ux43bux435ux434ux43eux432ux430ux442ux435ux43bux44cux43dux44bux445-ux443ux442ux43eux447ux43dux435ux43dux438ux439}{%
\subsubsection{2. Метод Последовательных
Уточнений}\label{ux43cux435ux442ux43eux434-ux43fux43eux441ux43bux435ux434ux43eux432ux430ux442ux435ux43bux44cux43dux44bux445-ux443ux442ux43eux447ux43dux435ux43dux438ux439}}

Метод последовательных уточнений начинается с формирования
``предварительного'' поля в узлах регулярной сетки, используя либо
ожидаемые прогностические значения, либо климатические нормы, или их
комбинации. Далее, это предварительное поле корректируется фактическими
данными наблюдений. Для этого значения из узлов сетки интерполируются на
станции, где сравниваются с фактическими измерениями, чтобы определить
невязки. По этим невязкам рассчитываются поправки, которые вносятся в
предварительное поле. Этот процесс итеративно повторяется, пока поле в
узлах сетки не будет приведено в соответствие с фактическими данными на
станциях.

\hypertarget{ux43cux435ux442ux43eux434-ux43eux43fux442ux438ux43cux430ux43bux44cux43dux43eux439-ux438ux43dux442ux435ux440ux43fux43eux43bux44fux446ux438ux438-1}{%
\subsubsection{3. Метод Оптимальной
Интерполяции}\label{ux43cux435ux442ux43eux434-ux43eux43fux442ux438ux43cux430ux43bux44cux43dux43eux439-ux438ux43dux442ux435ux440ux43fux43eux43bux44fux446ux438ux438-1}}

В методе оптимальной интерполяции значение метеорологической величины в
каждом узле расчетной сетки определяется как сумма климатической нормы в
этом узле и аномалии (отклонения от нормы). Аномалия в узле
рассчитывается путем суммирования аномалий на окружающих станциях с
определенными весовыми множителями. Эти весовые множители зависят от
взаимного расположения станций и узла: их вес уменьшается с удалением от
узла, а также при увеличении плотности близлежащих станций, используемых
для интерполяции. Весовые множители подбираются оптимальным образом в
статистическом смысле, исходя из условия минимизации ошибки
интерполяции, и для их нахождения используется пространственная
автокорреляционная функция анализируемой метеорологической величины.

\hypertarget{ux434ux43eux43fux43eux43bux43dux438ux442ux435ux43bux44cux43dux44bux435-ux43fux43eux434ux445ux43eux434ux44b-ux438-ux43fux440ux43eux431ux43bux435ux43cux430ux442ux438ux43aux430}{%
\subsection{Дополнительные Подходы и
Проблематика}\label{ux434ux43eux43fux43eux43bux43dux438ux442ux435ux43bux44cux43dux44bux435-ux43fux43eux434ux445ux43eux434ux44b-ux438-ux43fux440ux43eux431ux43bux435ux43cux430ux442ux438ux43aux430}}

В целом, численные методы прогноза погоды постоянно развиваются.
Современные подходы к решению дифференциальных уравнений включают
применение различных устойчивых схем, использование более коротких шагов
интегрирования по времени, а также метод расщепления (splitting method).
Метод расщепления упрощает процедуру решения, разбивая исходные
дифференциальные уравнения на более простые, учитывающие отдельные
физические факторы.

Однако, в рамках предоставленных источников, детальная информация о
применении \textbf{полулагранжевых} или \textbf{спектральных подходов}
непосредственно для численного решения уравнений гидротермодинамики
отсутствует. Хотя упоминается ``спектральный анализ временных рядов'' в
контексте анализа колебаний метеорологических величин и представление
периодических процессов в виде суммы гармоник (теорема Фурье) для
теплопроводности, это не описывает их использование как численных
методов для интегрирования основных прогностических уравнений атмосферы.
Источники сосредоточены на конечноразностных методах и различных схемах
интерполяции, используемых для подготовки данных и решения уравнений.

ewpage

\hypertarget{ux43fux430ux440ux430ux43cux435ux442ux440ux438ux437ux430ux446ux438ux44f-ux444ux438ux437ux438ux447ux435ux441ux43aux438ux445-ux43fux440ux43eux446ux435ux441ux441ux43eux432-ux432-ux43cux43eux434ux435ux43bux44fux445-ux43fux440ux43eux433ux43dux43eux437ux430}{%
\section{Параметризация Физических Процессов в Моделях
Прогноза}\label{ux43fux430ux440ux430ux43cux435ux442ux440ux438ux437ux430ux446ux438ux44f-ux444ux438ux437ux438ux447ux435ux441ux43aux438ux445-ux43fux440ux43eux446ux435ux441ux441ux43eux432-ux432-ux43cux43eux434ux435ux43bux44fux445-ux43fux440ux43eux433ux43dux43eux437ux430}}

В современных численных моделях прогноза погоды и климата, атмосферные
процессы, происходящие на масштабах меньших, чем разрешение расчетной
сетки (подсеточные процессы), не могут быть явно разрешены и требуют
параметризации. Это означает, что их интегральное влияние на
крупномасштабные параметры атмосферы должно быть описано через величины,
разрешаемые моделью. Ниже рассмотрим основные аспекты параметризации
ключевых физических процессов.

\hypertarget{ux43fux43eux434ux441ux435ux442ux43eux447ux43dux430ux44f-ux442ux443ux440ux431ux443ux43bux435ux43dux442ux43dux43eux441ux442ux44c}{%
\subsection{Подсеточная
Турбулентность}\label{ux43fux43eux434ux441ux435ux442ux43eux447ux43dux430ux44f-ux442ux443ux440ux431ux443ux43bux435ux43dux442ux43dux43eux441ux442ux44c}}

Турбулентность в атмосфере проявляется в виде резких, беспорядочных
колебаний скорости, давления и плотности, называемых пульсациями или
флуктуациями. Атмосферные движения, как правило, носят турбулентный
характер, что наиболее выражено в порывистости ветра. Поскольку эти
процессы происходят на масштабах значительно меньших, чем шаг сетки
прогностических моделей, их прямое моделирование невозможно.

При осреднении уравнений гидротермодинамики для учета турбулентного
движения, в систему уравнений вводятся новые неизвестные величины, так
называемые турбулентные напряжения или потоки. Их определение является
сложной задачей, требующей знания закономерностей пульсационного
движения.

Параметризация турбулентности основывается на аналогии с молекулярным
движением и на статистической теории турбулентности. В пограничном слое
атмосферы, где турбулентный обмен особенно интенсивен, для определения
коэффициентов турбулентного обмена привлекаются динамические уравнения
турбулентности. Эти коэффициенты характеризуют перенос импульса, тепла,
водяного пара, а также капель воды, кристаллов льда и загрязняющих
веществ. Интенсивность турбулентного обмена существенно влияет на
отклонение ветра от поля давления в пограничном слое. Кроме того, трение
в пограничном слое приводит к диссипации кинетической энергии, превращая
ее в тепловую.

\hypertarget{ux440ux430ux434ux438ux430ux446ux438ux43eux43dux43dux44bux435-ux43fux43eux442ux43eux43aux438}{%
\subsection{Радиационные
Потоки}\label{ux440ux430ux434ux438ux430ux446ux438ux43eux43dux43dux44bux435-ux43fux43eux442ux43eux43aux438}}

Радиационный теплообмен в атмосфере обусловлен поглощением и излучением
электромагнитных волн слоями воздуха. Солнечная радиация поступает в
атмосферу из очень малого телесного угла, и ее интенсивность в
коротковолновом диапазоне значительно превосходит излучение Земли и
атмосферы, в то время как длинноволновым излучением Солнца можно
пренебречь по сравнению с излучением Земли и атмосферы.

Расчет радиационных потоков является сложной задачей, поскольку спектр
поглощения атмосферных газов, в особенности водяного пара, имеет
линейчатую структуру и не может быть представлен аналитически. Воздух
слабо поглощает солнечную радиацию, а радиационный приток тепла в
частицу воздуха происходит преимущественно в длинноволновом диапазоне.
Особое внимание уделяется так называемому ``окнe прозрачности''
атмосферы в диапазоне длин волн от 8.5 до 12 мкм, где атмосфера
практически пропускает все излучение. Газы, поглощающие в этом окне
(например, метан), могут значительно изменять энергетический баланс
планеты.

В моделях для упрощения расчетов коэффициенты поглощения заменяются
функциями пропускания, зависящими от оптического пути. Облачность
оказывает существенное влияние на радиационный баланс земной
поверхности, изменяя как приток солнечной радиации, так и эффективное
излучение. Неравномерное распределение радиационного баланса по Земле
является основной причиной горизонтальных разностей температур, что, в
свою очередь, приводит к возникновению атмосферной циркуляции.

Земля как планета в целом находится в состоянии радиационного равновесия
(поток приходящей солнечной радиации уравновешен потоком отраженной
коротковолновой и уходящей длинноволновой радиации), однако сама
поверхность Земли в состоянии радиационного равновесия не находится.

\hypertarget{ux43aux440ux443ux43fux43dux43eux43cux430ux441ux448ux442ux430ux431ux43dux430ux44f-ux43aux43eux43dux434ux435ux43dux441ux430ux446ux438ux44f}{%
\subsection{Крупномасштабная
Конденсация}\label{ux43aux440ux443ux43fux43dux43eux43cux430ux441ux448ux442ux430ux431ux43dux430ux44f-ux43aux43eux43dux434ux435ux43dux441ux430ux446ux438ux44f}}

Процессы фазовых переходов воды -- конденсация, сублимация, испарение
капель воды и кристаллов льда -- играют ключевую роль в формировании
облаков и осадков и являются существенными источниками или стоками тепла
в атмосфере. Учет этих процессов в численных моделях является одной из
основных трудностей, поскольку они приводят к нелинейным уравнениям.

В прогностических моделях теплота фазовых переходов воды учитывается,
например, при построении прогностической кривой стратификации, которая
приближается к влажной адиабате при образовании кучевых облаков. Для
прогноза осадков необходимо учитывать ожидаемую форму и количество
облаков, толщину облачного слоя, интенсивность вертикальных движений
внутри облака, а также его микрофизическое строение и водность. В
средних и высоких широтах основным условием выпадения осадков является
наличие в облаке трех фаз -- водяного пара, капелек и ледяных элементов.

Несмотря на сложности, связанные с детальным учетом микрофизических
процессов (информация о которых часто отсутствует в оперативной
практике), делаются попытки косвенного расчета изменений агрегатного
состояния воды в облаке. Также в моделях учитывается уровень
конденсации, который может быть рассчитан на основе прогнозируемых
значений максимальной температуры и точки росы у поверхности земли.
Разрабатываются неадиабатические модели конвекции, способные более точно
предсказывать высоту нижней границы облаков.

\hypertarget{ux43aux43eux43dux432ux435ux43aux446ux438ux44f}{%
\subsection{Конвекция}\label{ux43aux43eux43dux432ux435ux43aux446ux438ux44f}}

Конвекция представляет собой упорядоченные местные вертикальные
движения, связанные с развитием кучевых и кучево-дождевых облаков. В
тропических широтах мощные конвективные системы, развивающиеся вдоль
внутритропической зоны конвергенции, являются основной причиной
изменчивости погоды. Эти ``горячие башни'' (мощные кучевые облака,
пробивающие инверсию пассатов и достигающие тропопаузы) играют важную
роль в тепло- и влагообмене между низкими и высокими широтами. Если
такие конвективные комплексы проходят над океаном с температурой воды
более 27°C, они могут получить дополнительную энергию и стать причиной
формирования тропических циклонов.

Поскольку конвективные процессы происходят на подсеточных масштабах, их
эффекты в прогностических моделях должны быть параметризованы. Развитие
конвекции происходит в слоях, где прогностическая кривая стратификации
отклоняется влево от влажной адиабаты, то есть γ \textgreater{} γwa, что
указывает на конвективную неустойчивость.

Современные методы прогноза конвективной облачности основаны на моделях
конвекции и использовании комплексных показателей для учета нелинейности
процессов. Количество развивающейся конвективной облачности
определяется, в первую очередь, вертикальными профилями температуры и
влажности. Несмотря на то что точный прогноз вертикальных профилей
температуры и влажности воздуха в оперативных условиях труднодостижим,
неадиабатические модели конвекции показывают хорошие результаты для
прогноза высоты нижней границы облаков. В схемах прогноза учитываются и
суточные прогнозы конвективных осадков.

\end{document}
